%
% \iffalse
%
%% lscape.dtx Copyright (C) 1994 1999-2000 David Carlisle
%%            Copyright (C) 2000--2015 David Carlisle, LaTeX3 Project
%%
%% This file is part of the Standard LaTeX `Graphics Bundle'.
%% It may be distributed under the terms of the LaTeX Project Public
%% License, as described in lppl.txt in the base LaTeX distribution.
%% Either version 1.3c or, at your option, any later version.
%%
%<*dtx>
          \ProvidesFile{lscape.dtx}
%</dtx>
%<package>\NeedsTeXFormat{LaTeX2e}[1994/06/01]
%<package>\ProvidesPackage{lscape}
%<driver> \ProvidesFile{lscape.drv}
% \fi
%         \ProvidesFile{lscape.dtx}
          [2000/10/22 v3.01 Landscape Pages (DPC)]
%
% \iffalse
%<*driver>
\documentclass{ltxdoc}
\usepackage{lscape}
\begin{document}
 \DocInput{lscape.dtx}
\end{document}
%</driver>
% \fi
%
% \GetFileInfo{lscape.dtx}
%
% \title{The \textsf{lscape} package\thanks{This file
%        has version number \fileversion, last
%        revised \filedate.}}
% \author{D. P. Carlisle}
% \date{\filedate}
%
% \CheckSum{69}
%
% \begin{landscape}
% \MaintainedByLaTeXTeam{graphics}
% \maketitle
%
% All text, within the \texttt{landscape} environment is rotated through
% 90 degrees. The environment may span several pages. It works well
% with, and was originally created for, use with \textsf{longtable} to
% produce long wide tables.
%
% All the work is done by the \textsf{graphics} package, the trick is to
% call it at the right place in the output routine to rotate the current
% page before the page head and foot are added.
%
%
% \StopEventually{\end{landscape}}
% %%%%%%%%%%%%%%%%%%%%%%%%%%%%%%%%%%%%%%%%%%%%%%%%%%%%%%%%%%%%%%%%%%%%
%
% \changes{v1.00}{1992/04/28}{Initial release}
% \changes{v2.00}{1992/11/18}{Support for footnotes and floats.
%         Special thanks to Sarah J. McQuay for testing this release}
% \changes{v3.00}{1994/10/05}{Update using LaTeX2e graphics package}
% \changes{v3.01}{1999/06/22}{Added better pdfTeX support
%    by Scott Pakin, from graphics/3063}
%
%
%    \begin{macrocode}
%<*package>
%    \end{macrocode}
%  Here is what Scott Pakin wrote about the pdftex option:
% \begin{quote}
%    I've made a minor improvement to lscape.dtx that I'd like to share
% with the world.  My addition makes lscape rotate the PDF "paper" --
% not just the text on the page -- when given the "pdftex" option.
% (Naturally, this works only with pdfLaTeX.)  The result is that the
% text is viewable online without the reader having to rotate his/her
% head 90 degrees.  The document still prints normally.
% \end{quote}
%
%
%    \begin{macrocode}
\newif\ifGin@pdftex
\Gin@pdftexfalse
\DeclareOption{pdftex}{%
  \PassOptionsToPackage\CurrentOption{graphics}
  \Gin@pdftextrue
}
\DeclareOption*{\PassOptionsToPackage\CurrentOption{graphics}}
\ProcessOptions
\RequirePackage{graphics}
%    \end{macrocode}
%
%    \begin{macrocode}
\let\LS@makecol=\@makecol
\let\LS@makefcolumn=\@makefcolumn
\def\LS@rot{%
  \setbox\@outputbox\vbox{\hbox{\rotatebox{90}{\box\@outputbox}}}}
%    \end{macrocode}
%
% First set up the page size for landscape.
%    \begin{macrocode}
\def\landscape{%
  \clearpage
  \begingroup
  \vsize=\textwidth
  \hsize=\textheight
  \linewidth=\hsize
  \columnwidth=\hsize
  \@colroom=\vsize
  \textheight=\vsize
  \@colht=\vsize
%    \end{macrocode}
%    Rotate the page body (text and float pages).
%    \begin{macrocode}
  \def\@makecol{\LS@makecol\LS@rot}%
  \def\@makefcolumn##1{\LS@makefcolumn{##1}\LS@rot}}
%    \end{macrocode}
%
%    Restore everything.  This is where the pdftex option makes a
%    difference.
%    \begin{macrocode}
\def\endlandscape{%
  \ifGin@pdftex
    \pdfpageattr{/Rotate 90}
  \fi
  \clearpage
  \endgroup
  \global\@colht=\textheight
  \global\vsize=\textheight
  \global\@colroom=\textheight}
%</package>
%    \end{macrocode}
%
%
% \Finale
%
