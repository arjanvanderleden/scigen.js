% \iffalse meta-comment
%
% File: scrindex.dtx
% Version: 2008/08/11 v1.1
% Info: Package index with KOMA-Script classes
%
% Copyright (C) 2008 by
%    Heiko Oberdiek <heiko.oberdiek at googlemail.com>
%
% This work may be distributed and/or modified under the
% conditions of the LaTeX Project Public License, either
% version 1.3c of this license or (at your option) any later
% version. This version of this license is in
%    http://www.latex-project.org/lppl/lppl-1-3c.txt
% and the latest version of this license is in
%    http://www.latex-project.org/lppl.txt
% and version 1.3 or later is part of all distributions of
% LaTeX version 2005/12/01 or later.
%
% This work has the LPPL maintenance status "maintained".
%
% This Current Maintainer of this work is Heiko Oberdiek.
%
% This work consists of the main source file scrindex.dtx
% and the derived files
%    scrindex.sty, scrindex.pdf, scrindex.ins, scrindex.drv,
%    scrindex-example1.tex, scrindex-example2.tex.
%
% Distribution:
%    CTAN:macros/latex/contrib/oberdiek/scrindex.dtx
%    CTAN:macros/latex/contrib/oberdiek/scrindex.pdf
%
% Unpacking:
%    (a) If scrindex.ins is present:
%           tex scrindex.ins
%    (b) Without scrindex.ins:
%           tex scrindex.dtx
%    (c) If you insist on using LaTeX
%           latex \let\install=y% \iffalse meta-comment
%
% File: scrindex.dtx
% Version: 2008/08/11 v1.1
% Info: Package index with KOMA-Script classes
%
% Copyright (C) 2008 by
%    Heiko Oberdiek <heiko.oberdiek at googlemail.com>
%
% This work may be distributed and/or modified under the
% conditions of the LaTeX Project Public License, either
% version 1.3c of this license or (at your option) any later
% version. This version of this license is in
%    http://www.latex-project.org/lppl/lppl-1-3c.txt
% and the latest version of this license is in
%    http://www.latex-project.org/lppl.txt
% and version 1.3 or later is part of all distributions of
% LaTeX version 2005/12/01 or later.
%
% This work has the LPPL maintenance status "maintained".
%
% This Current Maintainer of this work is Heiko Oberdiek.
%
% This work consists of the main source file scrindex.dtx
% and the derived files
%    scrindex.sty, scrindex.pdf, scrindex.ins, scrindex.drv,
%    scrindex-example1.tex, scrindex-example2.tex.
%
% Distribution:
%    CTAN:macros/latex/contrib/oberdiek/scrindex.dtx
%    CTAN:macros/latex/contrib/oberdiek/scrindex.pdf
%
% Unpacking:
%    (a) If scrindex.ins is present:
%           tex scrindex.ins
%    (b) Without scrindex.ins:
%           tex scrindex.dtx
%    (c) If you insist on using LaTeX
%           latex \let\install=y% \iffalse meta-comment
%
% File: scrindex.dtx
% Version: 2008/08/11 v1.1
% Info: Package index with KOMA-Script classes
%
% Copyright (C) 2008 by
%    Heiko Oberdiek <heiko.oberdiek at googlemail.com>
%
% This work may be distributed and/or modified under the
% conditions of the LaTeX Project Public License, either
% version 1.3c of this license or (at your option) any later
% version. This version of this license is in
%    http://www.latex-project.org/lppl/lppl-1-3c.txt
% and the latest version of this license is in
%    http://www.latex-project.org/lppl.txt
% and version 1.3 or later is part of all distributions of
% LaTeX version 2005/12/01 or later.
%
% This work has the LPPL maintenance status "maintained".
%
% This Current Maintainer of this work is Heiko Oberdiek.
%
% This work consists of the main source file scrindex.dtx
% and the derived files
%    scrindex.sty, scrindex.pdf, scrindex.ins, scrindex.drv,
%    scrindex-example1.tex, scrindex-example2.tex.
%
% Distribution:
%    CTAN:macros/latex/contrib/oberdiek/scrindex.dtx
%    CTAN:macros/latex/contrib/oberdiek/scrindex.pdf
%
% Unpacking:
%    (a) If scrindex.ins is present:
%           tex scrindex.ins
%    (b) Without scrindex.ins:
%           tex scrindex.dtx
%    (c) If you insist on using LaTeX
%           latex \let\install=y% \iffalse meta-comment
%
% File: scrindex.dtx
% Version: 2008/08/11 v1.1
% Info: Package index with KOMA-Script classes
%
% Copyright (C) 2008 by
%    Heiko Oberdiek <heiko.oberdiek at googlemail.com>
%
% This work may be distributed and/or modified under the
% conditions of the LaTeX Project Public License, either
% version 1.3c of this license or (at your option) any later
% version. This version of this license is in
%    http://www.latex-project.org/lppl/lppl-1-3c.txt
% and the latest version of this license is in
%    http://www.latex-project.org/lppl.txt
% and version 1.3 or later is part of all distributions of
% LaTeX version 2005/12/01 or later.
%
% This work has the LPPL maintenance status "maintained".
%
% This Current Maintainer of this work is Heiko Oberdiek.
%
% This work consists of the main source file scrindex.dtx
% and the derived files
%    scrindex.sty, scrindex.pdf, scrindex.ins, scrindex.drv,
%    scrindex-example1.tex, scrindex-example2.tex.
%
% Distribution:
%    CTAN:macros/latex/contrib/oberdiek/scrindex.dtx
%    CTAN:macros/latex/contrib/oberdiek/scrindex.pdf
%
% Unpacking:
%    (a) If scrindex.ins is present:
%           tex scrindex.ins
%    (b) Without scrindex.ins:
%           tex scrindex.dtx
%    (c) If you insist on using LaTeX
%           latex \let\install=y\input{scrindex.dtx}
%        (quote the arguments according to the demands of your shell)
%
% Documentation:
%    (a) If scrindex.drv is present:
%           latex scrindex.drv
%    (b) Without scrindex.drv:
%           latex scrindex.dtx; ...
%    The class ltxdoc loads the configuration file ltxdoc.cfg
%    if available. Here you can specify further options, e.g.
%    use A4 as paper format:
%       \PassOptionsToClass{a4paper}{article}
%
%    Programm calls to get the documentation (example):
%       pdflatex scrindex.dtx
%       makeindex -s gind.ist scrindex.idx
%       pdflatex scrindex.dtx
%       makeindex -s gind.ist scrindex.idx
%       pdflatex scrindex.dtx
%
% Installation:
%    TDS:tex/latex/oberdiek/scrindex.sty
%    TDS:doc/latex/oberdiek/scrindex.pdf
%    TDS:doc/latex/oberdiek/scrindex-example1.tex
%    TDS:doc/latex/oberdiek/scrindex-example2.tex
%    TDS:source/latex/oberdiek/scrindex.dtx
%
%<*ignore>
\begingroup
  \catcode123=1 %
  \catcode125=2 %
  \def\x{LaTeX2e}%
\expandafter\endgroup
\ifcase 0\ifx\install y1\fi\expandafter
         \ifx\csname processbatchFile\endcsname\relax\else1\fi
         \ifx\fmtname\x\else 1\fi\relax
\else\csname fi\endcsname
%</ignore>
%<*install>
\input docstrip.tex
\Msg{************************************************************************}
\Msg{* Installation}
\Msg{* Package: scrindex 2008/08/11 v1.1 Package index with KOMA-Script classes (HO)}
\Msg{************************************************************************}

\keepsilent
\askforoverwritefalse

\let\MetaPrefix\relax
\preamble

This is a generated file.

Project: scrindex
Version: 2008/08/11 v1.1

Copyright (C) 2008 by
   Heiko Oberdiek <heiko.oberdiek at googlemail.com>

This work may be distributed and/or modified under the
conditions of the LaTeX Project Public License, either
version 1.3c of this license or (at your option) any later
version. This version of this license is in
   http://www.latex-project.org/lppl/lppl-1-3c.txt
and the latest version of this license is in
   http://www.latex-project.org/lppl.txt
and version 1.3 or later is part of all distributions of
LaTeX version 2005/12/01 or later.

This work has the LPPL maintenance status "maintained".

This Current Maintainer of this work is Heiko Oberdiek.

This work consists of the main source file scrindex.dtx
and the derived files
   scrindex.sty, scrindex.pdf, scrindex.ins, scrindex.drv,
   scrindex-example1.tex, scrindex-example2.tex.

\endpreamble
\let\MetaPrefix\DoubleperCent

\generate{%
  \file{scrindex.ins}{\from{scrindex.dtx}{install}}%
  \file{scrindex.drv}{\from{scrindex.dtx}{driver}}%
  \usedir{tex/latex/oberdiek}%
  \file{scrindex.sty}{\from{scrindex.dtx}{package}}%
  \usedir{doc/latex/oberdiek}%
  \file{scrindex-example1.tex}{\from{scrindex.dtx}{example1}}%
  \file{scrindex-example2.tex}{\from{scrindex.dtx}{example2}}%
  \nopreamble
  \nopostamble
  \usedir{source/latex/oberdiek/catalogue}%
  \file{scrindex.xml}{\from{scrindex.dtx}{catalogue}}%
}

\catcode32=13\relax% active space
\let =\space%
\Msg{************************************************************************}
\Msg{*}
\Msg{* To finish the installation you have to move the following}
\Msg{* file into a directory searched by TeX:}
\Msg{*}
\Msg{*     scrindex.sty}
\Msg{*}
\Msg{* To produce the documentation run the file `scrindex.drv'}
\Msg{* through LaTeX.}
\Msg{*}
\Msg{* Happy TeXing!}
\Msg{*}
\Msg{************************************************************************}

\endbatchfile
%</install>
%<*ignore>
\fi
%</ignore>
%<*driver>
\NeedsTeXFormat{LaTeX2e}
\ProvidesFile{scrindex.drv}%
  [2008/08/11 v1.1 Package index with KOMA-Script classes (HO)]%
\documentclass{ltxdoc}
\usepackage{holtxdoc}[2011/11/22]
\usepackage{calc}
\begin{document}
  \DocInput{scrindex.dtx}%
\end{document}
%</driver>
% \fi
%
% \CheckSum{237}
%
% \CharacterTable
%  {Upper-case    \A\B\C\D\E\F\G\H\I\J\K\L\M\N\O\P\Q\R\S\T\U\V\W\X\Y\Z
%   Lower-case    \a\b\c\d\e\f\g\h\i\j\k\l\m\n\o\p\q\r\s\t\u\v\w\x\y\z
%   Digits        \0\1\2\3\4\5\6\7\8\9
%   Exclamation   \!     Double quote  \"     Hash (number) \#
%   Dollar        \$     Percent       \%     Ampersand     \&
%   Acute accent  \'     Left paren    \(     Right paren   \)
%   Asterisk      \*     Plus          \+     Comma         \,
%   Minus         \-     Point         \.     Solidus       \/
%   Colon         \:     Semicolon     \;     Less than     \<
%   Equals        \=     Greater than  \>     Question mark \?
%   Commercial at \@     Left bracket  \[     Backslash     \\
%   Right bracket \]     Circumflex    \^     Underscore    \_
%   Grave accent  \`     Left brace    \{     Vertical bar  \|
%   Right brace   \}     Tilde         \~}
%
% \GetFileInfo{scrindex.drv}
%
% \title{The \xpackage{scrindex} package}
% \date{2008/08/11 v1.1}
% \author{Heiko Oberdiek\\\xemail{heiko.oberdiek at googlemail.com}}
%
% \maketitle
%
% \begin{abstract}
% This package redefines environment `theindex' of package \xpackage{index},
% if a class from KOMA-Script is loaded. Also option \xoption{idxtotoc}
% is supported. Index preambles can be given either by means of package
% \xpackage{index} or via the interface provided by KOMA-Script.
% \end{abstract}
%
% \tableofcontents
%
% \section{Documentation}
%
% Package \xpackage{index}, written by David M.\ Jones, detects
% the standard classes |article|, |report|, and |book|. It
% redefines environment `theindex' for its needs.
% However, it does not know other classes such as KOMA-Script.
% This package closes the compatibiliy gap between KOMA-Script's
% classes and package \xpackage{index}.
%
% Environment |theindex| is redefined to support both package
% \xpackage{index} and KOMA-Script's classes. Thus both
% the prologe of package \xpackage{index} and the preamble
% of KOMA-Script's classes are available. Also class option |idxtotoc|
% of KOMA-Script is supported.
%
% \subsection{Usage}
%
% The package \xpackage{scrindex} is loaded without options:
%\begin{quote}
%\begin{verbatim}
%\usepackage{scrindex}
%\end{verbatim}
%\end{quote}
%
% It loads package \xpackage{index} and requests version 2004/01/20
% or later. \LaTeX's package interface allows multiple calls
% of the same package. The package is loaded at its first
% package loading command. At later times \LaTeX\ only checks
% options and a requested version date. Therefore it does not harm
% to add |\usepackage{index}| before or after |\usepackage{scrindex}|.
%
% Also the class does not matter. Environment |theindex| is only
% redefined for a supported class:
% \begin{itemize}
% \item |scrartcl|
% \item |scrreprt|
% \item |scrbook|
% \end{itemize}
%
% \subsection{Preambles}
%
% Both the prologue of package \xpackage{index} and the preamble
% of KOMA-Script's classes are supported. The position depends on
% the class.
%
% \subsubsection{Class \xclass{scrartcl}}
%
%    \begin{macrocode}
%<*example1>
\documentclass{scrartcl}
\usepackage{scrindex}
\setindexpreamble{Preamble of \texttt{scrartcl}\dotfill EOL}
\makeindex
\begin{document}
\section{First Section}
\index{first}
\index{section}
\printindex[default]%
  [Prologue of package \texttt{index}\dotfill EOL]%
\end{document}
%</example1>
%    \end{macrocode}
% The prologue of package \xpackage{index} is first set straight
% after the section title spanning both columns.
% Then the preamble of KOMA-Script follows
% in the first left column.
%
% \medskip
% \begin{quote}
%   \renewcommand*{\arraystretch}{1.2}
%   \begin{tabular}{|p{.45\linewidth}|p{.45\linewidth}|}
%   \hline
%   \multicolumn{2}{|l|}{\textbf{Index}}\\[1ex]
%   \multicolumn{2}{|p{.9\linewidth+2\tabcolsep}|}{^^A
%     Prologue of package \texttt{index}\dotfill EOL^^A
%   }\\[1ex]
%   \hline
%   Preamble of \texttt{scrartcl}\dotfill EOL&\\
%   first, 1&\\
%   section, 1&\\
%   \hline
%   \end{tabular}
% \end{quote}
%
% \subsubsection{Classes \xclass{scrreprt} and \xclass{scrbook}}
%
%    \begin{macrocode}
%<*example2>
\documentclass[openany]{scrbook}% or scrreprt
\usepackage{scrindex}
\setindexpreamble{Preamble of class \texttt{scrbook}\dotfill EOL}
\makeindex
\begin{document}
\chapter{First Chapter}
\index{first}
\index{chapter}
\printindex[default]%
  [Prologue of package \texttt{index}\dotfill EOL]%
\end{document}
%</example2>
%    \end{macrocode}
% The order of the two preambles are different for the classes
% \xclass{scrreprt} and \xclass{scrbook}. First KOMA-Script's
% chapter preamble is set, then the prologue of package \xpackage{index}
% follows. Both are set spanning both columns.
%
% \medskip
% \begin{quote}
%   \renewcommand*{\arraystretch}{1.2}
%   \begin{tabular}{|p{.45\linewidth}|p{.45\linewidth}|}
%   \hline
%   \multicolumn{2}{|l|}{\textbf{Index}}\\[1ex]
%   \multicolumn{2}{|p{.9\linewidth+2\tabcolsep}|}{^^A
%     Preamble of class \texttt{scrbook}\dotfill EOL^^A
%   }\\
%   \multicolumn{2}{|p{.9\linewidth+2\tabcolsep}|}{^^A
%     Prologue of package \xpackage{index}\dotfill EOL^^A
%   }\\[1ex]
%   \hline
%   chapter, 1&\\
%   first, 1&\\
%   \hline
%   \end{tabular}
% \end{quote}
%
% \StopEventually{
% }
%
% \section{Implementation}
%
%    \begin{macrocode}
%<*package>
\NeedsTeXFormat{LaTeX2e}
\ProvidesPackage{scrindex}
  [2008/08/11 v1.1 Package index with KOMA-Script classes (HO)]%
%    \end{macrocode}
%
%    \begin{macrocode}
\RequirePackage{index}[2004/01/20]%
%    \end{macrocode}
%
%    \begin{macrocode}
\@ifclassloaded{scrartcl}{%
  \renewenvironment{theindex}{%
    \edef\indexname{%
      \the\@nameuse{idxtitle@\@indextype}%
    }%
    \if@twocolumn
      \@restonecolfalse
    \else
      \@restonecoltrue
    \fi
    \idx@heading
    \thispagestyle{\indexpagestyle}%
    \columnseprule\z@
    \columnsep 35\p@
    \index@preamble\par\nobreak
    \parindent\z@
    \parskip\z@ \@plus .3\p@\relax
    \parfillskip\z@ \@plus 1fil\relax
    \let\item\@idxitem
  }{%
    \if@restonecol
      \onecolumn
    \else
      \clearpage
    \fi
  }%
  \@ifclasswith{scrartcl}{idxtotoc}{%
    \renewcommand*{\idx@heading}{%
      \twocolumn[%
        \addsec{\indexname}%
        \ifx\index@prologue\@empty
        \else
          \index@prologue
          \bigskip
        \fi
      ]%
      \@mkboth{\indexname}{\indexname}%
    }%
  }{%
    \renewcommand*{\idx@heading}{%
      \twocolumn[%
        \section*{\indexname}%
        \ifx\index@prologue\@empty
        \else
          \index@prologue
          \bigskip
        \fi
      ]%
      \@mkboth{\indexname}{\indexname}%
    }%
  }%
}{}
%    \end{macrocode}
%    \begin{macrocode}
\@ifclassloaded{scrreprt}{%
  \renewenvironment{theindex}{%
    \edef\indexname{%
      \the\@nameuse{idxtitle@\@indextype}%
    }%
    \if@twocolumn
      \@restonecolfalse
    \else
      \@restonecoltrue
    \fi
    \setchapterpreamble{\index@preamble}%
    \idx@heading
    \thispagestyle{\indexpagestyle}%
    \columnseprule\z@
    \columnsep 35\p@
    \parindent\z@
    \parskip\z@ \@plus .3\p@\relax
    \parfillskip\z@ \@plus 1fil\relax
    \let\item\@idxitem
  }{%
    \if@restonecol
      \onecolumn
    \else
      \clearpage
    \fi
  }%
  \@ifclasswith{scrreprt}{idxtotoc}{%
    \renewcommand*{\idx@heading}{%
      \if@openright
        \cleardoublepage
      \else
        \clearpage
      \fi
      \twocolumn[%
        \addchap{\indexname}%
        \ifx\index@prologue\@empty
        \else
          \index@prologue
          \bigskip
        \fi
      ]%
      \@mkboth{\indexname}{\indexname}%
    }%
  }{%
    \renewcommand*{\idx@heading}{%
      \if@openright
        \cleardoublepage
      \else
        \clearpage
      \fi
      \twocolumn[%
        \chapter*{\indexname}%
        \ifx\index@prologue\@empty
        \else
          \index@prologue
          \bigskip
        \fi
      ]%
      \@mkboth{\indexname}{\indexname}%
    }%
  }%
}{}
%    \end{macrocode}
%    \begin{macrocode}
\@ifclassloaded{scrbook}{%
  \renewenvironment{theindex}{%
    \edef\indexname{%
      \the\@nameuse{idxtitle@\@indextype}%
    }%
    \if@twocolumn
      \@restonecolfalse
    \else
      \@restonecoltrue
    \fi
    \setchapterpreamble{\index@preamble}%
    \idx@heading
    \thispagestyle{\indexpagestyle}%
    \columnseprule\z@
    \columnsep 35\p@
    \parindent\z@
    \parskip\z@ \@plus .3\p@\relax
    \parfillskip\z@ \@plus 1fil\relax
    \let\item\@idxitem
  }{%
    \if@restonecol
      \onecolumn
    \else
      \clearpage
    \fi
  }%
  \@ifclasswith{scrbook}{idxtotoc}{%
    \renewcommand*{\idx@heading}{%
      \if@openright
        \cleardoublepage
      \else
        \clearpage
      \fi
      \twocolumn[%
        \addchap{\indexname}%
        \ifx\index@prologue\@empty
        \else
          \index@prologue
          \bigskip
        \fi
      ]%
      \@mkboth{\indexname}{\indexname}%
    }%
  }{%
    \renewcommand*{\idx@heading}{%
      \if@openright
        \cleardoublepage
      \else
        \clearpage
      \fi
      \twocolumn[%
        \chapter*{\indexname}%
        \ifx\index@prologue\@empty
        \else
          \index@prologue
          \bigskip
        \fi
      ]%
      \@mkboth{\indexname}{\indexname}%
    }%
  }%
}{}
%    \end{macrocode}
%
%    \begin{macrocode}
%</package>
%    \end{macrocode}
%
% \section{Installation}
%
% \subsection{Download}
%
% \paragraph{Package.} This package is available on
% CTAN\footnote{\url{ftp://ftp.ctan.org/tex-archive/}}:
% \begin{description}
% \item[\CTAN{macros/latex/contrib/oberdiek/scrindex.dtx}] The source file.
% \item[\CTAN{macros/latex/contrib/oberdiek/scrindex.pdf}] Documentation.
% \end{description}
%
%
% \paragraph{Bundle.} All the packages of the bundle `oberdiek'
% are also available in a TDS compliant ZIP archive. There
% the packages are already unpacked and the documentation files
% are generated. The files and directories obey the TDS standard.
% \begin{description}
% \item[\CTAN{install/macros/latex/contrib/oberdiek.tds.zip}]
% \end{description}
% \emph{TDS} refers to the standard ``A Directory Structure
% for \TeX\ Files'' (\CTAN{tds/tds.pdf}). Directories
% with \xfile{texmf} in their name are usually organized this way.
%
% \subsection{Bundle installation}
%
% \paragraph{Unpacking.} Unpack the \xfile{oberdiek.tds.zip} in the
% TDS tree (also known as \xfile{texmf} tree) of your choice.
% Example (linux):
% \begin{quote}
%   |unzip oberdiek.tds.zip -d ~/texmf|
% \end{quote}
%
% \paragraph{Script installation.}
% Check the directory \xfile{TDS:scripts/oberdiek/} for
% scripts that need further installation steps.
% Package \xpackage{attachfile2} comes with the Perl script
% \xfile{pdfatfi.pl} that should be installed in such a way
% that it can be called as \texttt{pdfatfi}.
% Example (linux):
% \begin{quote}
%   |chmod +x scripts/oberdiek/pdfatfi.pl|\\
%   |cp scripts/oberdiek/pdfatfi.pl /usr/local/bin/|
% \end{quote}
%
% \subsection{Package installation}
%
% \paragraph{Unpacking.} The \xfile{.dtx} file is a self-extracting
% \docstrip\ archive. The files are extracted by running the
% \xfile{.dtx} through \plainTeX:
% \begin{quote}
%   \verb|tex scrindex.dtx|
% \end{quote}
%
% \paragraph{TDS.} Now the different files must be moved into
% the different directories in your installation TDS tree
% (also known as \xfile{texmf} tree):
% \begin{quote}
% \def\t{^^A
% \begin{tabular}{@{}>{\ttfamily}l@{ $\rightarrow$ }>{\ttfamily}l@{}}
%   scrindex.sty & tex/latex/oberdiek/scrindex.sty\\
%   scrindex.pdf & doc/latex/oberdiek/scrindex.pdf\\
%   scrindex-example1.tex & doc/latex/oberdiek/scrindex-example1.tex\\
%   scrindex-example2.tex & doc/latex/oberdiek/scrindex-example2.tex\\
%   scrindex.dtx & source/latex/oberdiek/scrindex.dtx\\
% \end{tabular}^^A
% }^^A
% \sbox0{\t}^^A
% \ifdim\wd0>\linewidth
%   \begingroup
%     \advance\linewidth by\leftmargin
%     \advance\linewidth by\rightmargin
%   \edef\x{\endgroup
%     \def\noexpand\lw{\the\linewidth}^^A
%   }\x
%   \def\lwbox{^^A
%     \leavevmode
%     \hbox to \linewidth{^^A
%       \kern-\leftmargin\relax
%       \hss
%       \usebox0
%       \hss
%       \kern-\rightmargin\relax
%     }^^A
%   }^^A
%   \ifdim\wd0>\lw
%     \sbox0{\small\t}^^A
%     \ifdim\wd0>\linewidth
%       \ifdim\wd0>\lw
%         \sbox0{\footnotesize\t}^^A
%         \ifdim\wd0>\linewidth
%           \ifdim\wd0>\lw
%             \sbox0{\scriptsize\t}^^A
%             \ifdim\wd0>\linewidth
%               \ifdim\wd0>\lw
%                 \sbox0{\tiny\t}^^A
%                 \ifdim\wd0>\linewidth
%                   \lwbox
%                 \else
%                   \usebox0
%                 \fi
%               \else
%                 \lwbox
%               \fi
%             \else
%               \usebox0
%             \fi
%           \else
%             \lwbox
%           \fi
%         \else
%           \usebox0
%         \fi
%       \else
%         \lwbox
%       \fi
%     \else
%       \usebox0
%     \fi
%   \else
%     \lwbox
%   \fi
% \else
%   \usebox0
% \fi
% \end{quote}
% If you have a \xfile{docstrip.cfg} that configures and enables \docstrip's
% TDS installing feature, then some files can already be in the right
% place, see the documentation of \docstrip.
%
% \subsection{Refresh file name databases}
%
% If your \TeX~distribution
% (\teTeX, \mikTeX, \dots) relies on file name databases, you must refresh
% these. For example, \teTeX\ users run \verb|texhash| or
% \verb|mktexlsr|.
%
% \subsection{Some details for the interested}
%
% \paragraph{Attached source.}
%
% The PDF documentation on CTAN also includes the
% \xfile{.dtx} source file. It can be extracted by
% AcrobatReader 6 or higher. Another option is \textsf{pdftk},
% e.g. unpack the file into the current directory:
% \begin{quote}
%   \verb|pdftk scrindex.pdf unpack_files output .|
% \end{quote}
%
% \paragraph{Unpacking with \LaTeX.}
% The \xfile{.dtx} chooses its action depending on the format:
% \begin{description}
% \item[\plainTeX:] Run \docstrip\ and extract the files.
% \item[\LaTeX:] Generate the documentation.
% \end{description}
% If you insist on using \LaTeX\ for \docstrip\ (really,
% \docstrip\ does not need \LaTeX), then inform the autodetect routine
% about your intention:
% \begin{quote}
%   \verb|latex \let\install=y\input{scrindex.dtx}|
% \end{quote}
% Do not forget to quote the argument according to the demands
% of your shell.
%
% \paragraph{Generating the documentation.}
% You can use both the \xfile{.dtx} or the \xfile{.drv} to generate
% the documentation. The process can be configured by the
% configuration file \xfile{ltxdoc.cfg}. For instance, put this
% line into this file, if you want to have A4 as paper format:
% \begin{quote}
%   \verb|\PassOptionsToClass{a4paper}{article}|
% \end{quote}
% An example follows how to generate the
% documentation with pdf\LaTeX:
% \begin{quote}
%\begin{verbatim}
%pdflatex scrindex.dtx
%makeindex -s gind.ist scrindex.idx
%pdflatex scrindex.dtx
%makeindex -s gind.ist scrindex.idx
%pdflatex scrindex.dtx
%\end{verbatim}
% \end{quote}
%
% \section{Catalogue}
%
% The following XML file can be used as source for the
% \href{http://mirror.ctan.org/help/Catalogue/catalogue.html}{\TeX\ Catalogue}.
% The elements \texttt{caption} and \texttt{description} are imported
% from the original XML file from the Catalogue.
% The name of the XML file in the Catalogue is \xfile{scrindex.xml}.
%    \begin{macrocode}
%<*catalogue>
<?xml version='1.0' encoding='us-ascii'?>
<!DOCTYPE entry SYSTEM 'catalogue.dtd'>
<entry datestamp='$Date$' modifier='$Author$' id='scrindex'>
  <name>scrindex</name>
  <caption>Make index package work with Koma-script classes.</caption>
  <authorref id='auth:oberdiek'/>
  <copyright owner='Heiko Oberdiek' year='2008'/>
  <license type='lppl1.3'/>
  <version number='1.1'/>
  <description>
    This package redefines environment `theindex' of package `index',
    if a class from <xref refid='koma-script'>KOMA-Script</xref> is loaded.
    Also option `idxtotoc' is supported. Index preambles can be given
    either by means of package `index' or via the interface provided
    by <xref refid='koma-script'>KOMA-Script</xref>.
    <p/>
    The package is part of the <xref refid='oberdiek'>oberdiek</xref>
    bundle.
  </description>
  <documentation details='Package documentation'
      href='ctan:/macros/latex/contrib/oberdiek/scrindex.pdf'/>
  <ctan file='true' path='/macros/latex/contrib/oberdiek/scrindex.dtx'/>
  <miktex location='oberdiek'/>
  <texlive location='oberdiek'/>
  <install path='/macros/latex/contrib/oberdiek/oberdiek.tds.zip'/>
</entry>
%</catalogue>
%    \end{macrocode}
%
% \begin{History}
%   \begin{Version}{2008/07/07 v1.0}
%   \item
%     First version, also published in newsgroup \xnewsgroup{de.comp.text.tex}:\\
%     \URL{``\link{Re: Z\"ahler bei \cs{index}}''}^^A
%     {http://groups.google.com/group/de.comp.text.tex/msg/39575b5e2f29be1e}
%   \end{Version}
%   \begin{Version}{2008/08/11 v1.1}
%   \item
%     Code is not changed.
%   \item
%     URLs updated.
%   \end{Version}
% \end{History}
%
% \PrintIndex
%
% \Finale
\endinput

%        (quote the arguments according to the demands of your shell)
%
% Documentation:
%    (a) If scrindex.drv is present:
%           latex scrindex.drv
%    (b) Without scrindex.drv:
%           latex scrindex.dtx; ...
%    The class ltxdoc loads the configuration file ltxdoc.cfg
%    if available. Here you can specify further options, e.g.
%    use A4 as paper format:
%       \PassOptionsToClass{a4paper}{article}
%
%    Programm calls to get the documentation (example):
%       pdflatex scrindex.dtx
%       makeindex -s gind.ist scrindex.idx
%       pdflatex scrindex.dtx
%       makeindex -s gind.ist scrindex.idx
%       pdflatex scrindex.dtx
%
% Installation:
%    TDS:tex/latex/oberdiek/scrindex.sty
%    TDS:doc/latex/oberdiek/scrindex.pdf
%    TDS:doc/latex/oberdiek/scrindex-example1.tex
%    TDS:doc/latex/oberdiek/scrindex-example2.tex
%    TDS:source/latex/oberdiek/scrindex.dtx
%
%<*ignore>
\begingroup
  \catcode123=1 %
  \catcode125=2 %
  \def\x{LaTeX2e}%
\expandafter\endgroup
\ifcase 0\ifx\install y1\fi\expandafter
         \ifx\csname processbatchFile\endcsname\relax\else1\fi
         \ifx\fmtname\x\else 1\fi\relax
\else\csname fi\endcsname
%</ignore>
%<*install>
\input docstrip.tex
\Msg{************************************************************************}
\Msg{* Installation}
\Msg{* Package: scrindex 2008/08/11 v1.1 Package index with KOMA-Script classes (HO)}
\Msg{************************************************************************}

\keepsilent
\askforoverwritefalse

\let\MetaPrefix\relax
\preamble

This is a generated file.

Project: scrindex
Version: 2008/08/11 v1.1

Copyright (C) 2008 by
   Heiko Oberdiek <heiko.oberdiek at googlemail.com>

This work may be distributed and/or modified under the
conditions of the LaTeX Project Public License, either
version 1.3c of this license or (at your option) any later
version. This version of this license is in
   http://www.latex-project.org/lppl/lppl-1-3c.txt
and the latest version of this license is in
   http://www.latex-project.org/lppl.txt
and version 1.3 or later is part of all distributions of
LaTeX version 2005/12/01 or later.

This work has the LPPL maintenance status "maintained".

This Current Maintainer of this work is Heiko Oberdiek.

This work consists of the main source file scrindex.dtx
and the derived files
   scrindex.sty, scrindex.pdf, scrindex.ins, scrindex.drv,
   scrindex-example1.tex, scrindex-example2.tex.

\endpreamble
\let\MetaPrefix\DoubleperCent

\generate{%
  \file{scrindex.ins}{\from{scrindex.dtx}{install}}%
  \file{scrindex.drv}{\from{scrindex.dtx}{driver}}%
  \usedir{tex/latex/oberdiek}%
  \file{scrindex.sty}{\from{scrindex.dtx}{package}}%
  \usedir{doc/latex/oberdiek}%
  \file{scrindex-example1.tex}{\from{scrindex.dtx}{example1}}%
  \file{scrindex-example2.tex}{\from{scrindex.dtx}{example2}}%
  \nopreamble
  \nopostamble
  \usedir{source/latex/oberdiek/catalogue}%
  \file{scrindex.xml}{\from{scrindex.dtx}{catalogue}}%
}

\catcode32=13\relax% active space
\let =\space%
\Msg{************************************************************************}
\Msg{*}
\Msg{* To finish the installation you have to move the following}
\Msg{* file into a directory searched by TeX:}
\Msg{*}
\Msg{*     scrindex.sty}
\Msg{*}
\Msg{* To produce the documentation run the file `scrindex.drv'}
\Msg{* through LaTeX.}
\Msg{*}
\Msg{* Happy TeXing!}
\Msg{*}
\Msg{************************************************************************}

\endbatchfile
%</install>
%<*ignore>
\fi
%</ignore>
%<*driver>
\NeedsTeXFormat{LaTeX2e}
\ProvidesFile{scrindex.drv}%
  [2008/08/11 v1.1 Package index with KOMA-Script classes (HO)]%
\documentclass{ltxdoc}
\usepackage{holtxdoc}[2011/11/22]
\usepackage{calc}
\begin{document}
  \DocInput{scrindex.dtx}%
\end{document}
%</driver>
% \fi
%
% \CheckSum{237}
%
% \CharacterTable
%  {Upper-case    \A\B\C\D\E\F\G\H\I\J\K\L\M\N\O\P\Q\R\S\T\U\V\W\X\Y\Z
%   Lower-case    \a\b\c\d\e\f\g\h\i\j\k\l\m\n\o\p\q\r\s\t\u\v\w\x\y\z
%   Digits        \0\1\2\3\4\5\6\7\8\9
%   Exclamation   \!     Double quote  \"     Hash (number) \#
%   Dollar        \$     Percent       \%     Ampersand     \&
%   Acute accent  \'     Left paren    \(     Right paren   \)
%   Asterisk      \*     Plus          \+     Comma         \,
%   Minus         \-     Point         \.     Solidus       \/
%   Colon         \:     Semicolon     \;     Less than     \<
%   Equals        \=     Greater than  \>     Question mark \?
%   Commercial at \@     Left bracket  \[     Backslash     \\
%   Right bracket \]     Circumflex    \^     Underscore    \_
%   Grave accent  \`     Left brace    \{     Vertical bar  \|
%   Right brace   \}     Tilde         \~}
%
% \GetFileInfo{scrindex.drv}
%
% \title{The \xpackage{scrindex} package}
% \date{2008/08/11 v1.1}
% \author{Heiko Oberdiek\\\xemail{heiko.oberdiek at googlemail.com}}
%
% \maketitle
%
% \begin{abstract}
% This package redefines environment `theindex' of package \xpackage{index},
% if a class from KOMA-Script is loaded. Also option \xoption{idxtotoc}
% is supported. Index preambles can be given either by means of package
% \xpackage{index} or via the interface provided by KOMA-Script.
% \end{abstract}
%
% \tableofcontents
%
% \section{Documentation}
%
% Package \xpackage{index}, written by David M.\ Jones, detects
% the standard classes |article|, |report|, and |book|. It
% redefines environment `theindex' for its needs.
% However, it does not know other classes such as KOMA-Script.
% This package closes the compatibiliy gap between KOMA-Script's
% classes and package \xpackage{index}.
%
% Environment |theindex| is redefined to support both package
% \xpackage{index} and KOMA-Script's classes. Thus both
% the prologe of package \xpackage{index} and the preamble
% of KOMA-Script's classes are available. Also class option |idxtotoc|
% of KOMA-Script is supported.
%
% \subsection{Usage}
%
% The package \xpackage{scrindex} is loaded without options:
%\begin{quote}
%\begin{verbatim}
%\usepackage{scrindex}
%\end{verbatim}
%\end{quote}
%
% It loads package \xpackage{index} and requests version 2004/01/20
% or later. \LaTeX's package interface allows multiple calls
% of the same package. The package is loaded at its first
% package loading command. At later times \LaTeX\ only checks
% options and a requested version date. Therefore it does not harm
% to add |\usepackage{index}| before or after |\usepackage{scrindex}|.
%
% Also the class does not matter. Environment |theindex| is only
% redefined for a supported class:
% \begin{itemize}
% \item |scrartcl|
% \item |scrreprt|
% \item |scrbook|
% \end{itemize}
%
% \subsection{Preambles}
%
% Both the prologue of package \xpackage{index} and the preamble
% of KOMA-Script's classes are supported. The position depends on
% the class.
%
% \subsubsection{Class \xclass{scrartcl}}
%
%    \begin{macrocode}
%<*example1>
\documentclass{scrartcl}
\usepackage{scrindex}
\setindexpreamble{Preamble of \texttt{scrartcl}\dotfill EOL}
\makeindex
\begin{document}
\section{First Section}
\index{first}
\index{section}
\printindex[default]%
  [Prologue of package \texttt{index}\dotfill EOL]%
\end{document}
%</example1>
%    \end{macrocode}
% The prologue of package \xpackage{index} is first set straight
% after the section title spanning both columns.
% Then the preamble of KOMA-Script follows
% in the first left column.
%
% \medskip
% \begin{quote}
%   \renewcommand*{\arraystretch}{1.2}
%   \begin{tabular}{|p{.45\linewidth}|p{.45\linewidth}|}
%   \hline
%   \multicolumn{2}{|l|}{\textbf{Index}}\\[1ex]
%   \multicolumn{2}{|p{.9\linewidth+2\tabcolsep}|}{^^A
%     Prologue of package \texttt{index}\dotfill EOL^^A
%   }\\[1ex]
%   \hline
%   Preamble of \texttt{scrartcl}\dotfill EOL&\\
%   first, 1&\\
%   section, 1&\\
%   \hline
%   \end{tabular}
% \end{quote}
%
% \subsubsection{Classes \xclass{scrreprt} and \xclass{scrbook}}
%
%    \begin{macrocode}
%<*example2>
\documentclass[openany]{scrbook}% or scrreprt
\usepackage{scrindex}
\setindexpreamble{Preamble of class \texttt{scrbook}\dotfill EOL}
\makeindex
\begin{document}
\chapter{First Chapter}
\index{first}
\index{chapter}
\printindex[default]%
  [Prologue of package \texttt{index}\dotfill EOL]%
\end{document}
%</example2>
%    \end{macrocode}
% The order of the two preambles are different for the classes
% \xclass{scrreprt} and \xclass{scrbook}. First KOMA-Script's
% chapter preamble is set, then the prologue of package \xpackage{index}
% follows. Both are set spanning both columns.
%
% \medskip
% \begin{quote}
%   \renewcommand*{\arraystretch}{1.2}
%   \begin{tabular}{|p{.45\linewidth}|p{.45\linewidth}|}
%   \hline
%   \multicolumn{2}{|l|}{\textbf{Index}}\\[1ex]
%   \multicolumn{2}{|p{.9\linewidth+2\tabcolsep}|}{^^A
%     Preamble of class \texttt{scrbook}\dotfill EOL^^A
%   }\\
%   \multicolumn{2}{|p{.9\linewidth+2\tabcolsep}|}{^^A
%     Prologue of package \xpackage{index}\dotfill EOL^^A
%   }\\[1ex]
%   \hline
%   chapter, 1&\\
%   first, 1&\\
%   \hline
%   \end{tabular}
% \end{quote}
%
% \StopEventually{
% }
%
% \section{Implementation}
%
%    \begin{macrocode}
%<*package>
\NeedsTeXFormat{LaTeX2e}
\ProvidesPackage{scrindex}
  [2008/08/11 v1.1 Package index with KOMA-Script classes (HO)]%
%    \end{macrocode}
%
%    \begin{macrocode}
\RequirePackage{index}[2004/01/20]%
%    \end{macrocode}
%
%    \begin{macrocode}
\@ifclassloaded{scrartcl}{%
  \renewenvironment{theindex}{%
    \edef\indexname{%
      \the\@nameuse{idxtitle@\@indextype}%
    }%
    \if@twocolumn
      \@restonecolfalse
    \else
      \@restonecoltrue
    \fi
    \idx@heading
    \thispagestyle{\indexpagestyle}%
    \columnseprule\z@
    \columnsep 35\p@
    \index@preamble\par\nobreak
    \parindent\z@
    \parskip\z@ \@plus .3\p@\relax
    \parfillskip\z@ \@plus 1fil\relax
    \let\item\@idxitem
  }{%
    \if@restonecol
      \onecolumn
    \else
      \clearpage
    \fi
  }%
  \@ifclasswith{scrartcl}{idxtotoc}{%
    \renewcommand*{\idx@heading}{%
      \twocolumn[%
        \addsec{\indexname}%
        \ifx\index@prologue\@empty
        \else
          \index@prologue
          \bigskip
        \fi
      ]%
      \@mkboth{\indexname}{\indexname}%
    }%
  }{%
    \renewcommand*{\idx@heading}{%
      \twocolumn[%
        \section*{\indexname}%
        \ifx\index@prologue\@empty
        \else
          \index@prologue
          \bigskip
        \fi
      ]%
      \@mkboth{\indexname}{\indexname}%
    }%
  }%
}{}
%    \end{macrocode}
%    \begin{macrocode}
\@ifclassloaded{scrreprt}{%
  \renewenvironment{theindex}{%
    \edef\indexname{%
      \the\@nameuse{idxtitle@\@indextype}%
    }%
    \if@twocolumn
      \@restonecolfalse
    \else
      \@restonecoltrue
    \fi
    \setchapterpreamble{\index@preamble}%
    \idx@heading
    \thispagestyle{\indexpagestyle}%
    \columnseprule\z@
    \columnsep 35\p@
    \parindent\z@
    \parskip\z@ \@plus .3\p@\relax
    \parfillskip\z@ \@plus 1fil\relax
    \let\item\@idxitem
  }{%
    \if@restonecol
      \onecolumn
    \else
      \clearpage
    \fi
  }%
  \@ifclasswith{scrreprt}{idxtotoc}{%
    \renewcommand*{\idx@heading}{%
      \if@openright
        \cleardoublepage
      \else
        \clearpage
      \fi
      \twocolumn[%
        \addchap{\indexname}%
        \ifx\index@prologue\@empty
        \else
          \index@prologue
          \bigskip
        \fi
      ]%
      \@mkboth{\indexname}{\indexname}%
    }%
  }{%
    \renewcommand*{\idx@heading}{%
      \if@openright
        \cleardoublepage
      \else
        \clearpage
      \fi
      \twocolumn[%
        \chapter*{\indexname}%
        \ifx\index@prologue\@empty
        \else
          \index@prologue
          \bigskip
        \fi
      ]%
      \@mkboth{\indexname}{\indexname}%
    }%
  }%
}{}
%    \end{macrocode}
%    \begin{macrocode}
\@ifclassloaded{scrbook}{%
  \renewenvironment{theindex}{%
    \edef\indexname{%
      \the\@nameuse{idxtitle@\@indextype}%
    }%
    \if@twocolumn
      \@restonecolfalse
    \else
      \@restonecoltrue
    \fi
    \setchapterpreamble{\index@preamble}%
    \idx@heading
    \thispagestyle{\indexpagestyle}%
    \columnseprule\z@
    \columnsep 35\p@
    \parindent\z@
    \parskip\z@ \@plus .3\p@\relax
    \parfillskip\z@ \@plus 1fil\relax
    \let\item\@idxitem
  }{%
    \if@restonecol
      \onecolumn
    \else
      \clearpage
    \fi
  }%
  \@ifclasswith{scrbook}{idxtotoc}{%
    \renewcommand*{\idx@heading}{%
      \if@openright
        \cleardoublepage
      \else
        \clearpage
      \fi
      \twocolumn[%
        \addchap{\indexname}%
        \ifx\index@prologue\@empty
        \else
          \index@prologue
          \bigskip
        \fi
      ]%
      \@mkboth{\indexname}{\indexname}%
    }%
  }{%
    \renewcommand*{\idx@heading}{%
      \if@openright
        \cleardoublepage
      \else
        \clearpage
      \fi
      \twocolumn[%
        \chapter*{\indexname}%
        \ifx\index@prologue\@empty
        \else
          \index@prologue
          \bigskip
        \fi
      ]%
      \@mkboth{\indexname}{\indexname}%
    }%
  }%
}{}
%    \end{macrocode}
%
%    \begin{macrocode}
%</package>
%    \end{macrocode}
%
% \section{Installation}
%
% \subsection{Download}
%
% \paragraph{Package.} This package is available on
% CTAN\footnote{\url{ftp://ftp.ctan.org/tex-archive/}}:
% \begin{description}
% \item[\CTAN{macros/latex/contrib/oberdiek/scrindex.dtx}] The source file.
% \item[\CTAN{macros/latex/contrib/oberdiek/scrindex.pdf}] Documentation.
% \end{description}
%
%
% \paragraph{Bundle.} All the packages of the bundle `oberdiek'
% are also available in a TDS compliant ZIP archive. There
% the packages are already unpacked and the documentation files
% are generated. The files and directories obey the TDS standard.
% \begin{description}
% \item[\CTAN{install/macros/latex/contrib/oberdiek.tds.zip}]
% \end{description}
% \emph{TDS} refers to the standard ``A Directory Structure
% for \TeX\ Files'' (\CTAN{tds/tds.pdf}). Directories
% with \xfile{texmf} in their name are usually organized this way.
%
% \subsection{Bundle installation}
%
% \paragraph{Unpacking.} Unpack the \xfile{oberdiek.tds.zip} in the
% TDS tree (also known as \xfile{texmf} tree) of your choice.
% Example (linux):
% \begin{quote}
%   |unzip oberdiek.tds.zip -d ~/texmf|
% \end{quote}
%
% \paragraph{Script installation.}
% Check the directory \xfile{TDS:scripts/oberdiek/} for
% scripts that need further installation steps.
% Package \xpackage{attachfile2} comes with the Perl script
% \xfile{pdfatfi.pl} that should be installed in such a way
% that it can be called as \texttt{pdfatfi}.
% Example (linux):
% \begin{quote}
%   |chmod +x scripts/oberdiek/pdfatfi.pl|\\
%   |cp scripts/oberdiek/pdfatfi.pl /usr/local/bin/|
% \end{quote}
%
% \subsection{Package installation}
%
% \paragraph{Unpacking.} The \xfile{.dtx} file is a self-extracting
% \docstrip\ archive. The files are extracted by running the
% \xfile{.dtx} through \plainTeX:
% \begin{quote}
%   \verb|tex scrindex.dtx|
% \end{quote}
%
% \paragraph{TDS.} Now the different files must be moved into
% the different directories in your installation TDS tree
% (also known as \xfile{texmf} tree):
% \begin{quote}
% \def\t{^^A
% \begin{tabular}{@{}>{\ttfamily}l@{ $\rightarrow$ }>{\ttfamily}l@{}}
%   scrindex.sty & tex/latex/oberdiek/scrindex.sty\\
%   scrindex.pdf & doc/latex/oberdiek/scrindex.pdf\\
%   scrindex-example1.tex & doc/latex/oberdiek/scrindex-example1.tex\\
%   scrindex-example2.tex & doc/latex/oberdiek/scrindex-example2.tex\\
%   scrindex.dtx & source/latex/oberdiek/scrindex.dtx\\
% \end{tabular}^^A
% }^^A
% \sbox0{\t}^^A
% \ifdim\wd0>\linewidth
%   \begingroup
%     \advance\linewidth by\leftmargin
%     \advance\linewidth by\rightmargin
%   \edef\x{\endgroup
%     \def\noexpand\lw{\the\linewidth}^^A
%   }\x
%   \def\lwbox{^^A
%     \leavevmode
%     \hbox to \linewidth{^^A
%       \kern-\leftmargin\relax
%       \hss
%       \usebox0
%       \hss
%       \kern-\rightmargin\relax
%     }^^A
%   }^^A
%   \ifdim\wd0>\lw
%     \sbox0{\small\t}^^A
%     \ifdim\wd0>\linewidth
%       \ifdim\wd0>\lw
%         \sbox0{\footnotesize\t}^^A
%         \ifdim\wd0>\linewidth
%           \ifdim\wd0>\lw
%             \sbox0{\scriptsize\t}^^A
%             \ifdim\wd0>\linewidth
%               \ifdim\wd0>\lw
%                 \sbox0{\tiny\t}^^A
%                 \ifdim\wd0>\linewidth
%                   \lwbox
%                 \else
%                   \usebox0
%                 \fi
%               \else
%                 \lwbox
%               \fi
%             \else
%               \usebox0
%             \fi
%           \else
%             \lwbox
%           \fi
%         \else
%           \usebox0
%         \fi
%       \else
%         \lwbox
%       \fi
%     \else
%       \usebox0
%     \fi
%   \else
%     \lwbox
%   \fi
% \else
%   \usebox0
% \fi
% \end{quote}
% If you have a \xfile{docstrip.cfg} that configures and enables \docstrip's
% TDS installing feature, then some files can already be in the right
% place, see the documentation of \docstrip.
%
% \subsection{Refresh file name databases}
%
% If your \TeX~distribution
% (\teTeX, \mikTeX, \dots) relies on file name databases, you must refresh
% these. For example, \teTeX\ users run \verb|texhash| or
% \verb|mktexlsr|.
%
% \subsection{Some details for the interested}
%
% \paragraph{Attached source.}
%
% The PDF documentation on CTAN also includes the
% \xfile{.dtx} source file. It can be extracted by
% AcrobatReader 6 or higher. Another option is \textsf{pdftk},
% e.g. unpack the file into the current directory:
% \begin{quote}
%   \verb|pdftk scrindex.pdf unpack_files output .|
% \end{quote}
%
% \paragraph{Unpacking with \LaTeX.}
% The \xfile{.dtx} chooses its action depending on the format:
% \begin{description}
% \item[\plainTeX:] Run \docstrip\ and extract the files.
% \item[\LaTeX:] Generate the documentation.
% \end{description}
% If you insist on using \LaTeX\ for \docstrip\ (really,
% \docstrip\ does not need \LaTeX), then inform the autodetect routine
% about your intention:
% \begin{quote}
%   \verb|latex \let\install=y% \iffalse meta-comment
%
% File: scrindex.dtx
% Version: 2008/08/11 v1.1
% Info: Package index with KOMA-Script classes
%
% Copyright (C) 2008 by
%    Heiko Oberdiek <heiko.oberdiek at googlemail.com>
%
% This work may be distributed and/or modified under the
% conditions of the LaTeX Project Public License, either
% version 1.3c of this license or (at your option) any later
% version. This version of this license is in
%    http://www.latex-project.org/lppl/lppl-1-3c.txt
% and the latest version of this license is in
%    http://www.latex-project.org/lppl.txt
% and version 1.3 or later is part of all distributions of
% LaTeX version 2005/12/01 or later.
%
% This work has the LPPL maintenance status "maintained".
%
% This Current Maintainer of this work is Heiko Oberdiek.
%
% This work consists of the main source file scrindex.dtx
% and the derived files
%    scrindex.sty, scrindex.pdf, scrindex.ins, scrindex.drv,
%    scrindex-example1.tex, scrindex-example2.tex.
%
% Distribution:
%    CTAN:macros/latex/contrib/oberdiek/scrindex.dtx
%    CTAN:macros/latex/contrib/oberdiek/scrindex.pdf
%
% Unpacking:
%    (a) If scrindex.ins is present:
%           tex scrindex.ins
%    (b) Without scrindex.ins:
%           tex scrindex.dtx
%    (c) If you insist on using LaTeX
%           latex \let\install=y\input{scrindex.dtx}
%        (quote the arguments according to the demands of your shell)
%
% Documentation:
%    (a) If scrindex.drv is present:
%           latex scrindex.drv
%    (b) Without scrindex.drv:
%           latex scrindex.dtx; ...
%    The class ltxdoc loads the configuration file ltxdoc.cfg
%    if available. Here you can specify further options, e.g.
%    use A4 as paper format:
%       \PassOptionsToClass{a4paper}{article}
%
%    Programm calls to get the documentation (example):
%       pdflatex scrindex.dtx
%       makeindex -s gind.ist scrindex.idx
%       pdflatex scrindex.dtx
%       makeindex -s gind.ist scrindex.idx
%       pdflatex scrindex.dtx
%
% Installation:
%    TDS:tex/latex/oberdiek/scrindex.sty
%    TDS:doc/latex/oberdiek/scrindex.pdf
%    TDS:doc/latex/oberdiek/scrindex-example1.tex
%    TDS:doc/latex/oberdiek/scrindex-example2.tex
%    TDS:source/latex/oberdiek/scrindex.dtx
%
%<*ignore>
\begingroup
  \catcode123=1 %
  \catcode125=2 %
  \def\x{LaTeX2e}%
\expandafter\endgroup
\ifcase 0\ifx\install y1\fi\expandafter
         \ifx\csname processbatchFile\endcsname\relax\else1\fi
         \ifx\fmtname\x\else 1\fi\relax
\else\csname fi\endcsname
%</ignore>
%<*install>
\input docstrip.tex
\Msg{************************************************************************}
\Msg{* Installation}
\Msg{* Package: scrindex 2008/08/11 v1.1 Package index with KOMA-Script classes (HO)}
\Msg{************************************************************************}

\keepsilent
\askforoverwritefalse

\let\MetaPrefix\relax
\preamble

This is a generated file.

Project: scrindex
Version: 2008/08/11 v1.1

Copyright (C) 2008 by
   Heiko Oberdiek <heiko.oberdiek at googlemail.com>

This work may be distributed and/or modified under the
conditions of the LaTeX Project Public License, either
version 1.3c of this license or (at your option) any later
version. This version of this license is in
   http://www.latex-project.org/lppl/lppl-1-3c.txt
and the latest version of this license is in
   http://www.latex-project.org/lppl.txt
and version 1.3 or later is part of all distributions of
LaTeX version 2005/12/01 or later.

This work has the LPPL maintenance status "maintained".

This Current Maintainer of this work is Heiko Oberdiek.

This work consists of the main source file scrindex.dtx
and the derived files
   scrindex.sty, scrindex.pdf, scrindex.ins, scrindex.drv,
   scrindex-example1.tex, scrindex-example2.tex.

\endpreamble
\let\MetaPrefix\DoubleperCent

\generate{%
  \file{scrindex.ins}{\from{scrindex.dtx}{install}}%
  \file{scrindex.drv}{\from{scrindex.dtx}{driver}}%
  \usedir{tex/latex/oberdiek}%
  \file{scrindex.sty}{\from{scrindex.dtx}{package}}%
  \usedir{doc/latex/oberdiek}%
  \file{scrindex-example1.tex}{\from{scrindex.dtx}{example1}}%
  \file{scrindex-example2.tex}{\from{scrindex.dtx}{example2}}%
  \nopreamble
  \nopostamble
  \usedir{source/latex/oberdiek/catalogue}%
  \file{scrindex.xml}{\from{scrindex.dtx}{catalogue}}%
}

\catcode32=13\relax% active space
\let =\space%
\Msg{************************************************************************}
\Msg{*}
\Msg{* To finish the installation you have to move the following}
\Msg{* file into a directory searched by TeX:}
\Msg{*}
\Msg{*     scrindex.sty}
\Msg{*}
\Msg{* To produce the documentation run the file `scrindex.drv'}
\Msg{* through LaTeX.}
\Msg{*}
\Msg{* Happy TeXing!}
\Msg{*}
\Msg{************************************************************************}

\endbatchfile
%</install>
%<*ignore>
\fi
%</ignore>
%<*driver>
\NeedsTeXFormat{LaTeX2e}
\ProvidesFile{scrindex.drv}%
  [2008/08/11 v1.1 Package index with KOMA-Script classes (HO)]%
\documentclass{ltxdoc}
\usepackage{holtxdoc}[2011/11/22]
\usepackage{calc}
\begin{document}
  \DocInput{scrindex.dtx}%
\end{document}
%</driver>
% \fi
%
% \CheckSum{237}
%
% \CharacterTable
%  {Upper-case    \A\B\C\D\E\F\G\H\I\J\K\L\M\N\O\P\Q\R\S\T\U\V\W\X\Y\Z
%   Lower-case    \a\b\c\d\e\f\g\h\i\j\k\l\m\n\o\p\q\r\s\t\u\v\w\x\y\z
%   Digits        \0\1\2\3\4\5\6\7\8\9
%   Exclamation   \!     Double quote  \"     Hash (number) \#
%   Dollar        \$     Percent       \%     Ampersand     \&
%   Acute accent  \'     Left paren    \(     Right paren   \)
%   Asterisk      \*     Plus          \+     Comma         \,
%   Minus         \-     Point         \.     Solidus       \/
%   Colon         \:     Semicolon     \;     Less than     \<
%   Equals        \=     Greater than  \>     Question mark \?
%   Commercial at \@     Left bracket  \[     Backslash     \\
%   Right bracket \]     Circumflex    \^     Underscore    \_
%   Grave accent  \`     Left brace    \{     Vertical bar  \|
%   Right brace   \}     Tilde         \~}
%
% \GetFileInfo{scrindex.drv}
%
% \title{The \xpackage{scrindex} package}
% \date{2008/08/11 v1.1}
% \author{Heiko Oberdiek\\\xemail{heiko.oberdiek at googlemail.com}}
%
% \maketitle
%
% \begin{abstract}
% This package redefines environment `theindex' of package \xpackage{index},
% if a class from KOMA-Script is loaded. Also option \xoption{idxtotoc}
% is supported. Index preambles can be given either by means of package
% \xpackage{index} or via the interface provided by KOMA-Script.
% \end{abstract}
%
% \tableofcontents
%
% \section{Documentation}
%
% Package \xpackage{index}, written by David M.\ Jones, detects
% the standard classes |article|, |report|, and |book|. It
% redefines environment `theindex' for its needs.
% However, it does not know other classes such as KOMA-Script.
% This package closes the compatibiliy gap between KOMA-Script's
% classes and package \xpackage{index}.
%
% Environment |theindex| is redefined to support both package
% \xpackage{index} and KOMA-Script's classes. Thus both
% the prologe of package \xpackage{index} and the preamble
% of KOMA-Script's classes are available. Also class option |idxtotoc|
% of KOMA-Script is supported.
%
% \subsection{Usage}
%
% The package \xpackage{scrindex} is loaded without options:
%\begin{quote}
%\begin{verbatim}
%\usepackage{scrindex}
%\end{verbatim}
%\end{quote}
%
% It loads package \xpackage{index} and requests version 2004/01/20
% or later. \LaTeX's package interface allows multiple calls
% of the same package. The package is loaded at its first
% package loading command. At later times \LaTeX\ only checks
% options and a requested version date. Therefore it does not harm
% to add |\usepackage{index}| before or after |\usepackage{scrindex}|.
%
% Also the class does not matter. Environment |theindex| is only
% redefined for a supported class:
% \begin{itemize}
% \item |scrartcl|
% \item |scrreprt|
% \item |scrbook|
% \end{itemize}
%
% \subsection{Preambles}
%
% Both the prologue of package \xpackage{index} and the preamble
% of KOMA-Script's classes are supported. The position depends on
% the class.
%
% \subsubsection{Class \xclass{scrartcl}}
%
%    \begin{macrocode}
%<*example1>
\documentclass{scrartcl}
\usepackage{scrindex}
\setindexpreamble{Preamble of \texttt{scrartcl}\dotfill EOL}
\makeindex
\begin{document}
\section{First Section}
\index{first}
\index{section}
\printindex[default]%
  [Prologue of package \texttt{index}\dotfill EOL]%
\end{document}
%</example1>
%    \end{macrocode}
% The prologue of package \xpackage{index} is first set straight
% after the section title spanning both columns.
% Then the preamble of KOMA-Script follows
% in the first left column.
%
% \medskip
% \begin{quote}
%   \renewcommand*{\arraystretch}{1.2}
%   \begin{tabular}{|p{.45\linewidth}|p{.45\linewidth}|}
%   \hline
%   \multicolumn{2}{|l|}{\textbf{Index}}\\[1ex]
%   \multicolumn{2}{|p{.9\linewidth+2\tabcolsep}|}{^^A
%     Prologue of package \texttt{index}\dotfill EOL^^A
%   }\\[1ex]
%   \hline
%   Preamble of \texttt{scrartcl}\dotfill EOL&\\
%   first, 1&\\
%   section, 1&\\
%   \hline
%   \end{tabular}
% \end{quote}
%
% \subsubsection{Classes \xclass{scrreprt} and \xclass{scrbook}}
%
%    \begin{macrocode}
%<*example2>
\documentclass[openany]{scrbook}% or scrreprt
\usepackage{scrindex}
\setindexpreamble{Preamble of class \texttt{scrbook}\dotfill EOL}
\makeindex
\begin{document}
\chapter{First Chapter}
\index{first}
\index{chapter}
\printindex[default]%
  [Prologue of package \texttt{index}\dotfill EOL]%
\end{document}
%</example2>
%    \end{macrocode}
% The order of the two preambles are different for the classes
% \xclass{scrreprt} and \xclass{scrbook}. First KOMA-Script's
% chapter preamble is set, then the prologue of package \xpackage{index}
% follows. Both are set spanning both columns.
%
% \medskip
% \begin{quote}
%   \renewcommand*{\arraystretch}{1.2}
%   \begin{tabular}{|p{.45\linewidth}|p{.45\linewidth}|}
%   \hline
%   \multicolumn{2}{|l|}{\textbf{Index}}\\[1ex]
%   \multicolumn{2}{|p{.9\linewidth+2\tabcolsep}|}{^^A
%     Preamble of class \texttt{scrbook}\dotfill EOL^^A
%   }\\
%   \multicolumn{2}{|p{.9\linewidth+2\tabcolsep}|}{^^A
%     Prologue of package \xpackage{index}\dotfill EOL^^A
%   }\\[1ex]
%   \hline
%   chapter, 1&\\
%   first, 1&\\
%   \hline
%   \end{tabular}
% \end{quote}
%
% \StopEventually{
% }
%
% \section{Implementation}
%
%    \begin{macrocode}
%<*package>
\NeedsTeXFormat{LaTeX2e}
\ProvidesPackage{scrindex}
  [2008/08/11 v1.1 Package index with KOMA-Script classes (HO)]%
%    \end{macrocode}
%
%    \begin{macrocode}
\RequirePackage{index}[2004/01/20]%
%    \end{macrocode}
%
%    \begin{macrocode}
\@ifclassloaded{scrartcl}{%
  \renewenvironment{theindex}{%
    \edef\indexname{%
      \the\@nameuse{idxtitle@\@indextype}%
    }%
    \if@twocolumn
      \@restonecolfalse
    \else
      \@restonecoltrue
    \fi
    \idx@heading
    \thispagestyle{\indexpagestyle}%
    \columnseprule\z@
    \columnsep 35\p@
    \index@preamble\par\nobreak
    \parindent\z@
    \parskip\z@ \@plus .3\p@\relax
    \parfillskip\z@ \@plus 1fil\relax
    \let\item\@idxitem
  }{%
    \if@restonecol
      \onecolumn
    \else
      \clearpage
    \fi
  }%
  \@ifclasswith{scrartcl}{idxtotoc}{%
    \renewcommand*{\idx@heading}{%
      \twocolumn[%
        \addsec{\indexname}%
        \ifx\index@prologue\@empty
        \else
          \index@prologue
          \bigskip
        \fi
      ]%
      \@mkboth{\indexname}{\indexname}%
    }%
  }{%
    \renewcommand*{\idx@heading}{%
      \twocolumn[%
        \section*{\indexname}%
        \ifx\index@prologue\@empty
        \else
          \index@prologue
          \bigskip
        \fi
      ]%
      \@mkboth{\indexname}{\indexname}%
    }%
  }%
}{}
%    \end{macrocode}
%    \begin{macrocode}
\@ifclassloaded{scrreprt}{%
  \renewenvironment{theindex}{%
    \edef\indexname{%
      \the\@nameuse{idxtitle@\@indextype}%
    }%
    \if@twocolumn
      \@restonecolfalse
    \else
      \@restonecoltrue
    \fi
    \setchapterpreamble{\index@preamble}%
    \idx@heading
    \thispagestyle{\indexpagestyle}%
    \columnseprule\z@
    \columnsep 35\p@
    \parindent\z@
    \parskip\z@ \@plus .3\p@\relax
    \parfillskip\z@ \@plus 1fil\relax
    \let\item\@idxitem
  }{%
    \if@restonecol
      \onecolumn
    \else
      \clearpage
    \fi
  }%
  \@ifclasswith{scrreprt}{idxtotoc}{%
    \renewcommand*{\idx@heading}{%
      \if@openright
        \cleardoublepage
      \else
        \clearpage
      \fi
      \twocolumn[%
        \addchap{\indexname}%
        \ifx\index@prologue\@empty
        \else
          \index@prologue
          \bigskip
        \fi
      ]%
      \@mkboth{\indexname}{\indexname}%
    }%
  }{%
    \renewcommand*{\idx@heading}{%
      \if@openright
        \cleardoublepage
      \else
        \clearpage
      \fi
      \twocolumn[%
        \chapter*{\indexname}%
        \ifx\index@prologue\@empty
        \else
          \index@prologue
          \bigskip
        \fi
      ]%
      \@mkboth{\indexname}{\indexname}%
    }%
  }%
}{}
%    \end{macrocode}
%    \begin{macrocode}
\@ifclassloaded{scrbook}{%
  \renewenvironment{theindex}{%
    \edef\indexname{%
      \the\@nameuse{idxtitle@\@indextype}%
    }%
    \if@twocolumn
      \@restonecolfalse
    \else
      \@restonecoltrue
    \fi
    \setchapterpreamble{\index@preamble}%
    \idx@heading
    \thispagestyle{\indexpagestyle}%
    \columnseprule\z@
    \columnsep 35\p@
    \parindent\z@
    \parskip\z@ \@plus .3\p@\relax
    \parfillskip\z@ \@plus 1fil\relax
    \let\item\@idxitem
  }{%
    \if@restonecol
      \onecolumn
    \else
      \clearpage
    \fi
  }%
  \@ifclasswith{scrbook}{idxtotoc}{%
    \renewcommand*{\idx@heading}{%
      \if@openright
        \cleardoublepage
      \else
        \clearpage
      \fi
      \twocolumn[%
        \addchap{\indexname}%
        \ifx\index@prologue\@empty
        \else
          \index@prologue
          \bigskip
        \fi
      ]%
      \@mkboth{\indexname}{\indexname}%
    }%
  }{%
    \renewcommand*{\idx@heading}{%
      \if@openright
        \cleardoublepage
      \else
        \clearpage
      \fi
      \twocolumn[%
        \chapter*{\indexname}%
        \ifx\index@prologue\@empty
        \else
          \index@prologue
          \bigskip
        \fi
      ]%
      \@mkboth{\indexname}{\indexname}%
    }%
  }%
}{}
%    \end{macrocode}
%
%    \begin{macrocode}
%</package>
%    \end{macrocode}
%
% \section{Installation}
%
% \subsection{Download}
%
% \paragraph{Package.} This package is available on
% CTAN\footnote{\url{ftp://ftp.ctan.org/tex-archive/}}:
% \begin{description}
% \item[\CTAN{macros/latex/contrib/oberdiek/scrindex.dtx}] The source file.
% \item[\CTAN{macros/latex/contrib/oberdiek/scrindex.pdf}] Documentation.
% \end{description}
%
%
% \paragraph{Bundle.} All the packages of the bundle `oberdiek'
% are also available in a TDS compliant ZIP archive. There
% the packages are already unpacked and the documentation files
% are generated. The files and directories obey the TDS standard.
% \begin{description}
% \item[\CTAN{install/macros/latex/contrib/oberdiek.tds.zip}]
% \end{description}
% \emph{TDS} refers to the standard ``A Directory Structure
% for \TeX\ Files'' (\CTAN{tds/tds.pdf}). Directories
% with \xfile{texmf} in their name are usually organized this way.
%
% \subsection{Bundle installation}
%
% \paragraph{Unpacking.} Unpack the \xfile{oberdiek.tds.zip} in the
% TDS tree (also known as \xfile{texmf} tree) of your choice.
% Example (linux):
% \begin{quote}
%   |unzip oberdiek.tds.zip -d ~/texmf|
% \end{quote}
%
% \paragraph{Script installation.}
% Check the directory \xfile{TDS:scripts/oberdiek/} for
% scripts that need further installation steps.
% Package \xpackage{attachfile2} comes with the Perl script
% \xfile{pdfatfi.pl} that should be installed in such a way
% that it can be called as \texttt{pdfatfi}.
% Example (linux):
% \begin{quote}
%   |chmod +x scripts/oberdiek/pdfatfi.pl|\\
%   |cp scripts/oberdiek/pdfatfi.pl /usr/local/bin/|
% \end{quote}
%
% \subsection{Package installation}
%
% \paragraph{Unpacking.} The \xfile{.dtx} file is a self-extracting
% \docstrip\ archive. The files are extracted by running the
% \xfile{.dtx} through \plainTeX:
% \begin{quote}
%   \verb|tex scrindex.dtx|
% \end{quote}
%
% \paragraph{TDS.} Now the different files must be moved into
% the different directories in your installation TDS tree
% (also known as \xfile{texmf} tree):
% \begin{quote}
% \def\t{^^A
% \begin{tabular}{@{}>{\ttfamily}l@{ $\rightarrow$ }>{\ttfamily}l@{}}
%   scrindex.sty & tex/latex/oberdiek/scrindex.sty\\
%   scrindex.pdf & doc/latex/oberdiek/scrindex.pdf\\
%   scrindex-example1.tex & doc/latex/oberdiek/scrindex-example1.tex\\
%   scrindex-example2.tex & doc/latex/oberdiek/scrindex-example2.tex\\
%   scrindex.dtx & source/latex/oberdiek/scrindex.dtx\\
% \end{tabular}^^A
% }^^A
% \sbox0{\t}^^A
% \ifdim\wd0>\linewidth
%   \begingroup
%     \advance\linewidth by\leftmargin
%     \advance\linewidth by\rightmargin
%   \edef\x{\endgroup
%     \def\noexpand\lw{\the\linewidth}^^A
%   }\x
%   \def\lwbox{^^A
%     \leavevmode
%     \hbox to \linewidth{^^A
%       \kern-\leftmargin\relax
%       \hss
%       \usebox0
%       \hss
%       \kern-\rightmargin\relax
%     }^^A
%   }^^A
%   \ifdim\wd0>\lw
%     \sbox0{\small\t}^^A
%     \ifdim\wd0>\linewidth
%       \ifdim\wd0>\lw
%         \sbox0{\footnotesize\t}^^A
%         \ifdim\wd0>\linewidth
%           \ifdim\wd0>\lw
%             \sbox0{\scriptsize\t}^^A
%             \ifdim\wd0>\linewidth
%               \ifdim\wd0>\lw
%                 \sbox0{\tiny\t}^^A
%                 \ifdim\wd0>\linewidth
%                   \lwbox
%                 \else
%                   \usebox0
%                 \fi
%               \else
%                 \lwbox
%               \fi
%             \else
%               \usebox0
%             \fi
%           \else
%             \lwbox
%           \fi
%         \else
%           \usebox0
%         \fi
%       \else
%         \lwbox
%       \fi
%     \else
%       \usebox0
%     \fi
%   \else
%     \lwbox
%   \fi
% \else
%   \usebox0
% \fi
% \end{quote}
% If you have a \xfile{docstrip.cfg} that configures and enables \docstrip's
% TDS installing feature, then some files can already be in the right
% place, see the documentation of \docstrip.
%
% \subsection{Refresh file name databases}
%
% If your \TeX~distribution
% (\teTeX, \mikTeX, \dots) relies on file name databases, you must refresh
% these. For example, \teTeX\ users run \verb|texhash| or
% \verb|mktexlsr|.
%
% \subsection{Some details for the interested}
%
% \paragraph{Attached source.}
%
% The PDF documentation on CTAN also includes the
% \xfile{.dtx} source file. It can be extracted by
% AcrobatReader 6 or higher. Another option is \textsf{pdftk},
% e.g. unpack the file into the current directory:
% \begin{quote}
%   \verb|pdftk scrindex.pdf unpack_files output .|
% \end{quote}
%
% \paragraph{Unpacking with \LaTeX.}
% The \xfile{.dtx} chooses its action depending on the format:
% \begin{description}
% \item[\plainTeX:] Run \docstrip\ and extract the files.
% \item[\LaTeX:] Generate the documentation.
% \end{description}
% If you insist on using \LaTeX\ for \docstrip\ (really,
% \docstrip\ does not need \LaTeX), then inform the autodetect routine
% about your intention:
% \begin{quote}
%   \verb|latex \let\install=y\input{scrindex.dtx}|
% \end{quote}
% Do not forget to quote the argument according to the demands
% of your shell.
%
% \paragraph{Generating the documentation.}
% You can use both the \xfile{.dtx} or the \xfile{.drv} to generate
% the documentation. The process can be configured by the
% configuration file \xfile{ltxdoc.cfg}. For instance, put this
% line into this file, if you want to have A4 as paper format:
% \begin{quote}
%   \verb|\PassOptionsToClass{a4paper}{article}|
% \end{quote}
% An example follows how to generate the
% documentation with pdf\LaTeX:
% \begin{quote}
%\begin{verbatim}
%pdflatex scrindex.dtx
%makeindex -s gind.ist scrindex.idx
%pdflatex scrindex.dtx
%makeindex -s gind.ist scrindex.idx
%pdflatex scrindex.dtx
%\end{verbatim}
% \end{quote}
%
% \section{Catalogue}
%
% The following XML file can be used as source for the
% \href{http://mirror.ctan.org/help/Catalogue/catalogue.html}{\TeX\ Catalogue}.
% The elements \texttt{caption} and \texttt{description} are imported
% from the original XML file from the Catalogue.
% The name of the XML file in the Catalogue is \xfile{scrindex.xml}.
%    \begin{macrocode}
%<*catalogue>
<?xml version='1.0' encoding='us-ascii'?>
<!DOCTYPE entry SYSTEM 'catalogue.dtd'>
<entry datestamp='$Date$' modifier='$Author$' id='scrindex'>
  <name>scrindex</name>
  <caption>Make index package work with Koma-script classes.</caption>
  <authorref id='auth:oberdiek'/>
  <copyright owner='Heiko Oberdiek' year='2008'/>
  <license type='lppl1.3'/>
  <version number='1.1'/>
  <description>
    This package redefines environment `theindex' of package `index',
    if a class from <xref refid='koma-script'>KOMA-Script</xref> is loaded.
    Also option `idxtotoc' is supported. Index preambles can be given
    either by means of package `index' or via the interface provided
    by <xref refid='koma-script'>KOMA-Script</xref>.
    <p/>
    The package is part of the <xref refid='oberdiek'>oberdiek</xref>
    bundle.
  </description>
  <documentation details='Package documentation'
      href='ctan:/macros/latex/contrib/oberdiek/scrindex.pdf'/>
  <ctan file='true' path='/macros/latex/contrib/oberdiek/scrindex.dtx'/>
  <miktex location='oberdiek'/>
  <texlive location='oberdiek'/>
  <install path='/macros/latex/contrib/oberdiek/oberdiek.tds.zip'/>
</entry>
%</catalogue>
%    \end{macrocode}
%
% \begin{History}
%   \begin{Version}{2008/07/07 v1.0}
%   \item
%     First version, also published in newsgroup \xnewsgroup{de.comp.text.tex}:\\
%     \URL{``\link{Re: Z\"ahler bei \cs{index}}''}^^A
%     {http://groups.google.com/group/de.comp.text.tex/msg/39575b5e2f29be1e}
%   \end{Version}
%   \begin{Version}{2008/08/11 v1.1}
%   \item
%     Code is not changed.
%   \item
%     URLs updated.
%   \end{Version}
% \end{History}
%
% \PrintIndex
%
% \Finale
\endinput
|
% \end{quote}
% Do not forget to quote the argument according to the demands
% of your shell.
%
% \paragraph{Generating the documentation.}
% You can use both the \xfile{.dtx} or the \xfile{.drv} to generate
% the documentation. The process can be configured by the
% configuration file \xfile{ltxdoc.cfg}. For instance, put this
% line into this file, if you want to have A4 as paper format:
% \begin{quote}
%   \verb|\PassOptionsToClass{a4paper}{article}|
% \end{quote}
% An example follows how to generate the
% documentation with pdf\LaTeX:
% \begin{quote}
%\begin{verbatim}
%pdflatex scrindex.dtx
%makeindex -s gind.ist scrindex.idx
%pdflatex scrindex.dtx
%makeindex -s gind.ist scrindex.idx
%pdflatex scrindex.dtx
%\end{verbatim}
% \end{quote}
%
% \section{Catalogue}
%
% The following XML file can be used as source for the
% \href{http://mirror.ctan.org/help/Catalogue/catalogue.html}{\TeX\ Catalogue}.
% The elements \texttt{caption} and \texttt{description} are imported
% from the original XML file from the Catalogue.
% The name of the XML file in the Catalogue is \xfile{scrindex.xml}.
%    \begin{macrocode}
%<*catalogue>
<?xml version='1.0' encoding='us-ascii'?>
<!DOCTYPE entry SYSTEM 'catalogue.dtd'>
<entry datestamp='$Date$' modifier='$Author$' id='scrindex'>
  <name>scrindex</name>
  <caption>Make index package work with Koma-script classes.</caption>
  <authorref id='auth:oberdiek'/>
  <copyright owner='Heiko Oberdiek' year='2008'/>
  <license type='lppl1.3'/>
  <version number='1.1'/>
  <description>
    This package redefines environment `theindex' of package `index',
    if a class from <xref refid='koma-script'>KOMA-Script</xref> is loaded.
    Also option `idxtotoc' is supported. Index preambles can be given
    either by means of package `index' or via the interface provided
    by <xref refid='koma-script'>KOMA-Script</xref>.
    <p/>
    The package is part of the <xref refid='oberdiek'>oberdiek</xref>
    bundle.
  </description>
  <documentation details='Package documentation'
      href='ctan:/macros/latex/contrib/oberdiek/scrindex.pdf'/>
  <ctan file='true' path='/macros/latex/contrib/oberdiek/scrindex.dtx'/>
  <miktex location='oberdiek'/>
  <texlive location='oberdiek'/>
  <install path='/macros/latex/contrib/oberdiek/oberdiek.tds.zip'/>
</entry>
%</catalogue>
%    \end{macrocode}
%
% \begin{History}
%   \begin{Version}{2008/07/07 v1.0}
%   \item
%     First version, also published in newsgroup \xnewsgroup{de.comp.text.tex}:\\
%     \URL{``\link{Re: Z\"ahler bei \cs{index}}''}^^A
%     {http://groups.google.com/group/de.comp.text.tex/msg/39575b5e2f29be1e}
%   \end{Version}
%   \begin{Version}{2008/08/11 v1.1}
%   \item
%     Code is not changed.
%   \item
%     URLs updated.
%   \end{Version}
% \end{History}
%
% \PrintIndex
%
% \Finale
\endinput

%        (quote the arguments according to the demands of your shell)
%
% Documentation:
%    (a) If scrindex.drv is present:
%           latex scrindex.drv
%    (b) Without scrindex.drv:
%           latex scrindex.dtx; ...
%    The class ltxdoc loads the configuration file ltxdoc.cfg
%    if available. Here you can specify further options, e.g.
%    use A4 as paper format:
%       \PassOptionsToClass{a4paper}{article}
%
%    Programm calls to get the documentation (example):
%       pdflatex scrindex.dtx
%       makeindex -s gind.ist scrindex.idx
%       pdflatex scrindex.dtx
%       makeindex -s gind.ist scrindex.idx
%       pdflatex scrindex.dtx
%
% Installation:
%    TDS:tex/latex/oberdiek/scrindex.sty
%    TDS:doc/latex/oberdiek/scrindex.pdf
%    TDS:doc/latex/oberdiek/scrindex-example1.tex
%    TDS:doc/latex/oberdiek/scrindex-example2.tex
%    TDS:source/latex/oberdiek/scrindex.dtx
%
%<*ignore>
\begingroup
  \catcode123=1 %
  \catcode125=2 %
  \def\x{LaTeX2e}%
\expandafter\endgroup
\ifcase 0\ifx\install y1\fi\expandafter
         \ifx\csname processbatchFile\endcsname\relax\else1\fi
         \ifx\fmtname\x\else 1\fi\relax
\else\csname fi\endcsname
%</ignore>
%<*install>
\input docstrip.tex
\Msg{************************************************************************}
\Msg{* Installation}
\Msg{* Package: scrindex 2008/08/11 v1.1 Package index with KOMA-Script classes (HO)}
\Msg{************************************************************************}

\keepsilent
\askforoverwritefalse

\let\MetaPrefix\relax
\preamble

This is a generated file.

Project: scrindex
Version: 2008/08/11 v1.1

Copyright (C) 2008 by
   Heiko Oberdiek <heiko.oberdiek at googlemail.com>

This work may be distributed and/or modified under the
conditions of the LaTeX Project Public License, either
version 1.3c of this license or (at your option) any later
version. This version of this license is in
   http://www.latex-project.org/lppl/lppl-1-3c.txt
and the latest version of this license is in
   http://www.latex-project.org/lppl.txt
and version 1.3 or later is part of all distributions of
LaTeX version 2005/12/01 or later.

This work has the LPPL maintenance status "maintained".

This Current Maintainer of this work is Heiko Oberdiek.

This work consists of the main source file scrindex.dtx
and the derived files
   scrindex.sty, scrindex.pdf, scrindex.ins, scrindex.drv,
   scrindex-example1.tex, scrindex-example2.tex.

\endpreamble
\let\MetaPrefix\DoubleperCent

\generate{%
  \file{scrindex.ins}{\from{scrindex.dtx}{install}}%
  \file{scrindex.drv}{\from{scrindex.dtx}{driver}}%
  \usedir{tex/latex/oberdiek}%
  \file{scrindex.sty}{\from{scrindex.dtx}{package}}%
  \usedir{doc/latex/oberdiek}%
  \file{scrindex-example1.tex}{\from{scrindex.dtx}{example1}}%
  \file{scrindex-example2.tex}{\from{scrindex.dtx}{example2}}%
  \nopreamble
  \nopostamble
  \usedir{source/latex/oberdiek/catalogue}%
  \file{scrindex.xml}{\from{scrindex.dtx}{catalogue}}%
}

\catcode32=13\relax% active space
\let =\space%
\Msg{************************************************************************}
\Msg{*}
\Msg{* To finish the installation you have to move the following}
\Msg{* file into a directory searched by TeX:}
\Msg{*}
\Msg{*     scrindex.sty}
\Msg{*}
\Msg{* To produce the documentation run the file `scrindex.drv'}
\Msg{* through LaTeX.}
\Msg{*}
\Msg{* Happy TeXing!}
\Msg{*}
\Msg{************************************************************************}

\endbatchfile
%</install>
%<*ignore>
\fi
%</ignore>
%<*driver>
\NeedsTeXFormat{LaTeX2e}
\ProvidesFile{scrindex.drv}%
  [2008/08/11 v1.1 Package index with KOMA-Script classes (HO)]%
\documentclass{ltxdoc}
\usepackage{holtxdoc}[2011/11/22]
\usepackage{calc}
\begin{document}
  \DocInput{scrindex.dtx}%
\end{document}
%</driver>
% \fi
%
% \CheckSum{237}
%
% \CharacterTable
%  {Upper-case    \A\B\C\D\E\F\G\H\I\J\K\L\M\N\O\P\Q\R\S\T\U\V\W\X\Y\Z
%   Lower-case    \a\b\c\d\e\f\g\h\i\j\k\l\m\n\o\p\q\r\s\t\u\v\w\x\y\z
%   Digits        \0\1\2\3\4\5\6\7\8\9
%   Exclamation   \!     Double quote  \"     Hash (number) \#
%   Dollar        \$     Percent       \%     Ampersand     \&
%   Acute accent  \'     Left paren    \(     Right paren   \)
%   Asterisk      \*     Plus          \+     Comma         \,
%   Minus         \-     Point         \.     Solidus       \/
%   Colon         \:     Semicolon     \;     Less than     \<
%   Equals        \=     Greater than  \>     Question mark \?
%   Commercial at \@     Left bracket  \[     Backslash     \\
%   Right bracket \]     Circumflex    \^     Underscore    \_
%   Grave accent  \`     Left brace    \{     Vertical bar  \|
%   Right brace   \}     Tilde         \~}
%
% \GetFileInfo{scrindex.drv}
%
% \title{The \xpackage{scrindex} package}
% \date{2008/08/11 v1.1}
% \author{Heiko Oberdiek\\\xemail{heiko.oberdiek at googlemail.com}}
%
% \maketitle
%
% \begin{abstract}
% This package redefines environment `theindex' of package \xpackage{index},
% if a class from KOMA-Script is loaded. Also option \xoption{idxtotoc}
% is supported. Index preambles can be given either by means of package
% \xpackage{index} or via the interface provided by KOMA-Script.
% \end{abstract}
%
% \tableofcontents
%
% \section{Documentation}
%
% Package \xpackage{index}, written by David M.\ Jones, detects
% the standard classes |article|, |report|, and |book|. It
% redefines environment `theindex' for its needs.
% However, it does not know other classes such as KOMA-Script.
% This package closes the compatibiliy gap between KOMA-Script's
% classes and package \xpackage{index}.
%
% Environment |theindex| is redefined to support both package
% \xpackage{index} and KOMA-Script's classes. Thus both
% the prologe of package \xpackage{index} and the preamble
% of KOMA-Script's classes are available. Also class option |idxtotoc|
% of KOMA-Script is supported.
%
% \subsection{Usage}
%
% The package \xpackage{scrindex} is loaded without options:
%\begin{quote}
%\begin{verbatim}
%\usepackage{scrindex}
%\end{verbatim}
%\end{quote}
%
% It loads package \xpackage{index} and requests version 2004/01/20
% or later. \LaTeX's package interface allows multiple calls
% of the same package. The package is loaded at its first
% package loading command. At later times \LaTeX\ only checks
% options and a requested version date. Therefore it does not harm
% to add |\usepackage{index}| before or after |\usepackage{scrindex}|.
%
% Also the class does not matter. Environment |theindex| is only
% redefined for a supported class:
% \begin{itemize}
% \item |scrartcl|
% \item |scrreprt|
% \item |scrbook|
% \end{itemize}
%
% \subsection{Preambles}
%
% Both the prologue of package \xpackage{index} and the preamble
% of KOMA-Script's classes are supported. The position depends on
% the class.
%
% \subsubsection{Class \xclass{scrartcl}}
%
%    \begin{macrocode}
%<*example1>
\documentclass{scrartcl}
\usepackage{scrindex}
\setindexpreamble{Preamble of \texttt{scrartcl}\dotfill EOL}
\makeindex
\begin{document}
\section{First Section}
\index{first}
\index{section}
\printindex[default]%
  [Prologue of package \texttt{index}\dotfill EOL]%
\end{document}
%</example1>
%    \end{macrocode}
% The prologue of package \xpackage{index} is first set straight
% after the section title spanning both columns.
% Then the preamble of KOMA-Script follows
% in the first left column.
%
% \medskip
% \begin{quote}
%   \renewcommand*{\arraystretch}{1.2}
%   \begin{tabular}{|p{.45\linewidth}|p{.45\linewidth}|}
%   \hline
%   \multicolumn{2}{|l|}{\textbf{Index}}\\[1ex]
%   \multicolumn{2}{|p{.9\linewidth+2\tabcolsep}|}{^^A
%     Prologue of package \texttt{index}\dotfill EOL^^A
%   }\\[1ex]
%   \hline
%   Preamble of \texttt{scrartcl}\dotfill EOL&\\
%   first, 1&\\
%   section, 1&\\
%   \hline
%   \end{tabular}
% \end{quote}
%
% \subsubsection{Classes \xclass{scrreprt} and \xclass{scrbook}}
%
%    \begin{macrocode}
%<*example2>
\documentclass[openany]{scrbook}% or scrreprt
\usepackage{scrindex}
\setindexpreamble{Preamble of class \texttt{scrbook}\dotfill EOL}
\makeindex
\begin{document}
\chapter{First Chapter}
\index{first}
\index{chapter}
\printindex[default]%
  [Prologue of package \texttt{index}\dotfill EOL]%
\end{document}
%</example2>
%    \end{macrocode}
% The order of the two preambles are different for the classes
% \xclass{scrreprt} and \xclass{scrbook}. First KOMA-Script's
% chapter preamble is set, then the prologue of package \xpackage{index}
% follows. Both are set spanning both columns.
%
% \medskip
% \begin{quote}
%   \renewcommand*{\arraystretch}{1.2}
%   \begin{tabular}{|p{.45\linewidth}|p{.45\linewidth}|}
%   \hline
%   \multicolumn{2}{|l|}{\textbf{Index}}\\[1ex]
%   \multicolumn{2}{|p{.9\linewidth+2\tabcolsep}|}{^^A
%     Preamble of class \texttt{scrbook}\dotfill EOL^^A
%   }\\
%   \multicolumn{2}{|p{.9\linewidth+2\tabcolsep}|}{^^A
%     Prologue of package \xpackage{index}\dotfill EOL^^A
%   }\\[1ex]
%   \hline
%   chapter, 1&\\
%   first, 1&\\
%   \hline
%   \end{tabular}
% \end{quote}
%
% \StopEventually{
% }
%
% \section{Implementation}
%
%    \begin{macrocode}
%<*package>
\NeedsTeXFormat{LaTeX2e}
\ProvidesPackage{scrindex}
  [2008/08/11 v1.1 Package index with KOMA-Script classes (HO)]%
%    \end{macrocode}
%
%    \begin{macrocode}
\RequirePackage{index}[2004/01/20]%
%    \end{macrocode}
%
%    \begin{macrocode}
\@ifclassloaded{scrartcl}{%
  \renewenvironment{theindex}{%
    \edef\indexname{%
      \the\@nameuse{idxtitle@\@indextype}%
    }%
    \if@twocolumn
      \@restonecolfalse
    \else
      \@restonecoltrue
    \fi
    \idx@heading
    \thispagestyle{\indexpagestyle}%
    \columnseprule\z@
    \columnsep 35\p@
    \index@preamble\par\nobreak
    \parindent\z@
    \parskip\z@ \@plus .3\p@\relax
    \parfillskip\z@ \@plus 1fil\relax
    \let\item\@idxitem
  }{%
    \if@restonecol
      \onecolumn
    \else
      \clearpage
    \fi
  }%
  \@ifclasswith{scrartcl}{idxtotoc}{%
    \renewcommand*{\idx@heading}{%
      \twocolumn[%
        \addsec{\indexname}%
        \ifx\index@prologue\@empty
        \else
          \index@prologue
          \bigskip
        \fi
      ]%
      \@mkboth{\indexname}{\indexname}%
    }%
  }{%
    \renewcommand*{\idx@heading}{%
      \twocolumn[%
        \section*{\indexname}%
        \ifx\index@prologue\@empty
        \else
          \index@prologue
          \bigskip
        \fi
      ]%
      \@mkboth{\indexname}{\indexname}%
    }%
  }%
}{}
%    \end{macrocode}
%    \begin{macrocode}
\@ifclassloaded{scrreprt}{%
  \renewenvironment{theindex}{%
    \edef\indexname{%
      \the\@nameuse{idxtitle@\@indextype}%
    }%
    \if@twocolumn
      \@restonecolfalse
    \else
      \@restonecoltrue
    \fi
    \setchapterpreamble{\index@preamble}%
    \idx@heading
    \thispagestyle{\indexpagestyle}%
    \columnseprule\z@
    \columnsep 35\p@
    \parindent\z@
    \parskip\z@ \@plus .3\p@\relax
    \parfillskip\z@ \@plus 1fil\relax
    \let\item\@idxitem
  }{%
    \if@restonecol
      \onecolumn
    \else
      \clearpage
    \fi
  }%
  \@ifclasswith{scrreprt}{idxtotoc}{%
    \renewcommand*{\idx@heading}{%
      \if@openright
        \cleardoublepage
      \else
        \clearpage
      \fi
      \twocolumn[%
        \addchap{\indexname}%
        \ifx\index@prologue\@empty
        \else
          \index@prologue
          \bigskip
        \fi
      ]%
      \@mkboth{\indexname}{\indexname}%
    }%
  }{%
    \renewcommand*{\idx@heading}{%
      \if@openright
        \cleardoublepage
      \else
        \clearpage
      \fi
      \twocolumn[%
        \chapter*{\indexname}%
        \ifx\index@prologue\@empty
        \else
          \index@prologue
          \bigskip
        \fi
      ]%
      \@mkboth{\indexname}{\indexname}%
    }%
  }%
}{}
%    \end{macrocode}
%    \begin{macrocode}
\@ifclassloaded{scrbook}{%
  \renewenvironment{theindex}{%
    \edef\indexname{%
      \the\@nameuse{idxtitle@\@indextype}%
    }%
    \if@twocolumn
      \@restonecolfalse
    \else
      \@restonecoltrue
    \fi
    \setchapterpreamble{\index@preamble}%
    \idx@heading
    \thispagestyle{\indexpagestyle}%
    \columnseprule\z@
    \columnsep 35\p@
    \parindent\z@
    \parskip\z@ \@plus .3\p@\relax
    \parfillskip\z@ \@plus 1fil\relax
    \let\item\@idxitem
  }{%
    \if@restonecol
      \onecolumn
    \else
      \clearpage
    \fi
  }%
  \@ifclasswith{scrbook}{idxtotoc}{%
    \renewcommand*{\idx@heading}{%
      \if@openright
        \cleardoublepage
      \else
        \clearpage
      \fi
      \twocolumn[%
        \addchap{\indexname}%
        \ifx\index@prologue\@empty
        \else
          \index@prologue
          \bigskip
        \fi
      ]%
      \@mkboth{\indexname}{\indexname}%
    }%
  }{%
    \renewcommand*{\idx@heading}{%
      \if@openright
        \cleardoublepage
      \else
        \clearpage
      \fi
      \twocolumn[%
        \chapter*{\indexname}%
        \ifx\index@prologue\@empty
        \else
          \index@prologue
          \bigskip
        \fi
      ]%
      \@mkboth{\indexname}{\indexname}%
    }%
  }%
}{}
%    \end{macrocode}
%
%    \begin{macrocode}
%</package>
%    \end{macrocode}
%
% \section{Installation}
%
% \subsection{Download}
%
% \paragraph{Package.} This package is available on
% CTAN\footnote{\url{ftp://ftp.ctan.org/tex-archive/}}:
% \begin{description}
% \item[\CTAN{macros/latex/contrib/oberdiek/scrindex.dtx}] The source file.
% \item[\CTAN{macros/latex/contrib/oberdiek/scrindex.pdf}] Documentation.
% \end{description}
%
%
% \paragraph{Bundle.} All the packages of the bundle `oberdiek'
% are also available in a TDS compliant ZIP archive. There
% the packages are already unpacked and the documentation files
% are generated. The files and directories obey the TDS standard.
% \begin{description}
% \item[\CTAN{install/macros/latex/contrib/oberdiek.tds.zip}]
% \end{description}
% \emph{TDS} refers to the standard ``A Directory Structure
% for \TeX\ Files'' (\CTAN{tds/tds.pdf}). Directories
% with \xfile{texmf} in their name are usually organized this way.
%
% \subsection{Bundle installation}
%
% \paragraph{Unpacking.} Unpack the \xfile{oberdiek.tds.zip} in the
% TDS tree (also known as \xfile{texmf} tree) of your choice.
% Example (linux):
% \begin{quote}
%   |unzip oberdiek.tds.zip -d ~/texmf|
% \end{quote}
%
% \paragraph{Script installation.}
% Check the directory \xfile{TDS:scripts/oberdiek/} for
% scripts that need further installation steps.
% Package \xpackage{attachfile2} comes with the Perl script
% \xfile{pdfatfi.pl} that should be installed in such a way
% that it can be called as \texttt{pdfatfi}.
% Example (linux):
% \begin{quote}
%   |chmod +x scripts/oberdiek/pdfatfi.pl|\\
%   |cp scripts/oberdiek/pdfatfi.pl /usr/local/bin/|
% \end{quote}
%
% \subsection{Package installation}
%
% \paragraph{Unpacking.} The \xfile{.dtx} file is a self-extracting
% \docstrip\ archive. The files are extracted by running the
% \xfile{.dtx} through \plainTeX:
% \begin{quote}
%   \verb|tex scrindex.dtx|
% \end{quote}
%
% \paragraph{TDS.} Now the different files must be moved into
% the different directories in your installation TDS tree
% (also known as \xfile{texmf} tree):
% \begin{quote}
% \def\t{^^A
% \begin{tabular}{@{}>{\ttfamily}l@{ $\rightarrow$ }>{\ttfamily}l@{}}
%   scrindex.sty & tex/latex/oberdiek/scrindex.sty\\
%   scrindex.pdf & doc/latex/oberdiek/scrindex.pdf\\
%   scrindex-example1.tex & doc/latex/oberdiek/scrindex-example1.tex\\
%   scrindex-example2.tex & doc/latex/oberdiek/scrindex-example2.tex\\
%   scrindex.dtx & source/latex/oberdiek/scrindex.dtx\\
% \end{tabular}^^A
% }^^A
% \sbox0{\t}^^A
% \ifdim\wd0>\linewidth
%   \begingroup
%     \advance\linewidth by\leftmargin
%     \advance\linewidth by\rightmargin
%   \edef\x{\endgroup
%     \def\noexpand\lw{\the\linewidth}^^A
%   }\x
%   \def\lwbox{^^A
%     \leavevmode
%     \hbox to \linewidth{^^A
%       \kern-\leftmargin\relax
%       \hss
%       \usebox0
%       \hss
%       \kern-\rightmargin\relax
%     }^^A
%   }^^A
%   \ifdim\wd0>\lw
%     \sbox0{\small\t}^^A
%     \ifdim\wd0>\linewidth
%       \ifdim\wd0>\lw
%         \sbox0{\footnotesize\t}^^A
%         \ifdim\wd0>\linewidth
%           \ifdim\wd0>\lw
%             \sbox0{\scriptsize\t}^^A
%             \ifdim\wd0>\linewidth
%               \ifdim\wd0>\lw
%                 \sbox0{\tiny\t}^^A
%                 \ifdim\wd0>\linewidth
%                   \lwbox
%                 \else
%                   \usebox0
%                 \fi
%               \else
%                 \lwbox
%               \fi
%             \else
%               \usebox0
%             \fi
%           \else
%             \lwbox
%           \fi
%         \else
%           \usebox0
%         \fi
%       \else
%         \lwbox
%       \fi
%     \else
%       \usebox0
%     \fi
%   \else
%     \lwbox
%   \fi
% \else
%   \usebox0
% \fi
% \end{quote}
% If you have a \xfile{docstrip.cfg} that configures and enables \docstrip's
% TDS installing feature, then some files can already be in the right
% place, see the documentation of \docstrip.
%
% \subsection{Refresh file name databases}
%
% If your \TeX~distribution
% (\teTeX, \mikTeX, \dots) relies on file name databases, you must refresh
% these. For example, \teTeX\ users run \verb|texhash| or
% \verb|mktexlsr|.
%
% \subsection{Some details for the interested}
%
% \paragraph{Attached source.}
%
% The PDF documentation on CTAN also includes the
% \xfile{.dtx} source file. It can be extracted by
% AcrobatReader 6 or higher. Another option is \textsf{pdftk},
% e.g. unpack the file into the current directory:
% \begin{quote}
%   \verb|pdftk scrindex.pdf unpack_files output .|
% \end{quote}
%
% \paragraph{Unpacking with \LaTeX.}
% The \xfile{.dtx} chooses its action depending on the format:
% \begin{description}
% \item[\plainTeX:] Run \docstrip\ and extract the files.
% \item[\LaTeX:] Generate the documentation.
% \end{description}
% If you insist on using \LaTeX\ for \docstrip\ (really,
% \docstrip\ does not need \LaTeX), then inform the autodetect routine
% about your intention:
% \begin{quote}
%   \verb|latex \let\install=y% \iffalse meta-comment
%
% File: scrindex.dtx
% Version: 2008/08/11 v1.1
% Info: Package index with KOMA-Script classes
%
% Copyright (C) 2008 by
%    Heiko Oberdiek <heiko.oberdiek at googlemail.com>
%
% This work may be distributed and/or modified under the
% conditions of the LaTeX Project Public License, either
% version 1.3c of this license or (at your option) any later
% version. This version of this license is in
%    http://www.latex-project.org/lppl/lppl-1-3c.txt
% and the latest version of this license is in
%    http://www.latex-project.org/lppl.txt
% and version 1.3 or later is part of all distributions of
% LaTeX version 2005/12/01 or later.
%
% This work has the LPPL maintenance status "maintained".
%
% This Current Maintainer of this work is Heiko Oberdiek.
%
% This work consists of the main source file scrindex.dtx
% and the derived files
%    scrindex.sty, scrindex.pdf, scrindex.ins, scrindex.drv,
%    scrindex-example1.tex, scrindex-example2.tex.
%
% Distribution:
%    CTAN:macros/latex/contrib/oberdiek/scrindex.dtx
%    CTAN:macros/latex/contrib/oberdiek/scrindex.pdf
%
% Unpacking:
%    (a) If scrindex.ins is present:
%           tex scrindex.ins
%    (b) Without scrindex.ins:
%           tex scrindex.dtx
%    (c) If you insist on using LaTeX
%           latex \let\install=y% \iffalse meta-comment
%
% File: scrindex.dtx
% Version: 2008/08/11 v1.1
% Info: Package index with KOMA-Script classes
%
% Copyright (C) 2008 by
%    Heiko Oberdiek <heiko.oberdiek at googlemail.com>
%
% This work may be distributed and/or modified under the
% conditions of the LaTeX Project Public License, either
% version 1.3c of this license or (at your option) any later
% version. This version of this license is in
%    http://www.latex-project.org/lppl/lppl-1-3c.txt
% and the latest version of this license is in
%    http://www.latex-project.org/lppl.txt
% and version 1.3 or later is part of all distributions of
% LaTeX version 2005/12/01 or later.
%
% This work has the LPPL maintenance status "maintained".
%
% This Current Maintainer of this work is Heiko Oberdiek.
%
% This work consists of the main source file scrindex.dtx
% and the derived files
%    scrindex.sty, scrindex.pdf, scrindex.ins, scrindex.drv,
%    scrindex-example1.tex, scrindex-example2.tex.
%
% Distribution:
%    CTAN:macros/latex/contrib/oberdiek/scrindex.dtx
%    CTAN:macros/latex/contrib/oberdiek/scrindex.pdf
%
% Unpacking:
%    (a) If scrindex.ins is present:
%           tex scrindex.ins
%    (b) Without scrindex.ins:
%           tex scrindex.dtx
%    (c) If you insist on using LaTeX
%           latex \let\install=y\input{scrindex.dtx}
%        (quote the arguments according to the demands of your shell)
%
% Documentation:
%    (a) If scrindex.drv is present:
%           latex scrindex.drv
%    (b) Without scrindex.drv:
%           latex scrindex.dtx; ...
%    The class ltxdoc loads the configuration file ltxdoc.cfg
%    if available. Here you can specify further options, e.g.
%    use A4 as paper format:
%       \PassOptionsToClass{a4paper}{article}
%
%    Programm calls to get the documentation (example):
%       pdflatex scrindex.dtx
%       makeindex -s gind.ist scrindex.idx
%       pdflatex scrindex.dtx
%       makeindex -s gind.ist scrindex.idx
%       pdflatex scrindex.dtx
%
% Installation:
%    TDS:tex/latex/oberdiek/scrindex.sty
%    TDS:doc/latex/oberdiek/scrindex.pdf
%    TDS:doc/latex/oberdiek/scrindex-example1.tex
%    TDS:doc/latex/oberdiek/scrindex-example2.tex
%    TDS:source/latex/oberdiek/scrindex.dtx
%
%<*ignore>
\begingroup
  \catcode123=1 %
  \catcode125=2 %
  \def\x{LaTeX2e}%
\expandafter\endgroup
\ifcase 0\ifx\install y1\fi\expandafter
         \ifx\csname processbatchFile\endcsname\relax\else1\fi
         \ifx\fmtname\x\else 1\fi\relax
\else\csname fi\endcsname
%</ignore>
%<*install>
\input docstrip.tex
\Msg{************************************************************************}
\Msg{* Installation}
\Msg{* Package: scrindex 2008/08/11 v1.1 Package index with KOMA-Script classes (HO)}
\Msg{************************************************************************}

\keepsilent
\askforoverwritefalse

\let\MetaPrefix\relax
\preamble

This is a generated file.

Project: scrindex
Version: 2008/08/11 v1.1

Copyright (C) 2008 by
   Heiko Oberdiek <heiko.oberdiek at googlemail.com>

This work may be distributed and/or modified under the
conditions of the LaTeX Project Public License, either
version 1.3c of this license or (at your option) any later
version. This version of this license is in
   http://www.latex-project.org/lppl/lppl-1-3c.txt
and the latest version of this license is in
   http://www.latex-project.org/lppl.txt
and version 1.3 or later is part of all distributions of
LaTeX version 2005/12/01 or later.

This work has the LPPL maintenance status "maintained".

This Current Maintainer of this work is Heiko Oberdiek.

This work consists of the main source file scrindex.dtx
and the derived files
   scrindex.sty, scrindex.pdf, scrindex.ins, scrindex.drv,
   scrindex-example1.tex, scrindex-example2.tex.

\endpreamble
\let\MetaPrefix\DoubleperCent

\generate{%
  \file{scrindex.ins}{\from{scrindex.dtx}{install}}%
  \file{scrindex.drv}{\from{scrindex.dtx}{driver}}%
  \usedir{tex/latex/oberdiek}%
  \file{scrindex.sty}{\from{scrindex.dtx}{package}}%
  \usedir{doc/latex/oberdiek}%
  \file{scrindex-example1.tex}{\from{scrindex.dtx}{example1}}%
  \file{scrindex-example2.tex}{\from{scrindex.dtx}{example2}}%
  \nopreamble
  \nopostamble
  \usedir{source/latex/oberdiek/catalogue}%
  \file{scrindex.xml}{\from{scrindex.dtx}{catalogue}}%
}

\catcode32=13\relax% active space
\let =\space%
\Msg{************************************************************************}
\Msg{*}
\Msg{* To finish the installation you have to move the following}
\Msg{* file into a directory searched by TeX:}
\Msg{*}
\Msg{*     scrindex.sty}
\Msg{*}
\Msg{* To produce the documentation run the file `scrindex.drv'}
\Msg{* through LaTeX.}
\Msg{*}
\Msg{* Happy TeXing!}
\Msg{*}
\Msg{************************************************************************}

\endbatchfile
%</install>
%<*ignore>
\fi
%</ignore>
%<*driver>
\NeedsTeXFormat{LaTeX2e}
\ProvidesFile{scrindex.drv}%
  [2008/08/11 v1.1 Package index with KOMA-Script classes (HO)]%
\documentclass{ltxdoc}
\usepackage{holtxdoc}[2011/11/22]
\usepackage{calc}
\begin{document}
  \DocInput{scrindex.dtx}%
\end{document}
%</driver>
% \fi
%
% \CheckSum{237}
%
% \CharacterTable
%  {Upper-case    \A\B\C\D\E\F\G\H\I\J\K\L\M\N\O\P\Q\R\S\T\U\V\W\X\Y\Z
%   Lower-case    \a\b\c\d\e\f\g\h\i\j\k\l\m\n\o\p\q\r\s\t\u\v\w\x\y\z
%   Digits        \0\1\2\3\4\5\6\7\8\9
%   Exclamation   \!     Double quote  \"     Hash (number) \#
%   Dollar        \$     Percent       \%     Ampersand     \&
%   Acute accent  \'     Left paren    \(     Right paren   \)
%   Asterisk      \*     Plus          \+     Comma         \,
%   Minus         \-     Point         \.     Solidus       \/
%   Colon         \:     Semicolon     \;     Less than     \<
%   Equals        \=     Greater than  \>     Question mark \?
%   Commercial at \@     Left bracket  \[     Backslash     \\
%   Right bracket \]     Circumflex    \^     Underscore    \_
%   Grave accent  \`     Left brace    \{     Vertical bar  \|
%   Right brace   \}     Tilde         \~}
%
% \GetFileInfo{scrindex.drv}
%
% \title{The \xpackage{scrindex} package}
% \date{2008/08/11 v1.1}
% \author{Heiko Oberdiek\\\xemail{heiko.oberdiek at googlemail.com}}
%
% \maketitle
%
% \begin{abstract}
% This package redefines environment `theindex' of package \xpackage{index},
% if a class from KOMA-Script is loaded. Also option \xoption{idxtotoc}
% is supported. Index preambles can be given either by means of package
% \xpackage{index} or via the interface provided by KOMA-Script.
% \end{abstract}
%
% \tableofcontents
%
% \section{Documentation}
%
% Package \xpackage{index}, written by David M.\ Jones, detects
% the standard classes |article|, |report|, and |book|. It
% redefines environment `theindex' for its needs.
% However, it does not know other classes such as KOMA-Script.
% This package closes the compatibiliy gap between KOMA-Script's
% classes and package \xpackage{index}.
%
% Environment |theindex| is redefined to support both package
% \xpackage{index} and KOMA-Script's classes. Thus both
% the prologe of package \xpackage{index} and the preamble
% of KOMA-Script's classes are available. Also class option |idxtotoc|
% of KOMA-Script is supported.
%
% \subsection{Usage}
%
% The package \xpackage{scrindex} is loaded without options:
%\begin{quote}
%\begin{verbatim}
%\usepackage{scrindex}
%\end{verbatim}
%\end{quote}
%
% It loads package \xpackage{index} and requests version 2004/01/20
% or later. \LaTeX's package interface allows multiple calls
% of the same package. The package is loaded at its first
% package loading command. At later times \LaTeX\ only checks
% options and a requested version date. Therefore it does not harm
% to add |\usepackage{index}| before or after |\usepackage{scrindex}|.
%
% Also the class does not matter. Environment |theindex| is only
% redefined for a supported class:
% \begin{itemize}
% \item |scrartcl|
% \item |scrreprt|
% \item |scrbook|
% \end{itemize}
%
% \subsection{Preambles}
%
% Both the prologue of package \xpackage{index} and the preamble
% of KOMA-Script's classes are supported. The position depends on
% the class.
%
% \subsubsection{Class \xclass{scrartcl}}
%
%    \begin{macrocode}
%<*example1>
\documentclass{scrartcl}
\usepackage{scrindex}
\setindexpreamble{Preamble of \texttt{scrartcl}\dotfill EOL}
\makeindex
\begin{document}
\section{First Section}
\index{first}
\index{section}
\printindex[default]%
  [Prologue of package \texttt{index}\dotfill EOL]%
\end{document}
%</example1>
%    \end{macrocode}
% The prologue of package \xpackage{index} is first set straight
% after the section title spanning both columns.
% Then the preamble of KOMA-Script follows
% in the first left column.
%
% \medskip
% \begin{quote}
%   \renewcommand*{\arraystretch}{1.2}
%   \begin{tabular}{|p{.45\linewidth}|p{.45\linewidth}|}
%   \hline
%   \multicolumn{2}{|l|}{\textbf{Index}}\\[1ex]
%   \multicolumn{2}{|p{.9\linewidth+2\tabcolsep}|}{^^A
%     Prologue of package \texttt{index}\dotfill EOL^^A
%   }\\[1ex]
%   \hline
%   Preamble of \texttt{scrartcl}\dotfill EOL&\\
%   first, 1&\\
%   section, 1&\\
%   \hline
%   \end{tabular}
% \end{quote}
%
% \subsubsection{Classes \xclass{scrreprt} and \xclass{scrbook}}
%
%    \begin{macrocode}
%<*example2>
\documentclass[openany]{scrbook}% or scrreprt
\usepackage{scrindex}
\setindexpreamble{Preamble of class \texttt{scrbook}\dotfill EOL}
\makeindex
\begin{document}
\chapter{First Chapter}
\index{first}
\index{chapter}
\printindex[default]%
  [Prologue of package \texttt{index}\dotfill EOL]%
\end{document}
%</example2>
%    \end{macrocode}
% The order of the two preambles are different for the classes
% \xclass{scrreprt} and \xclass{scrbook}. First KOMA-Script's
% chapter preamble is set, then the prologue of package \xpackage{index}
% follows. Both are set spanning both columns.
%
% \medskip
% \begin{quote}
%   \renewcommand*{\arraystretch}{1.2}
%   \begin{tabular}{|p{.45\linewidth}|p{.45\linewidth}|}
%   \hline
%   \multicolumn{2}{|l|}{\textbf{Index}}\\[1ex]
%   \multicolumn{2}{|p{.9\linewidth+2\tabcolsep}|}{^^A
%     Preamble of class \texttt{scrbook}\dotfill EOL^^A
%   }\\
%   \multicolumn{2}{|p{.9\linewidth+2\tabcolsep}|}{^^A
%     Prologue of package \xpackage{index}\dotfill EOL^^A
%   }\\[1ex]
%   \hline
%   chapter, 1&\\
%   first, 1&\\
%   \hline
%   \end{tabular}
% \end{quote}
%
% \StopEventually{
% }
%
% \section{Implementation}
%
%    \begin{macrocode}
%<*package>
\NeedsTeXFormat{LaTeX2e}
\ProvidesPackage{scrindex}
  [2008/08/11 v1.1 Package index with KOMA-Script classes (HO)]%
%    \end{macrocode}
%
%    \begin{macrocode}
\RequirePackage{index}[2004/01/20]%
%    \end{macrocode}
%
%    \begin{macrocode}
\@ifclassloaded{scrartcl}{%
  \renewenvironment{theindex}{%
    \edef\indexname{%
      \the\@nameuse{idxtitle@\@indextype}%
    }%
    \if@twocolumn
      \@restonecolfalse
    \else
      \@restonecoltrue
    \fi
    \idx@heading
    \thispagestyle{\indexpagestyle}%
    \columnseprule\z@
    \columnsep 35\p@
    \index@preamble\par\nobreak
    \parindent\z@
    \parskip\z@ \@plus .3\p@\relax
    \parfillskip\z@ \@plus 1fil\relax
    \let\item\@idxitem
  }{%
    \if@restonecol
      \onecolumn
    \else
      \clearpage
    \fi
  }%
  \@ifclasswith{scrartcl}{idxtotoc}{%
    \renewcommand*{\idx@heading}{%
      \twocolumn[%
        \addsec{\indexname}%
        \ifx\index@prologue\@empty
        \else
          \index@prologue
          \bigskip
        \fi
      ]%
      \@mkboth{\indexname}{\indexname}%
    }%
  }{%
    \renewcommand*{\idx@heading}{%
      \twocolumn[%
        \section*{\indexname}%
        \ifx\index@prologue\@empty
        \else
          \index@prologue
          \bigskip
        \fi
      ]%
      \@mkboth{\indexname}{\indexname}%
    }%
  }%
}{}
%    \end{macrocode}
%    \begin{macrocode}
\@ifclassloaded{scrreprt}{%
  \renewenvironment{theindex}{%
    \edef\indexname{%
      \the\@nameuse{idxtitle@\@indextype}%
    }%
    \if@twocolumn
      \@restonecolfalse
    \else
      \@restonecoltrue
    \fi
    \setchapterpreamble{\index@preamble}%
    \idx@heading
    \thispagestyle{\indexpagestyle}%
    \columnseprule\z@
    \columnsep 35\p@
    \parindent\z@
    \parskip\z@ \@plus .3\p@\relax
    \parfillskip\z@ \@plus 1fil\relax
    \let\item\@idxitem
  }{%
    \if@restonecol
      \onecolumn
    \else
      \clearpage
    \fi
  }%
  \@ifclasswith{scrreprt}{idxtotoc}{%
    \renewcommand*{\idx@heading}{%
      \if@openright
        \cleardoublepage
      \else
        \clearpage
      \fi
      \twocolumn[%
        \addchap{\indexname}%
        \ifx\index@prologue\@empty
        \else
          \index@prologue
          \bigskip
        \fi
      ]%
      \@mkboth{\indexname}{\indexname}%
    }%
  }{%
    \renewcommand*{\idx@heading}{%
      \if@openright
        \cleardoublepage
      \else
        \clearpage
      \fi
      \twocolumn[%
        \chapter*{\indexname}%
        \ifx\index@prologue\@empty
        \else
          \index@prologue
          \bigskip
        \fi
      ]%
      \@mkboth{\indexname}{\indexname}%
    }%
  }%
}{}
%    \end{macrocode}
%    \begin{macrocode}
\@ifclassloaded{scrbook}{%
  \renewenvironment{theindex}{%
    \edef\indexname{%
      \the\@nameuse{idxtitle@\@indextype}%
    }%
    \if@twocolumn
      \@restonecolfalse
    \else
      \@restonecoltrue
    \fi
    \setchapterpreamble{\index@preamble}%
    \idx@heading
    \thispagestyle{\indexpagestyle}%
    \columnseprule\z@
    \columnsep 35\p@
    \parindent\z@
    \parskip\z@ \@plus .3\p@\relax
    \parfillskip\z@ \@plus 1fil\relax
    \let\item\@idxitem
  }{%
    \if@restonecol
      \onecolumn
    \else
      \clearpage
    \fi
  }%
  \@ifclasswith{scrbook}{idxtotoc}{%
    \renewcommand*{\idx@heading}{%
      \if@openright
        \cleardoublepage
      \else
        \clearpage
      \fi
      \twocolumn[%
        \addchap{\indexname}%
        \ifx\index@prologue\@empty
        \else
          \index@prologue
          \bigskip
        \fi
      ]%
      \@mkboth{\indexname}{\indexname}%
    }%
  }{%
    \renewcommand*{\idx@heading}{%
      \if@openright
        \cleardoublepage
      \else
        \clearpage
      \fi
      \twocolumn[%
        \chapter*{\indexname}%
        \ifx\index@prologue\@empty
        \else
          \index@prologue
          \bigskip
        \fi
      ]%
      \@mkboth{\indexname}{\indexname}%
    }%
  }%
}{}
%    \end{macrocode}
%
%    \begin{macrocode}
%</package>
%    \end{macrocode}
%
% \section{Installation}
%
% \subsection{Download}
%
% \paragraph{Package.} This package is available on
% CTAN\footnote{\url{ftp://ftp.ctan.org/tex-archive/}}:
% \begin{description}
% \item[\CTAN{macros/latex/contrib/oberdiek/scrindex.dtx}] The source file.
% \item[\CTAN{macros/latex/contrib/oberdiek/scrindex.pdf}] Documentation.
% \end{description}
%
%
% \paragraph{Bundle.} All the packages of the bundle `oberdiek'
% are also available in a TDS compliant ZIP archive. There
% the packages are already unpacked and the documentation files
% are generated. The files and directories obey the TDS standard.
% \begin{description}
% \item[\CTAN{install/macros/latex/contrib/oberdiek.tds.zip}]
% \end{description}
% \emph{TDS} refers to the standard ``A Directory Structure
% for \TeX\ Files'' (\CTAN{tds/tds.pdf}). Directories
% with \xfile{texmf} in their name are usually organized this way.
%
% \subsection{Bundle installation}
%
% \paragraph{Unpacking.} Unpack the \xfile{oberdiek.tds.zip} in the
% TDS tree (also known as \xfile{texmf} tree) of your choice.
% Example (linux):
% \begin{quote}
%   |unzip oberdiek.tds.zip -d ~/texmf|
% \end{quote}
%
% \paragraph{Script installation.}
% Check the directory \xfile{TDS:scripts/oberdiek/} for
% scripts that need further installation steps.
% Package \xpackage{attachfile2} comes with the Perl script
% \xfile{pdfatfi.pl} that should be installed in such a way
% that it can be called as \texttt{pdfatfi}.
% Example (linux):
% \begin{quote}
%   |chmod +x scripts/oberdiek/pdfatfi.pl|\\
%   |cp scripts/oberdiek/pdfatfi.pl /usr/local/bin/|
% \end{quote}
%
% \subsection{Package installation}
%
% \paragraph{Unpacking.} The \xfile{.dtx} file is a self-extracting
% \docstrip\ archive. The files are extracted by running the
% \xfile{.dtx} through \plainTeX:
% \begin{quote}
%   \verb|tex scrindex.dtx|
% \end{quote}
%
% \paragraph{TDS.} Now the different files must be moved into
% the different directories in your installation TDS tree
% (also known as \xfile{texmf} tree):
% \begin{quote}
% \def\t{^^A
% \begin{tabular}{@{}>{\ttfamily}l@{ $\rightarrow$ }>{\ttfamily}l@{}}
%   scrindex.sty & tex/latex/oberdiek/scrindex.sty\\
%   scrindex.pdf & doc/latex/oberdiek/scrindex.pdf\\
%   scrindex-example1.tex & doc/latex/oberdiek/scrindex-example1.tex\\
%   scrindex-example2.tex & doc/latex/oberdiek/scrindex-example2.tex\\
%   scrindex.dtx & source/latex/oberdiek/scrindex.dtx\\
% \end{tabular}^^A
% }^^A
% \sbox0{\t}^^A
% \ifdim\wd0>\linewidth
%   \begingroup
%     \advance\linewidth by\leftmargin
%     \advance\linewidth by\rightmargin
%   \edef\x{\endgroup
%     \def\noexpand\lw{\the\linewidth}^^A
%   }\x
%   \def\lwbox{^^A
%     \leavevmode
%     \hbox to \linewidth{^^A
%       \kern-\leftmargin\relax
%       \hss
%       \usebox0
%       \hss
%       \kern-\rightmargin\relax
%     }^^A
%   }^^A
%   \ifdim\wd0>\lw
%     \sbox0{\small\t}^^A
%     \ifdim\wd0>\linewidth
%       \ifdim\wd0>\lw
%         \sbox0{\footnotesize\t}^^A
%         \ifdim\wd0>\linewidth
%           \ifdim\wd0>\lw
%             \sbox0{\scriptsize\t}^^A
%             \ifdim\wd0>\linewidth
%               \ifdim\wd0>\lw
%                 \sbox0{\tiny\t}^^A
%                 \ifdim\wd0>\linewidth
%                   \lwbox
%                 \else
%                   \usebox0
%                 \fi
%               \else
%                 \lwbox
%               \fi
%             \else
%               \usebox0
%             \fi
%           \else
%             \lwbox
%           \fi
%         \else
%           \usebox0
%         \fi
%       \else
%         \lwbox
%       \fi
%     \else
%       \usebox0
%     \fi
%   \else
%     \lwbox
%   \fi
% \else
%   \usebox0
% \fi
% \end{quote}
% If you have a \xfile{docstrip.cfg} that configures and enables \docstrip's
% TDS installing feature, then some files can already be in the right
% place, see the documentation of \docstrip.
%
% \subsection{Refresh file name databases}
%
% If your \TeX~distribution
% (\teTeX, \mikTeX, \dots) relies on file name databases, you must refresh
% these. For example, \teTeX\ users run \verb|texhash| or
% \verb|mktexlsr|.
%
% \subsection{Some details for the interested}
%
% \paragraph{Attached source.}
%
% The PDF documentation on CTAN also includes the
% \xfile{.dtx} source file. It can be extracted by
% AcrobatReader 6 or higher. Another option is \textsf{pdftk},
% e.g. unpack the file into the current directory:
% \begin{quote}
%   \verb|pdftk scrindex.pdf unpack_files output .|
% \end{quote}
%
% \paragraph{Unpacking with \LaTeX.}
% The \xfile{.dtx} chooses its action depending on the format:
% \begin{description}
% \item[\plainTeX:] Run \docstrip\ and extract the files.
% \item[\LaTeX:] Generate the documentation.
% \end{description}
% If you insist on using \LaTeX\ for \docstrip\ (really,
% \docstrip\ does not need \LaTeX), then inform the autodetect routine
% about your intention:
% \begin{quote}
%   \verb|latex \let\install=y\input{scrindex.dtx}|
% \end{quote}
% Do not forget to quote the argument according to the demands
% of your shell.
%
% \paragraph{Generating the documentation.}
% You can use both the \xfile{.dtx} or the \xfile{.drv} to generate
% the documentation. The process can be configured by the
% configuration file \xfile{ltxdoc.cfg}. For instance, put this
% line into this file, if you want to have A4 as paper format:
% \begin{quote}
%   \verb|\PassOptionsToClass{a4paper}{article}|
% \end{quote}
% An example follows how to generate the
% documentation with pdf\LaTeX:
% \begin{quote}
%\begin{verbatim}
%pdflatex scrindex.dtx
%makeindex -s gind.ist scrindex.idx
%pdflatex scrindex.dtx
%makeindex -s gind.ist scrindex.idx
%pdflatex scrindex.dtx
%\end{verbatim}
% \end{quote}
%
% \section{Catalogue}
%
% The following XML file can be used as source for the
% \href{http://mirror.ctan.org/help/Catalogue/catalogue.html}{\TeX\ Catalogue}.
% The elements \texttt{caption} and \texttt{description} are imported
% from the original XML file from the Catalogue.
% The name of the XML file in the Catalogue is \xfile{scrindex.xml}.
%    \begin{macrocode}
%<*catalogue>
<?xml version='1.0' encoding='us-ascii'?>
<!DOCTYPE entry SYSTEM 'catalogue.dtd'>
<entry datestamp='$Date$' modifier='$Author$' id='scrindex'>
  <name>scrindex</name>
  <caption>Make index package work with Koma-script classes.</caption>
  <authorref id='auth:oberdiek'/>
  <copyright owner='Heiko Oberdiek' year='2008'/>
  <license type='lppl1.3'/>
  <version number='1.1'/>
  <description>
    This package redefines environment `theindex' of package `index',
    if a class from <xref refid='koma-script'>KOMA-Script</xref> is loaded.
    Also option `idxtotoc' is supported. Index preambles can be given
    either by means of package `index' or via the interface provided
    by <xref refid='koma-script'>KOMA-Script</xref>.
    <p/>
    The package is part of the <xref refid='oberdiek'>oberdiek</xref>
    bundle.
  </description>
  <documentation details='Package documentation'
      href='ctan:/macros/latex/contrib/oberdiek/scrindex.pdf'/>
  <ctan file='true' path='/macros/latex/contrib/oberdiek/scrindex.dtx'/>
  <miktex location='oberdiek'/>
  <texlive location='oberdiek'/>
  <install path='/macros/latex/contrib/oberdiek/oberdiek.tds.zip'/>
</entry>
%</catalogue>
%    \end{macrocode}
%
% \begin{History}
%   \begin{Version}{2008/07/07 v1.0}
%   \item
%     First version, also published in newsgroup \xnewsgroup{de.comp.text.tex}:\\
%     \URL{``\link{Re: Z\"ahler bei \cs{index}}''}^^A
%     {http://groups.google.com/group/de.comp.text.tex/msg/39575b5e2f29be1e}
%   \end{Version}
%   \begin{Version}{2008/08/11 v1.1}
%   \item
%     Code is not changed.
%   \item
%     URLs updated.
%   \end{Version}
% \end{History}
%
% \PrintIndex
%
% \Finale
\endinput

%        (quote the arguments according to the demands of your shell)
%
% Documentation:
%    (a) If scrindex.drv is present:
%           latex scrindex.drv
%    (b) Without scrindex.drv:
%           latex scrindex.dtx; ...
%    The class ltxdoc loads the configuration file ltxdoc.cfg
%    if available. Here you can specify further options, e.g.
%    use A4 as paper format:
%       \PassOptionsToClass{a4paper}{article}
%
%    Programm calls to get the documentation (example):
%       pdflatex scrindex.dtx
%       makeindex -s gind.ist scrindex.idx
%       pdflatex scrindex.dtx
%       makeindex -s gind.ist scrindex.idx
%       pdflatex scrindex.dtx
%
% Installation:
%    TDS:tex/latex/oberdiek/scrindex.sty
%    TDS:doc/latex/oberdiek/scrindex.pdf
%    TDS:doc/latex/oberdiek/scrindex-example1.tex
%    TDS:doc/latex/oberdiek/scrindex-example2.tex
%    TDS:source/latex/oberdiek/scrindex.dtx
%
%<*ignore>
\begingroup
  \catcode123=1 %
  \catcode125=2 %
  \def\x{LaTeX2e}%
\expandafter\endgroup
\ifcase 0\ifx\install y1\fi\expandafter
         \ifx\csname processbatchFile\endcsname\relax\else1\fi
         \ifx\fmtname\x\else 1\fi\relax
\else\csname fi\endcsname
%</ignore>
%<*install>
\input docstrip.tex
\Msg{************************************************************************}
\Msg{* Installation}
\Msg{* Package: scrindex 2008/08/11 v1.1 Package index with KOMA-Script classes (HO)}
\Msg{************************************************************************}

\keepsilent
\askforoverwritefalse

\let\MetaPrefix\relax
\preamble

This is a generated file.

Project: scrindex
Version: 2008/08/11 v1.1

Copyright (C) 2008 by
   Heiko Oberdiek <heiko.oberdiek at googlemail.com>

This work may be distributed and/or modified under the
conditions of the LaTeX Project Public License, either
version 1.3c of this license or (at your option) any later
version. This version of this license is in
   http://www.latex-project.org/lppl/lppl-1-3c.txt
and the latest version of this license is in
   http://www.latex-project.org/lppl.txt
and version 1.3 or later is part of all distributions of
LaTeX version 2005/12/01 or later.

This work has the LPPL maintenance status "maintained".

This Current Maintainer of this work is Heiko Oberdiek.

This work consists of the main source file scrindex.dtx
and the derived files
   scrindex.sty, scrindex.pdf, scrindex.ins, scrindex.drv,
   scrindex-example1.tex, scrindex-example2.tex.

\endpreamble
\let\MetaPrefix\DoubleperCent

\generate{%
  \file{scrindex.ins}{\from{scrindex.dtx}{install}}%
  \file{scrindex.drv}{\from{scrindex.dtx}{driver}}%
  \usedir{tex/latex/oberdiek}%
  \file{scrindex.sty}{\from{scrindex.dtx}{package}}%
  \usedir{doc/latex/oberdiek}%
  \file{scrindex-example1.tex}{\from{scrindex.dtx}{example1}}%
  \file{scrindex-example2.tex}{\from{scrindex.dtx}{example2}}%
  \nopreamble
  \nopostamble
  \usedir{source/latex/oberdiek/catalogue}%
  \file{scrindex.xml}{\from{scrindex.dtx}{catalogue}}%
}

\catcode32=13\relax% active space
\let =\space%
\Msg{************************************************************************}
\Msg{*}
\Msg{* To finish the installation you have to move the following}
\Msg{* file into a directory searched by TeX:}
\Msg{*}
\Msg{*     scrindex.sty}
\Msg{*}
\Msg{* To produce the documentation run the file `scrindex.drv'}
\Msg{* through LaTeX.}
\Msg{*}
\Msg{* Happy TeXing!}
\Msg{*}
\Msg{************************************************************************}

\endbatchfile
%</install>
%<*ignore>
\fi
%</ignore>
%<*driver>
\NeedsTeXFormat{LaTeX2e}
\ProvidesFile{scrindex.drv}%
  [2008/08/11 v1.1 Package index with KOMA-Script classes (HO)]%
\documentclass{ltxdoc}
\usepackage{holtxdoc}[2011/11/22]
\usepackage{calc}
\begin{document}
  \DocInput{scrindex.dtx}%
\end{document}
%</driver>
% \fi
%
% \CheckSum{237}
%
% \CharacterTable
%  {Upper-case    \A\B\C\D\E\F\G\H\I\J\K\L\M\N\O\P\Q\R\S\T\U\V\W\X\Y\Z
%   Lower-case    \a\b\c\d\e\f\g\h\i\j\k\l\m\n\o\p\q\r\s\t\u\v\w\x\y\z
%   Digits        \0\1\2\3\4\5\6\7\8\9
%   Exclamation   \!     Double quote  \"     Hash (number) \#
%   Dollar        \$     Percent       \%     Ampersand     \&
%   Acute accent  \'     Left paren    \(     Right paren   \)
%   Asterisk      \*     Plus          \+     Comma         \,
%   Minus         \-     Point         \.     Solidus       \/
%   Colon         \:     Semicolon     \;     Less than     \<
%   Equals        \=     Greater than  \>     Question mark \?
%   Commercial at \@     Left bracket  \[     Backslash     \\
%   Right bracket \]     Circumflex    \^     Underscore    \_
%   Grave accent  \`     Left brace    \{     Vertical bar  \|
%   Right brace   \}     Tilde         \~}
%
% \GetFileInfo{scrindex.drv}
%
% \title{The \xpackage{scrindex} package}
% \date{2008/08/11 v1.1}
% \author{Heiko Oberdiek\\\xemail{heiko.oberdiek at googlemail.com}}
%
% \maketitle
%
% \begin{abstract}
% This package redefines environment `theindex' of package \xpackage{index},
% if a class from KOMA-Script is loaded. Also option \xoption{idxtotoc}
% is supported. Index preambles can be given either by means of package
% \xpackage{index} or via the interface provided by KOMA-Script.
% \end{abstract}
%
% \tableofcontents
%
% \section{Documentation}
%
% Package \xpackage{index}, written by David M.\ Jones, detects
% the standard classes |article|, |report|, and |book|. It
% redefines environment `theindex' for its needs.
% However, it does not know other classes such as KOMA-Script.
% This package closes the compatibiliy gap between KOMA-Script's
% classes and package \xpackage{index}.
%
% Environment |theindex| is redefined to support both package
% \xpackage{index} and KOMA-Script's classes. Thus both
% the prologe of package \xpackage{index} and the preamble
% of KOMA-Script's classes are available. Also class option |idxtotoc|
% of KOMA-Script is supported.
%
% \subsection{Usage}
%
% The package \xpackage{scrindex} is loaded without options:
%\begin{quote}
%\begin{verbatim}
%\usepackage{scrindex}
%\end{verbatim}
%\end{quote}
%
% It loads package \xpackage{index} and requests version 2004/01/20
% or later. \LaTeX's package interface allows multiple calls
% of the same package. The package is loaded at its first
% package loading command. At later times \LaTeX\ only checks
% options and a requested version date. Therefore it does not harm
% to add |\usepackage{index}| before or after |\usepackage{scrindex}|.
%
% Also the class does not matter. Environment |theindex| is only
% redefined for a supported class:
% \begin{itemize}
% \item |scrartcl|
% \item |scrreprt|
% \item |scrbook|
% \end{itemize}
%
% \subsection{Preambles}
%
% Both the prologue of package \xpackage{index} and the preamble
% of KOMA-Script's classes are supported. The position depends on
% the class.
%
% \subsubsection{Class \xclass{scrartcl}}
%
%    \begin{macrocode}
%<*example1>
\documentclass{scrartcl}
\usepackage{scrindex}
\setindexpreamble{Preamble of \texttt{scrartcl}\dotfill EOL}
\makeindex
\begin{document}
\section{First Section}
\index{first}
\index{section}
\printindex[default]%
  [Prologue of package \texttt{index}\dotfill EOL]%
\end{document}
%</example1>
%    \end{macrocode}
% The prologue of package \xpackage{index} is first set straight
% after the section title spanning both columns.
% Then the preamble of KOMA-Script follows
% in the first left column.
%
% \medskip
% \begin{quote}
%   \renewcommand*{\arraystretch}{1.2}
%   \begin{tabular}{|p{.45\linewidth}|p{.45\linewidth}|}
%   \hline
%   \multicolumn{2}{|l|}{\textbf{Index}}\\[1ex]
%   \multicolumn{2}{|p{.9\linewidth+2\tabcolsep}|}{^^A
%     Prologue of package \texttt{index}\dotfill EOL^^A
%   }\\[1ex]
%   \hline
%   Preamble of \texttt{scrartcl}\dotfill EOL&\\
%   first, 1&\\
%   section, 1&\\
%   \hline
%   \end{tabular}
% \end{quote}
%
% \subsubsection{Classes \xclass{scrreprt} and \xclass{scrbook}}
%
%    \begin{macrocode}
%<*example2>
\documentclass[openany]{scrbook}% or scrreprt
\usepackage{scrindex}
\setindexpreamble{Preamble of class \texttt{scrbook}\dotfill EOL}
\makeindex
\begin{document}
\chapter{First Chapter}
\index{first}
\index{chapter}
\printindex[default]%
  [Prologue of package \texttt{index}\dotfill EOL]%
\end{document}
%</example2>
%    \end{macrocode}
% The order of the two preambles are different for the classes
% \xclass{scrreprt} and \xclass{scrbook}. First KOMA-Script's
% chapter preamble is set, then the prologue of package \xpackage{index}
% follows. Both are set spanning both columns.
%
% \medskip
% \begin{quote}
%   \renewcommand*{\arraystretch}{1.2}
%   \begin{tabular}{|p{.45\linewidth}|p{.45\linewidth}|}
%   \hline
%   \multicolumn{2}{|l|}{\textbf{Index}}\\[1ex]
%   \multicolumn{2}{|p{.9\linewidth+2\tabcolsep}|}{^^A
%     Preamble of class \texttt{scrbook}\dotfill EOL^^A
%   }\\
%   \multicolumn{2}{|p{.9\linewidth+2\tabcolsep}|}{^^A
%     Prologue of package \xpackage{index}\dotfill EOL^^A
%   }\\[1ex]
%   \hline
%   chapter, 1&\\
%   first, 1&\\
%   \hline
%   \end{tabular}
% \end{quote}
%
% \StopEventually{
% }
%
% \section{Implementation}
%
%    \begin{macrocode}
%<*package>
\NeedsTeXFormat{LaTeX2e}
\ProvidesPackage{scrindex}
  [2008/08/11 v1.1 Package index with KOMA-Script classes (HO)]%
%    \end{macrocode}
%
%    \begin{macrocode}
\RequirePackage{index}[2004/01/20]%
%    \end{macrocode}
%
%    \begin{macrocode}
\@ifclassloaded{scrartcl}{%
  \renewenvironment{theindex}{%
    \edef\indexname{%
      \the\@nameuse{idxtitle@\@indextype}%
    }%
    \if@twocolumn
      \@restonecolfalse
    \else
      \@restonecoltrue
    \fi
    \idx@heading
    \thispagestyle{\indexpagestyle}%
    \columnseprule\z@
    \columnsep 35\p@
    \index@preamble\par\nobreak
    \parindent\z@
    \parskip\z@ \@plus .3\p@\relax
    \parfillskip\z@ \@plus 1fil\relax
    \let\item\@idxitem
  }{%
    \if@restonecol
      \onecolumn
    \else
      \clearpage
    \fi
  }%
  \@ifclasswith{scrartcl}{idxtotoc}{%
    \renewcommand*{\idx@heading}{%
      \twocolumn[%
        \addsec{\indexname}%
        \ifx\index@prologue\@empty
        \else
          \index@prologue
          \bigskip
        \fi
      ]%
      \@mkboth{\indexname}{\indexname}%
    }%
  }{%
    \renewcommand*{\idx@heading}{%
      \twocolumn[%
        \section*{\indexname}%
        \ifx\index@prologue\@empty
        \else
          \index@prologue
          \bigskip
        \fi
      ]%
      \@mkboth{\indexname}{\indexname}%
    }%
  }%
}{}
%    \end{macrocode}
%    \begin{macrocode}
\@ifclassloaded{scrreprt}{%
  \renewenvironment{theindex}{%
    \edef\indexname{%
      \the\@nameuse{idxtitle@\@indextype}%
    }%
    \if@twocolumn
      \@restonecolfalse
    \else
      \@restonecoltrue
    \fi
    \setchapterpreamble{\index@preamble}%
    \idx@heading
    \thispagestyle{\indexpagestyle}%
    \columnseprule\z@
    \columnsep 35\p@
    \parindent\z@
    \parskip\z@ \@plus .3\p@\relax
    \parfillskip\z@ \@plus 1fil\relax
    \let\item\@idxitem
  }{%
    \if@restonecol
      \onecolumn
    \else
      \clearpage
    \fi
  }%
  \@ifclasswith{scrreprt}{idxtotoc}{%
    \renewcommand*{\idx@heading}{%
      \if@openright
        \cleardoublepage
      \else
        \clearpage
      \fi
      \twocolumn[%
        \addchap{\indexname}%
        \ifx\index@prologue\@empty
        \else
          \index@prologue
          \bigskip
        \fi
      ]%
      \@mkboth{\indexname}{\indexname}%
    }%
  }{%
    \renewcommand*{\idx@heading}{%
      \if@openright
        \cleardoublepage
      \else
        \clearpage
      \fi
      \twocolumn[%
        \chapter*{\indexname}%
        \ifx\index@prologue\@empty
        \else
          \index@prologue
          \bigskip
        \fi
      ]%
      \@mkboth{\indexname}{\indexname}%
    }%
  }%
}{}
%    \end{macrocode}
%    \begin{macrocode}
\@ifclassloaded{scrbook}{%
  \renewenvironment{theindex}{%
    \edef\indexname{%
      \the\@nameuse{idxtitle@\@indextype}%
    }%
    \if@twocolumn
      \@restonecolfalse
    \else
      \@restonecoltrue
    \fi
    \setchapterpreamble{\index@preamble}%
    \idx@heading
    \thispagestyle{\indexpagestyle}%
    \columnseprule\z@
    \columnsep 35\p@
    \parindent\z@
    \parskip\z@ \@plus .3\p@\relax
    \parfillskip\z@ \@plus 1fil\relax
    \let\item\@idxitem
  }{%
    \if@restonecol
      \onecolumn
    \else
      \clearpage
    \fi
  }%
  \@ifclasswith{scrbook}{idxtotoc}{%
    \renewcommand*{\idx@heading}{%
      \if@openright
        \cleardoublepage
      \else
        \clearpage
      \fi
      \twocolumn[%
        \addchap{\indexname}%
        \ifx\index@prologue\@empty
        \else
          \index@prologue
          \bigskip
        \fi
      ]%
      \@mkboth{\indexname}{\indexname}%
    }%
  }{%
    \renewcommand*{\idx@heading}{%
      \if@openright
        \cleardoublepage
      \else
        \clearpage
      \fi
      \twocolumn[%
        \chapter*{\indexname}%
        \ifx\index@prologue\@empty
        \else
          \index@prologue
          \bigskip
        \fi
      ]%
      \@mkboth{\indexname}{\indexname}%
    }%
  }%
}{}
%    \end{macrocode}
%
%    \begin{macrocode}
%</package>
%    \end{macrocode}
%
% \section{Installation}
%
% \subsection{Download}
%
% \paragraph{Package.} This package is available on
% CTAN\footnote{\url{ftp://ftp.ctan.org/tex-archive/}}:
% \begin{description}
% \item[\CTAN{macros/latex/contrib/oberdiek/scrindex.dtx}] The source file.
% \item[\CTAN{macros/latex/contrib/oberdiek/scrindex.pdf}] Documentation.
% \end{description}
%
%
% \paragraph{Bundle.} All the packages of the bundle `oberdiek'
% are also available in a TDS compliant ZIP archive. There
% the packages are already unpacked and the documentation files
% are generated. The files and directories obey the TDS standard.
% \begin{description}
% \item[\CTAN{install/macros/latex/contrib/oberdiek.tds.zip}]
% \end{description}
% \emph{TDS} refers to the standard ``A Directory Structure
% for \TeX\ Files'' (\CTAN{tds/tds.pdf}). Directories
% with \xfile{texmf} in their name are usually organized this way.
%
% \subsection{Bundle installation}
%
% \paragraph{Unpacking.} Unpack the \xfile{oberdiek.tds.zip} in the
% TDS tree (also known as \xfile{texmf} tree) of your choice.
% Example (linux):
% \begin{quote}
%   |unzip oberdiek.tds.zip -d ~/texmf|
% \end{quote}
%
% \paragraph{Script installation.}
% Check the directory \xfile{TDS:scripts/oberdiek/} for
% scripts that need further installation steps.
% Package \xpackage{attachfile2} comes with the Perl script
% \xfile{pdfatfi.pl} that should be installed in such a way
% that it can be called as \texttt{pdfatfi}.
% Example (linux):
% \begin{quote}
%   |chmod +x scripts/oberdiek/pdfatfi.pl|\\
%   |cp scripts/oberdiek/pdfatfi.pl /usr/local/bin/|
% \end{quote}
%
% \subsection{Package installation}
%
% \paragraph{Unpacking.} The \xfile{.dtx} file is a self-extracting
% \docstrip\ archive. The files are extracted by running the
% \xfile{.dtx} through \plainTeX:
% \begin{quote}
%   \verb|tex scrindex.dtx|
% \end{quote}
%
% \paragraph{TDS.} Now the different files must be moved into
% the different directories in your installation TDS tree
% (also known as \xfile{texmf} tree):
% \begin{quote}
% \def\t{^^A
% \begin{tabular}{@{}>{\ttfamily}l@{ $\rightarrow$ }>{\ttfamily}l@{}}
%   scrindex.sty & tex/latex/oberdiek/scrindex.sty\\
%   scrindex.pdf & doc/latex/oberdiek/scrindex.pdf\\
%   scrindex-example1.tex & doc/latex/oberdiek/scrindex-example1.tex\\
%   scrindex-example2.tex & doc/latex/oberdiek/scrindex-example2.tex\\
%   scrindex.dtx & source/latex/oberdiek/scrindex.dtx\\
% \end{tabular}^^A
% }^^A
% \sbox0{\t}^^A
% \ifdim\wd0>\linewidth
%   \begingroup
%     \advance\linewidth by\leftmargin
%     \advance\linewidth by\rightmargin
%   \edef\x{\endgroup
%     \def\noexpand\lw{\the\linewidth}^^A
%   }\x
%   \def\lwbox{^^A
%     \leavevmode
%     \hbox to \linewidth{^^A
%       \kern-\leftmargin\relax
%       \hss
%       \usebox0
%       \hss
%       \kern-\rightmargin\relax
%     }^^A
%   }^^A
%   \ifdim\wd0>\lw
%     \sbox0{\small\t}^^A
%     \ifdim\wd0>\linewidth
%       \ifdim\wd0>\lw
%         \sbox0{\footnotesize\t}^^A
%         \ifdim\wd0>\linewidth
%           \ifdim\wd0>\lw
%             \sbox0{\scriptsize\t}^^A
%             \ifdim\wd0>\linewidth
%               \ifdim\wd0>\lw
%                 \sbox0{\tiny\t}^^A
%                 \ifdim\wd0>\linewidth
%                   \lwbox
%                 \else
%                   \usebox0
%                 \fi
%               \else
%                 \lwbox
%               \fi
%             \else
%               \usebox0
%             \fi
%           \else
%             \lwbox
%           \fi
%         \else
%           \usebox0
%         \fi
%       \else
%         \lwbox
%       \fi
%     \else
%       \usebox0
%     \fi
%   \else
%     \lwbox
%   \fi
% \else
%   \usebox0
% \fi
% \end{quote}
% If you have a \xfile{docstrip.cfg} that configures and enables \docstrip's
% TDS installing feature, then some files can already be in the right
% place, see the documentation of \docstrip.
%
% \subsection{Refresh file name databases}
%
% If your \TeX~distribution
% (\teTeX, \mikTeX, \dots) relies on file name databases, you must refresh
% these. For example, \teTeX\ users run \verb|texhash| or
% \verb|mktexlsr|.
%
% \subsection{Some details for the interested}
%
% \paragraph{Attached source.}
%
% The PDF documentation on CTAN also includes the
% \xfile{.dtx} source file. It can be extracted by
% AcrobatReader 6 or higher. Another option is \textsf{pdftk},
% e.g. unpack the file into the current directory:
% \begin{quote}
%   \verb|pdftk scrindex.pdf unpack_files output .|
% \end{quote}
%
% \paragraph{Unpacking with \LaTeX.}
% The \xfile{.dtx} chooses its action depending on the format:
% \begin{description}
% \item[\plainTeX:] Run \docstrip\ and extract the files.
% \item[\LaTeX:] Generate the documentation.
% \end{description}
% If you insist on using \LaTeX\ for \docstrip\ (really,
% \docstrip\ does not need \LaTeX), then inform the autodetect routine
% about your intention:
% \begin{quote}
%   \verb|latex \let\install=y% \iffalse meta-comment
%
% File: scrindex.dtx
% Version: 2008/08/11 v1.1
% Info: Package index with KOMA-Script classes
%
% Copyright (C) 2008 by
%    Heiko Oberdiek <heiko.oberdiek at googlemail.com>
%
% This work may be distributed and/or modified under the
% conditions of the LaTeX Project Public License, either
% version 1.3c of this license or (at your option) any later
% version. This version of this license is in
%    http://www.latex-project.org/lppl/lppl-1-3c.txt
% and the latest version of this license is in
%    http://www.latex-project.org/lppl.txt
% and version 1.3 or later is part of all distributions of
% LaTeX version 2005/12/01 or later.
%
% This work has the LPPL maintenance status "maintained".
%
% This Current Maintainer of this work is Heiko Oberdiek.
%
% This work consists of the main source file scrindex.dtx
% and the derived files
%    scrindex.sty, scrindex.pdf, scrindex.ins, scrindex.drv,
%    scrindex-example1.tex, scrindex-example2.tex.
%
% Distribution:
%    CTAN:macros/latex/contrib/oberdiek/scrindex.dtx
%    CTAN:macros/latex/contrib/oberdiek/scrindex.pdf
%
% Unpacking:
%    (a) If scrindex.ins is present:
%           tex scrindex.ins
%    (b) Without scrindex.ins:
%           tex scrindex.dtx
%    (c) If you insist on using LaTeX
%           latex \let\install=y\input{scrindex.dtx}
%        (quote the arguments according to the demands of your shell)
%
% Documentation:
%    (a) If scrindex.drv is present:
%           latex scrindex.drv
%    (b) Without scrindex.drv:
%           latex scrindex.dtx; ...
%    The class ltxdoc loads the configuration file ltxdoc.cfg
%    if available. Here you can specify further options, e.g.
%    use A4 as paper format:
%       \PassOptionsToClass{a4paper}{article}
%
%    Programm calls to get the documentation (example):
%       pdflatex scrindex.dtx
%       makeindex -s gind.ist scrindex.idx
%       pdflatex scrindex.dtx
%       makeindex -s gind.ist scrindex.idx
%       pdflatex scrindex.dtx
%
% Installation:
%    TDS:tex/latex/oberdiek/scrindex.sty
%    TDS:doc/latex/oberdiek/scrindex.pdf
%    TDS:doc/latex/oberdiek/scrindex-example1.tex
%    TDS:doc/latex/oberdiek/scrindex-example2.tex
%    TDS:source/latex/oberdiek/scrindex.dtx
%
%<*ignore>
\begingroup
  \catcode123=1 %
  \catcode125=2 %
  \def\x{LaTeX2e}%
\expandafter\endgroup
\ifcase 0\ifx\install y1\fi\expandafter
         \ifx\csname processbatchFile\endcsname\relax\else1\fi
         \ifx\fmtname\x\else 1\fi\relax
\else\csname fi\endcsname
%</ignore>
%<*install>
\input docstrip.tex
\Msg{************************************************************************}
\Msg{* Installation}
\Msg{* Package: scrindex 2008/08/11 v1.1 Package index with KOMA-Script classes (HO)}
\Msg{************************************************************************}

\keepsilent
\askforoverwritefalse

\let\MetaPrefix\relax
\preamble

This is a generated file.

Project: scrindex
Version: 2008/08/11 v1.1

Copyright (C) 2008 by
   Heiko Oberdiek <heiko.oberdiek at googlemail.com>

This work may be distributed and/or modified under the
conditions of the LaTeX Project Public License, either
version 1.3c of this license or (at your option) any later
version. This version of this license is in
   http://www.latex-project.org/lppl/lppl-1-3c.txt
and the latest version of this license is in
   http://www.latex-project.org/lppl.txt
and version 1.3 or later is part of all distributions of
LaTeX version 2005/12/01 or later.

This work has the LPPL maintenance status "maintained".

This Current Maintainer of this work is Heiko Oberdiek.

This work consists of the main source file scrindex.dtx
and the derived files
   scrindex.sty, scrindex.pdf, scrindex.ins, scrindex.drv,
   scrindex-example1.tex, scrindex-example2.tex.

\endpreamble
\let\MetaPrefix\DoubleperCent

\generate{%
  \file{scrindex.ins}{\from{scrindex.dtx}{install}}%
  \file{scrindex.drv}{\from{scrindex.dtx}{driver}}%
  \usedir{tex/latex/oberdiek}%
  \file{scrindex.sty}{\from{scrindex.dtx}{package}}%
  \usedir{doc/latex/oberdiek}%
  \file{scrindex-example1.tex}{\from{scrindex.dtx}{example1}}%
  \file{scrindex-example2.tex}{\from{scrindex.dtx}{example2}}%
  \nopreamble
  \nopostamble
  \usedir{source/latex/oberdiek/catalogue}%
  \file{scrindex.xml}{\from{scrindex.dtx}{catalogue}}%
}

\catcode32=13\relax% active space
\let =\space%
\Msg{************************************************************************}
\Msg{*}
\Msg{* To finish the installation you have to move the following}
\Msg{* file into a directory searched by TeX:}
\Msg{*}
\Msg{*     scrindex.sty}
\Msg{*}
\Msg{* To produce the documentation run the file `scrindex.drv'}
\Msg{* through LaTeX.}
\Msg{*}
\Msg{* Happy TeXing!}
\Msg{*}
\Msg{************************************************************************}

\endbatchfile
%</install>
%<*ignore>
\fi
%</ignore>
%<*driver>
\NeedsTeXFormat{LaTeX2e}
\ProvidesFile{scrindex.drv}%
  [2008/08/11 v1.1 Package index with KOMA-Script classes (HO)]%
\documentclass{ltxdoc}
\usepackage{holtxdoc}[2011/11/22]
\usepackage{calc}
\begin{document}
  \DocInput{scrindex.dtx}%
\end{document}
%</driver>
% \fi
%
% \CheckSum{237}
%
% \CharacterTable
%  {Upper-case    \A\B\C\D\E\F\G\H\I\J\K\L\M\N\O\P\Q\R\S\T\U\V\W\X\Y\Z
%   Lower-case    \a\b\c\d\e\f\g\h\i\j\k\l\m\n\o\p\q\r\s\t\u\v\w\x\y\z
%   Digits        \0\1\2\3\4\5\6\7\8\9
%   Exclamation   \!     Double quote  \"     Hash (number) \#
%   Dollar        \$     Percent       \%     Ampersand     \&
%   Acute accent  \'     Left paren    \(     Right paren   \)
%   Asterisk      \*     Plus          \+     Comma         \,
%   Minus         \-     Point         \.     Solidus       \/
%   Colon         \:     Semicolon     \;     Less than     \<
%   Equals        \=     Greater than  \>     Question mark \?
%   Commercial at \@     Left bracket  \[     Backslash     \\
%   Right bracket \]     Circumflex    \^     Underscore    \_
%   Grave accent  \`     Left brace    \{     Vertical bar  \|
%   Right brace   \}     Tilde         \~}
%
% \GetFileInfo{scrindex.drv}
%
% \title{The \xpackage{scrindex} package}
% \date{2008/08/11 v1.1}
% \author{Heiko Oberdiek\\\xemail{heiko.oberdiek at googlemail.com}}
%
% \maketitle
%
% \begin{abstract}
% This package redefines environment `theindex' of package \xpackage{index},
% if a class from KOMA-Script is loaded. Also option \xoption{idxtotoc}
% is supported. Index preambles can be given either by means of package
% \xpackage{index} or via the interface provided by KOMA-Script.
% \end{abstract}
%
% \tableofcontents
%
% \section{Documentation}
%
% Package \xpackage{index}, written by David M.\ Jones, detects
% the standard classes |article|, |report|, and |book|. It
% redefines environment `theindex' for its needs.
% However, it does not know other classes such as KOMA-Script.
% This package closes the compatibiliy gap between KOMA-Script's
% classes and package \xpackage{index}.
%
% Environment |theindex| is redefined to support both package
% \xpackage{index} and KOMA-Script's classes. Thus both
% the prologe of package \xpackage{index} and the preamble
% of KOMA-Script's classes are available. Also class option |idxtotoc|
% of KOMA-Script is supported.
%
% \subsection{Usage}
%
% The package \xpackage{scrindex} is loaded without options:
%\begin{quote}
%\begin{verbatim}
%\usepackage{scrindex}
%\end{verbatim}
%\end{quote}
%
% It loads package \xpackage{index} and requests version 2004/01/20
% or later. \LaTeX's package interface allows multiple calls
% of the same package. The package is loaded at its first
% package loading command. At later times \LaTeX\ only checks
% options and a requested version date. Therefore it does not harm
% to add |\usepackage{index}| before or after |\usepackage{scrindex}|.
%
% Also the class does not matter. Environment |theindex| is only
% redefined for a supported class:
% \begin{itemize}
% \item |scrartcl|
% \item |scrreprt|
% \item |scrbook|
% \end{itemize}
%
% \subsection{Preambles}
%
% Both the prologue of package \xpackage{index} and the preamble
% of KOMA-Script's classes are supported. The position depends on
% the class.
%
% \subsubsection{Class \xclass{scrartcl}}
%
%    \begin{macrocode}
%<*example1>
\documentclass{scrartcl}
\usepackage{scrindex}
\setindexpreamble{Preamble of \texttt{scrartcl}\dotfill EOL}
\makeindex
\begin{document}
\section{First Section}
\index{first}
\index{section}
\printindex[default]%
  [Prologue of package \texttt{index}\dotfill EOL]%
\end{document}
%</example1>
%    \end{macrocode}
% The prologue of package \xpackage{index} is first set straight
% after the section title spanning both columns.
% Then the preamble of KOMA-Script follows
% in the first left column.
%
% \medskip
% \begin{quote}
%   \renewcommand*{\arraystretch}{1.2}
%   \begin{tabular}{|p{.45\linewidth}|p{.45\linewidth}|}
%   \hline
%   \multicolumn{2}{|l|}{\textbf{Index}}\\[1ex]
%   \multicolumn{2}{|p{.9\linewidth+2\tabcolsep}|}{^^A
%     Prologue of package \texttt{index}\dotfill EOL^^A
%   }\\[1ex]
%   \hline
%   Preamble of \texttt{scrartcl}\dotfill EOL&\\
%   first, 1&\\
%   section, 1&\\
%   \hline
%   \end{tabular}
% \end{quote}
%
% \subsubsection{Classes \xclass{scrreprt} and \xclass{scrbook}}
%
%    \begin{macrocode}
%<*example2>
\documentclass[openany]{scrbook}% or scrreprt
\usepackage{scrindex}
\setindexpreamble{Preamble of class \texttt{scrbook}\dotfill EOL}
\makeindex
\begin{document}
\chapter{First Chapter}
\index{first}
\index{chapter}
\printindex[default]%
  [Prologue of package \texttt{index}\dotfill EOL]%
\end{document}
%</example2>
%    \end{macrocode}
% The order of the two preambles are different for the classes
% \xclass{scrreprt} and \xclass{scrbook}. First KOMA-Script's
% chapter preamble is set, then the prologue of package \xpackage{index}
% follows. Both are set spanning both columns.
%
% \medskip
% \begin{quote}
%   \renewcommand*{\arraystretch}{1.2}
%   \begin{tabular}{|p{.45\linewidth}|p{.45\linewidth}|}
%   \hline
%   \multicolumn{2}{|l|}{\textbf{Index}}\\[1ex]
%   \multicolumn{2}{|p{.9\linewidth+2\tabcolsep}|}{^^A
%     Preamble of class \texttt{scrbook}\dotfill EOL^^A
%   }\\
%   \multicolumn{2}{|p{.9\linewidth+2\tabcolsep}|}{^^A
%     Prologue of package \xpackage{index}\dotfill EOL^^A
%   }\\[1ex]
%   \hline
%   chapter, 1&\\
%   first, 1&\\
%   \hline
%   \end{tabular}
% \end{quote}
%
% \StopEventually{
% }
%
% \section{Implementation}
%
%    \begin{macrocode}
%<*package>
\NeedsTeXFormat{LaTeX2e}
\ProvidesPackage{scrindex}
  [2008/08/11 v1.1 Package index with KOMA-Script classes (HO)]%
%    \end{macrocode}
%
%    \begin{macrocode}
\RequirePackage{index}[2004/01/20]%
%    \end{macrocode}
%
%    \begin{macrocode}
\@ifclassloaded{scrartcl}{%
  \renewenvironment{theindex}{%
    \edef\indexname{%
      \the\@nameuse{idxtitle@\@indextype}%
    }%
    \if@twocolumn
      \@restonecolfalse
    \else
      \@restonecoltrue
    \fi
    \idx@heading
    \thispagestyle{\indexpagestyle}%
    \columnseprule\z@
    \columnsep 35\p@
    \index@preamble\par\nobreak
    \parindent\z@
    \parskip\z@ \@plus .3\p@\relax
    \parfillskip\z@ \@plus 1fil\relax
    \let\item\@idxitem
  }{%
    \if@restonecol
      \onecolumn
    \else
      \clearpage
    \fi
  }%
  \@ifclasswith{scrartcl}{idxtotoc}{%
    \renewcommand*{\idx@heading}{%
      \twocolumn[%
        \addsec{\indexname}%
        \ifx\index@prologue\@empty
        \else
          \index@prologue
          \bigskip
        \fi
      ]%
      \@mkboth{\indexname}{\indexname}%
    }%
  }{%
    \renewcommand*{\idx@heading}{%
      \twocolumn[%
        \section*{\indexname}%
        \ifx\index@prologue\@empty
        \else
          \index@prologue
          \bigskip
        \fi
      ]%
      \@mkboth{\indexname}{\indexname}%
    }%
  }%
}{}
%    \end{macrocode}
%    \begin{macrocode}
\@ifclassloaded{scrreprt}{%
  \renewenvironment{theindex}{%
    \edef\indexname{%
      \the\@nameuse{idxtitle@\@indextype}%
    }%
    \if@twocolumn
      \@restonecolfalse
    \else
      \@restonecoltrue
    \fi
    \setchapterpreamble{\index@preamble}%
    \idx@heading
    \thispagestyle{\indexpagestyle}%
    \columnseprule\z@
    \columnsep 35\p@
    \parindent\z@
    \parskip\z@ \@plus .3\p@\relax
    \parfillskip\z@ \@plus 1fil\relax
    \let\item\@idxitem
  }{%
    \if@restonecol
      \onecolumn
    \else
      \clearpage
    \fi
  }%
  \@ifclasswith{scrreprt}{idxtotoc}{%
    \renewcommand*{\idx@heading}{%
      \if@openright
        \cleardoublepage
      \else
        \clearpage
      \fi
      \twocolumn[%
        \addchap{\indexname}%
        \ifx\index@prologue\@empty
        \else
          \index@prologue
          \bigskip
        \fi
      ]%
      \@mkboth{\indexname}{\indexname}%
    }%
  }{%
    \renewcommand*{\idx@heading}{%
      \if@openright
        \cleardoublepage
      \else
        \clearpage
      \fi
      \twocolumn[%
        \chapter*{\indexname}%
        \ifx\index@prologue\@empty
        \else
          \index@prologue
          \bigskip
        \fi
      ]%
      \@mkboth{\indexname}{\indexname}%
    }%
  }%
}{}
%    \end{macrocode}
%    \begin{macrocode}
\@ifclassloaded{scrbook}{%
  \renewenvironment{theindex}{%
    \edef\indexname{%
      \the\@nameuse{idxtitle@\@indextype}%
    }%
    \if@twocolumn
      \@restonecolfalse
    \else
      \@restonecoltrue
    \fi
    \setchapterpreamble{\index@preamble}%
    \idx@heading
    \thispagestyle{\indexpagestyle}%
    \columnseprule\z@
    \columnsep 35\p@
    \parindent\z@
    \parskip\z@ \@plus .3\p@\relax
    \parfillskip\z@ \@plus 1fil\relax
    \let\item\@idxitem
  }{%
    \if@restonecol
      \onecolumn
    \else
      \clearpage
    \fi
  }%
  \@ifclasswith{scrbook}{idxtotoc}{%
    \renewcommand*{\idx@heading}{%
      \if@openright
        \cleardoublepage
      \else
        \clearpage
      \fi
      \twocolumn[%
        \addchap{\indexname}%
        \ifx\index@prologue\@empty
        \else
          \index@prologue
          \bigskip
        \fi
      ]%
      \@mkboth{\indexname}{\indexname}%
    }%
  }{%
    \renewcommand*{\idx@heading}{%
      \if@openright
        \cleardoublepage
      \else
        \clearpage
      \fi
      \twocolumn[%
        \chapter*{\indexname}%
        \ifx\index@prologue\@empty
        \else
          \index@prologue
          \bigskip
        \fi
      ]%
      \@mkboth{\indexname}{\indexname}%
    }%
  }%
}{}
%    \end{macrocode}
%
%    \begin{macrocode}
%</package>
%    \end{macrocode}
%
% \section{Installation}
%
% \subsection{Download}
%
% \paragraph{Package.} This package is available on
% CTAN\footnote{\url{ftp://ftp.ctan.org/tex-archive/}}:
% \begin{description}
% \item[\CTAN{macros/latex/contrib/oberdiek/scrindex.dtx}] The source file.
% \item[\CTAN{macros/latex/contrib/oberdiek/scrindex.pdf}] Documentation.
% \end{description}
%
%
% \paragraph{Bundle.} All the packages of the bundle `oberdiek'
% are also available in a TDS compliant ZIP archive. There
% the packages are already unpacked and the documentation files
% are generated. The files and directories obey the TDS standard.
% \begin{description}
% \item[\CTAN{install/macros/latex/contrib/oberdiek.tds.zip}]
% \end{description}
% \emph{TDS} refers to the standard ``A Directory Structure
% for \TeX\ Files'' (\CTAN{tds/tds.pdf}). Directories
% with \xfile{texmf} in their name are usually organized this way.
%
% \subsection{Bundle installation}
%
% \paragraph{Unpacking.} Unpack the \xfile{oberdiek.tds.zip} in the
% TDS tree (also known as \xfile{texmf} tree) of your choice.
% Example (linux):
% \begin{quote}
%   |unzip oberdiek.tds.zip -d ~/texmf|
% \end{quote}
%
% \paragraph{Script installation.}
% Check the directory \xfile{TDS:scripts/oberdiek/} for
% scripts that need further installation steps.
% Package \xpackage{attachfile2} comes with the Perl script
% \xfile{pdfatfi.pl} that should be installed in such a way
% that it can be called as \texttt{pdfatfi}.
% Example (linux):
% \begin{quote}
%   |chmod +x scripts/oberdiek/pdfatfi.pl|\\
%   |cp scripts/oberdiek/pdfatfi.pl /usr/local/bin/|
% \end{quote}
%
% \subsection{Package installation}
%
% \paragraph{Unpacking.} The \xfile{.dtx} file is a self-extracting
% \docstrip\ archive. The files are extracted by running the
% \xfile{.dtx} through \plainTeX:
% \begin{quote}
%   \verb|tex scrindex.dtx|
% \end{quote}
%
% \paragraph{TDS.} Now the different files must be moved into
% the different directories in your installation TDS tree
% (also known as \xfile{texmf} tree):
% \begin{quote}
% \def\t{^^A
% \begin{tabular}{@{}>{\ttfamily}l@{ $\rightarrow$ }>{\ttfamily}l@{}}
%   scrindex.sty & tex/latex/oberdiek/scrindex.sty\\
%   scrindex.pdf & doc/latex/oberdiek/scrindex.pdf\\
%   scrindex-example1.tex & doc/latex/oberdiek/scrindex-example1.tex\\
%   scrindex-example2.tex & doc/latex/oberdiek/scrindex-example2.tex\\
%   scrindex.dtx & source/latex/oberdiek/scrindex.dtx\\
% \end{tabular}^^A
% }^^A
% \sbox0{\t}^^A
% \ifdim\wd0>\linewidth
%   \begingroup
%     \advance\linewidth by\leftmargin
%     \advance\linewidth by\rightmargin
%   \edef\x{\endgroup
%     \def\noexpand\lw{\the\linewidth}^^A
%   }\x
%   \def\lwbox{^^A
%     \leavevmode
%     \hbox to \linewidth{^^A
%       \kern-\leftmargin\relax
%       \hss
%       \usebox0
%       \hss
%       \kern-\rightmargin\relax
%     }^^A
%   }^^A
%   \ifdim\wd0>\lw
%     \sbox0{\small\t}^^A
%     \ifdim\wd0>\linewidth
%       \ifdim\wd0>\lw
%         \sbox0{\footnotesize\t}^^A
%         \ifdim\wd0>\linewidth
%           \ifdim\wd0>\lw
%             \sbox0{\scriptsize\t}^^A
%             \ifdim\wd0>\linewidth
%               \ifdim\wd0>\lw
%                 \sbox0{\tiny\t}^^A
%                 \ifdim\wd0>\linewidth
%                   \lwbox
%                 \else
%                   \usebox0
%                 \fi
%               \else
%                 \lwbox
%               \fi
%             \else
%               \usebox0
%             \fi
%           \else
%             \lwbox
%           \fi
%         \else
%           \usebox0
%         \fi
%       \else
%         \lwbox
%       \fi
%     \else
%       \usebox0
%     \fi
%   \else
%     \lwbox
%   \fi
% \else
%   \usebox0
% \fi
% \end{quote}
% If you have a \xfile{docstrip.cfg} that configures and enables \docstrip's
% TDS installing feature, then some files can already be in the right
% place, see the documentation of \docstrip.
%
% \subsection{Refresh file name databases}
%
% If your \TeX~distribution
% (\teTeX, \mikTeX, \dots) relies on file name databases, you must refresh
% these. For example, \teTeX\ users run \verb|texhash| or
% \verb|mktexlsr|.
%
% \subsection{Some details for the interested}
%
% \paragraph{Attached source.}
%
% The PDF documentation on CTAN also includes the
% \xfile{.dtx} source file. It can be extracted by
% AcrobatReader 6 or higher. Another option is \textsf{pdftk},
% e.g. unpack the file into the current directory:
% \begin{quote}
%   \verb|pdftk scrindex.pdf unpack_files output .|
% \end{quote}
%
% \paragraph{Unpacking with \LaTeX.}
% The \xfile{.dtx} chooses its action depending on the format:
% \begin{description}
% \item[\plainTeX:] Run \docstrip\ and extract the files.
% \item[\LaTeX:] Generate the documentation.
% \end{description}
% If you insist on using \LaTeX\ for \docstrip\ (really,
% \docstrip\ does not need \LaTeX), then inform the autodetect routine
% about your intention:
% \begin{quote}
%   \verb|latex \let\install=y\input{scrindex.dtx}|
% \end{quote}
% Do not forget to quote the argument according to the demands
% of your shell.
%
% \paragraph{Generating the documentation.}
% You can use both the \xfile{.dtx} or the \xfile{.drv} to generate
% the documentation. The process can be configured by the
% configuration file \xfile{ltxdoc.cfg}. For instance, put this
% line into this file, if you want to have A4 as paper format:
% \begin{quote}
%   \verb|\PassOptionsToClass{a4paper}{article}|
% \end{quote}
% An example follows how to generate the
% documentation with pdf\LaTeX:
% \begin{quote}
%\begin{verbatim}
%pdflatex scrindex.dtx
%makeindex -s gind.ist scrindex.idx
%pdflatex scrindex.dtx
%makeindex -s gind.ist scrindex.idx
%pdflatex scrindex.dtx
%\end{verbatim}
% \end{quote}
%
% \section{Catalogue}
%
% The following XML file can be used as source for the
% \href{http://mirror.ctan.org/help/Catalogue/catalogue.html}{\TeX\ Catalogue}.
% The elements \texttt{caption} and \texttt{description} are imported
% from the original XML file from the Catalogue.
% The name of the XML file in the Catalogue is \xfile{scrindex.xml}.
%    \begin{macrocode}
%<*catalogue>
<?xml version='1.0' encoding='us-ascii'?>
<!DOCTYPE entry SYSTEM 'catalogue.dtd'>
<entry datestamp='$Date$' modifier='$Author$' id='scrindex'>
  <name>scrindex</name>
  <caption>Make index package work with Koma-script classes.</caption>
  <authorref id='auth:oberdiek'/>
  <copyright owner='Heiko Oberdiek' year='2008'/>
  <license type='lppl1.3'/>
  <version number='1.1'/>
  <description>
    This package redefines environment `theindex' of package `index',
    if a class from <xref refid='koma-script'>KOMA-Script</xref> is loaded.
    Also option `idxtotoc' is supported. Index preambles can be given
    either by means of package `index' or via the interface provided
    by <xref refid='koma-script'>KOMA-Script</xref>.
    <p/>
    The package is part of the <xref refid='oberdiek'>oberdiek</xref>
    bundle.
  </description>
  <documentation details='Package documentation'
      href='ctan:/macros/latex/contrib/oberdiek/scrindex.pdf'/>
  <ctan file='true' path='/macros/latex/contrib/oberdiek/scrindex.dtx'/>
  <miktex location='oberdiek'/>
  <texlive location='oberdiek'/>
  <install path='/macros/latex/contrib/oberdiek/oberdiek.tds.zip'/>
</entry>
%</catalogue>
%    \end{macrocode}
%
% \begin{History}
%   \begin{Version}{2008/07/07 v1.0}
%   \item
%     First version, also published in newsgroup \xnewsgroup{de.comp.text.tex}:\\
%     \URL{``\link{Re: Z\"ahler bei \cs{index}}''}^^A
%     {http://groups.google.com/group/de.comp.text.tex/msg/39575b5e2f29be1e}
%   \end{Version}
%   \begin{Version}{2008/08/11 v1.1}
%   \item
%     Code is not changed.
%   \item
%     URLs updated.
%   \end{Version}
% \end{History}
%
% \PrintIndex
%
% \Finale
\endinput
|
% \end{quote}
% Do not forget to quote the argument according to the demands
% of your shell.
%
% \paragraph{Generating the documentation.}
% You can use both the \xfile{.dtx} or the \xfile{.drv} to generate
% the documentation. The process can be configured by the
% configuration file \xfile{ltxdoc.cfg}. For instance, put this
% line into this file, if you want to have A4 as paper format:
% \begin{quote}
%   \verb|\PassOptionsToClass{a4paper}{article}|
% \end{quote}
% An example follows how to generate the
% documentation with pdf\LaTeX:
% \begin{quote}
%\begin{verbatim}
%pdflatex scrindex.dtx
%makeindex -s gind.ist scrindex.idx
%pdflatex scrindex.dtx
%makeindex -s gind.ist scrindex.idx
%pdflatex scrindex.dtx
%\end{verbatim}
% \end{quote}
%
% \section{Catalogue}
%
% The following XML file can be used as source for the
% \href{http://mirror.ctan.org/help/Catalogue/catalogue.html}{\TeX\ Catalogue}.
% The elements \texttt{caption} and \texttt{description} are imported
% from the original XML file from the Catalogue.
% The name of the XML file in the Catalogue is \xfile{scrindex.xml}.
%    \begin{macrocode}
%<*catalogue>
<?xml version='1.0' encoding='us-ascii'?>
<!DOCTYPE entry SYSTEM 'catalogue.dtd'>
<entry datestamp='$Date$' modifier='$Author$' id='scrindex'>
  <name>scrindex</name>
  <caption>Make index package work with Koma-script classes.</caption>
  <authorref id='auth:oberdiek'/>
  <copyright owner='Heiko Oberdiek' year='2008'/>
  <license type='lppl1.3'/>
  <version number='1.1'/>
  <description>
    This package redefines environment `theindex' of package `index',
    if a class from <xref refid='koma-script'>KOMA-Script</xref> is loaded.
    Also option `idxtotoc' is supported. Index preambles can be given
    either by means of package `index' or via the interface provided
    by <xref refid='koma-script'>KOMA-Script</xref>.
    <p/>
    The package is part of the <xref refid='oberdiek'>oberdiek</xref>
    bundle.
  </description>
  <documentation details='Package documentation'
      href='ctan:/macros/latex/contrib/oberdiek/scrindex.pdf'/>
  <ctan file='true' path='/macros/latex/contrib/oberdiek/scrindex.dtx'/>
  <miktex location='oberdiek'/>
  <texlive location='oberdiek'/>
  <install path='/macros/latex/contrib/oberdiek/oberdiek.tds.zip'/>
</entry>
%</catalogue>
%    \end{macrocode}
%
% \begin{History}
%   \begin{Version}{2008/07/07 v1.0}
%   \item
%     First version, also published in newsgroup \xnewsgroup{de.comp.text.tex}:\\
%     \URL{``\link{Re: Z\"ahler bei \cs{index}}''}^^A
%     {http://groups.google.com/group/de.comp.text.tex/msg/39575b5e2f29be1e}
%   \end{Version}
%   \begin{Version}{2008/08/11 v1.1}
%   \item
%     Code is not changed.
%   \item
%     URLs updated.
%   \end{Version}
% \end{History}
%
% \PrintIndex
%
% \Finale
\endinput
|
% \end{quote}
% Do not forget to quote the argument according to the demands
% of your shell.
%
% \paragraph{Generating the documentation.}
% You can use both the \xfile{.dtx} or the \xfile{.drv} to generate
% the documentation. The process can be configured by the
% configuration file \xfile{ltxdoc.cfg}. For instance, put this
% line into this file, if you want to have A4 as paper format:
% \begin{quote}
%   \verb|\PassOptionsToClass{a4paper}{article}|
% \end{quote}
% An example follows how to generate the
% documentation with pdf\LaTeX:
% \begin{quote}
%\begin{verbatim}
%pdflatex scrindex.dtx
%makeindex -s gind.ist scrindex.idx
%pdflatex scrindex.dtx
%makeindex -s gind.ist scrindex.idx
%pdflatex scrindex.dtx
%\end{verbatim}
% \end{quote}
%
% \section{Catalogue}
%
% The following XML file can be used as source for the
% \href{http://mirror.ctan.org/help/Catalogue/catalogue.html}{\TeX\ Catalogue}.
% The elements \texttt{caption} and \texttt{description} are imported
% from the original XML file from the Catalogue.
% The name of the XML file in the Catalogue is \xfile{scrindex.xml}.
%    \begin{macrocode}
%<*catalogue>
<?xml version='1.0' encoding='us-ascii'?>
<!DOCTYPE entry SYSTEM 'catalogue.dtd'>
<entry datestamp='$Date$' modifier='$Author$' id='scrindex'>
  <name>scrindex</name>
  <caption>Make index package work with Koma-script classes.</caption>
  <authorref id='auth:oberdiek'/>
  <copyright owner='Heiko Oberdiek' year='2008'/>
  <license type='lppl1.3'/>
  <version number='1.1'/>
  <description>
    This package redefines environment `theindex' of package `index',
    if a class from <xref refid='koma-script'>KOMA-Script</xref> is loaded.
    Also option `idxtotoc' is supported. Index preambles can be given
    either by means of package `index' or via the interface provided
    by <xref refid='koma-script'>KOMA-Script</xref>.
    <p/>
    The package is part of the <xref refid='oberdiek'>oberdiek</xref>
    bundle.
  </description>
  <documentation details='Package documentation'
      href='ctan:/macros/latex/contrib/oberdiek/scrindex.pdf'/>
  <ctan file='true' path='/macros/latex/contrib/oberdiek/scrindex.dtx'/>
  <miktex location='oberdiek'/>
  <texlive location='oberdiek'/>
  <install path='/macros/latex/contrib/oberdiek/oberdiek.tds.zip'/>
</entry>
%</catalogue>
%    \end{macrocode}
%
% \begin{History}
%   \begin{Version}{2008/07/07 v1.0}
%   \item
%     First version, also published in newsgroup \xnewsgroup{de.comp.text.tex}:\\
%     \URL{``\link{Re: Z\"ahler bei \cs{index}}''}^^A
%     {http://groups.google.com/group/de.comp.text.tex/msg/39575b5e2f29be1e}
%   \end{Version}
%   \begin{Version}{2008/08/11 v1.1}
%   \item
%     Code is not changed.
%   \item
%     URLs updated.
%   \end{Version}
% \end{History}
%
% \PrintIndex
%
% \Finale
\endinput

%        (quote the arguments according to the demands of your shell)
%
% Documentation:
%    (a) If scrindex.drv is present:
%           latex scrindex.drv
%    (b) Without scrindex.drv:
%           latex scrindex.dtx; ...
%    The class ltxdoc loads the configuration file ltxdoc.cfg
%    if available. Here you can specify further options, e.g.
%    use A4 as paper format:
%       \PassOptionsToClass{a4paper}{article}
%
%    Programm calls to get the documentation (example):
%       pdflatex scrindex.dtx
%       makeindex -s gind.ist scrindex.idx
%       pdflatex scrindex.dtx
%       makeindex -s gind.ist scrindex.idx
%       pdflatex scrindex.dtx
%
% Installation:
%    TDS:tex/latex/oberdiek/scrindex.sty
%    TDS:doc/latex/oberdiek/scrindex.pdf
%    TDS:doc/latex/oberdiek/scrindex-example1.tex
%    TDS:doc/latex/oberdiek/scrindex-example2.tex
%    TDS:source/latex/oberdiek/scrindex.dtx
%
%<*ignore>
\begingroup
  \catcode123=1 %
  \catcode125=2 %
  \def\x{LaTeX2e}%
\expandafter\endgroup
\ifcase 0\ifx\install y1\fi\expandafter
         \ifx\csname processbatchFile\endcsname\relax\else1\fi
         \ifx\fmtname\x\else 1\fi\relax
\else\csname fi\endcsname
%</ignore>
%<*install>
\input docstrip.tex
\Msg{************************************************************************}
\Msg{* Installation}
\Msg{* Package: scrindex 2008/08/11 v1.1 Package index with KOMA-Script classes (HO)}
\Msg{************************************************************************}

\keepsilent
\askforoverwritefalse

\let\MetaPrefix\relax
\preamble

This is a generated file.

Project: scrindex
Version: 2008/08/11 v1.1

Copyright (C) 2008 by
   Heiko Oberdiek <heiko.oberdiek at googlemail.com>

This work may be distributed and/or modified under the
conditions of the LaTeX Project Public License, either
version 1.3c of this license or (at your option) any later
version. This version of this license is in
   http://www.latex-project.org/lppl/lppl-1-3c.txt
and the latest version of this license is in
   http://www.latex-project.org/lppl.txt
and version 1.3 or later is part of all distributions of
LaTeX version 2005/12/01 or later.

This work has the LPPL maintenance status "maintained".

This Current Maintainer of this work is Heiko Oberdiek.

This work consists of the main source file scrindex.dtx
and the derived files
   scrindex.sty, scrindex.pdf, scrindex.ins, scrindex.drv,
   scrindex-example1.tex, scrindex-example2.tex.

\endpreamble
\let\MetaPrefix\DoubleperCent

\generate{%
  \file{scrindex.ins}{\from{scrindex.dtx}{install}}%
  \file{scrindex.drv}{\from{scrindex.dtx}{driver}}%
  \usedir{tex/latex/oberdiek}%
  \file{scrindex.sty}{\from{scrindex.dtx}{package}}%
  \usedir{doc/latex/oberdiek}%
  \file{scrindex-example1.tex}{\from{scrindex.dtx}{example1}}%
  \file{scrindex-example2.tex}{\from{scrindex.dtx}{example2}}%
  \nopreamble
  \nopostamble
  \usedir{source/latex/oberdiek/catalogue}%
  \file{scrindex.xml}{\from{scrindex.dtx}{catalogue}}%
}

\catcode32=13\relax% active space
\let =\space%
\Msg{************************************************************************}
\Msg{*}
\Msg{* To finish the installation you have to move the following}
\Msg{* file into a directory searched by TeX:}
\Msg{*}
\Msg{*     scrindex.sty}
\Msg{*}
\Msg{* To produce the documentation run the file `scrindex.drv'}
\Msg{* through LaTeX.}
\Msg{*}
\Msg{* Happy TeXing!}
\Msg{*}
\Msg{************************************************************************}

\endbatchfile
%</install>
%<*ignore>
\fi
%</ignore>
%<*driver>
\NeedsTeXFormat{LaTeX2e}
\ProvidesFile{scrindex.drv}%
  [2008/08/11 v1.1 Package index with KOMA-Script classes (HO)]%
\documentclass{ltxdoc}
\usepackage{holtxdoc}[2011/11/22]
\usepackage{calc}
\begin{document}
  \DocInput{scrindex.dtx}%
\end{document}
%</driver>
% \fi
%
% \CheckSum{237}
%
% \CharacterTable
%  {Upper-case    \A\B\C\D\E\F\G\H\I\J\K\L\M\N\O\P\Q\R\S\T\U\V\W\X\Y\Z
%   Lower-case    \a\b\c\d\e\f\g\h\i\j\k\l\m\n\o\p\q\r\s\t\u\v\w\x\y\z
%   Digits        \0\1\2\3\4\5\6\7\8\9
%   Exclamation   \!     Double quote  \"     Hash (number) \#
%   Dollar        \$     Percent       \%     Ampersand     \&
%   Acute accent  \'     Left paren    \(     Right paren   \)
%   Asterisk      \*     Plus          \+     Comma         \,
%   Minus         \-     Point         \.     Solidus       \/
%   Colon         \:     Semicolon     \;     Less than     \<
%   Equals        \=     Greater than  \>     Question mark \?
%   Commercial at \@     Left bracket  \[     Backslash     \\
%   Right bracket \]     Circumflex    \^     Underscore    \_
%   Grave accent  \`     Left brace    \{     Vertical bar  \|
%   Right brace   \}     Tilde         \~}
%
% \GetFileInfo{scrindex.drv}
%
% \title{The \xpackage{scrindex} package}
% \date{2008/08/11 v1.1}
% \author{Heiko Oberdiek\\\xemail{heiko.oberdiek at googlemail.com}}
%
% \maketitle
%
% \begin{abstract}
% This package redefines environment `theindex' of package \xpackage{index},
% if a class from KOMA-Script is loaded. Also option \xoption{idxtotoc}
% is supported. Index preambles can be given either by means of package
% \xpackage{index} or via the interface provided by KOMA-Script.
% \end{abstract}
%
% \tableofcontents
%
% \section{Documentation}
%
% Package \xpackage{index}, written by David M.\ Jones, detects
% the standard classes |article|, |report|, and |book|. It
% redefines environment `theindex' for its needs.
% However, it does not know other classes such as KOMA-Script.
% This package closes the compatibiliy gap between KOMA-Script's
% classes and package \xpackage{index}.
%
% Environment |theindex| is redefined to support both package
% \xpackage{index} and KOMA-Script's classes. Thus both
% the prologe of package \xpackage{index} and the preamble
% of KOMA-Script's classes are available. Also class option |idxtotoc|
% of KOMA-Script is supported.
%
% \subsection{Usage}
%
% The package \xpackage{scrindex} is loaded without options:
%\begin{quote}
%\begin{verbatim}
%\usepackage{scrindex}
%\end{verbatim}
%\end{quote}
%
% It loads package \xpackage{index} and requests version 2004/01/20
% or later. \LaTeX's package interface allows multiple calls
% of the same package. The package is loaded at its first
% package loading command. At later times \LaTeX\ only checks
% options and a requested version date. Therefore it does not harm
% to add |\usepackage{index}| before or after |\usepackage{scrindex}|.
%
% Also the class does not matter. Environment |theindex| is only
% redefined for a supported class:
% \begin{itemize}
% \item |scrartcl|
% \item |scrreprt|
% \item |scrbook|
% \end{itemize}
%
% \subsection{Preambles}
%
% Both the prologue of package \xpackage{index} and the preamble
% of KOMA-Script's classes are supported. The position depends on
% the class.
%
% \subsubsection{Class \xclass{scrartcl}}
%
%    \begin{macrocode}
%<*example1>
\documentclass{scrartcl}
\usepackage{scrindex}
\setindexpreamble{Preamble of \texttt{scrartcl}\dotfill EOL}
\makeindex
\begin{document}
\section{First Section}
\index{first}
\index{section}
\printindex[default]%
  [Prologue of package \texttt{index}\dotfill EOL]%
\end{document}
%</example1>
%    \end{macrocode}
% The prologue of package \xpackage{index} is first set straight
% after the section title spanning both columns.
% Then the preamble of KOMA-Script follows
% in the first left column.
%
% \medskip
% \begin{quote}
%   \renewcommand*{\arraystretch}{1.2}
%   \begin{tabular}{|p{.45\linewidth}|p{.45\linewidth}|}
%   \hline
%   \multicolumn{2}{|l|}{\textbf{Index}}\\[1ex]
%   \multicolumn{2}{|p{.9\linewidth+2\tabcolsep}|}{^^A
%     Prologue of package \texttt{index}\dotfill EOL^^A
%   }\\[1ex]
%   \hline
%   Preamble of \texttt{scrartcl}\dotfill EOL&\\
%   first, 1&\\
%   section, 1&\\
%   \hline
%   \end{tabular}
% \end{quote}
%
% \subsubsection{Classes \xclass{scrreprt} and \xclass{scrbook}}
%
%    \begin{macrocode}
%<*example2>
\documentclass[openany]{scrbook}% or scrreprt
\usepackage{scrindex}
\setindexpreamble{Preamble of class \texttt{scrbook}\dotfill EOL}
\makeindex
\begin{document}
\chapter{First Chapter}
\index{first}
\index{chapter}
\printindex[default]%
  [Prologue of package \texttt{index}\dotfill EOL]%
\end{document}
%</example2>
%    \end{macrocode}
% The order of the two preambles are different for the classes
% \xclass{scrreprt} and \xclass{scrbook}. First KOMA-Script's
% chapter preamble is set, then the prologue of package \xpackage{index}
% follows. Both are set spanning both columns.
%
% \medskip
% \begin{quote}
%   \renewcommand*{\arraystretch}{1.2}
%   \begin{tabular}{|p{.45\linewidth}|p{.45\linewidth}|}
%   \hline
%   \multicolumn{2}{|l|}{\textbf{Index}}\\[1ex]
%   \multicolumn{2}{|p{.9\linewidth+2\tabcolsep}|}{^^A
%     Preamble of class \texttt{scrbook}\dotfill EOL^^A
%   }\\
%   \multicolumn{2}{|p{.9\linewidth+2\tabcolsep}|}{^^A
%     Prologue of package \xpackage{index}\dotfill EOL^^A
%   }\\[1ex]
%   \hline
%   chapter, 1&\\
%   first, 1&\\
%   \hline
%   \end{tabular}
% \end{quote}
%
% \StopEventually{
% }
%
% \section{Implementation}
%
%    \begin{macrocode}
%<*package>
\NeedsTeXFormat{LaTeX2e}
\ProvidesPackage{scrindex}
  [2008/08/11 v1.1 Package index with KOMA-Script classes (HO)]%
%    \end{macrocode}
%
%    \begin{macrocode}
\RequirePackage{index}[2004/01/20]%
%    \end{macrocode}
%
%    \begin{macrocode}
\@ifclassloaded{scrartcl}{%
  \renewenvironment{theindex}{%
    \edef\indexname{%
      \the\@nameuse{idxtitle@\@indextype}%
    }%
    \if@twocolumn
      \@restonecolfalse
    \else
      \@restonecoltrue
    \fi
    \idx@heading
    \thispagestyle{\indexpagestyle}%
    \columnseprule\z@
    \columnsep 35\p@
    \index@preamble\par\nobreak
    \parindent\z@
    \parskip\z@ \@plus .3\p@\relax
    \parfillskip\z@ \@plus 1fil\relax
    \let\item\@idxitem
  }{%
    \if@restonecol
      \onecolumn
    \else
      \clearpage
    \fi
  }%
  \@ifclasswith{scrartcl}{idxtotoc}{%
    \renewcommand*{\idx@heading}{%
      \twocolumn[%
        \addsec{\indexname}%
        \ifx\index@prologue\@empty
        \else
          \index@prologue
          \bigskip
        \fi
      ]%
      \@mkboth{\indexname}{\indexname}%
    }%
  }{%
    \renewcommand*{\idx@heading}{%
      \twocolumn[%
        \section*{\indexname}%
        \ifx\index@prologue\@empty
        \else
          \index@prologue
          \bigskip
        \fi
      ]%
      \@mkboth{\indexname}{\indexname}%
    }%
  }%
}{}
%    \end{macrocode}
%    \begin{macrocode}
\@ifclassloaded{scrreprt}{%
  \renewenvironment{theindex}{%
    \edef\indexname{%
      \the\@nameuse{idxtitle@\@indextype}%
    }%
    \if@twocolumn
      \@restonecolfalse
    \else
      \@restonecoltrue
    \fi
    \setchapterpreamble{\index@preamble}%
    \idx@heading
    \thispagestyle{\indexpagestyle}%
    \columnseprule\z@
    \columnsep 35\p@
    \parindent\z@
    \parskip\z@ \@plus .3\p@\relax
    \parfillskip\z@ \@plus 1fil\relax
    \let\item\@idxitem
  }{%
    \if@restonecol
      \onecolumn
    \else
      \clearpage
    \fi
  }%
  \@ifclasswith{scrreprt}{idxtotoc}{%
    \renewcommand*{\idx@heading}{%
      \if@openright
        \cleardoublepage
      \else
        \clearpage
      \fi
      \twocolumn[%
        \addchap{\indexname}%
        \ifx\index@prologue\@empty
        \else
          \index@prologue
          \bigskip
        \fi
      ]%
      \@mkboth{\indexname}{\indexname}%
    }%
  }{%
    \renewcommand*{\idx@heading}{%
      \if@openright
        \cleardoublepage
      \else
        \clearpage
      \fi
      \twocolumn[%
        \chapter*{\indexname}%
        \ifx\index@prologue\@empty
        \else
          \index@prologue
          \bigskip
        \fi
      ]%
      \@mkboth{\indexname}{\indexname}%
    }%
  }%
}{}
%    \end{macrocode}
%    \begin{macrocode}
\@ifclassloaded{scrbook}{%
  \renewenvironment{theindex}{%
    \edef\indexname{%
      \the\@nameuse{idxtitle@\@indextype}%
    }%
    \if@twocolumn
      \@restonecolfalse
    \else
      \@restonecoltrue
    \fi
    \setchapterpreamble{\index@preamble}%
    \idx@heading
    \thispagestyle{\indexpagestyle}%
    \columnseprule\z@
    \columnsep 35\p@
    \parindent\z@
    \parskip\z@ \@plus .3\p@\relax
    \parfillskip\z@ \@plus 1fil\relax
    \let\item\@idxitem
  }{%
    \if@restonecol
      \onecolumn
    \else
      \clearpage
    \fi
  }%
  \@ifclasswith{scrbook}{idxtotoc}{%
    \renewcommand*{\idx@heading}{%
      \if@openright
        \cleardoublepage
      \else
        \clearpage
      \fi
      \twocolumn[%
        \addchap{\indexname}%
        \ifx\index@prologue\@empty
        \else
          \index@prologue
          \bigskip
        \fi
      ]%
      \@mkboth{\indexname}{\indexname}%
    }%
  }{%
    \renewcommand*{\idx@heading}{%
      \if@openright
        \cleardoublepage
      \else
        \clearpage
      \fi
      \twocolumn[%
        \chapter*{\indexname}%
        \ifx\index@prologue\@empty
        \else
          \index@prologue
          \bigskip
        \fi
      ]%
      \@mkboth{\indexname}{\indexname}%
    }%
  }%
}{}
%    \end{macrocode}
%
%    \begin{macrocode}
%</package>
%    \end{macrocode}
%
% \section{Installation}
%
% \subsection{Download}
%
% \paragraph{Package.} This package is available on
% CTAN\footnote{\url{ftp://ftp.ctan.org/tex-archive/}}:
% \begin{description}
% \item[\CTAN{macros/latex/contrib/oberdiek/scrindex.dtx}] The source file.
% \item[\CTAN{macros/latex/contrib/oberdiek/scrindex.pdf}] Documentation.
% \end{description}
%
%
% \paragraph{Bundle.} All the packages of the bundle `oberdiek'
% are also available in a TDS compliant ZIP archive. There
% the packages are already unpacked and the documentation files
% are generated. The files and directories obey the TDS standard.
% \begin{description}
% \item[\CTAN{install/macros/latex/contrib/oberdiek.tds.zip}]
% \end{description}
% \emph{TDS} refers to the standard ``A Directory Structure
% for \TeX\ Files'' (\CTAN{tds/tds.pdf}). Directories
% with \xfile{texmf} in their name are usually organized this way.
%
% \subsection{Bundle installation}
%
% \paragraph{Unpacking.} Unpack the \xfile{oberdiek.tds.zip} in the
% TDS tree (also known as \xfile{texmf} tree) of your choice.
% Example (linux):
% \begin{quote}
%   |unzip oberdiek.tds.zip -d ~/texmf|
% \end{quote}
%
% \paragraph{Script installation.}
% Check the directory \xfile{TDS:scripts/oberdiek/} for
% scripts that need further installation steps.
% Package \xpackage{attachfile2} comes with the Perl script
% \xfile{pdfatfi.pl} that should be installed in such a way
% that it can be called as \texttt{pdfatfi}.
% Example (linux):
% \begin{quote}
%   |chmod +x scripts/oberdiek/pdfatfi.pl|\\
%   |cp scripts/oberdiek/pdfatfi.pl /usr/local/bin/|
% \end{quote}
%
% \subsection{Package installation}
%
% \paragraph{Unpacking.} The \xfile{.dtx} file is a self-extracting
% \docstrip\ archive. The files are extracted by running the
% \xfile{.dtx} through \plainTeX:
% \begin{quote}
%   \verb|tex scrindex.dtx|
% \end{quote}
%
% \paragraph{TDS.} Now the different files must be moved into
% the different directories in your installation TDS tree
% (also known as \xfile{texmf} tree):
% \begin{quote}
% \def\t{^^A
% \begin{tabular}{@{}>{\ttfamily}l@{ $\rightarrow$ }>{\ttfamily}l@{}}
%   scrindex.sty & tex/latex/oberdiek/scrindex.sty\\
%   scrindex.pdf & doc/latex/oberdiek/scrindex.pdf\\
%   scrindex-example1.tex & doc/latex/oberdiek/scrindex-example1.tex\\
%   scrindex-example2.tex & doc/latex/oberdiek/scrindex-example2.tex\\
%   scrindex.dtx & source/latex/oberdiek/scrindex.dtx\\
% \end{tabular}^^A
% }^^A
% \sbox0{\t}^^A
% \ifdim\wd0>\linewidth
%   \begingroup
%     \advance\linewidth by\leftmargin
%     \advance\linewidth by\rightmargin
%   \edef\x{\endgroup
%     \def\noexpand\lw{\the\linewidth}^^A
%   }\x
%   \def\lwbox{^^A
%     \leavevmode
%     \hbox to \linewidth{^^A
%       \kern-\leftmargin\relax
%       \hss
%       \usebox0
%       \hss
%       \kern-\rightmargin\relax
%     }^^A
%   }^^A
%   \ifdim\wd0>\lw
%     \sbox0{\small\t}^^A
%     \ifdim\wd0>\linewidth
%       \ifdim\wd0>\lw
%         \sbox0{\footnotesize\t}^^A
%         \ifdim\wd0>\linewidth
%           \ifdim\wd0>\lw
%             \sbox0{\scriptsize\t}^^A
%             \ifdim\wd0>\linewidth
%               \ifdim\wd0>\lw
%                 \sbox0{\tiny\t}^^A
%                 \ifdim\wd0>\linewidth
%                   \lwbox
%                 \else
%                   \usebox0
%                 \fi
%               \else
%                 \lwbox
%               \fi
%             \else
%               \usebox0
%             \fi
%           \else
%             \lwbox
%           \fi
%         \else
%           \usebox0
%         \fi
%       \else
%         \lwbox
%       \fi
%     \else
%       \usebox0
%     \fi
%   \else
%     \lwbox
%   \fi
% \else
%   \usebox0
% \fi
% \end{quote}
% If you have a \xfile{docstrip.cfg} that configures and enables \docstrip's
% TDS installing feature, then some files can already be in the right
% place, see the documentation of \docstrip.
%
% \subsection{Refresh file name databases}
%
% If your \TeX~distribution
% (\teTeX, \mikTeX, \dots) relies on file name databases, you must refresh
% these. For example, \teTeX\ users run \verb|texhash| or
% \verb|mktexlsr|.
%
% \subsection{Some details for the interested}
%
% \paragraph{Attached source.}
%
% The PDF documentation on CTAN also includes the
% \xfile{.dtx} source file. It can be extracted by
% AcrobatReader 6 or higher. Another option is \textsf{pdftk},
% e.g. unpack the file into the current directory:
% \begin{quote}
%   \verb|pdftk scrindex.pdf unpack_files output .|
% \end{quote}
%
% \paragraph{Unpacking with \LaTeX.}
% The \xfile{.dtx} chooses its action depending on the format:
% \begin{description}
% \item[\plainTeX:] Run \docstrip\ and extract the files.
% \item[\LaTeX:] Generate the documentation.
% \end{description}
% If you insist on using \LaTeX\ for \docstrip\ (really,
% \docstrip\ does not need \LaTeX), then inform the autodetect routine
% about your intention:
% \begin{quote}
%   \verb|latex \let\install=y% \iffalse meta-comment
%
% File: scrindex.dtx
% Version: 2008/08/11 v1.1
% Info: Package index with KOMA-Script classes
%
% Copyright (C) 2008 by
%    Heiko Oberdiek <heiko.oberdiek at googlemail.com>
%
% This work may be distributed and/or modified under the
% conditions of the LaTeX Project Public License, either
% version 1.3c of this license or (at your option) any later
% version. This version of this license is in
%    http://www.latex-project.org/lppl/lppl-1-3c.txt
% and the latest version of this license is in
%    http://www.latex-project.org/lppl.txt
% and version 1.3 or later is part of all distributions of
% LaTeX version 2005/12/01 or later.
%
% This work has the LPPL maintenance status "maintained".
%
% This Current Maintainer of this work is Heiko Oberdiek.
%
% This work consists of the main source file scrindex.dtx
% and the derived files
%    scrindex.sty, scrindex.pdf, scrindex.ins, scrindex.drv,
%    scrindex-example1.tex, scrindex-example2.tex.
%
% Distribution:
%    CTAN:macros/latex/contrib/oberdiek/scrindex.dtx
%    CTAN:macros/latex/contrib/oberdiek/scrindex.pdf
%
% Unpacking:
%    (a) If scrindex.ins is present:
%           tex scrindex.ins
%    (b) Without scrindex.ins:
%           tex scrindex.dtx
%    (c) If you insist on using LaTeX
%           latex \let\install=y% \iffalse meta-comment
%
% File: scrindex.dtx
% Version: 2008/08/11 v1.1
% Info: Package index with KOMA-Script classes
%
% Copyright (C) 2008 by
%    Heiko Oberdiek <heiko.oberdiek at googlemail.com>
%
% This work may be distributed and/or modified under the
% conditions of the LaTeX Project Public License, either
% version 1.3c of this license or (at your option) any later
% version. This version of this license is in
%    http://www.latex-project.org/lppl/lppl-1-3c.txt
% and the latest version of this license is in
%    http://www.latex-project.org/lppl.txt
% and version 1.3 or later is part of all distributions of
% LaTeX version 2005/12/01 or later.
%
% This work has the LPPL maintenance status "maintained".
%
% This Current Maintainer of this work is Heiko Oberdiek.
%
% This work consists of the main source file scrindex.dtx
% and the derived files
%    scrindex.sty, scrindex.pdf, scrindex.ins, scrindex.drv,
%    scrindex-example1.tex, scrindex-example2.tex.
%
% Distribution:
%    CTAN:macros/latex/contrib/oberdiek/scrindex.dtx
%    CTAN:macros/latex/contrib/oberdiek/scrindex.pdf
%
% Unpacking:
%    (a) If scrindex.ins is present:
%           tex scrindex.ins
%    (b) Without scrindex.ins:
%           tex scrindex.dtx
%    (c) If you insist on using LaTeX
%           latex \let\install=y% \iffalse meta-comment
%
% File: scrindex.dtx
% Version: 2008/08/11 v1.1
% Info: Package index with KOMA-Script classes
%
% Copyright (C) 2008 by
%    Heiko Oberdiek <heiko.oberdiek at googlemail.com>
%
% This work may be distributed and/or modified under the
% conditions of the LaTeX Project Public License, either
% version 1.3c of this license or (at your option) any later
% version. This version of this license is in
%    http://www.latex-project.org/lppl/lppl-1-3c.txt
% and the latest version of this license is in
%    http://www.latex-project.org/lppl.txt
% and version 1.3 or later is part of all distributions of
% LaTeX version 2005/12/01 or later.
%
% This work has the LPPL maintenance status "maintained".
%
% This Current Maintainer of this work is Heiko Oberdiek.
%
% This work consists of the main source file scrindex.dtx
% and the derived files
%    scrindex.sty, scrindex.pdf, scrindex.ins, scrindex.drv,
%    scrindex-example1.tex, scrindex-example2.tex.
%
% Distribution:
%    CTAN:macros/latex/contrib/oberdiek/scrindex.dtx
%    CTAN:macros/latex/contrib/oberdiek/scrindex.pdf
%
% Unpacking:
%    (a) If scrindex.ins is present:
%           tex scrindex.ins
%    (b) Without scrindex.ins:
%           tex scrindex.dtx
%    (c) If you insist on using LaTeX
%           latex \let\install=y\input{scrindex.dtx}
%        (quote the arguments according to the demands of your shell)
%
% Documentation:
%    (a) If scrindex.drv is present:
%           latex scrindex.drv
%    (b) Without scrindex.drv:
%           latex scrindex.dtx; ...
%    The class ltxdoc loads the configuration file ltxdoc.cfg
%    if available. Here you can specify further options, e.g.
%    use A4 as paper format:
%       \PassOptionsToClass{a4paper}{article}
%
%    Programm calls to get the documentation (example):
%       pdflatex scrindex.dtx
%       makeindex -s gind.ist scrindex.idx
%       pdflatex scrindex.dtx
%       makeindex -s gind.ist scrindex.idx
%       pdflatex scrindex.dtx
%
% Installation:
%    TDS:tex/latex/oberdiek/scrindex.sty
%    TDS:doc/latex/oberdiek/scrindex.pdf
%    TDS:doc/latex/oberdiek/scrindex-example1.tex
%    TDS:doc/latex/oberdiek/scrindex-example2.tex
%    TDS:source/latex/oberdiek/scrindex.dtx
%
%<*ignore>
\begingroup
  \catcode123=1 %
  \catcode125=2 %
  \def\x{LaTeX2e}%
\expandafter\endgroup
\ifcase 0\ifx\install y1\fi\expandafter
         \ifx\csname processbatchFile\endcsname\relax\else1\fi
         \ifx\fmtname\x\else 1\fi\relax
\else\csname fi\endcsname
%</ignore>
%<*install>
\input docstrip.tex
\Msg{************************************************************************}
\Msg{* Installation}
\Msg{* Package: scrindex 2008/08/11 v1.1 Package index with KOMA-Script classes (HO)}
\Msg{************************************************************************}

\keepsilent
\askforoverwritefalse

\let\MetaPrefix\relax
\preamble

This is a generated file.

Project: scrindex
Version: 2008/08/11 v1.1

Copyright (C) 2008 by
   Heiko Oberdiek <heiko.oberdiek at googlemail.com>

This work may be distributed and/or modified under the
conditions of the LaTeX Project Public License, either
version 1.3c of this license or (at your option) any later
version. This version of this license is in
   http://www.latex-project.org/lppl/lppl-1-3c.txt
and the latest version of this license is in
   http://www.latex-project.org/lppl.txt
and version 1.3 or later is part of all distributions of
LaTeX version 2005/12/01 or later.

This work has the LPPL maintenance status "maintained".

This Current Maintainer of this work is Heiko Oberdiek.

This work consists of the main source file scrindex.dtx
and the derived files
   scrindex.sty, scrindex.pdf, scrindex.ins, scrindex.drv,
   scrindex-example1.tex, scrindex-example2.tex.

\endpreamble
\let\MetaPrefix\DoubleperCent

\generate{%
  \file{scrindex.ins}{\from{scrindex.dtx}{install}}%
  \file{scrindex.drv}{\from{scrindex.dtx}{driver}}%
  \usedir{tex/latex/oberdiek}%
  \file{scrindex.sty}{\from{scrindex.dtx}{package}}%
  \usedir{doc/latex/oberdiek}%
  \file{scrindex-example1.tex}{\from{scrindex.dtx}{example1}}%
  \file{scrindex-example2.tex}{\from{scrindex.dtx}{example2}}%
  \nopreamble
  \nopostamble
  \usedir{source/latex/oberdiek/catalogue}%
  \file{scrindex.xml}{\from{scrindex.dtx}{catalogue}}%
}

\catcode32=13\relax% active space
\let =\space%
\Msg{************************************************************************}
\Msg{*}
\Msg{* To finish the installation you have to move the following}
\Msg{* file into a directory searched by TeX:}
\Msg{*}
\Msg{*     scrindex.sty}
\Msg{*}
\Msg{* To produce the documentation run the file `scrindex.drv'}
\Msg{* through LaTeX.}
\Msg{*}
\Msg{* Happy TeXing!}
\Msg{*}
\Msg{************************************************************************}

\endbatchfile
%</install>
%<*ignore>
\fi
%</ignore>
%<*driver>
\NeedsTeXFormat{LaTeX2e}
\ProvidesFile{scrindex.drv}%
  [2008/08/11 v1.1 Package index with KOMA-Script classes (HO)]%
\documentclass{ltxdoc}
\usepackage{holtxdoc}[2011/11/22]
\usepackage{calc}
\begin{document}
  \DocInput{scrindex.dtx}%
\end{document}
%</driver>
% \fi
%
% \CheckSum{237}
%
% \CharacterTable
%  {Upper-case    \A\B\C\D\E\F\G\H\I\J\K\L\M\N\O\P\Q\R\S\T\U\V\W\X\Y\Z
%   Lower-case    \a\b\c\d\e\f\g\h\i\j\k\l\m\n\o\p\q\r\s\t\u\v\w\x\y\z
%   Digits        \0\1\2\3\4\5\6\7\8\9
%   Exclamation   \!     Double quote  \"     Hash (number) \#
%   Dollar        \$     Percent       \%     Ampersand     \&
%   Acute accent  \'     Left paren    \(     Right paren   \)
%   Asterisk      \*     Plus          \+     Comma         \,
%   Minus         \-     Point         \.     Solidus       \/
%   Colon         \:     Semicolon     \;     Less than     \<
%   Equals        \=     Greater than  \>     Question mark \?
%   Commercial at \@     Left bracket  \[     Backslash     \\
%   Right bracket \]     Circumflex    \^     Underscore    \_
%   Grave accent  \`     Left brace    \{     Vertical bar  \|
%   Right brace   \}     Tilde         \~}
%
% \GetFileInfo{scrindex.drv}
%
% \title{The \xpackage{scrindex} package}
% \date{2008/08/11 v1.1}
% \author{Heiko Oberdiek\\\xemail{heiko.oberdiek at googlemail.com}}
%
% \maketitle
%
% \begin{abstract}
% This package redefines environment `theindex' of package \xpackage{index},
% if a class from KOMA-Script is loaded. Also option \xoption{idxtotoc}
% is supported. Index preambles can be given either by means of package
% \xpackage{index} or via the interface provided by KOMA-Script.
% \end{abstract}
%
% \tableofcontents
%
% \section{Documentation}
%
% Package \xpackage{index}, written by David M.\ Jones, detects
% the standard classes |article|, |report|, and |book|. It
% redefines environment `theindex' for its needs.
% However, it does not know other classes such as KOMA-Script.
% This package closes the compatibiliy gap between KOMA-Script's
% classes and package \xpackage{index}.
%
% Environment |theindex| is redefined to support both package
% \xpackage{index} and KOMA-Script's classes. Thus both
% the prologe of package \xpackage{index} and the preamble
% of KOMA-Script's classes are available. Also class option |idxtotoc|
% of KOMA-Script is supported.
%
% \subsection{Usage}
%
% The package \xpackage{scrindex} is loaded without options:
%\begin{quote}
%\begin{verbatim}
%\usepackage{scrindex}
%\end{verbatim}
%\end{quote}
%
% It loads package \xpackage{index} and requests version 2004/01/20
% or later. \LaTeX's package interface allows multiple calls
% of the same package. The package is loaded at its first
% package loading command. At later times \LaTeX\ only checks
% options and a requested version date. Therefore it does not harm
% to add |\usepackage{index}| before or after |\usepackage{scrindex}|.
%
% Also the class does not matter. Environment |theindex| is only
% redefined for a supported class:
% \begin{itemize}
% \item |scrartcl|
% \item |scrreprt|
% \item |scrbook|
% \end{itemize}
%
% \subsection{Preambles}
%
% Both the prologue of package \xpackage{index} and the preamble
% of KOMA-Script's classes are supported. The position depends on
% the class.
%
% \subsubsection{Class \xclass{scrartcl}}
%
%    \begin{macrocode}
%<*example1>
\documentclass{scrartcl}
\usepackage{scrindex}
\setindexpreamble{Preamble of \texttt{scrartcl}\dotfill EOL}
\makeindex
\begin{document}
\section{First Section}
\index{first}
\index{section}
\printindex[default]%
  [Prologue of package \texttt{index}\dotfill EOL]%
\end{document}
%</example1>
%    \end{macrocode}
% The prologue of package \xpackage{index} is first set straight
% after the section title spanning both columns.
% Then the preamble of KOMA-Script follows
% in the first left column.
%
% \medskip
% \begin{quote}
%   \renewcommand*{\arraystretch}{1.2}
%   \begin{tabular}{|p{.45\linewidth}|p{.45\linewidth}|}
%   \hline
%   \multicolumn{2}{|l|}{\textbf{Index}}\\[1ex]
%   \multicolumn{2}{|p{.9\linewidth+2\tabcolsep}|}{^^A
%     Prologue of package \texttt{index}\dotfill EOL^^A
%   }\\[1ex]
%   \hline
%   Preamble of \texttt{scrartcl}\dotfill EOL&\\
%   first, 1&\\
%   section, 1&\\
%   \hline
%   \end{tabular}
% \end{quote}
%
% \subsubsection{Classes \xclass{scrreprt} and \xclass{scrbook}}
%
%    \begin{macrocode}
%<*example2>
\documentclass[openany]{scrbook}% or scrreprt
\usepackage{scrindex}
\setindexpreamble{Preamble of class \texttt{scrbook}\dotfill EOL}
\makeindex
\begin{document}
\chapter{First Chapter}
\index{first}
\index{chapter}
\printindex[default]%
  [Prologue of package \texttt{index}\dotfill EOL]%
\end{document}
%</example2>
%    \end{macrocode}
% The order of the two preambles are different for the classes
% \xclass{scrreprt} and \xclass{scrbook}. First KOMA-Script's
% chapter preamble is set, then the prologue of package \xpackage{index}
% follows. Both are set spanning both columns.
%
% \medskip
% \begin{quote}
%   \renewcommand*{\arraystretch}{1.2}
%   \begin{tabular}{|p{.45\linewidth}|p{.45\linewidth}|}
%   \hline
%   \multicolumn{2}{|l|}{\textbf{Index}}\\[1ex]
%   \multicolumn{2}{|p{.9\linewidth+2\tabcolsep}|}{^^A
%     Preamble of class \texttt{scrbook}\dotfill EOL^^A
%   }\\
%   \multicolumn{2}{|p{.9\linewidth+2\tabcolsep}|}{^^A
%     Prologue of package \xpackage{index}\dotfill EOL^^A
%   }\\[1ex]
%   \hline
%   chapter, 1&\\
%   first, 1&\\
%   \hline
%   \end{tabular}
% \end{quote}
%
% \StopEventually{
% }
%
% \section{Implementation}
%
%    \begin{macrocode}
%<*package>
\NeedsTeXFormat{LaTeX2e}
\ProvidesPackage{scrindex}
  [2008/08/11 v1.1 Package index with KOMA-Script classes (HO)]%
%    \end{macrocode}
%
%    \begin{macrocode}
\RequirePackage{index}[2004/01/20]%
%    \end{macrocode}
%
%    \begin{macrocode}
\@ifclassloaded{scrartcl}{%
  \renewenvironment{theindex}{%
    \edef\indexname{%
      \the\@nameuse{idxtitle@\@indextype}%
    }%
    \if@twocolumn
      \@restonecolfalse
    \else
      \@restonecoltrue
    \fi
    \idx@heading
    \thispagestyle{\indexpagestyle}%
    \columnseprule\z@
    \columnsep 35\p@
    \index@preamble\par\nobreak
    \parindent\z@
    \parskip\z@ \@plus .3\p@\relax
    \parfillskip\z@ \@plus 1fil\relax
    \let\item\@idxitem
  }{%
    \if@restonecol
      \onecolumn
    \else
      \clearpage
    \fi
  }%
  \@ifclasswith{scrartcl}{idxtotoc}{%
    \renewcommand*{\idx@heading}{%
      \twocolumn[%
        \addsec{\indexname}%
        \ifx\index@prologue\@empty
        \else
          \index@prologue
          \bigskip
        \fi
      ]%
      \@mkboth{\indexname}{\indexname}%
    }%
  }{%
    \renewcommand*{\idx@heading}{%
      \twocolumn[%
        \section*{\indexname}%
        \ifx\index@prologue\@empty
        \else
          \index@prologue
          \bigskip
        \fi
      ]%
      \@mkboth{\indexname}{\indexname}%
    }%
  }%
}{}
%    \end{macrocode}
%    \begin{macrocode}
\@ifclassloaded{scrreprt}{%
  \renewenvironment{theindex}{%
    \edef\indexname{%
      \the\@nameuse{idxtitle@\@indextype}%
    }%
    \if@twocolumn
      \@restonecolfalse
    \else
      \@restonecoltrue
    \fi
    \setchapterpreamble{\index@preamble}%
    \idx@heading
    \thispagestyle{\indexpagestyle}%
    \columnseprule\z@
    \columnsep 35\p@
    \parindent\z@
    \parskip\z@ \@plus .3\p@\relax
    \parfillskip\z@ \@plus 1fil\relax
    \let\item\@idxitem
  }{%
    \if@restonecol
      \onecolumn
    \else
      \clearpage
    \fi
  }%
  \@ifclasswith{scrreprt}{idxtotoc}{%
    \renewcommand*{\idx@heading}{%
      \if@openright
        \cleardoublepage
      \else
        \clearpage
      \fi
      \twocolumn[%
        \addchap{\indexname}%
        \ifx\index@prologue\@empty
        \else
          \index@prologue
          \bigskip
        \fi
      ]%
      \@mkboth{\indexname}{\indexname}%
    }%
  }{%
    \renewcommand*{\idx@heading}{%
      \if@openright
        \cleardoublepage
      \else
        \clearpage
      \fi
      \twocolumn[%
        \chapter*{\indexname}%
        \ifx\index@prologue\@empty
        \else
          \index@prologue
          \bigskip
        \fi
      ]%
      \@mkboth{\indexname}{\indexname}%
    }%
  }%
}{}
%    \end{macrocode}
%    \begin{macrocode}
\@ifclassloaded{scrbook}{%
  \renewenvironment{theindex}{%
    \edef\indexname{%
      \the\@nameuse{idxtitle@\@indextype}%
    }%
    \if@twocolumn
      \@restonecolfalse
    \else
      \@restonecoltrue
    \fi
    \setchapterpreamble{\index@preamble}%
    \idx@heading
    \thispagestyle{\indexpagestyle}%
    \columnseprule\z@
    \columnsep 35\p@
    \parindent\z@
    \parskip\z@ \@plus .3\p@\relax
    \parfillskip\z@ \@plus 1fil\relax
    \let\item\@idxitem
  }{%
    \if@restonecol
      \onecolumn
    \else
      \clearpage
    \fi
  }%
  \@ifclasswith{scrbook}{idxtotoc}{%
    \renewcommand*{\idx@heading}{%
      \if@openright
        \cleardoublepage
      \else
        \clearpage
      \fi
      \twocolumn[%
        \addchap{\indexname}%
        \ifx\index@prologue\@empty
        \else
          \index@prologue
          \bigskip
        \fi
      ]%
      \@mkboth{\indexname}{\indexname}%
    }%
  }{%
    \renewcommand*{\idx@heading}{%
      \if@openright
        \cleardoublepage
      \else
        \clearpage
      \fi
      \twocolumn[%
        \chapter*{\indexname}%
        \ifx\index@prologue\@empty
        \else
          \index@prologue
          \bigskip
        \fi
      ]%
      \@mkboth{\indexname}{\indexname}%
    }%
  }%
}{}
%    \end{macrocode}
%
%    \begin{macrocode}
%</package>
%    \end{macrocode}
%
% \section{Installation}
%
% \subsection{Download}
%
% \paragraph{Package.} This package is available on
% CTAN\footnote{\url{ftp://ftp.ctan.org/tex-archive/}}:
% \begin{description}
% \item[\CTAN{macros/latex/contrib/oberdiek/scrindex.dtx}] The source file.
% \item[\CTAN{macros/latex/contrib/oberdiek/scrindex.pdf}] Documentation.
% \end{description}
%
%
% \paragraph{Bundle.} All the packages of the bundle `oberdiek'
% are also available in a TDS compliant ZIP archive. There
% the packages are already unpacked and the documentation files
% are generated. The files and directories obey the TDS standard.
% \begin{description}
% \item[\CTAN{install/macros/latex/contrib/oberdiek.tds.zip}]
% \end{description}
% \emph{TDS} refers to the standard ``A Directory Structure
% for \TeX\ Files'' (\CTAN{tds/tds.pdf}). Directories
% with \xfile{texmf} in their name are usually organized this way.
%
% \subsection{Bundle installation}
%
% \paragraph{Unpacking.} Unpack the \xfile{oberdiek.tds.zip} in the
% TDS tree (also known as \xfile{texmf} tree) of your choice.
% Example (linux):
% \begin{quote}
%   |unzip oberdiek.tds.zip -d ~/texmf|
% \end{quote}
%
% \paragraph{Script installation.}
% Check the directory \xfile{TDS:scripts/oberdiek/} for
% scripts that need further installation steps.
% Package \xpackage{attachfile2} comes with the Perl script
% \xfile{pdfatfi.pl} that should be installed in such a way
% that it can be called as \texttt{pdfatfi}.
% Example (linux):
% \begin{quote}
%   |chmod +x scripts/oberdiek/pdfatfi.pl|\\
%   |cp scripts/oberdiek/pdfatfi.pl /usr/local/bin/|
% \end{quote}
%
% \subsection{Package installation}
%
% \paragraph{Unpacking.} The \xfile{.dtx} file is a self-extracting
% \docstrip\ archive. The files are extracted by running the
% \xfile{.dtx} through \plainTeX:
% \begin{quote}
%   \verb|tex scrindex.dtx|
% \end{quote}
%
% \paragraph{TDS.} Now the different files must be moved into
% the different directories in your installation TDS tree
% (also known as \xfile{texmf} tree):
% \begin{quote}
% \def\t{^^A
% \begin{tabular}{@{}>{\ttfamily}l@{ $\rightarrow$ }>{\ttfamily}l@{}}
%   scrindex.sty & tex/latex/oberdiek/scrindex.sty\\
%   scrindex.pdf & doc/latex/oberdiek/scrindex.pdf\\
%   scrindex-example1.tex & doc/latex/oberdiek/scrindex-example1.tex\\
%   scrindex-example2.tex & doc/latex/oberdiek/scrindex-example2.tex\\
%   scrindex.dtx & source/latex/oberdiek/scrindex.dtx\\
% \end{tabular}^^A
% }^^A
% \sbox0{\t}^^A
% \ifdim\wd0>\linewidth
%   \begingroup
%     \advance\linewidth by\leftmargin
%     \advance\linewidth by\rightmargin
%   \edef\x{\endgroup
%     \def\noexpand\lw{\the\linewidth}^^A
%   }\x
%   \def\lwbox{^^A
%     \leavevmode
%     \hbox to \linewidth{^^A
%       \kern-\leftmargin\relax
%       \hss
%       \usebox0
%       \hss
%       \kern-\rightmargin\relax
%     }^^A
%   }^^A
%   \ifdim\wd0>\lw
%     \sbox0{\small\t}^^A
%     \ifdim\wd0>\linewidth
%       \ifdim\wd0>\lw
%         \sbox0{\footnotesize\t}^^A
%         \ifdim\wd0>\linewidth
%           \ifdim\wd0>\lw
%             \sbox0{\scriptsize\t}^^A
%             \ifdim\wd0>\linewidth
%               \ifdim\wd0>\lw
%                 \sbox0{\tiny\t}^^A
%                 \ifdim\wd0>\linewidth
%                   \lwbox
%                 \else
%                   \usebox0
%                 \fi
%               \else
%                 \lwbox
%               \fi
%             \else
%               \usebox0
%             \fi
%           \else
%             \lwbox
%           \fi
%         \else
%           \usebox0
%         \fi
%       \else
%         \lwbox
%       \fi
%     \else
%       \usebox0
%     \fi
%   \else
%     \lwbox
%   \fi
% \else
%   \usebox0
% \fi
% \end{quote}
% If you have a \xfile{docstrip.cfg} that configures and enables \docstrip's
% TDS installing feature, then some files can already be in the right
% place, see the documentation of \docstrip.
%
% \subsection{Refresh file name databases}
%
% If your \TeX~distribution
% (\teTeX, \mikTeX, \dots) relies on file name databases, you must refresh
% these. For example, \teTeX\ users run \verb|texhash| or
% \verb|mktexlsr|.
%
% \subsection{Some details for the interested}
%
% \paragraph{Attached source.}
%
% The PDF documentation on CTAN also includes the
% \xfile{.dtx} source file. It can be extracted by
% AcrobatReader 6 or higher. Another option is \textsf{pdftk},
% e.g. unpack the file into the current directory:
% \begin{quote}
%   \verb|pdftk scrindex.pdf unpack_files output .|
% \end{quote}
%
% \paragraph{Unpacking with \LaTeX.}
% The \xfile{.dtx} chooses its action depending on the format:
% \begin{description}
% \item[\plainTeX:] Run \docstrip\ and extract the files.
% \item[\LaTeX:] Generate the documentation.
% \end{description}
% If you insist on using \LaTeX\ for \docstrip\ (really,
% \docstrip\ does not need \LaTeX), then inform the autodetect routine
% about your intention:
% \begin{quote}
%   \verb|latex \let\install=y\input{scrindex.dtx}|
% \end{quote}
% Do not forget to quote the argument according to the demands
% of your shell.
%
% \paragraph{Generating the documentation.}
% You can use both the \xfile{.dtx} or the \xfile{.drv} to generate
% the documentation. The process can be configured by the
% configuration file \xfile{ltxdoc.cfg}. For instance, put this
% line into this file, if you want to have A4 as paper format:
% \begin{quote}
%   \verb|\PassOptionsToClass{a4paper}{article}|
% \end{quote}
% An example follows how to generate the
% documentation with pdf\LaTeX:
% \begin{quote}
%\begin{verbatim}
%pdflatex scrindex.dtx
%makeindex -s gind.ist scrindex.idx
%pdflatex scrindex.dtx
%makeindex -s gind.ist scrindex.idx
%pdflatex scrindex.dtx
%\end{verbatim}
% \end{quote}
%
% \section{Catalogue}
%
% The following XML file can be used as source for the
% \href{http://mirror.ctan.org/help/Catalogue/catalogue.html}{\TeX\ Catalogue}.
% The elements \texttt{caption} and \texttt{description} are imported
% from the original XML file from the Catalogue.
% The name of the XML file in the Catalogue is \xfile{scrindex.xml}.
%    \begin{macrocode}
%<*catalogue>
<?xml version='1.0' encoding='us-ascii'?>
<!DOCTYPE entry SYSTEM 'catalogue.dtd'>
<entry datestamp='$Date$' modifier='$Author$' id='scrindex'>
  <name>scrindex</name>
  <caption>Make index package work with Koma-script classes.</caption>
  <authorref id='auth:oberdiek'/>
  <copyright owner='Heiko Oberdiek' year='2008'/>
  <license type='lppl1.3'/>
  <version number='1.1'/>
  <description>
    This package redefines environment `theindex' of package `index',
    if a class from <xref refid='koma-script'>KOMA-Script</xref> is loaded.
    Also option `idxtotoc' is supported. Index preambles can be given
    either by means of package `index' or via the interface provided
    by <xref refid='koma-script'>KOMA-Script</xref>.
    <p/>
    The package is part of the <xref refid='oberdiek'>oberdiek</xref>
    bundle.
  </description>
  <documentation details='Package documentation'
      href='ctan:/macros/latex/contrib/oberdiek/scrindex.pdf'/>
  <ctan file='true' path='/macros/latex/contrib/oberdiek/scrindex.dtx'/>
  <miktex location='oberdiek'/>
  <texlive location='oberdiek'/>
  <install path='/macros/latex/contrib/oberdiek/oberdiek.tds.zip'/>
</entry>
%</catalogue>
%    \end{macrocode}
%
% \begin{History}
%   \begin{Version}{2008/07/07 v1.0}
%   \item
%     First version, also published in newsgroup \xnewsgroup{de.comp.text.tex}:\\
%     \URL{``\link{Re: Z\"ahler bei \cs{index}}''}^^A
%     {http://groups.google.com/group/de.comp.text.tex/msg/39575b5e2f29be1e}
%   \end{Version}
%   \begin{Version}{2008/08/11 v1.1}
%   \item
%     Code is not changed.
%   \item
%     URLs updated.
%   \end{Version}
% \end{History}
%
% \PrintIndex
%
% \Finale
\endinput

%        (quote the arguments according to the demands of your shell)
%
% Documentation:
%    (a) If scrindex.drv is present:
%           latex scrindex.drv
%    (b) Without scrindex.drv:
%           latex scrindex.dtx; ...
%    The class ltxdoc loads the configuration file ltxdoc.cfg
%    if available. Here you can specify further options, e.g.
%    use A4 as paper format:
%       \PassOptionsToClass{a4paper}{article}
%
%    Programm calls to get the documentation (example):
%       pdflatex scrindex.dtx
%       makeindex -s gind.ist scrindex.idx
%       pdflatex scrindex.dtx
%       makeindex -s gind.ist scrindex.idx
%       pdflatex scrindex.dtx
%
% Installation:
%    TDS:tex/latex/oberdiek/scrindex.sty
%    TDS:doc/latex/oberdiek/scrindex.pdf
%    TDS:doc/latex/oberdiek/scrindex-example1.tex
%    TDS:doc/latex/oberdiek/scrindex-example2.tex
%    TDS:source/latex/oberdiek/scrindex.dtx
%
%<*ignore>
\begingroup
  \catcode123=1 %
  \catcode125=2 %
  \def\x{LaTeX2e}%
\expandafter\endgroup
\ifcase 0\ifx\install y1\fi\expandafter
         \ifx\csname processbatchFile\endcsname\relax\else1\fi
         \ifx\fmtname\x\else 1\fi\relax
\else\csname fi\endcsname
%</ignore>
%<*install>
\input docstrip.tex
\Msg{************************************************************************}
\Msg{* Installation}
\Msg{* Package: scrindex 2008/08/11 v1.1 Package index with KOMA-Script classes (HO)}
\Msg{************************************************************************}

\keepsilent
\askforoverwritefalse

\let\MetaPrefix\relax
\preamble

This is a generated file.

Project: scrindex
Version: 2008/08/11 v1.1

Copyright (C) 2008 by
   Heiko Oberdiek <heiko.oberdiek at googlemail.com>

This work may be distributed and/or modified under the
conditions of the LaTeX Project Public License, either
version 1.3c of this license or (at your option) any later
version. This version of this license is in
   http://www.latex-project.org/lppl/lppl-1-3c.txt
and the latest version of this license is in
   http://www.latex-project.org/lppl.txt
and version 1.3 or later is part of all distributions of
LaTeX version 2005/12/01 or later.

This work has the LPPL maintenance status "maintained".

This Current Maintainer of this work is Heiko Oberdiek.

This work consists of the main source file scrindex.dtx
and the derived files
   scrindex.sty, scrindex.pdf, scrindex.ins, scrindex.drv,
   scrindex-example1.tex, scrindex-example2.tex.

\endpreamble
\let\MetaPrefix\DoubleperCent

\generate{%
  \file{scrindex.ins}{\from{scrindex.dtx}{install}}%
  \file{scrindex.drv}{\from{scrindex.dtx}{driver}}%
  \usedir{tex/latex/oberdiek}%
  \file{scrindex.sty}{\from{scrindex.dtx}{package}}%
  \usedir{doc/latex/oberdiek}%
  \file{scrindex-example1.tex}{\from{scrindex.dtx}{example1}}%
  \file{scrindex-example2.tex}{\from{scrindex.dtx}{example2}}%
  \nopreamble
  \nopostamble
  \usedir{source/latex/oberdiek/catalogue}%
  \file{scrindex.xml}{\from{scrindex.dtx}{catalogue}}%
}

\catcode32=13\relax% active space
\let =\space%
\Msg{************************************************************************}
\Msg{*}
\Msg{* To finish the installation you have to move the following}
\Msg{* file into a directory searched by TeX:}
\Msg{*}
\Msg{*     scrindex.sty}
\Msg{*}
\Msg{* To produce the documentation run the file `scrindex.drv'}
\Msg{* through LaTeX.}
\Msg{*}
\Msg{* Happy TeXing!}
\Msg{*}
\Msg{************************************************************************}

\endbatchfile
%</install>
%<*ignore>
\fi
%</ignore>
%<*driver>
\NeedsTeXFormat{LaTeX2e}
\ProvidesFile{scrindex.drv}%
  [2008/08/11 v1.1 Package index with KOMA-Script classes (HO)]%
\documentclass{ltxdoc}
\usepackage{holtxdoc}[2011/11/22]
\usepackage{calc}
\begin{document}
  \DocInput{scrindex.dtx}%
\end{document}
%</driver>
% \fi
%
% \CheckSum{237}
%
% \CharacterTable
%  {Upper-case    \A\B\C\D\E\F\G\H\I\J\K\L\M\N\O\P\Q\R\S\T\U\V\W\X\Y\Z
%   Lower-case    \a\b\c\d\e\f\g\h\i\j\k\l\m\n\o\p\q\r\s\t\u\v\w\x\y\z
%   Digits        \0\1\2\3\4\5\6\7\8\9
%   Exclamation   \!     Double quote  \"     Hash (number) \#
%   Dollar        \$     Percent       \%     Ampersand     \&
%   Acute accent  \'     Left paren    \(     Right paren   \)
%   Asterisk      \*     Plus          \+     Comma         \,
%   Minus         \-     Point         \.     Solidus       \/
%   Colon         \:     Semicolon     \;     Less than     \<
%   Equals        \=     Greater than  \>     Question mark \?
%   Commercial at \@     Left bracket  \[     Backslash     \\
%   Right bracket \]     Circumflex    \^     Underscore    \_
%   Grave accent  \`     Left brace    \{     Vertical bar  \|
%   Right brace   \}     Tilde         \~}
%
% \GetFileInfo{scrindex.drv}
%
% \title{The \xpackage{scrindex} package}
% \date{2008/08/11 v1.1}
% \author{Heiko Oberdiek\\\xemail{heiko.oberdiek at googlemail.com}}
%
% \maketitle
%
% \begin{abstract}
% This package redefines environment `theindex' of package \xpackage{index},
% if a class from KOMA-Script is loaded. Also option \xoption{idxtotoc}
% is supported. Index preambles can be given either by means of package
% \xpackage{index} or via the interface provided by KOMA-Script.
% \end{abstract}
%
% \tableofcontents
%
% \section{Documentation}
%
% Package \xpackage{index}, written by David M.\ Jones, detects
% the standard classes |article|, |report|, and |book|. It
% redefines environment `theindex' for its needs.
% However, it does not know other classes such as KOMA-Script.
% This package closes the compatibiliy gap between KOMA-Script's
% classes and package \xpackage{index}.
%
% Environment |theindex| is redefined to support both package
% \xpackage{index} and KOMA-Script's classes. Thus both
% the prologe of package \xpackage{index} and the preamble
% of KOMA-Script's classes are available. Also class option |idxtotoc|
% of KOMA-Script is supported.
%
% \subsection{Usage}
%
% The package \xpackage{scrindex} is loaded without options:
%\begin{quote}
%\begin{verbatim}
%\usepackage{scrindex}
%\end{verbatim}
%\end{quote}
%
% It loads package \xpackage{index} and requests version 2004/01/20
% or later. \LaTeX's package interface allows multiple calls
% of the same package. The package is loaded at its first
% package loading command. At later times \LaTeX\ only checks
% options and a requested version date. Therefore it does not harm
% to add |\usepackage{index}| before or after |\usepackage{scrindex}|.
%
% Also the class does not matter. Environment |theindex| is only
% redefined for a supported class:
% \begin{itemize}
% \item |scrartcl|
% \item |scrreprt|
% \item |scrbook|
% \end{itemize}
%
% \subsection{Preambles}
%
% Both the prologue of package \xpackage{index} and the preamble
% of KOMA-Script's classes are supported. The position depends on
% the class.
%
% \subsubsection{Class \xclass{scrartcl}}
%
%    \begin{macrocode}
%<*example1>
\documentclass{scrartcl}
\usepackage{scrindex}
\setindexpreamble{Preamble of \texttt{scrartcl}\dotfill EOL}
\makeindex
\begin{document}
\section{First Section}
\index{first}
\index{section}
\printindex[default]%
  [Prologue of package \texttt{index}\dotfill EOL]%
\end{document}
%</example1>
%    \end{macrocode}
% The prologue of package \xpackage{index} is first set straight
% after the section title spanning both columns.
% Then the preamble of KOMA-Script follows
% in the first left column.
%
% \medskip
% \begin{quote}
%   \renewcommand*{\arraystretch}{1.2}
%   \begin{tabular}{|p{.45\linewidth}|p{.45\linewidth}|}
%   \hline
%   \multicolumn{2}{|l|}{\textbf{Index}}\\[1ex]
%   \multicolumn{2}{|p{.9\linewidth+2\tabcolsep}|}{^^A
%     Prologue of package \texttt{index}\dotfill EOL^^A
%   }\\[1ex]
%   \hline
%   Preamble of \texttt{scrartcl}\dotfill EOL&\\
%   first, 1&\\
%   section, 1&\\
%   \hline
%   \end{tabular}
% \end{quote}
%
% \subsubsection{Classes \xclass{scrreprt} and \xclass{scrbook}}
%
%    \begin{macrocode}
%<*example2>
\documentclass[openany]{scrbook}% or scrreprt
\usepackage{scrindex}
\setindexpreamble{Preamble of class \texttt{scrbook}\dotfill EOL}
\makeindex
\begin{document}
\chapter{First Chapter}
\index{first}
\index{chapter}
\printindex[default]%
  [Prologue of package \texttt{index}\dotfill EOL]%
\end{document}
%</example2>
%    \end{macrocode}
% The order of the two preambles are different for the classes
% \xclass{scrreprt} and \xclass{scrbook}. First KOMA-Script's
% chapter preamble is set, then the prologue of package \xpackage{index}
% follows. Both are set spanning both columns.
%
% \medskip
% \begin{quote}
%   \renewcommand*{\arraystretch}{1.2}
%   \begin{tabular}{|p{.45\linewidth}|p{.45\linewidth}|}
%   \hline
%   \multicolumn{2}{|l|}{\textbf{Index}}\\[1ex]
%   \multicolumn{2}{|p{.9\linewidth+2\tabcolsep}|}{^^A
%     Preamble of class \texttt{scrbook}\dotfill EOL^^A
%   }\\
%   \multicolumn{2}{|p{.9\linewidth+2\tabcolsep}|}{^^A
%     Prologue of package \xpackage{index}\dotfill EOL^^A
%   }\\[1ex]
%   \hline
%   chapter, 1&\\
%   first, 1&\\
%   \hline
%   \end{tabular}
% \end{quote}
%
% \StopEventually{
% }
%
% \section{Implementation}
%
%    \begin{macrocode}
%<*package>
\NeedsTeXFormat{LaTeX2e}
\ProvidesPackage{scrindex}
  [2008/08/11 v1.1 Package index with KOMA-Script classes (HO)]%
%    \end{macrocode}
%
%    \begin{macrocode}
\RequirePackage{index}[2004/01/20]%
%    \end{macrocode}
%
%    \begin{macrocode}
\@ifclassloaded{scrartcl}{%
  \renewenvironment{theindex}{%
    \edef\indexname{%
      \the\@nameuse{idxtitle@\@indextype}%
    }%
    \if@twocolumn
      \@restonecolfalse
    \else
      \@restonecoltrue
    \fi
    \idx@heading
    \thispagestyle{\indexpagestyle}%
    \columnseprule\z@
    \columnsep 35\p@
    \index@preamble\par\nobreak
    \parindent\z@
    \parskip\z@ \@plus .3\p@\relax
    \parfillskip\z@ \@plus 1fil\relax
    \let\item\@idxitem
  }{%
    \if@restonecol
      \onecolumn
    \else
      \clearpage
    \fi
  }%
  \@ifclasswith{scrartcl}{idxtotoc}{%
    \renewcommand*{\idx@heading}{%
      \twocolumn[%
        \addsec{\indexname}%
        \ifx\index@prologue\@empty
        \else
          \index@prologue
          \bigskip
        \fi
      ]%
      \@mkboth{\indexname}{\indexname}%
    }%
  }{%
    \renewcommand*{\idx@heading}{%
      \twocolumn[%
        \section*{\indexname}%
        \ifx\index@prologue\@empty
        \else
          \index@prologue
          \bigskip
        \fi
      ]%
      \@mkboth{\indexname}{\indexname}%
    }%
  }%
}{}
%    \end{macrocode}
%    \begin{macrocode}
\@ifclassloaded{scrreprt}{%
  \renewenvironment{theindex}{%
    \edef\indexname{%
      \the\@nameuse{idxtitle@\@indextype}%
    }%
    \if@twocolumn
      \@restonecolfalse
    \else
      \@restonecoltrue
    \fi
    \setchapterpreamble{\index@preamble}%
    \idx@heading
    \thispagestyle{\indexpagestyle}%
    \columnseprule\z@
    \columnsep 35\p@
    \parindent\z@
    \parskip\z@ \@plus .3\p@\relax
    \parfillskip\z@ \@plus 1fil\relax
    \let\item\@idxitem
  }{%
    \if@restonecol
      \onecolumn
    \else
      \clearpage
    \fi
  }%
  \@ifclasswith{scrreprt}{idxtotoc}{%
    \renewcommand*{\idx@heading}{%
      \if@openright
        \cleardoublepage
      \else
        \clearpage
      \fi
      \twocolumn[%
        \addchap{\indexname}%
        \ifx\index@prologue\@empty
        \else
          \index@prologue
          \bigskip
        \fi
      ]%
      \@mkboth{\indexname}{\indexname}%
    }%
  }{%
    \renewcommand*{\idx@heading}{%
      \if@openright
        \cleardoublepage
      \else
        \clearpage
      \fi
      \twocolumn[%
        \chapter*{\indexname}%
        \ifx\index@prologue\@empty
        \else
          \index@prologue
          \bigskip
        \fi
      ]%
      \@mkboth{\indexname}{\indexname}%
    }%
  }%
}{}
%    \end{macrocode}
%    \begin{macrocode}
\@ifclassloaded{scrbook}{%
  \renewenvironment{theindex}{%
    \edef\indexname{%
      \the\@nameuse{idxtitle@\@indextype}%
    }%
    \if@twocolumn
      \@restonecolfalse
    \else
      \@restonecoltrue
    \fi
    \setchapterpreamble{\index@preamble}%
    \idx@heading
    \thispagestyle{\indexpagestyle}%
    \columnseprule\z@
    \columnsep 35\p@
    \parindent\z@
    \parskip\z@ \@plus .3\p@\relax
    \parfillskip\z@ \@plus 1fil\relax
    \let\item\@idxitem
  }{%
    \if@restonecol
      \onecolumn
    \else
      \clearpage
    \fi
  }%
  \@ifclasswith{scrbook}{idxtotoc}{%
    \renewcommand*{\idx@heading}{%
      \if@openright
        \cleardoublepage
      \else
        \clearpage
      \fi
      \twocolumn[%
        \addchap{\indexname}%
        \ifx\index@prologue\@empty
        \else
          \index@prologue
          \bigskip
        \fi
      ]%
      \@mkboth{\indexname}{\indexname}%
    }%
  }{%
    \renewcommand*{\idx@heading}{%
      \if@openright
        \cleardoublepage
      \else
        \clearpage
      \fi
      \twocolumn[%
        \chapter*{\indexname}%
        \ifx\index@prologue\@empty
        \else
          \index@prologue
          \bigskip
        \fi
      ]%
      \@mkboth{\indexname}{\indexname}%
    }%
  }%
}{}
%    \end{macrocode}
%
%    \begin{macrocode}
%</package>
%    \end{macrocode}
%
% \section{Installation}
%
% \subsection{Download}
%
% \paragraph{Package.} This package is available on
% CTAN\footnote{\url{ftp://ftp.ctan.org/tex-archive/}}:
% \begin{description}
% \item[\CTAN{macros/latex/contrib/oberdiek/scrindex.dtx}] The source file.
% \item[\CTAN{macros/latex/contrib/oberdiek/scrindex.pdf}] Documentation.
% \end{description}
%
%
% \paragraph{Bundle.} All the packages of the bundle `oberdiek'
% are also available in a TDS compliant ZIP archive. There
% the packages are already unpacked and the documentation files
% are generated. The files and directories obey the TDS standard.
% \begin{description}
% \item[\CTAN{install/macros/latex/contrib/oberdiek.tds.zip}]
% \end{description}
% \emph{TDS} refers to the standard ``A Directory Structure
% for \TeX\ Files'' (\CTAN{tds/tds.pdf}). Directories
% with \xfile{texmf} in their name are usually organized this way.
%
% \subsection{Bundle installation}
%
% \paragraph{Unpacking.} Unpack the \xfile{oberdiek.tds.zip} in the
% TDS tree (also known as \xfile{texmf} tree) of your choice.
% Example (linux):
% \begin{quote}
%   |unzip oberdiek.tds.zip -d ~/texmf|
% \end{quote}
%
% \paragraph{Script installation.}
% Check the directory \xfile{TDS:scripts/oberdiek/} for
% scripts that need further installation steps.
% Package \xpackage{attachfile2} comes with the Perl script
% \xfile{pdfatfi.pl} that should be installed in such a way
% that it can be called as \texttt{pdfatfi}.
% Example (linux):
% \begin{quote}
%   |chmod +x scripts/oberdiek/pdfatfi.pl|\\
%   |cp scripts/oberdiek/pdfatfi.pl /usr/local/bin/|
% \end{quote}
%
% \subsection{Package installation}
%
% \paragraph{Unpacking.} The \xfile{.dtx} file is a self-extracting
% \docstrip\ archive. The files are extracted by running the
% \xfile{.dtx} through \plainTeX:
% \begin{quote}
%   \verb|tex scrindex.dtx|
% \end{quote}
%
% \paragraph{TDS.} Now the different files must be moved into
% the different directories in your installation TDS tree
% (also known as \xfile{texmf} tree):
% \begin{quote}
% \def\t{^^A
% \begin{tabular}{@{}>{\ttfamily}l@{ $\rightarrow$ }>{\ttfamily}l@{}}
%   scrindex.sty & tex/latex/oberdiek/scrindex.sty\\
%   scrindex.pdf & doc/latex/oberdiek/scrindex.pdf\\
%   scrindex-example1.tex & doc/latex/oberdiek/scrindex-example1.tex\\
%   scrindex-example2.tex & doc/latex/oberdiek/scrindex-example2.tex\\
%   scrindex.dtx & source/latex/oberdiek/scrindex.dtx\\
% \end{tabular}^^A
% }^^A
% \sbox0{\t}^^A
% \ifdim\wd0>\linewidth
%   \begingroup
%     \advance\linewidth by\leftmargin
%     \advance\linewidth by\rightmargin
%   \edef\x{\endgroup
%     \def\noexpand\lw{\the\linewidth}^^A
%   }\x
%   \def\lwbox{^^A
%     \leavevmode
%     \hbox to \linewidth{^^A
%       \kern-\leftmargin\relax
%       \hss
%       \usebox0
%       \hss
%       \kern-\rightmargin\relax
%     }^^A
%   }^^A
%   \ifdim\wd0>\lw
%     \sbox0{\small\t}^^A
%     \ifdim\wd0>\linewidth
%       \ifdim\wd0>\lw
%         \sbox0{\footnotesize\t}^^A
%         \ifdim\wd0>\linewidth
%           \ifdim\wd0>\lw
%             \sbox0{\scriptsize\t}^^A
%             \ifdim\wd0>\linewidth
%               \ifdim\wd0>\lw
%                 \sbox0{\tiny\t}^^A
%                 \ifdim\wd0>\linewidth
%                   \lwbox
%                 \else
%                   \usebox0
%                 \fi
%               \else
%                 \lwbox
%               \fi
%             \else
%               \usebox0
%             \fi
%           \else
%             \lwbox
%           \fi
%         \else
%           \usebox0
%         \fi
%       \else
%         \lwbox
%       \fi
%     \else
%       \usebox0
%     \fi
%   \else
%     \lwbox
%   \fi
% \else
%   \usebox0
% \fi
% \end{quote}
% If you have a \xfile{docstrip.cfg} that configures and enables \docstrip's
% TDS installing feature, then some files can already be in the right
% place, see the documentation of \docstrip.
%
% \subsection{Refresh file name databases}
%
% If your \TeX~distribution
% (\teTeX, \mikTeX, \dots) relies on file name databases, you must refresh
% these. For example, \teTeX\ users run \verb|texhash| or
% \verb|mktexlsr|.
%
% \subsection{Some details for the interested}
%
% \paragraph{Attached source.}
%
% The PDF documentation on CTAN also includes the
% \xfile{.dtx} source file. It can be extracted by
% AcrobatReader 6 or higher. Another option is \textsf{pdftk},
% e.g. unpack the file into the current directory:
% \begin{quote}
%   \verb|pdftk scrindex.pdf unpack_files output .|
% \end{quote}
%
% \paragraph{Unpacking with \LaTeX.}
% The \xfile{.dtx} chooses its action depending on the format:
% \begin{description}
% \item[\plainTeX:] Run \docstrip\ and extract the files.
% \item[\LaTeX:] Generate the documentation.
% \end{description}
% If you insist on using \LaTeX\ for \docstrip\ (really,
% \docstrip\ does not need \LaTeX), then inform the autodetect routine
% about your intention:
% \begin{quote}
%   \verb|latex \let\install=y% \iffalse meta-comment
%
% File: scrindex.dtx
% Version: 2008/08/11 v1.1
% Info: Package index with KOMA-Script classes
%
% Copyright (C) 2008 by
%    Heiko Oberdiek <heiko.oberdiek at googlemail.com>
%
% This work may be distributed and/or modified under the
% conditions of the LaTeX Project Public License, either
% version 1.3c of this license or (at your option) any later
% version. This version of this license is in
%    http://www.latex-project.org/lppl/lppl-1-3c.txt
% and the latest version of this license is in
%    http://www.latex-project.org/lppl.txt
% and version 1.3 or later is part of all distributions of
% LaTeX version 2005/12/01 or later.
%
% This work has the LPPL maintenance status "maintained".
%
% This Current Maintainer of this work is Heiko Oberdiek.
%
% This work consists of the main source file scrindex.dtx
% and the derived files
%    scrindex.sty, scrindex.pdf, scrindex.ins, scrindex.drv,
%    scrindex-example1.tex, scrindex-example2.tex.
%
% Distribution:
%    CTAN:macros/latex/contrib/oberdiek/scrindex.dtx
%    CTAN:macros/latex/contrib/oberdiek/scrindex.pdf
%
% Unpacking:
%    (a) If scrindex.ins is present:
%           tex scrindex.ins
%    (b) Without scrindex.ins:
%           tex scrindex.dtx
%    (c) If you insist on using LaTeX
%           latex \let\install=y\input{scrindex.dtx}
%        (quote the arguments according to the demands of your shell)
%
% Documentation:
%    (a) If scrindex.drv is present:
%           latex scrindex.drv
%    (b) Without scrindex.drv:
%           latex scrindex.dtx; ...
%    The class ltxdoc loads the configuration file ltxdoc.cfg
%    if available. Here you can specify further options, e.g.
%    use A4 as paper format:
%       \PassOptionsToClass{a4paper}{article}
%
%    Programm calls to get the documentation (example):
%       pdflatex scrindex.dtx
%       makeindex -s gind.ist scrindex.idx
%       pdflatex scrindex.dtx
%       makeindex -s gind.ist scrindex.idx
%       pdflatex scrindex.dtx
%
% Installation:
%    TDS:tex/latex/oberdiek/scrindex.sty
%    TDS:doc/latex/oberdiek/scrindex.pdf
%    TDS:doc/latex/oberdiek/scrindex-example1.tex
%    TDS:doc/latex/oberdiek/scrindex-example2.tex
%    TDS:source/latex/oberdiek/scrindex.dtx
%
%<*ignore>
\begingroup
  \catcode123=1 %
  \catcode125=2 %
  \def\x{LaTeX2e}%
\expandafter\endgroup
\ifcase 0\ifx\install y1\fi\expandafter
         \ifx\csname processbatchFile\endcsname\relax\else1\fi
         \ifx\fmtname\x\else 1\fi\relax
\else\csname fi\endcsname
%</ignore>
%<*install>
\input docstrip.tex
\Msg{************************************************************************}
\Msg{* Installation}
\Msg{* Package: scrindex 2008/08/11 v1.1 Package index with KOMA-Script classes (HO)}
\Msg{************************************************************************}

\keepsilent
\askforoverwritefalse

\let\MetaPrefix\relax
\preamble

This is a generated file.

Project: scrindex
Version: 2008/08/11 v1.1

Copyright (C) 2008 by
   Heiko Oberdiek <heiko.oberdiek at googlemail.com>

This work may be distributed and/or modified under the
conditions of the LaTeX Project Public License, either
version 1.3c of this license or (at your option) any later
version. This version of this license is in
   http://www.latex-project.org/lppl/lppl-1-3c.txt
and the latest version of this license is in
   http://www.latex-project.org/lppl.txt
and version 1.3 or later is part of all distributions of
LaTeX version 2005/12/01 or later.

This work has the LPPL maintenance status "maintained".

This Current Maintainer of this work is Heiko Oberdiek.

This work consists of the main source file scrindex.dtx
and the derived files
   scrindex.sty, scrindex.pdf, scrindex.ins, scrindex.drv,
   scrindex-example1.tex, scrindex-example2.tex.

\endpreamble
\let\MetaPrefix\DoubleperCent

\generate{%
  \file{scrindex.ins}{\from{scrindex.dtx}{install}}%
  \file{scrindex.drv}{\from{scrindex.dtx}{driver}}%
  \usedir{tex/latex/oberdiek}%
  \file{scrindex.sty}{\from{scrindex.dtx}{package}}%
  \usedir{doc/latex/oberdiek}%
  \file{scrindex-example1.tex}{\from{scrindex.dtx}{example1}}%
  \file{scrindex-example2.tex}{\from{scrindex.dtx}{example2}}%
  \nopreamble
  \nopostamble
  \usedir{source/latex/oberdiek/catalogue}%
  \file{scrindex.xml}{\from{scrindex.dtx}{catalogue}}%
}

\catcode32=13\relax% active space
\let =\space%
\Msg{************************************************************************}
\Msg{*}
\Msg{* To finish the installation you have to move the following}
\Msg{* file into a directory searched by TeX:}
\Msg{*}
\Msg{*     scrindex.sty}
\Msg{*}
\Msg{* To produce the documentation run the file `scrindex.drv'}
\Msg{* through LaTeX.}
\Msg{*}
\Msg{* Happy TeXing!}
\Msg{*}
\Msg{************************************************************************}

\endbatchfile
%</install>
%<*ignore>
\fi
%</ignore>
%<*driver>
\NeedsTeXFormat{LaTeX2e}
\ProvidesFile{scrindex.drv}%
  [2008/08/11 v1.1 Package index with KOMA-Script classes (HO)]%
\documentclass{ltxdoc}
\usepackage{holtxdoc}[2011/11/22]
\usepackage{calc}
\begin{document}
  \DocInput{scrindex.dtx}%
\end{document}
%</driver>
% \fi
%
% \CheckSum{237}
%
% \CharacterTable
%  {Upper-case    \A\B\C\D\E\F\G\H\I\J\K\L\M\N\O\P\Q\R\S\T\U\V\W\X\Y\Z
%   Lower-case    \a\b\c\d\e\f\g\h\i\j\k\l\m\n\o\p\q\r\s\t\u\v\w\x\y\z
%   Digits        \0\1\2\3\4\5\6\7\8\9
%   Exclamation   \!     Double quote  \"     Hash (number) \#
%   Dollar        \$     Percent       \%     Ampersand     \&
%   Acute accent  \'     Left paren    \(     Right paren   \)
%   Asterisk      \*     Plus          \+     Comma         \,
%   Minus         \-     Point         \.     Solidus       \/
%   Colon         \:     Semicolon     \;     Less than     \<
%   Equals        \=     Greater than  \>     Question mark \?
%   Commercial at \@     Left bracket  \[     Backslash     \\
%   Right bracket \]     Circumflex    \^     Underscore    \_
%   Grave accent  \`     Left brace    \{     Vertical bar  \|
%   Right brace   \}     Tilde         \~}
%
% \GetFileInfo{scrindex.drv}
%
% \title{The \xpackage{scrindex} package}
% \date{2008/08/11 v1.1}
% \author{Heiko Oberdiek\\\xemail{heiko.oberdiek at googlemail.com}}
%
% \maketitle
%
% \begin{abstract}
% This package redefines environment `theindex' of package \xpackage{index},
% if a class from KOMA-Script is loaded. Also option \xoption{idxtotoc}
% is supported. Index preambles can be given either by means of package
% \xpackage{index} or via the interface provided by KOMA-Script.
% \end{abstract}
%
% \tableofcontents
%
% \section{Documentation}
%
% Package \xpackage{index}, written by David M.\ Jones, detects
% the standard classes |article|, |report|, and |book|. It
% redefines environment `theindex' for its needs.
% However, it does not know other classes such as KOMA-Script.
% This package closes the compatibiliy gap between KOMA-Script's
% classes and package \xpackage{index}.
%
% Environment |theindex| is redefined to support both package
% \xpackage{index} and KOMA-Script's classes. Thus both
% the prologe of package \xpackage{index} and the preamble
% of KOMA-Script's classes are available. Also class option |idxtotoc|
% of KOMA-Script is supported.
%
% \subsection{Usage}
%
% The package \xpackage{scrindex} is loaded without options:
%\begin{quote}
%\begin{verbatim}
%\usepackage{scrindex}
%\end{verbatim}
%\end{quote}
%
% It loads package \xpackage{index} and requests version 2004/01/20
% or later. \LaTeX's package interface allows multiple calls
% of the same package. The package is loaded at its first
% package loading command. At later times \LaTeX\ only checks
% options and a requested version date. Therefore it does not harm
% to add |\usepackage{index}| before or after |\usepackage{scrindex}|.
%
% Also the class does not matter. Environment |theindex| is only
% redefined for a supported class:
% \begin{itemize}
% \item |scrartcl|
% \item |scrreprt|
% \item |scrbook|
% \end{itemize}
%
% \subsection{Preambles}
%
% Both the prologue of package \xpackage{index} and the preamble
% of KOMA-Script's classes are supported. The position depends on
% the class.
%
% \subsubsection{Class \xclass{scrartcl}}
%
%    \begin{macrocode}
%<*example1>
\documentclass{scrartcl}
\usepackage{scrindex}
\setindexpreamble{Preamble of \texttt{scrartcl}\dotfill EOL}
\makeindex
\begin{document}
\section{First Section}
\index{first}
\index{section}
\printindex[default]%
  [Prologue of package \texttt{index}\dotfill EOL]%
\end{document}
%</example1>
%    \end{macrocode}
% The prologue of package \xpackage{index} is first set straight
% after the section title spanning both columns.
% Then the preamble of KOMA-Script follows
% in the first left column.
%
% \medskip
% \begin{quote}
%   \renewcommand*{\arraystretch}{1.2}
%   \begin{tabular}{|p{.45\linewidth}|p{.45\linewidth}|}
%   \hline
%   \multicolumn{2}{|l|}{\textbf{Index}}\\[1ex]
%   \multicolumn{2}{|p{.9\linewidth+2\tabcolsep}|}{^^A
%     Prologue of package \texttt{index}\dotfill EOL^^A
%   }\\[1ex]
%   \hline
%   Preamble of \texttt{scrartcl}\dotfill EOL&\\
%   first, 1&\\
%   section, 1&\\
%   \hline
%   \end{tabular}
% \end{quote}
%
% \subsubsection{Classes \xclass{scrreprt} and \xclass{scrbook}}
%
%    \begin{macrocode}
%<*example2>
\documentclass[openany]{scrbook}% or scrreprt
\usepackage{scrindex}
\setindexpreamble{Preamble of class \texttt{scrbook}\dotfill EOL}
\makeindex
\begin{document}
\chapter{First Chapter}
\index{first}
\index{chapter}
\printindex[default]%
  [Prologue of package \texttt{index}\dotfill EOL]%
\end{document}
%</example2>
%    \end{macrocode}
% The order of the two preambles are different for the classes
% \xclass{scrreprt} and \xclass{scrbook}. First KOMA-Script's
% chapter preamble is set, then the prologue of package \xpackage{index}
% follows. Both are set spanning both columns.
%
% \medskip
% \begin{quote}
%   \renewcommand*{\arraystretch}{1.2}
%   \begin{tabular}{|p{.45\linewidth}|p{.45\linewidth}|}
%   \hline
%   \multicolumn{2}{|l|}{\textbf{Index}}\\[1ex]
%   \multicolumn{2}{|p{.9\linewidth+2\tabcolsep}|}{^^A
%     Preamble of class \texttt{scrbook}\dotfill EOL^^A
%   }\\
%   \multicolumn{2}{|p{.9\linewidth+2\tabcolsep}|}{^^A
%     Prologue of package \xpackage{index}\dotfill EOL^^A
%   }\\[1ex]
%   \hline
%   chapter, 1&\\
%   first, 1&\\
%   \hline
%   \end{tabular}
% \end{quote}
%
% \StopEventually{
% }
%
% \section{Implementation}
%
%    \begin{macrocode}
%<*package>
\NeedsTeXFormat{LaTeX2e}
\ProvidesPackage{scrindex}
  [2008/08/11 v1.1 Package index with KOMA-Script classes (HO)]%
%    \end{macrocode}
%
%    \begin{macrocode}
\RequirePackage{index}[2004/01/20]%
%    \end{macrocode}
%
%    \begin{macrocode}
\@ifclassloaded{scrartcl}{%
  \renewenvironment{theindex}{%
    \edef\indexname{%
      \the\@nameuse{idxtitle@\@indextype}%
    }%
    \if@twocolumn
      \@restonecolfalse
    \else
      \@restonecoltrue
    \fi
    \idx@heading
    \thispagestyle{\indexpagestyle}%
    \columnseprule\z@
    \columnsep 35\p@
    \index@preamble\par\nobreak
    \parindent\z@
    \parskip\z@ \@plus .3\p@\relax
    \parfillskip\z@ \@plus 1fil\relax
    \let\item\@idxitem
  }{%
    \if@restonecol
      \onecolumn
    \else
      \clearpage
    \fi
  }%
  \@ifclasswith{scrartcl}{idxtotoc}{%
    \renewcommand*{\idx@heading}{%
      \twocolumn[%
        \addsec{\indexname}%
        \ifx\index@prologue\@empty
        \else
          \index@prologue
          \bigskip
        \fi
      ]%
      \@mkboth{\indexname}{\indexname}%
    }%
  }{%
    \renewcommand*{\idx@heading}{%
      \twocolumn[%
        \section*{\indexname}%
        \ifx\index@prologue\@empty
        \else
          \index@prologue
          \bigskip
        \fi
      ]%
      \@mkboth{\indexname}{\indexname}%
    }%
  }%
}{}
%    \end{macrocode}
%    \begin{macrocode}
\@ifclassloaded{scrreprt}{%
  \renewenvironment{theindex}{%
    \edef\indexname{%
      \the\@nameuse{idxtitle@\@indextype}%
    }%
    \if@twocolumn
      \@restonecolfalse
    \else
      \@restonecoltrue
    \fi
    \setchapterpreamble{\index@preamble}%
    \idx@heading
    \thispagestyle{\indexpagestyle}%
    \columnseprule\z@
    \columnsep 35\p@
    \parindent\z@
    \parskip\z@ \@plus .3\p@\relax
    \parfillskip\z@ \@plus 1fil\relax
    \let\item\@idxitem
  }{%
    \if@restonecol
      \onecolumn
    \else
      \clearpage
    \fi
  }%
  \@ifclasswith{scrreprt}{idxtotoc}{%
    \renewcommand*{\idx@heading}{%
      \if@openright
        \cleardoublepage
      \else
        \clearpage
      \fi
      \twocolumn[%
        \addchap{\indexname}%
        \ifx\index@prologue\@empty
        \else
          \index@prologue
          \bigskip
        \fi
      ]%
      \@mkboth{\indexname}{\indexname}%
    }%
  }{%
    \renewcommand*{\idx@heading}{%
      \if@openright
        \cleardoublepage
      \else
        \clearpage
      \fi
      \twocolumn[%
        \chapter*{\indexname}%
        \ifx\index@prologue\@empty
        \else
          \index@prologue
          \bigskip
        \fi
      ]%
      \@mkboth{\indexname}{\indexname}%
    }%
  }%
}{}
%    \end{macrocode}
%    \begin{macrocode}
\@ifclassloaded{scrbook}{%
  \renewenvironment{theindex}{%
    \edef\indexname{%
      \the\@nameuse{idxtitle@\@indextype}%
    }%
    \if@twocolumn
      \@restonecolfalse
    \else
      \@restonecoltrue
    \fi
    \setchapterpreamble{\index@preamble}%
    \idx@heading
    \thispagestyle{\indexpagestyle}%
    \columnseprule\z@
    \columnsep 35\p@
    \parindent\z@
    \parskip\z@ \@plus .3\p@\relax
    \parfillskip\z@ \@plus 1fil\relax
    \let\item\@idxitem
  }{%
    \if@restonecol
      \onecolumn
    \else
      \clearpage
    \fi
  }%
  \@ifclasswith{scrbook}{idxtotoc}{%
    \renewcommand*{\idx@heading}{%
      \if@openright
        \cleardoublepage
      \else
        \clearpage
      \fi
      \twocolumn[%
        \addchap{\indexname}%
        \ifx\index@prologue\@empty
        \else
          \index@prologue
          \bigskip
        \fi
      ]%
      \@mkboth{\indexname}{\indexname}%
    }%
  }{%
    \renewcommand*{\idx@heading}{%
      \if@openright
        \cleardoublepage
      \else
        \clearpage
      \fi
      \twocolumn[%
        \chapter*{\indexname}%
        \ifx\index@prologue\@empty
        \else
          \index@prologue
          \bigskip
        \fi
      ]%
      \@mkboth{\indexname}{\indexname}%
    }%
  }%
}{}
%    \end{macrocode}
%
%    \begin{macrocode}
%</package>
%    \end{macrocode}
%
% \section{Installation}
%
% \subsection{Download}
%
% \paragraph{Package.} This package is available on
% CTAN\footnote{\url{ftp://ftp.ctan.org/tex-archive/}}:
% \begin{description}
% \item[\CTAN{macros/latex/contrib/oberdiek/scrindex.dtx}] The source file.
% \item[\CTAN{macros/latex/contrib/oberdiek/scrindex.pdf}] Documentation.
% \end{description}
%
%
% \paragraph{Bundle.} All the packages of the bundle `oberdiek'
% are also available in a TDS compliant ZIP archive. There
% the packages are already unpacked and the documentation files
% are generated. The files and directories obey the TDS standard.
% \begin{description}
% \item[\CTAN{install/macros/latex/contrib/oberdiek.tds.zip}]
% \end{description}
% \emph{TDS} refers to the standard ``A Directory Structure
% for \TeX\ Files'' (\CTAN{tds/tds.pdf}). Directories
% with \xfile{texmf} in their name are usually organized this way.
%
% \subsection{Bundle installation}
%
% \paragraph{Unpacking.} Unpack the \xfile{oberdiek.tds.zip} in the
% TDS tree (also known as \xfile{texmf} tree) of your choice.
% Example (linux):
% \begin{quote}
%   |unzip oberdiek.tds.zip -d ~/texmf|
% \end{quote}
%
% \paragraph{Script installation.}
% Check the directory \xfile{TDS:scripts/oberdiek/} for
% scripts that need further installation steps.
% Package \xpackage{attachfile2} comes with the Perl script
% \xfile{pdfatfi.pl} that should be installed in such a way
% that it can be called as \texttt{pdfatfi}.
% Example (linux):
% \begin{quote}
%   |chmod +x scripts/oberdiek/pdfatfi.pl|\\
%   |cp scripts/oberdiek/pdfatfi.pl /usr/local/bin/|
% \end{quote}
%
% \subsection{Package installation}
%
% \paragraph{Unpacking.} The \xfile{.dtx} file is a self-extracting
% \docstrip\ archive. The files are extracted by running the
% \xfile{.dtx} through \plainTeX:
% \begin{quote}
%   \verb|tex scrindex.dtx|
% \end{quote}
%
% \paragraph{TDS.} Now the different files must be moved into
% the different directories in your installation TDS tree
% (also known as \xfile{texmf} tree):
% \begin{quote}
% \def\t{^^A
% \begin{tabular}{@{}>{\ttfamily}l@{ $\rightarrow$ }>{\ttfamily}l@{}}
%   scrindex.sty & tex/latex/oberdiek/scrindex.sty\\
%   scrindex.pdf & doc/latex/oberdiek/scrindex.pdf\\
%   scrindex-example1.tex & doc/latex/oberdiek/scrindex-example1.tex\\
%   scrindex-example2.tex & doc/latex/oberdiek/scrindex-example2.tex\\
%   scrindex.dtx & source/latex/oberdiek/scrindex.dtx\\
% \end{tabular}^^A
% }^^A
% \sbox0{\t}^^A
% \ifdim\wd0>\linewidth
%   \begingroup
%     \advance\linewidth by\leftmargin
%     \advance\linewidth by\rightmargin
%   \edef\x{\endgroup
%     \def\noexpand\lw{\the\linewidth}^^A
%   }\x
%   \def\lwbox{^^A
%     \leavevmode
%     \hbox to \linewidth{^^A
%       \kern-\leftmargin\relax
%       \hss
%       \usebox0
%       \hss
%       \kern-\rightmargin\relax
%     }^^A
%   }^^A
%   \ifdim\wd0>\lw
%     \sbox0{\small\t}^^A
%     \ifdim\wd0>\linewidth
%       \ifdim\wd0>\lw
%         \sbox0{\footnotesize\t}^^A
%         \ifdim\wd0>\linewidth
%           \ifdim\wd0>\lw
%             \sbox0{\scriptsize\t}^^A
%             \ifdim\wd0>\linewidth
%               \ifdim\wd0>\lw
%                 \sbox0{\tiny\t}^^A
%                 \ifdim\wd0>\linewidth
%                   \lwbox
%                 \else
%                   \usebox0
%                 \fi
%               \else
%                 \lwbox
%               \fi
%             \else
%               \usebox0
%             \fi
%           \else
%             \lwbox
%           \fi
%         \else
%           \usebox0
%         \fi
%       \else
%         \lwbox
%       \fi
%     \else
%       \usebox0
%     \fi
%   \else
%     \lwbox
%   \fi
% \else
%   \usebox0
% \fi
% \end{quote}
% If you have a \xfile{docstrip.cfg} that configures and enables \docstrip's
% TDS installing feature, then some files can already be in the right
% place, see the documentation of \docstrip.
%
% \subsection{Refresh file name databases}
%
% If your \TeX~distribution
% (\teTeX, \mikTeX, \dots) relies on file name databases, you must refresh
% these. For example, \teTeX\ users run \verb|texhash| or
% \verb|mktexlsr|.
%
% \subsection{Some details for the interested}
%
% \paragraph{Attached source.}
%
% The PDF documentation on CTAN also includes the
% \xfile{.dtx} source file. It can be extracted by
% AcrobatReader 6 or higher. Another option is \textsf{pdftk},
% e.g. unpack the file into the current directory:
% \begin{quote}
%   \verb|pdftk scrindex.pdf unpack_files output .|
% \end{quote}
%
% \paragraph{Unpacking with \LaTeX.}
% The \xfile{.dtx} chooses its action depending on the format:
% \begin{description}
% \item[\plainTeX:] Run \docstrip\ and extract the files.
% \item[\LaTeX:] Generate the documentation.
% \end{description}
% If you insist on using \LaTeX\ for \docstrip\ (really,
% \docstrip\ does not need \LaTeX), then inform the autodetect routine
% about your intention:
% \begin{quote}
%   \verb|latex \let\install=y\input{scrindex.dtx}|
% \end{quote}
% Do not forget to quote the argument according to the demands
% of your shell.
%
% \paragraph{Generating the documentation.}
% You can use both the \xfile{.dtx} or the \xfile{.drv} to generate
% the documentation. The process can be configured by the
% configuration file \xfile{ltxdoc.cfg}. For instance, put this
% line into this file, if you want to have A4 as paper format:
% \begin{quote}
%   \verb|\PassOptionsToClass{a4paper}{article}|
% \end{quote}
% An example follows how to generate the
% documentation with pdf\LaTeX:
% \begin{quote}
%\begin{verbatim}
%pdflatex scrindex.dtx
%makeindex -s gind.ist scrindex.idx
%pdflatex scrindex.dtx
%makeindex -s gind.ist scrindex.idx
%pdflatex scrindex.dtx
%\end{verbatim}
% \end{quote}
%
% \section{Catalogue}
%
% The following XML file can be used as source for the
% \href{http://mirror.ctan.org/help/Catalogue/catalogue.html}{\TeX\ Catalogue}.
% The elements \texttt{caption} and \texttt{description} are imported
% from the original XML file from the Catalogue.
% The name of the XML file in the Catalogue is \xfile{scrindex.xml}.
%    \begin{macrocode}
%<*catalogue>
<?xml version='1.0' encoding='us-ascii'?>
<!DOCTYPE entry SYSTEM 'catalogue.dtd'>
<entry datestamp='$Date$' modifier='$Author$' id='scrindex'>
  <name>scrindex</name>
  <caption>Make index package work with Koma-script classes.</caption>
  <authorref id='auth:oberdiek'/>
  <copyright owner='Heiko Oberdiek' year='2008'/>
  <license type='lppl1.3'/>
  <version number='1.1'/>
  <description>
    This package redefines environment `theindex' of package `index',
    if a class from <xref refid='koma-script'>KOMA-Script</xref> is loaded.
    Also option `idxtotoc' is supported. Index preambles can be given
    either by means of package `index' or via the interface provided
    by <xref refid='koma-script'>KOMA-Script</xref>.
    <p/>
    The package is part of the <xref refid='oberdiek'>oberdiek</xref>
    bundle.
  </description>
  <documentation details='Package documentation'
      href='ctan:/macros/latex/contrib/oberdiek/scrindex.pdf'/>
  <ctan file='true' path='/macros/latex/contrib/oberdiek/scrindex.dtx'/>
  <miktex location='oberdiek'/>
  <texlive location='oberdiek'/>
  <install path='/macros/latex/contrib/oberdiek/oberdiek.tds.zip'/>
</entry>
%</catalogue>
%    \end{macrocode}
%
% \begin{History}
%   \begin{Version}{2008/07/07 v1.0}
%   \item
%     First version, also published in newsgroup \xnewsgroup{de.comp.text.tex}:\\
%     \URL{``\link{Re: Z\"ahler bei \cs{index}}''}^^A
%     {http://groups.google.com/group/de.comp.text.tex/msg/39575b5e2f29be1e}
%   \end{Version}
%   \begin{Version}{2008/08/11 v1.1}
%   \item
%     Code is not changed.
%   \item
%     URLs updated.
%   \end{Version}
% \end{History}
%
% \PrintIndex
%
% \Finale
\endinput
|
% \end{quote}
% Do not forget to quote the argument according to the demands
% of your shell.
%
% \paragraph{Generating the documentation.}
% You can use both the \xfile{.dtx} or the \xfile{.drv} to generate
% the documentation. The process can be configured by the
% configuration file \xfile{ltxdoc.cfg}. For instance, put this
% line into this file, if you want to have A4 as paper format:
% \begin{quote}
%   \verb|\PassOptionsToClass{a4paper}{article}|
% \end{quote}
% An example follows how to generate the
% documentation with pdf\LaTeX:
% \begin{quote}
%\begin{verbatim}
%pdflatex scrindex.dtx
%makeindex -s gind.ist scrindex.idx
%pdflatex scrindex.dtx
%makeindex -s gind.ist scrindex.idx
%pdflatex scrindex.dtx
%\end{verbatim}
% \end{quote}
%
% \section{Catalogue}
%
% The following XML file can be used as source for the
% \href{http://mirror.ctan.org/help/Catalogue/catalogue.html}{\TeX\ Catalogue}.
% The elements \texttt{caption} and \texttt{description} are imported
% from the original XML file from the Catalogue.
% The name of the XML file in the Catalogue is \xfile{scrindex.xml}.
%    \begin{macrocode}
%<*catalogue>
<?xml version='1.0' encoding='us-ascii'?>
<!DOCTYPE entry SYSTEM 'catalogue.dtd'>
<entry datestamp='$Date$' modifier='$Author$' id='scrindex'>
  <name>scrindex</name>
  <caption>Make index package work with Koma-script classes.</caption>
  <authorref id='auth:oberdiek'/>
  <copyright owner='Heiko Oberdiek' year='2008'/>
  <license type='lppl1.3'/>
  <version number='1.1'/>
  <description>
    This package redefines environment `theindex' of package `index',
    if a class from <xref refid='koma-script'>KOMA-Script</xref> is loaded.
    Also option `idxtotoc' is supported. Index preambles can be given
    either by means of package `index' or via the interface provided
    by <xref refid='koma-script'>KOMA-Script</xref>.
    <p/>
    The package is part of the <xref refid='oberdiek'>oberdiek</xref>
    bundle.
  </description>
  <documentation details='Package documentation'
      href='ctan:/macros/latex/contrib/oberdiek/scrindex.pdf'/>
  <ctan file='true' path='/macros/latex/contrib/oberdiek/scrindex.dtx'/>
  <miktex location='oberdiek'/>
  <texlive location='oberdiek'/>
  <install path='/macros/latex/contrib/oberdiek/oberdiek.tds.zip'/>
</entry>
%</catalogue>
%    \end{macrocode}
%
% \begin{History}
%   \begin{Version}{2008/07/07 v1.0}
%   \item
%     First version, also published in newsgroup \xnewsgroup{de.comp.text.tex}:\\
%     \URL{``\link{Re: Z\"ahler bei \cs{index}}''}^^A
%     {http://groups.google.com/group/de.comp.text.tex/msg/39575b5e2f29be1e}
%   \end{Version}
%   \begin{Version}{2008/08/11 v1.1}
%   \item
%     Code is not changed.
%   \item
%     URLs updated.
%   \end{Version}
% \end{History}
%
% \PrintIndex
%
% \Finale
\endinput

%        (quote the arguments according to the demands of your shell)
%
% Documentation:
%    (a) If scrindex.drv is present:
%           latex scrindex.drv
%    (b) Without scrindex.drv:
%           latex scrindex.dtx; ...
%    The class ltxdoc loads the configuration file ltxdoc.cfg
%    if available. Here you can specify further options, e.g.
%    use A4 as paper format:
%       \PassOptionsToClass{a4paper}{article}
%
%    Programm calls to get the documentation (example):
%       pdflatex scrindex.dtx
%       makeindex -s gind.ist scrindex.idx
%       pdflatex scrindex.dtx
%       makeindex -s gind.ist scrindex.idx
%       pdflatex scrindex.dtx
%
% Installation:
%    TDS:tex/latex/oberdiek/scrindex.sty
%    TDS:doc/latex/oberdiek/scrindex.pdf
%    TDS:doc/latex/oberdiek/scrindex-example1.tex
%    TDS:doc/latex/oberdiek/scrindex-example2.tex
%    TDS:source/latex/oberdiek/scrindex.dtx
%
%<*ignore>
\begingroup
  \catcode123=1 %
  \catcode125=2 %
  \def\x{LaTeX2e}%
\expandafter\endgroup
\ifcase 0\ifx\install y1\fi\expandafter
         \ifx\csname processbatchFile\endcsname\relax\else1\fi
         \ifx\fmtname\x\else 1\fi\relax
\else\csname fi\endcsname
%</ignore>
%<*install>
\input docstrip.tex
\Msg{************************************************************************}
\Msg{* Installation}
\Msg{* Package: scrindex 2008/08/11 v1.1 Package index with KOMA-Script classes (HO)}
\Msg{************************************************************************}

\keepsilent
\askforoverwritefalse

\let\MetaPrefix\relax
\preamble

This is a generated file.

Project: scrindex
Version: 2008/08/11 v1.1

Copyright (C) 2008 by
   Heiko Oberdiek <heiko.oberdiek at googlemail.com>

This work may be distributed and/or modified under the
conditions of the LaTeX Project Public License, either
version 1.3c of this license or (at your option) any later
version. This version of this license is in
   http://www.latex-project.org/lppl/lppl-1-3c.txt
and the latest version of this license is in
   http://www.latex-project.org/lppl.txt
and version 1.3 or later is part of all distributions of
LaTeX version 2005/12/01 or later.

This work has the LPPL maintenance status "maintained".

This Current Maintainer of this work is Heiko Oberdiek.

This work consists of the main source file scrindex.dtx
and the derived files
   scrindex.sty, scrindex.pdf, scrindex.ins, scrindex.drv,
   scrindex-example1.tex, scrindex-example2.tex.

\endpreamble
\let\MetaPrefix\DoubleperCent

\generate{%
  \file{scrindex.ins}{\from{scrindex.dtx}{install}}%
  \file{scrindex.drv}{\from{scrindex.dtx}{driver}}%
  \usedir{tex/latex/oberdiek}%
  \file{scrindex.sty}{\from{scrindex.dtx}{package}}%
  \usedir{doc/latex/oberdiek}%
  \file{scrindex-example1.tex}{\from{scrindex.dtx}{example1}}%
  \file{scrindex-example2.tex}{\from{scrindex.dtx}{example2}}%
  \nopreamble
  \nopostamble
  \usedir{source/latex/oberdiek/catalogue}%
  \file{scrindex.xml}{\from{scrindex.dtx}{catalogue}}%
}

\catcode32=13\relax% active space
\let =\space%
\Msg{************************************************************************}
\Msg{*}
\Msg{* To finish the installation you have to move the following}
\Msg{* file into a directory searched by TeX:}
\Msg{*}
\Msg{*     scrindex.sty}
\Msg{*}
\Msg{* To produce the documentation run the file `scrindex.drv'}
\Msg{* through LaTeX.}
\Msg{*}
\Msg{* Happy TeXing!}
\Msg{*}
\Msg{************************************************************************}

\endbatchfile
%</install>
%<*ignore>
\fi
%</ignore>
%<*driver>
\NeedsTeXFormat{LaTeX2e}
\ProvidesFile{scrindex.drv}%
  [2008/08/11 v1.1 Package index with KOMA-Script classes (HO)]%
\documentclass{ltxdoc}
\usepackage{holtxdoc}[2011/11/22]
\usepackage{calc}
\begin{document}
  \DocInput{scrindex.dtx}%
\end{document}
%</driver>
% \fi
%
% \CheckSum{237}
%
% \CharacterTable
%  {Upper-case    \A\B\C\D\E\F\G\H\I\J\K\L\M\N\O\P\Q\R\S\T\U\V\W\X\Y\Z
%   Lower-case    \a\b\c\d\e\f\g\h\i\j\k\l\m\n\o\p\q\r\s\t\u\v\w\x\y\z
%   Digits        \0\1\2\3\4\5\6\7\8\9
%   Exclamation   \!     Double quote  \"     Hash (number) \#
%   Dollar        \$     Percent       \%     Ampersand     \&
%   Acute accent  \'     Left paren    \(     Right paren   \)
%   Asterisk      \*     Plus          \+     Comma         \,
%   Minus         \-     Point         \.     Solidus       \/
%   Colon         \:     Semicolon     \;     Less than     \<
%   Equals        \=     Greater than  \>     Question mark \?
%   Commercial at \@     Left bracket  \[     Backslash     \\
%   Right bracket \]     Circumflex    \^     Underscore    \_
%   Grave accent  \`     Left brace    \{     Vertical bar  \|
%   Right brace   \}     Tilde         \~}
%
% \GetFileInfo{scrindex.drv}
%
% \title{The \xpackage{scrindex} package}
% \date{2008/08/11 v1.1}
% \author{Heiko Oberdiek\\\xemail{heiko.oberdiek at googlemail.com}}
%
% \maketitle
%
% \begin{abstract}
% This package redefines environment `theindex' of package \xpackage{index},
% if a class from KOMA-Script is loaded. Also option \xoption{idxtotoc}
% is supported. Index preambles can be given either by means of package
% \xpackage{index} or via the interface provided by KOMA-Script.
% \end{abstract}
%
% \tableofcontents
%
% \section{Documentation}
%
% Package \xpackage{index}, written by David M.\ Jones, detects
% the standard classes |article|, |report|, and |book|. It
% redefines environment `theindex' for its needs.
% However, it does not know other classes such as KOMA-Script.
% This package closes the compatibiliy gap between KOMA-Script's
% classes and package \xpackage{index}.
%
% Environment |theindex| is redefined to support both package
% \xpackage{index} and KOMA-Script's classes. Thus both
% the prologe of package \xpackage{index} and the preamble
% of KOMA-Script's classes are available. Also class option |idxtotoc|
% of KOMA-Script is supported.
%
% \subsection{Usage}
%
% The package \xpackage{scrindex} is loaded without options:
%\begin{quote}
%\begin{verbatim}
%\usepackage{scrindex}
%\end{verbatim}
%\end{quote}
%
% It loads package \xpackage{index} and requests version 2004/01/20
% or later. \LaTeX's package interface allows multiple calls
% of the same package. The package is loaded at its first
% package loading command. At later times \LaTeX\ only checks
% options and a requested version date. Therefore it does not harm
% to add |\usepackage{index}| before or after |\usepackage{scrindex}|.
%
% Also the class does not matter. Environment |theindex| is only
% redefined for a supported class:
% \begin{itemize}
% \item |scrartcl|
% \item |scrreprt|
% \item |scrbook|
% \end{itemize}
%
% \subsection{Preambles}
%
% Both the prologue of package \xpackage{index} and the preamble
% of KOMA-Script's classes are supported. The position depends on
% the class.
%
% \subsubsection{Class \xclass{scrartcl}}
%
%    \begin{macrocode}
%<*example1>
\documentclass{scrartcl}
\usepackage{scrindex}
\setindexpreamble{Preamble of \texttt{scrartcl}\dotfill EOL}
\makeindex
\begin{document}
\section{First Section}
\index{first}
\index{section}
\printindex[default]%
  [Prologue of package \texttt{index}\dotfill EOL]%
\end{document}
%</example1>
%    \end{macrocode}
% The prologue of package \xpackage{index} is first set straight
% after the section title spanning both columns.
% Then the preamble of KOMA-Script follows
% in the first left column.
%
% \medskip
% \begin{quote}
%   \renewcommand*{\arraystretch}{1.2}
%   \begin{tabular}{|p{.45\linewidth}|p{.45\linewidth}|}
%   \hline
%   \multicolumn{2}{|l|}{\textbf{Index}}\\[1ex]
%   \multicolumn{2}{|p{.9\linewidth+2\tabcolsep}|}{^^A
%     Prologue of package \texttt{index}\dotfill EOL^^A
%   }\\[1ex]
%   \hline
%   Preamble of \texttt{scrartcl}\dotfill EOL&\\
%   first, 1&\\
%   section, 1&\\
%   \hline
%   \end{tabular}
% \end{quote}
%
% \subsubsection{Classes \xclass{scrreprt} and \xclass{scrbook}}
%
%    \begin{macrocode}
%<*example2>
\documentclass[openany]{scrbook}% or scrreprt
\usepackage{scrindex}
\setindexpreamble{Preamble of class \texttt{scrbook}\dotfill EOL}
\makeindex
\begin{document}
\chapter{First Chapter}
\index{first}
\index{chapter}
\printindex[default]%
  [Prologue of package \texttt{index}\dotfill EOL]%
\end{document}
%</example2>
%    \end{macrocode}
% The order of the two preambles are different for the classes
% \xclass{scrreprt} and \xclass{scrbook}. First KOMA-Script's
% chapter preamble is set, then the prologue of package \xpackage{index}
% follows. Both are set spanning both columns.
%
% \medskip
% \begin{quote}
%   \renewcommand*{\arraystretch}{1.2}
%   \begin{tabular}{|p{.45\linewidth}|p{.45\linewidth}|}
%   \hline
%   \multicolumn{2}{|l|}{\textbf{Index}}\\[1ex]
%   \multicolumn{2}{|p{.9\linewidth+2\tabcolsep}|}{^^A
%     Preamble of class \texttt{scrbook}\dotfill EOL^^A
%   }\\
%   \multicolumn{2}{|p{.9\linewidth+2\tabcolsep}|}{^^A
%     Prologue of package \xpackage{index}\dotfill EOL^^A
%   }\\[1ex]
%   \hline
%   chapter, 1&\\
%   first, 1&\\
%   \hline
%   \end{tabular}
% \end{quote}
%
% \StopEventually{
% }
%
% \section{Implementation}
%
%    \begin{macrocode}
%<*package>
\NeedsTeXFormat{LaTeX2e}
\ProvidesPackage{scrindex}
  [2008/08/11 v1.1 Package index with KOMA-Script classes (HO)]%
%    \end{macrocode}
%
%    \begin{macrocode}
\RequirePackage{index}[2004/01/20]%
%    \end{macrocode}
%
%    \begin{macrocode}
\@ifclassloaded{scrartcl}{%
  \renewenvironment{theindex}{%
    \edef\indexname{%
      \the\@nameuse{idxtitle@\@indextype}%
    }%
    \if@twocolumn
      \@restonecolfalse
    \else
      \@restonecoltrue
    \fi
    \idx@heading
    \thispagestyle{\indexpagestyle}%
    \columnseprule\z@
    \columnsep 35\p@
    \index@preamble\par\nobreak
    \parindent\z@
    \parskip\z@ \@plus .3\p@\relax
    \parfillskip\z@ \@plus 1fil\relax
    \let\item\@idxitem
  }{%
    \if@restonecol
      \onecolumn
    \else
      \clearpage
    \fi
  }%
  \@ifclasswith{scrartcl}{idxtotoc}{%
    \renewcommand*{\idx@heading}{%
      \twocolumn[%
        \addsec{\indexname}%
        \ifx\index@prologue\@empty
        \else
          \index@prologue
          \bigskip
        \fi
      ]%
      \@mkboth{\indexname}{\indexname}%
    }%
  }{%
    \renewcommand*{\idx@heading}{%
      \twocolumn[%
        \section*{\indexname}%
        \ifx\index@prologue\@empty
        \else
          \index@prologue
          \bigskip
        \fi
      ]%
      \@mkboth{\indexname}{\indexname}%
    }%
  }%
}{}
%    \end{macrocode}
%    \begin{macrocode}
\@ifclassloaded{scrreprt}{%
  \renewenvironment{theindex}{%
    \edef\indexname{%
      \the\@nameuse{idxtitle@\@indextype}%
    }%
    \if@twocolumn
      \@restonecolfalse
    \else
      \@restonecoltrue
    \fi
    \setchapterpreamble{\index@preamble}%
    \idx@heading
    \thispagestyle{\indexpagestyle}%
    \columnseprule\z@
    \columnsep 35\p@
    \parindent\z@
    \parskip\z@ \@plus .3\p@\relax
    \parfillskip\z@ \@plus 1fil\relax
    \let\item\@idxitem
  }{%
    \if@restonecol
      \onecolumn
    \else
      \clearpage
    \fi
  }%
  \@ifclasswith{scrreprt}{idxtotoc}{%
    \renewcommand*{\idx@heading}{%
      \if@openright
        \cleardoublepage
      \else
        \clearpage
      \fi
      \twocolumn[%
        \addchap{\indexname}%
        \ifx\index@prologue\@empty
        \else
          \index@prologue
          \bigskip
        \fi
      ]%
      \@mkboth{\indexname}{\indexname}%
    }%
  }{%
    \renewcommand*{\idx@heading}{%
      \if@openright
        \cleardoublepage
      \else
        \clearpage
      \fi
      \twocolumn[%
        \chapter*{\indexname}%
        \ifx\index@prologue\@empty
        \else
          \index@prologue
          \bigskip
        \fi
      ]%
      \@mkboth{\indexname}{\indexname}%
    }%
  }%
}{}
%    \end{macrocode}
%    \begin{macrocode}
\@ifclassloaded{scrbook}{%
  \renewenvironment{theindex}{%
    \edef\indexname{%
      \the\@nameuse{idxtitle@\@indextype}%
    }%
    \if@twocolumn
      \@restonecolfalse
    \else
      \@restonecoltrue
    \fi
    \setchapterpreamble{\index@preamble}%
    \idx@heading
    \thispagestyle{\indexpagestyle}%
    \columnseprule\z@
    \columnsep 35\p@
    \parindent\z@
    \parskip\z@ \@plus .3\p@\relax
    \parfillskip\z@ \@plus 1fil\relax
    \let\item\@idxitem
  }{%
    \if@restonecol
      \onecolumn
    \else
      \clearpage
    \fi
  }%
  \@ifclasswith{scrbook}{idxtotoc}{%
    \renewcommand*{\idx@heading}{%
      \if@openright
        \cleardoublepage
      \else
        \clearpage
      \fi
      \twocolumn[%
        \addchap{\indexname}%
        \ifx\index@prologue\@empty
        \else
          \index@prologue
          \bigskip
        \fi
      ]%
      \@mkboth{\indexname}{\indexname}%
    }%
  }{%
    \renewcommand*{\idx@heading}{%
      \if@openright
        \cleardoublepage
      \else
        \clearpage
      \fi
      \twocolumn[%
        \chapter*{\indexname}%
        \ifx\index@prologue\@empty
        \else
          \index@prologue
          \bigskip
        \fi
      ]%
      \@mkboth{\indexname}{\indexname}%
    }%
  }%
}{}
%    \end{macrocode}
%
%    \begin{macrocode}
%</package>
%    \end{macrocode}
%
% \section{Installation}
%
% \subsection{Download}
%
% \paragraph{Package.} This package is available on
% CTAN\footnote{\url{ftp://ftp.ctan.org/tex-archive/}}:
% \begin{description}
% \item[\CTAN{macros/latex/contrib/oberdiek/scrindex.dtx}] The source file.
% \item[\CTAN{macros/latex/contrib/oberdiek/scrindex.pdf}] Documentation.
% \end{description}
%
%
% \paragraph{Bundle.} All the packages of the bundle `oberdiek'
% are also available in a TDS compliant ZIP archive. There
% the packages are already unpacked and the documentation files
% are generated. The files and directories obey the TDS standard.
% \begin{description}
% \item[\CTAN{install/macros/latex/contrib/oberdiek.tds.zip}]
% \end{description}
% \emph{TDS} refers to the standard ``A Directory Structure
% for \TeX\ Files'' (\CTAN{tds/tds.pdf}). Directories
% with \xfile{texmf} in their name are usually organized this way.
%
% \subsection{Bundle installation}
%
% \paragraph{Unpacking.} Unpack the \xfile{oberdiek.tds.zip} in the
% TDS tree (also known as \xfile{texmf} tree) of your choice.
% Example (linux):
% \begin{quote}
%   |unzip oberdiek.tds.zip -d ~/texmf|
% \end{quote}
%
% \paragraph{Script installation.}
% Check the directory \xfile{TDS:scripts/oberdiek/} for
% scripts that need further installation steps.
% Package \xpackage{attachfile2} comes with the Perl script
% \xfile{pdfatfi.pl} that should be installed in such a way
% that it can be called as \texttt{pdfatfi}.
% Example (linux):
% \begin{quote}
%   |chmod +x scripts/oberdiek/pdfatfi.pl|\\
%   |cp scripts/oberdiek/pdfatfi.pl /usr/local/bin/|
% \end{quote}
%
% \subsection{Package installation}
%
% \paragraph{Unpacking.} The \xfile{.dtx} file is a self-extracting
% \docstrip\ archive. The files are extracted by running the
% \xfile{.dtx} through \plainTeX:
% \begin{quote}
%   \verb|tex scrindex.dtx|
% \end{quote}
%
% \paragraph{TDS.} Now the different files must be moved into
% the different directories in your installation TDS tree
% (also known as \xfile{texmf} tree):
% \begin{quote}
% \def\t{^^A
% \begin{tabular}{@{}>{\ttfamily}l@{ $\rightarrow$ }>{\ttfamily}l@{}}
%   scrindex.sty & tex/latex/oberdiek/scrindex.sty\\
%   scrindex.pdf & doc/latex/oberdiek/scrindex.pdf\\
%   scrindex-example1.tex & doc/latex/oberdiek/scrindex-example1.tex\\
%   scrindex-example2.tex & doc/latex/oberdiek/scrindex-example2.tex\\
%   scrindex.dtx & source/latex/oberdiek/scrindex.dtx\\
% \end{tabular}^^A
% }^^A
% \sbox0{\t}^^A
% \ifdim\wd0>\linewidth
%   \begingroup
%     \advance\linewidth by\leftmargin
%     \advance\linewidth by\rightmargin
%   \edef\x{\endgroup
%     \def\noexpand\lw{\the\linewidth}^^A
%   }\x
%   \def\lwbox{^^A
%     \leavevmode
%     \hbox to \linewidth{^^A
%       \kern-\leftmargin\relax
%       \hss
%       \usebox0
%       \hss
%       \kern-\rightmargin\relax
%     }^^A
%   }^^A
%   \ifdim\wd0>\lw
%     \sbox0{\small\t}^^A
%     \ifdim\wd0>\linewidth
%       \ifdim\wd0>\lw
%         \sbox0{\footnotesize\t}^^A
%         \ifdim\wd0>\linewidth
%           \ifdim\wd0>\lw
%             \sbox0{\scriptsize\t}^^A
%             \ifdim\wd0>\linewidth
%               \ifdim\wd0>\lw
%                 \sbox0{\tiny\t}^^A
%                 \ifdim\wd0>\linewidth
%                   \lwbox
%                 \else
%                   \usebox0
%                 \fi
%               \else
%                 \lwbox
%               \fi
%             \else
%               \usebox0
%             \fi
%           \else
%             \lwbox
%           \fi
%         \else
%           \usebox0
%         \fi
%       \else
%         \lwbox
%       \fi
%     \else
%       \usebox0
%     \fi
%   \else
%     \lwbox
%   \fi
% \else
%   \usebox0
% \fi
% \end{quote}
% If you have a \xfile{docstrip.cfg} that configures and enables \docstrip's
% TDS installing feature, then some files can already be in the right
% place, see the documentation of \docstrip.
%
% \subsection{Refresh file name databases}
%
% If your \TeX~distribution
% (\teTeX, \mikTeX, \dots) relies on file name databases, you must refresh
% these. For example, \teTeX\ users run \verb|texhash| or
% \verb|mktexlsr|.
%
% \subsection{Some details for the interested}
%
% \paragraph{Attached source.}
%
% The PDF documentation on CTAN also includes the
% \xfile{.dtx} source file. It can be extracted by
% AcrobatReader 6 or higher. Another option is \textsf{pdftk},
% e.g. unpack the file into the current directory:
% \begin{quote}
%   \verb|pdftk scrindex.pdf unpack_files output .|
% \end{quote}
%
% \paragraph{Unpacking with \LaTeX.}
% The \xfile{.dtx} chooses its action depending on the format:
% \begin{description}
% \item[\plainTeX:] Run \docstrip\ and extract the files.
% \item[\LaTeX:] Generate the documentation.
% \end{description}
% If you insist on using \LaTeX\ for \docstrip\ (really,
% \docstrip\ does not need \LaTeX), then inform the autodetect routine
% about your intention:
% \begin{quote}
%   \verb|latex \let\install=y% \iffalse meta-comment
%
% File: scrindex.dtx
% Version: 2008/08/11 v1.1
% Info: Package index with KOMA-Script classes
%
% Copyright (C) 2008 by
%    Heiko Oberdiek <heiko.oberdiek at googlemail.com>
%
% This work may be distributed and/or modified under the
% conditions of the LaTeX Project Public License, either
% version 1.3c of this license or (at your option) any later
% version. This version of this license is in
%    http://www.latex-project.org/lppl/lppl-1-3c.txt
% and the latest version of this license is in
%    http://www.latex-project.org/lppl.txt
% and version 1.3 or later is part of all distributions of
% LaTeX version 2005/12/01 or later.
%
% This work has the LPPL maintenance status "maintained".
%
% This Current Maintainer of this work is Heiko Oberdiek.
%
% This work consists of the main source file scrindex.dtx
% and the derived files
%    scrindex.sty, scrindex.pdf, scrindex.ins, scrindex.drv,
%    scrindex-example1.tex, scrindex-example2.tex.
%
% Distribution:
%    CTAN:macros/latex/contrib/oberdiek/scrindex.dtx
%    CTAN:macros/latex/contrib/oberdiek/scrindex.pdf
%
% Unpacking:
%    (a) If scrindex.ins is present:
%           tex scrindex.ins
%    (b) Without scrindex.ins:
%           tex scrindex.dtx
%    (c) If you insist on using LaTeX
%           latex \let\install=y% \iffalse meta-comment
%
% File: scrindex.dtx
% Version: 2008/08/11 v1.1
% Info: Package index with KOMA-Script classes
%
% Copyright (C) 2008 by
%    Heiko Oberdiek <heiko.oberdiek at googlemail.com>
%
% This work may be distributed and/or modified under the
% conditions of the LaTeX Project Public License, either
% version 1.3c of this license or (at your option) any later
% version. This version of this license is in
%    http://www.latex-project.org/lppl/lppl-1-3c.txt
% and the latest version of this license is in
%    http://www.latex-project.org/lppl.txt
% and version 1.3 or later is part of all distributions of
% LaTeX version 2005/12/01 or later.
%
% This work has the LPPL maintenance status "maintained".
%
% This Current Maintainer of this work is Heiko Oberdiek.
%
% This work consists of the main source file scrindex.dtx
% and the derived files
%    scrindex.sty, scrindex.pdf, scrindex.ins, scrindex.drv,
%    scrindex-example1.tex, scrindex-example2.tex.
%
% Distribution:
%    CTAN:macros/latex/contrib/oberdiek/scrindex.dtx
%    CTAN:macros/latex/contrib/oberdiek/scrindex.pdf
%
% Unpacking:
%    (a) If scrindex.ins is present:
%           tex scrindex.ins
%    (b) Without scrindex.ins:
%           tex scrindex.dtx
%    (c) If you insist on using LaTeX
%           latex \let\install=y\input{scrindex.dtx}
%        (quote the arguments according to the demands of your shell)
%
% Documentation:
%    (a) If scrindex.drv is present:
%           latex scrindex.drv
%    (b) Without scrindex.drv:
%           latex scrindex.dtx; ...
%    The class ltxdoc loads the configuration file ltxdoc.cfg
%    if available. Here you can specify further options, e.g.
%    use A4 as paper format:
%       \PassOptionsToClass{a4paper}{article}
%
%    Programm calls to get the documentation (example):
%       pdflatex scrindex.dtx
%       makeindex -s gind.ist scrindex.idx
%       pdflatex scrindex.dtx
%       makeindex -s gind.ist scrindex.idx
%       pdflatex scrindex.dtx
%
% Installation:
%    TDS:tex/latex/oberdiek/scrindex.sty
%    TDS:doc/latex/oberdiek/scrindex.pdf
%    TDS:doc/latex/oberdiek/scrindex-example1.tex
%    TDS:doc/latex/oberdiek/scrindex-example2.tex
%    TDS:source/latex/oberdiek/scrindex.dtx
%
%<*ignore>
\begingroup
  \catcode123=1 %
  \catcode125=2 %
  \def\x{LaTeX2e}%
\expandafter\endgroup
\ifcase 0\ifx\install y1\fi\expandafter
         \ifx\csname processbatchFile\endcsname\relax\else1\fi
         \ifx\fmtname\x\else 1\fi\relax
\else\csname fi\endcsname
%</ignore>
%<*install>
\input docstrip.tex
\Msg{************************************************************************}
\Msg{* Installation}
\Msg{* Package: scrindex 2008/08/11 v1.1 Package index with KOMA-Script classes (HO)}
\Msg{************************************************************************}

\keepsilent
\askforoverwritefalse

\let\MetaPrefix\relax
\preamble

This is a generated file.

Project: scrindex
Version: 2008/08/11 v1.1

Copyright (C) 2008 by
   Heiko Oberdiek <heiko.oberdiek at googlemail.com>

This work may be distributed and/or modified under the
conditions of the LaTeX Project Public License, either
version 1.3c of this license or (at your option) any later
version. This version of this license is in
   http://www.latex-project.org/lppl/lppl-1-3c.txt
and the latest version of this license is in
   http://www.latex-project.org/lppl.txt
and version 1.3 or later is part of all distributions of
LaTeX version 2005/12/01 or later.

This work has the LPPL maintenance status "maintained".

This Current Maintainer of this work is Heiko Oberdiek.

This work consists of the main source file scrindex.dtx
and the derived files
   scrindex.sty, scrindex.pdf, scrindex.ins, scrindex.drv,
   scrindex-example1.tex, scrindex-example2.tex.

\endpreamble
\let\MetaPrefix\DoubleperCent

\generate{%
  \file{scrindex.ins}{\from{scrindex.dtx}{install}}%
  \file{scrindex.drv}{\from{scrindex.dtx}{driver}}%
  \usedir{tex/latex/oberdiek}%
  \file{scrindex.sty}{\from{scrindex.dtx}{package}}%
  \usedir{doc/latex/oberdiek}%
  \file{scrindex-example1.tex}{\from{scrindex.dtx}{example1}}%
  \file{scrindex-example2.tex}{\from{scrindex.dtx}{example2}}%
  \nopreamble
  \nopostamble
  \usedir{source/latex/oberdiek/catalogue}%
  \file{scrindex.xml}{\from{scrindex.dtx}{catalogue}}%
}

\catcode32=13\relax% active space
\let =\space%
\Msg{************************************************************************}
\Msg{*}
\Msg{* To finish the installation you have to move the following}
\Msg{* file into a directory searched by TeX:}
\Msg{*}
\Msg{*     scrindex.sty}
\Msg{*}
\Msg{* To produce the documentation run the file `scrindex.drv'}
\Msg{* through LaTeX.}
\Msg{*}
\Msg{* Happy TeXing!}
\Msg{*}
\Msg{************************************************************************}

\endbatchfile
%</install>
%<*ignore>
\fi
%</ignore>
%<*driver>
\NeedsTeXFormat{LaTeX2e}
\ProvidesFile{scrindex.drv}%
  [2008/08/11 v1.1 Package index with KOMA-Script classes (HO)]%
\documentclass{ltxdoc}
\usepackage{holtxdoc}[2011/11/22]
\usepackage{calc}
\begin{document}
  \DocInput{scrindex.dtx}%
\end{document}
%</driver>
% \fi
%
% \CheckSum{237}
%
% \CharacterTable
%  {Upper-case    \A\B\C\D\E\F\G\H\I\J\K\L\M\N\O\P\Q\R\S\T\U\V\W\X\Y\Z
%   Lower-case    \a\b\c\d\e\f\g\h\i\j\k\l\m\n\o\p\q\r\s\t\u\v\w\x\y\z
%   Digits        \0\1\2\3\4\5\6\7\8\9
%   Exclamation   \!     Double quote  \"     Hash (number) \#
%   Dollar        \$     Percent       \%     Ampersand     \&
%   Acute accent  \'     Left paren    \(     Right paren   \)
%   Asterisk      \*     Plus          \+     Comma         \,
%   Minus         \-     Point         \.     Solidus       \/
%   Colon         \:     Semicolon     \;     Less than     \<
%   Equals        \=     Greater than  \>     Question mark \?
%   Commercial at \@     Left bracket  \[     Backslash     \\
%   Right bracket \]     Circumflex    \^     Underscore    \_
%   Grave accent  \`     Left brace    \{     Vertical bar  \|
%   Right brace   \}     Tilde         \~}
%
% \GetFileInfo{scrindex.drv}
%
% \title{The \xpackage{scrindex} package}
% \date{2008/08/11 v1.1}
% \author{Heiko Oberdiek\\\xemail{heiko.oberdiek at googlemail.com}}
%
% \maketitle
%
% \begin{abstract}
% This package redefines environment `theindex' of package \xpackage{index},
% if a class from KOMA-Script is loaded. Also option \xoption{idxtotoc}
% is supported. Index preambles can be given either by means of package
% \xpackage{index} or via the interface provided by KOMA-Script.
% \end{abstract}
%
% \tableofcontents
%
% \section{Documentation}
%
% Package \xpackage{index}, written by David M.\ Jones, detects
% the standard classes |article|, |report|, and |book|. It
% redefines environment `theindex' for its needs.
% However, it does not know other classes such as KOMA-Script.
% This package closes the compatibiliy gap between KOMA-Script's
% classes and package \xpackage{index}.
%
% Environment |theindex| is redefined to support both package
% \xpackage{index} and KOMA-Script's classes. Thus both
% the prologe of package \xpackage{index} and the preamble
% of KOMA-Script's classes are available. Also class option |idxtotoc|
% of KOMA-Script is supported.
%
% \subsection{Usage}
%
% The package \xpackage{scrindex} is loaded without options:
%\begin{quote}
%\begin{verbatim}
%\usepackage{scrindex}
%\end{verbatim}
%\end{quote}
%
% It loads package \xpackage{index} and requests version 2004/01/20
% or later. \LaTeX's package interface allows multiple calls
% of the same package. The package is loaded at its first
% package loading command. At later times \LaTeX\ only checks
% options and a requested version date. Therefore it does not harm
% to add |\usepackage{index}| before or after |\usepackage{scrindex}|.
%
% Also the class does not matter. Environment |theindex| is only
% redefined for a supported class:
% \begin{itemize}
% \item |scrartcl|
% \item |scrreprt|
% \item |scrbook|
% \end{itemize}
%
% \subsection{Preambles}
%
% Both the prologue of package \xpackage{index} and the preamble
% of KOMA-Script's classes are supported. The position depends on
% the class.
%
% \subsubsection{Class \xclass{scrartcl}}
%
%    \begin{macrocode}
%<*example1>
\documentclass{scrartcl}
\usepackage{scrindex}
\setindexpreamble{Preamble of \texttt{scrartcl}\dotfill EOL}
\makeindex
\begin{document}
\section{First Section}
\index{first}
\index{section}
\printindex[default]%
  [Prologue of package \texttt{index}\dotfill EOL]%
\end{document}
%</example1>
%    \end{macrocode}
% The prologue of package \xpackage{index} is first set straight
% after the section title spanning both columns.
% Then the preamble of KOMA-Script follows
% in the first left column.
%
% \medskip
% \begin{quote}
%   \renewcommand*{\arraystretch}{1.2}
%   \begin{tabular}{|p{.45\linewidth}|p{.45\linewidth}|}
%   \hline
%   \multicolumn{2}{|l|}{\textbf{Index}}\\[1ex]
%   \multicolumn{2}{|p{.9\linewidth+2\tabcolsep}|}{^^A
%     Prologue of package \texttt{index}\dotfill EOL^^A
%   }\\[1ex]
%   \hline
%   Preamble of \texttt{scrartcl}\dotfill EOL&\\
%   first, 1&\\
%   section, 1&\\
%   \hline
%   \end{tabular}
% \end{quote}
%
% \subsubsection{Classes \xclass{scrreprt} and \xclass{scrbook}}
%
%    \begin{macrocode}
%<*example2>
\documentclass[openany]{scrbook}% or scrreprt
\usepackage{scrindex}
\setindexpreamble{Preamble of class \texttt{scrbook}\dotfill EOL}
\makeindex
\begin{document}
\chapter{First Chapter}
\index{first}
\index{chapter}
\printindex[default]%
  [Prologue of package \texttt{index}\dotfill EOL]%
\end{document}
%</example2>
%    \end{macrocode}
% The order of the two preambles are different for the classes
% \xclass{scrreprt} and \xclass{scrbook}. First KOMA-Script's
% chapter preamble is set, then the prologue of package \xpackage{index}
% follows. Both are set spanning both columns.
%
% \medskip
% \begin{quote}
%   \renewcommand*{\arraystretch}{1.2}
%   \begin{tabular}{|p{.45\linewidth}|p{.45\linewidth}|}
%   \hline
%   \multicolumn{2}{|l|}{\textbf{Index}}\\[1ex]
%   \multicolumn{2}{|p{.9\linewidth+2\tabcolsep}|}{^^A
%     Preamble of class \texttt{scrbook}\dotfill EOL^^A
%   }\\
%   \multicolumn{2}{|p{.9\linewidth+2\tabcolsep}|}{^^A
%     Prologue of package \xpackage{index}\dotfill EOL^^A
%   }\\[1ex]
%   \hline
%   chapter, 1&\\
%   first, 1&\\
%   \hline
%   \end{tabular}
% \end{quote}
%
% \StopEventually{
% }
%
% \section{Implementation}
%
%    \begin{macrocode}
%<*package>
\NeedsTeXFormat{LaTeX2e}
\ProvidesPackage{scrindex}
  [2008/08/11 v1.1 Package index with KOMA-Script classes (HO)]%
%    \end{macrocode}
%
%    \begin{macrocode}
\RequirePackage{index}[2004/01/20]%
%    \end{macrocode}
%
%    \begin{macrocode}
\@ifclassloaded{scrartcl}{%
  \renewenvironment{theindex}{%
    \edef\indexname{%
      \the\@nameuse{idxtitle@\@indextype}%
    }%
    \if@twocolumn
      \@restonecolfalse
    \else
      \@restonecoltrue
    \fi
    \idx@heading
    \thispagestyle{\indexpagestyle}%
    \columnseprule\z@
    \columnsep 35\p@
    \index@preamble\par\nobreak
    \parindent\z@
    \parskip\z@ \@plus .3\p@\relax
    \parfillskip\z@ \@plus 1fil\relax
    \let\item\@idxitem
  }{%
    \if@restonecol
      \onecolumn
    \else
      \clearpage
    \fi
  }%
  \@ifclasswith{scrartcl}{idxtotoc}{%
    \renewcommand*{\idx@heading}{%
      \twocolumn[%
        \addsec{\indexname}%
        \ifx\index@prologue\@empty
        \else
          \index@prologue
          \bigskip
        \fi
      ]%
      \@mkboth{\indexname}{\indexname}%
    }%
  }{%
    \renewcommand*{\idx@heading}{%
      \twocolumn[%
        \section*{\indexname}%
        \ifx\index@prologue\@empty
        \else
          \index@prologue
          \bigskip
        \fi
      ]%
      \@mkboth{\indexname}{\indexname}%
    }%
  }%
}{}
%    \end{macrocode}
%    \begin{macrocode}
\@ifclassloaded{scrreprt}{%
  \renewenvironment{theindex}{%
    \edef\indexname{%
      \the\@nameuse{idxtitle@\@indextype}%
    }%
    \if@twocolumn
      \@restonecolfalse
    \else
      \@restonecoltrue
    \fi
    \setchapterpreamble{\index@preamble}%
    \idx@heading
    \thispagestyle{\indexpagestyle}%
    \columnseprule\z@
    \columnsep 35\p@
    \parindent\z@
    \parskip\z@ \@plus .3\p@\relax
    \parfillskip\z@ \@plus 1fil\relax
    \let\item\@idxitem
  }{%
    \if@restonecol
      \onecolumn
    \else
      \clearpage
    \fi
  }%
  \@ifclasswith{scrreprt}{idxtotoc}{%
    \renewcommand*{\idx@heading}{%
      \if@openright
        \cleardoublepage
      \else
        \clearpage
      \fi
      \twocolumn[%
        \addchap{\indexname}%
        \ifx\index@prologue\@empty
        \else
          \index@prologue
          \bigskip
        \fi
      ]%
      \@mkboth{\indexname}{\indexname}%
    }%
  }{%
    \renewcommand*{\idx@heading}{%
      \if@openright
        \cleardoublepage
      \else
        \clearpage
      \fi
      \twocolumn[%
        \chapter*{\indexname}%
        \ifx\index@prologue\@empty
        \else
          \index@prologue
          \bigskip
        \fi
      ]%
      \@mkboth{\indexname}{\indexname}%
    }%
  }%
}{}
%    \end{macrocode}
%    \begin{macrocode}
\@ifclassloaded{scrbook}{%
  \renewenvironment{theindex}{%
    \edef\indexname{%
      \the\@nameuse{idxtitle@\@indextype}%
    }%
    \if@twocolumn
      \@restonecolfalse
    \else
      \@restonecoltrue
    \fi
    \setchapterpreamble{\index@preamble}%
    \idx@heading
    \thispagestyle{\indexpagestyle}%
    \columnseprule\z@
    \columnsep 35\p@
    \parindent\z@
    \parskip\z@ \@plus .3\p@\relax
    \parfillskip\z@ \@plus 1fil\relax
    \let\item\@idxitem
  }{%
    \if@restonecol
      \onecolumn
    \else
      \clearpage
    \fi
  }%
  \@ifclasswith{scrbook}{idxtotoc}{%
    \renewcommand*{\idx@heading}{%
      \if@openright
        \cleardoublepage
      \else
        \clearpage
      \fi
      \twocolumn[%
        \addchap{\indexname}%
        \ifx\index@prologue\@empty
        \else
          \index@prologue
          \bigskip
        \fi
      ]%
      \@mkboth{\indexname}{\indexname}%
    }%
  }{%
    \renewcommand*{\idx@heading}{%
      \if@openright
        \cleardoublepage
      \else
        \clearpage
      \fi
      \twocolumn[%
        \chapter*{\indexname}%
        \ifx\index@prologue\@empty
        \else
          \index@prologue
          \bigskip
        \fi
      ]%
      \@mkboth{\indexname}{\indexname}%
    }%
  }%
}{}
%    \end{macrocode}
%
%    \begin{macrocode}
%</package>
%    \end{macrocode}
%
% \section{Installation}
%
% \subsection{Download}
%
% \paragraph{Package.} This package is available on
% CTAN\footnote{\url{ftp://ftp.ctan.org/tex-archive/}}:
% \begin{description}
% \item[\CTAN{macros/latex/contrib/oberdiek/scrindex.dtx}] The source file.
% \item[\CTAN{macros/latex/contrib/oberdiek/scrindex.pdf}] Documentation.
% \end{description}
%
%
% \paragraph{Bundle.} All the packages of the bundle `oberdiek'
% are also available in a TDS compliant ZIP archive. There
% the packages are already unpacked and the documentation files
% are generated. The files and directories obey the TDS standard.
% \begin{description}
% \item[\CTAN{install/macros/latex/contrib/oberdiek.tds.zip}]
% \end{description}
% \emph{TDS} refers to the standard ``A Directory Structure
% for \TeX\ Files'' (\CTAN{tds/tds.pdf}). Directories
% with \xfile{texmf} in their name are usually organized this way.
%
% \subsection{Bundle installation}
%
% \paragraph{Unpacking.} Unpack the \xfile{oberdiek.tds.zip} in the
% TDS tree (also known as \xfile{texmf} tree) of your choice.
% Example (linux):
% \begin{quote}
%   |unzip oberdiek.tds.zip -d ~/texmf|
% \end{quote}
%
% \paragraph{Script installation.}
% Check the directory \xfile{TDS:scripts/oberdiek/} for
% scripts that need further installation steps.
% Package \xpackage{attachfile2} comes with the Perl script
% \xfile{pdfatfi.pl} that should be installed in such a way
% that it can be called as \texttt{pdfatfi}.
% Example (linux):
% \begin{quote}
%   |chmod +x scripts/oberdiek/pdfatfi.pl|\\
%   |cp scripts/oberdiek/pdfatfi.pl /usr/local/bin/|
% \end{quote}
%
% \subsection{Package installation}
%
% \paragraph{Unpacking.} The \xfile{.dtx} file is a self-extracting
% \docstrip\ archive. The files are extracted by running the
% \xfile{.dtx} through \plainTeX:
% \begin{quote}
%   \verb|tex scrindex.dtx|
% \end{quote}
%
% \paragraph{TDS.} Now the different files must be moved into
% the different directories in your installation TDS tree
% (also known as \xfile{texmf} tree):
% \begin{quote}
% \def\t{^^A
% \begin{tabular}{@{}>{\ttfamily}l@{ $\rightarrow$ }>{\ttfamily}l@{}}
%   scrindex.sty & tex/latex/oberdiek/scrindex.sty\\
%   scrindex.pdf & doc/latex/oberdiek/scrindex.pdf\\
%   scrindex-example1.tex & doc/latex/oberdiek/scrindex-example1.tex\\
%   scrindex-example2.tex & doc/latex/oberdiek/scrindex-example2.tex\\
%   scrindex.dtx & source/latex/oberdiek/scrindex.dtx\\
% \end{tabular}^^A
% }^^A
% \sbox0{\t}^^A
% \ifdim\wd0>\linewidth
%   \begingroup
%     \advance\linewidth by\leftmargin
%     \advance\linewidth by\rightmargin
%   \edef\x{\endgroup
%     \def\noexpand\lw{\the\linewidth}^^A
%   }\x
%   \def\lwbox{^^A
%     \leavevmode
%     \hbox to \linewidth{^^A
%       \kern-\leftmargin\relax
%       \hss
%       \usebox0
%       \hss
%       \kern-\rightmargin\relax
%     }^^A
%   }^^A
%   \ifdim\wd0>\lw
%     \sbox0{\small\t}^^A
%     \ifdim\wd0>\linewidth
%       \ifdim\wd0>\lw
%         \sbox0{\footnotesize\t}^^A
%         \ifdim\wd0>\linewidth
%           \ifdim\wd0>\lw
%             \sbox0{\scriptsize\t}^^A
%             \ifdim\wd0>\linewidth
%               \ifdim\wd0>\lw
%                 \sbox0{\tiny\t}^^A
%                 \ifdim\wd0>\linewidth
%                   \lwbox
%                 \else
%                   \usebox0
%                 \fi
%               \else
%                 \lwbox
%               \fi
%             \else
%               \usebox0
%             \fi
%           \else
%             \lwbox
%           \fi
%         \else
%           \usebox0
%         \fi
%       \else
%         \lwbox
%       \fi
%     \else
%       \usebox0
%     \fi
%   \else
%     \lwbox
%   \fi
% \else
%   \usebox0
% \fi
% \end{quote}
% If you have a \xfile{docstrip.cfg} that configures and enables \docstrip's
% TDS installing feature, then some files can already be in the right
% place, see the documentation of \docstrip.
%
% \subsection{Refresh file name databases}
%
% If your \TeX~distribution
% (\teTeX, \mikTeX, \dots) relies on file name databases, you must refresh
% these. For example, \teTeX\ users run \verb|texhash| or
% \verb|mktexlsr|.
%
% \subsection{Some details for the interested}
%
% \paragraph{Attached source.}
%
% The PDF documentation on CTAN also includes the
% \xfile{.dtx} source file. It can be extracted by
% AcrobatReader 6 or higher. Another option is \textsf{pdftk},
% e.g. unpack the file into the current directory:
% \begin{quote}
%   \verb|pdftk scrindex.pdf unpack_files output .|
% \end{quote}
%
% \paragraph{Unpacking with \LaTeX.}
% The \xfile{.dtx} chooses its action depending on the format:
% \begin{description}
% \item[\plainTeX:] Run \docstrip\ and extract the files.
% \item[\LaTeX:] Generate the documentation.
% \end{description}
% If you insist on using \LaTeX\ for \docstrip\ (really,
% \docstrip\ does not need \LaTeX), then inform the autodetect routine
% about your intention:
% \begin{quote}
%   \verb|latex \let\install=y\input{scrindex.dtx}|
% \end{quote}
% Do not forget to quote the argument according to the demands
% of your shell.
%
% \paragraph{Generating the documentation.}
% You can use both the \xfile{.dtx} or the \xfile{.drv} to generate
% the documentation. The process can be configured by the
% configuration file \xfile{ltxdoc.cfg}. For instance, put this
% line into this file, if you want to have A4 as paper format:
% \begin{quote}
%   \verb|\PassOptionsToClass{a4paper}{article}|
% \end{quote}
% An example follows how to generate the
% documentation with pdf\LaTeX:
% \begin{quote}
%\begin{verbatim}
%pdflatex scrindex.dtx
%makeindex -s gind.ist scrindex.idx
%pdflatex scrindex.dtx
%makeindex -s gind.ist scrindex.idx
%pdflatex scrindex.dtx
%\end{verbatim}
% \end{quote}
%
% \section{Catalogue}
%
% The following XML file can be used as source for the
% \href{http://mirror.ctan.org/help/Catalogue/catalogue.html}{\TeX\ Catalogue}.
% The elements \texttt{caption} and \texttt{description} are imported
% from the original XML file from the Catalogue.
% The name of the XML file in the Catalogue is \xfile{scrindex.xml}.
%    \begin{macrocode}
%<*catalogue>
<?xml version='1.0' encoding='us-ascii'?>
<!DOCTYPE entry SYSTEM 'catalogue.dtd'>
<entry datestamp='$Date$' modifier='$Author$' id='scrindex'>
  <name>scrindex</name>
  <caption>Make index package work with Koma-script classes.</caption>
  <authorref id='auth:oberdiek'/>
  <copyright owner='Heiko Oberdiek' year='2008'/>
  <license type='lppl1.3'/>
  <version number='1.1'/>
  <description>
    This package redefines environment `theindex' of package `index',
    if a class from <xref refid='koma-script'>KOMA-Script</xref> is loaded.
    Also option `idxtotoc' is supported. Index preambles can be given
    either by means of package `index' or via the interface provided
    by <xref refid='koma-script'>KOMA-Script</xref>.
    <p/>
    The package is part of the <xref refid='oberdiek'>oberdiek</xref>
    bundle.
  </description>
  <documentation details='Package documentation'
      href='ctan:/macros/latex/contrib/oberdiek/scrindex.pdf'/>
  <ctan file='true' path='/macros/latex/contrib/oberdiek/scrindex.dtx'/>
  <miktex location='oberdiek'/>
  <texlive location='oberdiek'/>
  <install path='/macros/latex/contrib/oberdiek/oberdiek.tds.zip'/>
</entry>
%</catalogue>
%    \end{macrocode}
%
% \begin{History}
%   \begin{Version}{2008/07/07 v1.0}
%   \item
%     First version, also published in newsgroup \xnewsgroup{de.comp.text.tex}:\\
%     \URL{``\link{Re: Z\"ahler bei \cs{index}}''}^^A
%     {http://groups.google.com/group/de.comp.text.tex/msg/39575b5e2f29be1e}
%   \end{Version}
%   \begin{Version}{2008/08/11 v1.1}
%   \item
%     Code is not changed.
%   \item
%     URLs updated.
%   \end{Version}
% \end{History}
%
% \PrintIndex
%
% \Finale
\endinput

%        (quote the arguments according to the demands of your shell)
%
% Documentation:
%    (a) If scrindex.drv is present:
%           latex scrindex.drv
%    (b) Without scrindex.drv:
%           latex scrindex.dtx; ...
%    The class ltxdoc loads the configuration file ltxdoc.cfg
%    if available. Here you can specify further options, e.g.
%    use A4 as paper format:
%       \PassOptionsToClass{a4paper}{article}
%
%    Programm calls to get the documentation (example):
%       pdflatex scrindex.dtx
%       makeindex -s gind.ist scrindex.idx
%       pdflatex scrindex.dtx
%       makeindex -s gind.ist scrindex.idx
%       pdflatex scrindex.dtx
%
% Installation:
%    TDS:tex/latex/oberdiek/scrindex.sty
%    TDS:doc/latex/oberdiek/scrindex.pdf
%    TDS:doc/latex/oberdiek/scrindex-example1.tex
%    TDS:doc/latex/oberdiek/scrindex-example2.tex
%    TDS:source/latex/oberdiek/scrindex.dtx
%
%<*ignore>
\begingroup
  \catcode123=1 %
  \catcode125=2 %
  \def\x{LaTeX2e}%
\expandafter\endgroup
\ifcase 0\ifx\install y1\fi\expandafter
         \ifx\csname processbatchFile\endcsname\relax\else1\fi
         \ifx\fmtname\x\else 1\fi\relax
\else\csname fi\endcsname
%</ignore>
%<*install>
\input docstrip.tex
\Msg{************************************************************************}
\Msg{* Installation}
\Msg{* Package: scrindex 2008/08/11 v1.1 Package index with KOMA-Script classes (HO)}
\Msg{************************************************************************}

\keepsilent
\askforoverwritefalse

\let\MetaPrefix\relax
\preamble

This is a generated file.

Project: scrindex
Version: 2008/08/11 v1.1

Copyright (C) 2008 by
   Heiko Oberdiek <heiko.oberdiek at googlemail.com>

This work may be distributed and/or modified under the
conditions of the LaTeX Project Public License, either
version 1.3c of this license or (at your option) any later
version. This version of this license is in
   http://www.latex-project.org/lppl/lppl-1-3c.txt
and the latest version of this license is in
   http://www.latex-project.org/lppl.txt
and version 1.3 or later is part of all distributions of
LaTeX version 2005/12/01 or later.

This work has the LPPL maintenance status "maintained".

This Current Maintainer of this work is Heiko Oberdiek.

This work consists of the main source file scrindex.dtx
and the derived files
   scrindex.sty, scrindex.pdf, scrindex.ins, scrindex.drv,
   scrindex-example1.tex, scrindex-example2.tex.

\endpreamble
\let\MetaPrefix\DoubleperCent

\generate{%
  \file{scrindex.ins}{\from{scrindex.dtx}{install}}%
  \file{scrindex.drv}{\from{scrindex.dtx}{driver}}%
  \usedir{tex/latex/oberdiek}%
  \file{scrindex.sty}{\from{scrindex.dtx}{package}}%
  \usedir{doc/latex/oberdiek}%
  \file{scrindex-example1.tex}{\from{scrindex.dtx}{example1}}%
  \file{scrindex-example2.tex}{\from{scrindex.dtx}{example2}}%
  \nopreamble
  \nopostamble
  \usedir{source/latex/oberdiek/catalogue}%
  \file{scrindex.xml}{\from{scrindex.dtx}{catalogue}}%
}

\catcode32=13\relax% active space
\let =\space%
\Msg{************************************************************************}
\Msg{*}
\Msg{* To finish the installation you have to move the following}
\Msg{* file into a directory searched by TeX:}
\Msg{*}
\Msg{*     scrindex.sty}
\Msg{*}
\Msg{* To produce the documentation run the file `scrindex.drv'}
\Msg{* through LaTeX.}
\Msg{*}
\Msg{* Happy TeXing!}
\Msg{*}
\Msg{************************************************************************}

\endbatchfile
%</install>
%<*ignore>
\fi
%</ignore>
%<*driver>
\NeedsTeXFormat{LaTeX2e}
\ProvidesFile{scrindex.drv}%
  [2008/08/11 v1.1 Package index with KOMA-Script classes (HO)]%
\documentclass{ltxdoc}
\usepackage{holtxdoc}[2011/11/22]
\usepackage{calc}
\begin{document}
  \DocInput{scrindex.dtx}%
\end{document}
%</driver>
% \fi
%
% \CheckSum{237}
%
% \CharacterTable
%  {Upper-case    \A\B\C\D\E\F\G\H\I\J\K\L\M\N\O\P\Q\R\S\T\U\V\W\X\Y\Z
%   Lower-case    \a\b\c\d\e\f\g\h\i\j\k\l\m\n\o\p\q\r\s\t\u\v\w\x\y\z
%   Digits        \0\1\2\3\4\5\6\7\8\9
%   Exclamation   \!     Double quote  \"     Hash (number) \#
%   Dollar        \$     Percent       \%     Ampersand     \&
%   Acute accent  \'     Left paren    \(     Right paren   \)
%   Asterisk      \*     Plus          \+     Comma         \,
%   Minus         \-     Point         \.     Solidus       \/
%   Colon         \:     Semicolon     \;     Less than     \<
%   Equals        \=     Greater than  \>     Question mark \?
%   Commercial at \@     Left bracket  \[     Backslash     \\
%   Right bracket \]     Circumflex    \^     Underscore    \_
%   Grave accent  \`     Left brace    \{     Vertical bar  \|
%   Right brace   \}     Tilde         \~}
%
% \GetFileInfo{scrindex.drv}
%
% \title{The \xpackage{scrindex} package}
% \date{2008/08/11 v1.1}
% \author{Heiko Oberdiek\\\xemail{heiko.oberdiek at googlemail.com}}
%
% \maketitle
%
% \begin{abstract}
% This package redefines environment `theindex' of package \xpackage{index},
% if a class from KOMA-Script is loaded. Also option \xoption{idxtotoc}
% is supported. Index preambles can be given either by means of package
% \xpackage{index} or via the interface provided by KOMA-Script.
% \end{abstract}
%
% \tableofcontents
%
% \section{Documentation}
%
% Package \xpackage{index}, written by David M.\ Jones, detects
% the standard classes |article|, |report|, and |book|. It
% redefines environment `theindex' for its needs.
% However, it does not know other classes such as KOMA-Script.
% This package closes the compatibiliy gap between KOMA-Script's
% classes and package \xpackage{index}.
%
% Environment |theindex| is redefined to support both package
% \xpackage{index} and KOMA-Script's classes. Thus both
% the prologe of package \xpackage{index} and the preamble
% of KOMA-Script's classes are available. Also class option |idxtotoc|
% of KOMA-Script is supported.
%
% \subsection{Usage}
%
% The package \xpackage{scrindex} is loaded without options:
%\begin{quote}
%\begin{verbatim}
%\usepackage{scrindex}
%\end{verbatim}
%\end{quote}
%
% It loads package \xpackage{index} and requests version 2004/01/20
% or later. \LaTeX's package interface allows multiple calls
% of the same package. The package is loaded at its first
% package loading command. At later times \LaTeX\ only checks
% options and a requested version date. Therefore it does not harm
% to add |\usepackage{index}| before or after |\usepackage{scrindex}|.
%
% Also the class does not matter. Environment |theindex| is only
% redefined for a supported class:
% \begin{itemize}
% \item |scrartcl|
% \item |scrreprt|
% \item |scrbook|
% \end{itemize}
%
% \subsection{Preambles}
%
% Both the prologue of package \xpackage{index} and the preamble
% of KOMA-Script's classes are supported. The position depends on
% the class.
%
% \subsubsection{Class \xclass{scrartcl}}
%
%    \begin{macrocode}
%<*example1>
\documentclass{scrartcl}
\usepackage{scrindex}
\setindexpreamble{Preamble of \texttt{scrartcl}\dotfill EOL}
\makeindex
\begin{document}
\section{First Section}
\index{first}
\index{section}
\printindex[default]%
  [Prologue of package \texttt{index}\dotfill EOL]%
\end{document}
%</example1>
%    \end{macrocode}
% The prologue of package \xpackage{index} is first set straight
% after the section title spanning both columns.
% Then the preamble of KOMA-Script follows
% in the first left column.
%
% \medskip
% \begin{quote}
%   \renewcommand*{\arraystretch}{1.2}
%   \begin{tabular}{|p{.45\linewidth}|p{.45\linewidth}|}
%   \hline
%   \multicolumn{2}{|l|}{\textbf{Index}}\\[1ex]
%   \multicolumn{2}{|p{.9\linewidth+2\tabcolsep}|}{^^A
%     Prologue of package \texttt{index}\dotfill EOL^^A
%   }\\[1ex]
%   \hline
%   Preamble of \texttt{scrartcl}\dotfill EOL&\\
%   first, 1&\\
%   section, 1&\\
%   \hline
%   \end{tabular}
% \end{quote}
%
% \subsubsection{Classes \xclass{scrreprt} and \xclass{scrbook}}
%
%    \begin{macrocode}
%<*example2>
\documentclass[openany]{scrbook}% or scrreprt
\usepackage{scrindex}
\setindexpreamble{Preamble of class \texttt{scrbook}\dotfill EOL}
\makeindex
\begin{document}
\chapter{First Chapter}
\index{first}
\index{chapter}
\printindex[default]%
  [Prologue of package \texttt{index}\dotfill EOL]%
\end{document}
%</example2>
%    \end{macrocode}
% The order of the two preambles are different for the classes
% \xclass{scrreprt} and \xclass{scrbook}. First KOMA-Script's
% chapter preamble is set, then the prologue of package \xpackage{index}
% follows. Both are set spanning both columns.
%
% \medskip
% \begin{quote}
%   \renewcommand*{\arraystretch}{1.2}
%   \begin{tabular}{|p{.45\linewidth}|p{.45\linewidth}|}
%   \hline
%   \multicolumn{2}{|l|}{\textbf{Index}}\\[1ex]
%   \multicolumn{2}{|p{.9\linewidth+2\tabcolsep}|}{^^A
%     Preamble of class \texttt{scrbook}\dotfill EOL^^A
%   }\\
%   \multicolumn{2}{|p{.9\linewidth+2\tabcolsep}|}{^^A
%     Prologue of package \xpackage{index}\dotfill EOL^^A
%   }\\[1ex]
%   \hline
%   chapter, 1&\\
%   first, 1&\\
%   \hline
%   \end{tabular}
% \end{quote}
%
% \StopEventually{
% }
%
% \section{Implementation}
%
%    \begin{macrocode}
%<*package>
\NeedsTeXFormat{LaTeX2e}
\ProvidesPackage{scrindex}
  [2008/08/11 v1.1 Package index with KOMA-Script classes (HO)]%
%    \end{macrocode}
%
%    \begin{macrocode}
\RequirePackage{index}[2004/01/20]%
%    \end{macrocode}
%
%    \begin{macrocode}
\@ifclassloaded{scrartcl}{%
  \renewenvironment{theindex}{%
    \edef\indexname{%
      \the\@nameuse{idxtitle@\@indextype}%
    }%
    \if@twocolumn
      \@restonecolfalse
    \else
      \@restonecoltrue
    \fi
    \idx@heading
    \thispagestyle{\indexpagestyle}%
    \columnseprule\z@
    \columnsep 35\p@
    \index@preamble\par\nobreak
    \parindent\z@
    \parskip\z@ \@plus .3\p@\relax
    \parfillskip\z@ \@plus 1fil\relax
    \let\item\@idxitem
  }{%
    \if@restonecol
      \onecolumn
    \else
      \clearpage
    \fi
  }%
  \@ifclasswith{scrartcl}{idxtotoc}{%
    \renewcommand*{\idx@heading}{%
      \twocolumn[%
        \addsec{\indexname}%
        \ifx\index@prologue\@empty
        \else
          \index@prologue
          \bigskip
        \fi
      ]%
      \@mkboth{\indexname}{\indexname}%
    }%
  }{%
    \renewcommand*{\idx@heading}{%
      \twocolumn[%
        \section*{\indexname}%
        \ifx\index@prologue\@empty
        \else
          \index@prologue
          \bigskip
        \fi
      ]%
      \@mkboth{\indexname}{\indexname}%
    }%
  }%
}{}
%    \end{macrocode}
%    \begin{macrocode}
\@ifclassloaded{scrreprt}{%
  \renewenvironment{theindex}{%
    \edef\indexname{%
      \the\@nameuse{idxtitle@\@indextype}%
    }%
    \if@twocolumn
      \@restonecolfalse
    \else
      \@restonecoltrue
    \fi
    \setchapterpreamble{\index@preamble}%
    \idx@heading
    \thispagestyle{\indexpagestyle}%
    \columnseprule\z@
    \columnsep 35\p@
    \parindent\z@
    \parskip\z@ \@plus .3\p@\relax
    \parfillskip\z@ \@plus 1fil\relax
    \let\item\@idxitem
  }{%
    \if@restonecol
      \onecolumn
    \else
      \clearpage
    \fi
  }%
  \@ifclasswith{scrreprt}{idxtotoc}{%
    \renewcommand*{\idx@heading}{%
      \if@openright
        \cleardoublepage
      \else
        \clearpage
      \fi
      \twocolumn[%
        \addchap{\indexname}%
        \ifx\index@prologue\@empty
        \else
          \index@prologue
          \bigskip
        \fi
      ]%
      \@mkboth{\indexname}{\indexname}%
    }%
  }{%
    \renewcommand*{\idx@heading}{%
      \if@openright
        \cleardoublepage
      \else
        \clearpage
      \fi
      \twocolumn[%
        \chapter*{\indexname}%
        \ifx\index@prologue\@empty
        \else
          \index@prologue
          \bigskip
        \fi
      ]%
      \@mkboth{\indexname}{\indexname}%
    }%
  }%
}{}
%    \end{macrocode}
%    \begin{macrocode}
\@ifclassloaded{scrbook}{%
  \renewenvironment{theindex}{%
    \edef\indexname{%
      \the\@nameuse{idxtitle@\@indextype}%
    }%
    \if@twocolumn
      \@restonecolfalse
    \else
      \@restonecoltrue
    \fi
    \setchapterpreamble{\index@preamble}%
    \idx@heading
    \thispagestyle{\indexpagestyle}%
    \columnseprule\z@
    \columnsep 35\p@
    \parindent\z@
    \parskip\z@ \@plus .3\p@\relax
    \parfillskip\z@ \@plus 1fil\relax
    \let\item\@idxitem
  }{%
    \if@restonecol
      \onecolumn
    \else
      \clearpage
    \fi
  }%
  \@ifclasswith{scrbook}{idxtotoc}{%
    \renewcommand*{\idx@heading}{%
      \if@openright
        \cleardoublepage
      \else
        \clearpage
      \fi
      \twocolumn[%
        \addchap{\indexname}%
        \ifx\index@prologue\@empty
        \else
          \index@prologue
          \bigskip
        \fi
      ]%
      \@mkboth{\indexname}{\indexname}%
    }%
  }{%
    \renewcommand*{\idx@heading}{%
      \if@openright
        \cleardoublepage
      \else
        \clearpage
      \fi
      \twocolumn[%
        \chapter*{\indexname}%
        \ifx\index@prologue\@empty
        \else
          \index@prologue
          \bigskip
        \fi
      ]%
      \@mkboth{\indexname}{\indexname}%
    }%
  }%
}{}
%    \end{macrocode}
%
%    \begin{macrocode}
%</package>
%    \end{macrocode}
%
% \section{Installation}
%
% \subsection{Download}
%
% \paragraph{Package.} This package is available on
% CTAN\footnote{\url{ftp://ftp.ctan.org/tex-archive/}}:
% \begin{description}
% \item[\CTAN{macros/latex/contrib/oberdiek/scrindex.dtx}] The source file.
% \item[\CTAN{macros/latex/contrib/oberdiek/scrindex.pdf}] Documentation.
% \end{description}
%
%
% \paragraph{Bundle.} All the packages of the bundle `oberdiek'
% are also available in a TDS compliant ZIP archive. There
% the packages are already unpacked and the documentation files
% are generated. The files and directories obey the TDS standard.
% \begin{description}
% \item[\CTAN{install/macros/latex/contrib/oberdiek.tds.zip}]
% \end{description}
% \emph{TDS} refers to the standard ``A Directory Structure
% for \TeX\ Files'' (\CTAN{tds/tds.pdf}). Directories
% with \xfile{texmf} in their name are usually organized this way.
%
% \subsection{Bundle installation}
%
% \paragraph{Unpacking.} Unpack the \xfile{oberdiek.tds.zip} in the
% TDS tree (also known as \xfile{texmf} tree) of your choice.
% Example (linux):
% \begin{quote}
%   |unzip oberdiek.tds.zip -d ~/texmf|
% \end{quote}
%
% \paragraph{Script installation.}
% Check the directory \xfile{TDS:scripts/oberdiek/} for
% scripts that need further installation steps.
% Package \xpackage{attachfile2} comes with the Perl script
% \xfile{pdfatfi.pl} that should be installed in such a way
% that it can be called as \texttt{pdfatfi}.
% Example (linux):
% \begin{quote}
%   |chmod +x scripts/oberdiek/pdfatfi.pl|\\
%   |cp scripts/oberdiek/pdfatfi.pl /usr/local/bin/|
% \end{quote}
%
% \subsection{Package installation}
%
% \paragraph{Unpacking.} The \xfile{.dtx} file is a self-extracting
% \docstrip\ archive. The files are extracted by running the
% \xfile{.dtx} through \plainTeX:
% \begin{quote}
%   \verb|tex scrindex.dtx|
% \end{quote}
%
% \paragraph{TDS.} Now the different files must be moved into
% the different directories in your installation TDS tree
% (also known as \xfile{texmf} tree):
% \begin{quote}
% \def\t{^^A
% \begin{tabular}{@{}>{\ttfamily}l@{ $\rightarrow$ }>{\ttfamily}l@{}}
%   scrindex.sty & tex/latex/oberdiek/scrindex.sty\\
%   scrindex.pdf & doc/latex/oberdiek/scrindex.pdf\\
%   scrindex-example1.tex & doc/latex/oberdiek/scrindex-example1.tex\\
%   scrindex-example2.tex & doc/latex/oberdiek/scrindex-example2.tex\\
%   scrindex.dtx & source/latex/oberdiek/scrindex.dtx\\
% \end{tabular}^^A
% }^^A
% \sbox0{\t}^^A
% \ifdim\wd0>\linewidth
%   \begingroup
%     \advance\linewidth by\leftmargin
%     \advance\linewidth by\rightmargin
%   \edef\x{\endgroup
%     \def\noexpand\lw{\the\linewidth}^^A
%   }\x
%   \def\lwbox{^^A
%     \leavevmode
%     \hbox to \linewidth{^^A
%       \kern-\leftmargin\relax
%       \hss
%       \usebox0
%       \hss
%       \kern-\rightmargin\relax
%     }^^A
%   }^^A
%   \ifdim\wd0>\lw
%     \sbox0{\small\t}^^A
%     \ifdim\wd0>\linewidth
%       \ifdim\wd0>\lw
%         \sbox0{\footnotesize\t}^^A
%         \ifdim\wd0>\linewidth
%           \ifdim\wd0>\lw
%             \sbox0{\scriptsize\t}^^A
%             \ifdim\wd0>\linewidth
%               \ifdim\wd0>\lw
%                 \sbox0{\tiny\t}^^A
%                 \ifdim\wd0>\linewidth
%                   \lwbox
%                 \else
%                   \usebox0
%                 \fi
%               \else
%                 \lwbox
%               \fi
%             \else
%               \usebox0
%             \fi
%           \else
%             \lwbox
%           \fi
%         \else
%           \usebox0
%         \fi
%       \else
%         \lwbox
%       \fi
%     \else
%       \usebox0
%     \fi
%   \else
%     \lwbox
%   \fi
% \else
%   \usebox0
% \fi
% \end{quote}
% If you have a \xfile{docstrip.cfg} that configures and enables \docstrip's
% TDS installing feature, then some files can already be in the right
% place, see the documentation of \docstrip.
%
% \subsection{Refresh file name databases}
%
% If your \TeX~distribution
% (\teTeX, \mikTeX, \dots) relies on file name databases, you must refresh
% these. For example, \teTeX\ users run \verb|texhash| or
% \verb|mktexlsr|.
%
% \subsection{Some details for the interested}
%
% \paragraph{Attached source.}
%
% The PDF documentation on CTAN also includes the
% \xfile{.dtx} source file. It can be extracted by
% AcrobatReader 6 or higher. Another option is \textsf{pdftk},
% e.g. unpack the file into the current directory:
% \begin{quote}
%   \verb|pdftk scrindex.pdf unpack_files output .|
% \end{quote}
%
% \paragraph{Unpacking with \LaTeX.}
% The \xfile{.dtx} chooses its action depending on the format:
% \begin{description}
% \item[\plainTeX:] Run \docstrip\ and extract the files.
% \item[\LaTeX:] Generate the documentation.
% \end{description}
% If you insist on using \LaTeX\ for \docstrip\ (really,
% \docstrip\ does not need \LaTeX), then inform the autodetect routine
% about your intention:
% \begin{quote}
%   \verb|latex \let\install=y% \iffalse meta-comment
%
% File: scrindex.dtx
% Version: 2008/08/11 v1.1
% Info: Package index with KOMA-Script classes
%
% Copyright (C) 2008 by
%    Heiko Oberdiek <heiko.oberdiek at googlemail.com>
%
% This work may be distributed and/or modified under the
% conditions of the LaTeX Project Public License, either
% version 1.3c of this license or (at your option) any later
% version. This version of this license is in
%    http://www.latex-project.org/lppl/lppl-1-3c.txt
% and the latest version of this license is in
%    http://www.latex-project.org/lppl.txt
% and version 1.3 or later is part of all distributions of
% LaTeX version 2005/12/01 or later.
%
% This work has the LPPL maintenance status "maintained".
%
% This Current Maintainer of this work is Heiko Oberdiek.
%
% This work consists of the main source file scrindex.dtx
% and the derived files
%    scrindex.sty, scrindex.pdf, scrindex.ins, scrindex.drv,
%    scrindex-example1.tex, scrindex-example2.tex.
%
% Distribution:
%    CTAN:macros/latex/contrib/oberdiek/scrindex.dtx
%    CTAN:macros/latex/contrib/oberdiek/scrindex.pdf
%
% Unpacking:
%    (a) If scrindex.ins is present:
%           tex scrindex.ins
%    (b) Without scrindex.ins:
%           tex scrindex.dtx
%    (c) If you insist on using LaTeX
%           latex \let\install=y\input{scrindex.dtx}
%        (quote the arguments according to the demands of your shell)
%
% Documentation:
%    (a) If scrindex.drv is present:
%           latex scrindex.drv
%    (b) Without scrindex.drv:
%           latex scrindex.dtx; ...
%    The class ltxdoc loads the configuration file ltxdoc.cfg
%    if available. Here you can specify further options, e.g.
%    use A4 as paper format:
%       \PassOptionsToClass{a4paper}{article}
%
%    Programm calls to get the documentation (example):
%       pdflatex scrindex.dtx
%       makeindex -s gind.ist scrindex.idx
%       pdflatex scrindex.dtx
%       makeindex -s gind.ist scrindex.idx
%       pdflatex scrindex.dtx
%
% Installation:
%    TDS:tex/latex/oberdiek/scrindex.sty
%    TDS:doc/latex/oberdiek/scrindex.pdf
%    TDS:doc/latex/oberdiek/scrindex-example1.tex
%    TDS:doc/latex/oberdiek/scrindex-example2.tex
%    TDS:source/latex/oberdiek/scrindex.dtx
%
%<*ignore>
\begingroup
  \catcode123=1 %
  \catcode125=2 %
  \def\x{LaTeX2e}%
\expandafter\endgroup
\ifcase 0\ifx\install y1\fi\expandafter
         \ifx\csname processbatchFile\endcsname\relax\else1\fi
         \ifx\fmtname\x\else 1\fi\relax
\else\csname fi\endcsname
%</ignore>
%<*install>
\input docstrip.tex
\Msg{************************************************************************}
\Msg{* Installation}
\Msg{* Package: scrindex 2008/08/11 v1.1 Package index with KOMA-Script classes (HO)}
\Msg{************************************************************************}

\keepsilent
\askforoverwritefalse

\let\MetaPrefix\relax
\preamble

This is a generated file.

Project: scrindex
Version: 2008/08/11 v1.1

Copyright (C) 2008 by
   Heiko Oberdiek <heiko.oberdiek at googlemail.com>

This work may be distributed and/or modified under the
conditions of the LaTeX Project Public License, either
version 1.3c of this license or (at your option) any later
version. This version of this license is in
   http://www.latex-project.org/lppl/lppl-1-3c.txt
and the latest version of this license is in
   http://www.latex-project.org/lppl.txt
and version 1.3 or later is part of all distributions of
LaTeX version 2005/12/01 or later.

This work has the LPPL maintenance status "maintained".

This Current Maintainer of this work is Heiko Oberdiek.

This work consists of the main source file scrindex.dtx
and the derived files
   scrindex.sty, scrindex.pdf, scrindex.ins, scrindex.drv,
   scrindex-example1.tex, scrindex-example2.tex.

\endpreamble
\let\MetaPrefix\DoubleperCent

\generate{%
  \file{scrindex.ins}{\from{scrindex.dtx}{install}}%
  \file{scrindex.drv}{\from{scrindex.dtx}{driver}}%
  \usedir{tex/latex/oberdiek}%
  \file{scrindex.sty}{\from{scrindex.dtx}{package}}%
  \usedir{doc/latex/oberdiek}%
  \file{scrindex-example1.tex}{\from{scrindex.dtx}{example1}}%
  \file{scrindex-example2.tex}{\from{scrindex.dtx}{example2}}%
  \nopreamble
  \nopostamble
  \usedir{source/latex/oberdiek/catalogue}%
  \file{scrindex.xml}{\from{scrindex.dtx}{catalogue}}%
}

\catcode32=13\relax% active space
\let =\space%
\Msg{************************************************************************}
\Msg{*}
\Msg{* To finish the installation you have to move the following}
\Msg{* file into a directory searched by TeX:}
\Msg{*}
\Msg{*     scrindex.sty}
\Msg{*}
\Msg{* To produce the documentation run the file `scrindex.drv'}
\Msg{* through LaTeX.}
\Msg{*}
\Msg{* Happy TeXing!}
\Msg{*}
\Msg{************************************************************************}

\endbatchfile
%</install>
%<*ignore>
\fi
%</ignore>
%<*driver>
\NeedsTeXFormat{LaTeX2e}
\ProvidesFile{scrindex.drv}%
  [2008/08/11 v1.1 Package index with KOMA-Script classes (HO)]%
\documentclass{ltxdoc}
\usepackage{holtxdoc}[2011/11/22]
\usepackage{calc}
\begin{document}
  \DocInput{scrindex.dtx}%
\end{document}
%</driver>
% \fi
%
% \CheckSum{237}
%
% \CharacterTable
%  {Upper-case    \A\B\C\D\E\F\G\H\I\J\K\L\M\N\O\P\Q\R\S\T\U\V\W\X\Y\Z
%   Lower-case    \a\b\c\d\e\f\g\h\i\j\k\l\m\n\o\p\q\r\s\t\u\v\w\x\y\z
%   Digits        \0\1\2\3\4\5\6\7\8\9
%   Exclamation   \!     Double quote  \"     Hash (number) \#
%   Dollar        \$     Percent       \%     Ampersand     \&
%   Acute accent  \'     Left paren    \(     Right paren   \)
%   Asterisk      \*     Plus          \+     Comma         \,
%   Minus         \-     Point         \.     Solidus       \/
%   Colon         \:     Semicolon     \;     Less than     \<
%   Equals        \=     Greater than  \>     Question mark \?
%   Commercial at \@     Left bracket  \[     Backslash     \\
%   Right bracket \]     Circumflex    \^     Underscore    \_
%   Grave accent  \`     Left brace    \{     Vertical bar  \|
%   Right brace   \}     Tilde         \~}
%
% \GetFileInfo{scrindex.drv}
%
% \title{The \xpackage{scrindex} package}
% \date{2008/08/11 v1.1}
% \author{Heiko Oberdiek\\\xemail{heiko.oberdiek at googlemail.com}}
%
% \maketitle
%
% \begin{abstract}
% This package redefines environment `theindex' of package \xpackage{index},
% if a class from KOMA-Script is loaded. Also option \xoption{idxtotoc}
% is supported. Index preambles can be given either by means of package
% \xpackage{index} or via the interface provided by KOMA-Script.
% \end{abstract}
%
% \tableofcontents
%
% \section{Documentation}
%
% Package \xpackage{index}, written by David M.\ Jones, detects
% the standard classes |article|, |report|, and |book|. It
% redefines environment `theindex' for its needs.
% However, it does not know other classes such as KOMA-Script.
% This package closes the compatibiliy gap between KOMA-Script's
% classes and package \xpackage{index}.
%
% Environment |theindex| is redefined to support both package
% \xpackage{index} and KOMA-Script's classes. Thus both
% the prologe of package \xpackage{index} and the preamble
% of KOMA-Script's classes are available. Also class option |idxtotoc|
% of KOMA-Script is supported.
%
% \subsection{Usage}
%
% The package \xpackage{scrindex} is loaded without options:
%\begin{quote}
%\begin{verbatim}
%\usepackage{scrindex}
%\end{verbatim}
%\end{quote}
%
% It loads package \xpackage{index} and requests version 2004/01/20
% or later. \LaTeX's package interface allows multiple calls
% of the same package. The package is loaded at its first
% package loading command. At later times \LaTeX\ only checks
% options and a requested version date. Therefore it does not harm
% to add |\usepackage{index}| before or after |\usepackage{scrindex}|.
%
% Also the class does not matter. Environment |theindex| is only
% redefined for a supported class:
% \begin{itemize}
% \item |scrartcl|
% \item |scrreprt|
% \item |scrbook|
% \end{itemize}
%
% \subsection{Preambles}
%
% Both the prologue of package \xpackage{index} and the preamble
% of KOMA-Script's classes are supported. The position depends on
% the class.
%
% \subsubsection{Class \xclass{scrartcl}}
%
%    \begin{macrocode}
%<*example1>
\documentclass{scrartcl}
\usepackage{scrindex}
\setindexpreamble{Preamble of \texttt{scrartcl}\dotfill EOL}
\makeindex
\begin{document}
\section{First Section}
\index{first}
\index{section}
\printindex[default]%
  [Prologue of package \texttt{index}\dotfill EOL]%
\end{document}
%</example1>
%    \end{macrocode}
% The prologue of package \xpackage{index} is first set straight
% after the section title spanning both columns.
% Then the preamble of KOMA-Script follows
% in the first left column.
%
% \medskip
% \begin{quote}
%   \renewcommand*{\arraystretch}{1.2}
%   \begin{tabular}{|p{.45\linewidth}|p{.45\linewidth}|}
%   \hline
%   \multicolumn{2}{|l|}{\textbf{Index}}\\[1ex]
%   \multicolumn{2}{|p{.9\linewidth+2\tabcolsep}|}{^^A
%     Prologue of package \texttt{index}\dotfill EOL^^A
%   }\\[1ex]
%   \hline
%   Preamble of \texttt{scrartcl}\dotfill EOL&\\
%   first, 1&\\
%   section, 1&\\
%   \hline
%   \end{tabular}
% \end{quote}
%
% \subsubsection{Classes \xclass{scrreprt} and \xclass{scrbook}}
%
%    \begin{macrocode}
%<*example2>
\documentclass[openany]{scrbook}% or scrreprt
\usepackage{scrindex}
\setindexpreamble{Preamble of class \texttt{scrbook}\dotfill EOL}
\makeindex
\begin{document}
\chapter{First Chapter}
\index{first}
\index{chapter}
\printindex[default]%
  [Prologue of package \texttt{index}\dotfill EOL]%
\end{document}
%</example2>
%    \end{macrocode}
% The order of the two preambles are different for the classes
% \xclass{scrreprt} and \xclass{scrbook}. First KOMA-Script's
% chapter preamble is set, then the prologue of package \xpackage{index}
% follows. Both are set spanning both columns.
%
% \medskip
% \begin{quote}
%   \renewcommand*{\arraystretch}{1.2}
%   \begin{tabular}{|p{.45\linewidth}|p{.45\linewidth}|}
%   \hline
%   \multicolumn{2}{|l|}{\textbf{Index}}\\[1ex]
%   \multicolumn{2}{|p{.9\linewidth+2\tabcolsep}|}{^^A
%     Preamble of class \texttt{scrbook}\dotfill EOL^^A
%   }\\
%   \multicolumn{2}{|p{.9\linewidth+2\tabcolsep}|}{^^A
%     Prologue of package \xpackage{index}\dotfill EOL^^A
%   }\\[1ex]
%   \hline
%   chapter, 1&\\
%   first, 1&\\
%   \hline
%   \end{tabular}
% \end{quote}
%
% \StopEventually{
% }
%
% \section{Implementation}
%
%    \begin{macrocode}
%<*package>
\NeedsTeXFormat{LaTeX2e}
\ProvidesPackage{scrindex}
  [2008/08/11 v1.1 Package index with KOMA-Script classes (HO)]%
%    \end{macrocode}
%
%    \begin{macrocode}
\RequirePackage{index}[2004/01/20]%
%    \end{macrocode}
%
%    \begin{macrocode}
\@ifclassloaded{scrartcl}{%
  \renewenvironment{theindex}{%
    \edef\indexname{%
      \the\@nameuse{idxtitle@\@indextype}%
    }%
    \if@twocolumn
      \@restonecolfalse
    \else
      \@restonecoltrue
    \fi
    \idx@heading
    \thispagestyle{\indexpagestyle}%
    \columnseprule\z@
    \columnsep 35\p@
    \index@preamble\par\nobreak
    \parindent\z@
    \parskip\z@ \@plus .3\p@\relax
    \parfillskip\z@ \@plus 1fil\relax
    \let\item\@idxitem
  }{%
    \if@restonecol
      \onecolumn
    \else
      \clearpage
    \fi
  }%
  \@ifclasswith{scrartcl}{idxtotoc}{%
    \renewcommand*{\idx@heading}{%
      \twocolumn[%
        \addsec{\indexname}%
        \ifx\index@prologue\@empty
        \else
          \index@prologue
          \bigskip
        \fi
      ]%
      \@mkboth{\indexname}{\indexname}%
    }%
  }{%
    \renewcommand*{\idx@heading}{%
      \twocolumn[%
        \section*{\indexname}%
        \ifx\index@prologue\@empty
        \else
          \index@prologue
          \bigskip
        \fi
      ]%
      \@mkboth{\indexname}{\indexname}%
    }%
  }%
}{}
%    \end{macrocode}
%    \begin{macrocode}
\@ifclassloaded{scrreprt}{%
  \renewenvironment{theindex}{%
    \edef\indexname{%
      \the\@nameuse{idxtitle@\@indextype}%
    }%
    \if@twocolumn
      \@restonecolfalse
    \else
      \@restonecoltrue
    \fi
    \setchapterpreamble{\index@preamble}%
    \idx@heading
    \thispagestyle{\indexpagestyle}%
    \columnseprule\z@
    \columnsep 35\p@
    \parindent\z@
    \parskip\z@ \@plus .3\p@\relax
    \parfillskip\z@ \@plus 1fil\relax
    \let\item\@idxitem
  }{%
    \if@restonecol
      \onecolumn
    \else
      \clearpage
    \fi
  }%
  \@ifclasswith{scrreprt}{idxtotoc}{%
    \renewcommand*{\idx@heading}{%
      \if@openright
        \cleardoublepage
      \else
        \clearpage
      \fi
      \twocolumn[%
        \addchap{\indexname}%
        \ifx\index@prologue\@empty
        \else
          \index@prologue
          \bigskip
        \fi
      ]%
      \@mkboth{\indexname}{\indexname}%
    }%
  }{%
    \renewcommand*{\idx@heading}{%
      \if@openright
        \cleardoublepage
      \else
        \clearpage
      \fi
      \twocolumn[%
        \chapter*{\indexname}%
        \ifx\index@prologue\@empty
        \else
          \index@prologue
          \bigskip
        \fi
      ]%
      \@mkboth{\indexname}{\indexname}%
    }%
  }%
}{}
%    \end{macrocode}
%    \begin{macrocode}
\@ifclassloaded{scrbook}{%
  \renewenvironment{theindex}{%
    \edef\indexname{%
      \the\@nameuse{idxtitle@\@indextype}%
    }%
    \if@twocolumn
      \@restonecolfalse
    \else
      \@restonecoltrue
    \fi
    \setchapterpreamble{\index@preamble}%
    \idx@heading
    \thispagestyle{\indexpagestyle}%
    \columnseprule\z@
    \columnsep 35\p@
    \parindent\z@
    \parskip\z@ \@plus .3\p@\relax
    \parfillskip\z@ \@plus 1fil\relax
    \let\item\@idxitem
  }{%
    \if@restonecol
      \onecolumn
    \else
      \clearpage
    \fi
  }%
  \@ifclasswith{scrbook}{idxtotoc}{%
    \renewcommand*{\idx@heading}{%
      \if@openright
        \cleardoublepage
      \else
        \clearpage
      \fi
      \twocolumn[%
        \addchap{\indexname}%
        \ifx\index@prologue\@empty
        \else
          \index@prologue
          \bigskip
        \fi
      ]%
      \@mkboth{\indexname}{\indexname}%
    }%
  }{%
    \renewcommand*{\idx@heading}{%
      \if@openright
        \cleardoublepage
      \else
        \clearpage
      \fi
      \twocolumn[%
        \chapter*{\indexname}%
        \ifx\index@prologue\@empty
        \else
          \index@prologue
          \bigskip
        \fi
      ]%
      \@mkboth{\indexname}{\indexname}%
    }%
  }%
}{}
%    \end{macrocode}
%
%    \begin{macrocode}
%</package>
%    \end{macrocode}
%
% \section{Installation}
%
% \subsection{Download}
%
% \paragraph{Package.} This package is available on
% CTAN\footnote{\url{ftp://ftp.ctan.org/tex-archive/}}:
% \begin{description}
% \item[\CTAN{macros/latex/contrib/oberdiek/scrindex.dtx}] The source file.
% \item[\CTAN{macros/latex/contrib/oberdiek/scrindex.pdf}] Documentation.
% \end{description}
%
%
% \paragraph{Bundle.} All the packages of the bundle `oberdiek'
% are also available in a TDS compliant ZIP archive. There
% the packages are already unpacked and the documentation files
% are generated. The files and directories obey the TDS standard.
% \begin{description}
% \item[\CTAN{install/macros/latex/contrib/oberdiek.tds.zip}]
% \end{description}
% \emph{TDS} refers to the standard ``A Directory Structure
% for \TeX\ Files'' (\CTAN{tds/tds.pdf}). Directories
% with \xfile{texmf} in their name are usually organized this way.
%
% \subsection{Bundle installation}
%
% \paragraph{Unpacking.} Unpack the \xfile{oberdiek.tds.zip} in the
% TDS tree (also known as \xfile{texmf} tree) of your choice.
% Example (linux):
% \begin{quote}
%   |unzip oberdiek.tds.zip -d ~/texmf|
% \end{quote}
%
% \paragraph{Script installation.}
% Check the directory \xfile{TDS:scripts/oberdiek/} for
% scripts that need further installation steps.
% Package \xpackage{attachfile2} comes with the Perl script
% \xfile{pdfatfi.pl} that should be installed in such a way
% that it can be called as \texttt{pdfatfi}.
% Example (linux):
% \begin{quote}
%   |chmod +x scripts/oberdiek/pdfatfi.pl|\\
%   |cp scripts/oberdiek/pdfatfi.pl /usr/local/bin/|
% \end{quote}
%
% \subsection{Package installation}
%
% \paragraph{Unpacking.} The \xfile{.dtx} file is a self-extracting
% \docstrip\ archive. The files are extracted by running the
% \xfile{.dtx} through \plainTeX:
% \begin{quote}
%   \verb|tex scrindex.dtx|
% \end{quote}
%
% \paragraph{TDS.} Now the different files must be moved into
% the different directories in your installation TDS tree
% (also known as \xfile{texmf} tree):
% \begin{quote}
% \def\t{^^A
% \begin{tabular}{@{}>{\ttfamily}l@{ $\rightarrow$ }>{\ttfamily}l@{}}
%   scrindex.sty & tex/latex/oberdiek/scrindex.sty\\
%   scrindex.pdf & doc/latex/oberdiek/scrindex.pdf\\
%   scrindex-example1.tex & doc/latex/oberdiek/scrindex-example1.tex\\
%   scrindex-example2.tex & doc/latex/oberdiek/scrindex-example2.tex\\
%   scrindex.dtx & source/latex/oberdiek/scrindex.dtx\\
% \end{tabular}^^A
% }^^A
% \sbox0{\t}^^A
% \ifdim\wd0>\linewidth
%   \begingroup
%     \advance\linewidth by\leftmargin
%     \advance\linewidth by\rightmargin
%   \edef\x{\endgroup
%     \def\noexpand\lw{\the\linewidth}^^A
%   }\x
%   \def\lwbox{^^A
%     \leavevmode
%     \hbox to \linewidth{^^A
%       \kern-\leftmargin\relax
%       \hss
%       \usebox0
%       \hss
%       \kern-\rightmargin\relax
%     }^^A
%   }^^A
%   \ifdim\wd0>\lw
%     \sbox0{\small\t}^^A
%     \ifdim\wd0>\linewidth
%       \ifdim\wd0>\lw
%         \sbox0{\footnotesize\t}^^A
%         \ifdim\wd0>\linewidth
%           \ifdim\wd0>\lw
%             \sbox0{\scriptsize\t}^^A
%             \ifdim\wd0>\linewidth
%               \ifdim\wd0>\lw
%                 \sbox0{\tiny\t}^^A
%                 \ifdim\wd0>\linewidth
%                   \lwbox
%                 \else
%                   \usebox0
%                 \fi
%               \else
%                 \lwbox
%               \fi
%             \else
%               \usebox0
%             \fi
%           \else
%             \lwbox
%           \fi
%         \else
%           \usebox0
%         \fi
%       \else
%         \lwbox
%       \fi
%     \else
%       \usebox0
%     \fi
%   \else
%     \lwbox
%   \fi
% \else
%   \usebox0
% \fi
% \end{quote}
% If you have a \xfile{docstrip.cfg} that configures and enables \docstrip's
% TDS installing feature, then some files can already be in the right
% place, see the documentation of \docstrip.
%
% \subsection{Refresh file name databases}
%
% If your \TeX~distribution
% (\teTeX, \mikTeX, \dots) relies on file name databases, you must refresh
% these. For example, \teTeX\ users run \verb|texhash| or
% \verb|mktexlsr|.
%
% \subsection{Some details for the interested}
%
% \paragraph{Attached source.}
%
% The PDF documentation on CTAN also includes the
% \xfile{.dtx} source file. It can be extracted by
% AcrobatReader 6 or higher. Another option is \textsf{pdftk},
% e.g. unpack the file into the current directory:
% \begin{quote}
%   \verb|pdftk scrindex.pdf unpack_files output .|
% \end{quote}
%
% \paragraph{Unpacking with \LaTeX.}
% The \xfile{.dtx} chooses its action depending on the format:
% \begin{description}
% \item[\plainTeX:] Run \docstrip\ and extract the files.
% \item[\LaTeX:] Generate the documentation.
% \end{description}
% If you insist on using \LaTeX\ for \docstrip\ (really,
% \docstrip\ does not need \LaTeX), then inform the autodetect routine
% about your intention:
% \begin{quote}
%   \verb|latex \let\install=y\input{scrindex.dtx}|
% \end{quote}
% Do not forget to quote the argument according to the demands
% of your shell.
%
% \paragraph{Generating the documentation.}
% You can use both the \xfile{.dtx} or the \xfile{.drv} to generate
% the documentation. The process can be configured by the
% configuration file \xfile{ltxdoc.cfg}. For instance, put this
% line into this file, if you want to have A4 as paper format:
% \begin{quote}
%   \verb|\PassOptionsToClass{a4paper}{article}|
% \end{quote}
% An example follows how to generate the
% documentation with pdf\LaTeX:
% \begin{quote}
%\begin{verbatim}
%pdflatex scrindex.dtx
%makeindex -s gind.ist scrindex.idx
%pdflatex scrindex.dtx
%makeindex -s gind.ist scrindex.idx
%pdflatex scrindex.dtx
%\end{verbatim}
% \end{quote}
%
% \section{Catalogue}
%
% The following XML file can be used as source for the
% \href{http://mirror.ctan.org/help/Catalogue/catalogue.html}{\TeX\ Catalogue}.
% The elements \texttt{caption} and \texttt{description} are imported
% from the original XML file from the Catalogue.
% The name of the XML file in the Catalogue is \xfile{scrindex.xml}.
%    \begin{macrocode}
%<*catalogue>
<?xml version='1.0' encoding='us-ascii'?>
<!DOCTYPE entry SYSTEM 'catalogue.dtd'>
<entry datestamp='$Date$' modifier='$Author$' id='scrindex'>
  <name>scrindex</name>
  <caption>Make index package work with Koma-script classes.</caption>
  <authorref id='auth:oberdiek'/>
  <copyright owner='Heiko Oberdiek' year='2008'/>
  <license type='lppl1.3'/>
  <version number='1.1'/>
  <description>
    This package redefines environment `theindex' of package `index',
    if a class from <xref refid='koma-script'>KOMA-Script</xref> is loaded.
    Also option `idxtotoc' is supported. Index preambles can be given
    either by means of package `index' or via the interface provided
    by <xref refid='koma-script'>KOMA-Script</xref>.
    <p/>
    The package is part of the <xref refid='oberdiek'>oberdiek</xref>
    bundle.
  </description>
  <documentation details='Package documentation'
      href='ctan:/macros/latex/contrib/oberdiek/scrindex.pdf'/>
  <ctan file='true' path='/macros/latex/contrib/oberdiek/scrindex.dtx'/>
  <miktex location='oberdiek'/>
  <texlive location='oberdiek'/>
  <install path='/macros/latex/contrib/oberdiek/oberdiek.tds.zip'/>
</entry>
%</catalogue>
%    \end{macrocode}
%
% \begin{History}
%   \begin{Version}{2008/07/07 v1.0}
%   \item
%     First version, also published in newsgroup \xnewsgroup{de.comp.text.tex}:\\
%     \URL{``\link{Re: Z\"ahler bei \cs{index}}''}^^A
%     {http://groups.google.com/group/de.comp.text.tex/msg/39575b5e2f29be1e}
%   \end{Version}
%   \begin{Version}{2008/08/11 v1.1}
%   \item
%     Code is not changed.
%   \item
%     URLs updated.
%   \end{Version}
% \end{History}
%
% \PrintIndex
%
% \Finale
\endinput
|
% \end{quote}
% Do not forget to quote the argument according to the demands
% of your shell.
%
% \paragraph{Generating the documentation.}
% You can use both the \xfile{.dtx} or the \xfile{.drv} to generate
% the documentation. The process can be configured by the
% configuration file \xfile{ltxdoc.cfg}. For instance, put this
% line into this file, if you want to have A4 as paper format:
% \begin{quote}
%   \verb|\PassOptionsToClass{a4paper}{article}|
% \end{quote}
% An example follows how to generate the
% documentation with pdf\LaTeX:
% \begin{quote}
%\begin{verbatim}
%pdflatex scrindex.dtx
%makeindex -s gind.ist scrindex.idx
%pdflatex scrindex.dtx
%makeindex -s gind.ist scrindex.idx
%pdflatex scrindex.dtx
%\end{verbatim}
% \end{quote}
%
% \section{Catalogue}
%
% The following XML file can be used as source for the
% \href{http://mirror.ctan.org/help/Catalogue/catalogue.html}{\TeX\ Catalogue}.
% The elements \texttt{caption} and \texttt{description} are imported
% from the original XML file from the Catalogue.
% The name of the XML file in the Catalogue is \xfile{scrindex.xml}.
%    \begin{macrocode}
%<*catalogue>
<?xml version='1.0' encoding='us-ascii'?>
<!DOCTYPE entry SYSTEM 'catalogue.dtd'>
<entry datestamp='$Date$' modifier='$Author$' id='scrindex'>
  <name>scrindex</name>
  <caption>Make index package work with Koma-script classes.</caption>
  <authorref id='auth:oberdiek'/>
  <copyright owner='Heiko Oberdiek' year='2008'/>
  <license type='lppl1.3'/>
  <version number='1.1'/>
  <description>
    This package redefines environment `theindex' of package `index',
    if a class from <xref refid='koma-script'>KOMA-Script</xref> is loaded.
    Also option `idxtotoc' is supported. Index preambles can be given
    either by means of package `index' or via the interface provided
    by <xref refid='koma-script'>KOMA-Script</xref>.
    <p/>
    The package is part of the <xref refid='oberdiek'>oberdiek</xref>
    bundle.
  </description>
  <documentation details='Package documentation'
      href='ctan:/macros/latex/contrib/oberdiek/scrindex.pdf'/>
  <ctan file='true' path='/macros/latex/contrib/oberdiek/scrindex.dtx'/>
  <miktex location='oberdiek'/>
  <texlive location='oberdiek'/>
  <install path='/macros/latex/contrib/oberdiek/oberdiek.tds.zip'/>
</entry>
%</catalogue>
%    \end{macrocode}
%
% \begin{History}
%   \begin{Version}{2008/07/07 v1.0}
%   \item
%     First version, also published in newsgroup \xnewsgroup{de.comp.text.tex}:\\
%     \URL{``\link{Re: Z\"ahler bei \cs{index}}''}^^A
%     {http://groups.google.com/group/de.comp.text.tex/msg/39575b5e2f29be1e}
%   \end{Version}
%   \begin{Version}{2008/08/11 v1.1}
%   \item
%     Code is not changed.
%   \item
%     URLs updated.
%   \end{Version}
% \end{History}
%
% \PrintIndex
%
% \Finale
\endinput
|
% \end{quote}
% Do not forget to quote the argument according to the demands
% of your shell.
%
% \paragraph{Generating the documentation.}
% You can use both the \xfile{.dtx} or the \xfile{.drv} to generate
% the documentation. The process can be configured by the
% configuration file \xfile{ltxdoc.cfg}. For instance, put this
% line into this file, if you want to have A4 as paper format:
% \begin{quote}
%   \verb|\PassOptionsToClass{a4paper}{article}|
% \end{quote}
% An example follows how to generate the
% documentation with pdf\LaTeX:
% \begin{quote}
%\begin{verbatim}
%pdflatex scrindex.dtx
%makeindex -s gind.ist scrindex.idx
%pdflatex scrindex.dtx
%makeindex -s gind.ist scrindex.idx
%pdflatex scrindex.dtx
%\end{verbatim}
% \end{quote}
%
% \section{Catalogue}
%
% The following XML file can be used as source for the
% \href{http://mirror.ctan.org/help/Catalogue/catalogue.html}{\TeX\ Catalogue}.
% The elements \texttt{caption} and \texttt{description} are imported
% from the original XML file from the Catalogue.
% The name of the XML file in the Catalogue is \xfile{scrindex.xml}.
%    \begin{macrocode}
%<*catalogue>
<?xml version='1.0' encoding='us-ascii'?>
<!DOCTYPE entry SYSTEM 'catalogue.dtd'>
<entry datestamp='$Date$' modifier='$Author$' id='scrindex'>
  <name>scrindex</name>
  <caption>Make index package work with Koma-script classes.</caption>
  <authorref id='auth:oberdiek'/>
  <copyright owner='Heiko Oberdiek' year='2008'/>
  <license type='lppl1.3'/>
  <version number='1.1'/>
  <description>
    This package redefines environment `theindex' of package `index',
    if a class from <xref refid='koma-script'>KOMA-Script</xref> is loaded.
    Also option `idxtotoc' is supported. Index preambles can be given
    either by means of package `index' or via the interface provided
    by <xref refid='koma-script'>KOMA-Script</xref>.
    <p/>
    The package is part of the <xref refid='oberdiek'>oberdiek</xref>
    bundle.
  </description>
  <documentation details='Package documentation'
      href='ctan:/macros/latex/contrib/oberdiek/scrindex.pdf'/>
  <ctan file='true' path='/macros/latex/contrib/oberdiek/scrindex.dtx'/>
  <miktex location='oberdiek'/>
  <texlive location='oberdiek'/>
  <install path='/macros/latex/contrib/oberdiek/oberdiek.tds.zip'/>
</entry>
%</catalogue>
%    \end{macrocode}
%
% \begin{History}
%   \begin{Version}{2008/07/07 v1.0}
%   \item
%     First version, also published in newsgroup \xnewsgroup{de.comp.text.tex}:\\
%     \URL{``\link{Re: Z\"ahler bei \cs{index}}''}^^A
%     {http://groups.google.com/group/de.comp.text.tex/msg/39575b5e2f29be1e}
%   \end{Version}
%   \begin{Version}{2008/08/11 v1.1}
%   \item
%     Code is not changed.
%   \item
%     URLs updated.
%   \end{Version}
% \end{History}
%
% \PrintIndex
%
% \Finale
\endinput
|
% \end{quote}
% Do not forget to quote the argument according to the demands
% of your shell.
%
% \paragraph{Generating the documentation.}
% You can use both the \xfile{.dtx} or the \xfile{.drv} to generate
% the documentation. The process can be configured by the
% configuration file \xfile{ltxdoc.cfg}. For instance, put this
% line into this file, if you want to have A4 as paper format:
% \begin{quote}
%   \verb|\PassOptionsToClass{a4paper}{article}|
% \end{quote}
% An example follows how to generate the
% documentation with pdf\LaTeX:
% \begin{quote}
%\begin{verbatim}
%pdflatex scrindex.dtx
%makeindex -s gind.ist scrindex.idx
%pdflatex scrindex.dtx
%makeindex -s gind.ist scrindex.idx
%pdflatex scrindex.dtx
%\end{verbatim}
% \end{quote}
%
% \section{Catalogue}
%
% The following XML file can be used as source for the
% \href{http://mirror.ctan.org/help/Catalogue/catalogue.html}{\TeX\ Catalogue}.
% The elements \texttt{caption} and \texttt{description} are imported
% from the original XML file from the Catalogue.
% The name of the XML file in the Catalogue is \xfile{scrindex.xml}.
%    \begin{macrocode}
%<*catalogue>
<?xml version='1.0' encoding='us-ascii'?>
<!DOCTYPE entry SYSTEM 'catalogue.dtd'>
<entry datestamp='$Date$' modifier='$Author$' id='scrindex'>
  <name>scrindex</name>
  <caption>Make index package work with Koma-script classes.</caption>
  <authorref id='auth:oberdiek'/>
  <copyright owner='Heiko Oberdiek' year='2008'/>
  <license type='lppl1.3'/>
  <version number='1.1'/>
  <description>
    This package redefines environment `theindex' of package `index',
    if a class from <xref refid='koma-script'>KOMA-Script</xref> is loaded.
    Also option `idxtotoc' is supported. Index preambles can be given
    either by means of package `index' or via the interface provided
    by <xref refid='koma-script'>KOMA-Script</xref>.
    <p/>
    The package is part of the <xref refid='oberdiek'>oberdiek</xref>
    bundle.
  </description>
  <documentation details='Package documentation'
      href='ctan:/macros/latex/contrib/oberdiek/scrindex.pdf'/>
  <ctan file='true' path='/macros/latex/contrib/oberdiek/scrindex.dtx'/>
  <miktex location='oberdiek'/>
  <texlive location='oberdiek'/>
  <install path='/macros/latex/contrib/oberdiek/oberdiek.tds.zip'/>
</entry>
%</catalogue>
%    \end{macrocode}
%
% \begin{History}
%   \begin{Version}{2008/07/07 v1.0}
%   \item
%     First version, also published in newsgroup \xnewsgroup{de.comp.text.tex}:\\
%     \URL{``\link{Re: Z\"ahler bei \cs{index}}''}^^A
%     {http://groups.google.com/group/de.comp.text.tex/msg/39575b5e2f29be1e}
%   \end{Version}
%   \begin{Version}{2008/08/11 v1.1}
%   \item
%     Code is not changed.
%   \item
%     URLs updated.
%   \end{Version}
% \end{History}
%
% \PrintIndex
%
% \Finale
\endinput
