% \iffalse meta-comment
%
% File: pdfescape.dtx
% Version: 2011/11/25 v1.13
% Info: Implements pdfTeX's escape features
%
% Copyright (C) 2007, 2010, 2011 by
%    Heiko Oberdiek <heiko.oberdiek at googlemail.com>
%
% This work may be distributed and/or modified under the
% conditions of the LaTeX Project Public License, either
% version 1.3c of this license or (at your option) any later
% version. This version of this license is in
%    http://www.latex-project.org/lppl/lppl-1-3c.txt
% and the latest version of this license is in
%    http://www.latex-project.org/lppl.txt
% and version 1.3 or later is part of all distributions of
% LaTeX version 2005/12/01 or later.
%
% This work has the LPPL maintenance status "maintained".
%
% This Current Maintainer of this work is Heiko Oberdiek.
%
% The Base Interpreter refers to any `TeX-Format',
% because some files are installed in TDS:tex/generic//.
%
% This work consists of the main source file pdfescape.dtx
% and the derived files
%    pdfescape.sty, pdfescape.pdf, pdfescape.ins, pdfescape.drv,
%    pdfescape-test1.tex, pdfescape-test2.tex,
%    pdfescape-test3.tex, pdfescape-test4.tex,
%    pdfescape-test5.tex, pdfescape-test6.tex.
%
% Distribution:
%    CTAN:macros/latex/contrib/oberdiek/pdfescape.dtx
%    CTAN:macros/latex/contrib/oberdiek/pdfescape.pdf
%
% Unpacking:
%    (a) If pdfescape.ins is present:
%           tex pdfescape.ins
%    (b) Without pdfescape.ins:
%           tex pdfescape.dtx
%    (c) If you insist on using LaTeX
%           latex \let\install=y% \iffalse meta-comment
%
% File: pdfescape.dtx
% Version: 2011/11/25 v1.13
% Info: Implements pdfTeX's escape features
%
% Copyright (C) 2007, 2010, 2011 by
%    Heiko Oberdiek <heiko.oberdiek at googlemail.com>
%
% This work may be distributed and/or modified under the
% conditions of the LaTeX Project Public License, either
% version 1.3c of this license or (at your option) any later
% version. This version of this license is in
%    http://www.latex-project.org/lppl/lppl-1-3c.txt
% and the latest version of this license is in
%    http://www.latex-project.org/lppl.txt
% and version 1.3 or later is part of all distributions of
% LaTeX version 2005/12/01 or later.
%
% This work has the LPPL maintenance status "maintained".
%
% This Current Maintainer of this work is Heiko Oberdiek.
%
% The Base Interpreter refers to any `TeX-Format',
% because some files are installed in TDS:tex/generic//.
%
% This work consists of the main source file pdfescape.dtx
% and the derived files
%    pdfescape.sty, pdfescape.pdf, pdfescape.ins, pdfescape.drv,
%    pdfescape-test1.tex, pdfescape-test2.tex,
%    pdfescape-test3.tex, pdfescape-test4.tex,
%    pdfescape-test5.tex, pdfescape-test6.tex.
%
% Distribution:
%    CTAN:macros/latex/contrib/oberdiek/pdfescape.dtx
%    CTAN:macros/latex/contrib/oberdiek/pdfescape.pdf
%
% Unpacking:
%    (a) If pdfescape.ins is present:
%           tex pdfescape.ins
%    (b) Without pdfescape.ins:
%           tex pdfescape.dtx
%    (c) If you insist on using LaTeX
%           latex \let\install=y% \iffalse meta-comment
%
% File: pdfescape.dtx
% Version: 2011/11/25 v1.13
% Info: Implements pdfTeX's escape features
%
% Copyright (C) 2007, 2010, 2011 by
%    Heiko Oberdiek <heiko.oberdiek at googlemail.com>
%
% This work may be distributed and/or modified under the
% conditions of the LaTeX Project Public License, either
% version 1.3c of this license or (at your option) any later
% version. This version of this license is in
%    http://www.latex-project.org/lppl/lppl-1-3c.txt
% and the latest version of this license is in
%    http://www.latex-project.org/lppl.txt
% and version 1.3 or later is part of all distributions of
% LaTeX version 2005/12/01 or later.
%
% This work has the LPPL maintenance status "maintained".
%
% This Current Maintainer of this work is Heiko Oberdiek.
%
% The Base Interpreter refers to any `TeX-Format',
% because some files are installed in TDS:tex/generic//.
%
% This work consists of the main source file pdfescape.dtx
% and the derived files
%    pdfescape.sty, pdfescape.pdf, pdfescape.ins, pdfescape.drv,
%    pdfescape-test1.tex, pdfescape-test2.tex,
%    pdfescape-test3.tex, pdfescape-test4.tex,
%    pdfescape-test5.tex, pdfescape-test6.tex.
%
% Distribution:
%    CTAN:macros/latex/contrib/oberdiek/pdfescape.dtx
%    CTAN:macros/latex/contrib/oberdiek/pdfescape.pdf
%
% Unpacking:
%    (a) If pdfescape.ins is present:
%           tex pdfescape.ins
%    (b) Without pdfescape.ins:
%           tex pdfescape.dtx
%    (c) If you insist on using LaTeX
%           latex \let\install=y% \iffalse meta-comment
%
% File: pdfescape.dtx
% Version: 2011/11/25 v1.13
% Info: Implements pdfTeX's escape features
%
% Copyright (C) 2007, 2010, 2011 by
%    Heiko Oberdiek <heiko.oberdiek at googlemail.com>
%
% This work may be distributed and/or modified under the
% conditions of the LaTeX Project Public License, either
% version 1.3c of this license or (at your option) any later
% version. This version of this license is in
%    http://www.latex-project.org/lppl/lppl-1-3c.txt
% and the latest version of this license is in
%    http://www.latex-project.org/lppl.txt
% and version 1.3 or later is part of all distributions of
% LaTeX version 2005/12/01 or later.
%
% This work has the LPPL maintenance status "maintained".
%
% This Current Maintainer of this work is Heiko Oberdiek.
%
% The Base Interpreter refers to any `TeX-Format',
% because some files are installed in TDS:tex/generic//.
%
% This work consists of the main source file pdfescape.dtx
% and the derived files
%    pdfescape.sty, pdfescape.pdf, pdfescape.ins, pdfescape.drv,
%    pdfescape-test1.tex, pdfescape-test2.tex,
%    pdfescape-test3.tex, pdfescape-test4.tex,
%    pdfescape-test5.tex, pdfescape-test6.tex.
%
% Distribution:
%    CTAN:macros/latex/contrib/oberdiek/pdfescape.dtx
%    CTAN:macros/latex/contrib/oberdiek/pdfescape.pdf
%
% Unpacking:
%    (a) If pdfescape.ins is present:
%           tex pdfescape.ins
%    (b) Without pdfescape.ins:
%           tex pdfescape.dtx
%    (c) If you insist on using LaTeX
%           latex \let\install=y\input{pdfescape.dtx}
%        (quote the arguments according to the demands of your shell)
%
% Documentation:
%    (a) If pdfescape.drv is present:
%           latex pdfescape.drv
%    (b) Without pdfescape.drv:
%           latex pdfescape.dtx; ...
%    The class ltxdoc loads the configuration file ltxdoc.cfg
%    if available. Here you can specify further options, e.g.
%    use A4 as paper format:
%       \PassOptionsToClass{a4paper}{article}
%
%    Programm calls to get the documentation (example):
%       pdflatex pdfescape.dtx
%       makeindex -s gind.ist pdfescape.idx
%       pdflatex pdfescape.dtx
%       makeindex -s gind.ist pdfescape.idx
%       pdflatex pdfescape.dtx
%
% Installation:
%    TDS:tex/generic/oberdiek/pdfescape.sty
%    TDS:doc/latex/oberdiek/pdfescape.pdf
%    TDS:doc/latex/oberdiek/test/pdfescape-test1.tex
%    TDS:doc/latex/oberdiek/test/pdfescape-test2.tex
%    TDS:doc/latex/oberdiek/test/pdfescape-test3.tex
%    TDS:doc/latex/oberdiek/test/pdfescape-test4.tex
%    TDS:doc/latex/oberdiek/test/pdfescape-test5.tex
%    TDS:doc/latex/oberdiek/test/pdfescape-test6.tex
%    TDS:source/latex/oberdiek/pdfescape.dtx
%
%<*ignore>
\begingroup
  \catcode123=1 %
  \catcode125=2 %
  \def\x{LaTeX2e}%
\expandafter\endgroup
\ifcase 0\ifx\install y1\fi\expandafter
         \ifx\csname processbatchFile\endcsname\relax\else1\fi
         \ifx\fmtname\x\else 1\fi\relax
\else\csname fi\endcsname
%</ignore>
%<*install>
\input docstrip.tex
\Msg{************************************************************************}
\Msg{* Installation}
\Msg{* Package: pdfescape 2011/11/25 v1.13 Implements pdfTeX's escape features (HO)}
\Msg{************************************************************************}

\keepsilent
\askforoverwritefalse

\let\MetaPrefix\relax
\preamble

This is a generated file.

Project: pdfescape
Version: 2011/11/25 v1.13

Copyright (C) 2007, 2010, 2011 by
   Heiko Oberdiek <heiko.oberdiek at googlemail.com>

This work may be distributed and/or modified under the
conditions of the LaTeX Project Public License, either
version 1.3c of this license or (at your option) any later
version. This version of this license is in
   http://www.latex-project.org/lppl/lppl-1-3c.txt
and the latest version of this license is in
   http://www.latex-project.org/lppl.txt
and version 1.3 or later is part of all distributions of
LaTeX version 2005/12/01 or later.

This work has the LPPL maintenance status "maintained".

This Current Maintainer of this work is Heiko Oberdiek.

The Base Interpreter refers to any `TeX-Format',
because some files are installed in TDS:tex/generic//.

This work consists of the main source file pdfescape.dtx
and the derived files
   pdfescape.sty, pdfescape.pdf, pdfescape.ins, pdfescape.drv,
   pdfescape-test1.tex, pdfescape-test2.tex,
   pdfescape-test3.tex, pdfescape-test4.tex,
   pdfescape-test5.tex, pdfescape-test6.tex.

\endpreamble
\let\MetaPrefix\DoubleperCent

\generate{%
  \file{pdfescape.ins}{\from{pdfescape.dtx}{install}}%
  \file{pdfescape.drv}{\from{pdfescape.dtx}{driver}}%
  \usedir{tex/generic/oberdiek}%
  \file{pdfescape.sty}{\from{pdfescape.dtx}{package}}%
  \usedir{doc/latex/oberdiek/test}%
  \file{pdfescape-test1.tex}{\from{pdfescape.dtx}{test1}}%
  \file{pdfescape-test2.tex}{\from{pdfescape.dtx}{test2}}%
  \file{pdfescape-test3.tex}{\from{pdfescape.dtx}{test3}}%
  \file{pdfescape-test4.tex}{\from{pdfescape.dtx}{test4}}%
  \file{pdfescape-test5.tex}{\from{pdfescape.dtx}{test5}}%
  \file{pdfescape-test6.tex}{\from{pdfescape.dtx}{test6}}%
  \nopreamble
  \nopostamble
  \usedir{source/latex/oberdiek/catalogue}%
  \file{pdfescape.xml}{\from{pdfescape.dtx}{catalogue}}%
}

\catcode32=13\relax% active space
\let =\space%
\Msg{************************************************************************}
\Msg{*}
\Msg{* To finish the installation you have to move the following}
\Msg{* file into a directory searched by TeX:}
\Msg{*}
\Msg{*     pdfescape.sty}
\Msg{*}
\Msg{* To produce the documentation run the file `pdfescape.drv'}
\Msg{* through LaTeX.}
\Msg{*}
\Msg{* Happy TeXing!}
\Msg{*}
\Msg{************************************************************************}

\endbatchfile
%</install>
%<*ignore>
\fi
%</ignore>
%<*driver>
\NeedsTeXFormat{LaTeX2e}
\ProvidesFile{pdfescape.drv}%
  [2011/11/25 v1.13 Implements pdfTeX's escape features (HO)]%
\documentclass{ltxdoc}
\usepackage{holtxdoc}[2011/11/22]
\begin{document}
  \DocInput{pdfescape.dtx}%
\end{document}
%</driver>
% \fi
%
% \CheckSum{2516}
%
% \CharacterTable
%  {Upper-case    \A\B\C\D\E\F\G\H\I\J\K\L\M\N\O\P\Q\R\S\T\U\V\W\X\Y\Z
%   Lower-case    \a\b\c\d\e\f\g\h\i\j\k\l\m\n\o\p\q\r\s\t\u\v\w\x\y\z
%   Digits        \0\1\2\3\4\5\6\7\8\9
%   Exclamation   \!     Double quote  \"     Hash (number) \#
%   Dollar        \$     Percent       \%     Ampersand     \&
%   Acute accent  \'     Left paren    \(     Right paren   \)
%   Asterisk      \*     Plus          \+     Comma         \,
%   Minus         \-     Point         \.     Solidus       \/
%   Colon         \:     Semicolon     \;     Less than     \<
%   Equals        \=     Greater than  \>     Question mark \?
%   Commercial at \@     Left bracket  \[     Backslash     \\
%   Right bracket \]     Circumflex    \^     Underscore    \_
%   Grave accent  \`     Left brace    \{     Vertical bar  \|
%   Right brace   \}     Tilde         \~}
%
% \GetFileInfo{pdfescape.drv}
%
% \title{The \xpackage{pdfescape} package}
% \date{2011/11/25 v1.13}
% \author{Heiko Oberdiek\\\xemail{heiko.oberdiek at googlemail.com}}
%
% \maketitle
%
% \begin{abstract}
% This package implements \pdfTeX's escape features
% (\cs{pdfescapehex}, \cs{pdfunescapehex}, \cs{pdfescapename},
% \cs{pdfescapestring}) using \TeX\ or \eTeX.
% \end{abstract}
%
% \tableofcontents
%
% \section{Documentation}
%
% \begin{declcs}{EdefEscapeHex} \M{cmd} \M{string}\\
%  \cs{EdefUnescapeHex} \M{cmd} \M{string}\\
%  \cs{EdefEscapeName} \M{cmd} \M{string}\\
%  \cs{EdefEscapeString} \M{cmd} \M{string}
% \end{declcs}
% These commands converts \meta{string} and stores the result
% in macro \meta{cmd}. The conversion result is the same
% as the conversion of the corresponding \pdfTeX's primitives.
% Note that the argument \meta{string} is expanded before
% the conversion.
%
% For example, if \pdfTeX\ \textgreater= 1.30 is present,
% then \cs{EdefEscapeHex}
% becomes to:
% \begin{quote}
%\begin{verbatim}
%\def\EdefEscapeHex#1#2{%
%  \edef#1{\pdfescapehex{#2}}%
%}
%\end{verbatim}
% \end{quote}
% The package provides implementations for the case that \pdfTeX\ is
% not present (or too old). Even \eTeX\ can be missing, however it is
% used if it is detected.
%
% \paragraph{Babel.}
% The input strings may contain shorthand characters of package
% \xpackage{babel}.
%
% \subsection{Additional unescape macros}
%
% \begin{declcs}{EdefUnescapeName} \M{cmd} \M{string}
% \end{declcs}
% Sequences of a hash sign with two hexadecimal digits are
% converted to the corresponding character (PDF-1.2).
% A hash sign that is not followed by two hexadecimal digits
% is left unchanged. The catcodes in the result string follow
% TeX's conventions. The space has catcode 10 (space) and
% the other characters have catcode 12 (other).
%
% \begin{declcs}{EdefUnescapeString} \M{cmd} \M{string}
% \end{declcs}
% Macro \meta{cmd} stores the unescaped string in \meta{string}.
% All the rules for literal strings are implemented, see
% PDF specification.
% The catcodes in the result string follow TeX's conventions.
%
% \subsection{Sanitizing macro}
%
% \begin{declcs}{EdefSanitize} \M{cmd} \M{string}
% \end{declcs}
% Argument \meta{string} is expanded, converted to a string
% of tokens with catcode 12 (other) and space tokens, and
% stored in macro \meta{cmd}.
%
% \StopEventually{
% }
%
% \section{Implementation}
%
%    \begin{macrocode}
%<*package>
%    \end{macrocode}
%
% \subsection{Reload check and package identification}
%    Reload check, especially if the package is not used with \LaTeX.
%    \begin{macrocode}
\begingroup\catcode61\catcode48\catcode32=10\relax%
  \catcode13=5 % ^^M
  \endlinechar=13 %
  \catcode35=6 % #
  \catcode39=12 % '
  \catcode44=12 % ,
  \catcode45=12 % -
  \catcode46=12 % .
  \catcode58=12 % :
  \catcode64=11 % @
  \catcode123=1 % {
  \catcode125=2 % }
  \expandafter\let\expandafter\x\csname ver@pdfescape.sty\endcsname
  \ifx\x\relax % plain-TeX, first loading
  \else
    \def\empty{}%
    \ifx\x\empty % LaTeX, first loading,
      % variable is initialized, but \ProvidesPackage not yet seen
    \else
      \expandafter\ifx\csname PackageInfo\endcsname\relax
        \def\x#1#2{%
          \immediate\write-1{Package #1 Info: #2.}%
        }%
      \else
        \def\x#1#2{\PackageInfo{#1}{#2, stopped}}%
      \fi
      \x{pdfescape}{The package is already loaded}%
      \aftergroup\endinput
    \fi
  \fi
\endgroup%
%    \end{macrocode}
%    Package identification:
%    \begin{macrocode}
\begingroup\catcode61\catcode48\catcode32=10\relax%
  \catcode13=5 % ^^M
  \endlinechar=13 %
  \catcode35=6 % #
  \catcode39=12 % '
  \catcode40=12 % (
  \catcode41=12 % )
  \catcode44=12 % ,
  \catcode45=12 % -
  \catcode46=12 % .
  \catcode47=12 % /
  \catcode58=12 % :
  \catcode64=11 % @
  \catcode91=12 % [
  \catcode93=12 % ]
  \catcode123=1 % {
  \catcode125=2 % }
  \expandafter\ifx\csname ProvidesPackage\endcsname\relax
    \def\x#1#2#3[#4]{\endgroup
      \immediate\write-1{Package: #3 #4}%
      \xdef#1{#4}%
    }%
  \else
    \def\x#1#2[#3]{\endgroup
      #2[{#3}]%
      \ifx#1\@undefined
        \xdef#1{#3}%
      \fi
      \ifx#1\relax
        \xdef#1{#3}%
      \fi
    }%
  \fi
\expandafter\x\csname ver@pdfescape.sty\endcsname
\ProvidesPackage{pdfescape}%
  [2011/11/25 v1.13 Implements pdfTeX's escape features (HO)]%
%    \end{macrocode}
%
% \subsection{Catcodes}
%
%    \begin{macrocode}
\begingroup\catcode61\catcode48\catcode32=10\relax%
  \catcode13=5 % ^^M
  \endlinechar=13 %
  \catcode123 1 % {
  \catcode125 2 % }
  \catcode64 11 %
  \def\x{\endgroup
    \expandafter\edef\csname PE@AtEnd\endcsname{%
      \endlinechar=\the\endlinechar\relax
      \catcode13=\the\catcode13\relax
      \catcode32=\the\catcode32\relax
      \catcode35=\the\catcode35\relax
      \catcode61=\the\catcode61\relax
      \catcode64=\the\catcode64\relax
      \catcode123=\the\catcode123\relax
      \catcode125=\the\catcode125\relax
    }%
  }%
\x\catcode61\catcode48\catcode32=10\relax%
\catcode13=5 % ^^M
\endlinechar=13 %
\catcode35=6 % #
\catcode64=11 % @
\catcode123=1 % {
\catcode125=2 % }
\def\TMP@EnsureCode#1#2#3{%
  \edef\PE@AtEnd{%
    \PE@AtEnd
    #1#2=\the#1#2\relax
  }%
  #1#2=#3\relax
}
\TMP@EnsureCode\catcode{0}{12}% ^^@
\TMP@EnsureCode\catcode{34}{12}% "
\TMP@EnsureCode\catcode{36}{3}% $
\TMP@EnsureCode\catcode{38}{4}% &
\TMP@EnsureCode\catcode{39}{12}% '
\TMP@EnsureCode\catcode{42}{12}% *
\TMP@EnsureCode\catcode{45}{12}% -
\TMP@EnsureCode\catcode{46}{12}% .
\TMP@EnsureCode\catcode{47}{12}% /
\TMP@EnsureCode\catcode{60}{12}% <
\TMP@EnsureCode\catcode{62}{12}% >
\TMP@EnsureCode\catcode{91}{12}% [
\TMP@EnsureCode\catcode{93}{12}% ]
\TMP@EnsureCode\catcode{94}{7}% ^
\TMP@EnsureCode\catcode{96}{12}% `
\TMP@EnsureCode\uccode{34}{0}% "
\TMP@EnsureCode\uccode{48}{0}% 0
\TMP@EnsureCode\uccode{61}{0}% =
\edef\PE@AtEnd{\PE@AtEnd\noexpand\endinput}
%    \end{macrocode}
%
% \subsection{Load package}
%
%    \begin{macrocode}
\begingroup\expandafter\expandafter\expandafter\endgroup
\expandafter\ifx\csname RequirePackage\endcsname\relax
  \def\TMP@RequirePackage#1[#2]{%
    \begingroup\expandafter\expandafter\expandafter\endgroup
    \expandafter\ifx\csname ver@#1.sty\endcsname\relax
      \input #1.sty\relax
    \fi
  }%
  \TMP@RequirePackage{ltxcmds}[2010/04/08]%
\else
  \RequirePackage{ltxcmds}[2010/04/08]%
\fi
%    \end{macrocode}
%
% \subsection{Sanitizing}
%
%    \begin{macro}{\EdefSanitize}
%    Macro \cs{EdefSanitize} takes |#2|, entirely converts
%    it to token with catcode 12 (other) and stores the
%    result in macro |#1|.
%    \begin{macrocode}
\begingroup\expandafter\expandafter\expandafter\endgroup
\expandafter\ifx\csname detokenize\endcsname\relax
  \long\def\EdefSanitize#1#2{%
    \begingroup
      \csname @safe@activestrue\endcsname
      \edef#1{#2}%
      \PE@onelevel@sanitize#1%
    \expandafter\endgroup
    \expandafter\def\expandafter#1\expandafter{#1}%
  }%
  \begingroup\expandafter\expandafter\expandafter\endgroup
  \expandafter\ifx\csname @onelevel@sanitize\endcsname\relax
    \def\PE@onelevel@sanitize#1{%
      \edef#1{\expandafter\PE@strip@prefix\meaning#1}%
    }%
    \def\PE@strip@prefix#1>{}%
  \else
    \let\PE@onelevel@sanitize\@onelevel@sanitize
  \fi
\else
  \long\def\EdefSanitize#1#2{%
    \begingroup
      \csname @safe@activestrue\endcsname
      \edef#1{#2}%
    \expandafter\endgroup
    \expandafter\def\expandafter#1\expandafter{%
      \detokenize\expandafter{#1}%
    }%
  }%
  \def\PE@onelevel@sanitize#1{%
    \edef#1{\detokenize\expandafter{#1}}%
  }%
\fi
%    \end{macrocode}
%    \end{macro}
%    \begin{macro}{\PE@sanitize}
%    Macro \cs{PE@sanitize} is only defined for
%    compatibility with version 1.4. Its use is deprecated.
%    \begin{macrocode}
\let\PE@sanitize\EdefSanitize
%    \end{macrocode}
%    \end{macro}
%
% \subsubsection{Space characters}
%
%    \begin{macro}{\PE@space@other}
%    \begin{macrocode}
\begingroup
  \catcode`\ =12\relax%
\def\x{\endgroup\def\PE@space@other{ }}\x\relax
%    \end{macrocode}
%    \end{macro}
%    \begin{macro}{\PE@space@space}
%    \begin{macrocode}
\def\PE@space@space{ }
%    \end{macrocode}
%    \end{macro}
%
% \subsubsection{Space normalization}
%
%    \begin{macro}{\PE@SanitizeSpaceOther}
%    \begin{macrocode}
\def\PE@SanitizeSpaceOther#1{%
  \edef#1{\expandafter\PE@SpaceToOther#1 \relax}%
}
%    \end{macrocode}
%    \end{macro}
%
%    \begin{macro}{\PE@SpaceToOther}
%    \begin{macrocode}
\def\PE@SpaceToOther#1 #2\relax{%
  #1%
  \ifx\\#2\\%
  \else
    \PE@space@other
    \ltx@ReturnAfterFi{%
      \PE@SpaceToOther#2\relax
    }%
  \fi
}
%    \end{macrocode}
%    \end{macro}
%
% \subsection{\cs{EdefUnescapeName}}
%
%    \begin{macro}{\EdefUnescapeName}
%    \begin{macrocode}
\def\EdefUnescapeName#1#2{%
  \EdefSanitize#1{#2}%
  \PE@SanitizeSpaceOther#1%
  \PE@UnescapeName#1%
  \PE@onelevel@sanitize#1%
}
%    \end{macrocode}
%    \end{macro}
%
%    \begin{macro}{\PE@UnescapeName}
%    \begin{macrocode}
\begingroup
  \catcode`\$=6  % hash
  \catcode`\#=12 % other
  \gdef\PE@UnescapeName$1{%
    \begingroup
      \PE@InitUccodeHexDigit
      \def\PE@result{}%
      \expandafter\PE@DeName$1#\relax\relax
    \expandafter\endgroup
    \expandafter\def\expandafter$1\expandafter{\PE@result}%
  }%
  \gdef\PE@DeName$1#$2$3{%
    \ifx\relax$2%
      \edef\PE@result{\PE@result$1}%
      \let\PE@next\relax
    \else
      \ifx\relax$3%
        % wrong escape sequence in input
        \edef\PE@result{\PE@result$1#}%
        \let\PE@next\relax
      \else
        \uppercase{%
          \def\PE@testA{$2}%
          \def\PE@testB{$3}%
        }%
        \ifcase\ifcase\expandafter\PE@TestUcHexDigit\PE@testA
                 \ifcase\expandafter\PE@TestUcHexDigit\PE@testB
                   \ltx@zero
                 \else
                   \ltx@one
                 \fi
               \else
                 \ltx@one
               \fi
          \uccode\ltx@zero="\PE@testA\PE@testB\relax
          \uppercase{%
            \def\PE@temp{^^@}%
          }%
          \uccode\ltx@zero=\ltx@zero
          \edef\PE@result{\PE@result$1\PE@temp}%
          \let\PE@next\PE@DeName
        \else
          % wrong escape sequence in input
          \edef\PE@result{\PE@result$1#}%
          \def\PE@next{\PE@DeName$2$3}%
        \fi
      \fi
    \fi
    \PE@next
  }%
\endgroup
%    \end{macrocode}
%    \end{macro}
%
%    \begin{macro}{\PE@InitUccodeHexDigit}
%    \begin{macrocode}
\def\PE@InitUccodeHexDigit{%
  \uccode`a=`A\relax
  \uccode`b=`B\relax
  \uccode`c=`C\relax
  \uccode`d=`D\relax
  \uccode`e=`E\relax
  \uccode`f=`F\relax
  \uccode`A=\ltx@zero
  \uccode`B=\ltx@zero
  \uccode`C=\ltx@zero
  \uccode`D=\ltx@zero
  \uccode`E=\ltx@zero
  \uccode`F=\ltx@zero
  \uccode`0=\ltx@zero
  \uccode`1=\ltx@zero
  \uccode`2=\ltx@zero
  \uccode`3=\ltx@zero
  \uccode`4=\ltx@zero
  \uccode`5=\ltx@zero
  \uccode`6=\ltx@zero
  \uccode`7=\ltx@zero
  \uccode`8=\ltx@zero
  \uccode`9=\ltx@zero
}
%    \end{macrocode}
%    \end{macro}
%
%    \begin{macro}{\PE@TestUcHexDigit}
%    \begin{macrocode}
\def\PE@TestUcHexDigit#1{%
  \ifnum`#1<48 % 0
    \ltx@one
  \else
    \ifnum`#1>70 % F
      \ltx@one
    \else
      \ifnum`#1>57 % 9
        \ifnum`#1<65 % A
          \ltx@one
        \else
          \ltx@zero
        \fi
      \else
        \ltx@zero
      \fi
    \fi
  \fi
}
%    \end{macrocode}
%    \end{macro}
%
% \subsection{\cs{EdefUnescapeString}}
%
%    \begin{macro}{\EdefUnescapeString}
%    \begin{macrocode}
\def\EdefUnescapeString#1#2{%
  \EdefSanitize#1{#2}%
  \PE@SanitizeSpaceOther#1%
  \PE@NormalizeLineEnd#1%
  \PE@UnescapeString#1%
  \PE@onelevel@sanitize#1%
}
%    \end{macrocode}
%    \end{macro}
%
%    \begin{macrocode}
\begingroup
  \uccode`\8=10 % lf
  \uccode`\9=13 % cr
\def\x#1#2{\endgroup
%    \end{macrocode}
%    \begin{macro}{\PE@NormalizeLineEnd}
%    \begin{macrocode}
  \def\PE@NormalizeLineEnd##1{%
    \def\PE@result{}%
    \expandafter\PE@@NormalizeLineEnd##1#2\relax
    \let##1\PE@result
  }%
%    \end{macrocode}
%    \end{macro}
%    \begin{macro}{\PE@@NormalizeLineEnd}
%    \begin{macrocode}
  \def\PE@@NormalizeLineEnd##1#2##2{%
    \ifx\relax##2%
      \edef\PE@result{\PE@result##1}%
      \let\PE@next\relax
    \else
      \edef\PE@result{\PE@result##1#1}%
      \ifx#1##2% lf
        \let\PE@next\PE@@NormalizeLineEnd
      \else
        \def\PE@next{\PE@@NormalizeLineEnd##2}%
      \fi
    \fi
    \PE@next
  }%
}%
\uppercase{%
  \x 89%
}
%    \end{macrocode}
%    \end{macro}
%
%    \begin{macrocode}
\begingroup
  \catcode`\|=0 %
  \catcode`\\=12 %
%    \end{macrocode}
%    \begin{macro}{\PE@UnescapeString}
%    \begin{macrocode}
  |gdef|PE@UnescapeString#1{%
    |begingroup
      |def|PE@result{}%
      |expandafter|PE@DeString#1\|relax
    |expandafter|endgroup
    |expandafter|def|expandafter#1|expandafter{|PE@result}%
  }%
%    \end{macrocode}
%    \end{macro}
%    \begin{macro}{\PE@DeString}
%    \begin{macrocode}
  |gdef|PE@DeString#1\#2{%
    |ifx|relax#2%
      |edef|PE@result{|PE@result#1}%
      |let|PE@next|relax
    |else
      |if n#2%
        |uccode|ltx@zero=10 %
      |else|if r#2%
        |uccode|ltx@zero=13 %
      |else|if t#2%
        |uccode|ltx@zero=9 %
      |else|if b#2%
        |uccode|ltx@zero=8 %
      |else|if f#2%
        |uccode|ltx@zero=12 %
      |else
        |uccode|ltx@zero=|ltx@zero
      |fi|fi|fi|fi|fi
      |ifnum|uccode|ltx@zero>|ltx@zero
        |uppercase{%
          |edef|PE@temp{^^@}%
        }%
        |edef|PE@result{|PE@result#1|PE@temp}%
        |let|PE@next|PE@DeString
      |else
        |if\#2% backslash
          |edef|PE@result{|PE@result#1}%
          |let|PE@next|PE@CheckEndBackslash
        |else
          |ifnum`#2=10 % linefeed
            |edef|PE@result{|PE@result#1}%
            |let|PE@next|PE@DeString
          |else
            |ifcase|PE@TestOctDigit#2%
              |edef|PE@result{|PE@result#1}%
              |def|PE@next{|PE@OctI#2}%
            |else
              |edef|PE@result{|PE@result#1#2}%
              |let|PE@next|PE@DeString
            |fi
          |fi
        |fi
      |fi
    |fi
    |PE@next
  }%
%    \end{macrocode}
%    \end{macro}
%    \begin{macro}{\PE@CheckEndBackslash}
%    \begin{macrocode}
  |gdef|PE@CheckEndBackslash#1{%
    |ifx|relax#1%
    |else
      |edef|PE@result{|PE@result\}%
      |expandafter|PE@DeString|expandafter#1%
    |fi
  }%
%    \end{macrocode}
%    \end{macro}
%    \begin{macrocode}
|endgroup
%    \end{macrocode}
%    \begin{macro}{\PE@TestOctDigit}
%    \begin{macrocode}
\def\PE@TestOctDigit#1{%
  \ifnum`#1<48 % 0
    \ltx@one
  \else
    \ifnum`#1>55 % 7
      \ltx@one
    \else
      \ltx@zero
    \fi
  \fi
}
%    \end{macrocode}
%    \end{macro}
%    \begin{macro}{\PE@OctI}
%    \begin{macrocode}
\def\PE@OctI#1#2{%
  \ifcase\PE@TestOctDigit#2%
    \def\PE@next{\PE@OctII{#1#2}}%
  \else
    \def\PE@next{\PE@OctAll#1#2}%
  \fi
  \PE@next
}
%    \end{macrocode}
%    \end{macro}
%    \begin{macro}{\PE@OctII}
%    \begin{macrocode}
\def\PE@OctII#1#2{%
  \ifcase\PE@TestOctDigit#2%
    \def\PE@next{\PE@OctIII#1#2}%
  \else
    \def\PE@next{\PE@OctAll{#1}#2}%
  \fi
  \PE@next
}
%    \end{macrocode}
%    \end{macro}
%    \begin{macrocode}
\ltx@IfUndefined{numexpr}{%
  \catcode`\$=9 %
  \catcode`\&=14 %
}{%
  \catcode`\$=14 %
  \catcode`\&=9 %
}
%    \end{macrocode}
%    \begin{macro}{\PE@OctIII}
%    \begin{macrocode}
\def\PE@OctIII#1#2#3{%
  \ifnum#1<4 %
    \def\PE@next{\PE@OctAll{#1#2#3}}%
  \else
$   \count\ltx@cclv#1 %
$   \advance\count\ltx@cclv -4 %
    \edef\PE@next{%
      \noexpand\PE@OctAll{%
$       \the\count\ltx@cclv
&       \the\numexpr#1-4\relax
        #2#3%
      }%
    }%
  \fi
  \PE@next
}
%    \end{macrocode}
%    \end{macro}
%    \begin{macro}{\PE@OctAll}
%    \begin{macrocode}
\def\PE@OctAll#1{%
  \uccode\ltx@zero='#1\relax
  \uppercase{%
    \edef\PE@result{\PE@result^^@}%
  }%
  \PE@DeString
}
%    \end{macrocode}
%    \end{macro}
%
% \subsection{User macros (\pdfTeX\space analogues)}
%
%    \begin{macrocode}
\begingroup\expandafter\expandafter\expandafter\endgroup
\expandafter\ifx\csname RequirePackage\endcsname\relax
  \def\TMP@RequirePackage#1[#2]{%
    \begingroup\expandafter\expandafter\expandafter\endgroup
    \expandafter\ifx\csname ver@#1.sty\endcsname\relax
      \input #1.sty\relax
    \fi
  }%
  \TMP@RequirePackage{pdftexcmds}[2007/11/11]%
\else
  \RequirePackage{pdftexcmds}[2007/11/11]%
\fi
%    \end{macrocode}
%
%    \begin{macrocode}
\begingroup\expandafter\expandafter\expandafter\endgroup
\expandafter\ifx\csname pdf@escapehex\endcsname\relax
%    \end{macrocode}
%
%    \begin{macro}{\EdefEscapeHex}
%    \begin{macrocode}
  \long\def\EdefEscapeHex#1#2{%
    \EdefSanitize#1{#2}%
    \PE@SanitizeSpaceOther#1%
    \PE@EscapeHex#1%
  }%
%    \end{macrocode}
%    \end{macro}
%    \begin{macro}{\EdefUnescapeHex}
%    \begin{macrocode}
  \def\EdefUnescapeHex#1#2{%
    \EdefSanitize#1{#2}%
    \PE@UnescapeHex#1%
  }%
%    \end{macrocode}
%    \end{macro}
%    \begin{macro}{\EdefEscapeName}
%    \begin{macrocode}
  \long\def\EdefEscapeName#1#2{%
    \EdefSanitize#1{#2}%
    \PE@SanitizeSpaceOther#1%
    \PE@EscapeName#1%
  }%
%    \end{macrocode}
%    \end{macro}
%    \begin{macro}{\EdefEscapeString}
%    \begin{macrocode}
  \long\def\EdefEscapeString#1#2{%
    \EdefSanitize#1{#2}%
    \PE@SanitizeSpaceOther#1%
    \PE@EscapeString#1%
  }%
%    \end{macrocode}
%    \end{macro}
%
%    \begin{macrocode}
\else
%    \end{macrocode}
%
%    \begin{macro}{\PE@edefbabel}
%    Help macro that adds support for \xpackage{babel}'s shorthand
%    characters.
%    \begin{macrocode}
  \long\def\PE@edefbabel#1#2#3{%
    \begingroup
      \csname @save@activestrue\endcsname
      \edef#1{#2{#3}}%
    \expandafter\endgroup
    \expandafter\def\expandafter#1\expandafter{#1}%
  }%
%    \end{macrocode}
%    \end{macro}
%
%    \begin{macro}{\EdefEscapeHex}
%    \begin{macrocode}
  \long\def\EdefEscapeHex#1#2{%
    \PE@edefbabel#1\pdf@escapehex{#2}%
  }%
%    \end{macrocode}
%    \end{macro}
%    \begin{macro}{\EdefUnescapeHex}
%    \begin{macrocode}
  \def\EdefUnescapeHex#1#2{%
    \PE@edefbabel#1\pdf@unescapehex{#2}%
  }%
%    \end{macrocode}
%    \end{macro}
%    \begin{macro}{\EdefEscapeName}
%    \begin{macrocode}
  \long\def\EdefEscapeName#1#2{%
    \PE@edefbabel#1\pdf@escapename{#2}%
  }%
%    \end{macrocode}
%    \end{macro}
%    \begin{macro}{\EdefEscapeString}
%    \begin{macrocode}
  \long\def\EdefEscapeString#1#2{%
    \PE@edefbabel#1\pdf@escapestring{#2}%
  }%
%    \end{macrocode}
%    \end{macro}
%
%    \begin{macrocode}
  \expandafter\PE@AtEnd
\fi%
%    \end{macrocode}
%
% \subsection{Help macros}
%
% \subsubsection{Characters}
%
%    Special characters with catcode 12 (other) are created and
%    stored in macros.
%    \begin{macro}{\PE@hash}
%    \begin{macrocode}
\edef\PE@hash{\string#}
%    \end{macrocode}
%    \end{macro}
%    \begin{macro}{\PE@backslash}
%    \begin{macrocode}
\begingroup
  \escapechar=-1 %
\edef\x{\endgroup
  \def\noexpand\PE@backslash{\string\\}%
}
\x
%    \end{macrocode}
%    \end{macro}
%
% \subsubsection{Switch for \eTeX}
%
%    \begin{macrocode}
\ltx@newif\ifPE@etex
\begingroup\expandafter\expandafter\expandafter\endgroup
\expandafter\ifx\csname numexpr\endcsname\relax
\else
  \PE@etextrue
\fi
%    \end{macrocode}
%
% \subsection{Conversions}
%
% \subsubsection{Conversion to hex string}
%
%    \begin{macro}{\PE@EscapeHex}
%    \begin{macrocode}
\ifPE@etex
  \def\PE@EscapeHex#1{%
    \edef#1{\expandafter\PE@ToHex#1\relax}%
  }%
\else
  \def\PE@EscapeHex#1{%
    \def\PE@result{}%
    \expandafter\PE@ToHex#1\relax
    \let#1\PE@result
  }%
\fi
%    \end{macrocode}
%    \end{macro}
%    \begin{macro}{\PE@ToHex}
%    \begin{macrocode}
\def\PE@ToHex#1{%
  \ifx\relax#1%
  \else
    \PE@HexChar{#1}%
    \expandafter\PE@ToHex
  \fi
}%
%    \end{macrocode}
%    \end{macro}
%    \begin{macro}{\PE@HexChar}
%    \begin{macrocode}
\ifPE@etex
  \def\PE@HexChar#1{%
    \PE@HexDigit{\numexpr\dimexpr.0625\dimexpr`#1sp\relax\relax\relax}%
    \PE@HexDigit{%
      \numexpr`#1-16*\dimexpr.0625\dimexpr`#1sp\relax\relax\relax
    }%
  }%
\else
  \def\PE@HexChar#1{%
    \dimen0=`#1sp%
    \dimen2=.0625\dimen0 %
    \advance\dimen0-16\dimen2 %
    \edef\PE@result{%
      \PE@result
      \PE@HexDigit{\dimen2 }%
      \PE@HexDigit{\dimen0 }%
    }%
  }%
\fi
%    \end{macrocode}
%    \end{macro}
%    \begin{macro}{\PE@HexDigit}
%    \begin{macrocode}
\def\PE@HexDigit#1{%
  \expandafter\string
  \ifcase#1%
    0\or 1\or 2\or 3\or 4\or 5\or 6\or 7\or 8\or 9\or
    A\or B\or C\or D\or E\or F%
  \fi
}
%    \end{macrocode}
%    \end{macro}
%
% \subsubsection{Character code to octal number}
%
%    \begin{macro}{\PE@OctChar}
%    \begin{macrocode}
\ifPE@etex
  \def\PE@OctChar#1{%
    \expandafter\PE@@OctChar
        \the\numexpr\dimexpr.015625\dimexpr`#1sp\relax\relax
            \expandafter\relax
        \expandafter\relax
        \the\numexpr\dimexpr.125\dimexpr`#1sp\relax\relax\relax
        \relax
        #1%
  }%
  \def\PE@@OctChar#1\relax#2\relax#3{%
    \PE@backslash
    #1%
    \the\numexpr#2-8*#1\relax
    \the\numexpr\dimexpr`#3sp\relax-8*#2\relax
  }%
\else
  \def\PE@OctChar#1{%
    \dimen0=`#1sp%
    \dimen2=.125\dimen0 %
    \dimen4=.125\dimen2 %
    \advance\dimen0-8\dimen2 %
    \advance\dimen2-8\dimen4 %
    \edef\PE@result{%
      \PE@result
      \PE@backslash
      \number\dimen4 %
      \number\dimen2 %
      \number\dimen0 %
    }%
  }%
\fi
%    \end{macrocode}
%    \end{macro}
%
% \subsubsection{Unpack hex string}
%
%    \begin{macro}{\PE@UnescapeHex}
%    \begin{macrocode}
\def\PE@UnescapeHex#1{%
  \begingroup
    \PE@InitUccodeHexDigit
    \def\PE@result{}%
    \expandafter\PE@DeHex#1\relax\relax
  \expandafter\endgroup
  \expandafter\def\expandafter#1\expandafter{\PE@result}%
}
%    \end{macrocode}
%    \end{macro}
%    \begin{macro}{\PE@DeHex}
%    \begin{macrocode}
\def\PE@DeHex#1#2{%
  \ifx#2\relax
    \ifx#1\relax
      \let\PE@next\relax
    \else
      \uppercase{%
        \def\PE@testA{#1}%
      }%
      \ifcase\expandafter\PE@TestUcHexDigit\PE@testA
        \def\PE@next{%
          \PE@DeHex#10\relax\relax
        }%
      \else
        \let\PE@next\relax
      \fi
    \fi
  \else
    \uppercase{%
      \def\PE@testA{#1}%
      \def\PE@testB{#2}%
    }%
    \ifcase\expandafter\PE@TestUcHexDigit\PE@testA
      \ifcase\expandafter\PE@TestUcHexDigit\PE@testB
        \uccode\ltx@zero="\PE@testA\PE@testB\relax
        \ifnum\uccode\ltx@zero=32 %
          \let\PE@temp\PE@space@space
        \else
          \uppercase{%
            \def\PE@temp{^^@}%
          }%
        \fi
        \edef\PE@result{\PE@result\PE@temp}%
        \let\PE@next\PE@DeHex
      \else
        % invalid input sequence
        \def\PE@next{%
          \PE@DeHex#1%
        }%
      \fi
    \else
      % invalid input sequence
      \def\PE@next{\PE@DeHex#2}%
    \fi
  \fi
  \PE@next
}
%    \end{macrocode}
%    \end{macro}
%
% \subsubsection{Conversion to PDF name}
%
%    \begin{macro}{\PE@EscapeName}
%    \begin{macrocode}
\ifPE@etex
  \def\PE@EscapeName#1{%
    \edef#1{\expandafter\PE@EscapeNameTokens#1\relax}%
  }%
\else
  \def\PE@EscapeName#1{%
    \def\PE@result{}%
    \expandafter\PE@EscapeNameTokens#1\relax
    \let#1\PE@result
  }%
\fi
%    \end{macrocode}
%    \end{macro}
%    \begin{macro}{\PE@EscapeNameTokens}
%    \begin{macrocode}
\def\PE@EscapeNameTokens#1{%
  \ifx\relax#1%
  \else
    \ifnum`#1<33 %
      \ifcase`#1 %
        % drop illegal zero
      \else
        \PE@EscapeNameAdd\PE@hash
        \PE@HexChar#1%
      \fi
    \else
      \ifnum`#1>126 %
        \PE@EscapeNameAdd\PE@hash
        \PE@HexChar#1%
      \else  \ifnum`#1=35  \PE@EscapeNameHashChar 23% #
        \else\ifnum`#1=37  \PE@EscapeNameHashChar 25% %
        \else\ifnum`#1=40  \PE@EscapeNameHashChar 28% (
        \else\ifnum`#1=41  \PE@EscapeNameHashChar 29% )
        \else\ifnum`#1=47  \PE@EscapeNameHashChar 2F% /
        \else\ifnum`#1=60  \PE@EscapeNameHashChar 3C% <
        \else\ifnum`#1=62  \PE@EscapeNameHashChar 3E% >
        \else\ifnum`#1=91  \PE@EscapeNameHashChar 5B% [
        \else\ifnum`#1=93  \PE@EscapeNameHashChar 5D% ]
        \else\ifnum`#1=123 \PE@EscapeNameHashChar 7B% {
        \else\ifnum`#1=125 \PE@EscapeNameHashChar 7D% }
        \else
          \PE@EscapeNameAdd{#1}%
        \fi\fi\fi\fi\fi\fi\fi\fi\fi\fi\fi
      \fi
    \fi
    \expandafter\PE@EscapeNameTokens
  \fi
}%
\def\PE@EscapeNameHashChar#1#2{%
  \PE@EscapeNameAdd{\PE@hash\string#1\string#2}%
}%
%    \end{macrocode}
%    \end{macro}
%    \begin{macro}{\PE@EscapeNameAdd}
%    \begin{macrocode}
\ifPE@etex
  \def\PE@EscapeNameAdd#1{#1}%
\else
  \def\PE@EscapeNameAdd#1{%
    \edef\PE@result{%
      \PE@result
      #1%
    }%
  }%
\fi
%    \end{macrocode}
%    \end{macro}
%
% \subsubsection{Conversion to PDF string}
%
%    \begin{macro}{\PE@EscapeString}
%    \begin{macrocode}
\ifPE@etex
  \def\PE@EscapeString#1{%
    \edef#1{\expandafter\PE@EscapeStringTokens#1\relax}%
  }%
\else
  \def\PE@EscapeString#1{%
    \begingroup
      \def\PE@result{}%
      \expandafter\PE@EscapeStringTokens#1\relax
    \expandafter\endgroup
    \expandafter\def\expandafter#1\expandafter{\PE@result}%
  }%
\fi
%    \end{macrocode}
%    \end{macro}
%    \begin{macro}{\PE@EscapeStringTokens}
%    \begin{macrocode}
\def\PE@EscapeStringTokens#1{%
  \ifx\relax#1%
  \else
    \ifnum`#1<33 %
      \PE@OctChar#1%
    \else
      \ifnum`#1>126 %
        \PE@OctChar#1%
      \else  \ifnum`#1=40 \PE@EscapeStringAdd{\string\(}% (
        \else\ifnum`#1=41 \PE@EscapeStringAdd{\string\)}% )
        \else\ifnum`#1=92 \PE@EscapeStringAdd{\string\\}% \
        \else
          \PE@EscapeStringAdd{#1}%
        \fi\fi\fi
      \fi
    \fi
    \expandafter\PE@EscapeStringTokens
  \fi
}%
%    \end{macrocode}
%    \end{macro}
%    \begin{macro}{\PE@EscapeStringAdd}
%    \begin{macrocode}
\ifPE@etex
  \def\PE@EscapeStringAdd#1{#1}%
\else
  \def\PE@EscapeStringAdd#1{%
    \edef\PE@result{%
      \PE@result
      #1%
    }%
  }%
\fi
%    \end{macrocode}
%    \end{macro}
%
%    \begin{macrocode}
\PE@AtEnd%
%    \end{macrocode}
%
%    \begin{macrocode}
%</package>
%    \end{macrocode}
%
% \section{Test}
%
% \subsection{Catcode checks for loading}
%
%    \begin{macrocode}
%<*test1>
%    \end{macrocode}
%    \begin{macrocode}
\catcode`\{=1 %
\catcode`\}=2 %
\catcode`\#=6 %
\catcode`\@=11 %
\expandafter\ifx\csname count@\endcsname\relax
  \countdef\count@=255 %
\fi
\expandafter\ifx\csname @gobble\endcsname\relax
  \long\def\@gobble#1{}%
\fi
\expandafter\ifx\csname @firstofone\endcsname\relax
  \long\def\@firstofone#1{#1}%
\fi
\expandafter\ifx\csname loop\endcsname\relax
  \expandafter\@firstofone
\else
  \expandafter\@gobble
\fi
{%
  \def\loop#1\repeat{%
    \def\body{#1}%
    \iterate
  }%
  \def\iterate{%
    \body
      \let\next\iterate
    \else
      \let\next\relax
    \fi
    \next
  }%
  \let\repeat=\fi
}%
\def\RestoreCatcodes{}
\count@=0 %
\loop
  \edef\RestoreCatcodes{%
    \RestoreCatcodes
    \catcode\the\count@=\the\catcode\count@\relax
  }%
\ifnum\count@<255 %
  \advance\count@ 1 %
\repeat

\def\RangeCatcodeInvalid#1#2{%
  \count@=#1\relax
  \loop
    \catcode\count@=15 %
  \ifnum\count@<#2\relax
    \advance\count@ 1 %
  \repeat
}
\def\RangeCatcodeCheck#1#2#3{%
  \count@=#1\relax
  \loop
    \ifnum#3=\catcode\count@
    \else
      \errmessage{%
        Character \the\count@\space
        with wrong catcode \the\catcode\count@\space
        instead of \number#3%
      }%
    \fi
  \ifnum\count@<#2\relax
    \advance\count@ 1 %
  \repeat
}
\def\space{ }
\expandafter\ifx\csname LoadCommand\endcsname\relax
  \def\LoadCommand{\input pdfescape.sty\relax}%
\fi
\def\Test{%
  \RangeCatcodeInvalid{0}{47}%
  \RangeCatcodeInvalid{58}{64}%
  \RangeCatcodeInvalid{91}{96}%
  \RangeCatcodeInvalid{123}{255}%
  \catcode`\@=12 %
  \catcode`\\=0 %
  \catcode`\%=14 %
  \LoadCommand
  \RangeCatcodeCheck{0}{36}{15}%
  \RangeCatcodeCheck{37}{37}{14}%
  \RangeCatcodeCheck{38}{47}{15}%
  \RangeCatcodeCheck{48}{57}{12}%
  \RangeCatcodeCheck{58}{63}{15}%
  \RangeCatcodeCheck{64}{64}{12}%
  \RangeCatcodeCheck{65}{90}{11}%
  \RangeCatcodeCheck{91}{91}{15}%
  \RangeCatcodeCheck{92}{92}{0}%
  \RangeCatcodeCheck{93}{96}{15}%
  \RangeCatcodeCheck{97}{122}{11}%
  \RangeCatcodeCheck{123}{255}{15}%
  \RestoreCatcodes
}
\Test
\csname @@end\endcsname
\end
%    \end{macrocode}
%    \begin{macrocode}
%</test1>
%    \end{macrocode}
%
% \subsection{Macro tests}
%
%    \begin{macrocode}
%<*test2|test3|test4|test5>
\NeedsTeXFormat{LaTeX2e}
\makeatletter
%    \end{macrocode}
%
% \subsection{Test with \cs{pdfescape...} commands}
%
%    \begin{macrocode}
%<*test2>
\ProvidesFile{pdfescape-test2.tex}%
    [2011/11/25 v1.13 Test with \string\pdfescape... commands]%
%</test2>
%    \end{macrocode}
%
% \subsection{Test without \cs{pdfescape...}, with \eTeX}
%
%    \begin{macrocode}
%<*test3>
\ProvidesFile{pdfescape-test3.tex}%
    [2011/11/25 v1.13 Test without \string\pdfescape..., with e-TeX]%
%</test3>
%    \end{macrocode}
%
% \subsection{Test without \cs{pdfescape...} and \eTeX}
%
%    \begin{macrocode}
%<*test4>
\ProvidesFile{pdfescape-test4.tex}%
    [2011/11/25 v1.13 Test without \string\pdfescape... and e-TeX]%
%</test4>
%    \end{macrocode}
%
% \subsection{Test with \LuaTeX}
%
%    \begin{macrocode}
%<*test5>
\ProvidesFile{pdfescape-test5.tex}%
    [2011/11/25 v1.13 Test with LuaTeX]%
%</test5>
%    \end{macrocode}
%
% \subsection{Check/ensure test preconditions}
%
% \subsubsection{Check \cs{pdfescape...}}
%
%    \begin{macrocode}
%<*test2>
\@ifundefined{pdfescapehex}{%
  \PackageError{pdfescape-test2}{%
    Missing \string\pdfescape... commands%
  }{Test aborted.}%
  \stop
}{}
%</test2>
%    \end{macrocode}
%    \begin{macrocode}
%<*test3|test4>
\let\pdfescapehex\@undefined
\let\pdfunescapehex\@undefined
\let\pdfescapename\@undefined
\let\pdfescapestring\@undefined
%</test3|test4>
%    \end{macrocode}
%
% \subsubsection{Check \eTeX}
%
%    \begin{macrocode}
%<*test3>
\@ifundefined{numexpr}{%
  \PackageError{pdfescape-test3}{%
    Missing \eTeX
  }{Test aborted.}%
  \stop
}{}
%</test3>
%    \end{macrocode}
% Package \xpackage{qstest} uses \eTeX, thus \eTeX's features
% can only be disabled later during loading of package \xpackage{pdfescape}.
%
% \subsubsection{Check \LuaTeX}
%
%    \begin{macrocode}
%<*test5>
\@ifundefined{directlua}{%
  \PackageError{pdfescape-test5}{%
    Missing LuaTeX%
  }{Test aborted.}%
  \stop
}{}
%</test5>
%    \end{macrocode}
%
% \subsection{Common part}
%
%    The files for testing uses the framework, provided by the
%    new package \xpackage{qstest} of David Kastrup.
%    \begin{macrocode}
\RequirePackage{qstest}
\IncludeTests{*}
\LogTests{log}{*}{*}

\newcommand*{\ExpectVar}[2]{%
  \ifx#1#2%
  \else
    \begingroup
      \@onelevel@sanitize#1%
      \@onelevel@sanitize#2%
      \typeout{[#1] <> [#2]}% hash-ok
    \endgroup
  \fi
  \Expect*{\ifx#1#2true\else false\fi}{true}%
}

\makeatletter
\begingroup
  \gdef\AllBytes{}%
  \count@=0 %
  \catcode0=12 %
  \@whilenum\count@<256 \do{%
    \lccode0=\count@
    \ifnum\count@=32 %
      \xdef\AllBytes{\AllBytes\space}%
    \else
      \lowercase{%
        \xdef\AllBytes{\AllBytes^^@}%
      }%
    \fi
    \advance\count@ by 1 %
  }%
\endgroup
\newcommand*{\AllBytesHex}{%
  000102030405060708090A0B0C0D0E0F%
  101112131415161718191A1B1C1D1E1F%
  202122232425262728292A2B2C2D2E2F%
  303132333435363738393A3B3C3D3E3F%
  404142434445464748494A4B4C4D4E4F%
  505152535455565758595A5B5C5D5E5F%
  606162636465666768696A6B6C6D6E6F%
  707172737475767778797A7B7C7D7E7F%
  808182838485868788898A8B8C8D8E8F%
  909192939495969798999A9B9C9D9E9F%
  A0A1A2A3A4A5A6A7A8A9AAABACADAEAF%
  B0B1B2B3B4B5B6B7B8B9BABBBCBDBEBF%
  C0C1C2C3C4C5C6C7C8C9CACBCCCDCECF%
  D0D1D2D3D4D5D6D7D8D9DADBDCDDDEDF%
  E0E1E2E3E4E5E6E7E8E9EAEBECEDEEEF%
  F0F1F2F3F4F5F6F7F8F9FAFBFCFDFEFF%
}
\@onelevel@sanitize\AllBytesHex
\expandafter\lowercase\expandafter{%
  \expandafter\newcommand\expandafter*\expandafter\AllBytesHexLC
      \expandafter{\AllBytesHex}%
}
\newcommand*{\AllBytesName}{}
\begingroup
  \catcode`\#=12 %
  \xdef\AllBytesName{%
    #01#02#03#04#05#06#07#08#09#0A#0B#0C#0D#0E#0F%
    #10#11#12#13#14#15#16#17#18#19#1A#1B#1C#1D#1E#1F%
    #20!"#23$#25&'#28#29*+,-.#2F%
    0123456789:;#3C=#3E?%
    @ABCDEFGHIJKLMNO%
    PQRSTUVWXYZ#5B\@backslashchar#5D^_%
    `abcdefghijklmno%
    pqrstuvwxyz#7B|#7D\string~#7F%
    #80#81#82#83#84#85#86#87#88#89#8A#8B#8C#8D#8E#8F%
    #90#91#92#93#94#95#96#97#98#99#9A#9B#9C#9D#9E#9F%
    #A0#A1#A2#A3#A4#A5#A6#A7#A8#A9#AA#AB#AC#AD#AE#AF%
    #B0#B1#B2#B3#B4#B5#B6#B7#B8#B9#BA#BB#BC#BD#BE#BF%
    #C0#C1#C2#C3#C4#C5#C6#C7#C8#C9#CA#CB#CC#CD#CE#CF%
    #D0#D1#D2#D3#D4#D5#D6#D7#D8#D9#DA#DB#DC#DD#DE#DF%
    #E0#E1#E2#E3#E4#E5#E6#E7#E8#E9#EA#EB#EC#ED#EE#EF%
    #F0#F1#F2#F3#F4#F5#F6#F7#F8#F9#FA#FB#FC#FD#FE#FF%
  }%
\endgroup
\@onelevel@sanitize\AllBytesName

\newcommand*{\AllBytesString}{}
\begingroup
  \def\|{|}%
  \edef\%{\@percentchar}%
  \catcode`\|=0 %
  \catcode`\#=12 %
  \catcode`\~=12 %
  \catcode`\\=12 %
  |xdef|AllBytesString{%
    \000\001\002\003\004\005\006\007\010\011\012\013\014\015\016\017%
    \020\021\022\023\024\025\026\027\030\031\032\033\034\035\036\037%
    \040!"#$|%&'\(\)*+,-./%
    0123456789:;<=>?%
    @ABCDEFGHIJKLMNO%
    PQRSTUVWXYZ[\\]^_%
    `abcdefghijklmno%
    pqrstuvwxyz{||}~\177%
    \200\201\202\203\204\205\206\207\210\211\212\213\214\215\216\217%
    \220\221\222\223\224\225\226\227\230\231\232\233\234\235\236\237%
    \240\241\242\243\244\245\246\247\250\251\252\253\254\255\256\257%
    \260\261\262\263\264\265\266\267\270\271\272\273\274\275\276\277%
    \300\301\302\303\304\305\306\307\310\311\312\313\314\315\316\317%
    \320\321\322\323\324\325\326\327\330\331\332\333\334\335\336\337%
    \340\341\342\343\344\345\346\347\350\351\352\353\354\355\356\357%
    \360\361\362\363\364\365\366\367\370\371\372\373\374\375\376\377%
  }%
|endgroup
\@onelevel@sanitize\AllBytesString

%<*test4>
\let\org@detokenize\detokenize
\let\detokenize\@undefined
\let\org@numexpr\numexpr
\let\numexpr\@undefined
%</test4>
\RequirePackage{pdfescape}
%<*test4>
\let\detokenize\org@detokenize
\let\numexpr\org@numexpr
%</test4>

\begin{qstest}{all-hex}{\AllBytes, escapehex}
  \EdefEscapeHex\x{\AllBytes}%
  \Expect*{\x}*{\AllBytesHex}%
  \ExpectVar\x\AllBytesHex
\end{qstest}

\begin{qstest}{all-unhex}{\AllBytesHex, unescapehex}
  \EdefUnescapeHex\x{\AllBytesHex}%
  \Expect*{\x}*{\AllBytes}%
  \ExpectVar\x\AllBytes
\end{qstest}

\begin{qstest}{all-unhex-lc}{\AllBytesHexLC, unescapehex, lowercase}
  \EdefUnescapeHex\x{\AllBytesHexLC}%
  \Expect*{\x}*{\AllBytes}%
  \ExpectVar\x\AllBytes
\end{qstest}

\begin{qstest}{unhex-incomplete}{unescapehex, incomplete}
  \EdefUnescapeHex\x{4}%
  \Expect*{\x}{@}%
\end{qstest}

\begin{qstest}{unhex-space}{unescapehex, space}
  \EdefUnescapeHex\x{20}%
  \Expect*{\x}{ }%
  \ExpectVar\x\space
\end{qstest}

\begin{qstest}{unhex-spaces}{unescapehex, spaces}
  \EdefUnescapeHex\x{204020204120}%
  \def\y#1{%
    \edef\z{#1\string @#1#1\string A#1}%
  }\y{ }%
  \Expect*{\x}*{\z}%
  \ExpectVar\x\z
\end{qstest}

\begin{qstest}{unhex-hash}{unescapehex, hash}
  \catcode`\#=12 %
  \EdefUnescapeHex\x{#20}%
  \ExpectVar\x\space
\end{qstest}

\begin{qstest}{unhex-invalid}{unescapehex, invalid}
  \def\test#1#2{%
    \EdefUnescapeHex\x{#1}%
    \edef\y{#2}%
    \@onelevel@sanitize\y
    \ExpectVar\x\y
  }%
%<*test2>
  \edef\x{\pdfunescapehex{4X}}%
  \edef\y{\string @}%
  \ifx\x\y
  \else
    \def~{\space}%
    \typeout{************************************}%
    \typeout{* Your pdfTeX contains bug 777.~~~~*}%
    \typeout{* This test is redefined as dummy, *}%
    \typeout{* because it triggers the bug.~~~~~*}%
    \typeout{************************************}%
    \def\test#1#2{}%
  \fi
%</test2>
  \test{X}{}%
  \test{XY}{}%
  \test{XYZ}{}%
  \test{A}{^^a0}%
  \test{AX}{^^a0}%
  \test{XA}{^^a0}%
  \test{XXAXX}{^^a0}%
\end{qstest}

\begin{qstest}{all-name}{\AllBytes, escapename}
  \EdefEscapeName\x{\AllBytes}%
  \Expect*{\x}*{\AllBytesName}%
  \ExpectVar\x\AllBytesName
\end{qstest}

\begin{qstest}{all-string}{\AllBytes, escapestring}
  \EdefEscapeString\x{\AllBytes}%
  \Expect*{\x}*{\AllBytesString}%
  \ExpectVar\x\AllBytesString
\end{qstest}

\begin{qstest}{uchexdigit}{unescape, uppercase hex digit}
  \catcode`\@=11 %
  \catcode0=12 %
  \def\test#1#2{%
    \uccode0=#1\relax
    \uppercase{%
      \def\x{^^@}%
    }%
    \Expect*{%
       \ifcase\expandafter\PE@TestUcHexDigit\x
         true%
       \else
         false%
       \fi
    }{#2}%
  }%
  \def\range#1#2#3{%
    \count0=#1\relax
    \loop
    \ifnum\count0<#2\relax
      \test{\count0}{#3}%
      \advance\count0 by 1 %
    \repeat
  }%
  \range{0}{47}{false}%
  \range{48}{57}{true}%
  \range{58}{64}{false}%
  \range{65}{70}{true}%
  \range{71}{255}{false}%
\end{qstest}

\begin{qstest}{unescapename}{unescapename}
  \def\test#1#2{%
    \EdefUnescapeName\x{#1}%
    \edef\y{#2}%
    \@onelevel@sanitize\y
    \ExpectVar\x\y
  }%
  \catcode`\#=12 %
  \catcode0=12 %
  \test{}{}%
  \test{x}{x}%
  \test{xy}{xy}%
  \test{#}{#}%
  \test{##}{##}%
  \test{###}{###}%
  \test{####}{####}%
  \test{#x}{#x}%
  \test{#xy}{#xy}%
  \test{#1}{#1}%
  \test{#40}{@}%
  \test{#400}{@0}%
  \test{#4x0}{#4x0}%
  \test{#ab}{^^ab}%
  \test{#00}{^^@}%
  \test{x#40y#40z}{x@y@z}%
  \test{#40#40#40#40}{@@@@}%
  \test{a#x}{a#x}%
  \test{a#xy}{a#xy}%
  \test{a#1}{a#1}%
  \test{a#40}{a@}%
  \test{a#400}{a@0}%
  \test{#20}{ }%
  \test{a#20}{a }%
  \test{a#20b}{a b}%
  \test{a#20#20#20b}{a \space\space b}%
\end{qstest}

\begin{qstest}{unescapestring}{unescapestring}
  \def\test#1#2{%
    \EdefUnescapeString\x{#1}%
    \edef\y{#2}%
    \@onelevel@sanitize\y
    \ExpectVar\x\y
  }%
  \catcode0=12 %
  \def\DefChar#1#2{%
    \begingroup
      \uccode0=#2\relax
    \uppercase{\endgroup
      \def#1{^^@}%
    }%
  }%
  \DefChar\nul{0}%
  \DefChar\one{1}%
  \DefChar\bel{8}%
  \DefChar\tab{9}%
  \DefChar\lf{10}%
  \DefChar\ff{12}%
  \DefChar\cr{13}%
  \DefChar\\{92}%
  \test{}{}%
  \test{a}{a}%
  \test{\\}{}%
  \test{\\\\}{\\}%
  \test{\\\\y}{\\y}%
  \test{\\000}{\nul}%
  \test{\\b}{\bel}%
  \test{\\t}{\tab}%
  \test{\\n}{\lf}%
  \test{\\f}{\ff}%
  \test{\\r}{\cr}%
  \test{\\(}{(}%
  \test{\\)}{)}%
  \test{\\040}{ }%
  \test{\\100}{@}%
  \test{\\40}{ }%
  \test{\\1}{\one}%
  \test{\\01}{\one}%
  \test{\\001}{\one}%
  \test{\\18}{\one8}%
  \test{\\018}{\one8}%
  \test{\\0018}{\one8}%
  \test{x\\}{x}%
  \test{x\\\\}{x\\}%
  \test{x\\\\y}{x\\y}%
  \test{x\\000}{x\nul}%
  \test{x\\b}{x\bel}%
  \test{x\\t}{x\tab}%
  \test{x\\n}{x\lf}%
  \test{x\\f}{x\ff}%
  \test{x\\r}{x\cr}%
  \test{x\\(}{x(}%
  \test{x\\)}{x)}%
  \test{x\\040}{x }%
  \test{x\\100}{x@}%
  \test{x\\40}{x }%
  \test{x\\1}{x\one}%
  \test{x\\01}{x\one}%
  \test{x\\001}{x\one}%
  \test{x\\18}{x\one8}%
  \test{x\\018}{x\one8}%
  \test{x\\0018}{x\one8}%
  \test{\\b\\t\\n\\f\\r\\(\\)\\\\\\000\\040}{%
    \bel\tab\lf\ff\cr()\\\nul\space
  }%
  \test{\\\lf}{}%
  \test{x\\\lf}{x}%
  \test{\cr}{\lf}%
  \test{\cr\lf}{\lf}%
  \test{\lf}{\lf}%
  \test{\lf\cr}{\lf\lf}%
  \test{x\cr}{x\lf}%
  \test{x\cr\lf}{x\lf}%
  \test{x\lf}{x\lf}%
  \test{x\lf\cr}{x\lf\lf}%
  \test{x\\\cr\lf y\cr}{xy\lf}%
%
  \test{\\409}{ 9}%
  \test{\\800}{800}%
  \test{\\900}{900}%
  \test{\\578}{/8}%
  \test{\\477}{?}%
  \test{\\377}{^^ff}%
  \test{\\777}{^^ff}%
  \test{\\7777}{^^ff7}%
\end{qstest}
\stop
%    \end{macrocode}
%
%    \begin{macrocode}
%</test2|test3|test4|test5>
%    \end{macrocode}
%
% \subsubsection{Test for \hologo{iniTeX}}
%
%    \begin{macrocode}
%<*test6>
\input pdfescape.sty\relax
\catcode`\{=1 %
\catcode`\}=2 %
\catcode`\#=6 %
\catcode`\^=7 %
\catcode`\@=11 %
%    \end{macrocode}
%    \begin{macrocode}
\begingroup
  \catcode`\@=11 %
  \countdef\count@=255 %
  \def\space{ }%
  \long\def\@whilenum#1\do #2{%
    \ifnum #1\relax
      #2\relax
      \@iwhilenum{#1\relax#2\relax}%
    \fi
  }%
  \long\def\@iwhilenum#1{%
    \ifnum #1%
      \expandafter\@iwhilenum
    \else
      \expandafter\ltx@gobble
    \fi
    {#1}%
  }%
  \gdef\AllBytes{}%
  \count@=0 %
  \catcode0=12 %
  \@whilenum\count@<256 \do{%
    \lccode0=\count@
    \ifnum\count@=32 %
      \xdef\AllBytes{\AllBytes\space}%
    \else
      \lowercase{%
        \xdef\AllBytes{\AllBytes^^@}%
      }%
    \fi
    \advance\count@ by 1 %
  }%
\endgroup
\def\AllBytesHex{%
  000102030405060708090A0B0C0D0E0F%
  101112131415161718191A1B1C1D1E1F%
  202122232425262728292A2B2C2D2E2F%
  303132333435363738393A3B3C3D3E3F%
  404142434445464748494A4B4C4D4E4F%
  505152535455565758595A5B5C5D5E5F%
  606162636465666768696A6B6C6D6E6F%
  707172737475767778797A7B7C7D7E7F%
  808182838485868788898A8B8C8D8E8F%
  909192939495969798999A9B9C9D9E9F%
  A0A1A2A3A4A5A6A7A8A9AAABACADAEAF%
  B0B1B2B3B4B5B6B7B8B9BABBBCBDBEBF%
  C0C1C2C3C4C5C6C7C8C9CACBCCCDCECF%
  D0D1D2D3D4D5D6D7D8D9DADBDCDDDEDF%
  E0E1E2E3E4E5E6E7E8E9EAEBECEDEEEF%
  F0F1F2F3F4F5F6F7F8F9FAFBFCFDFEFF%
}
\ltx@onelevel@sanitize\AllBytesHex
\expandafter\lowercase\expandafter{%
  \expandafter\def\expandafter\AllBytesHexLC
      \expandafter{\AllBytesHex}%
}
\begingroup
  \catcode`\#=12 %
  \xdef\AllBytesName{%
    #01#02#03#04#05#06#07#08#09#0A#0B#0C#0D#0E#0F%
    #10#11#12#13#14#15#16#17#18#19#1A#1B#1C#1D#1E#1F%
    #20!"#23$#25&'#28#29*+,-.#2F%
    0123456789:;#3C=#3E?%
    @ABCDEFGHIJKLMNO%
    PQRSTUVWXYZ#5B\ltx@backslashchar#5D^_%
    `abcdefghijklmno%
    pqrstuvwxyz#7B|#7D\string~#7F%
    #80#81#82#83#84#85#86#87#88#89#8A#8B#8C#8D#8E#8F%
    #90#91#92#93#94#95#96#97#98#99#9A#9B#9C#9D#9E#9F%
    #A0#A1#A2#A3#A4#A5#A6#A7#A8#A9#AA#AB#AC#AD#AE#AF%
    #B0#B1#B2#B3#B4#B5#B6#B7#B8#B9#BA#BB#BC#BD#BE#BF%
    #C0#C1#C2#C3#C4#C5#C6#C7#C8#C9#CA#CB#CC#CD#CE#CF%
    #D0#D1#D2#D3#D4#D5#D6#D7#D8#D9#DA#DB#DC#DD#DE#DF%
    #E0#E1#E2#E3#E4#E5#E6#E7#E8#E9#EA#EB#EC#ED#EE#EF%
    #F0#F1#F2#F3#F4#F5#F6#F7#F8#F9#FA#FB#FC#FD#FE#FF%
  }%
\endgroup
\ltx@onelevel@sanitize\AllBytesName
\edef\AllBytesFromName{\expandafter\ltx@gobble\AllBytes}

\begingroup
  \def\|{|}%
  \edef\%{\ltx@percentchar}%
  \catcode`\|=0 %
  \catcode`\#=12 %
  \catcode`\~=12 %
  \catcode`\\=12 %
  |xdef|AllBytesString{%
    \000\001\002\003\004\005\006\007\010\011\012\013\014\015\016\017%
    \020\021\022\023\024\025\026\027\030\031\032\033\034\035\036\037%
    \040!"#$|%&'\(\)*+,-./%
    0123456789:;<=>?%
    @ABCDEFGHIJKLMNO%
    PQRSTUVWXYZ[\\]^_%
    `abcdefghijklmno%
    pqrstuvwxyz{||}~\177%
    \200\201\202\203\204\205\206\207\210\211\212\213\214\215\216\217%
    \220\221\222\223\224\225\226\227\230\231\232\233\234\235\236\237%
    \240\241\242\243\244\245\246\247\250\251\252\253\254\255\256\257%
    \260\261\262\263\264\265\266\267\270\271\272\273\274\275\276\277%
    \300\301\302\303\304\305\306\307\310\311\312\313\314\315\316\317%
    \320\321\322\323\324\325\326\327\330\331\332\333\334\335\336\337%
    \340\341\342\343\344\345\346\347\350\351\352\353\354\355\356\357%
    \360\361\362\363\364\365\366\367\370\371\372\373\374\375\376\377%
  }%
|endgroup
\ltx@onelevel@sanitize\AllBytesString
%    \end{macrocode}
%    \begin{macrocode}
\def\msg#{\immediate\write16}
\def\Test#1#2#3{%
  \begingroup
    #1\TestResult{#2}%
    \ifx\TestResult#3%
    \else
      \newlinechar=10 %
      \msg{Expect:^^J#3}%
      \msg{Result:^^J\TestResult}%
      \errmessage{\string#2 -\string#1-> \string#3}%
    \fi
  \endgroup
}
\Test\EdefEscapeHex\AllBytes\AllBytesHex
\Test\EdefUnescapeHex\AllBytesHex\AllBytes
\Test\EdefEscapeName\AllBytes\AllBytesName
\Test\EdefUnescapeName\AllBytesName\AllBytesFromName
\Test\EdefEscapeString\AllBytes\AllBytesString
\Test\EdefUnescapeString\AllBytesString\AllBytes
%    \end{macrocode}
%    \begin{macrocode}
\csname @@end\endcsname\end
%</test6>
%    \end{macrocode}
%
% \section{Installation}
%
% \subsection{Download}
%
% \paragraph{Package.} This package is available on
% CTAN\footnote{\url{ftp://ftp.ctan.org/tex-archive/}}:
% \begin{description}
% \item[\CTAN{macros/latex/contrib/oberdiek/pdfescape.dtx}] The source file.
% \item[\CTAN{macros/latex/contrib/oberdiek/pdfescape.pdf}] Documentation.
% \end{description}
%
%
% \paragraph{Bundle.} All the packages of the bundle `oberdiek'
% are also available in a TDS compliant ZIP archive. There
% the packages are already unpacked and the documentation files
% are generated. The files and directories obey the TDS standard.
% \begin{description}
% \item[\CTAN{install/macros/latex/contrib/oberdiek.tds.zip}]
% \end{description}
% \emph{TDS} refers to the standard ``A Directory Structure
% for \TeX\ Files'' (\CTAN{tds/tds.pdf}). Directories
% with \xfile{texmf} in their name are usually organized this way.
%
% \subsection{Bundle installation}
%
% \paragraph{Unpacking.} Unpack the \xfile{oberdiek.tds.zip} in the
% TDS tree (also known as \xfile{texmf} tree) of your choice.
% Example (linux):
% \begin{quote}
%   |unzip oberdiek.tds.zip -d ~/texmf|
% \end{quote}
%
% \paragraph{Script installation.}
% Check the directory \xfile{TDS:scripts/oberdiek/} for
% scripts that need further installation steps.
% Package \xpackage{attachfile2} comes with the Perl script
% \xfile{pdfatfi.pl} that should be installed in such a way
% that it can be called as \texttt{pdfatfi}.
% Example (linux):
% \begin{quote}
%   |chmod +x scripts/oberdiek/pdfatfi.pl|\\
%   |cp scripts/oberdiek/pdfatfi.pl /usr/local/bin/|
% \end{quote}
%
% \subsection{Package installation}
%
% \paragraph{Unpacking.} The \xfile{.dtx} file is a self-extracting
% \docstrip\ archive. The files are extracted by running the
% \xfile{.dtx} through \plainTeX:
% \begin{quote}
%   \verb|tex pdfescape.dtx|
% \end{quote}
%
% \paragraph{TDS.} Now the different files must be moved into
% the different directories in your installation TDS tree
% (also known as \xfile{texmf} tree):
% \begin{quote}
% \def\t{^^A
% \begin{tabular}{@{}>{\ttfamily}l@{ $\rightarrow$ }>{\ttfamily}l@{}}
%   pdfescape.sty & tex/generic/oberdiek/pdfescape.sty\\
%   pdfescape.pdf & doc/latex/oberdiek/pdfescape.pdf\\
%   test/pdfescape-test1.tex & doc/latex/oberdiek/test/pdfescape-test1.tex\\
%   test/pdfescape-test2.tex & doc/latex/oberdiek/test/pdfescape-test2.tex\\
%   test/pdfescape-test3.tex & doc/latex/oberdiek/test/pdfescape-test3.tex\\
%   test/pdfescape-test4.tex & doc/latex/oberdiek/test/pdfescape-test4.tex\\
%   test/pdfescape-test5.tex & doc/latex/oberdiek/test/pdfescape-test5.tex\\
%   test/pdfescape-test6.tex & doc/latex/oberdiek/test/pdfescape-test6.tex\\
%   pdfescape.dtx & source/latex/oberdiek/pdfescape.dtx\\
% \end{tabular}^^A
% }^^A
% \sbox0{\t}^^A
% \ifdim\wd0>\linewidth
%   \begingroup
%     \advance\linewidth by\leftmargin
%     \advance\linewidth by\rightmargin
%   \edef\x{\endgroup
%     \def\noexpand\lw{\the\linewidth}^^A
%   }\x
%   \def\lwbox{^^A
%     \leavevmode
%     \hbox to \linewidth{^^A
%       \kern-\leftmargin\relax
%       \hss
%       \usebox0
%       \hss
%       \kern-\rightmargin\relax
%     }^^A
%   }^^A
%   \ifdim\wd0>\lw
%     \sbox0{\small\t}^^A
%     \ifdim\wd0>\linewidth
%       \ifdim\wd0>\lw
%         \sbox0{\footnotesize\t}^^A
%         \ifdim\wd0>\linewidth
%           \ifdim\wd0>\lw
%             \sbox0{\scriptsize\t}^^A
%             \ifdim\wd0>\linewidth
%               \ifdim\wd0>\lw
%                 \sbox0{\tiny\t}^^A
%                 \ifdim\wd0>\linewidth
%                   \lwbox
%                 \else
%                   \usebox0
%                 \fi
%               \else
%                 \lwbox
%               \fi
%             \else
%               \usebox0
%             \fi
%           \else
%             \lwbox
%           \fi
%         \else
%           \usebox0
%         \fi
%       \else
%         \lwbox
%       \fi
%     \else
%       \usebox0
%     \fi
%   \else
%     \lwbox
%   \fi
% \else
%   \usebox0
% \fi
% \end{quote}
% If you have a \xfile{docstrip.cfg} that configures and enables \docstrip's
% TDS installing feature, then some files can already be in the right
% place, see the documentation of \docstrip.
%
% \subsection{Refresh file name databases}
%
% If your \TeX~distribution
% (\teTeX, \mikTeX, \dots) relies on file name databases, you must refresh
% these. For example, \teTeX\ users run \verb|texhash| or
% \verb|mktexlsr|.
%
% \subsection{Some details for the interested}
%
% \paragraph{Attached source.}
%
% The PDF documentation on CTAN also includes the
% \xfile{.dtx} source file. It can be extracted by
% AcrobatReader 6 or higher. Another option is \textsf{pdftk},
% e.g. unpack the file into the current directory:
% \begin{quote}
%   \verb|pdftk pdfescape.pdf unpack_files output .|
% \end{quote}
%
% \paragraph{Unpacking with \LaTeX.}
% The \xfile{.dtx} chooses its action depending on the format:
% \begin{description}
% \item[\plainTeX:] Run \docstrip\ and extract the files.
% \item[\LaTeX:] Generate the documentation.
% \end{description}
% If you insist on using \LaTeX\ for \docstrip\ (really,
% \docstrip\ does not need \LaTeX), then inform the autodetect routine
% about your intention:
% \begin{quote}
%   \verb|latex \let\install=y\input{pdfescape.dtx}|
% \end{quote}
% Do not forget to quote the argument according to the demands
% of your shell.
%
% \paragraph{Generating the documentation.}
% You can use both the \xfile{.dtx} or the \xfile{.drv} to generate
% the documentation. The process can be configured by the
% configuration file \xfile{ltxdoc.cfg}. For instance, put this
% line into this file, if you want to have A4 as paper format:
% \begin{quote}
%   \verb|\PassOptionsToClass{a4paper}{article}|
% \end{quote}
% An example follows how to generate the
% documentation with pdf\LaTeX:
% \begin{quote}
%\begin{verbatim}
%pdflatex pdfescape.dtx
%makeindex -s gind.ist pdfescape.idx
%pdflatex pdfescape.dtx
%makeindex -s gind.ist pdfescape.idx
%pdflatex pdfescape.dtx
%\end{verbatim}
% \end{quote}
%
% \section{Catalogue}
%
% The following XML file can be used as source for the
% \href{http://mirror.ctan.org/help/Catalogue/catalogue.html}{\TeX\ Catalogue}.
% The elements \texttt{caption} and \texttt{description} are imported
% from the original XML file from the Catalogue.
% The name of the XML file in the Catalogue is \xfile{pdfescape.xml}.
%    \begin{macrocode}
%<*catalogue>
<?xml version='1.0' encoding='us-ascii'?>
<!DOCTYPE entry SYSTEM 'catalogue.dtd'>
<entry datestamp='$Date$' modifier='$Author$' id='pdfescape'>
  <name>pdfescape</name>
  <caption>Implements pdfTeX's escape features using TeX or e-TeX.</caption>
  <authorref id='auth:oberdiek'/>
  <copyright owner='Heiko Oberdiek' year='2007,2010,2011'/>
  <license type='lppl1.3'/>
  <version number='1.13'/>
  <description>
    This package implements <xref refid='pdftex'>pdfTeX</xref>'s
    escape features (<tt>\pdfescapehex</tt>, <tt>\pdfunescapehex</tt>,
    <tt>\pdfescapename</tt>, <tt>\pdfescapestring</tt>) using TeX or
    e-TeX.
    <p/>
    The package is part of the <xref refid='oberdiek'>oberdiek</xref>
    bundle.
  </description>
  <documentation details='Package documentation'
      href='ctan:/macros/latex/contrib/oberdiek/pdfescape.pdf'/>
  <ctan file='true' path='/macros/latex/contrib/oberdiek/pdfescape.dtx'/>
  <miktex location='oberdiek'/>
  <texlive location='oberdiek'/>
  <install path='/macros/latex/contrib/oberdiek/oberdiek.tds.zip'/>
</entry>
%</catalogue>
%    \end{macrocode}
%
% \begin{History}
%   \begin{Version}{2007/02/21 v1.0}
%   \item
%     First version.
%   \end{Version}
%   \begin{Version}{2007/02/25 v1.1}
%   \item
%     Test files added.
%   \item
%     \cs{EdefUnescapeHex} supports lowercase letters.
%   \item
%     Fix: \verb|\EdefEscapeName{^^@}|
%   \item
%     Fix: \verb|\EdefEscapeName{\string#}|
%   \item
%     Fix for \cs{EdefUnescapeHex} in case of incomplete hex string.
%   \item
%     Fix: \cs{EdefUnescapeHex} generates space tokens with catcode
%     10 (space) in all cases.
%   \item
%     Fix: \cs{EdefEscapeHex} and \cs{EdefEscapeName} now generate
%     tokens with catcode 12 (other) only.
%   \end{Version}
%   \begin{Version}{2007/03/20 v1.2}
%   \item
%     Fix: Wrong year in \cs{ProvidesPackage}.
%   \end{Version}
%   \begin{Version}{2007/04/11 v1.3}
%   \item
%     Line ends sanitized.
%   \end{Version}
%   \begin{Version}{2007/04/21 v1.4}
%   \item
%     \cs{EdefUnescapeName} and \cs{EdefUnescapeString} added.
%   \end{Version}
%   \begin{Version}{2007/08/27 v1.5}
%   \item
%     \cs{EdefSanitize} added (replaces \cs{PE@sanitize}).
%   \end{Version}
%   \begin{Version}{2007/09/09 v1.6}
%   \item
%     Fix in catcode setup.
%   \end{Version}
%   \begin{Version}{2007/10/27 v1.7}
%   \item
%     More efficient \cs{EdefSanitize}.
%   \end{Version}
%   \begin{Version}{2007/11/11 v1.8}
%   \item
%     Use of package \xpackage{pdftexcmds} for \hologo{LuaTeX} support.
%   \end{Version}
%   \begin{Version}{2010/03/01 v1.9}
%   \item
%     Compatibility with \hologo{iniTeX}.
%   \end{Version}
%   \begin{Version}{2010/11/12 v1.10}
%   \item
%     Use of package \xpackage{ltxcmds}.
%   \item
%     Fix for compatibility with \hologo{iniTeX}.
%   \end{Version}
%   \begin{Version}{2011/01/30 v1.11}
%   \item
%     Already loaded package files are not input in \hologo{plainTeX}.
%   \end{Version}
%   \begin{Version}{2011/04/04 v1.12}
%   \item
%     Further fixes for compatibility for \hologo{iniTeX}.
%   \item
%     Test file for \hologo{iniTeX} added.
%   \end{Version}
%   \begin{Version}{2011/11/25 v1.13}
%   \item
%     Higher order bit of octal sequences in \cs{EdefUnescapeString}
%     ignored according to the PDF specification
%     (Bug found by Bruno Le Floch).
%   \end{Version}
% \end{History}
%
% \PrintIndex
%
% \Finale
\endinput

%        (quote the arguments according to the demands of your shell)
%
% Documentation:
%    (a) If pdfescape.drv is present:
%           latex pdfescape.drv
%    (b) Without pdfescape.drv:
%           latex pdfescape.dtx; ...
%    The class ltxdoc loads the configuration file ltxdoc.cfg
%    if available. Here you can specify further options, e.g.
%    use A4 as paper format:
%       \PassOptionsToClass{a4paper}{article}
%
%    Programm calls to get the documentation (example):
%       pdflatex pdfescape.dtx
%       makeindex -s gind.ist pdfescape.idx
%       pdflatex pdfescape.dtx
%       makeindex -s gind.ist pdfescape.idx
%       pdflatex pdfescape.dtx
%
% Installation:
%    TDS:tex/generic/oberdiek/pdfescape.sty
%    TDS:doc/latex/oberdiek/pdfescape.pdf
%    TDS:doc/latex/oberdiek/test/pdfescape-test1.tex
%    TDS:doc/latex/oberdiek/test/pdfescape-test2.tex
%    TDS:doc/latex/oberdiek/test/pdfescape-test3.tex
%    TDS:doc/latex/oberdiek/test/pdfescape-test4.tex
%    TDS:doc/latex/oberdiek/test/pdfescape-test5.tex
%    TDS:doc/latex/oberdiek/test/pdfescape-test6.tex
%    TDS:source/latex/oberdiek/pdfescape.dtx
%
%<*ignore>
\begingroup
  \catcode123=1 %
  \catcode125=2 %
  \def\x{LaTeX2e}%
\expandafter\endgroup
\ifcase 0\ifx\install y1\fi\expandafter
         \ifx\csname processbatchFile\endcsname\relax\else1\fi
         \ifx\fmtname\x\else 1\fi\relax
\else\csname fi\endcsname
%</ignore>
%<*install>
\input docstrip.tex
\Msg{************************************************************************}
\Msg{* Installation}
\Msg{* Package: pdfescape 2011/11/25 v1.13 Implements pdfTeX's escape features (HO)}
\Msg{************************************************************************}

\keepsilent
\askforoverwritefalse

\let\MetaPrefix\relax
\preamble

This is a generated file.

Project: pdfescape
Version: 2011/11/25 v1.13

Copyright (C) 2007, 2010, 2011 by
   Heiko Oberdiek <heiko.oberdiek at googlemail.com>

This work may be distributed and/or modified under the
conditions of the LaTeX Project Public License, either
version 1.3c of this license or (at your option) any later
version. This version of this license is in
   http://www.latex-project.org/lppl/lppl-1-3c.txt
and the latest version of this license is in
   http://www.latex-project.org/lppl.txt
and version 1.3 or later is part of all distributions of
LaTeX version 2005/12/01 or later.

This work has the LPPL maintenance status "maintained".

This Current Maintainer of this work is Heiko Oberdiek.

The Base Interpreter refers to any `TeX-Format',
because some files are installed in TDS:tex/generic//.

This work consists of the main source file pdfescape.dtx
and the derived files
   pdfescape.sty, pdfescape.pdf, pdfescape.ins, pdfescape.drv,
   pdfescape-test1.tex, pdfescape-test2.tex,
   pdfescape-test3.tex, pdfescape-test4.tex,
   pdfescape-test5.tex, pdfescape-test6.tex.

\endpreamble
\let\MetaPrefix\DoubleperCent

\generate{%
  \file{pdfescape.ins}{\from{pdfescape.dtx}{install}}%
  \file{pdfescape.drv}{\from{pdfescape.dtx}{driver}}%
  \usedir{tex/generic/oberdiek}%
  \file{pdfescape.sty}{\from{pdfescape.dtx}{package}}%
  \usedir{doc/latex/oberdiek/test}%
  \file{pdfescape-test1.tex}{\from{pdfescape.dtx}{test1}}%
  \file{pdfescape-test2.tex}{\from{pdfescape.dtx}{test2}}%
  \file{pdfescape-test3.tex}{\from{pdfescape.dtx}{test3}}%
  \file{pdfescape-test4.tex}{\from{pdfescape.dtx}{test4}}%
  \file{pdfescape-test5.tex}{\from{pdfescape.dtx}{test5}}%
  \file{pdfescape-test6.tex}{\from{pdfescape.dtx}{test6}}%
  \nopreamble
  \nopostamble
  \usedir{source/latex/oberdiek/catalogue}%
  \file{pdfescape.xml}{\from{pdfescape.dtx}{catalogue}}%
}

\catcode32=13\relax% active space
\let =\space%
\Msg{************************************************************************}
\Msg{*}
\Msg{* To finish the installation you have to move the following}
\Msg{* file into a directory searched by TeX:}
\Msg{*}
\Msg{*     pdfescape.sty}
\Msg{*}
\Msg{* To produce the documentation run the file `pdfescape.drv'}
\Msg{* through LaTeX.}
\Msg{*}
\Msg{* Happy TeXing!}
\Msg{*}
\Msg{************************************************************************}

\endbatchfile
%</install>
%<*ignore>
\fi
%</ignore>
%<*driver>
\NeedsTeXFormat{LaTeX2e}
\ProvidesFile{pdfescape.drv}%
  [2011/11/25 v1.13 Implements pdfTeX's escape features (HO)]%
\documentclass{ltxdoc}
\usepackage{holtxdoc}[2011/11/22]
\begin{document}
  \DocInput{pdfescape.dtx}%
\end{document}
%</driver>
% \fi
%
% \CheckSum{2516}
%
% \CharacterTable
%  {Upper-case    \A\B\C\D\E\F\G\H\I\J\K\L\M\N\O\P\Q\R\S\T\U\V\W\X\Y\Z
%   Lower-case    \a\b\c\d\e\f\g\h\i\j\k\l\m\n\o\p\q\r\s\t\u\v\w\x\y\z
%   Digits        \0\1\2\3\4\5\6\7\8\9
%   Exclamation   \!     Double quote  \"     Hash (number) \#
%   Dollar        \$     Percent       \%     Ampersand     \&
%   Acute accent  \'     Left paren    \(     Right paren   \)
%   Asterisk      \*     Plus          \+     Comma         \,
%   Minus         \-     Point         \.     Solidus       \/
%   Colon         \:     Semicolon     \;     Less than     \<
%   Equals        \=     Greater than  \>     Question mark \?
%   Commercial at \@     Left bracket  \[     Backslash     \\
%   Right bracket \]     Circumflex    \^     Underscore    \_
%   Grave accent  \`     Left brace    \{     Vertical bar  \|
%   Right brace   \}     Tilde         \~}
%
% \GetFileInfo{pdfescape.drv}
%
% \title{The \xpackage{pdfescape} package}
% \date{2011/11/25 v1.13}
% \author{Heiko Oberdiek\\\xemail{heiko.oberdiek at googlemail.com}}
%
% \maketitle
%
% \begin{abstract}
% This package implements \pdfTeX's escape features
% (\cs{pdfescapehex}, \cs{pdfunescapehex}, \cs{pdfescapename},
% \cs{pdfescapestring}) using \TeX\ or \eTeX.
% \end{abstract}
%
% \tableofcontents
%
% \section{Documentation}
%
% \begin{declcs}{EdefEscapeHex} \M{cmd} \M{string}\\
%  \cs{EdefUnescapeHex} \M{cmd} \M{string}\\
%  \cs{EdefEscapeName} \M{cmd} \M{string}\\
%  \cs{EdefEscapeString} \M{cmd} \M{string}
% \end{declcs}
% These commands converts \meta{string} and stores the result
% in macro \meta{cmd}. The conversion result is the same
% as the conversion of the corresponding \pdfTeX's primitives.
% Note that the argument \meta{string} is expanded before
% the conversion.
%
% For example, if \pdfTeX\ \textgreater= 1.30 is present,
% then \cs{EdefEscapeHex}
% becomes to:
% \begin{quote}
%\begin{verbatim}
%\def\EdefEscapeHex#1#2{%
%  \edef#1{\pdfescapehex{#2}}%
%}
%\end{verbatim}
% \end{quote}
% The package provides implementations for the case that \pdfTeX\ is
% not present (or too old). Even \eTeX\ can be missing, however it is
% used if it is detected.
%
% \paragraph{Babel.}
% The input strings may contain shorthand characters of package
% \xpackage{babel}.
%
% \subsection{Additional unescape macros}
%
% \begin{declcs}{EdefUnescapeName} \M{cmd} \M{string}
% \end{declcs}
% Sequences of a hash sign with two hexadecimal digits are
% converted to the corresponding character (PDF-1.2).
% A hash sign that is not followed by two hexadecimal digits
% is left unchanged. The catcodes in the result string follow
% TeX's conventions. The space has catcode 10 (space) and
% the other characters have catcode 12 (other).
%
% \begin{declcs}{EdefUnescapeString} \M{cmd} \M{string}
% \end{declcs}
% Macro \meta{cmd} stores the unescaped string in \meta{string}.
% All the rules for literal strings are implemented, see
% PDF specification.
% The catcodes in the result string follow TeX's conventions.
%
% \subsection{Sanitizing macro}
%
% \begin{declcs}{EdefSanitize} \M{cmd} \M{string}
% \end{declcs}
% Argument \meta{string} is expanded, converted to a string
% of tokens with catcode 12 (other) and space tokens, and
% stored in macro \meta{cmd}.
%
% \StopEventually{
% }
%
% \section{Implementation}
%
%    \begin{macrocode}
%<*package>
%    \end{macrocode}
%
% \subsection{Reload check and package identification}
%    Reload check, especially if the package is not used with \LaTeX.
%    \begin{macrocode}
\begingroup\catcode61\catcode48\catcode32=10\relax%
  \catcode13=5 % ^^M
  \endlinechar=13 %
  \catcode35=6 % #
  \catcode39=12 % '
  \catcode44=12 % ,
  \catcode45=12 % -
  \catcode46=12 % .
  \catcode58=12 % :
  \catcode64=11 % @
  \catcode123=1 % {
  \catcode125=2 % }
  \expandafter\let\expandafter\x\csname ver@pdfescape.sty\endcsname
  \ifx\x\relax % plain-TeX, first loading
  \else
    \def\empty{}%
    \ifx\x\empty % LaTeX, first loading,
      % variable is initialized, but \ProvidesPackage not yet seen
    \else
      \expandafter\ifx\csname PackageInfo\endcsname\relax
        \def\x#1#2{%
          \immediate\write-1{Package #1 Info: #2.}%
        }%
      \else
        \def\x#1#2{\PackageInfo{#1}{#2, stopped}}%
      \fi
      \x{pdfescape}{The package is already loaded}%
      \aftergroup\endinput
    \fi
  \fi
\endgroup%
%    \end{macrocode}
%    Package identification:
%    \begin{macrocode}
\begingroup\catcode61\catcode48\catcode32=10\relax%
  \catcode13=5 % ^^M
  \endlinechar=13 %
  \catcode35=6 % #
  \catcode39=12 % '
  \catcode40=12 % (
  \catcode41=12 % )
  \catcode44=12 % ,
  \catcode45=12 % -
  \catcode46=12 % .
  \catcode47=12 % /
  \catcode58=12 % :
  \catcode64=11 % @
  \catcode91=12 % [
  \catcode93=12 % ]
  \catcode123=1 % {
  \catcode125=2 % }
  \expandafter\ifx\csname ProvidesPackage\endcsname\relax
    \def\x#1#2#3[#4]{\endgroup
      \immediate\write-1{Package: #3 #4}%
      \xdef#1{#4}%
    }%
  \else
    \def\x#1#2[#3]{\endgroup
      #2[{#3}]%
      \ifx#1\@undefined
        \xdef#1{#3}%
      \fi
      \ifx#1\relax
        \xdef#1{#3}%
      \fi
    }%
  \fi
\expandafter\x\csname ver@pdfescape.sty\endcsname
\ProvidesPackage{pdfescape}%
  [2011/11/25 v1.13 Implements pdfTeX's escape features (HO)]%
%    \end{macrocode}
%
% \subsection{Catcodes}
%
%    \begin{macrocode}
\begingroup\catcode61\catcode48\catcode32=10\relax%
  \catcode13=5 % ^^M
  \endlinechar=13 %
  \catcode123 1 % {
  \catcode125 2 % }
  \catcode64 11 %
  \def\x{\endgroup
    \expandafter\edef\csname PE@AtEnd\endcsname{%
      \endlinechar=\the\endlinechar\relax
      \catcode13=\the\catcode13\relax
      \catcode32=\the\catcode32\relax
      \catcode35=\the\catcode35\relax
      \catcode61=\the\catcode61\relax
      \catcode64=\the\catcode64\relax
      \catcode123=\the\catcode123\relax
      \catcode125=\the\catcode125\relax
    }%
  }%
\x\catcode61\catcode48\catcode32=10\relax%
\catcode13=5 % ^^M
\endlinechar=13 %
\catcode35=6 % #
\catcode64=11 % @
\catcode123=1 % {
\catcode125=2 % }
\def\TMP@EnsureCode#1#2#3{%
  \edef\PE@AtEnd{%
    \PE@AtEnd
    #1#2=\the#1#2\relax
  }%
  #1#2=#3\relax
}
\TMP@EnsureCode\catcode{0}{12}% ^^@
\TMP@EnsureCode\catcode{34}{12}% "
\TMP@EnsureCode\catcode{36}{3}% $
\TMP@EnsureCode\catcode{38}{4}% &
\TMP@EnsureCode\catcode{39}{12}% '
\TMP@EnsureCode\catcode{42}{12}% *
\TMP@EnsureCode\catcode{45}{12}% -
\TMP@EnsureCode\catcode{46}{12}% .
\TMP@EnsureCode\catcode{47}{12}% /
\TMP@EnsureCode\catcode{60}{12}% <
\TMP@EnsureCode\catcode{62}{12}% >
\TMP@EnsureCode\catcode{91}{12}% [
\TMP@EnsureCode\catcode{93}{12}% ]
\TMP@EnsureCode\catcode{94}{7}% ^
\TMP@EnsureCode\catcode{96}{12}% `
\TMP@EnsureCode\uccode{34}{0}% "
\TMP@EnsureCode\uccode{48}{0}% 0
\TMP@EnsureCode\uccode{61}{0}% =
\edef\PE@AtEnd{\PE@AtEnd\noexpand\endinput}
%    \end{macrocode}
%
% \subsection{Load package}
%
%    \begin{macrocode}
\begingroup\expandafter\expandafter\expandafter\endgroup
\expandafter\ifx\csname RequirePackage\endcsname\relax
  \def\TMP@RequirePackage#1[#2]{%
    \begingroup\expandafter\expandafter\expandafter\endgroup
    \expandafter\ifx\csname ver@#1.sty\endcsname\relax
      \input #1.sty\relax
    \fi
  }%
  \TMP@RequirePackage{ltxcmds}[2010/04/08]%
\else
  \RequirePackage{ltxcmds}[2010/04/08]%
\fi
%    \end{macrocode}
%
% \subsection{Sanitizing}
%
%    \begin{macro}{\EdefSanitize}
%    Macro \cs{EdefSanitize} takes |#2|, entirely converts
%    it to token with catcode 12 (other) and stores the
%    result in macro |#1|.
%    \begin{macrocode}
\begingroup\expandafter\expandafter\expandafter\endgroup
\expandafter\ifx\csname detokenize\endcsname\relax
  \long\def\EdefSanitize#1#2{%
    \begingroup
      \csname @safe@activestrue\endcsname
      \edef#1{#2}%
      \PE@onelevel@sanitize#1%
    \expandafter\endgroup
    \expandafter\def\expandafter#1\expandafter{#1}%
  }%
  \begingroup\expandafter\expandafter\expandafter\endgroup
  \expandafter\ifx\csname @onelevel@sanitize\endcsname\relax
    \def\PE@onelevel@sanitize#1{%
      \edef#1{\expandafter\PE@strip@prefix\meaning#1}%
    }%
    \def\PE@strip@prefix#1>{}%
  \else
    \let\PE@onelevel@sanitize\@onelevel@sanitize
  \fi
\else
  \long\def\EdefSanitize#1#2{%
    \begingroup
      \csname @safe@activestrue\endcsname
      \edef#1{#2}%
    \expandafter\endgroup
    \expandafter\def\expandafter#1\expandafter{%
      \detokenize\expandafter{#1}%
    }%
  }%
  \def\PE@onelevel@sanitize#1{%
    \edef#1{\detokenize\expandafter{#1}}%
  }%
\fi
%    \end{macrocode}
%    \end{macro}
%    \begin{macro}{\PE@sanitize}
%    Macro \cs{PE@sanitize} is only defined for
%    compatibility with version 1.4. Its use is deprecated.
%    \begin{macrocode}
\let\PE@sanitize\EdefSanitize
%    \end{macrocode}
%    \end{macro}
%
% \subsubsection{Space characters}
%
%    \begin{macro}{\PE@space@other}
%    \begin{macrocode}
\begingroup
  \catcode`\ =12\relax%
\def\x{\endgroup\def\PE@space@other{ }}\x\relax
%    \end{macrocode}
%    \end{macro}
%    \begin{macro}{\PE@space@space}
%    \begin{macrocode}
\def\PE@space@space{ }
%    \end{macrocode}
%    \end{macro}
%
% \subsubsection{Space normalization}
%
%    \begin{macro}{\PE@SanitizeSpaceOther}
%    \begin{macrocode}
\def\PE@SanitizeSpaceOther#1{%
  \edef#1{\expandafter\PE@SpaceToOther#1 \relax}%
}
%    \end{macrocode}
%    \end{macro}
%
%    \begin{macro}{\PE@SpaceToOther}
%    \begin{macrocode}
\def\PE@SpaceToOther#1 #2\relax{%
  #1%
  \ifx\\#2\\%
  \else
    \PE@space@other
    \ltx@ReturnAfterFi{%
      \PE@SpaceToOther#2\relax
    }%
  \fi
}
%    \end{macrocode}
%    \end{macro}
%
% \subsection{\cs{EdefUnescapeName}}
%
%    \begin{macro}{\EdefUnescapeName}
%    \begin{macrocode}
\def\EdefUnescapeName#1#2{%
  \EdefSanitize#1{#2}%
  \PE@SanitizeSpaceOther#1%
  \PE@UnescapeName#1%
  \PE@onelevel@sanitize#1%
}
%    \end{macrocode}
%    \end{macro}
%
%    \begin{macro}{\PE@UnescapeName}
%    \begin{macrocode}
\begingroup
  \catcode`\$=6  % hash
  \catcode`\#=12 % other
  \gdef\PE@UnescapeName$1{%
    \begingroup
      \PE@InitUccodeHexDigit
      \def\PE@result{}%
      \expandafter\PE@DeName$1#\relax\relax
    \expandafter\endgroup
    \expandafter\def\expandafter$1\expandafter{\PE@result}%
  }%
  \gdef\PE@DeName$1#$2$3{%
    \ifx\relax$2%
      \edef\PE@result{\PE@result$1}%
      \let\PE@next\relax
    \else
      \ifx\relax$3%
        % wrong escape sequence in input
        \edef\PE@result{\PE@result$1#}%
        \let\PE@next\relax
      \else
        \uppercase{%
          \def\PE@testA{$2}%
          \def\PE@testB{$3}%
        }%
        \ifcase\ifcase\expandafter\PE@TestUcHexDigit\PE@testA
                 \ifcase\expandafter\PE@TestUcHexDigit\PE@testB
                   \ltx@zero
                 \else
                   \ltx@one
                 \fi
               \else
                 \ltx@one
               \fi
          \uccode\ltx@zero="\PE@testA\PE@testB\relax
          \uppercase{%
            \def\PE@temp{^^@}%
          }%
          \uccode\ltx@zero=\ltx@zero
          \edef\PE@result{\PE@result$1\PE@temp}%
          \let\PE@next\PE@DeName
        \else
          % wrong escape sequence in input
          \edef\PE@result{\PE@result$1#}%
          \def\PE@next{\PE@DeName$2$3}%
        \fi
      \fi
    \fi
    \PE@next
  }%
\endgroup
%    \end{macrocode}
%    \end{macro}
%
%    \begin{macro}{\PE@InitUccodeHexDigit}
%    \begin{macrocode}
\def\PE@InitUccodeHexDigit{%
  \uccode`a=`A\relax
  \uccode`b=`B\relax
  \uccode`c=`C\relax
  \uccode`d=`D\relax
  \uccode`e=`E\relax
  \uccode`f=`F\relax
  \uccode`A=\ltx@zero
  \uccode`B=\ltx@zero
  \uccode`C=\ltx@zero
  \uccode`D=\ltx@zero
  \uccode`E=\ltx@zero
  \uccode`F=\ltx@zero
  \uccode`0=\ltx@zero
  \uccode`1=\ltx@zero
  \uccode`2=\ltx@zero
  \uccode`3=\ltx@zero
  \uccode`4=\ltx@zero
  \uccode`5=\ltx@zero
  \uccode`6=\ltx@zero
  \uccode`7=\ltx@zero
  \uccode`8=\ltx@zero
  \uccode`9=\ltx@zero
}
%    \end{macrocode}
%    \end{macro}
%
%    \begin{macro}{\PE@TestUcHexDigit}
%    \begin{macrocode}
\def\PE@TestUcHexDigit#1{%
  \ifnum`#1<48 % 0
    \ltx@one
  \else
    \ifnum`#1>70 % F
      \ltx@one
    \else
      \ifnum`#1>57 % 9
        \ifnum`#1<65 % A
          \ltx@one
        \else
          \ltx@zero
        \fi
      \else
        \ltx@zero
      \fi
    \fi
  \fi
}
%    \end{macrocode}
%    \end{macro}
%
% \subsection{\cs{EdefUnescapeString}}
%
%    \begin{macro}{\EdefUnescapeString}
%    \begin{macrocode}
\def\EdefUnescapeString#1#2{%
  \EdefSanitize#1{#2}%
  \PE@SanitizeSpaceOther#1%
  \PE@NormalizeLineEnd#1%
  \PE@UnescapeString#1%
  \PE@onelevel@sanitize#1%
}
%    \end{macrocode}
%    \end{macro}
%
%    \begin{macrocode}
\begingroup
  \uccode`\8=10 % lf
  \uccode`\9=13 % cr
\def\x#1#2{\endgroup
%    \end{macrocode}
%    \begin{macro}{\PE@NormalizeLineEnd}
%    \begin{macrocode}
  \def\PE@NormalizeLineEnd##1{%
    \def\PE@result{}%
    \expandafter\PE@@NormalizeLineEnd##1#2\relax
    \let##1\PE@result
  }%
%    \end{macrocode}
%    \end{macro}
%    \begin{macro}{\PE@@NormalizeLineEnd}
%    \begin{macrocode}
  \def\PE@@NormalizeLineEnd##1#2##2{%
    \ifx\relax##2%
      \edef\PE@result{\PE@result##1}%
      \let\PE@next\relax
    \else
      \edef\PE@result{\PE@result##1#1}%
      \ifx#1##2% lf
        \let\PE@next\PE@@NormalizeLineEnd
      \else
        \def\PE@next{\PE@@NormalizeLineEnd##2}%
      \fi
    \fi
    \PE@next
  }%
}%
\uppercase{%
  \x 89%
}
%    \end{macrocode}
%    \end{macro}
%
%    \begin{macrocode}
\begingroup
  \catcode`\|=0 %
  \catcode`\\=12 %
%    \end{macrocode}
%    \begin{macro}{\PE@UnescapeString}
%    \begin{macrocode}
  |gdef|PE@UnescapeString#1{%
    |begingroup
      |def|PE@result{}%
      |expandafter|PE@DeString#1\|relax
    |expandafter|endgroup
    |expandafter|def|expandafter#1|expandafter{|PE@result}%
  }%
%    \end{macrocode}
%    \end{macro}
%    \begin{macro}{\PE@DeString}
%    \begin{macrocode}
  |gdef|PE@DeString#1\#2{%
    |ifx|relax#2%
      |edef|PE@result{|PE@result#1}%
      |let|PE@next|relax
    |else
      |if n#2%
        |uccode|ltx@zero=10 %
      |else|if r#2%
        |uccode|ltx@zero=13 %
      |else|if t#2%
        |uccode|ltx@zero=9 %
      |else|if b#2%
        |uccode|ltx@zero=8 %
      |else|if f#2%
        |uccode|ltx@zero=12 %
      |else
        |uccode|ltx@zero=|ltx@zero
      |fi|fi|fi|fi|fi
      |ifnum|uccode|ltx@zero>|ltx@zero
        |uppercase{%
          |edef|PE@temp{^^@}%
        }%
        |edef|PE@result{|PE@result#1|PE@temp}%
        |let|PE@next|PE@DeString
      |else
        |if\#2% backslash
          |edef|PE@result{|PE@result#1}%
          |let|PE@next|PE@CheckEndBackslash
        |else
          |ifnum`#2=10 % linefeed
            |edef|PE@result{|PE@result#1}%
            |let|PE@next|PE@DeString
          |else
            |ifcase|PE@TestOctDigit#2%
              |edef|PE@result{|PE@result#1}%
              |def|PE@next{|PE@OctI#2}%
            |else
              |edef|PE@result{|PE@result#1#2}%
              |let|PE@next|PE@DeString
            |fi
          |fi
        |fi
      |fi
    |fi
    |PE@next
  }%
%    \end{macrocode}
%    \end{macro}
%    \begin{macro}{\PE@CheckEndBackslash}
%    \begin{macrocode}
  |gdef|PE@CheckEndBackslash#1{%
    |ifx|relax#1%
    |else
      |edef|PE@result{|PE@result\}%
      |expandafter|PE@DeString|expandafter#1%
    |fi
  }%
%    \end{macrocode}
%    \end{macro}
%    \begin{macrocode}
|endgroup
%    \end{macrocode}
%    \begin{macro}{\PE@TestOctDigit}
%    \begin{macrocode}
\def\PE@TestOctDigit#1{%
  \ifnum`#1<48 % 0
    \ltx@one
  \else
    \ifnum`#1>55 % 7
      \ltx@one
    \else
      \ltx@zero
    \fi
  \fi
}
%    \end{macrocode}
%    \end{macro}
%    \begin{macro}{\PE@OctI}
%    \begin{macrocode}
\def\PE@OctI#1#2{%
  \ifcase\PE@TestOctDigit#2%
    \def\PE@next{\PE@OctII{#1#2}}%
  \else
    \def\PE@next{\PE@OctAll#1#2}%
  \fi
  \PE@next
}
%    \end{macrocode}
%    \end{macro}
%    \begin{macro}{\PE@OctII}
%    \begin{macrocode}
\def\PE@OctII#1#2{%
  \ifcase\PE@TestOctDigit#2%
    \def\PE@next{\PE@OctIII#1#2}%
  \else
    \def\PE@next{\PE@OctAll{#1}#2}%
  \fi
  \PE@next
}
%    \end{macrocode}
%    \end{macro}
%    \begin{macrocode}
\ltx@IfUndefined{numexpr}{%
  \catcode`\$=9 %
  \catcode`\&=14 %
}{%
  \catcode`\$=14 %
  \catcode`\&=9 %
}
%    \end{macrocode}
%    \begin{macro}{\PE@OctIII}
%    \begin{macrocode}
\def\PE@OctIII#1#2#3{%
  \ifnum#1<4 %
    \def\PE@next{\PE@OctAll{#1#2#3}}%
  \else
$   \count\ltx@cclv#1 %
$   \advance\count\ltx@cclv -4 %
    \edef\PE@next{%
      \noexpand\PE@OctAll{%
$       \the\count\ltx@cclv
&       \the\numexpr#1-4\relax
        #2#3%
      }%
    }%
  \fi
  \PE@next
}
%    \end{macrocode}
%    \end{macro}
%    \begin{macro}{\PE@OctAll}
%    \begin{macrocode}
\def\PE@OctAll#1{%
  \uccode\ltx@zero='#1\relax
  \uppercase{%
    \edef\PE@result{\PE@result^^@}%
  }%
  \PE@DeString
}
%    \end{macrocode}
%    \end{macro}
%
% \subsection{User macros (\pdfTeX\space analogues)}
%
%    \begin{macrocode}
\begingroup\expandafter\expandafter\expandafter\endgroup
\expandafter\ifx\csname RequirePackage\endcsname\relax
  \def\TMP@RequirePackage#1[#2]{%
    \begingroup\expandafter\expandafter\expandafter\endgroup
    \expandafter\ifx\csname ver@#1.sty\endcsname\relax
      \input #1.sty\relax
    \fi
  }%
  \TMP@RequirePackage{pdftexcmds}[2007/11/11]%
\else
  \RequirePackage{pdftexcmds}[2007/11/11]%
\fi
%    \end{macrocode}
%
%    \begin{macrocode}
\begingroup\expandafter\expandafter\expandafter\endgroup
\expandafter\ifx\csname pdf@escapehex\endcsname\relax
%    \end{macrocode}
%
%    \begin{macro}{\EdefEscapeHex}
%    \begin{macrocode}
  \long\def\EdefEscapeHex#1#2{%
    \EdefSanitize#1{#2}%
    \PE@SanitizeSpaceOther#1%
    \PE@EscapeHex#1%
  }%
%    \end{macrocode}
%    \end{macro}
%    \begin{macro}{\EdefUnescapeHex}
%    \begin{macrocode}
  \def\EdefUnescapeHex#1#2{%
    \EdefSanitize#1{#2}%
    \PE@UnescapeHex#1%
  }%
%    \end{macrocode}
%    \end{macro}
%    \begin{macro}{\EdefEscapeName}
%    \begin{macrocode}
  \long\def\EdefEscapeName#1#2{%
    \EdefSanitize#1{#2}%
    \PE@SanitizeSpaceOther#1%
    \PE@EscapeName#1%
  }%
%    \end{macrocode}
%    \end{macro}
%    \begin{macro}{\EdefEscapeString}
%    \begin{macrocode}
  \long\def\EdefEscapeString#1#2{%
    \EdefSanitize#1{#2}%
    \PE@SanitizeSpaceOther#1%
    \PE@EscapeString#1%
  }%
%    \end{macrocode}
%    \end{macro}
%
%    \begin{macrocode}
\else
%    \end{macrocode}
%
%    \begin{macro}{\PE@edefbabel}
%    Help macro that adds support for \xpackage{babel}'s shorthand
%    characters.
%    \begin{macrocode}
  \long\def\PE@edefbabel#1#2#3{%
    \begingroup
      \csname @save@activestrue\endcsname
      \edef#1{#2{#3}}%
    \expandafter\endgroup
    \expandafter\def\expandafter#1\expandafter{#1}%
  }%
%    \end{macrocode}
%    \end{macro}
%
%    \begin{macro}{\EdefEscapeHex}
%    \begin{macrocode}
  \long\def\EdefEscapeHex#1#2{%
    \PE@edefbabel#1\pdf@escapehex{#2}%
  }%
%    \end{macrocode}
%    \end{macro}
%    \begin{macro}{\EdefUnescapeHex}
%    \begin{macrocode}
  \def\EdefUnescapeHex#1#2{%
    \PE@edefbabel#1\pdf@unescapehex{#2}%
  }%
%    \end{macrocode}
%    \end{macro}
%    \begin{macro}{\EdefEscapeName}
%    \begin{macrocode}
  \long\def\EdefEscapeName#1#2{%
    \PE@edefbabel#1\pdf@escapename{#2}%
  }%
%    \end{macrocode}
%    \end{macro}
%    \begin{macro}{\EdefEscapeString}
%    \begin{macrocode}
  \long\def\EdefEscapeString#1#2{%
    \PE@edefbabel#1\pdf@escapestring{#2}%
  }%
%    \end{macrocode}
%    \end{macro}
%
%    \begin{macrocode}
  \expandafter\PE@AtEnd
\fi%
%    \end{macrocode}
%
% \subsection{Help macros}
%
% \subsubsection{Characters}
%
%    Special characters with catcode 12 (other) are created and
%    stored in macros.
%    \begin{macro}{\PE@hash}
%    \begin{macrocode}
\edef\PE@hash{\string#}
%    \end{macrocode}
%    \end{macro}
%    \begin{macro}{\PE@backslash}
%    \begin{macrocode}
\begingroup
  \escapechar=-1 %
\edef\x{\endgroup
  \def\noexpand\PE@backslash{\string\\}%
}
\x
%    \end{macrocode}
%    \end{macro}
%
% \subsubsection{Switch for \eTeX}
%
%    \begin{macrocode}
\ltx@newif\ifPE@etex
\begingroup\expandafter\expandafter\expandafter\endgroup
\expandafter\ifx\csname numexpr\endcsname\relax
\else
  \PE@etextrue
\fi
%    \end{macrocode}
%
% \subsection{Conversions}
%
% \subsubsection{Conversion to hex string}
%
%    \begin{macro}{\PE@EscapeHex}
%    \begin{macrocode}
\ifPE@etex
  \def\PE@EscapeHex#1{%
    \edef#1{\expandafter\PE@ToHex#1\relax}%
  }%
\else
  \def\PE@EscapeHex#1{%
    \def\PE@result{}%
    \expandafter\PE@ToHex#1\relax
    \let#1\PE@result
  }%
\fi
%    \end{macrocode}
%    \end{macro}
%    \begin{macro}{\PE@ToHex}
%    \begin{macrocode}
\def\PE@ToHex#1{%
  \ifx\relax#1%
  \else
    \PE@HexChar{#1}%
    \expandafter\PE@ToHex
  \fi
}%
%    \end{macrocode}
%    \end{macro}
%    \begin{macro}{\PE@HexChar}
%    \begin{macrocode}
\ifPE@etex
  \def\PE@HexChar#1{%
    \PE@HexDigit{\numexpr\dimexpr.0625\dimexpr`#1sp\relax\relax\relax}%
    \PE@HexDigit{%
      \numexpr`#1-16*\dimexpr.0625\dimexpr`#1sp\relax\relax\relax
    }%
  }%
\else
  \def\PE@HexChar#1{%
    \dimen0=`#1sp%
    \dimen2=.0625\dimen0 %
    \advance\dimen0-16\dimen2 %
    \edef\PE@result{%
      \PE@result
      \PE@HexDigit{\dimen2 }%
      \PE@HexDigit{\dimen0 }%
    }%
  }%
\fi
%    \end{macrocode}
%    \end{macro}
%    \begin{macro}{\PE@HexDigit}
%    \begin{macrocode}
\def\PE@HexDigit#1{%
  \expandafter\string
  \ifcase#1%
    0\or 1\or 2\or 3\or 4\or 5\or 6\or 7\or 8\or 9\or
    A\or B\or C\or D\or E\or F%
  \fi
}
%    \end{macrocode}
%    \end{macro}
%
% \subsubsection{Character code to octal number}
%
%    \begin{macro}{\PE@OctChar}
%    \begin{macrocode}
\ifPE@etex
  \def\PE@OctChar#1{%
    \expandafter\PE@@OctChar
        \the\numexpr\dimexpr.015625\dimexpr`#1sp\relax\relax
            \expandafter\relax
        \expandafter\relax
        \the\numexpr\dimexpr.125\dimexpr`#1sp\relax\relax\relax
        \relax
        #1%
  }%
  \def\PE@@OctChar#1\relax#2\relax#3{%
    \PE@backslash
    #1%
    \the\numexpr#2-8*#1\relax
    \the\numexpr\dimexpr`#3sp\relax-8*#2\relax
  }%
\else
  \def\PE@OctChar#1{%
    \dimen0=`#1sp%
    \dimen2=.125\dimen0 %
    \dimen4=.125\dimen2 %
    \advance\dimen0-8\dimen2 %
    \advance\dimen2-8\dimen4 %
    \edef\PE@result{%
      \PE@result
      \PE@backslash
      \number\dimen4 %
      \number\dimen2 %
      \number\dimen0 %
    }%
  }%
\fi
%    \end{macrocode}
%    \end{macro}
%
% \subsubsection{Unpack hex string}
%
%    \begin{macro}{\PE@UnescapeHex}
%    \begin{macrocode}
\def\PE@UnescapeHex#1{%
  \begingroup
    \PE@InitUccodeHexDigit
    \def\PE@result{}%
    \expandafter\PE@DeHex#1\relax\relax
  \expandafter\endgroup
  \expandafter\def\expandafter#1\expandafter{\PE@result}%
}
%    \end{macrocode}
%    \end{macro}
%    \begin{macro}{\PE@DeHex}
%    \begin{macrocode}
\def\PE@DeHex#1#2{%
  \ifx#2\relax
    \ifx#1\relax
      \let\PE@next\relax
    \else
      \uppercase{%
        \def\PE@testA{#1}%
      }%
      \ifcase\expandafter\PE@TestUcHexDigit\PE@testA
        \def\PE@next{%
          \PE@DeHex#10\relax\relax
        }%
      \else
        \let\PE@next\relax
      \fi
    \fi
  \else
    \uppercase{%
      \def\PE@testA{#1}%
      \def\PE@testB{#2}%
    }%
    \ifcase\expandafter\PE@TestUcHexDigit\PE@testA
      \ifcase\expandafter\PE@TestUcHexDigit\PE@testB
        \uccode\ltx@zero="\PE@testA\PE@testB\relax
        \ifnum\uccode\ltx@zero=32 %
          \let\PE@temp\PE@space@space
        \else
          \uppercase{%
            \def\PE@temp{^^@}%
          }%
        \fi
        \edef\PE@result{\PE@result\PE@temp}%
        \let\PE@next\PE@DeHex
      \else
        % invalid input sequence
        \def\PE@next{%
          \PE@DeHex#1%
        }%
      \fi
    \else
      % invalid input sequence
      \def\PE@next{\PE@DeHex#2}%
    \fi
  \fi
  \PE@next
}
%    \end{macrocode}
%    \end{macro}
%
% \subsubsection{Conversion to PDF name}
%
%    \begin{macro}{\PE@EscapeName}
%    \begin{macrocode}
\ifPE@etex
  \def\PE@EscapeName#1{%
    \edef#1{\expandafter\PE@EscapeNameTokens#1\relax}%
  }%
\else
  \def\PE@EscapeName#1{%
    \def\PE@result{}%
    \expandafter\PE@EscapeNameTokens#1\relax
    \let#1\PE@result
  }%
\fi
%    \end{macrocode}
%    \end{macro}
%    \begin{macro}{\PE@EscapeNameTokens}
%    \begin{macrocode}
\def\PE@EscapeNameTokens#1{%
  \ifx\relax#1%
  \else
    \ifnum`#1<33 %
      \ifcase`#1 %
        % drop illegal zero
      \else
        \PE@EscapeNameAdd\PE@hash
        \PE@HexChar#1%
      \fi
    \else
      \ifnum`#1>126 %
        \PE@EscapeNameAdd\PE@hash
        \PE@HexChar#1%
      \else  \ifnum`#1=35  \PE@EscapeNameHashChar 23% #
        \else\ifnum`#1=37  \PE@EscapeNameHashChar 25% %
        \else\ifnum`#1=40  \PE@EscapeNameHashChar 28% (
        \else\ifnum`#1=41  \PE@EscapeNameHashChar 29% )
        \else\ifnum`#1=47  \PE@EscapeNameHashChar 2F% /
        \else\ifnum`#1=60  \PE@EscapeNameHashChar 3C% <
        \else\ifnum`#1=62  \PE@EscapeNameHashChar 3E% >
        \else\ifnum`#1=91  \PE@EscapeNameHashChar 5B% [
        \else\ifnum`#1=93  \PE@EscapeNameHashChar 5D% ]
        \else\ifnum`#1=123 \PE@EscapeNameHashChar 7B% {
        \else\ifnum`#1=125 \PE@EscapeNameHashChar 7D% }
        \else
          \PE@EscapeNameAdd{#1}%
        \fi\fi\fi\fi\fi\fi\fi\fi\fi\fi\fi
      \fi
    \fi
    \expandafter\PE@EscapeNameTokens
  \fi
}%
\def\PE@EscapeNameHashChar#1#2{%
  \PE@EscapeNameAdd{\PE@hash\string#1\string#2}%
}%
%    \end{macrocode}
%    \end{macro}
%    \begin{macro}{\PE@EscapeNameAdd}
%    \begin{macrocode}
\ifPE@etex
  \def\PE@EscapeNameAdd#1{#1}%
\else
  \def\PE@EscapeNameAdd#1{%
    \edef\PE@result{%
      \PE@result
      #1%
    }%
  }%
\fi
%    \end{macrocode}
%    \end{macro}
%
% \subsubsection{Conversion to PDF string}
%
%    \begin{macro}{\PE@EscapeString}
%    \begin{macrocode}
\ifPE@etex
  \def\PE@EscapeString#1{%
    \edef#1{\expandafter\PE@EscapeStringTokens#1\relax}%
  }%
\else
  \def\PE@EscapeString#1{%
    \begingroup
      \def\PE@result{}%
      \expandafter\PE@EscapeStringTokens#1\relax
    \expandafter\endgroup
    \expandafter\def\expandafter#1\expandafter{\PE@result}%
  }%
\fi
%    \end{macrocode}
%    \end{macro}
%    \begin{macro}{\PE@EscapeStringTokens}
%    \begin{macrocode}
\def\PE@EscapeStringTokens#1{%
  \ifx\relax#1%
  \else
    \ifnum`#1<33 %
      \PE@OctChar#1%
    \else
      \ifnum`#1>126 %
        \PE@OctChar#1%
      \else  \ifnum`#1=40 \PE@EscapeStringAdd{\string\(}% (
        \else\ifnum`#1=41 \PE@EscapeStringAdd{\string\)}% )
        \else\ifnum`#1=92 \PE@EscapeStringAdd{\string\\}% \
        \else
          \PE@EscapeStringAdd{#1}%
        \fi\fi\fi
      \fi
    \fi
    \expandafter\PE@EscapeStringTokens
  \fi
}%
%    \end{macrocode}
%    \end{macro}
%    \begin{macro}{\PE@EscapeStringAdd}
%    \begin{macrocode}
\ifPE@etex
  \def\PE@EscapeStringAdd#1{#1}%
\else
  \def\PE@EscapeStringAdd#1{%
    \edef\PE@result{%
      \PE@result
      #1%
    }%
  }%
\fi
%    \end{macrocode}
%    \end{macro}
%
%    \begin{macrocode}
\PE@AtEnd%
%    \end{macrocode}
%
%    \begin{macrocode}
%</package>
%    \end{macrocode}
%
% \section{Test}
%
% \subsection{Catcode checks for loading}
%
%    \begin{macrocode}
%<*test1>
%    \end{macrocode}
%    \begin{macrocode}
\catcode`\{=1 %
\catcode`\}=2 %
\catcode`\#=6 %
\catcode`\@=11 %
\expandafter\ifx\csname count@\endcsname\relax
  \countdef\count@=255 %
\fi
\expandafter\ifx\csname @gobble\endcsname\relax
  \long\def\@gobble#1{}%
\fi
\expandafter\ifx\csname @firstofone\endcsname\relax
  \long\def\@firstofone#1{#1}%
\fi
\expandafter\ifx\csname loop\endcsname\relax
  \expandafter\@firstofone
\else
  \expandafter\@gobble
\fi
{%
  \def\loop#1\repeat{%
    \def\body{#1}%
    \iterate
  }%
  \def\iterate{%
    \body
      \let\next\iterate
    \else
      \let\next\relax
    \fi
    \next
  }%
  \let\repeat=\fi
}%
\def\RestoreCatcodes{}
\count@=0 %
\loop
  \edef\RestoreCatcodes{%
    \RestoreCatcodes
    \catcode\the\count@=\the\catcode\count@\relax
  }%
\ifnum\count@<255 %
  \advance\count@ 1 %
\repeat

\def\RangeCatcodeInvalid#1#2{%
  \count@=#1\relax
  \loop
    \catcode\count@=15 %
  \ifnum\count@<#2\relax
    \advance\count@ 1 %
  \repeat
}
\def\RangeCatcodeCheck#1#2#3{%
  \count@=#1\relax
  \loop
    \ifnum#3=\catcode\count@
    \else
      \errmessage{%
        Character \the\count@\space
        with wrong catcode \the\catcode\count@\space
        instead of \number#3%
      }%
    \fi
  \ifnum\count@<#2\relax
    \advance\count@ 1 %
  \repeat
}
\def\space{ }
\expandafter\ifx\csname LoadCommand\endcsname\relax
  \def\LoadCommand{\input pdfescape.sty\relax}%
\fi
\def\Test{%
  \RangeCatcodeInvalid{0}{47}%
  \RangeCatcodeInvalid{58}{64}%
  \RangeCatcodeInvalid{91}{96}%
  \RangeCatcodeInvalid{123}{255}%
  \catcode`\@=12 %
  \catcode`\\=0 %
  \catcode`\%=14 %
  \LoadCommand
  \RangeCatcodeCheck{0}{36}{15}%
  \RangeCatcodeCheck{37}{37}{14}%
  \RangeCatcodeCheck{38}{47}{15}%
  \RangeCatcodeCheck{48}{57}{12}%
  \RangeCatcodeCheck{58}{63}{15}%
  \RangeCatcodeCheck{64}{64}{12}%
  \RangeCatcodeCheck{65}{90}{11}%
  \RangeCatcodeCheck{91}{91}{15}%
  \RangeCatcodeCheck{92}{92}{0}%
  \RangeCatcodeCheck{93}{96}{15}%
  \RangeCatcodeCheck{97}{122}{11}%
  \RangeCatcodeCheck{123}{255}{15}%
  \RestoreCatcodes
}
\Test
\csname @@end\endcsname
\end
%    \end{macrocode}
%    \begin{macrocode}
%</test1>
%    \end{macrocode}
%
% \subsection{Macro tests}
%
%    \begin{macrocode}
%<*test2|test3|test4|test5>
\NeedsTeXFormat{LaTeX2e}
\makeatletter
%    \end{macrocode}
%
% \subsection{Test with \cs{pdfescape...} commands}
%
%    \begin{macrocode}
%<*test2>
\ProvidesFile{pdfescape-test2.tex}%
    [2011/11/25 v1.13 Test with \string\pdfescape... commands]%
%</test2>
%    \end{macrocode}
%
% \subsection{Test without \cs{pdfescape...}, with \eTeX}
%
%    \begin{macrocode}
%<*test3>
\ProvidesFile{pdfescape-test3.tex}%
    [2011/11/25 v1.13 Test without \string\pdfescape..., with e-TeX]%
%</test3>
%    \end{macrocode}
%
% \subsection{Test without \cs{pdfescape...} and \eTeX}
%
%    \begin{macrocode}
%<*test4>
\ProvidesFile{pdfescape-test4.tex}%
    [2011/11/25 v1.13 Test without \string\pdfescape... and e-TeX]%
%</test4>
%    \end{macrocode}
%
% \subsection{Test with \LuaTeX}
%
%    \begin{macrocode}
%<*test5>
\ProvidesFile{pdfescape-test5.tex}%
    [2011/11/25 v1.13 Test with LuaTeX]%
%</test5>
%    \end{macrocode}
%
% \subsection{Check/ensure test preconditions}
%
% \subsubsection{Check \cs{pdfescape...}}
%
%    \begin{macrocode}
%<*test2>
\@ifundefined{pdfescapehex}{%
  \PackageError{pdfescape-test2}{%
    Missing \string\pdfescape... commands%
  }{Test aborted.}%
  \stop
}{}
%</test2>
%    \end{macrocode}
%    \begin{macrocode}
%<*test3|test4>
\let\pdfescapehex\@undefined
\let\pdfunescapehex\@undefined
\let\pdfescapename\@undefined
\let\pdfescapestring\@undefined
%</test3|test4>
%    \end{macrocode}
%
% \subsubsection{Check \eTeX}
%
%    \begin{macrocode}
%<*test3>
\@ifundefined{numexpr}{%
  \PackageError{pdfescape-test3}{%
    Missing \eTeX
  }{Test aborted.}%
  \stop
}{}
%</test3>
%    \end{macrocode}
% Package \xpackage{qstest} uses \eTeX, thus \eTeX's features
% can only be disabled later during loading of package \xpackage{pdfescape}.
%
% \subsubsection{Check \LuaTeX}
%
%    \begin{macrocode}
%<*test5>
\@ifundefined{directlua}{%
  \PackageError{pdfescape-test5}{%
    Missing LuaTeX%
  }{Test aborted.}%
  \stop
}{}
%</test5>
%    \end{macrocode}
%
% \subsection{Common part}
%
%    The files for testing uses the framework, provided by the
%    new package \xpackage{qstest} of David Kastrup.
%    \begin{macrocode}
\RequirePackage{qstest}
\IncludeTests{*}
\LogTests{log}{*}{*}

\newcommand*{\ExpectVar}[2]{%
  \ifx#1#2%
  \else
    \begingroup
      \@onelevel@sanitize#1%
      \@onelevel@sanitize#2%
      \typeout{[#1] <> [#2]}% hash-ok
    \endgroup
  \fi
  \Expect*{\ifx#1#2true\else false\fi}{true}%
}

\makeatletter
\begingroup
  \gdef\AllBytes{}%
  \count@=0 %
  \catcode0=12 %
  \@whilenum\count@<256 \do{%
    \lccode0=\count@
    \ifnum\count@=32 %
      \xdef\AllBytes{\AllBytes\space}%
    \else
      \lowercase{%
        \xdef\AllBytes{\AllBytes^^@}%
      }%
    \fi
    \advance\count@ by 1 %
  }%
\endgroup
\newcommand*{\AllBytesHex}{%
  000102030405060708090A0B0C0D0E0F%
  101112131415161718191A1B1C1D1E1F%
  202122232425262728292A2B2C2D2E2F%
  303132333435363738393A3B3C3D3E3F%
  404142434445464748494A4B4C4D4E4F%
  505152535455565758595A5B5C5D5E5F%
  606162636465666768696A6B6C6D6E6F%
  707172737475767778797A7B7C7D7E7F%
  808182838485868788898A8B8C8D8E8F%
  909192939495969798999A9B9C9D9E9F%
  A0A1A2A3A4A5A6A7A8A9AAABACADAEAF%
  B0B1B2B3B4B5B6B7B8B9BABBBCBDBEBF%
  C0C1C2C3C4C5C6C7C8C9CACBCCCDCECF%
  D0D1D2D3D4D5D6D7D8D9DADBDCDDDEDF%
  E0E1E2E3E4E5E6E7E8E9EAEBECEDEEEF%
  F0F1F2F3F4F5F6F7F8F9FAFBFCFDFEFF%
}
\@onelevel@sanitize\AllBytesHex
\expandafter\lowercase\expandafter{%
  \expandafter\newcommand\expandafter*\expandafter\AllBytesHexLC
      \expandafter{\AllBytesHex}%
}
\newcommand*{\AllBytesName}{}
\begingroup
  \catcode`\#=12 %
  \xdef\AllBytesName{%
    #01#02#03#04#05#06#07#08#09#0A#0B#0C#0D#0E#0F%
    #10#11#12#13#14#15#16#17#18#19#1A#1B#1C#1D#1E#1F%
    #20!"#23$#25&'#28#29*+,-.#2F%
    0123456789:;#3C=#3E?%
    @ABCDEFGHIJKLMNO%
    PQRSTUVWXYZ#5B\@backslashchar#5D^_%
    `abcdefghijklmno%
    pqrstuvwxyz#7B|#7D\string~#7F%
    #80#81#82#83#84#85#86#87#88#89#8A#8B#8C#8D#8E#8F%
    #90#91#92#93#94#95#96#97#98#99#9A#9B#9C#9D#9E#9F%
    #A0#A1#A2#A3#A4#A5#A6#A7#A8#A9#AA#AB#AC#AD#AE#AF%
    #B0#B1#B2#B3#B4#B5#B6#B7#B8#B9#BA#BB#BC#BD#BE#BF%
    #C0#C1#C2#C3#C4#C5#C6#C7#C8#C9#CA#CB#CC#CD#CE#CF%
    #D0#D1#D2#D3#D4#D5#D6#D7#D8#D9#DA#DB#DC#DD#DE#DF%
    #E0#E1#E2#E3#E4#E5#E6#E7#E8#E9#EA#EB#EC#ED#EE#EF%
    #F0#F1#F2#F3#F4#F5#F6#F7#F8#F9#FA#FB#FC#FD#FE#FF%
  }%
\endgroup
\@onelevel@sanitize\AllBytesName

\newcommand*{\AllBytesString}{}
\begingroup
  \def\|{|}%
  \edef\%{\@percentchar}%
  \catcode`\|=0 %
  \catcode`\#=12 %
  \catcode`\~=12 %
  \catcode`\\=12 %
  |xdef|AllBytesString{%
    \000\001\002\003\004\005\006\007\010\011\012\013\014\015\016\017%
    \020\021\022\023\024\025\026\027\030\031\032\033\034\035\036\037%
    \040!"#$|%&'\(\)*+,-./%
    0123456789:;<=>?%
    @ABCDEFGHIJKLMNO%
    PQRSTUVWXYZ[\\]^_%
    `abcdefghijklmno%
    pqrstuvwxyz{||}~\177%
    \200\201\202\203\204\205\206\207\210\211\212\213\214\215\216\217%
    \220\221\222\223\224\225\226\227\230\231\232\233\234\235\236\237%
    \240\241\242\243\244\245\246\247\250\251\252\253\254\255\256\257%
    \260\261\262\263\264\265\266\267\270\271\272\273\274\275\276\277%
    \300\301\302\303\304\305\306\307\310\311\312\313\314\315\316\317%
    \320\321\322\323\324\325\326\327\330\331\332\333\334\335\336\337%
    \340\341\342\343\344\345\346\347\350\351\352\353\354\355\356\357%
    \360\361\362\363\364\365\366\367\370\371\372\373\374\375\376\377%
  }%
|endgroup
\@onelevel@sanitize\AllBytesString

%<*test4>
\let\org@detokenize\detokenize
\let\detokenize\@undefined
\let\org@numexpr\numexpr
\let\numexpr\@undefined
%</test4>
\RequirePackage{pdfescape}
%<*test4>
\let\detokenize\org@detokenize
\let\numexpr\org@numexpr
%</test4>

\begin{qstest}{all-hex}{\AllBytes, escapehex}
  \EdefEscapeHex\x{\AllBytes}%
  \Expect*{\x}*{\AllBytesHex}%
  \ExpectVar\x\AllBytesHex
\end{qstest}

\begin{qstest}{all-unhex}{\AllBytesHex, unescapehex}
  \EdefUnescapeHex\x{\AllBytesHex}%
  \Expect*{\x}*{\AllBytes}%
  \ExpectVar\x\AllBytes
\end{qstest}

\begin{qstest}{all-unhex-lc}{\AllBytesHexLC, unescapehex, lowercase}
  \EdefUnescapeHex\x{\AllBytesHexLC}%
  \Expect*{\x}*{\AllBytes}%
  \ExpectVar\x\AllBytes
\end{qstest}

\begin{qstest}{unhex-incomplete}{unescapehex, incomplete}
  \EdefUnescapeHex\x{4}%
  \Expect*{\x}{@}%
\end{qstest}

\begin{qstest}{unhex-space}{unescapehex, space}
  \EdefUnescapeHex\x{20}%
  \Expect*{\x}{ }%
  \ExpectVar\x\space
\end{qstest}

\begin{qstest}{unhex-spaces}{unescapehex, spaces}
  \EdefUnescapeHex\x{204020204120}%
  \def\y#1{%
    \edef\z{#1\string @#1#1\string A#1}%
  }\y{ }%
  \Expect*{\x}*{\z}%
  \ExpectVar\x\z
\end{qstest}

\begin{qstest}{unhex-hash}{unescapehex, hash}
  \catcode`\#=12 %
  \EdefUnescapeHex\x{#20}%
  \ExpectVar\x\space
\end{qstest}

\begin{qstest}{unhex-invalid}{unescapehex, invalid}
  \def\test#1#2{%
    \EdefUnescapeHex\x{#1}%
    \edef\y{#2}%
    \@onelevel@sanitize\y
    \ExpectVar\x\y
  }%
%<*test2>
  \edef\x{\pdfunescapehex{4X}}%
  \edef\y{\string @}%
  \ifx\x\y
  \else
    \def~{\space}%
    \typeout{************************************}%
    \typeout{* Your pdfTeX contains bug 777.~~~~*}%
    \typeout{* This test is redefined as dummy, *}%
    \typeout{* because it triggers the bug.~~~~~*}%
    \typeout{************************************}%
    \def\test#1#2{}%
  \fi
%</test2>
  \test{X}{}%
  \test{XY}{}%
  \test{XYZ}{}%
  \test{A}{^^a0}%
  \test{AX}{^^a0}%
  \test{XA}{^^a0}%
  \test{XXAXX}{^^a0}%
\end{qstest}

\begin{qstest}{all-name}{\AllBytes, escapename}
  \EdefEscapeName\x{\AllBytes}%
  \Expect*{\x}*{\AllBytesName}%
  \ExpectVar\x\AllBytesName
\end{qstest}

\begin{qstest}{all-string}{\AllBytes, escapestring}
  \EdefEscapeString\x{\AllBytes}%
  \Expect*{\x}*{\AllBytesString}%
  \ExpectVar\x\AllBytesString
\end{qstest}

\begin{qstest}{uchexdigit}{unescape, uppercase hex digit}
  \catcode`\@=11 %
  \catcode0=12 %
  \def\test#1#2{%
    \uccode0=#1\relax
    \uppercase{%
      \def\x{^^@}%
    }%
    \Expect*{%
       \ifcase\expandafter\PE@TestUcHexDigit\x
         true%
       \else
         false%
       \fi
    }{#2}%
  }%
  \def\range#1#2#3{%
    \count0=#1\relax
    \loop
    \ifnum\count0<#2\relax
      \test{\count0}{#3}%
      \advance\count0 by 1 %
    \repeat
  }%
  \range{0}{47}{false}%
  \range{48}{57}{true}%
  \range{58}{64}{false}%
  \range{65}{70}{true}%
  \range{71}{255}{false}%
\end{qstest}

\begin{qstest}{unescapename}{unescapename}
  \def\test#1#2{%
    \EdefUnescapeName\x{#1}%
    \edef\y{#2}%
    \@onelevel@sanitize\y
    \ExpectVar\x\y
  }%
  \catcode`\#=12 %
  \catcode0=12 %
  \test{}{}%
  \test{x}{x}%
  \test{xy}{xy}%
  \test{#}{#}%
  \test{##}{##}%
  \test{###}{###}%
  \test{####}{####}%
  \test{#x}{#x}%
  \test{#xy}{#xy}%
  \test{#1}{#1}%
  \test{#40}{@}%
  \test{#400}{@0}%
  \test{#4x0}{#4x0}%
  \test{#ab}{^^ab}%
  \test{#00}{^^@}%
  \test{x#40y#40z}{x@y@z}%
  \test{#40#40#40#40}{@@@@}%
  \test{a#x}{a#x}%
  \test{a#xy}{a#xy}%
  \test{a#1}{a#1}%
  \test{a#40}{a@}%
  \test{a#400}{a@0}%
  \test{#20}{ }%
  \test{a#20}{a }%
  \test{a#20b}{a b}%
  \test{a#20#20#20b}{a \space\space b}%
\end{qstest}

\begin{qstest}{unescapestring}{unescapestring}
  \def\test#1#2{%
    \EdefUnescapeString\x{#1}%
    \edef\y{#2}%
    \@onelevel@sanitize\y
    \ExpectVar\x\y
  }%
  \catcode0=12 %
  \def\DefChar#1#2{%
    \begingroup
      \uccode0=#2\relax
    \uppercase{\endgroup
      \def#1{^^@}%
    }%
  }%
  \DefChar\nul{0}%
  \DefChar\one{1}%
  \DefChar\bel{8}%
  \DefChar\tab{9}%
  \DefChar\lf{10}%
  \DefChar\ff{12}%
  \DefChar\cr{13}%
  \DefChar\\{92}%
  \test{}{}%
  \test{a}{a}%
  \test{\\}{}%
  \test{\\\\}{\\}%
  \test{\\\\y}{\\y}%
  \test{\\000}{\nul}%
  \test{\\b}{\bel}%
  \test{\\t}{\tab}%
  \test{\\n}{\lf}%
  \test{\\f}{\ff}%
  \test{\\r}{\cr}%
  \test{\\(}{(}%
  \test{\\)}{)}%
  \test{\\040}{ }%
  \test{\\100}{@}%
  \test{\\40}{ }%
  \test{\\1}{\one}%
  \test{\\01}{\one}%
  \test{\\001}{\one}%
  \test{\\18}{\one8}%
  \test{\\018}{\one8}%
  \test{\\0018}{\one8}%
  \test{x\\}{x}%
  \test{x\\\\}{x\\}%
  \test{x\\\\y}{x\\y}%
  \test{x\\000}{x\nul}%
  \test{x\\b}{x\bel}%
  \test{x\\t}{x\tab}%
  \test{x\\n}{x\lf}%
  \test{x\\f}{x\ff}%
  \test{x\\r}{x\cr}%
  \test{x\\(}{x(}%
  \test{x\\)}{x)}%
  \test{x\\040}{x }%
  \test{x\\100}{x@}%
  \test{x\\40}{x }%
  \test{x\\1}{x\one}%
  \test{x\\01}{x\one}%
  \test{x\\001}{x\one}%
  \test{x\\18}{x\one8}%
  \test{x\\018}{x\one8}%
  \test{x\\0018}{x\one8}%
  \test{\\b\\t\\n\\f\\r\\(\\)\\\\\\000\\040}{%
    \bel\tab\lf\ff\cr()\\\nul\space
  }%
  \test{\\\lf}{}%
  \test{x\\\lf}{x}%
  \test{\cr}{\lf}%
  \test{\cr\lf}{\lf}%
  \test{\lf}{\lf}%
  \test{\lf\cr}{\lf\lf}%
  \test{x\cr}{x\lf}%
  \test{x\cr\lf}{x\lf}%
  \test{x\lf}{x\lf}%
  \test{x\lf\cr}{x\lf\lf}%
  \test{x\\\cr\lf y\cr}{xy\lf}%
%
  \test{\\409}{ 9}%
  \test{\\800}{800}%
  \test{\\900}{900}%
  \test{\\578}{/8}%
  \test{\\477}{?}%
  \test{\\377}{^^ff}%
  \test{\\777}{^^ff}%
  \test{\\7777}{^^ff7}%
\end{qstest}
\stop
%    \end{macrocode}
%
%    \begin{macrocode}
%</test2|test3|test4|test5>
%    \end{macrocode}
%
% \subsubsection{Test for \hologo{iniTeX}}
%
%    \begin{macrocode}
%<*test6>
\input pdfescape.sty\relax
\catcode`\{=1 %
\catcode`\}=2 %
\catcode`\#=6 %
\catcode`\^=7 %
\catcode`\@=11 %
%    \end{macrocode}
%    \begin{macrocode}
\begingroup
  \catcode`\@=11 %
  \countdef\count@=255 %
  \def\space{ }%
  \long\def\@whilenum#1\do #2{%
    \ifnum #1\relax
      #2\relax
      \@iwhilenum{#1\relax#2\relax}%
    \fi
  }%
  \long\def\@iwhilenum#1{%
    \ifnum #1%
      \expandafter\@iwhilenum
    \else
      \expandafter\ltx@gobble
    \fi
    {#1}%
  }%
  \gdef\AllBytes{}%
  \count@=0 %
  \catcode0=12 %
  \@whilenum\count@<256 \do{%
    \lccode0=\count@
    \ifnum\count@=32 %
      \xdef\AllBytes{\AllBytes\space}%
    \else
      \lowercase{%
        \xdef\AllBytes{\AllBytes^^@}%
      }%
    \fi
    \advance\count@ by 1 %
  }%
\endgroup
\def\AllBytesHex{%
  000102030405060708090A0B0C0D0E0F%
  101112131415161718191A1B1C1D1E1F%
  202122232425262728292A2B2C2D2E2F%
  303132333435363738393A3B3C3D3E3F%
  404142434445464748494A4B4C4D4E4F%
  505152535455565758595A5B5C5D5E5F%
  606162636465666768696A6B6C6D6E6F%
  707172737475767778797A7B7C7D7E7F%
  808182838485868788898A8B8C8D8E8F%
  909192939495969798999A9B9C9D9E9F%
  A0A1A2A3A4A5A6A7A8A9AAABACADAEAF%
  B0B1B2B3B4B5B6B7B8B9BABBBCBDBEBF%
  C0C1C2C3C4C5C6C7C8C9CACBCCCDCECF%
  D0D1D2D3D4D5D6D7D8D9DADBDCDDDEDF%
  E0E1E2E3E4E5E6E7E8E9EAEBECEDEEEF%
  F0F1F2F3F4F5F6F7F8F9FAFBFCFDFEFF%
}
\ltx@onelevel@sanitize\AllBytesHex
\expandafter\lowercase\expandafter{%
  \expandafter\def\expandafter\AllBytesHexLC
      \expandafter{\AllBytesHex}%
}
\begingroup
  \catcode`\#=12 %
  \xdef\AllBytesName{%
    #01#02#03#04#05#06#07#08#09#0A#0B#0C#0D#0E#0F%
    #10#11#12#13#14#15#16#17#18#19#1A#1B#1C#1D#1E#1F%
    #20!"#23$#25&'#28#29*+,-.#2F%
    0123456789:;#3C=#3E?%
    @ABCDEFGHIJKLMNO%
    PQRSTUVWXYZ#5B\ltx@backslashchar#5D^_%
    `abcdefghijklmno%
    pqrstuvwxyz#7B|#7D\string~#7F%
    #80#81#82#83#84#85#86#87#88#89#8A#8B#8C#8D#8E#8F%
    #90#91#92#93#94#95#96#97#98#99#9A#9B#9C#9D#9E#9F%
    #A0#A1#A2#A3#A4#A5#A6#A7#A8#A9#AA#AB#AC#AD#AE#AF%
    #B0#B1#B2#B3#B4#B5#B6#B7#B8#B9#BA#BB#BC#BD#BE#BF%
    #C0#C1#C2#C3#C4#C5#C6#C7#C8#C9#CA#CB#CC#CD#CE#CF%
    #D0#D1#D2#D3#D4#D5#D6#D7#D8#D9#DA#DB#DC#DD#DE#DF%
    #E0#E1#E2#E3#E4#E5#E6#E7#E8#E9#EA#EB#EC#ED#EE#EF%
    #F0#F1#F2#F3#F4#F5#F6#F7#F8#F9#FA#FB#FC#FD#FE#FF%
  }%
\endgroup
\ltx@onelevel@sanitize\AllBytesName
\edef\AllBytesFromName{\expandafter\ltx@gobble\AllBytes}

\begingroup
  \def\|{|}%
  \edef\%{\ltx@percentchar}%
  \catcode`\|=0 %
  \catcode`\#=12 %
  \catcode`\~=12 %
  \catcode`\\=12 %
  |xdef|AllBytesString{%
    \000\001\002\003\004\005\006\007\010\011\012\013\014\015\016\017%
    \020\021\022\023\024\025\026\027\030\031\032\033\034\035\036\037%
    \040!"#$|%&'\(\)*+,-./%
    0123456789:;<=>?%
    @ABCDEFGHIJKLMNO%
    PQRSTUVWXYZ[\\]^_%
    `abcdefghijklmno%
    pqrstuvwxyz{||}~\177%
    \200\201\202\203\204\205\206\207\210\211\212\213\214\215\216\217%
    \220\221\222\223\224\225\226\227\230\231\232\233\234\235\236\237%
    \240\241\242\243\244\245\246\247\250\251\252\253\254\255\256\257%
    \260\261\262\263\264\265\266\267\270\271\272\273\274\275\276\277%
    \300\301\302\303\304\305\306\307\310\311\312\313\314\315\316\317%
    \320\321\322\323\324\325\326\327\330\331\332\333\334\335\336\337%
    \340\341\342\343\344\345\346\347\350\351\352\353\354\355\356\357%
    \360\361\362\363\364\365\366\367\370\371\372\373\374\375\376\377%
  }%
|endgroup
\ltx@onelevel@sanitize\AllBytesString
%    \end{macrocode}
%    \begin{macrocode}
\def\msg#{\immediate\write16}
\def\Test#1#2#3{%
  \begingroup
    #1\TestResult{#2}%
    \ifx\TestResult#3%
    \else
      \newlinechar=10 %
      \msg{Expect:^^J#3}%
      \msg{Result:^^J\TestResult}%
      \errmessage{\string#2 -\string#1-> \string#3}%
    \fi
  \endgroup
}
\Test\EdefEscapeHex\AllBytes\AllBytesHex
\Test\EdefUnescapeHex\AllBytesHex\AllBytes
\Test\EdefEscapeName\AllBytes\AllBytesName
\Test\EdefUnescapeName\AllBytesName\AllBytesFromName
\Test\EdefEscapeString\AllBytes\AllBytesString
\Test\EdefUnescapeString\AllBytesString\AllBytes
%    \end{macrocode}
%    \begin{macrocode}
\csname @@end\endcsname\end
%</test6>
%    \end{macrocode}
%
% \section{Installation}
%
% \subsection{Download}
%
% \paragraph{Package.} This package is available on
% CTAN\footnote{\url{ftp://ftp.ctan.org/tex-archive/}}:
% \begin{description}
% \item[\CTAN{macros/latex/contrib/oberdiek/pdfescape.dtx}] The source file.
% \item[\CTAN{macros/latex/contrib/oberdiek/pdfescape.pdf}] Documentation.
% \end{description}
%
%
% \paragraph{Bundle.} All the packages of the bundle `oberdiek'
% are also available in a TDS compliant ZIP archive. There
% the packages are already unpacked and the documentation files
% are generated. The files and directories obey the TDS standard.
% \begin{description}
% \item[\CTAN{install/macros/latex/contrib/oberdiek.tds.zip}]
% \end{description}
% \emph{TDS} refers to the standard ``A Directory Structure
% for \TeX\ Files'' (\CTAN{tds/tds.pdf}). Directories
% with \xfile{texmf} in their name are usually organized this way.
%
% \subsection{Bundle installation}
%
% \paragraph{Unpacking.} Unpack the \xfile{oberdiek.tds.zip} in the
% TDS tree (also known as \xfile{texmf} tree) of your choice.
% Example (linux):
% \begin{quote}
%   |unzip oberdiek.tds.zip -d ~/texmf|
% \end{quote}
%
% \paragraph{Script installation.}
% Check the directory \xfile{TDS:scripts/oberdiek/} for
% scripts that need further installation steps.
% Package \xpackage{attachfile2} comes with the Perl script
% \xfile{pdfatfi.pl} that should be installed in such a way
% that it can be called as \texttt{pdfatfi}.
% Example (linux):
% \begin{quote}
%   |chmod +x scripts/oberdiek/pdfatfi.pl|\\
%   |cp scripts/oberdiek/pdfatfi.pl /usr/local/bin/|
% \end{quote}
%
% \subsection{Package installation}
%
% \paragraph{Unpacking.} The \xfile{.dtx} file is a self-extracting
% \docstrip\ archive. The files are extracted by running the
% \xfile{.dtx} through \plainTeX:
% \begin{quote}
%   \verb|tex pdfescape.dtx|
% \end{quote}
%
% \paragraph{TDS.} Now the different files must be moved into
% the different directories in your installation TDS tree
% (also known as \xfile{texmf} tree):
% \begin{quote}
% \def\t{^^A
% \begin{tabular}{@{}>{\ttfamily}l@{ $\rightarrow$ }>{\ttfamily}l@{}}
%   pdfescape.sty & tex/generic/oberdiek/pdfescape.sty\\
%   pdfescape.pdf & doc/latex/oberdiek/pdfescape.pdf\\
%   test/pdfescape-test1.tex & doc/latex/oberdiek/test/pdfescape-test1.tex\\
%   test/pdfescape-test2.tex & doc/latex/oberdiek/test/pdfescape-test2.tex\\
%   test/pdfescape-test3.tex & doc/latex/oberdiek/test/pdfescape-test3.tex\\
%   test/pdfescape-test4.tex & doc/latex/oberdiek/test/pdfescape-test4.tex\\
%   test/pdfescape-test5.tex & doc/latex/oberdiek/test/pdfescape-test5.tex\\
%   test/pdfescape-test6.tex & doc/latex/oberdiek/test/pdfescape-test6.tex\\
%   pdfescape.dtx & source/latex/oberdiek/pdfescape.dtx\\
% \end{tabular}^^A
% }^^A
% \sbox0{\t}^^A
% \ifdim\wd0>\linewidth
%   \begingroup
%     \advance\linewidth by\leftmargin
%     \advance\linewidth by\rightmargin
%   \edef\x{\endgroup
%     \def\noexpand\lw{\the\linewidth}^^A
%   }\x
%   \def\lwbox{^^A
%     \leavevmode
%     \hbox to \linewidth{^^A
%       \kern-\leftmargin\relax
%       \hss
%       \usebox0
%       \hss
%       \kern-\rightmargin\relax
%     }^^A
%   }^^A
%   \ifdim\wd0>\lw
%     \sbox0{\small\t}^^A
%     \ifdim\wd0>\linewidth
%       \ifdim\wd0>\lw
%         \sbox0{\footnotesize\t}^^A
%         \ifdim\wd0>\linewidth
%           \ifdim\wd0>\lw
%             \sbox0{\scriptsize\t}^^A
%             \ifdim\wd0>\linewidth
%               \ifdim\wd0>\lw
%                 \sbox0{\tiny\t}^^A
%                 \ifdim\wd0>\linewidth
%                   \lwbox
%                 \else
%                   \usebox0
%                 \fi
%               \else
%                 \lwbox
%               \fi
%             \else
%               \usebox0
%             \fi
%           \else
%             \lwbox
%           \fi
%         \else
%           \usebox0
%         \fi
%       \else
%         \lwbox
%       \fi
%     \else
%       \usebox0
%     \fi
%   \else
%     \lwbox
%   \fi
% \else
%   \usebox0
% \fi
% \end{quote}
% If you have a \xfile{docstrip.cfg} that configures and enables \docstrip's
% TDS installing feature, then some files can already be in the right
% place, see the documentation of \docstrip.
%
% \subsection{Refresh file name databases}
%
% If your \TeX~distribution
% (\teTeX, \mikTeX, \dots) relies on file name databases, you must refresh
% these. For example, \teTeX\ users run \verb|texhash| or
% \verb|mktexlsr|.
%
% \subsection{Some details for the interested}
%
% \paragraph{Attached source.}
%
% The PDF documentation on CTAN also includes the
% \xfile{.dtx} source file. It can be extracted by
% AcrobatReader 6 or higher. Another option is \textsf{pdftk},
% e.g. unpack the file into the current directory:
% \begin{quote}
%   \verb|pdftk pdfescape.pdf unpack_files output .|
% \end{quote}
%
% \paragraph{Unpacking with \LaTeX.}
% The \xfile{.dtx} chooses its action depending on the format:
% \begin{description}
% \item[\plainTeX:] Run \docstrip\ and extract the files.
% \item[\LaTeX:] Generate the documentation.
% \end{description}
% If you insist on using \LaTeX\ for \docstrip\ (really,
% \docstrip\ does not need \LaTeX), then inform the autodetect routine
% about your intention:
% \begin{quote}
%   \verb|latex \let\install=y% \iffalse meta-comment
%
% File: pdfescape.dtx
% Version: 2011/11/25 v1.13
% Info: Implements pdfTeX's escape features
%
% Copyright (C) 2007, 2010, 2011 by
%    Heiko Oberdiek <heiko.oberdiek at googlemail.com>
%
% This work may be distributed and/or modified under the
% conditions of the LaTeX Project Public License, either
% version 1.3c of this license or (at your option) any later
% version. This version of this license is in
%    http://www.latex-project.org/lppl/lppl-1-3c.txt
% and the latest version of this license is in
%    http://www.latex-project.org/lppl.txt
% and version 1.3 or later is part of all distributions of
% LaTeX version 2005/12/01 or later.
%
% This work has the LPPL maintenance status "maintained".
%
% This Current Maintainer of this work is Heiko Oberdiek.
%
% The Base Interpreter refers to any `TeX-Format',
% because some files are installed in TDS:tex/generic//.
%
% This work consists of the main source file pdfescape.dtx
% and the derived files
%    pdfescape.sty, pdfescape.pdf, pdfescape.ins, pdfescape.drv,
%    pdfescape-test1.tex, pdfescape-test2.tex,
%    pdfescape-test3.tex, pdfescape-test4.tex,
%    pdfescape-test5.tex, pdfescape-test6.tex.
%
% Distribution:
%    CTAN:macros/latex/contrib/oberdiek/pdfescape.dtx
%    CTAN:macros/latex/contrib/oberdiek/pdfescape.pdf
%
% Unpacking:
%    (a) If pdfescape.ins is present:
%           tex pdfescape.ins
%    (b) Without pdfescape.ins:
%           tex pdfescape.dtx
%    (c) If you insist on using LaTeX
%           latex \let\install=y\input{pdfescape.dtx}
%        (quote the arguments according to the demands of your shell)
%
% Documentation:
%    (a) If pdfescape.drv is present:
%           latex pdfescape.drv
%    (b) Without pdfescape.drv:
%           latex pdfescape.dtx; ...
%    The class ltxdoc loads the configuration file ltxdoc.cfg
%    if available. Here you can specify further options, e.g.
%    use A4 as paper format:
%       \PassOptionsToClass{a4paper}{article}
%
%    Programm calls to get the documentation (example):
%       pdflatex pdfescape.dtx
%       makeindex -s gind.ist pdfescape.idx
%       pdflatex pdfescape.dtx
%       makeindex -s gind.ist pdfescape.idx
%       pdflatex pdfescape.dtx
%
% Installation:
%    TDS:tex/generic/oberdiek/pdfescape.sty
%    TDS:doc/latex/oberdiek/pdfescape.pdf
%    TDS:doc/latex/oberdiek/test/pdfescape-test1.tex
%    TDS:doc/latex/oberdiek/test/pdfescape-test2.tex
%    TDS:doc/latex/oberdiek/test/pdfescape-test3.tex
%    TDS:doc/latex/oberdiek/test/pdfescape-test4.tex
%    TDS:doc/latex/oberdiek/test/pdfescape-test5.tex
%    TDS:doc/latex/oberdiek/test/pdfescape-test6.tex
%    TDS:source/latex/oberdiek/pdfescape.dtx
%
%<*ignore>
\begingroup
  \catcode123=1 %
  \catcode125=2 %
  \def\x{LaTeX2e}%
\expandafter\endgroup
\ifcase 0\ifx\install y1\fi\expandafter
         \ifx\csname processbatchFile\endcsname\relax\else1\fi
         \ifx\fmtname\x\else 1\fi\relax
\else\csname fi\endcsname
%</ignore>
%<*install>
\input docstrip.tex
\Msg{************************************************************************}
\Msg{* Installation}
\Msg{* Package: pdfescape 2011/11/25 v1.13 Implements pdfTeX's escape features (HO)}
\Msg{************************************************************************}

\keepsilent
\askforoverwritefalse

\let\MetaPrefix\relax
\preamble

This is a generated file.

Project: pdfescape
Version: 2011/11/25 v1.13

Copyright (C) 2007, 2010, 2011 by
   Heiko Oberdiek <heiko.oberdiek at googlemail.com>

This work may be distributed and/or modified under the
conditions of the LaTeX Project Public License, either
version 1.3c of this license or (at your option) any later
version. This version of this license is in
   http://www.latex-project.org/lppl/lppl-1-3c.txt
and the latest version of this license is in
   http://www.latex-project.org/lppl.txt
and version 1.3 or later is part of all distributions of
LaTeX version 2005/12/01 or later.

This work has the LPPL maintenance status "maintained".

This Current Maintainer of this work is Heiko Oberdiek.

The Base Interpreter refers to any `TeX-Format',
because some files are installed in TDS:tex/generic//.

This work consists of the main source file pdfescape.dtx
and the derived files
   pdfescape.sty, pdfescape.pdf, pdfescape.ins, pdfescape.drv,
   pdfescape-test1.tex, pdfescape-test2.tex,
   pdfescape-test3.tex, pdfescape-test4.tex,
   pdfescape-test5.tex, pdfescape-test6.tex.

\endpreamble
\let\MetaPrefix\DoubleperCent

\generate{%
  \file{pdfescape.ins}{\from{pdfescape.dtx}{install}}%
  \file{pdfescape.drv}{\from{pdfescape.dtx}{driver}}%
  \usedir{tex/generic/oberdiek}%
  \file{pdfescape.sty}{\from{pdfescape.dtx}{package}}%
  \usedir{doc/latex/oberdiek/test}%
  \file{pdfescape-test1.tex}{\from{pdfescape.dtx}{test1}}%
  \file{pdfescape-test2.tex}{\from{pdfescape.dtx}{test2}}%
  \file{pdfescape-test3.tex}{\from{pdfescape.dtx}{test3}}%
  \file{pdfescape-test4.tex}{\from{pdfescape.dtx}{test4}}%
  \file{pdfescape-test5.tex}{\from{pdfescape.dtx}{test5}}%
  \file{pdfescape-test6.tex}{\from{pdfescape.dtx}{test6}}%
  \nopreamble
  \nopostamble
  \usedir{source/latex/oberdiek/catalogue}%
  \file{pdfescape.xml}{\from{pdfescape.dtx}{catalogue}}%
}

\catcode32=13\relax% active space
\let =\space%
\Msg{************************************************************************}
\Msg{*}
\Msg{* To finish the installation you have to move the following}
\Msg{* file into a directory searched by TeX:}
\Msg{*}
\Msg{*     pdfescape.sty}
\Msg{*}
\Msg{* To produce the documentation run the file `pdfescape.drv'}
\Msg{* through LaTeX.}
\Msg{*}
\Msg{* Happy TeXing!}
\Msg{*}
\Msg{************************************************************************}

\endbatchfile
%</install>
%<*ignore>
\fi
%</ignore>
%<*driver>
\NeedsTeXFormat{LaTeX2e}
\ProvidesFile{pdfescape.drv}%
  [2011/11/25 v1.13 Implements pdfTeX's escape features (HO)]%
\documentclass{ltxdoc}
\usepackage{holtxdoc}[2011/11/22]
\begin{document}
  \DocInput{pdfescape.dtx}%
\end{document}
%</driver>
% \fi
%
% \CheckSum{2516}
%
% \CharacterTable
%  {Upper-case    \A\B\C\D\E\F\G\H\I\J\K\L\M\N\O\P\Q\R\S\T\U\V\W\X\Y\Z
%   Lower-case    \a\b\c\d\e\f\g\h\i\j\k\l\m\n\o\p\q\r\s\t\u\v\w\x\y\z
%   Digits        \0\1\2\3\4\5\6\7\8\9
%   Exclamation   \!     Double quote  \"     Hash (number) \#
%   Dollar        \$     Percent       \%     Ampersand     \&
%   Acute accent  \'     Left paren    \(     Right paren   \)
%   Asterisk      \*     Plus          \+     Comma         \,
%   Minus         \-     Point         \.     Solidus       \/
%   Colon         \:     Semicolon     \;     Less than     \<
%   Equals        \=     Greater than  \>     Question mark \?
%   Commercial at \@     Left bracket  \[     Backslash     \\
%   Right bracket \]     Circumflex    \^     Underscore    \_
%   Grave accent  \`     Left brace    \{     Vertical bar  \|
%   Right brace   \}     Tilde         \~}
%
% \GetFileInfo{pdfescape.drv}
%
% \title{The \xpackage{pdfescape} package}
% \date{2011/11/25 v1.13}
% \author{Heiko Oberdiek\\\xemail{heiko.oberdiek at googlemail.com}}
%
% \maketitle
%
% \begin{abstract}
% This package implements \pdfTeX's escape features
% (\cs{pdfescapehex}, \cs{pdfunescapehex}, \cs{pdfescapename},
% \cs{pdfescapestring}) using \TeX\ or \eTeX.
% \end{abstract}
%
% \tableofcontents
%
% \section{Documentation}
%
% \begin{declcs}{EdefEscapeHex} \M{cmd} \M{string}\\
%  \cs{EdefUnescapeHex} \M{cmd} \M{string}\\
%  \cs{EdefEscapeName} \M{cmd} \M{string}\\
%  \cs{EdefEscapeString} \M{cmd} \M{string}
% \end{declcs}
% These commands converts \meta{string} and stores the result
% in macro \meta{cmd}. The conversion result is the same
% as the conversion of the corresponding \pdfTeX's primitives.
% Note that the argument \meta{string} is expanded before
% the conversion.
%
% For example, if \pdfTeX\ \textgreater= 1.30 is present,
% then \cs{EdefEscapeHex}
% becomes to:
% \begin{quote}
%\begin{verbatim}
%\def\EdefEscapeHex#1#2{%
%  \edef#1{\pdfescapehex{#2}}%
%}
%\end{verbatim}
% \end{quote}
% The package provides implementations for the case that \pdfTeX\ is
% not present (or too old). Even \eTeX\ can be missing, however it is
% used if it is detected.
%
% \paragraph{Babel.}
% The input strings may contain shorthand characters of package
% \xpackage{babel}.
%
% \subsection{Additional unescape macros}
%
% \begin{declcs}{EdefUnescapeName} \M{cmd} \M{string}
% \end{declcs}
% Sequences of a hash sign with two hexadecimal digits are
% converted to the corresponding character (PDF-1.2).
% A hash sign that is not followed by two hexadecimal digits
% is left unchanged. The catcodes in the result string follow
% TeX's conventions. The space has catcode 10 (space) and
% the other characters have catcode 12 (other).
%
% \begin{declcs}{EdefUnescapeString} \M{cmd} \M{string}
% \end{declcs}
% Macro \meta{cmd} stores the unescaped string in \meta{string}.
% All the rules for literal strings are implemented, see
% PDF specification.
% The catcodes in the result string follow TeX's conventions.
%
% \subsection{Sanitizing macro}
%
% \begin{declcs}{EdefSanitize} \M{cmd} \M{string}
% \end{declcs}
% Argument \meta{string} is expanded, converted to a string
% of tokens with catcode 12 (other) and space tokens, and
% stored in macro \meta{cmd}.
%
% \StopEventually{
% }
%
% \section{Implementation}
%
%    \begin{macrocode}
%<*package>
%    \end{macrocode}
%
% \subsection{Reload check and package identification}
%    Reload check, especially if the package is not used with \LaTeX.
%    \begin{macrocode}
\begingroup\catcode61\catcode48\catcode32=10\relax%
  \catcode13=5 % ^^M
  \endlinechar=13 %
  \catcode35=6 % #
  \catcode39=12 % '
  \catcode44=12 % ,
  \catcode45=12 % -
  \catcode46=12 % .
  \catcode58=12 % :
  \catcode64=11 % @
  \catcode123=1 % {
  \catcode125=2 % }
  \expandafter\let\expandafter\x\csname ver@pdfescape.sty\endcsname
  \ifx\x\relax % plain-TeX, first loading
  \else
    \def\empty{}%
    \ifx\x\empty % LaTeX, first loading,
      % variable is initialized, but \ProvidesPackage not yet seen
    \else
      \expandafter\ifx\csname PackageInfo\endcsname\relax
        \def\x#1#2{%
          \immediate\write-1{Package #1 Info: #2.}%
        }%
      \else
        \def\x#1#2{\PackageInfo{#1}{#2, stopped}}%
      \fi
      \x{pdfescape}{The package is already loaded}%
      \aftergroup\endinput
    \fi
  \fi
\endgroup%
%    \end{macrocode}
%    Package identification:
%    \begin{macrocode}
\begingroup\catcode61\catcode48\catcode32=10\relax%
  \catcode13=5 % ^^M
  \endlinechar=13 %
  \catcode35=6 % #
  \catcode39=12 % '
  \catcode40=12 % (
  \catcode41=12 % )
  \catcode44=12 % ,
  \catcode45=12 % -
  \catcode46=12 % .
  \catcode47=12 % /
  \catcode58=12 % :
  \catcode64=11 % @
  \catcode91=12 % [
  \catcode93=12 % ]
  \catcode123=1 % {
  \catcode125=2 % }
  \expandafter\ifx\csname ProvidesPackage\endcsname\relax
    \def\x#1#2#3[#4]{\endgroup
      \immediate\write-1{Package: #3 #4}%
      \xdef#1{#4}%
    }%
  \else
    \def\x#1#2[#3]{\endgroup
      #2[{#3}]%
      \ifx#1\@undefined
        \xdef#1{#3}%
      \fi
      \ifx#1\relax
        \xdef#1{#3}%
      \fi
    }%
  \fi
\expandafter\x\csname ver@pdfescape.sty\endcsname
\ProvidesPackage{pdfescape}%
  [2011/11/25 v1.13 Implements pdfTeX's escape features (HO)]%
%    \end{macrocode}
%
% \subsection{Catcodes}
%
%    \begin{macrocode}
\begingroup\catcode61\catcode48\catcode32=10\relax%
  \catcode13=5 % ^^M
  \endlinechar=13 %
  \catcode123 1 % {
  \catcode125 2 % }
  \catcode64 11 %
  \def\x{\endgroup
    \expandafter\edef\csname PE@AtEnd\endcsname{%
      \endlinechar=\the\endlinechar\relax
      \catcode13=\the\catcode13\relax
      \catcode32=\the\catcode32\relax
      \catcode35=\the\catcode35\relax
      \catcode61=\the\catcode61\relax
      \catcode64=\the\catcode64\relax
      \catcode123=\the\catcode123\relax
      \catcode125=\the\catcode125\relax
    }%
  }%
\x\catcode61\catcode48\catcode32=10\relax%
\catcode13=5 % ^^M
\endlinechar=13 %
\catcode35=6 % #
\catcode64=11 % @
\catcode123=1 % {
\catcode125=2 % }
\def\TMP@EnsureCode#1#2#3{%
  \edef\PE@AtEnd{%
    \PE@AtEnd
    #1#2=\the#1#2\relax
  }%
  #1#2=#3\relax
}
\TMP@EnsureCode\catcode{0}{12}% ^^@
\TMP@EnsureCode\catcode{34}{12}% "
\TMP@EnsureCode\catcode{36}{3}% $
\TMP@EnsureCode\catcode{38}{4}% &
\TMP@EnsureCode\catcode{39}{12}% '
\TMP@EnsureCode\catcode{42}{12}% *
\TMP@EnsureCode\catcode{45}{12}% -
\TMP@EnsureCode\catcode{46}{12}% .
\TMP@EnsureCode\catcode{47}{12}% /
\TMP@EnsureCode\catcode{60}{12}% <
\TMP@EnsureCode\catcode{62}{12}% >
\TMP@EnsureCode\catcode{91}{12}% [
\TMP@EnsureCode\catcode{93}{12}% ]
\TMP@EnsureCode\catcode{94}{7}% ^
\TMP@EnsureCode\catcode{96}{12}% `
\TMP@EnsureCode\uccode{34}{0}% "
\TMP@EnsureCode\uccode{48}{0}% 0
\TMP@EnsureCode\uccode{61}{0}% =
\edef\PE@AtEnd{\PE@AtEnd\noexpand\endinput}
%    \end{macrocode}
%
% \subsection{Load package}
%
%    \begin{macrocode}
\begingroup\expandafter\expandafter\expandafter\endgroup
\expandafter\ifx\csname RequirePackage\endcsname\relax
  \def\TMP@RequirePackage#1[#2]{%
    \begingroup\expandafter\expandafter\expandafter\endgroup
    \expandafter\ifx\csname ver@#1.sty\endcsname\relax
      \input #1.sty\relax
    \fi
  }%
  \TMP@RequirePackage{ltxcmds}[2010/04/08]%
\else
  \RequirePackage{ltxcmds}[2010/04/08]%
\fi
%    \end{macrocode}
%
% \subsection{Sanitizing}
%
%    \begin{macro}{\EdefSanitize}
%    Macro \cs{EdefSanitize} takes |#2|, entirely converts
%    it to token with catcode 12 (other) and stores the
%    result in macro |#1|.
%    \begin{macrocode}
\begingroup\expandafter\expandafter\expandafter\endgroup
\expandafter\ifx\csname detokenize\endcsname\relax
  \long\def\EdefSanitize#1#2{%
    \begingroup
      \csname @safe@activestrue\endcsname
      \edef#1{#2}%
      \PE@onelevel@sanitize#1%
    \expandafter\endgroup
    \expandafter\def\expandafter#1\expandafter{#1}%
  }%
  \begingroup\expandafter\expandafter\expandafter\endgroup
  \expandafter\ifx\csname @onelevel@sanitize\endcsname\relax
    \def\PE@onelevel@sanitize#1{%
      \edef#1{\expandafter\PE@strip@prefix\meaning#1}%
    }%
    \def\PE@strip@prefix#1>{}%
  \else
    \let\PE@onelevel@sanitize\@onelevel@sanitize
  \fi
\else
  \long\def\EdefSanitize#1#2{%
    \begingroup
      \csname @safe@activestrue\endcsname
      \edef#1{#2}%
    \expandafter\endgroup
    \expandafter\def\expandafter#1\expandafter{%
      \detokenize\expandafter{#1}%
    }%
  }%
  \def\PE@onelevel@sanitize#1{%
    \edef#1{\detokenize\expandafter{#1}}%
  }%
\fi
%    \end{macrocode}
%    \end{macro}
%    \begin{macro}{\PE@sanitize}
%    Macro \cs{PE@sanitize} is only defined for
%    compatibility with version 1.4. Its use is deprecated.
%    \begin{macrocode}
\let\PE@sanitize\EdefSanitize
%    \end{macrocode}
%    \end{macro}
%
% \subsubsection{Space characters}
%
%    \begin{macro}{\PE@space@other}
%    \begin{macrocode}
\begingroup
  \catcode`\ =12\relax%
\def\x{\endgroup\def\PE@space@other{ }}\x\relax
%    \end{macrocode}
%    \end{macro}
%    \begin{macro}{\PE@space@space}
%    \begin{macrocode}
\def\PE@space@space{ }
%    \end{macrocode}
%    \end{macro}
%
% \subsubsection{Space normalization}
%
%    \begin{macro}{\PE@SanitizeSpaceOther}
%    \begin{macrocode}
\def\PE@SanitizeSpaceOther#1{%
  \edef#1{\expandafter\PE@SpaceToOther#1 \relax}%
}
%    \end{macrocode}
%    \end{macro}
%
%    \begin{macro}{\PE@SpaceToOther}
%    \begin{macrocode}
\def\PE@SpaceToOther#1 #2\relax{%
  #1%
  \ifx\\#2\\%
  \else
    \PE@space@other
    \ltx@ReturnAfterFi{%
      \PE@SpaceToOther#2\relax
    }%
  \fi
}
%    \end{macrocode}
%    \end{macro}
%
% \subsection{\cs{EdefUnescapeName}}
%
%    \begin{macro}{\EdefUnescapeName}
%    \begin{macrocode}
\def\EdefUnescapeName#1#2{%
  \EdefSanitize#1{#2}%
  \PE@SanitizeSpaceOther#1%
  \PE@UnescapeName#1%
  \PE@onelevel@sanitize#1%
}
%    \end{macrocode}
%    \end{macro}
%
%    \begin{macro}{\PE@UnescapeName}
%    \begin{macrocode}
\begingroup
  \catcode`\$=6  % hash
  \catcode`\#=12 % other
  \gdef\PE@UnescapeName$1{%
    \begingroup
      \PE@InitUccodeHexDigit
      \def\PE@result{}%
      \expandafter\PE@DeName$1#\relax\relax
    \expandafter\endgroup
    \expandafter\def\expandafter$1\expandafter{\PE@result}%
  }%
  \gdef\PE@DeName$1#$2$3{%
    \ifx\relax$2%
      \edef\PE@result{\PE@result$1}%
      \let\PE@next\relax
    \else
      \ifx\relax$3%
        % wrong escape sequence in input
        \edef\PE@result{\PE@result$1#}%
        \let\PE@next\relax
      \else
        \uppercase{%
          \def\PE@testA{$2}%
          \def\PE@testB{$3}%
        }%
        \ifcase\ifcase\expandafter\PE@TestUcHexDigit\PE@testA
                 \ifcase\expandafter\PE@TestUcHexDigit\PE@testB
                   \ltx@zero
                 \else
                   \ltx@one
                 \fi
               \else
                 \ltx@one
               \fi
          \uccode\ltx@zero="\PE@testA\PE@testB\relax
          \uppercase{%
            \def\PE@temp{^^@}%
          }%
          \uccode\ltx@zero=\ltx@zero
          \edef\PE@result{\PE@result$1\PE@temp}%
          \let\PE@next\PE@DeName
        \else
          % wrong escape sequence in input
          \edef\PE@result{\PE@result$1#}%
          \def\PE@next{\PE@DeName$2$3}%
        \fi
      \fi
    \fi
    \PE@next
  }%
\endgroup
%    \end{macrocode}
%    \end{macro}
%
%    \begin{macro}{\PE@InitUccodeHexDigit}
%    \begin{macrocode}
\def\PE@InitUccodeHexDigit{%
  \uccode`a=`A\relax
  \uccode`b=`B\relax
  \uccode`c=`C\relax
  \uccode`d=`D\relax
  \uccode`e=`E\relax
  \uccode`f=`F\relax
  \uccode`A=\ltx@zero
  \uccode`B=\ltx@zero
  \uccode`C=\ltx@zero
  \uccode`D=\ltx@zero
  \uccode`E=\ltx@zero
  \uccode`F=\ltx@zero
  \uccode`0=\ltx@zero
  \uccode`1=\ltx@zero
  \uccode`2=\ltx@zero
  \uccode`3=\ltx@zero
  \uccode`4=\ltx@zero
  \uccode`5=\ltx@zero
  \uccode`6=\ltx@zero
  \uccode`7=\ltx@zero
  \uccode`8=\ltx@zero
  \uccode`9=\ltx@zero
}
%    \end{macrocode}
%    \end{macro}
%
%    \begin{macro}{\PE@TestUcHexDigit}
%    \begin{macrocode}
\def\PE@TestUcHexDigit#1{%
  \ifnum`#1<48 % 0
    \ltx@one
  \else
    \ifnum`#1>70 % F
      \ltx@one
    \else
      \ifnum`#1>57 % 9
        \ifnum`#1<65 % A
          \ltx@one
        \else
          \ltx@zero
        \fi
      \else
        \ltx@zero
      \fi
    \fi
  \fi
}
%    \end{macrocode}
%    \end{macro}
%
% \subsection{\cs{EdefUnescapeString}}
%
%    \begin{macro}{\EdefUnescapeString}
%    \begin{macrocode}
\def\EdefUnescapeString#1#2{%
  \EdefSanitize#1{#2}%
  \PE@SanitizeSpaceOther#1%
  \PE@NormalizeLineEnd#1%
  \PE@UnescapeString#1%
  \PE@onelevel@sanitize#1%
}
%    \end{macrocode}
%    \end{macro}
%
%    \begin{macrocode}
\begingroup
  \uccode`\8=10 % lf
  \uccode`\9=13 % cr
\def\x#1#2{\endgroup
%    \end{macrocode}
%    \begin{macro}{\PE@NormalizeLineEnd}
%    \begin{macrocode}
  \def\PE@NormalizeLineEnd##1{%
    \def\PE@result{}%
    \expandafter\PE@@NormalizeLineEnd##1#2\relax
    \let##1\PE@result
  }%
%    \end{macrocode}
%    \end{macro}
%    \begin{macro}{\PE@@NormalizeLineEnd}
%    \begin{macrocode}
  \def\PE@@NormalizeLineEnd##1#2##2{%
    \ifx\relax##2%
      \edef\PE@result{\PE@result##1}%
      \let\PE@next\relax
    \else
      \edef\PE@result{\PE@result##1#1}%
      \ifx#1##2% lf
        \let\PE@next\PE@@NormalizeLineEnd
      \else
        \def\PE@next{\PE@@NormalizeLineEnd##2}%
      \fi
    \fi
    \PE@next
  }%
}%
\uppercase{%
  \x 89%
}
%    \end{macrocode}
%    \end{macro}
%
%    \begin{macrocode}
\begingroup
  \catcode`\|=0 %
  \catcode`\\=12 %
%    \end{macrocode}
%    \begin{macro}{\PE@UnescapeString}
%    \begin{macrocode}
  |gdef|PE@UnescapeString#1{%
    |begingroup
      |def|PE@result{}%
      |expandafter|PE@DeString#1\|relax
    |expandafter|endgroup
    |expandafter|def|expandafter#1|expandafter{|PE@result}%
  }%
%    \end{macrocode}
%    \end{macro}
%    \begin{macro}{\PE@DeString}
%    \begin{macrocode}
  |gdef|PE@DeString#1\#2{%
    |ifx|relax#2%
      |edef|PE@result{|PE@result#1}%
      |let|PE@next|relax
    |else
      |if n#2%
        |uccode|ltx@zero=10 %
      |else|if r#2%
        |uccode|ltx@zero=13 %
      |else|if t#2%
        |uccode|ltx@zero=9 %
      |else|if b#2%
        |uccode|ltx@zero=8 %
      |else|if f#2%
        |uccode|ltx@zero=12 %
      |else
        |uccode|ltx@zero=|ltx@zero
      |fi|fi|fi|fi|fi
      |ifnum|uccode|ltx@zero>|ltx@zero
        |uppercase{%
          |edef|PE@temp{^^@}%
        }%
        |edef|PE@result{|PE@result#1|PE@temp}%
        |let|PE@next|PE@DeString
      |else
        |if\#2% backslash
          |edef|PE@result{|PE@result#1}%
          |let|PE@next|PE@CheckEndBackslash
        |else
          |ifnum`#2=10 % linefeed
            |edef|PE@result{|PE@result#1}%
            |let|PE@next|PE@DeString
          |else
            |ifcase|PE@TestOctDigit#2%
              |edef|PE@result{|PE@result#1}%
              |def|PE@next{|PE@OctI#2}%
            |else
              |edef|PE@result{|PE@result#1#2}%
              |let|PE@next|PE@DeString
            |fi
          |fi
        |fi
      |fi
    |fi
    |PE@next
  }%
%    \end{macrocode}
%    \end{macro}
%    \begin{macro}{\PE@CheckEndBackslash}
%    \begin{macrocode}
  |gdef|PE@CheckEndBackslash#1{%
    |ifx|relax#1%
    |else
      |edef|PE@result{|PE@result\}%
      |expandafter|PE@DeString|expandafter#1%
    |fi
  }%
%    \end{macrocode}
%    \end{macro}
%    \begin{macrocode}
|endgroup
%    \end{macrocode}
%    \begin{macro}{\PE@TestOctDigit}
%    \begin{macrocode}
\def\PE@TestOctDigit#1{%
  \ifnum`#1<48 % 0
    \ltx@one
  \else
    \ifnum`#1>55 % 7
      \ltx@one
    \else
      \ltx@zero
    \fi
  \fi
}
%    \end{macrocode}
%    \end{macro}
%    \begin{macro}{\PE@OctI}
%    \begin{macrocode}
\def\PE@OctI#1#2{%
  \ifcase\PE@TestOctDigit#2%
    \def\PE@next{\PE@OctII{#1#2}}%
  \else
    \def\PE@next{\PE@OctAll#1#2}%
  \fi
  \PE@next
}
%    \end{macrocode}
%    \end{macro}
%    \begin{macro}{\PE@OctII}
%    \begin{macrocode}
\def\PE@OctII#1#2{%
  \ifcase\PE@TestOctDigit#2%
    \def\PE@next{\PE@OctIII#1#2}%
  \else
    \def\PE@next{\PE@OctAll{#1}#2}%
  \fi
  \PE@next
}
%    \end{macrocode}
%    \end{macro}
%    \begin{macrocode}
\ltx@IfUndefined{numexpr}{%
  \catcode`\$=9 %
  \catcode`\&=14 %
}{%
  \catcode`\$=14 %
  \catcode`\&=9 %
}
%    \end{macrocode}
%    \begin{macro}{\PE@OctIII}
%    \begin{macrocode}
\def\PE@OctIII#1#2#3{%
  \ifnum#1<4 %
    \def\PE@next{\PE@OctAll{#1#2#3}}%
  \else
$   \count\ltx@cclv#1 %
$   \advance\count\ltx@cclv -4 %
    \edef\PE@next{%
      \noexpand\PE@OctAll{%
$       \the\count\ltx@cclv
&       \the\numexpr#1-4\relax
        #2#3%
      }%
    }%
  \fi
  \PE@next
}
%    \end{macrocode}
%    \end{macro}
%    \begin{macro}{\PE@OctAll}
%    \begin{macrocode}
\def\PE@OctAll#1{%
  \uccode\ltx@zero='#1\relax
  \uppercase{%
    \edef\PE@result{\PE@result^^@}%
  }%
  \PE@DeString
}
%    \end{macrocode}
%    \end{macro}
%
% \subsection{User macros (\pdfTeX\space analogues)}
%
%    \begin{macrocode}
\begingroup\expandafter\expandafter\expandafter\endgroup
\expandafter\ifx\csname RequirePackage\endcsname\relax
  \def\TMP@RequirePackage#1[#2]{%
    \begingroup\expandafter\expandafter\expandafter\endgroup
    \expandafter\ifx\csname ver@#1.sty\endcsname\relax
      \input #1.sty\relax
    \fi
  }%
  \TMP@RequirePackage{pdftexcmds}[2007/11/11]%
\else
  \RequirePackage{pdftexcmds}[2007/11/11]%
\fi
%    \end{macrocode}
%
%    \begin{macrocode}
\begingroup\expandafter\expandafter\expandafter\endgroup
\expandafter\ifx\csname pdf@escapehex\endcsname\relax
%    \end{macrocode}
%
%    \begin{macro}{\EdefEscapeHex}
%    \begin{macrocode}
  \long\def\EdefEscapeHex#1#2{%
    \EdefSanitize#1{#2}%
    \PE@SanitizeSpaceOther#1%
    \PE@EscapeHex#1%
  }%
%    \end{macrocode}
%    \end{macro}
%    \begin{macro}{\EdefUnescapeHex}
%    \begin{macrocode}
  \def\EdefUnescapeHex#1#2{%
    \EdefSanitize#1{#2}%
    \PE@UnescapeHex#1%
  }%
%    \end{macrocode}
%    \end{macro}
%    \begin{macro}{\EdefEscapeName}
%    \begin{macrocode}
  \long\def\EdefEscapeName#1#2{%
    \EdefSanitize#1{#2}%
    \PE@SanitizeSpaceOther#1%
    \PE@EscapeName#1%
  }%
%    \end{macrocode}
%    \end{macro}
%    \begin{macro}{\EdefEscapeString}
%    \begin{macrocode}
  \long\def\EdefEscapeString#1#2{%
    \EdefSanitize#1{#2}%
    \PE@SanitizeSpaceOther#1%
    \PE@EscapeString#1%
  }%
%    \end{macrocode}
%    \end{macro}
%
%    \begin{macrocode}
\else
%    \end{macrocode}
%
%    \begin{macro}{\PE@edefbabel}
%    Help macro that adds support for \xpackage{babel}'s shorthand
%    characters.
%    \begin{macrocode}
  \long\def\PE@edefbabel#1#2#3{%
    \begingroup
      \csname @save@activestrue\endcsname
      \edef#1{#2{#3}}%
    \expandafter\endgroup
    \expandafter\def\expandafter#1\expandafter{#1}%
  }%
%    \end{macrocode}
%    \end{macro}
%
%    \begin{macro}{\EdefEscapeHex}
%    \begin{macrocode}
  \long\def\EdefEscapeHex#1#2{%
    \PE@edefbabel#1\pdf@escapehex{#2}%
  }%
%    \end{macrocode}
%    \end{macro}
%    \begin{macro}{\EdefUnescapeHex}
%    \begin{macrocode}
  \def\EdefUnescapeHex#1#2{%
    \PE@edefbabel#1\pdf@unescapehex{#2}%
  }%
%    \end{macrocode}
%    \end{macro}
%    \begin{macro}{\EdefEscapeName}
%    \begin{macrocode}
  \long\def\EdefEscapeName#1#2{%
    \PE@edefbabel#1\pdf@escapename{#2}%
  }%
%    \end{macrocode}
%    \end{macro}
%    \begin{macro}{\EdefEscapeString}
%    \begin{macrocode}
  \long\def\EdefEscapeString#1#2{%
    \PE@edefbabel#1\pdf@escapestring{#2}%
  }%
%    \end{macrocode}
%    \end{macro}
%
%    \begin{macrocode}
  \expandafter\PE@AtEnd
\fi%
%    \end{macrocode}
%
% \subsection{Help macros}
%
% \subsubsection{Characters}
%
%    Special characters with catcode 12 (other) are created and
%    stored in macros.
%    \begin{macro}{\PE@hash}
%    \begin{macrocode}
\edef\PE@hash{\string#}
%    \end{macrocode}
%    \end{macro}
%    \begin{macro}{\PE@backslash}
%    \begin{macrocode}
\begingroup
  \escapechar=-1 %
\edef\x{\endgroup
  \def\noexpand\PE@backslash{\string\\}%
}
\x
%    \end{macrocode}
%    \end{macro}
%
% \subsubsection{Switch for \eTeX}
%
%    \begin{macrocode}
\ltx@newif\ifPE@etex
\begingroup\expandafter\expandafter\expandafter\endgroup
\expandafter\ifx\csname numexpr\endcsname\relax
\else
  \PE@etextrue
\fi
%    \end{macrocode}
%
% \subsection{Conversions}
%
% \subsubsection{Conversion to hex string}
%
%    \begin{macro}{\PE@EscapeHex}
%    \begin{macrocode}
\ifPE@etex
  \def\PE@EscapeHex#1{%
    \edef#1{\expandafter\PE@ToHex#1\relax}%
  }%
\else
  \def\PE@EscapeHex#1{%
    \def\PE@result{}%
    \expandafter\PE@ToHex#1\relax
    \let#1\PE@result
  }%
\fi
%    \end{macrocode}
%    \end{macro}
%    \begin{macro}{\PE@ToHex}
%    \begin{macrocode}
\def\PE@ToHex#1{%
  \ifx\relax#1%
  \else
    \PE@HexChar{#1}%
    \expandafter\PE@ToHex
  \fi
}%
%    \end{macrocode}
%    \end{macro}
%    \begin{macro}{\PE@HexChar}
%    \begin{macrocode}
\ifPE@etex
  \def\PE@HexChar#1{%
    \PE@HexDigit{\numexpr\dimexpr.0625\dimexpr`#1sp\relax\relax\relax}%
    \PE@HexDigit{%
      \numexpr`#1-16*\dimexpr.0625\dimexpr`#1sp\relax\relax\relax
    }%
  }%
\else
  \def\PE@HexChar#1{%
    \dimen0=`#1sp%
    \dimen2=.0625\dimen0 %
    \advance\dimen0-16\dimen2 %
    \edef\PE@result{%
      \PE@result
      \PE@HexDigit{\dimen2 }%
      \PE@HexDigit{\dimen0 }%
    }%
  }%
\fi
%    \end{macrocode}
%    \end{macro}
%    \begin{macro}{\PE@HexDigit}
%    \begin{macrocode}
\def\PE@HexDigit#1{%
  \expandafter\string
  \ifcase#1%
    0\or 1\or 2\or 3\or 4\or 5\or 6\or 7\or 8\or 9\or
    A\or B\or C\or D\or E\or F%
  \fi
}
%    \end{macrocode}
%    \end{macro}
%
% \subsubsection{Character code to octal number}
%
%    \begin{macro}{\PE@OctChar}
%    \begin{macrocode}
\ifPE@etex
  \def\PE@OctChar#1{%
    \expandafter\PE@@OctChar
        \the\numexpr\dimexpr.015625\dimexpr`#1sp\relax\relax
            \expandafter\relax
        \expandafter\relax
        \the\numexpr\dimexpr.125\dimexpr`#1sp\relax\relax\relax
        \relax
        #1%
  }%
  \def\PE@@OctChar#1\relax#2\relax#3{%
    \PE@backslash
    #1%
    \the\numexpr#2-8*#1\relax
    \the\numexpr\dimexpr`#3sp\relax-8*#2\relax
  }%
\else
  \def\PE@OctChar#1{%
    \dimen0=`#1sp%
    \dimen2=.125\dimen0 %
    \dimen4=.125\dimen2 %
    \advance\dimen0-8\dimen2 %
    \advance\dimen2-8\dimen4 %
    \edef\PE@result{%
      \PE@result
      \PE@backslash
      \number\dimen4 %
      \number\dimen2 %
      \number\dimen0 %
    }%
  }%
\fi
%    \end{macrocode}
%    \end{macro}
%
% \subsubsection{Unpack hex string}
%
%    \begin{macro}{\PE@UnescapeHex}
%    \begin{macrocode}
\def\PE@UnescapeHex#1{%
  \begingroup
    \PE@InitUccodeHexDigit
    \def\PE@result{}%
    \expandafter\PE@DeHex#1\relax\relax
  \expandafter\endgroup
  \expandafter\def\expandafter#1\expandafter{\PE@result}%
}
%    \end{macrocode}
%    \end{macro}
%    \begin{macro}{\PE@DeHex}
%    \begin{macrocode}
\def\PE@DeHex#1#2{%
  \ifx#2\relax
    \ifx#1\relax
      \let\PE@next\relax
    \else
      \uppercase{%
        \def\PE@testA{#1}%
      }%
      \ifcase\expandafter\PE@TestUcHexDigit\PE@testA
        \def\PE@next{%
          \PE@DeHex#10\relax\relax
        }%
      \else
        \let\PE@next\relax
      \fi
    \fi
  \else
    \uppercase{%
      \def\PE@testA{#1}%
      \def\PE@testB{#2}%
    }%
    \ifcase\expandafter\PE@TestUcHexDigit\PE@testA
      \ifcase\expandafter\PE@TestUcHexDigit\PE@testB
        \uccode\ltx@zero="\PE@testA\PE@testB\relax
        \ifnum\uccode\ltx@zero=32 %
          \let\PE@temp\PE@space@space
        \else
          \uppercase{%
            \def\PE@temp{^^@}%
          }%
        \fi
        \edef\PE@result{\PE@result\PE@temp}%
        \let\PE@next\PE@DeHex
      \else
        % invalid input sequence
        \def\PE@next{%
          \PE@DeHex#1%
        }%
      \fi
    \else
      % invalid input sequence
      \def\PE@next{\PE@DeHex#2}%
    \fi
  \fi
  \PE@next
}
%    \end{macrocode}
%    \end{macro}
%
% \subsubsection{Conversion to PDF name}
%
%    \begin{macro}{\PE@EscapeName}
%    \begin{macrocode}
\ifPE@etex
  \def\PE@EscapeName#1{%
    \edef#1{\expandafter\PE@EscapeNameTokens#1\relax}%
  }%
\else
  \def\PE@EscapeName#1{%
    \def\PE@result{}%
    \expandafter\PE@EscapeNameTokens#1\relax
    \let#1\PE@result
  }%
\fi
%    \end{macrocode}
%    \end{macro}
%    \begin{macro}{\PE@EscapeNameTokens}
%    \begin{macrocode}
\def\PE@EscapeNameTokens#1{%
  \ifx\relax#1%
  \else
    \ifnum`#1<33 %
      \ifcase`#1 %
        % drop illegal zero
      \else
        \PE@EscapeNameAdd\PE@hash
        \PE@HexChar#1%
      \fi
    \else
      \ifnum`#1>126 %
        \PE@EscapeNameAdd\PE@hash
        \PE@HexChar#1%
      \else  \ifnum`#1=35  \PE@EscapeNameHashChar 23% #
        \else\ifnum`#1=37  \PE@EscapeNameHashChar 25% %
        \else\ifnum`#1=40  \PE@EscapeNameHashChar 28% (
        \else\ifnum`#1=41  \PE@EscapeNameHashChar 29% )
        \else\ifnum`#1=47  \PE@EscapeNameHashChar 2F% /
        \else\ifnum`#1=60  \PE@EscapeNameHashChar 3C% <
        \else\ifnum`#1=62  \PE@EscapeNameHashChar 3E% >
        \else\ifnum`#1=91  \PE@EscapeNameHashChar 5B% [
        \else\ifnum`#1=93  \PE@EscapeNameHashChar 5D% ]
        \else\ifnum`#1=123 \PE@EscapeNameHashChar 7B% {
        \else\ifnum`#1=125 \PE@EscapeNameHashChar 7D% }
        \else
          \PE@EscapeNameAdd{#1}%
        \fi\fi\fi\fi\fi\fi\fi\fi\fi\fi\fi
      \fi
    \fi
    \expandafter\PE@EscapeNameTokens
  \fi
}%
\def\PE@EscapeNameHashChar#1#2{%
  \PE@EscapeNameAdd{\PE@hash\string#1\string#2}%
}%
%    \end{macrocode}
%    \end{macro}
%    \begin{macro}{\PE@EscapeNameAdd}
%    \begin{macrocode}
\ifPE@etex
  \def\PE@EscapeNameAdd#1{#1}%
\else
  \def\PE@EscapeNameAdd#1{%
    \edef\PE@result{%
      \PE@result
      #1%
    }%
  }%
\fi
%    \end{macrocode}
%    \end{macro}
%
% \subsubsection{Conversion to PDF string}
%
%    \begin{macro}{\PE@EscapeString}
%    \begin{macrocode}
\ifPE@etex
  \def\PE@EscapeString#1{%
    \edef#1{\expandafter\PE@EscapeStringTokens#1\relax}%
  }%
\else
  \def\PE@EscapeString#1{%
    \begingroup
      \def\PE@result{}%
      \expandafter\PE@EscapeStringTokens#1\relax
    \expandafter\endgroup
    \expandafter\def\expandafter#1\expandafter{\PE@result}%
  }%
\fi
%    \end{macrocode}
%    \end{macro}
%    \begin{macro}{\PE@EscapeStringTokens}
%    \begin{macrocode}
\def\PE@EscapeStringTokens#1{%
  \ifx\relax#1%
  \else
    \ifnum`#1<33 %
      \PE@OctChar#1%
    \else
      \ifnum`#1>126 %
        \PE@OctChar#1%
      \else  \ifnum`#1=40 \PE@EscapeStringAdd{\string\(}% (
        \else\ifnum`#1=41 \PE@EscapeStringAdd{\string\)}% )
        \else\ifnum`#1=92 \PE@EscapeStringAdd{\string\\}% \
        \else
          \PE@EscapeStringAdd{#1}%
        \fi\fi\fi
      \fi
    \fi
    \expandafter\PE@EscapeStringTokens
  \fi
}%
%    \end{macrocode}
%    \end{macro}
%    \begin{macro}{\PE@EscapeStringAdd}
%    \begin{macrocode}
\ifPE@etex
  \def\PE@EscapeStringAdd#1{#1}%
\else
  \def\PE@EscapeStringAdd#1{%
    \edef\PE@result{%
      \PE@result
      #1%
    }%
  }%
\fi
%    \end{macrocode}
%    \end{macro}
%
%    \begin{macrocode}
\PE@AtEnd%
%    \end{macrocode}
%
%    \begin{macrocode}
%</package>
%    \end{macrocode}
%
% \section{Test}
%
% \subsection{Catcode checks for loading}
%
%    \begin{macrocode}
%<*test1>
%    \end{macrocode}
%    \begin{macrocode}
\catcode`\{=1 %
\catcode`\}=2 %
\catcode`\#=6 %
\catcode`\@=11 %
\expandafter\ifx\csname count@\endcsname\relax
  \countdef\count@=255 %
\fi
\expandafter\ifx\csname @gobble\endcsname\relax
  \long\def\@gobble#1{}%
\fi
\expandafter\ifx\csname @firstofone\endcsname\relax
  \long\def\@firstofone#1{#1}%
\fi
\expandafter\ifx\csname loop\endcsname\relax
  \expandafter\@firstofone
\else
  \expandafter\@gobble
\fi
{%
  \def\loop#1\repeat{%
    \def\body{#1}%
    \iterate
  }%
  \def\iterate{%
    \body
      \let\next\iterate
    \else
      \let\next\relax
    \fi
    \next
  }%
  \let\repeat=\fi
}%
\def\RestoreCatcodes{}
\count@=0 %
\loop
  \edef\RestoreCatcodes{%
    \RestoreCatcodes
    \catcode\the\count@=\the\catcode\count@\relax
  }%
\ifnum\count@<255 %
  \advance\count@ 1 %
\repeat

\def\RangeCatcodeInvalid#1#2{%
  \count@=#1\relax
  \loop
    \catcode\count@=15 %
  \ifnum\count@<#2\relax
    \advance\count@ 1 %
  \repeat
}
\def\RangeCatcodeCheck#1#2#3{%
  \count@=#1\relax
  \loop
    \ifnum#3=\catcode\count@
    \else
      \errmessage{%
        Character \the\count@\space
        with wrong catcode \the\catcode\count@\space
        instead of \number#3%
      }%
    \fi
  \ifnum\count@<#2\relax
    \advance\count@ 1 %
  \repeat
}
\def\space{ }
\expandafter\ifx\csname LoadCommand\endcsname\relax
  \def\LoadCommand{\input pdfescape.sty\relax}%
\fi
\def\Test{%
  \RangeCatcodeInvalid{0}{47}%
  \RangeCatcodeInvalid{58}{64}%
  \RangeCatcodeInvalid{91}{96}%
  \RangeCatcodeInvalid{123}{255}%
  \catcode`\@=12 %
  \catcode`\\=0 %
  \catcode`\%=14 %
  \LoadCommand
  \RangeCatcodeCheck{0}{36}{15}%
  \RangeCatcodeCheck{37}{37}{14}%
  \RangeCatcodeCheck{38}{47}{15}%
  \RangeCatcodeCheck{48}{57}{12}%
  \RangeCatcodeCheck{58}{63}{15}%
  \RangeCatcodeCheck{64}{64}{12}%
  \RangeCatcodeCheck{65}{90}{11}%
  \RangeCatcodeCheck{91}{91}{15}%
  \RangeCatcodeCheck{92}{92}{0}%
  \RangeCatcodeCheck{93}{96}{15}%
  \RangeCatcodeCheck{97}{122}{11}%
  \RangeCatcodeCheck{123}{255}{15}%
  \RestoreCatcodes
}
\Test
\csname @@end\endcsname
\end
%    \end{macrocode}
%    \begin{macrocode}
%</test1>
%    \end{macrocode}
%
% \subsection{Macro tests}
%
%    \begin{macrocode}
%<*test2|test3|test4|test5>
\NeedsTeXFormat{LaTeX2e}
\makeatletter
%    \end{macrocode}
%
% \subsection{Test with \cs{pdfescape...} commands}
%
%    \begin{macrocode}
%<*test2>
\ProvidesFile{pdfescape-test2.tex}%
    [2011/11/25 v1.13 Test with \string\pdfescape... commands]%
%</test2>
%    \end{macrocode}
%
% \subsection{Test without \cs{pdfescape...}, with \eTeX}
%
%    \begin{macrocode}
%<*test3>
\ProvidesFile{pdfescape-test3.tex}%
    [2011/11/25 v1.13 Test without \string\pdfescape..., with e-TeX]%
%</test3>
%    \end{macrocode}
%
% \subsection{Test without \cs{pdfescape...} and \eTeX}
%
%    \begin{macrocode}
%<*test4>
\ProvidesFile{pdfescape-test4.tex}%
    [2011/11/25 v1.13 Test without \string\pdfescape... and e-TeX]%
%</test4>
%    \end{macrocode}
%
% \subsection{Test with \LuaTeX}
%
%    \begin{macrocode}
%<*test5>
\ProvidesFile{pdfescape-test5.tex}%
    [2011/11/25 v1.13 Test with LuaTeX]%
%</test5>
%    \end{macrocode}
%
% \subsection{Check/ensure test preconditions}
%
% \subsubsection{Check \cs{pdfescape...}}
%
%    \begin{macrocode}
%<*test2>
\@ifundefined{pdfescapehex}{%
  \PackageError{pdfescape-test2}{%
    Missing \string\pdfescape... commands%
  }{Test aborted.}%
  \stop
}{}
%</test2>
%    \end{macrocode}
%    \begin{macrocode}
%<*test3|test4>
\let\pdfescapehex\@undefined
\let\pdfunescapehex\@undefined
\let\pdfescapename\@undefined
\let\pdfescapestring\@undefined
%</test3|test4>
%    \end{macrocode}
%
% \subsubsection{Check \eTeX}
%
%    \begin{macrocode}
%<*test3>
\@ifundefined{numexpr}{%
  \PackageError{pdfescape-test3}{%
    Missing \eTeX
  }{Test aborted.}%
  \stop
}{}
%</test3>
%    \end{macrocode}
% Package \xpackage{qstest} uses \eTeX, thus \eTeX's features
% can only be disabled later during loading of package \xpackage{pdfescape}.
%
% \subsubsection{Check \LuaTeX}
%
%    \begin{macrocode}
%<*test5>
\@ifundefined{directlua}{%
  \PackageError{pdfescape-test5}{%
    Missing LuaTeX%
  }{Test aborted.}%
  \stop
}{}
%</test5>
%    \end{macrocode}
%
% \subsection{Common part}
%
%    The files for testing uses the framework, provided by the
%    new package \xpackage{qstest} of David Kastrup.
%    \begin{macrocode}
\RequirePackage{qstest}
\IncludeTests{*}
\LogTests{log}{*}{*}

\newcommand*{\ExpectVar}[2]{%
  \ifx#1#2%
  \else
    \begingroup
      \@onelevel@sanitize#1%
      \@onelevel@sanitize#2%
      \typeout{[#1] <> [#2]}% hash-ok
    \endgroup
  \fi
  \Expect*{\ifx#1#2true\else false\fi}{true}%
}

\makeatletter
\begingroup
  \gdef\AllBytes{}%
  \count@=0 %
  \catcode0=12 %
  \@whilenum\count@<256 \do{%
    \lccode0=\count@
    \ifnum\count@=32 %
      \xdef\AllBytes{\AllBytes\space}%
    \else
      \lowercase{%
        \xdef\AllBytes{\AllBytes^^@}%
      }%
    \fi
    \advance\count@ by 1 %
  }%
\endgroup
\newcommand*{\AllBytesHex}{%
  000102030405060708090A0B0C0D0E0F%
  101112131415161718191A1B1C1D1E1F%
  202122232425262728292A2B2C2D2E2F%
  303132333435363738393A3B3C3D3E3F%
  404142434445464748494A4B4C4D4E4F%
  505152535455565758595A5B5C5D5E5F%
  606162636465666768696A6B6C6D6E6F%
  707172737475767778797A7B7C7D7E7F%
  808182838485868788898A8B8C8D8E8F%
  909192939495969798999A9B9C9D9E9F%
  A0A1A2A3A4A5A6A7A8A9AAABACADAEAF%
  B0B1B2B3B4B5B6B7B8B9BABBBCBDBEBF%
  C0C1C2C3C4C5C6C7C8C9CACBCCCDCECF%
  D0D1D2D3D4D5D6D7D8D9DADBDCDDDEDF%
  E0E1E2E3E4E5E6E7E8E9EAEBECEDEEEF%
  F0F1F2F3F4F5F6F7F8F9FAFBFCFDFEFF%
}
\@onelevel@sanitize\AllBytesHex
\expandafter\lowercase\expandafter{%
  \expandafter\newcommand\expandafter*\expandafter\AllBytesHexLC
      \expandafter{\AllBytesHex}%
}
\newcommand*{\AllBytesName}{}
\begingroup
  \catcode`\#=12 %
  \xdef\AllBytesName{%
    #01#02#03#04#05#06#07#08#09#0A#0B#0C#0D#0E#0F%
    #10#11#12#13#14#15#16#17#18#19#1A#1B#1C#1D#1E#1F%
    #20!"#23$#25&'#28#29*+,-.#2F%
    0123456789:;#3C=#3E?%
    @ABCDEFGHIJKLMNO%
    PQRSTUVWXYZ#5B\@backslashchar#5D^_%
    `abcdefghijklmno%
    pqrstuvwxyz#7B|#7D\string~#7F%
    #80#81#82#83#84#85#86#87#88#89#8A#8B#8C#8D#8E#8F%
    #90#91#92#93#94#95#96#97#98#99#9A#9B#9C#9D#9E#9F%
    #A0#A1#A2#A3#A4#A5#A6#A7#A8#A9#AA#AB#AC#AD#AE#AF%
    #B0#B1#B2#B3#B4#B5#B6#B7#B8#B9#BA#BB#BC#BD#BE#BF%
    #C0#C1#C2#C3#C4#C5#C6#C7#C8#C9#CA#CB#CC#CD#CE#CF%
    #D0#D1#D2#D3#D4#D5#D6#D7#D8#D9#DA#DB#DC#DD#DE#DF%
    #E0#E1#E2#E3#E4#E5#E6#E7#E8#E9#EA#EB#EC#ED#EE#EF%
    #F0#F1#F2#F3#F4#F5#F6#F7#F8#F9#FA#FB#FC#FD#FE#FF%
  }%
\endgroup
\@onelevel@sanitize\AllBytesName

\newcommand*{\AllBytesString}{}
\begingroup
  \def\|{|}%
  \edef\%{\@percentchar}%
  \catcode`\|=0 %
  \catcode`\#=12 %
  \catcode`\~=12 %
  \catcode`\\=12 %
  |xdef|AllBytesString{%
    \000\001\002\003\004\005\006\007\010\011\012\013\014\015\016\017%
    \020\021\022\023\024\025\026\027\030\031\032\033\034\035\036\037%
    \040!"#$|%&'\(\)*+,-./%
    0123456789:;<=>?%
    @ABCDEFGHIJKLMNO%
    PQRSTUVWXYZ[\\]^_%
    `abcdefghijklmno%
    pqrstuvwxyz{||}~\177%
    \200\201\202\203\204\205\206\207\210\211\212\213\214\215\216\217%
    \220\221\222\223\224\225\226\227\230\231\232\233\234\235\236\237%
    \240\241\242\243\244\245\246\247\250\251\252\253\254\255\256\257%
    \260\261\262\263\264\265\266\267\270\271\272\273\274\275\276\277%
    \300\301\302\303\304\305\306\307\310\311\312\313\314\315\316\317%
    \320\321\322\323\324\325\326\327\330\331\332\333\334\335\336\337%
    \340\341\342\343\344\345\346\347\350\351\352\353\354\355\356\357%
    \360\361\362\363\364\365\366\367\370\371\372\373\374\375\376\377%
  }%
|endgroup
\@onelevel@sanitize\AllBytesString

%<*test4>
\let\org@detokenize\detokenize
\let\detokenize\@undefined
\let\org@numexpr\numexpr
\let\numexpr\@undefined
%</test4>
\RequirePackage{pdfescape}
%<*test4>
\let\detokenize\org@detokenize
\let\numexpr\org@numexpr
%</test4>

\begin{qstest}{all-hex}{\AllBytes, escapehex}
  \EdefEscapeHex\x{\AllBytes}%
  \Expect*{\x}*{\AllBytesHex}%
  \ExpectVar\x\AllBytesHex
\end{qstest}

\begin{qstest}{all-unhex}{\AllBytesHex, unescapehex}
  \EdefUnescapeHex\x{\AllBytesHex}%
  \Expect*{\x}*{\AllBytes}%
  \ExpectVar\x\AllBytes
\end{qstest}

\begin{qstest}{all-unhex-lc}{\AllBytesHexLC, unescapehex, lowercase}
  \EdefUnescapeHex\x{\AllBytesHexLC}%
  \Expect*{\x}*{\AllBytes}%
  \ExpectVar\x\AllBytes
\end{qstest}

\begin{qstest}{unhex-incomplete}{unescapehex, incomplete}
  \EdefUnescapeHex\x{4}%
  \Expect*{\x}{@}%
\end{qstest}

\begin{qstest}{unhex-space}{unescapehex, space}
  \EdefUnescapeHex\x{20}%
  \Expect*{\x}{ }%
  \ExpectVar\x\space
\end{qstest}

\begin{qstest}{unhex-spaces}{unescapehex, spaces}
  \EdefUnescapeHex\x{204020204120}%
  \def\y#1{%
    \edef\z{#1\string @#1#1\string A#1}%
  }\y{ }%
  \Expect*{\x}*{\z}%
  \ExpectVar\x\z
\end{qstest}

\begin{qstest}{unhex-hash}{unescapehex, hash}
  \catcode`\#=12 %
  \EdefUnescapeHex\x{#20}%
  \ExpectVar\x\space
\end{qstest}

\begin{qstest}{unhex-invalid}{unescapehex, invalid}
  \def\test#1#2{%
    \EdefUnescapeHex\x{#1}%
    \edef\y{#2}%
    \@onelevel@sanitize\y
    \ExpectVar\x\y
  }%
%<*test2>
  \edef\x{\pdfunescapehex{4X}}%
  \edef\y{\string @}%
  \ifx\x\y
  \else
    \def~{\space}%
    \typeout{************************************}%
    \typeout{* Your pdfTeX contains bug 777.~~~~*}%
    \typeout{* This test is redefined as dummy, *}%
    \typeout{* because it triggers the bug.~~~~~*}%
    \typeout{************************************}%
    \def\test#1#2{}%
  \fi
%</test2>
  \test{X}{}%
  \test{XY}{}%
  \test{XYZ}{}%
  \test{A}{^^a0}%
  \test{AX}{^^a0}%
  \test{XA}{^^a0}%
  \test{XXAXX}{^^a0}%
\end{qstest}

\begin{qstest}{all-name}{\AllBytes, escapename}
  \EdefEscapeName\x{\AllBytes}%
  \Expect*{\x}*{\AllBytesName}%
  \ExpectVar\x\AllBytesName
\end{qstest}

\begin{qstest}{all-string}{\AllBytes, escapestring}
  \EdefEscapeString\x{\AllBytes}%
  \Expect*{\x}*{\AllBytesString}%
  \ExpectVar\x\AllBytesString
\end{qstest}

\begin{qstest}{uchexdigit}{unescape, uppercase hex digit}
  \catcode`\@=11 %
  \catcode0=12 %
  \def\test#1#2{%
    \uccode0=#1\relax
    \uppercase{%
      \def\x{^^@}%
    }%
    \Expect*{%
       \ifcase\expandafter\PE@TestUcHexDigit\x
         true%
       \else
         false%
       \fi
    }{#2}%
  }%
  \def\range#1#2#3{%
    \count0=#1\relax
    \loop
    \ifnum\count0<#2\relax
      \test{\count0}{#3}%
      \advance\count0 by 1 %
    \repeat
  }%
  \range{0}{47}{false}%
  \range{48}{57}{true}%
  \range{58}{64}{false}%
  \range{65}{70}{true}%
  \range{71}{255}{false}%
\end{qstest}

\begin{qstest}{unescapename}{unescapename}
  \def\test#1#2{%
    \EdefUnescapeName\x{#1}%
    \edef\y{#2}%
    \@onelevel@sanitize\y
    \ExpectVar\x\y
  }%
  \catcode`\#=12 %
  \catcode0=12 %
  \test{}{}%
  \test{x}{x}%
  \test{xy}{xy}%
  \test{#}{#}%
  \test{##}{##}%
  \test{###}{###}%
  \test{####}{####}%
  \test{#x}{#x}%
  \test{#xy}{#xy}%
  \test{#1}{#1}%
  \test{#40}{@}%
  \test{#400}{@0}%
  \test{#4x0}{#4x0}%
  \test{#ab}{^^ab}%
  \test{#00}{^^@}%
  \test{x#40y#40z}{x@y@z}%
  \test{#40#40#40#40}{@@@@}%
  \test{a#x}{a#x}%
  \test{a#xy}{a#xy}%
  \test{a#1}{a#1}%
  \test{a#40}{a@}%
  \test{a#400}{a@0}%
  \test{#20}{ }%
  \test{a#20}{a }%
  \test{a#20b}{a b}%
  \test{a#20#20#20b}{a \space\space b}%
\end{qstest}

\begin{qstest}{unescapestring}{unescapestring}
  \def\test#1#2{%
    \EdefUnescapeString\x{#1}%
    \edef\y{#2}%
    \@onelevel@sanitize\y
    \ExpectVar\x\y
  }%
  \catcode0=12 %
  \def\DefChar#1#2{%
    \begingroup
      \uccode0=#2\relax
    \uppercase{\endgroup
      \def#1{^^@}%
    }%
  }%
  \DefChar\nul{0}%
  \DefChar\one{1}%
  \DefChar\bel{8}%
  \DefChar\tab{9}%
  \DefChar\lf{10}%
  \DefChar\ff{12}%
  \DefChar\cr{13}%
  \DefChar\\{92}%
  \test{}{}%
  \test{a}{a}%
  \test{\\}{}%
  \test{\\\\}{\\}%
  \test{\\\\y}{\\y}%
  \test{\\000}{\nul}%
  \test{\\b}{\bel}%
  \test{\\t}{\tab}%
  \test{\\n}{\lf}%
  \test{\\f}{\ff}%
  \test{\\r}{\cr}%
  \test{\\(}{(}%
  \test{\\)}{)}%
  \test{\\040}{ }%
  \test{\\100}{@}%
  \test{\\40}{ }%
  \test{\\1}{\one}%
  \test{\\01}{\one}%
  \test{\\001}{\one}%
  \test{\\18}{\one8}%
  \test{\\018}{\one8}%
  \test{\\0018}{\one8}%
  \test{x\\}{x}%
  \test{x\\\\}{x\\}%
  \test{x\\\\y}{x\\y}%
  \test{x\\000}{x\nul}%
  \test{x\\b}{x\bel}%
  \test{x\\t}{x\tab}%
  \test{x\\n}{x\lf}%
  \test{x\\f}{x\ff}%
  \test{x\\r}{x\cr}%
  \test{x\\(}{x(}%
  \test{x\\)}{x)}%
  \test{x\\040}{x }%
  \test{x\\100}{x@}%
  \test{x\\40}{x }%
  \test{x\\1}{x\one}%
  \test{x\\01}{x\one}%
  \test{x\\001}{x\one}%
  \test{x\\18}{x\one8}%
  \test{x\\018}{x\one8}%
  \test{x\\0018}{x\one8}%
  \test{\\b\\t\\n\\f\\r\\(\\)\\\\\\000\\040}{%
    \bel\tab\lf\ff\cr()\\\nul\space
  }%
  \test{\\\lf}{}%
  \test{x\\\lf}{x}%
  \test{\cr}{\lf}%
  \test{\cr\lf}{\lf}%
  \test{\lf}{\lf}%
  \test{\lf\cr}{\lf\lf}%
  \test{x\cr}{x\lf}%
  \test{x\cr\lf}{x\lf}%
  \test{x\lf}{x\lf}%
  \test{x\lf\cr}{x\lf\lf}%
  \test{x\\\cr\lf y\cr}{xy\lf}%
%
  \test{\\409}{ 9}%
  \test{\\800}{800}%
  \test{\\900}{900}%
  \test{\\578}{/8}%
  \test{\\477}{?}%
  \test{\\377}{^^ff}%
  \test{\\777}{^^ff}%
  \test{\\7777}{^^ff7}%
\end{qstest}
\stop
%    \end{macrocode}
%
%    \begin{macrocode}
%</test2|test3|test4|test5>
%    \end{macrocode}
%
% \subsubsection{Test for \hologo{iniTeX}}
%
%    \begin{macrocode}
%<*test6>
\input pdfescape.sty\relax
\catcode`\{=1 %
\catcode`\}=2 %
\catcode`\#=6 %
\catcode`\^=7 %
\catcode`\@=11 %
%    \end{macrocode}
%    \begin{macrocode}
\begingroup
  \catcode`\@=11 %
  \countdef\count@=255 %
  \def\space{ }%
  \long\def\@whilenum#1\do #2{%
    \ifnum #1\relax
      #2\relax
      \@iwhilenum{#1\relax#2\relax}%
    \fi
  }%
  \long\def\@iwhilenum#1{%
    \ifnum #1%
      \expandafter\@iwhilenum
    \else
      \expandafter\ltx@gobble
    \fi
    {#1}%
  }%
  \gdef\AllBytes{}%
  \count@=0 %
  \catcode0=12 %
  \@whilenum\count@<256 \do{%
    \lccode0=\count@
    \ifnum\count@=32 %
      \xdef\AllBytes{\AllBytes\space}%
    \else
      \lowercase{%
        \xdef\AllBytes{\AllBytes^^@}%
      }%
    \fi
    \advance\count@ by 1 %
  }%
\endgroup
\def\AllBytesHex{%
  000102030405060708090A0B0C0D0E0F%
  101112131415161718191A1B1C1D1E1F%
  202122232425262728292A2B2C2D2E2F%
  303132333435363738393A3B3C3D3E3F%
  404142434445464748494A4B4C4D4E4F%
  505152535455565758595A5B5C5D5E5F%
  606162636465666768696A6B6C6D6E6F%
  707172737475767778797A7B7C7D7E7F%
  808182838485868788898A8B8C8D8E8F%
  909192939495969798999A9B9C9D9E9F%
  A0A1A2A3A4A5A6A7A8A9AAABACADAEAF%
  B0B1B2B3B4B5B6B7B8B9BABBBCBDBEBF%
  C0C1C2C3C4C5C6C7C8C9CACBCCCDCECF%
  D0D1D2D3D4D5D6D7D8D9DADBDCDDDEDF%
  E0E1E2E3E4E5E6E7E8E9EAEBECEDEEEF%
  F0F1F2F3F4F5F6F7F8F9FAFBFCFDFEFF%
}
\ltx@onelevel@sanitize\AllBytesHex
\expandafter\lowercase\expandafter{%
  \expandafter\def\expandafter\AllBytesHexLC
      \expandafter{\AllBytesHex}%
}
\begingroup
  \catcode`\#=12 %
  \xdef\AllBytesName{%
    #01#02#03#04#05#06#07#08#09#0A#0B#0C#0D#0E#0F%
    #10#11#12#13#14#15#16#17#18#19#1A#1B#1C#1D#1E#1F%
    #20!"#23$#25&'#28#29*+,-.#2F%
    0123456789:;#3C=#3E?%
    @ABCDEFGHIJKLMNO%
    PQRSTUVWXYZ#5B\ltx@backslashchar#5D^_%
    `abcdefghijklmno%
    pqrstuvwxyz#7B|#7D\string~#7F%
    #80#81#82#83#84#85#86#87#88#89#8A#8B#8C#8D#8E#8F%
    #90#91#92#93#94#95#96#97#98#99#9A#9B#9C#9D#9E#9F%
    #A0#A1#A2#A3#A4#A5#A6#A7#A8#A9#AA#AB#AC#AD#AE#AF%
    #B0#B1#B2#B3#B4#B5#B6#B7#B8#B9#BA#BB#BC#BD#BE#BF%
    #C0#C1#C2#C3#C4#C5#C6#C7#C8#C9#CA#CB#CC#CD#CE#CF%
    #D0#D1#D2#D3#D4#D5#D6#D7#D8#D9#DA#DB#DC#DD#DE#DF%
    #E0#E1#E2#E3#E4#E5#E6#E7#E8#E9#EA#EB#EC#ED#EE#EF%
    #F0#F1#F2#F3#F4#F5#F6#F7#F8#F9#FA#FB#FC#FD#FE#FF%
  }%
\endgroup
\ltx@onelevel@sanitize\AllBytesName
\edef\AllBytesFromName{\expandafter\ltx@gobble\AllBytes}

\begingroup
  \def\|{|}%
  \edef\%{\ltx@percentchar}%
  \catcode`\|=0 %
  \catcode`\#=12 %
  \catcode`\~=12 %
  \catcode`\\=12 %
  |xdef|AllBytesString{%
    \000\001\002\003\004\005\006\007\010\011\012\013\014\015\016\017%
    \020\021\022\023\024\025\026\027\030\031\032\033\034\035\036\037%
    \040!"#$|%&'\(\)*+,-./%
    0123456789:;<=>?%
    @ABCDEFGHIJKLMNO%
    PQRSTUVWXYZ[\\]^_%
    `abcdefghijklmno%
    pqrstuvwxyz{||}~\177%
    \200\201\202\203\204\205\206\207\210\211\212\213\214\215\216\217%
    \220\221\222\223\224\225\226\227\230\231\232\233\234\235\236\237%
    \240\241\242\243\244\245\246\247\250\251\252\253\254\255\256\257%
    \260\261\262\263\264\265\266\267\270\271\272\273\274\275\276\277%
    \300\301\302\303\304\305\306\307\310\311\312\313\314\315\316\317%
    \320\321\322\323\324\325\326\327\330\331\332\333\334\335\336\337%
    \340\341\342\343\344\345\346\347\350\351\352\353\354\355\356\357%
    \360\361\362\363\364\365\366\367\370\371\372\373\374\375\376\377%
  }%
|endgroup
\ltx@onelevel@sanitize\AllBytesString
%    \end{macrocode}
%    \begin{macrocode}
\def\msg#{\immediate\write16}
\def\Test#1#2#3{%
  \begingroup
    #1\TestResult{#2}%
    \ifx\TestResult#3%
    \else
      \newlinechar=10 %
      \msg{Expect:^^J#3}%
      \msg{Result:^^J\TestResult}%
      \errmessage{\string#2 -\string#1-> \string#3}%
    \fi
  \endgroup
}
\Test\EdefEscapeHex\AllBytes\AllBytesHex
\Test\EdefUnescapeHex\AllBytesHex\AllBytes
\Test\EdefEscapeName\AllBytes\AllBytesName
\Test\EdefUnescapeName\AllBytesName\AllBytesFromName
\Test\EdefEscapeString\AllBytes\AllBytesString
\Test\EdefUnescapeString\AllBytesString\AllBytes
%    \end{macrocode}
%    \begin{macrocode}
\csname @@end\endcsname\end
%</test6>
%    \end{macrocode}
%
% \section{Installation}
%
% \subsection{Download}
%
% \paragraph{Package.} This package is available on
% CTAN\footnote{\url{ftp://ftp.ctan.org/tex-archive/}}:
% \begin{description}
% \item[\CTAN{macros/latex/contrib/oberdiek/pdfescape.dtx}] The source file.
% \item[\CTAN{macros/latex/contrib/oberdiek/pdfescape.pdf}] Documentation.
% \end{description}
%
%
% \paragraph{Bundle.} All the packages of the bundle `oberdiek'
% are also available in a TDS compliant ZIP archive. There
% the packages are already unpacked and the documentation files
% are generated. The files and directories obey the TDS standard.
% \begin{description}
% \item[\CTAN{install/macros/latex/contrib/oberdiek.tds.zip}]
% \end{description}
% \emph{TDS} refers to the standard ``A Directory Structure
% for \TeX\ Files'' (\CTAN{tds/tds.pdf}). Directories
% with \xfile{texmf} in their name are usually organized this way.
%
% \subsection{Bundle installation}
%
% \paragraph{Unpacking.} Unpack the \xfile{oberdiek.tds.zip} in the
% TDS tree (also known as \xfile{texmf} tree) of your choice.
% Example (linux):
% \begin{quote}
%   |unzip oberdiek.tds.zip -d ~/texmf|
% \end{quote}
%
% \paragraph{Script installation.}
% Check the directory \xfile{TDS:scripts/oberdiek/} for
% scripts that need further installation steps.
% Package \xpackage{attachfile2} comes with the Perl script
% \xfile{pdfatfi.pl} that should be installed in such a way
% that it can be called as \texttt{pdfatfi}.
% Example (linux):
% \begin{quote}
%   |chmod +x scripts/oberdiek/pdfatfi.pl|\\
%   |cp scripts/oberdiek/pdfatfi.pl /usr/local/bin/|
% \end{quote}
%
% \subsection{Package installation}
%
% \paragraph{Unpacking.} The \xfile{.dtx} file is a self-extracting
% \docstrip\ archive. The files are extracted by running the
% \xfile{.dtx} through \plainTeX:
% \begin{quote}
%   \verb|tex pdfescape.dtx|
% \end{quote}
%
% \paragraph{TDS.} Now the different files must be moved into
% the different directories in your installation TDS tree
% (also known as \xfile{texmf} tree):
% \begin{quote}
% \def\t{^^A
% \begin{tabular}{@{}>{\ttfamily}l@{ $\rightarrow$ }>{\ttfamily}l@{}}
%   pdfescape.sty & tex/generic/oberdiek/pdfescape.sty\\
%   pdfescape.pdf & doc/latex/oberdiek/pdfescape.pdf\\
%   test/pdfescape-test1.tex & doc/latex/oberdiek/test/pdfescape-test1.tex\\
%   test/pdfescape-test2.tex & doc/latex/oberdiek/test/pdfescape-test2.tex\\
%   test/pdfescape-test3.tex & doc/latex/oberdiek/test/pdfescape-test3.tex\\
%   test/pdfescape-test4.tex & doc/latex/oberdiek/test/pdfescape-test4.tex\\
%   test/pdfescape-test5.tex & doc/latex/oberdiek/test/pdfescape-test5.tex\\
%   test/pdfescape-test6.tex & doc/latex/oberdiek/test/pdfescape-test6.tex\\
%   pdfescape.dtx & source/latex/oberdiek/pdfescape.dtx\\
% \end{tabular}^^A
% }^^A
% \sbox0{\t}^^A
% \ifdim\wd0>\linewidth
%   \begingroup
%     \advance\linewidth by\leftmargin
%     \advance\linewidth by\rightmargin
%   \edef\x{\endgroup
%     \def\noexpand\lw{\the\linewidth}^^A
%   }\x
%   \def\lwbox{^^A
%     \leavevmode
%     \hbox to \linewidth{^^A
%       \kern-\leftmargin\relax
%       \hss
%       \usebox0
%       \hss
%       \kern-\rightmargin\relax
%     }^^A
%   }^^A
%   \ifdim\wd0>\lw
%     \sbox0{\small\t}^^A
%     \ifdim\wd0>\linewidth
%       \ifdim\wd0>\lw
%         \sbox0{\footnotesize\t}^^A
%         \ifdim\wd0>\linewidth
%           \ifdim\wd0>\lw
%             \sbox0{\scriptsize\t}^^A
%             \ifdim\wd0>\linewidth
%               \ifdim\wd0>\lw
%                 \sbox0{\tiny\t}^^A
%                 \ifdim\wd0>\linewidth
%                   \lwbox
%                 \else
%                   \usebox0
%                 \fi
%               \else
%                 \lwbox
%               \fi
%             \else
%               \usebox0
%             \fi
%           \else
%             \lwbox
%           \fi
%         \else
%           \usebox0
%         \fi
%       \else
%         \lwbox
%       \fi
%     \else
%       \usebox0
%     \fi
%   \else
%     \lwbox
%   \fi
% \else
%   \usebox0
% \fi
% \end{quote}
% If you have a \xfile{docstrip.cfg} that configures and enables \docstrip's
% TDS installing feature, then some files can already be in the right
% place, see the documentation of \docstrip.
%
% \subsection{Refresh file name databases}
%
% If your \TeX~distribution
% (\teTeX, \mikTeX, \dots) relies on file name databases, you must refresh
% these. For example, \teTeX\ users run \verb|texhash| or
% \verb|mktexlsr|.
%
% \subsection{Some details for the interested}
%
% \paragraph{Attached source.}
%
% The PDF documentation on CTAN also includes the
% \xfile{.dtx} source file. It can be extracted by
% AcrobatReader 6 or higher. Another option is \textsf{pdftk},
% e.g. unpack the file into the current directory:
% \begin{quote}
%   \verb|pdftk pdfescape.pdf unpack_files output .|
% \end{quote}
%
% \paragraph{Unpacking with \LaTeX.}
% The \xfile{.dtx} chooses its action depending on the format:
% \begin{description}
% \item[\plainTeX:] Run \docstrip\ and extract the files.
% \item[\LaTeX:] Generate the documentation.
% \end{description}
% If you insist on using \LaTeX\ for \docstrip\ (really,
% \docstrip\ does not need \LaTeX), then inform the autodetect routine
% about your intention:
% \begin{quote}
%   \verb|latex \let\install=y\input{pdfescape.dtx}|
% \end{quote}
% Do not forget to quote the argument according to the demands
% of your shell.
%
% \paragraph{Generating the documentation.}
% You can use both the \xfile{.dtx} or the \xfile{.drv} to generate
% the documentation. The process can be configured by the
% configuration file \xfile{ltxdoc.cfg}. For instance, put this
% line into this file, if you want to have A4 as paper format:
% \begin{quote}
%   \verb|\PassOptionsToClass{a4paper}{article}|
% \end{quote}
% An example follows how to generate the
% documentation with pdf\LaTeX:
% \begin{quote}
%\begin{verbatim}
%pdflatex pdfescape.dtx
%makeindex -s gind.ist pdfescape.idx
%pdflatex pdfescape.dtx
%makeindex -s gind.ist pdfescape.idx
%pdflatex pdfescape.dtx
%\end{verbatim}
% \end{quote}
%
% \section{Catalogue}
%
% The following XML file can be used as source for the
% \href{http://mirror.ctan.org/help/Catalogue/catalogue.html}{\TeX\ Catalogue}.
% The elements \texttt{caption} and \texttt{description} are imported
% from the original XML file from the Catalogue.
% The name of the XML file in the Catalogue is \xfile{pdfescape.xml}.
%    \begin{macrocode}
%<*catalogue>
<?xml version='1.0' encoding='us-ascii'?>
<!DOCTYPE entry SYSTEM 'catalogue.dtd'>
<entry datestamp='$Date$' modifier='$Author$' id='pdfescape'>
  <name>pdfescape</name>
  <caption>Implements pdfTeX's escape features using TeX or e-TeX.</caption>
  <authorref id='auth:oberdiek'/>
  <copyright owner='Heiko Oberdiek' year='2007,2010,2011'/>
  <license type='lppl1.3'/>
  <version number='1.13'/>
  <description>
    This package implements <xref refid='pdftex'>pdfTeX</xref>'s
    escape features (<tt>\pdfescapehex</tt>, <tt>\pdfunescapehex</tt>,
    <tt>\pdfescapename</tt>, <tt>\pdfescapestring</tt>) using TeX or
    e-TeX.
    <p/>
    The package is part of the <xref refid='oberdiek'>oberdiek</xref>
    bundle.
  </description>
  <documentation details='Package documentation'
      href='ctan:/macros/latex/contrib/oberdiek/pdfescape.pdf'/>
  <ctan file='true' path='/macros/latex/contrib/oberdiek/pdfescape.dtx'/>
  <miktex location='oberdiek'/>
  <texlive location='oberdiek'/>
  <install path='/macros/latex/contrib/oberdiek/oberdiek.tds.zip'/>
</entry>
%</catalogue>
%    \end{macrocode}
%
% \begin{History}
%   \begin{Version}{2007/02/21 v1.0}
%   \item
%     First version.
%   \end{Version}
%   \begin{Version}{2007/02/25 v1.1}
%   \item
%     Test files added.
%   \item
%     \cs{EdefUnescapeHex} supports lowercase letters.
%   \item
%     Fix: \verb|\EdefEscapeName{^^@}|
%   \item
%     Fix: \verb|\EdefEscapeName{\string#}|
%   \item
%     Fix for \cs{EdefUnescapeHex} in case of incomplete hex string.
%   \item
%     Fix: \cs{EdefUnescapeHex} generates space tokens with catcode
%     10 (space) in all cases.
%   \item
%     Fix: \cs{EdefEscapeHex} and \cs{EdefEscapeName} now generate
%     tokens with catcode 12 (other) only.
%   \end{Version}
%   \begin{Version}{2007/03/20 v1.2}
%   \item
%     Fix: Wrong year in \cs{ProvidesPackage}.
%   \end{Version}
%   \begin{Version}{2007/04/11 v1.3}
%   \item
%     Line ends sanitized.
%   \end{Version}
%   \begin{Version}{2007/04/21 v1.4}
%   \item
%     \cs{EdefUnescapeName} and \cs{EdefUnescapeString} added.
%   \end{Version}
%   \begin{Version}{2007/08/27 v1.5}
%   \item
%     \cs{EdefSanitize} added (replaces \cs{PE@sanitize}).
%   \end{Version}
%   \begin{Version}{2007/09/09 v1.6}
%   \item
%     Fix in catcode setup.
%   \end{Version}
%   \begin{Version}{2007/10/27 v1.7}
%   \item
%     More efficient \cs{EdefSanitize}.
%   \end{Version}
%   \begin{Version}{2007/11/11 v1.8}
%   \item
%     Use of package \xpackage{pdftexcmds} for \hologo{LuaTeX} support.
%   \end{Version}
%   \begin{Version}{2010/03/01 v1.9}
%   \item
%     Compatibility with \hologo{iniTeX}.
%   \end{Version}
%   \begin{Version}{2010/11/12 v1.10}
%   \item
%     Use of package \xpackage{ltxcmds}.
%   \item
%     Fix for compatibility with \hologo{iniTeX}.
%   \end{Version}
%   \begin{Version}{2011/01/30 v1.11}
%   \item
%     Already loaded package files are not input in \hologo{plainTeX}.
%   \end{Version}
%   \begin{Version}{2011/04/04 v1.12}
%   \item
%     Further fixes for compatibility for \hologo{iniTeX}.
%   \item
%     Test file for \hologo{iniTeX} added.
%   \end{Version}
%   \begin{Version}{2011/11/25 v1.13}
%   \item
%     Higher order bit of octal sequences in \cs{EdefUnescapeString}
%     ignored according to the PDF specification
%     (Bug found by Bruno Le Floch).
%   \end{Version}
% \end{History}
%
% \PrintIndex
%
% \Finale
\endinput
|
% \end{quote}
% Do not forget to quote the argument according to the demands
% of your shell.
%
% \paragraph{Generating the documentation.}
% You can use both the \xfile{.dtx} or the \xfile{.drv} to generate
% the documentation. The process can be configured by the
% configuration file \xfile{ltxdoc.cfg}. For instance, put this
% line into this file, if you want to have A4 as paper format:
% \begin{quote}
%   \verb|\PassOptionsToClass{a4paper}{article}|
% \end{quote}
% An example follows how to generate the
% documentation with pdf\LaTeX:
% \begin{quote}
%\begin{verbatim}
%pdflatex pdfescape.dtx
%makeindex -s gind.ist pdfescape.idx
%pdflatex pdfescape.dtx
%makeindex -s gind.ist pdfescape.idx
%pdflatex pdfescape.dtx
%\end{verbatim}
% \end{quote}
%
% \section{Catalogue}
%
% The following XML file can be used as source for the
% \href{http://mirror.ctan.org/help/Catalogue/catalogue.html}{\TeX\ Catalogue}.
% The elements \texttt{caption} and \texttt{description} are imported
% from the original XML file from the Catalogue.
% The name of the XML file in the Catalogue is \xfile{pdfescape.xml}.
%    \begin{macrocode}
%<*catalogue>
<?xml version='1.0' encoding='us-ascii'?>
<!DOCTYPE entry SYSTEM 'catalogue.dtd'>
<entry datestamp='$Date$' modifier='$Author$' id='pdfescape'>
  <name>pdfescape</name>
  <caption>Implements pdfTeX's escape features using TeX or e-TeX.</caption>
  <authorref id='auth:oberdiek'/>
  <copyright owner='Heiko Oberdiek' year='2007,2010,2011'/>
  <license type='lppl1.3'/>
  <version number='1.13'/>
  <description>
    This package implements <xref refid='pdftex'>pdfTeX</xref>'s
    escape features (<tt>\pdfescapehex</tt>, <tt>\pdfunescapehex</tt>,
    <tt>\pdfescapename</tt>, <tt>\pdfescapestring</tt>) using TeX or
    e-TeX.
    <p/>
    The package is part of the <xref refid='oberdiek'>oberdiek</xref>
    bundle.
  </description>
  <documentation details='Package documentation'
      href='ctan:/macros/latex/contrib/oberdiek/pdfescape.pdf'/>
  <ctan file='true' path='/macros/latex/contrib/oberdiek/pdfescape.dtx'/>
  <miktex location='oberdiek'/>
  <texlive location='oberdiek'/>
  <install path='/macros/latex/contrib/oberdiek/oberdiek.tds.zip'/>
</entry>
%</catalogue>
%    \end{macrocode}
%
% \begin{History}
%   \begin{Version}{2007/02/21 v1.0}
%   \item
%     First version.
%   \end{Version}
%   \begin{Version}{2007/02/25 v1.1}
%   \item
%     Test files added.
%   \item
%     \cs{EdefUnescapeHex} supports lowercase letters.
%   \item
%     Fix: \verb|\EdefEscapeName{^^@}|
%   \item
%     Fix: \verb|\EdefEscapeName{\string#}|
%   \item
%     Fix for \cs{EdefUnescapeHex} in case of incomplete hex string.
%   \item
%     Fix: \cs{EdefUnescapeHex} generates space tokens with catcode
%     10 (space) in all cases.
%   \item
%     Fix: \cs{EdefEscapeHex} and \cs{EdefEscapeName} now generate
%     tokens with catcode 12 (other) only.
%   \end{Version}
%   \begin{Version}{2007/03/20 v1.2}
%   \item
%     Fix: Wrong year in \cs{ProvidesPackage}.
%   \end{Version}
%   \begin{Version}{2007/04/11 v1.3}
%   \item
%     Line ends sanitized.
%   \end{Version}
%   \begin{Version}{2007/04/21 v1.4}
%   \item
%     \cs{EdefUnescapeName} and \cs{EdefUnescapeString} added.
%   \end{Version}
%   \begin{Version}{2007/08/27 v1.5}
%   \item
%     \cs{EdefSanitize} added (replaces \cs{PE@sanitize}).
%   \end{Version}
%   \begin{Version}{2007/09/09 v1.6}
%   \item
%     Fix in catcode setup.
%   \end{Version}
%   \begin{Version}{2007/10/27 v1.7}
%   \item
%     More efficient \cs{EdefSanitize}.
%   \end{Version}
%   \begin{Version}{2007/11/11 v1.8}
%   \item
%     Use of package \xpackage{pdftexcmds} for \hologo{LuaTeX} support.
%   \end{Version}
%   \begin{Version}{2010/03/01 v1.9}
%   \item
%     Compatibility with \hologo{iniTeX}.
%   \end{Version}
%   \begin{Version}{2010/11/12 v1.10}
%   \item
%     Use of package \xpackage{ltxcmds}.
%   \item
%     Fix for compatibility with \hologo{iniTeX}.
%   \end{Version}
%   \begin{Version}{2011/01/30 v1.11}
%   \item
%     Already loaded package files are not input in \hologo{plainTeX}.
%   \end{Version}
%   \begin{Version}{2011/04/04 v1.12}
%   \item
%     Further fixes for compatibility for \hologo{iniTeX}.
%   \item
%     Test file for \hologo{iniTeX} added.
%   \end{Version}
%   \begin{Version}{2011/11/25 v1.13}
%   \item
%     Higher order bit of octal sequences in \cs{EdefUnescapeString}
%     ignored according to the PDF specification
%     (Bug found by Bruno Le Floch).
%   \end{Version}
% \end{History}
%
% \PrintIndex
%
% \Finale
\endinput

%        (quote the arguments according to the demands of your shell)
%
% Documentation:
%    (a) If pdfescape.drv is present:
%           latex pdfescape.drv
%    (b) Without pdfescape.drv:
%           latex pdfescape.dtx; ...
%    The class ltxdoc loads the configuration file ltxdoc.cfg
%    if available. Here you can specify further options, e.g.
%    use A4 as paper format:
%       \PassOptionsToClass{a4paper}{article}
%
%    Programm calls to get the documentation (example):
%       pdflatex pdfescape.dtx
%       makeindex -s gind.ist pdfescape.idx
%       pdflatex pdfescape.dtx
%       makeindex -s gind.ist pdfescape.idx
%       pdflatex pdfescape.dtx
%
% Installation:
%    TDS:tex/generic/oberdiek/pdfescape.sty
%    TDS:doc/latex/oberdiek/pdfescape.pdf
%    TDS:doc/latex/oberdiek/test/pdfescape-test1.tex
%    TDS:doc/latex/oberdiek/test/pdfescape-test2.tex
%    TDS:doc/latex/oberdiek/test/pdfescape-test3.tex
%    TDS:doc/latex/oberdiek/test/pdfescape-test4.tex
%    TDS:doc/latex/oberdiek/test/pdfescape-test5.tex
%    TDS:doc/latex/oberdiek/test/pdfescape-test6.tex
%    TDS:source/latex/oberdiek/pdfescape.dtx
%
%<*ignore>
\begingroup
  \catcode123=1 %
  \catcode125=2 %
  \def\x{LaTeX2e}%
\expandafter\endgroup
\ifcase 0\ifx\install y1\fi\expandafter
         \ifx\csname processbatchFile\endcsname\relax\else1\fi
         \ifx\fmtname\x\else 1\fi\relax
\else\csname fi\endcsname
%</ignore>
%<*install>
\input docstrip.tex
\Msg{************************************************************************}
\Msg{* Installation}
\Msg{* Package: pdfescape 2011/11/25 v1.13 Implements pdfTeX's escape features (HO)}
\Msg{************************************************************************}

\keepsilent
\askforoverwritefalse

\let\MetaPrefix\relax
\preamble

This is a generated file.

Project: pdfescape
Version: 2011/11/25 v1.13

Copyright (C) 2007, 2010, 2011 by
   Heiko Oberdiek <heiko.oberdiek at googlemail.com>

This work may be distributed and/or modified under the
conditions of the LaTeX Project Public License, either
version 1.3c of this license or (at your option) any later
version. This version of this license is in
   http://www.latex-project.org/lppl/lppl-1-3c.txt
and the latest version of this license is in
   http://www.latex-project.org/lppl.txt
and version 1.3 or later is part of all distributions of
LaTeX version 2005/12/01 or later.

This work has the LPPL maintenance status "maintained".

This Current Maintainer of this work is Heiko Oberdiek.

The Base Interpreter refers to any `TeX-Format',
because some files are installed in TDS:tex/generic//.

This work consists of the main source file pdfescape.dtx
and the derived files
   pdfescape.sty, pdfescape.pdf, pdfescape.ins, pdfescape.drv,
   pdfescape-test1.tex, pdfescape-test2.tex,
   pdfescape-test3.tex, pdfescape-test4.tex,
   pdfescape-test5.tex, pdfescape-test6.tex.

\endpreamble
\let\MetaPrefix\DoubleperCent

\generate{%
  \file{pdfescape.ins}{\from{pdfescape.dtx}{install}}%
  \file{pdfescape.drv}{\from{pdfescape.dtx}{driver}}%
  \usedir{tex/generic/oberdiek}%
  \file{pdfescape.sty}{\from{pdfescape.dtx}{package}}%
  \usedir{doc/latex/oberdiek/test}%
  \file{pdfescape-test1.tex}{\from{pdfescape.dtx}{test1}}%
  \file{pdfescape-test2.tex}{\from{pdfescape.dtx}{test2}}%
  \file{pdfescape-test3.tex}{\from{pdfescape.dtx}{test3}}%
  \file{pdfescape-test4.tex}{\from{pdfescape.dtx}{test4}}%
  \file{pdfescape-test5.tex}{\from{pdfescape.dtx}{test5}}%
  \file{pdfescape-test6.tex}{\from{pdfescape.dtx}{test6}}%
  \nopreamble
  \nopostamble
  \usedir{source/latex/oberdiek/catalogue}%
  \file{pdfescape.xml}{\from{pdfescape.dtx}{catalogue}}%
}

\catcode32=13\relax% active space
\let =\space%
\Msg{************************************************************************}
\Msg{*}
\Msg{* To finish the installation you have to move the following}
\Msg{* file into a directory searched by TeX:}
\Msg{*}
\Msg{*     pdfescape.sty}
\Msg{*}
\Msg{* To produce the documentation run the file `pdfescape.drv'}
\Msg{* through LaTeX.}
\Msg{*}
\Msg{* Happy TeXing!}
\Msg{*}
\Msg{************************************************************************}

\endbatchfile
%</install>
%<*ignore>
\fi
%</ignore>
%<*driver>
\NeedsTeXFormat{LaTeX2e}
\ProvidesFile{pdfescape.drv}%
  [2011/11/25 v1.13 Implements pdfTeX's escape features (HO)]%
\documentclass{ltxdoc}
\usepackage{holtxdoc}[2011/11/22]
\begin{document}
  \DocInput{pdfescape.dtx}%
\end{document}
%</driver>
% \fi
%
% \CheckSum{2516}
%
% \CharacterTable
%  {Upper-case    \A\B\C\D\E\F\G\H\I\J\K\L\M\N\O\P\Q\R\S\T\U\V\W\X\Y\Z
%   Lower-case    \a\b\c\d\e\f\g\h\i\j\k\l\m\n\o\p\q\r\s\t\u\v\w\x\y\z
%   Digits        \0\1\2\3\4\5\6\7\8\9
%   Exclamation   \!     Double quote  \"     Hash (number) \#
%   Dollar        \$     Percent       \%     Ampersand     \&
%   Acute accent  \'     Left paren    \(     Right paren   \)
%   Asterisk      \*     Plus          \+     Comma         \,
%   Minus         \-     Point         \.     Solidus       \/
%   Colon         \:     Semicolon     \;     Less than     \<
%   Equals        \=     Greater than  \>     Question mark \?
%   Commercial at \@     Left bracket  \[     Backslash     \\
%   Right bracket \]     Circumflex    \^     Underscore    \_
%   Grave accent  \`     Left brace    \{     Vertical bar  \|
%   Right brace   \}     Tilde         \~}
%
% \GetFileInfo{pdfescape.drv}
%
% \title{The \xpackage{pdfescape} package}
% \date{2011/11/25 v1.13}
% \author{Heiko Oberdiek\\\xemail{heiko.oberdiek at googlemail.com}}
%
% \maketitle
%
% \begin{abstract}
% This package implements \pdfTeX's escape features
% (\cs{pdfescapehex}, \cs{pdfunescapehex}, \cs{pdfescapename},
% \cs{pdfescapestring}) using \TeX\ or \eTeX.
% \end{abstract}
%
% \tableofcontents
%
% \section{Documentation}
%
% \begin{declcs}{EdefEscapeHex} \M{cmd} \M{string}\\
%  \cs{EdefUnescapeHex} \M{cmd} \M{string}\\
%  \cs{EdefEscapeName} \M{cmd} \M{string}\\
%  \cs{EdefEscapeString} \M{cmd} \M{string}
% \end{declcs}
% These commands converts \meta{string} and stores the result
% in macro \meta{cmd}. The conversion result is the same
% as the conversion of the corresponding \pdfTeX's primitives.
% Note that the argument \meta{string} is expanded before
% the conversion.
%
% For example, if \pdfTeX\ \textgreater= 1.30 is present,
% then \cs{EdefEscapeHex}
% becomes to:
% \begin{quote}
%\begin{verbatim}
%\def\EdefEscapeHex#1#2{%
%  \edef#1{\pdfescapehex{#2}}%
%}
%\end{verbatim}
% \end{quote}
% The package provides implementations for the case that \pdfTeX\ is
% not present (or too old). Even \eTeX\ can be missing, however it is
% used if it is detected.
%
% \paragraph{Babel.}
% The input strings may contain shorthand characters of package
% \xpackage{babel}.
%
% \subsection{Additional unescape macros}
%
% \begin{declcs}{EdefUnescapeName} \M{cmd} \M{string}
% \end{declcs}
% Sequences of a hash sign with two hexadecimal digits are
% converted to the corresponding character (PDF-1.2).
% A hash sign that is not followed by two hexadecimal digits
% is left unchanged. The catcodes in the result string follow
% TeX's conventions. The space has catcode 10 (space) and
% the other characters have catcode 12 (other).
%
% \begin{declcs}{EdefUnescapeString} \M{cmd} \M{string}
% \end{declcs}
% Macro \meta{cmd} stores the unescaped string in \meta{string}.
% All the rules for literal strings are implemented, see
% PDF specification.
% The catcodes in the result string follow TeX's conventions.
%
% \subsection{Sanitizing macro}
%
% \begin{declcs}{EdefSanitize} \M{cmd} \M{string}
% \end{declcs}
% Argument \meta{string} is expanded, converted to a string
% of tokens with catcode 12 (other) and space tokens, and
% stored in macro \meta{cmd}.
%
% \StopEventually{
% }
%
% \section{Implementation}
%
%    \begin{macrocode}
%<*package>
%    \end{macrocode}
%
% \subsection{Reload check and package identification}
%    Reload check, especially if the package is not used with \LaTeX.
%    \begin{macrocode}
\begingroup\catcode61\catcode48\catcode32=10\relax%
  \catcode13=5 % ^^M
  \endlinechar=13 %
  \catcode35=6 % #
  \catcode39=12 % '
  \catcode44=12 % ,
  \catcode45=12 % -
  \catcode46=12 % .
  \catcode58=12 % :
  \catcode64=11 % @
  \catcode123=1 % {
  \catcode125=2 % }
  \expandafter\let\expandafter\x\csname ver@pdfescape.sty\endcsname
  \ifx\x\relax % plain-TeX, first loading
  \else
    \def\empty{}%
    \ifx\x\empty % LaTeX, first loading,
      % variable is initialized, but \ProvidesPackage not yet seen
    \else
      \expandafter\ifx\csname PackageInfo\endcsname\relax
        \def\x#1#2{%
          \immediate\write-1{Package #1 Info: #2.}%
        }%
      \else
        \def\x#1#2{\PackageInfo{#1}{#2, stopped}}%
      \fi
      \x{pdfescape}{The package is already loaded}%
      \aftergroup\endinput
    \fi
  \fi
\endgroup%
%    \end{macrocode}
%    Package identification:
%    \begin{macrocode}
\begingroup\catcode61\catcode48\catcode32=10\relax%
  \catcode13=5 % ^^M
  \endlinechar=13 %
  \catcode35=6 % #
  \catcode39=12 % '
  \catcode40=12 % (
  \catcode41=12 % )
  \catcode44=12 % ,
  \catcode45=12 % -
  \catcode46=12 % .
  \catcode47=12 % /
  \catcode58=12 % :
  \catcode64=11 % @
  \catcode91=12 % [
  \catcode93=12 % ]
  \catcode123=1 % {
  \catcode125=2 % }
  \expandafter\ifx\csname ProvidesPackage\endcsname\relax
    \def\x#1#2#3[#4]{\endgroup
      \immediate\write-1{Package: #3 #4}%
      \xdef#1{#4}%
    }%
  \else
    \def\x#1#2[#3]{\endgroup
      #2[{#3}]%
      \ifx#1\@undefined
        \xdef#1{#3}%
      \fi
      \ifx#1\relax
        \xdef#1{#3}%
      \fi
    }%
  \fi
\expandafter\x\csname ver@pdfescape.sty\endcsname
\ProvidesPackage{pdfescape}%
  [2011/11/25 v1.13 Implements pdfTeX's escape features (HO)]%
%    \end{macrocode}
%
% \subsection{Catcodes}
%
%    \begin{macrocode}
\begingroup\catcode61\catcode48\catcode32=10\relax%
  \catcode13=5 % ^^M
  \endlinechar=13 %
  \catcode123 1 % {
  \catcode125 2 % }
  \catcode64 11 %
  \def\x{\endgroup
    \expandafter\edef\csname PE@AtEnd\endcsname{%
      \endlinechar=\the\endlinechar\relax
      \catcode13=\the\catcode13\relax
      \catcode32=\the\catcode32\relax
      \catcode35=\the\catcode35\relax
      \catcode61=\the\catcode61\relax
      \catcode64=\the\catcode64\relax
      \catcode123=\the\catcode123\relax
      \catcode125=\the\catcode125\relax
    }%
  }%
\x\catcode61\catcode48\catcode32=10\relax%
\catcode13=5 % ^^M
\endlinechar=13 %
\catcode35=6 % #
\catcode64=11 % @
\catcode123=1 % {
\catcode125=2 % }
\def\TMP@EnsureCode#1#2#3{%
  \edef\PE@AtEnd{%
    \PE@AtEnd
    #1#2=\the#1#2\relax
  }%
  #1#2=#3\relax
}
\TMP@EnsureCode\catcode{0}{12}% ^^@
\TMP@EnsureCode\catcode{34}{12}% "
\TMP@EnsureCode\catcode{36}{3}% $
\TMP@EnsureCode\catcode{38}{4}% &
\TMP@EnsureCode\catcode{39}{12}% '
\TMP@EnsureCode\catcode{42}{12}% *
\TMP@EnsureCode\catcode{45}{12}% -
\TMP@EnsureCode\catcode{46}{12}% .
\TMP@EnsureCode\catcode{47}{12}% /
\TMP@EnsureCode\catcode{60}{12}% <
\TMP@EnsureCode\catcode{62}{12}% >
\TMP@EnsureCode\catcode{91}{12}% [
\TMP@EnsureCode\catcode{93}{12}% ]
\TMP@EnsureCode\catcode{94}{7}% ^
\TMP@EnsureCode\catcode{96}{12}% `
\TMP@EnsureCode\uccode{34}{0}% "
\TMP@EnsureCode\uccode{48}{0}% 0
\TMP@EnsureCode\uccode{61}{0}% =
\edef\PE@AtEnd{\PE@AtEnd\noexpand\endinput}
%    \end{macrocode}
%
% \subsection{Load package}
%
%    \begin{macrocode}
\begingroup\expandafter\expandafter\expandafter\endgroup
\expandafter\ifx\csname RequirePackage\endcsname\relax
  \def\TMP@RequirePackage#1[#2]{%
    \begingroup\expandafter\expandafter\expandafter\endgroup
    \expandafter\ifx\csname ver@#1.sty\endcsname\relax
      \input #1.sty\relax
    \fi
  }%
  \TMP@RequirePackage{ltxcmds}[2010/04/08]%
\else
  \RequirePackage{ltxcmds}[2010/04/08]%
\fi
%    \end{macrocode}
%
% \subsection{Sanitizing}
%
%    \begin{macro}{\EdefSanitize}
%    Macro \cs{EdefSanitize} takes |#2|, entirely converts
%    it to token with catcode 12 (other) and stores the
%    result in macro |#1|.
%    \begin{macrocode}
\begingroup\expandafter\expandafter\expandafter\endgroup
\expandafter\ifx\csname detokenize\endcsname\relax
  \long\def\EdefSanitize#1#2{%
    \begingroup
      \csname @safe@activestrue\endcsname
      \edef#1{#2}%
      \PE@onelevel@sanitize#1%
    \expandafter\endgroup
    \expandafter\def\expandafter#1\expandafter{#1}%
  }%
  \begingroup\expandafter\expandafter\expandafter\endgroup
  \expandafter\ifx\csname @onelevel@sanitize\endcsname\relax
    \def\PE@onelevel@sanitize#1{%
      \edef#1{\expandafter\PE@strip@prefix\meaning#1}%
    }%
    \def\PE@strip@prefix#1>{}%
  \else
    \let\PE@onelevel@sanitize\@onelevel@sanitize
  \fi
\else
  \long\def\EdefSanitize#1#2{%
    \begingroup
      \csname @safe@activestrue\endcsname
      \edef#1{#2}%
    \expandafter\endgroup
    \expandafter\def\expandafter#1\expandafter{%
      \detokenize\expandafter{#1}%
    }%
  }%
  \def\PE@onelevel@sanitize#1{%
    \edef#1{\detokenize\expandafter{#1}}%
  }%
\fi
%    \end{macrocode}
%    \end{macro}
%    \begin{macro}{\PE@sanitize}
%    Macro \cs{PE@sanitize} is only defined for
%    compatibility with version 1.4. Its use is deprecated.
%    \begin{macrocode}
\let\PE@sanitize\EdefSanitize
%    \end{macrocode}
%    \end{macro}
%
% \subsubsection{Space characters}
%
%    \begin{macro}{\PE@space@other}
%    \begin{macrocode}
\begingroup
  \catcode`\ =12\relax%
\def\x{\endgroup\def\PE@space@other{ }}\x\relax
%    \end{macrocode}
%    \end{macro}
%    \begin{macro}{\PE@space@space}
%    \begin{macrocode}
\def\PE@space@space{ }
%    \end{macrocode}
%    \end{macro}
%
% \subsubsection{Space normalization}
%
%    \begin{macro}{\PE@SanitizeSpaceOther}
%    \begin{macrocode}
\def\PE@SanitizeSpaceOther#1{%
  \edef#1{\expandafter\PE@SpaceToOther#1 \relax}%
}
%    \end{macrocode}
%    \end{macro}
%
%    \begin{macro}{\PE@SpaceToOther}
%    \begin{macrocode}
\def\PE@SpaceToOther#1 #2\relax{%
  #1%
  \ifx\\#2\\%
  \else
    \PE@space@other
    \ltx@ReturnAfterFi{%
      \PE@SpaceToOther#2\relax
    }%
  \fi
}
%    \end{macrocode}
%    \end{macro}
%
% \subsection{\cs{EdefUnescapeName}}
%
%    \begin{macro}{\EdefUnescapeName}
%    \begin{macrocode}
\def\EdefUnescapeName#1#2{%
  \EdefSanitize#1{#2}%
  \PE@SanitizeSpaceOther#1%
  \PE@UnescapeName#1%
  \PE@onelevel@sanitize#1%
}
%    \end{macrocode}
%    \end{macro}
%
%    \begin{macro}{\PE@UnescapeName}
%    \begin{macrocode}
\begingroup
  \catcode`\$=6  % hash
  \catcode`\#=12 % other
  \gdef\PE@UnescapeName$1{%
    \begingroup
      \PE@InitUccodeHexDigit
      \def\PE@result{}%
      \expandafter\PE@DeName$1#\relax\relax
    \expandafter\endgroup
    \expandafter\def\expandafter$1\expandafter{\PE@result}%
  }%
  \gdef\PE@DeName$1#$2$3{%
    \ifx\relax$2%
      \edef\PE@result{\PE@result$1}%
      \let\PE@next\relax
    \else
      \ifx\relax$3%
        % wrong escape sequence in input
        \edef\PE@result{\PE@result$1#}%
        \let\PE@next\relax
      \else
        \uppercase{%
          \def\PE@testA{$2}%
          \def\PE@testB{$3}%
        }%
        \ifcase\ifcase\expandafter\PE@TestUcHexDigit\PE@testA
                 \ifcase\expandafter\PE@TestUcHexDigit\PE@testB
                   \ltx@zero
                 \else
                   \ltx@one
                 \fi
               \else
                 \ltx@one
               \fi
          \uccode\ltx@zero="\PE@testA\PE@testB\relax
          \uppercase{%
            \def\PE@temp{^^@}%
          }%
          \uccode\ltx@zero=\ltx@zero
          \edef\PE@result{\PE@result$1\PE@temp}%
          \let\PE@next\PE@DeName
        \else
          % wrong escape sequence in input
          \edef\PE@result{\PE@result$1#}%
          \def\PE@next{\PE@DeName$2$3}%
        \fi
      \fi
    \fi
    \PE@next
  }%
\endgroup
%    \end{macrocode}
%    \end{macro}
%
%    \begin{macro}{\PE@InitUccodeHexDigit}
%    \begin{macrocode}
\def\PE@InitUccodeHexDigit{%
  \uccode`a=`A\relax
  \uccode`b=`B\relax
  \uccode`c=`C\relax
  \uccode`d=`D\relax
  \uccode`e=`E\relax
  \uccode`f=`F\relax
  \uccode`A=\ltx@zero
  \uccode`B=\ltx@zero
  \uccode`C=\ltx@zero
  \uccode`D=\ltx@zero
  \uccode`E=\ltx@zero
  \uccode`F=\ltx@zero
  \uccode`0=\ltx@zero
  \uccode`1=\ltx@zero
  \uccode`2=\ltx@zero
  \uccode`3=\ltx@zero
  \uccode`4=\ltx@zero
  \uccode`5=\ltx@zero
  \uccode`6=\ltx@zero
  \uccode`7=\ltx@zero
  \uccode`8=\ltx@zero
  \uccode`9=\ltx@zero
}
%    \end{macrocode}
%    \end{macro}
%
%    \begin{macro}{\PE@TestUcHexDigit}
%    \begin{macrocode}
\def\PE@TestUcHexDigit#1{%
  \ifnum`#1<48 % 0
    \ltx@one
  \else
    \ifnum`#1>70 % F
      \ltx@one
    \else
      \ifnum`#1>57 % 9
        \ifnum`#1<65 % A
          \ltx@one
        \else
          \ltx@zero
        \fi
      \else
        \ltx@zero
      \fi
    \fi
  \fi
}
%    \end{macrocode}
%    \end{macro}
%
% \subsection{\cs{EdefUnescapeString}}
%
%    \begin{macro}{\EdefUnescapeString}
%    \begin{macrocode}
\def\EdefUnescapeString#1#2{%
  \EdefSanitize#1{#2}%
  \PE@SanitizeSpaceOther#1%
  \PE@NormalizeLineEnd#1%
  \PE@UnescapeString#1%
  \PE@onelevel@sanitize#1%
}
%    \end{macrocode}
%    \end{macro}
%
%    \begin{macrocode}
\begingroup
  \uccode`\8=10 % lf
  \uccode`\9=13 % cr
\def\x#1#2{\endgroup
%    \end{macrocode}
%    \begin{macro}{\PE@NormalizeLineEnd}
%    \begin{macrocode}
  \def\PE@NormalizeLineEnd##1{%
    \def\PE@result{}%
    \expandafter\PE@@NormalizeLineEnd##1#2\relax
    \let##1\PE@result
  }%
%    \end{macrocode}
%    \end{macro}
%    \begin{macro}{\PE@@NormalizeLineEnd}
%    \begin{macrocode}
  \def\PE@@NormalizeLineEnd##1#2##2{%
    \ifx\relax##2%
      \edef\PE@result{\PE@result##1}%
      \let\PE@next\relax
    \else
      \edef\PE@result{\PE@result##1#1}%
      \ifx#1##2% lf
        \let\PE@next\PE@@NormalizeLineEnd
      \else
        \def\PE@next{\PE@@NormalizeLineEnd##2}%
      \fi
    \fi
    \PE@next
  }%
}%
\uppercase{%
  \x 89%
}
%    \end{macrocode}
%    \end{macro}
%
%    \begin{macrocode}
\begingroup
  \catcode`\|=0 %
  \catcode`\\=12 %
%    \end{macrocode}
%    \begin{macro}{\PE@UnescapeString}
%    \begin{macrocode}
  |gdef|PE@UnescapeString#1{%
    |begingroup
      |def|PE@result{}%
      |expandafter|PE@DeString#1\|relax
    |expandafter|endgroup
    |expandafter|def|expandafter#1|expandafter{|PE@result}%
  }%
%    \end{macrocode}
%    \end{macro}
%    \begin{macro}{\PE@DeString}
%    \begin{macrocode}
  |gdef|PE@DeString#1\#2{%
    |ifx|relax#2%
      |edef|PE@result{|PE@result#1}%
      |let|PE@next|relax
    |else
      |if n#2%
        |uccode|ltx@zero=10 %
      |else|if r#2%
        |uccode|ltx@zero=13 %
      |else|if t#2%
        |uccode|ltx@zero=9 %
      |else|if b#2%
        |uccode|ltx@zero=8 %
      |else|if f#2%
        |uccode|ltx@zero=12 %
      |else
        |uccode|ltx@zero=|ltx@zero
      |fi|fi|fi|fi|fi
      |ifnum|uccode|ltx@zero>|ltx@zero
        |uppercase{%
          |edef|PE@temp{^^@}%
        }%
        |edef|PE@result{|PE@result#1|PE@temp}%
        |let|PE@next|PE@DeString
      |else
        |if\#2% backslash
          |edef|PE@result{|PE@result#1}%
          |let|PE@next|PE@CheckEndBackslash
        |else
          |ifnum`#2=10 % linefeed
            |edef|PE@result{|PE@result#1}%
            |let|PE@next|PE@DeString
          |else
            |ifcase|PE@TestOctDigit#2%
              |edef|PE@result{|PE@result#1}%
              |def|PE@next{|PE@OctI#2}%
            |else
              |edef|PE@result{|PE@result#1#2}%
              |let|PE@next|PE@DeString
            |fi
          |fi
        |fi
      |fi
    |fi
    |PE@next
  }%
%    \end{macrocode}
%    \end{macro}
%    \begin{macro}{\PE@CheckEndBackslash}
%    \begin{macrocode}
  |gdef|PE@CheckEndBackslash#1{%
    |ifx|relax#1%
    |else
      |edef|PE@result{|PE@result\}%
      |expandafter|PE@DeString|expandafter#1%
    |fi
  }%
%    \end{macrocode}
%    \end{macro}
%    \begin{macrocode}
|endgroup
%    \end{macrocode}
%    \begin{macro}{\PE@TestOctDigit}
%    \begin{macrocode}
\def\PE@TestOctDigit#1{%
  \ifnum`#1<48 % 0
    \ltx@one
  \else
    \ifnum`#1>55 % 7
      \ltx@one
    \else
      \ltx@zero
    \fi
  \fi
}
%    \end{macrocode}
%    \end{macro}
%    \begin{macro}{\PE@OctI}
%    \begin{macrocode}
\def\PE@OctI#1#2{%
  \ifcase\PE@TestOctDigit#2%
    \def\PE@next{\PE@OctII{#1#2}}%
  \else
    \def\PE@next{\PE@OctAll#1#2}%
  \fi
  \PE@next
}
%    \end{macrocode}
%    \end{macro}
%    \begin{macro}{\PE@OctII}
%    \begin{macrocode}
\def\PE@OctII#1#2{%
  \ifcase\PE@TestOctDigit#2%
    \def\PE@next{\PE@OctIII#1#2}%
  \else
    \def\PE@next{\PE@OctAll{#1}#2}%
  \fi
  \PE@next
}
%    \end{macrocode}
%    \end{macro}
%    \begin{macrocode}
\ltx@IfUndefined{numexpr}{%
  \catcode`\$=9 %
  \catcode`\&=14 %
}{%
  \catcode`\$=14 %
  \catcode`\&=9 %
}
%    \end{macrocode}
%    \begin{macro}{\PE@OctIII}
%    \begin{macrocode}
\def\PE@OctIII#1#2#3{%
  \ifnum#1<4 %
    \def\PE@next{\PE@OctAll{#1#2#3}}%
  \else
$   \count\ltx@cclv#1 %
$   \advance\count\ltx@cclv -4 %
    \edef\PE@next{%
      \noexpand\PE@OctAll{%
$       \the\count\ltx@cclv
&       \the\numexpr#1-4\relax
        #2#3%
      }%
    }%
  \fi
  \PE@next
}
%    \end{macrocode}
%    \end{macro}
%    \begin{macro}{\PE@OctAll}
%    \begin{macrocode}
\def\PE@OctAll#1{%
  \uccode\ltx@zero='#1\relax
  \uppercase{%
    \edef\PE@result{\PE@result^^@}%
  }%
  \PE@DeString
}
%    \end{macrocode}
%    \end{macro}
%
% \subsection{User macros (\pdfTeX\space analogues)}
%
%    \begin{macrocode}
\begingroup\expandafter\expandafter\expandafter\endgroup
\expandafter\ifx\csname RequirePackage\endcsname\relax
  \def\TMP@RequirePackage#1[#2]{%
    \begingroup\expandafter\expandafter\expandafter\endgroup
    \expandafter\ifx\csname ver@#1.sty\endcsname\relax
      \input #1.sty\relax
    \fi
  }%
  \TMP@RequirePackage{pdftexcmds}[2007/11/11]%
\else
  \RequirePackage{pdftexcmds}[2007/11/11]%
\fi
%    \end{macrocode}
%
%    \begin{macrocode}
\begingroup\expandafter\expandafter\expandafter\endgroup
\expandafter\ifx\csname pdf@escapehex\endcsname\relax
%    \end{macrocode}
%
%    \begin{macro}{\EdefEscapeHex}
%    \begin{macrocode}
  \long\def\EdefEscapeHex#1#2{%
    \EdefSanitize#1{#2}%
    \PE@SanitizeSpaceOther#1%
    \PE@EscapeHex#1%
  }%
%    \end{macrocode}
%    \end{macro}
%    \begin{macro}{\EdefUnescapeHex}
%    \begin{macrocode}
  \def\EdefUnescapeHex#1#2{%
    \EdefSanitize#1{#2}%
    \PE@UnescapeHex#1%
  }%
%    \end{macrocode}
%    \end{macro}
%    \begin{macro}{\EdefEscapeName}
%    \begin{macrocode}
  \long\def\EdefEscapeName#1#2{%
    \EdefSanitize#1{#2}%
    \PE@SanitizeSpaceOther#1%
    \PE@EscapeName#1%
  }%
%    \end{macrocode}
%    \end{macro}
%    \begin{macro}{\EdefEscapeString}
%    \begin{macrocode}
  \long\def\EdefEscapeString#1#2{%
    \EdefSanitize#1{#2}%
    \PE@SanitizeSpaceOther#1%
    \PE@EscapeString#1%
  }%
%    \end{macrocode}
%    \end{macro}
%
%    \begin{macrocode}
\else
%    \end{macrocode}
%
%    \begin{macro}{\PE@edefbabel}
%    Help macro that adds support for \xpackage{babel}'s shorthand
%    characters.
%    \begin{macrocode}
  \long\def\PE@edefbabel#1#2#3{%
    \begingroup
      \csname @save@activestrue\endcsname
      \edef#1{#2{#3}}%
    \expandafter\endgroup
    \expandafter\def\expandafter#1\expandafter{#1}%
  }%
%    \end{macrocode}
%    \end{macro}
%
%    \begin{macro}{\EdefEscapeHex}
%    \begin{macrocode}
  \long\def\EdefEscapeHex#1#2{%
    \PE@edefbabel#1\pdf@escapehex{#2}%
  }%
%    \end{macrocode}
%    \end{macro}
%    \begin{macro}{\EdefUnescapeHex}
%    \begin{macrocode}
  \def\EdefUnescapeHex#1#2{%
    \PE@edefbabel#1\pdf@unescapehex{#2}%
  }%
%    \end{macrocode}
%    \end{macro}
%    \begin{macro}{\EdefEscapeName}
%    \begin{macrocode}
  \long\def\EdefEscapeName#1#2{%
    \PE@edefbabel#1\pdf@escapename{#2}%
  }%
%    \end{macrocode}
%    \end{macro}
%    \begin{macro}{\EdefEscapeString}
%    \begin{macrocode}
  \long\def\EdefEscapeString#1#2{%
    \PE@edefbabel#1\pdf@escapestring{#2}%
  }%
%    \end{macrocode}
%    \end{macro}
%
%    \begin{macrocode}
  \expandafter\PE@AtEnd
\fi%
%    \end{macrocode}
%
% \subsection{Help macros}
%
% \subsubsection{Characters}
%
%    Special characters with catcode 12 (other) are created and
%    stored in macros.
%    \begin{macro}{\PE@hash}
%    \begin{macrocode}
\edef\PE@hash{\string#}
%    \end{macrocode}
%    \end{macro}
%    \begin{macro}{\PE@backslash}
%    \begin{macrocode}
\begingroup
  \escapechar=-1 %
\edef\x{\endgroup
  \def\noexpand\PE@backslash{\string\\}%
}
\x
%    \end{macrocode}
%    \end{macro}
%
% \subsubsection{Switch for \eTeX}
%
%    \begin{macrocode}
\ltx@newif\ifPE@etex
\begingroup\expandafter\expandafter\expandafter\endgroup
\expandafter\ifx\csname numexpr\endcsname\relax
\else
  \PE@etextrue
\fi
%    \end{macrocode}
%
% \subsection{Conversions}
%
% \subsubsection{Conversion to hex string}
%
%    \begin{macro}{\PE@EscapeHex}
%    \begin{macrocode}
\ifPE@etex
  \def\PE@EscapeHex#1{%
    \edef#1{\expandafter\PE@ToHex#1\relax}%
  }%
\else
  \def\PE@EscapeHex#1{%
    \def\PE@result{}%
    \expandafter\PE@ToHex#1\relax
    \let#1\PE@result
  }%
\fi
%    \end{macrocode}
%    \end{macro}
%    \begin{macro}{\PE@ToHex}
%    \begin{macrocode}
\def\PE@ToHex#1{%
  \ifx\relax#1%
  \else
    \PE@HexChar{#1}%
    \expandafter\PE@ToHex
  \fi
}%
%    \end{macrocode}
%    \end{macro}
%    \begin{macro}{\PE@HexChar}
%    \begin{macrocode}
\ifPE@etex
  \def\PE@HexChar#1{%
    \PE@HexDigit{\numexpr\dimexpr.0625\dimexpr`#1sp\relax\relax\relax}%
    \PE@HexDigit{%
      \numexpr`#1-16*\dimexpr.0625\dimexpr`#1sp\relax\relax\relax
    }%
  }%
\else
  \def\PE@HexChar#1{%
    \dimen0=`#1sp%
    \dimen2=.0625\dimen0 %
    \advance\dimen0-16\dimen2 %
    \edef\PE@result{%
      \PE@result
      \PE@HexDigit{\dimen2 }%
      \PE@HexDigit{\dimen0 }%
    }%
  }%
\fi
%    \end{macrocode}
%    \end{macro}
%    \begin{macro}{\PE@HexDigit}
%    \begin{macrocode}
\def\PE@HexDigit#1{%
  \expandafter\string
  \ifcase#1%
    0\or 1\or 2\or 3\or 4\or 5\or 6\or 7\or 8\or 9\or
    A\or B\or C\or D\or E\or F%
  \fi
}
%    \end{macrocode}
%    \end{macro}
%
% \subsubsection{Character code to octal number}
%
%    \begin{macro}{\PE@OctChar}
%    \begin{macrocode}
\ifPE@etex
  \def\PE@OctChar#1{%
    \expandafter\PE@@OctChar
        \the\numexpr\dimexpr.015625\dimexpr`#1sp\relax\relax
            \expandafter\relax
        \expandafter\relax
        \the\numexpr\dimexpr.125\dimexpr`#1sp\relax\relax\relax
        \relax
        #1%
  }%
  \def\PE@@OctChar#1\relax#2\relax#3{%
    \PE@backslash
    #1%
    \the\numexpr#2-8*#1\relax
    \the\numexpr\dimexpr`#3sp\relax-8*#2\relax
  }%
\else
  \def\PE@OctChar#1{%
    \dimen0=`#1sp%
    \dimen2=.125\dimen0 %
    \dimen4=.125\dimen2 %
    \advance\dimen0-8\dimen2 %
    \advance\dimen2-8\dimen4 %
    \edef\PE@result{%
      \PE@result
      \PE@backslash
      \number\dimen4 %
      \number\dimen2 %
      \number\dimen0 %
    }%
  }%
\fi
%    \end{macrocode}
%    \end{macro}
%
% \subsubsection{Unpack hex string}
%
%    \begin{macro}{\PE@UnescapeHex}
%    \begin{macrocode}
\def\PE@UnescapeHex#1{%
  \begingroup
    \PE@InitUccodeHexDigit
    \def\PE@result{}%
    \expandafter\PE@DeHex#1\relax\relax
  \expandafter\endgroup
  \expandafter\def\expandafter#1\expandafter{\PE@result}%
}
%    \end{macrocode}
%    \end{macro}
%    \begin{macro}{\PE@DeHex}
%    \begin{macrocode}
\def\PE@DeHex#1#2{%
  \ifx#2\relax
    \ifx#1\relax
      \let\PE@next\relax
    \else
      \uppercase{%
        \def\PE@testA{#1}%
      }%
      \ifcase\expandafter\PE@TestUcHexDigit\PE@testA
        \def\PE@next{%
          \PE@DeHex#10\relax\relax
        }%
      \else
        \let\PE@next\relax
      \fi
    \fi
  \else
    \uppercase{%
      \def\PE@testA{#1}%
      \def\PE@testB{#2}%
    }%
    \ifcase\expandafter\PE@TestUcHexDigit\PE@testA
      \ifcase\expandafter\PE@TestUcHexDigit\PE@testB
        \uccode\ltx@zero="\PE@testA\PE@testB\relax
        \ifnum\uccode\ltx@zero=32 %
          \let\PE@temp\PE@space@space
        \else
          \uppercase{%
            \def\PE@temp{^^@}%
          }%
        \fi
        \edef\PE@result{\PE@result\PE@temp}%
        \let\PE@next\PE@DeHex
      \else
        % invalid input sequence
        \def\PE@next{%
          \PE@DeHex#1%
        }%
      \fi
    \else
      % invalid input sequence
      \def\PE@next{\PE@DeHex#2}%
    \fi
  \fi
  \PE@next
}
%    \end{macrocode}
%    \end{macro}
%
% \subsubsection{Conversion to PDF name}
%
%    \begin{macro}{\PE@EscapeName}
%    \begin{macrocode}
\ifPE@etex
  \def\PE@EscapeName#1{%
    \edef#1{\expandafter\PE@EscapeNameTokens#1\relax}%
  }%
\else
  \def\PE@EscapeName#1{%
    \def\PE@result{}%
    \expandafter\PE@EscapeNameTokens#1\relax
    \let#1\PE@result
  }%
\fi
%    \end{macrocode}
%    \end{macro}
%    \begin{macro}{\PE@EscapeNameTokens}
%    \begin{macrocode}
\def\PE@EscapeNameTokens#1{%
  \ifx\relax#1%
  \else
    \ifnum`#1<33 %
      \ifcase`#1 %
        % drop illegal zero
      \else
        \PE@EscapeNameAdd\PE@hash
        \PE@HexChar#1%
      \fi
    \else
      \ifnum`#1>126 %
        \PE@EscapeNameAdd\PE@hash
        \PE@HexChar#1%
      \else  \ifnum`#1=35  \PE@EscapeNameHashChar 23% #
        \else\ifnum`#1=37  \PE@EscapeNameHashChar 25% %
        \else\ifnum`#1=40  \PE@EscapeNameHashChar 28% (
        \else\ifnum`#1=41  \PE@EscapeNameHashChar 29% )
        \else\ifnum`#1=47  \PE@EscapeNameHashChar 2F% /
        \else\ifnum`#1=60  \PE@EscapeNameHashChar 3C% <
        \else\ifnum`#1=62  \PE@EscapeNameHashChar 3E% >
        \else\ifnum`#1=91  \PE@EscapeNameHashChar 5B% [
        \else\ifnum`#1=93  \PE@EscapeNameHashChar 5D% ]
        \else\ifnum`#1=123 \PE@EscapeNameHashChar 7B% {
        \else\ifnum`#1=125 \PE@EscapeNameHashChar 7D% }
        \else
          \PE@EscapeNameAdd{#1}%
        \fi\fi\fi\fi\fi\fi\fi\fi\fi\fi\fi
      \fi
    \fi
    \expandafter\PE@EscapeNameTokens
  \fi
}%
\def\PE@EscapeNameHashChar#1#2{%
  \PE@EscapeNameAdd{\PE@hash\string#1\string#2}%
}%
%    \end{macrocode}
%    \end{macro}
%    \begin{macro}{\PE@EscapeNameAdd}
%    \begin{macrocode}
\ifPE@etex
  \def\PE@EscapeNameAdd#1{#1}%
\else
  \def\PE@EscapeNameAdd#1{%
    \edef\PE@result{%
      \PE@result
      #1%
    }%
  }%
\fi
%    \end{macrocode}
%    \end{macro}
%
% \subsubsection{Conversion to PDF string}
%
%    \begin{macro}{\PE@EscapeString}
%    \begin{macrocode}
\ifPE@etex
  \def\PE@EscapeString#1{%
    \edef#1{\expandafter\PE@EscapeStringTokens#1\relax}%
  }%
\else
  \def\PE@EscapeString#1{%
    \begingroup
      \def\PE@result{}%
      \expandafter\PE@EscapeStringTokens#1\relax
    \expandafter\endgroup
    \expandafter\def\expandafter#1\expandafter{\PE@result}%
  }%
\fi
%    \end{macrocode}
%    \end{macro}
%    \begin{macro}{\PE@EscapeStringTokens}
%    \begin{macrocode}
\def\PE@EscapeStringTokens#1{%
  \ifx\relax#1%
  \else
    \ifnum`#1<33 %
      \PE@OctChar#1%
    \else
      \ifnum`#1>126 %
        \PE@OctChar#1%
      \else  \ifnum`#1=40 \PE@EscapeStringAdd{\string\(}% (
        \else\ifnum`#1=41 \PE@EscapeStringAdd{\string\)}% )
        \else\ifnum`#1=92 \PE@EscapeStringAdd{\string\\}% \
        \else
          \PE@EscapeStringAdd{#1}%
        \fi\fi\fi
      \fi
    \fi
    \expandafter\PE@EscapeStringTokens
  \fi
}%
%    \end{macrocode}
%    \end{macro}
%    \begin{macro}{\PE@EscapeStringAdd}
%    \begin{macrocode}
\ifPE@etex
  \def\PE@EscapeStringAdd#1{#1}%
\else
  \def\PE@EscapeStringAdd#1{%
    \edef\PE@result{%
      \PE@result
      #1%
    }%
  }%
\fi
%    \end{macrocode}
%    \end{macro}
%
%    \begin{macrocode}
\PE@AtEnd%
%    \end{macrocode}
%
%    \begin{macrocode}
%</package>
%    \end{macrocode}
%
% \section{Test}
%
% \subsection{Catcode checks for loading}
%
%    \begin{macrocode}
%<*test1>
%    \end{macrocode}
%    \begin{macrocode}
\catcode`\{=1 %
\catcode`\}=2 %
\catcode`\#=6 %
\catcode`\@=11 %
\expandafter\ifx\csname count@\endcsname\relax
  \countdef\count@=255 %
\fi
\expandafter\ifx\csname @gobble\endcsname\relax
  \long\def\@gobble#1{}%
\fi
\expandafter\ifx\csname @firstofone\endcsname\relax
  \long\def\@firstofone#1{#1}%
\fi
\expandafter\ifx\csname loop\endcsname\relax
  \expandafter\@firstofone
\else
  \expandafter\@gobble
\fi
{%
  \def\loop#1\repeat{%
    \def\body{#1}%
    \iterate
  }%
  \def\iterate{%
    \body
      \let\next\iterate
    \else
      \let\next\relax
    \fi
    \next
  }%
  \let\repeat=\fi
}%
\def\RestoreCatcodes{}
\count@=0 %
\loop
  \edef\RestoreCatcodes{%
    \RestoreCatcodes
    \catcode\the\count@=\the\catcode\count@\relax
  }%
\ifnum\count@<255 %
  \advance\count@ 1 %
\repeat

\def\RangeCatcodeInvalid#1#2{%
  \count@=#1\relax
  \loop
    \catcode\count@=15 %
  \ifnum\count@<#2\relax
    \advance\count@ 1 %
  \repeat
}
\def\RangeCatcodeCheck#1#2#3{%
  \count@=#1\relax
  \loop
    \ifnum#3=\catcode\count@
    \else
      \errmessage{%
        Character \the\count@\space
        with wrong catcode \the\catcode\count@\space
        instead of \number#3%
      }%
    \fi
  \ifnum\count@<#2\relax
    \advance\count@ 1 %
  \repeat
}
\def\space{ }
\expandafter\ifx\csname LoadCommand\endcsname\relax
  \def\LoadCommand{\input pdfescape.sty\relax}%
\fi
\def\Test{%
  \RangeCatcodeInvalid{0}{47}%
  \RangeCatcodeInvalid{58}{64}%
  \RangeCatcodeInvalid{91}{96}%
  \RangeCatcodeInvalid{123}{255}%
  \catcode`\@=12 %
  \catcode`\\=0 %
  \catcode`\%=14 %
  \LoadCommand
  \RangeCatcodeCheck{0}{36}{15}%
  \RangeCatcodeCheck{37}{37}{14}%
  \RangeCatcodeCheck{38}{47}{15}%
  \RangeCatcodeCheck{48}{57}{12}%
  \RangeCatcodeCheck{58}{63}{15}%
  \RangeCatcodeCheck{64}{64}{12}%
  \RangeCatcodeCheck{65}{90}{11}%
  \RangeCatcodeCheck{91}{91}{15}%
  \RangeCatcodeCheck{92}{92}{0}%
  \RangeCatcodeCheck{93}{96}{15}%
  \RangeCatcodeCheck{97}{122}{11}%
  \RangeCatcodeCheck{123}{255}{15}%
  \RestoreCatcodes
}
\Test
\csname @@end\endcsname
\end
%    \end{macrocode}
%    \begin{macrocode}
%</test1>
%    \end{macrocode}
%
% \subsection{Macro tests}
%
%    \begin{macrocode}
%<*test2|test3|test4|test5>
\NeedsTeXFormat{LaTeX2e}
\makeatletter
%    \end{macrocode}
%
% \subsection{Test with \cs{pdfescape...} commands}
%
%    \begin{macrocode}
%<*test2>
\ProvidesFile{pdfescape-test2.tex}%
    [2011/11/25 v1.13 Test with \string\pdfescape... commands]%
%</test2>
%    \end{macrocode}
%
% \subsection{Test without \cs{pdfescape...}, with \eTeX}
%
%    \begin{macrocode}
%<*test3>
\ProvidesFile{pdfescape-test3.tex}%
    [2011/11/25 v1.13 Test without \string\pdfescape..., with e-TeX]%
%</test3>
%    \end{macrocode}
%
% \subsection{Test without \cs{pdfescape...} and \eTeX}
%
%    \begin{macrocode}
%<*test4>
\ProvidesFile{pdfescape-test4.tex}%
    [2011/11/25 v1.13 Test without \string\pdfescape... and e-TeX]%
%</test4>
%    \end{macrocode}
%
% \subsection{Test with \LuaTeX}
%
%    \begin{macrocode}
%<*test5>
\ProvidesFile{pdfescape-test5.tex}%
    [2011/11/25 v1.13 Test with LuaTeX]%
%</test5>
%    \end{macrocode}
%
% \subsection{Check/ensure test preconditions}
%
% \subsubsection{Check \cs{pdfescape...}}
%
%    \begin{macrocode}
%<*test2>
\@ifundefined{pdfescapehex}{%
  \PackageError{pdfescape-test2}{%
    Missing \string\pdfescape... commands%
  }{Test aborted.}%
  \stop
}{}
%</test2>
%    \end{macrocode}
%    \begin{macrocode}
%<*test3|test4>
\let\pdfescapehex\@undefined
\let\pdfunescapehex\@undefined
\let\pdfescapename\@undefined
\let\pdfescapestring\@undefined
%</test3|test4>
%    \end{macrocode}
%
% \subsubsection{Check \eTeX}
%
%    \begin{macrocode}
%<*test3>
\@ifundefined{numexpr}{%
  \PackageError{pdfescape-test3}{%
    Missing \eTeX
  }{Test aborted.}%
  \stop
}{}
%</test3>
%    \end{macrocode}
% Package \xpackage{qstest} uses \eTeX, thus \eTeX's features
% can only be disabled later during loading of package \xpackage{pdfescape}.
%
% \subsubsection{Check \LuaTeX}
%
%    \begin{macrocode}
%<*test5>
\@ifundefined{directlua}{%
  \PackageError{pdfescape-test5}{%
    Missing LuaTeX%
  }{Test aborted.}%
  \stop
}{}
%</test5>
%    \end{macrocode}
%
% \subsection{Common part}
%
%    The files for testing uses the framework, provided by the
%    new package \xpackage{qstest} of David Kastrup.
%    \begin{macrocode}
\RequirePackage{qstest}
\IncludeTests{*}
\LogTests{log}{*}{*}

\newcommand*{\ExpectVar}[2]{%
  \ifx#1#2%
  \else
    \begingroup
      \@onelevel@sanitize#1%
      \@onelevel@sanitize#2%
      \typeout{[#1] <> [#2]}% hash-ok
    \endgroup
  \fi
  \Expect*{\ifx#1#2true\else false\fi}{true}%
}

\makeatletter
\begingroup
  \gdef\AllBytes{}%
  \count@=0 %
  \catcode0=12 %
  \@whilenum\count@<256 \do{%
    \lccode0=\count@
    \ifnum\count@=32 %
      \xdef\AllBytes{\AllBytes\space}%
    \else
      \lowercase{%
        \xdef\AllBytes{\AllBytes^^@}%
      }%
    \fi
    \advance\count@ by 1 %
  }%
\endgroup
\newcommand*{\AllBytesHex}{%
  000102030405060708090A0B0C0D0E0F%
  101112131415161718191A1B1C1D1E1F%
  202122232425262728292A2B2C2D2E2F%
  303132333435363738393A3B3C3D3E3F%
  404142434445464748494A4B4C4D4E4F%
  505152535455565758595A5B5C5D5E5F%
  606162636465666768696A6B6C6D6E6F%
  707172737475767778797A7B7C7D7E7F%
  808182838485868788898A8B8C8D8E8F%
  909192939495969798999A9B9C9D9E9F%
  A0A1A2A3A4A5A6A7A8A9AAABACADAEAF%
  B0B1B2B3B4B5B6B7B8B9BABBBCBDBEBF%
  C0C1C2C3C4C5C6C7C8C9CACBCCCDCECF%
  D0D1D2D3D4D5D6D7D8D9DADBDCDDDEDF%
  E0E1E2E3E4E5E6E7E8E9EAEBECEDEEEF%
  F0F1F2F3F4F5F6F7F8F9FAFBFCFDFEFF%
}
\@onelevel@sanitize\AllBytesHex
\expandafter\lowercase\expandafter{%
  \expandafter\newcommand\expandafter*\expandafter\AllBytesHexLC
      \expandafter{\AllBytesHex}%
}
\newcommand*{\AllBytesName}{}
\begingroup
  \catcode`\#=12 %
  \xdef\AllBytesName{%
    #01#02#03#04#05#06#07#08#09#0A#0B#0C#0D#0E#0F%
    #10#11#12#13#14#15#16#17#18#19#1A#1B#1C#1D#1E#1F%
    #20!"#23$#25&'#28#29*+,-.#2F%
    0123456789:;#3C=#3E?%
    @ABCDEFGHIJKLMNO%
    PQRSTUVWXYZ#5B\@backslashchar#5D^_%
    `abcdefghijklmno%
    pqrstuvwxyz#7B|#7D\string~#7F%
    #80#81#82#83#84#85#86#87#88#89#8A#8B#8C#8D#8E#8F%
    #90#91#92#93#94#95#96#97#98#99#9A#9B#9C#9D#9E#9F%
    #A0#A1#A2#A3#A4#A5#A6#A7#A8#A9#AA#AB#AC#AD#AE#AF%
    #B0#B1#B2#B3#B4#B5#B6#B7#B8#B9#BA#BB#BC#BD#BE#BF%
    #C0#C1#C2#C3#C4#C5#C6#C7#C8#C9#CA#CB#CC#CD#CE#CF%
    #D0#D1#D2#D3#D4#D5#D6#D7#D8#D9#DA#DB#DC#DD#DE#DF%
    #E0#E1#E2#E3#E4#E5#E6#E7#E8#E9#EA#EB#EC#ED#EE#EF%
    #F0#F1#F2#F3#F4#F5#F6#F7#F8#F9#FA#FB#FC#FD#FE#FF%
  }%
\endgroup
\@onelevel@sanitize\AllBytesName

\newcommand*{\AllBytesString}{}
\begingroup
  \def\|{|}%
  \edef\%{\@percentchar}%
  \catcode`\|=0 %
  \catcode`\#=12 %
  \catcode`\~=12 %
  \catcode`\\=12 %
  |xdef|AllBytesString{%
    \000\001\002\003\004\005\006\007\010\011\012\013\014\015\016\017%
    \020\021\022\023\024\025\026\027\030\031\032\033\034\035\036\037%
    \040!"#$|%&'\(\)*+,-./%
    0123456789:;<=>?%
    @ABCDEFGHIJKLMNO%
    PQRSTUVWXYZ[\\]^_%
    `abcdefghijklmno%
    pqrstuvwxyz{||}~\177%
    \200\201\202\203\204\205\206\207\210\211\212\213\214\215\216\217%
    \220\221\222\223\224\225\226\227\230\231\232\233\234\235\236\237%
    \240\241\242\243\244\245\246\247\250\251\252\253\254\255\256\257%
    \260\261\262\263\264\265\266\267\270\271\272\273\274\275\276\277%
    \300\301\302\303\304\305\306\307\310\311\312\313\314\315\316\317%
    \320\321\322\323\324\325\326\327\330\331\332\333\334\335\336\337%
    \340\341\342\343\344\345\346\347\350\351\352\353\354\355\356\357%
    \360\361\362\363\364\365\366\367\370\371\372\373\374\375\376\377%
  }%
|endgroup
\@onelevel@sanitize\AllBytesString

%<*test4>
\let\org@detokenize\detokenize
\let\detokenize\@undefined
\let\org@numexpr\numexpr
\let\numexpr\@undefined
%</test4>
\RequirePackage{pdfescape}
%<*test4>
\let\detokenize\org@detokenize
\let\numexpr\org@numexpr
%</test4>

\begin{qstest}{all-hex}{\AllBytes, escapehex}
  \EdefEscapeHex\x{\AllBytes}%
  \Expect*{\x}*{\AllBytesHex}%
  \ExpectVar\x\AllBytesHex
\end{qstest}

\begin{qstest}{all-unhex}{\AllBytesHex, unescapehex}
  \EdefUnescapeHex\x{\AllBytesHex}%
  \Expect*{\x}*{\AllBytes}%
  \ExpectVar\x\AllBytes
\end{qstest}

\begin{qstest}{all-unhex-lc}{\AllBytesHexLC, unescapehex, lowercase}
  \EdefUnescapeHex\x{\AllBytesHexLC}%
  \Expect*{\x}*{\AllBytes}%
  \ExpectVar\x\AllBytes
\end{qstest}

\begin{qstest}{unhex-incomplete}{unescapehex, incomplete}
  \EdefUnescapeHex\x{4}%
  \Expect*{\x}{@}%
\end{qstest}

\begin{qstest}{unhex-space}{unescapehex, space}
  \EdefUnescapeHex\x{20}%
  \Expect*{\x}{ }%
  \ExpectVar\x\space
\end{qstest}

\begin{qstest}{unhex-spaces}{unescapehex, spaces}
  \EdefUnescapeHex\x{204020204120}%
  \def\y#1{%
    \edef\z{#1\string @#1#1\string A#1}%
  }\y{ }%
  \Expect*{\x}*{\z}%
  \ExpectVar\x\z
\end{qstest}

\begin{qstest}{unhex-hash}{unescapehex, hash}
  \catcode`\#=12 %
  \EdefUnescapeHex\x{#20}%
  \ExpectVar\x\space
\end{qstest}

\begin{qstest}{unhex-invalid}{unescapehex, invalid}
  \def\test#1#2{%
    \EdefUnescapeHex\x{#1}%
    \edef\y{#2}%
    \@onelevel@sanitize\y
    \ExpectVar\x\y
  }%
%<*test2>
  \edef\x{\pdfunescapehex{4X}}%
  \edef\y{\string @}%
  \ifx\x\y
  \else
    \def~{\space}%
    \typeout{************************************}%
    \typeout{* Your pdfTeX contains bug 777.~~~~*}%
    \typeout{* This test is redefined as dummy, *}%
    \typeout{* because it triggers the bug.~~~~~*}%
    \typeout{************************************}%
    \def\test#1#2{}%
  \fi
%</test2>
  \test{X}{}%
  \test{XY}{}%
  \test{XYZ}{}%
  \test{A}{^^a0}%
  \test{AX}{^^a0}%
  \test{XA}{^^a0}%
  \test{XXAXX}{^^a0}%
\end{qstest}

\begin{qstest}{all-name}{\AllBytes, escapename}
  \EdefEscapeName\x{\AllBytes}%
  \Expect*{\x}*{\AllBytesName}%
  \ExpectVar\x\AllBytesName
\end{qstest}

\begin{qstest}{all-string}{\AllBytes, escapestring}
  \EdefEscapeString\x{\AllBytes}%
  \Expect*{\x}*{\AllBytesString}%
  \ExpectVar\x\AllBytesString
\end{qstest}

\begin{qstest}{uchexdigit}{unescape, uppercase hex digit}
  \catcode`\@=11 %
  \catcode0=12 %
  \def\test#1#2{%
    \uccode0=#1\relax
    \uppercase{%
      \def\x{^^@}%
    }%
    \Expect*{%
       \ifcase\expandafter\PE@TestUcHexDigit\x
         true%
       \else
         false%
       \fi
    }{#2}%
  }%
  \def\range#1#2#3{%
    \count0=#1\relax
    \loop
    \ifnum\count0<#2\relax
      \test{\count0}{#3}%
      \advance\count0 by 1 %
    \repeat
  }%
  \range{0}{47}{false}%
  \range{48}{57}{true}%
  \range{58}{64}{false}%
  \range{65}{70}{true}%
  \range{71}{255}{false}%
\end{qstest}

\begin{qstest}{unescapename}{unescapename}
  \def\test#1#2{%
    \EdefUnescapeName\x{#1}%
    \edef\y{#2}%
    \@onelevel@sanitize\y
    \ExpectVar\x\y
  }%
  \catcode`\#=12 %
  \catcode0=12 %
  \test{}{}%
  \test{x}{x}%
  \test{xy}{xy}%
  \test{#}{#}%
  \test{##}{##}%
  \test{###}{###}%
  \test{####}{####}%
  \test{#x}{#x}%
  \test{#xy}{#xy}%
  \test{#1}{#1}%
  \test{#40}{@}%
  \test{#400}{@0}%
  \test{#4x0}{#4x0}%
  \test{#ab}{^^ab}%
  \test{#00}{^^@}%
  \test{x#40y#40z}{x@y@z}%
  \test{#40#40#40#40}{@@@@}%
  \test{a#x}{a#x}%
  \test{a#xy}{a#xy}%
  \test{a#1}{a#1}%
  \test{a#40}{a@}%
  \test{a#400}{a@0}%
  \test{#20}{ }%
  \test{a#20}{a }%
  \test{a#20b}{a b}%
  \test{a#20#20#20b}{a \space\space b}%
\end{qstest}

\begin{qstest}{unescapestring}{unescapestring}
  \def\test#1#2{%
    \EdefUnescapeString\x{#1}%
    \edef\y{#2}%
    \@onelevel@sanitize\y
    \ExpectVar\x\y
  }%
  \catcode0=12 %
  \def\DefChar#1#2{%
    \begingroup
      \uccode0=#2\relax
    \uppercase{\endgroup
      \def#1{^^@}%
    }%
  }%
  \DefChar\nul{0}%
  \DefChar\one{1}%
  \DefChar\bel{8}%
  \DefChar\tab{9}%
  \DefChar\lf{10}%
  \DefChar\ff{12}%
  \DefChar\cr{13}%
  \DefChar\\{92}%
  \test{}{}%
  \test{a}{a}%
  \test{\\}{}%
  \test{\\\\}{\\}%
  \test{\\\\y}{\\y}%
  \test{\\000}{\nul}%
  \test{\\b}{\bel}%
  \test{\\t}{\tab}%
  \test{\\n}{\lf}%
  \test{\\f}{\ff}%
  \test{\\r}{\cr}%
  \test{\\(}{(}%
  \test{\\)}{)}%
  \test{\\040}{ }%
  \test{\\100}{@}%
  \test{\\40}{ }%
  \test{\\1}{\one}%
  \test{\\01}{\one}%
  \test{\\001}{\one}%
  \test{\\18}{\one8}%
  \test{\\018}{\one8}%
  \test{\\0018}{\one8}%
  \test{x\\}{x}%
  \test{x\\\\}{x\\}%
  \test{x\\\\y}{x\\y}%
  \test{x\\000}{x\nul}%
  \test{x\\b}{x\bel}%
  \test{x\\t}{x\tab}%
  \test{x\\n}{x\lf}%
  \test{x\\f}{x\ff}%
  \test{x\\r}{x\cr}%
  \test{x\\(}{x(}%
  \test{x\\)}{x)}%
  \test{x\\040}{x }%
  \test{x\\100}{x@}%
  \test{x\\40}{x }%
  \test{x\\1}{x\one}%
  \test{x\\01}{x\one}%
  \test{x\\001}{x\one}%
  \test{x\\18}{x\one8}%
  \test{x\\018}{x\one8}%
  \test{x\\0018}{x\one8}%
  \test{\\b\\t\\n\\f\\r\\(\\)\\\\\\000\\040}{%
    \bel\tab\lf\ff\cr()\\\nul\space
  }%
  \test{\\\lf}{}%
  \test{x\\\lf}{x}%
  \test{\cr}{\lf}%
  \test{\cr\lf}{\lf}%
  \test{\lf}{\lf}%
  \test{\lf\cr}{\lf\lf}%
  \test{x\cr}{x\lf}%
  \test{x\cr\lf}{x\lf}%
  \test{x\lf}{x\lf}%
  \test{x\lf\cr}{x\lf\lf}%
  \test{x\\\cr\lf y\cr}{xy\lf}%
%
  \test{\\409}{ 9}%
  \test{\\800}{800}%
  \test{\\900}{900}%
  \test{\\578}{/8}%
  \test{\\477}{?}%
  \test{\\377}{^^ff}%
  \test{\\777}{^^ff}%
  \test{\\7777}{^^ff7}%
\end{qstest}
\stop
%    \end{macrocode}
%
%    \begin{macrocode}
%</test2|test3|test4|test5>
%    \end{macrocode}
%
% \subsubsection{Test for \hologo{iniTeX}}
%
%    \begin{macrocode}
%<*test6>
\input pdfescape.sty\relax
\catcode`\{=1 %
\catcode`\}=2 %
\catcode`\#=6 %
\catcode`\^=7 %
\catcode`\@=11 %
%    \end{macrocode}
%    \begin{macrocode}
\begingroup
  \catcode`\@=11 %
  \countdef\count@=255 %
  \def\space{ }%
  \long\def\@whilenum#1\do #2{%
    \ifnum #1\relax
      #2\relax
      \@iwhilenum{#1\relax#2\relax}%
    \fi
  }%
  \long\def\@iwhilenum#1{%
    \ifnum #1%
      \expandafter\@iwhilenum
    \else
      \expandafter\ltx@gobble
    \fi
    {#1}%
  }%
  \gdef\AllBytes{}%
  \count@=0 %
  \catcode0=12 %
  \@whilenum\count@<256 \do{%
    \lccode0=\count@
    \ifnum\count@=32 %
      \xdef\AllBytes{\AllBytes\space}%
    \else
      \lowercase{%
        \xdef\AllBytes{\AllBytes^^@}%
      }%
    \fi
    \advance\count@ by 1 %
  }%
\endgroup
\def\AllBytesHex{%
  000102030405060708090A0B0C0D0E0F%
  101112131415161718191A1B1C1D1E1F%
  202122232425262728292A2B2C2D2E2F%
  303132333435363738393A3B3C3D3E3F%
  404142434445464748494A4B4C4D4E4F%
  505152535455565758595A5B5C5D5E5F%
  606162636465666768696A6B6C6D6E6F%
  707172737475767778797A7B7C7D7E7F%
  808182838485868788898A8B8C8D8E8F%
  909192939495969798999A9B9C9D9E9F%
  A0A1A2A3A4A5A6A7A8A9AAABACADAEAF%
  B0B1B2B3B4B5B6B7B8B9BABBBCBDBEBF%
  C0C1C2C3C4C5C6C7C8C9CACBCCCDCECF%
  D0D1D2D3D4D5D6D7D8D9DADBDCDDDEDF%
  E0E1E2E3E4E5E6E7E8E9EAEBECEDEEEF%
  F0F1F2F3F4F5F6F7F8F9FAFBFCFDFEFF%
}
\ltx@onelevel@sanitize\AllBytesHex
\expandafter\lowercase\expandafter{%
  \expandafter\def\expandafter\AllBytesHexLC
      \expandafter{\AllBytesHex}%
}
\begingroup
  \catcode`\#=12 %
  \xdef\AllBytesName{%
    #01#02#03#04#05#06#07#08#09#0A#0B#0C#0D#0E#0F%
    #10#11#12#13#14#15#16#17#18#19#1A#1B#1C#1D#1E#1F%
    #20!"#23$#25&'#28#29*+,-.#2F%
    0123456789:;#3C=#3E?%
    @ABCDEFGHIJKLMNO%
    PQRSTUVWXYZ#5B\ltx@backslashchar#5D^_%
    `abcdefghijklmno%
    pqrstuvwxyz#7B|#7D\string~#7F%
    #80#81#82#83#84#85#86#87#88#89#8A#8B#8C#8D#8E#8F%
    #90#91#92#93#94#95#96#97#98#99#9A#9B#9C#9D#9E#9F%
    #A0#A1#A2#A3#A4#A5#A6#A7#A8#A9#AA#AB#AC#AD#AE#AF%
    #B0#B1#B2#B3#B4#B5#B6#B7#B8#B9#BA#BB#BC#BD#BE#BF%
    #C0#C1#C2#C3#C4#C5#C6#C7#C8#C9#CA#CB#CC#CD#CE#CF%
    #D0#D1#D2#D3#D4#D5#D6#D7#D8#D9#DA#DB#DC#DD#DE#DF%
    #E0#E1#E2#E3#E4#E5#E6#E7#E8#E9#EA#EB#EC#ED#EE#EF%
    #F0#F1#F2#F3#F4#F5#F6#F7#F8#F9#FA#FB#FC#FD#FE#FF%
  }%
\endgroup
\ltx@onelevel@sanitize\AllBytesName
\edef\AllBytesFromName{\expandafter\ltx@gobble\AllBytes}

\begingroup
  \def\|{|}%
  \edef\%{\ltx@percentchar}%
  \catcode`\|=0 %
  \catcode`\#=12 %
  \catcode`\~=12 %
  \catcode`\\=12 %
  |xdef|AllBytesString{%
    \000\001\002\003\004\005\006\007\010\011\012\013\014\015\016\017%
    \020\021\022\023\024\025\026\027\030\031\032\033\034\035\036\037%
    \040!"#$|%&'\(\)*+,-./%
    0123456789:;<=>?%
    @ABCDEFGHIJKLMNO%
    PQRSTUVWXYZ[\\]^_%
    `abcdefghijklmno%
    pqrstuvwxyz{||}~\177%
    \200\201\202\203\204\205\206\207\210\211\212\213\214\215\216\217%
    \220\221\222\223\224\225\226\227\230\231\232\233\234\235\236\237%
    \240\241\242\243\244\245\246\247\250\251\252\253\254\255\256\257%
    \260\261\262\263\264\265\266\267\270\271\272\273\274\275\276\277%
    \300\301\302\303\304\305\306\307\310\311\312\313\314\315\316\317%
    \320\321\322\323\324\325\326\327\330\331\332\333\334\335\336\337%
    \340\341\342\343\344\345\346\347\350\351\352\353\354\355\356\357%
    \360\361\362\363\364\365\366\367\370\371\372\373\374\375\376\377%
  }%
|endgroup
\ltx@onelevel@sanitize\AllBytesString
%    \end{macrocode}
%    \begin{macrocode}
\def\msg#{\immediate\write16}
\def\Test#1#2#3{%
  \begingroup
    #1\TestResult{#2}%
    \ifx\TestResult#3%
    \else
      \newlinechar=10 %
      \msg{Expect:^^J#3}%
      \msg{Result:^^J\TestResult}%
      \errmessage{\string#2 -\string#1-> \string#3}%
    \fi
  \endgroup
}
\Test\EdefEscapeHex\AllBytes\AllBytesHex
\Test\EdefUnescapeHex\AllBytesHex\AllBytes
\Test\EdefEscapeName\AllBytes\AllBytesName
\Test\EdefUnescapeName\AllBytesName\AllBytesFromName
\Test\EdefEscapeString\AllBytes\AllBytesString
\Test\EdefUnescapeString\AllBytesString\AllBytes
%    \end{macrocode}
%    \begin{macrocode}
\csname @@end\endcsname\end
%</test6>
%    \end{macrocode}
%
% \section{Installation}
%
% \subsection{Download}
%
% \paragraph{Package.} This package is available on
% CTAN\footnote{\url{ftp://ftp.ctan.org/tex-archive/}}:
% \begin{description}
% \item[\CTAN{macros/latex/contrib/oberdiek/pdfescape.dtx}] The source file.
% \item[\CTAN{macros/latex/contrib/oberdiek/pdfescape.pdf}] Documentation.
% \end{description}
%
%
% \paragraph{Bundle.} All the packages of the bundle `oberdiek'
% are also available in a TDS compliant ZIP archive. There
% the packages are already unpacked and the documentation files
% are generated. The files and directories obey the TDS standard.
% \begin{description}
% \item[\CTAN{install/macros/latex/contrib/oberdiek.tds.zip}]
% \end{description}
% \emph{TDS} refers to the standard ``A Directory Structure
% for \TeX\ Files'' (\CTAN{tds/tds.pdf}). Directories
% with \xfile{texmf} in their name are usually organized this way.
%
% \subsection{Bundle installation}
%
% \paragraph{Unpacking.} Unpack the \xfile{oberdiek.tds.zip} in the
% TDS tree (also known as \xfile{texmf} tree) of your choice.
% Example (linux):
% \begin{quote}
%   |unzip oberdiek.tds.zip -d ~/texmf|
% \end{quote}
%
% \paragraph{Script installation.}
% Check the directory \xfile{TDS:scripts/oberdiek/} for
% scripts that need further installation steps.
% Package \xpackage{attachfile2} comes with the Perl script
% \xfile{pdfatfi.pl} that should be installed in such a way
% that it can be called as \texttt{pdfatfi}.
% Example (linux):
% \begin{quote}
%   |chmod +x scripts/oberdiek/pdfatfi.pl|\\
%   |cp scripts/oberdiek/pdfatfi.pl /usr/local/bin/|
% \end{quote}
%
% \subsection{Package installation}
%
% \paragraph{Unpacking.} The \xfile{.dtx} file is a self-extracting
% \docstrip\ archive. The files are extracted by running the
% \xfile{.dtx} through \plainTeX:
% \begin{quote}
%   \verb|tex pdfescape.dtx|
% \end{quote}
%
% \paragraph{TDS.} Now the different files must be moved into
% the different directories in your installation TDS tree
% (also known as \xfile{texmf} tree):
% \begin{quote}
% \def\t{^^A
% \begin{tabular}{@{}>{\ttfamily}l@{ $\rightarrow$ }>{\ttfamily}l@{}}
%   pdfescape.sty & tex/generic/oberdiek/pdfescape.sty\\
%   pdfescape.pdf & doc/latex/oberdiek/pdfescape.pdf\\
%   test/pdfescape-test1.tex & doc/latex/oberdiek/test/pdfescape-test1.tex\\
%   test/pdfescape-test2.tex & doc/latex/oberdiek/test/pdfescape-test2.tex\\
%   test/pdfescape-test3.tex & doc/latex/oberdiek/test/pdfescape-test3.tex\\
%   test/pdfescape-test4.tex & doc/latex/oberdiek/test/pdfescape-test4.tex\\
%   test/pdfescape-test5.tex & doc/latex/oberdiek/test/pdfescape-test5.tex\\
%   test/pdfescape-test6.tex & doc/latex/oberdiek/test/pdfescape-test6.tex\\
%   pdfescape.dtx & source/latex/oberdiek/pdfescape.dtx\\
% \end{tabular}^^A
% }^^A
% \sbox0{\t}^^A
% \ifdim\wd0>\linewidth
%   \begingroup
%     \advance\linewidth by\leftmargin
%     \advance\linewidth by\rightmargin
%   \edef\x{\endgroup
%     \def\noexpand\lw{\the\linewidth}^^A
%   }\x
%   \def\lwbox{^^A
%     \leavevmode
%     \hbox to \linewidth{^^A
%       \kern-\leftmargin\relax
%       \hss
%       \usebox0
%       \hss
%       \kern-\rightmargin\relax
%     }^^A
%   }^^A
%   \ifdim\wd0>\lw
%     \sbox0{\small\t}^^A
%     \ifdim\wd0>\linewidth
%       \ifdim\wd0>\lw
%         \sbox0{\footnotesize\t}^^A
%         \ifdim\wd0>\linewidth
%           \ifdim\wd0>\lw
%             \sbox0{\scriptsize\t}^^A
%             \ifdim\wd0>\linewidth
%               \ifdim\wd0>\lw
%                 \sbox0{\tiny\t}^^A
%                 \ifdim\wd0>\linewidth
%                   \lwbox
%                 \else
%                   \usebox0
%                 \fi
%               \else
%                 \lwbox
%               \fi
%             \else
%               \usebox0
%             \fi
%           \else
%             \lwbox
%           \fi
%         \else
%           \usebox0
%         \fi
%       \else
%         \lwbox
%       \fi
%     \else
%       \usebox0
%     \fi
%   \else
%     \lwbox
%   \fi
% \else
%   \usebox0
% \fi
% \end{quote}
% If you have a \xfile{docstrip.cfg} that configures and enables \docstrip's
% TDS installing feature, then some files can already be in the right
% place, see the documentation of \docstrip.
%
% \subsection{Refresh file name databases}
%
% If your \TeX~distribution
% (\teTeX, \mikTeX, \dots) relies on file name databases, you must refresh
% these. For example, \teTeX\ users run \verb|texhash| or
% \verb|mktexlsr|.
%
% \subsection{Some details for the interested}
%
% \paragraph{Attached source.}
%
% The PDF documentation on CTAN also includes the
% \xfile{.dtx} source file. It can be extracted by
% AcrobatReader 6 or higher. Another option is \textsf{pdftk},
% e.g. unpack the file into the current directory:
% \begin{quote}
%   \verb|pdftk pdfescape.pdf unpack_files output .|
% \end{quote}
%
% \paragraph{Unpacking with \LaTeX.}
% The \xfile{.dtx} chooses its action depending on the format:
% \begin{description}
% \item[\plainTeX:] Run \docstrip\ and extract the files.
% \item[\LaTeX:] Generate the documentation.
% \end{description}
% If you insist on using \LaTeX\ for \docstrip\ (really,
% \docstrip\ does not need \LaTeX), then inform the autodetect routine
% about your intention:
% \begin{quote}
%   \verb|latex \let\install=y% \iffalse meta-comment
%
% File: pdfescape.dtx
% Version: 2011/11/25 v1.13
% Info: Implements pdfTeX's escape features
%
% Copyright (C) 2007, 2010, 2011 by
%    Heiko Oberdiek <heiko.oberdiek at googlemail.com>
%
% This work may be distributed and/or modified under the
% conditions of the LaTeX Project Public License, either
% version 1.3c of this license or (at your option) any later
% version. This version of this license is in
%    http://www.latex-project.org/lppl/lppl-1-3c.txt
% and the latest version of this license is in
%    http://www.latex-project.org/lppl.txt
% and version 1.3 or later is part of all distributions of
% LaTeX version 2005/12/01 or later.
%
% This work has the LPPL maintenance status "maintained".
%
% This Current Maintainer of this work is Heiko Oberdiek.
%
% The Base Interpreter refers to any `TeX-Format',
% because some files are installed in TDS:tex/generic//.
%
% This work consists of the main source file pdfescape.dtx
% and the derived files
%    pdfescape.sty, pdfescape.pdf, pdfescape.ins, pdfescape.drv,
%    pdfescape-test1.tex, pdfescape-test2.tex,
%    pdfescape-test3.tex, pdfescape-test4.tex,
%    pdfescape-test5.tex, pdfescape-test6.tex.
%
% Distribution:
%    CTAN:macros/latex/contrib/oberdiek/pdfescape.dtx
%    CTAN:macros/latex/contrib/oberdiek/pdfescape.pdf
%
% Unpacking:
%    (a) If pdfescape.ins is present:
%           tex pdfescape.ins
%    (b) Without pdfescape.ins:
%           tex pdfescape.dtx
%    (c) If you insist on using LaTeX
%           latex \let\install=y% \iffalse meta-comment
%
% File: pdfescape.dtx
% Version: 2011/11/25 v1.13
% Info: Implements pdfTeX's escape features
%
% Copyright (C) 2007, 2010, 2011 by
%    Heiko Oberdiek <heiko.oberdiek at googlemail.com>
%
% This work may be distributed and/or modified under the
% conditions of the LaTeX Project Public License, either
% version 1.3c of this license or (at your option) any later
% version. This version of this license is in
%    http://www.latex-project.org/lppl/lppl-1-3c.txt
% and the latest version of this license is in
%    http://www.latex-project.org/lppl.txt
% and version 1.3 or later is part of all distributions of
% LaTeX version 2005/12/01 or later.
%
% This work has the LPPL maintenance status "maintained".
%
% This Current Maintainer of this work is Heiko Oberdiek.
%
% The Base Interpreter refers to any `TeX-Format',
% because some files are installed in TDS:tex/generic//.
%
% This work consists of the main source file pdfescape.dtx
% and the derived files
%    pdfescape.sty, pdfescape.pdf, pdfescape.ins, pdfescape.drv,
%    pdfescape-test1.tex, pdfescape-test2.tex,
%    pdfescape-test3.tex, pdfescape-test4.tex,
%    pdfescape-test5.tex, pdfescape-test6.tex.
%
% Distribution:
%    CTAN:macros/latex/contrib/oberdiek/pdfescape.dtx
%    CTAN:macros/latex/contrib/oberdiek/pdfescape.pdf
%
% Unpacking:
%    (a) If pdfescape.ins is present:
%           tex pdfescape.ins
%    (b) Without pdfescape.ins:
%           tex pdfescape.dtx
%    (c) If you insist on using LaTeX
%           latex \let\install=y\input{pdfescape.dtx}
%        (quote the arguments according to the demands of your shell)
%
% Documentation:
%    (a) If pdfescape.drv is present:
%           latex pdfescape.drv
%    (b) Without pdfescape.drv:
%           latex pdfescape.dtx; ...
%    The class ltxdoc loads the configuration file ltxdoc.cfg
%    if available. Here you can specify further options, e.g.
%    use A4 as paper format:
%       \PassOptionsToClass{a4paper}{article}
%
%    Programm calls to get the documentation (example):
%       pdflatex pdfescape.dtx
%       makeindex -s gind.ist pdfescape.idx
%       pdflatex pdfescape.dtx
%       makeindex -s gind.ist pdfescape.idx
%       pdflatex pdfescape.dtx
%
% Installation:
%    TDS:tex/generic/oberdiek/pdfescape.sty
%    TDS:doc/latex/oberdiek/pdfescape.pdf
%    TDS:doc/latex/oberdiek/test/pdfescape-test1.tex
%    TDS:doc/latex/oberdiek/test/pdfescape-test2.tex
%    TDS:doc/latex/oberdiek/test/pdfescape-test3.tex
%    TDS:doc/latex/oberdiek/test/pdfescape-test4.tex
%    TDS:doc/latex/oberdiek/test/pdfescape-test5.tex
%    TDS:doc/latex/oberdiek/test/pdfescape-test6.tex
%    TDS:source/latex/oberdiek/pdfescape.dtx
%
%<*ignore>
\begingroup
  \catcode123=1 %
  \catcode125=2 %
  \def\x{LaTeX2e}%
\expandafter\endgroup
\ifcase 0\ifx\install y1\fi\expandafter
         \ifx\csname processbatchFile\endcsname\relax\else1\fi
         \ifx\fmtname\x\else 1\fi\relax
\else\csname fi\endcsname
%</ignore>
%<*install>
\input docstrip.tex
\Msg{************************************************************************}
\Msg{* Installation}
\Msg{* Package: pdfescape 2011/11/25 v1.13 Implements pdfTeX's escape features (HO)}
\Msg{************************************************************************}

\keepsilent
\askforoverwritefalse

\let\MetaPrefix\relax
\preamble

This is a generated file.

Project: pdfescape
Version: 2011/11/25 v1.13

Copyright (C) 2007, 2010, 2011 by
   Heiko Oberdiek <heiko.oberdiek at googlemail.com>

This work may be distributed and/or modified under the
conditions of the LaTeX Project Public License, either
version 1.3c of this license or (at your option) any later
version. This version of this license is in
   http://www.latex-project.org/lppl/lppl-1-3c.txt
and the latest version of this license is in
   http://www.latex-project.org/lppl.txt
and version 1.3 or later is part of all distributions of
LaTeX version 2005/12/01 or later.

This work has the LPPL maintenance status "maintained".

This Current Maintainer of this work is Heiko Oberdiek.

The Base Interpreter refers to any `TeX-Format',
because some files are installed in TDS:tex/generic//.

This work consists of the main source file pdfescape.dtx
and the derived files
   pdfescape.sty, pdfescape.pdf, pdfescape.ins, pdfescape.drv,
   pdfescape-test1.tex, pdfescape-test2.tex,
   pdfescape-test3.tex, pdfescape-test4.tex,
   pdfescape-test5.tex, pdfescape-test6.tex.

\endpreamble
\let\MetaPrefix\DoubleperCent

\generate{%
  \file{pdfescape.ins}{\from{pdfescape.dtx}{install}}%
  \file{pdfescape.drv}{\from{pdfescape.dtx}{driver}}%
  \usedir{tex/generic/oberdiek}%
  \file{pdfescape.sty}{\from{pdfescape.dtx}{package}}%
  \usedir{doc/latex/oberdiek/test}%
  \file{pdfescape-test1.tex}{\from{pdfescape.dtx}{test1}}%
  \file{pdfescape-test2.tex}{\from{pdfescape.dtx}{test2}}%
  \file{pdfescape-test3.tex}{\from{pdfescape.dtx}{test3}}%
  \file{pdfescape-test4.tex}{\from{pdfescape.dtx}{test4}}%
  \file{pdfescape-test5.tex}{\from{pdfescape.dtx}{test5}}%
  \file{pdfescape-test6.tex}{\from{pdfescape.dtx}{test6}}%
  \nopreamble
  \nopostamble
  \usedir{source/latex/oberdiek/catalogue}%
  \file{pdfescape.xml}{\from{pdfescape.dtx}{catalogue}}%
}

\catcode32=13\relax% active space
\let =\space%
\Msg{************************************************************************}
\Msg{*}
\Msg{* To finish the installation you have to move the following}
\Msg{* file into a directory searched by TeX:}
\Msg{*}
\Msg{*     pdfescape.sty}
\Msg{*}
\Msg{* To produce the documentation run the file `pdfescape.drv'}
\Msg{* through LaTeX.}
\Msg{*}
\Msg{* Happy TeXing!}
\Msg{*}
\Msg{************************************************************************}

\endbatchfile
%</install>
%<*ignore>
\fi
%</ignore>
%<*driver>
\NeedsTeXFormat{LaTeX2e}
\ProvidesFile{pdfescape.drv}%
  [2011/11/25 v1.13 Implements pdfTeX's escape features (HO)]%
\documentclass{ltxdoc}
\usepackage{holtxdoc}[2011/11/22]
\begin{document}
  \DocInput{pdfescape.dtx}%
\end{document}
%</driver>
% \fi
%
% \CheckSum{2516}
%
% \CharacterTable
%  {Upper-case    \A\B\C\D\E\F\G\H\I\J\K\L\M\N\O\P\Q\R\S\T\U\V\W\X\Y\Z
%   Lower-case    \a\b\c\d\e\f\g\h\i\j\k\l\m\n\o\p\q\r\s\t\u\v\w\x\y\z
%   Digits        \0\1\2\3\4\5\6\7\8\9
%   Exclamation   \!     Double quote  \"     Hash (number) \#
%   Dollar        \$     Percent       \%     Ampersand     \&
%   Acute accent  \'     Left paren    \(     Right paren   \)
%   Asterisk      \*     Plus          \+     Comma         \,
%   Minus         \-     Point         \.     Solidus       \/
%   Colon         \:     Semicolon     \;     Less than     \<
%   Equals        \=     Greater than  \>     Question mark \?
%   Commercial at \@     Left bracket  \[     Backslash     \\
%   Right bracket \]     Circumflex    \^     Underscore    \_
%   Grave accent  \`     Left brace    \{     Vertical bar  \|
%   Right brace   \}     Tilde         \~}
%
% \GetFileInfo{pdfescape.drv}
%
% \title{The \xpackage{pdfescape} package}
% \date{2011/11/25 v1.13}
% \author{Heiko Oberdiek\\\xemail{heiko.oberdiek at googlemail.com}}
%
% \maketitle
%
% \begin{abstract}
% This package implements \pdfTeX's escape features
% (\cs{pdfescapehex}, \cs{pdfunescapehex}, \cs{pdfescapename},
% \cs{pdfescapestring}) using \TeX\ or \eTeX.
% \end{abstract}
%
% \tableofcontents
%
% \section{Documentation}
%
% \begin{declcs}{EdefEscapeHex} \M{cmd} \M{string}\\
%  \cs{EdefUnescapeHex} \M{cmd} \M{string}\\
%  \cs{EdefEscapeName} \M{cmd} \M{string}\\
%  \cs{EdefEscapeString} \M{cmd} \M{string}
% \end{declcs}
% These commands converts \meta{string} and stores the result
% in macro \meta{cmd}. The conversion result is the same
% as the conversion of the corresponding \pdfTeX's primitives.
% Note that the argument \meta{string} is expanded before
% the conversion.
%
% For example, if \pdfTeX\ \textgreater= 1.30 is present,
% then \cs{EdefEscapeHex}
% becomes to:
% \begin{quote}
%\begin{verbatim}
%\def\EdefEscapeHex#1#2{%
%  \edef#1{\pdfescapehex{#2}}%
%}
%\end{verbatim}
% \end{quote}
% The package provides implementations for the case that \pdfTeX\ is
% not present (or too old). Even \eTeX\ can be missing, however it is
% used if it is detected.
%
% \paragraph{Babel.}
% The input strings may contain shorthand characters of package
% \xpackage{babel}.
%
% \subsection{Additional unescape macros}
%
% \begin{declcs}{EdefUnescapeName} \M{cmd} \M{string}
% \end{declcs}
% Sequences of a hash sign with two hexadecimal digits are
% converted to the corresponding character (PDF-1.2).
% A hash sign that is not followed by two hexadecimal digits
% is left unchanged. The catcodes in the result string follow
% TeX's conventions. The space has catcode 10 (space) and
% the other characters have catcode 12 (other).
%
% \begin{declcs}{EdefUnescapeString} \M{cmd} \M{string}
% \end{declcs}
% Macro \meta{cmd} stores the unescaped string in \meta{string}.
% All the rules for literal strings are implemented, see
% PDF specification.
% The catcodes in the result string follow TeX's conventions.
%
% \subsection{Sanitizing macro}
%
% \begin{declcs}{EdefSanitize} \M{cmd} \M{string}
% \end{declcs}
% Argument \meta{string} is expanded, converted to a string
% of tokens with catcode 12 (other) and space tokens, and
% stored in macro \meta{cmd}.
%
% \StopEventually{
% }
%
% \section{Implementation}
%
%    \begin{macrocode}
%<*package>
%    \end{macrocode}
%
% \subsection{Reload check and package identification}
%    Reload check, especially if the package is not used with \LaTeX.
%    \begin{macrocode}
\begingroup\catcode61\catcode48\catcode32=10\relax%
  \catcode13=5 % ^^M
  \endlinechar=13 %
  \catcode35=6 % #
  \catcode39=12 % '
  \catcode44=12 % ,
  \catcode45=12 % -
  \catcode46=12 % .
  \catcode58=12 % :
  \catcode64=11 % @
  \catcode123=1 % {
  \catcode125=2 % }
  \expandafter\let\expandafter\x\csname ver@pdfescape.sty\endcsname
  \ifx\x\relax % plain-TeX, first loading
  \else
    \def\empty{}%
    \ifx\x\empty % LaTeX, first loading,
      % variable is initialized, but \ProvidesPackage not yet seen
    \else
      \expandafter\ifx\csname PackageInfo\endcsname\relax
        \def\x#1#2{%
          \immediate\write-1{Package #1 Info: #2.}%
        }%
      \else
        \def\x#1#2{\PackageInfo{#1}{#2, stopped}}%
      \fi
      \x{pdfescape}{The package is already loaded}%
      \aftergroup\endinput
    \fi
  \fi
\endgroup%
%    \end{macrocode}
%    Package identification:
%    \begin{macrocode}
\begingroup\catcode61\catcode48\catcode32=10\relax%
  \catcode13=5 % ^^M
  \endlinechar=13 %
  \catcode35=6 % #
  \catcode39=12 % '
  \catcode40=12 % (
  \catcode41=12 % )
  \catcode44=12 % ,
  \catcode45=12 % -
  \catcode46=12 % .
  \catcode47=12 % /
  \catcode58=12 % :
  \catcode64=11 % @
  \catcode91=12 % [
  \catcode93=12 % ]
  \catcode123=1 % {
  \catcode125=2 % }
  \expandafter\ifx\csname ProvidesPackage\endcsname\relax
    \def\x#1#2#3[#4]{\endgroup
      \immediate\write-1{Package: #3 #4}%
      \xdef#1{#4}%
    }%
  \else
    \def\x#1#2[#3]{\endgroup
      #2[{#3}]%
      \ifx#1\@undefined
        \xdef#1{#3}%
      \fi
      \ifx#1\relax
        \xdef#1{#3}%
      \fi
    }%
  \fi
\expandafter\x\csname ver@pdfescape.sty\endcsname
\ProvidesPackage{pdfescape}%
  [2011/11/25 v1.13 Implements pdfTeX's escape features (HO)]%
%    \end{macrocode}
%
% \subsection{Catcodes}
%
%    \begin{macrocode}
\begingroup\catcode61\catcode48\catcode32=10\relax%
  \catcode13=5 % ^^M
  \endlinechar=13 %
  \catcode123 1 % {
  \catcode125 2 % }
  \catcode64 11 %
  \def\x{\endgroup
    \expandafter\edef\csname PE@AtEnd\endcsname{%
      \endlinechar=\the\endlinechar\relax
      \catcode13=\the\catcode13\relax
      \catcode32=\the\catcode32\relax
      \catcode35=\the\catcode35\relax
      \catcode61=\the\catcode61\relax
      \catcode64=\the\catcode64\relax
      \catcode123=\the\catcode123\relax
      \catcode125=\the\catcode125\relax
    }%
  }%
\x\catcode61\catcode48\catcode32=10\relax%
\catcode13=5 % ^^M
\endlinechar=13 %
\catcode35=6 % #
\catcode64=11 % @
\catcode123=1 % {
\catcode125=2 % }
\def\TMP@EnsureCode#1#2#3{%
  \edef\PE@AtEnd{%
    \PE@AtEnd
    #1#2=\the#1#2\relax
  }%
  #1#2=#3\relax
}
\TMP@EnsureCode\catcode{0}{12}% ^^@
\TMP@EnsureCode\catcode{34}{12}% "
\TMP@EnsureCode\catcode{36}{3}% $
\TMP@EnsureCode\catcode{38}{4}% &
\TMP@EnsureCode\catcode{39}{12}% '
\TMP@EnsureCode\catcode{42}{12}% *
\TMP@EnsureCode\catcode{45}{12}% -
\TMP@EnsureCode\catcode{46}{12}% .
\TMP@EnsureCode\catcode{47}{12}% /
\TMP@EnsureCode\catcode{60}{12}% <
\TMP@EnsureCode\catcode{62}{12}% >
\TMP@EnsureCode\catcode{91}{12}% [
\TMP@EnsureCode\catcode{93}{12}% ]
\TMP@EnsureCode\catcode{94}{7}% ^
\TMP@EnsureCode\catcode{96}{12}% `
\TMP@EnsureCode\uccode{34}{0}% "
\TMP@EnsureCode\uccode{48}{0}% 0
\TMP@EnsureCode\uccode{61}{0}% =
\edef\PE@AtEnd{\PE@AtEnd\noexpand\endinput}
%    \end{macrocode}
%
% \subsection{Load package}
%
%    \begin{macrocode}
\begingroup\expandafter\expandafter\expandafter\endgroup
\expandafter\ifx\csname RequirePackage\endcsname\relax
  \def\TMP@RequirePackage#1[#2]{%
    \begingroup\expandafter\expandafter\expandafter\endgroup
    \expandafter\ifx\csname ver@#1.sty\endcsname\relax
      \input #1.sty\relax
    \fi
  }%
  \TMP@RequirePackage{ltxcmds}[2010/04/08]%
\else
  \RequirePackage{ltxcmds}[2010/04/08]%
\fi
%    \end{macrocode}
%
% \subsection{Sanitizing}
%
%    \begin{macro}{\EdefSanitize}
%    Macro \cs{EdefSanitize} takes |#2|, entirely converts
%    it to token with catcode 12 (other) and stores the
%    result in macro |#1|.
%    \begin{macrocode}
\begingroup\expandafter\expandafter\expandafter\endgroup
\expandafter\ifx\csname detokenize\endcsname\relax
  \long\def\EdefSanitize#1#2{%
    \begingroup
      \csname @safe@activestrue\endcsname
      \edef#1{#2}%
      \PE@onelevel@sanitize#1%
    \expandafter\endgroup
    \expandafter\def\expandafter#1\expandafter{#1}%
  }%
  \begingroup\expandafter\expandafter\expandafter\endgroup
  \expandafter\ifx\csname @onelevel@sanitize\endcsname\relax
    \def\PE@onelevel@sanitize#1{%
      \edef#1{\expandafter\PE@strip@prefix\meaning#1}%
    }%
    \def\PE@strip@prefix#1>{}%
  \else
    \let\PE@onelevel@sanitize\@onelevel@sanitize
  \fi
\else
  \long\def\EdefSanitize#1#2{%
    \begingroup
      \csname @safe@activestrue\endcsname
      \edef#1{#2}%
    \expandafter\endgroup
    \expandafter\def\expandafter#1\expandafter{%
      \detokenize\expandafter{#1}%
    }%
  }%
  \def\PE@onelevel@sanitize#1{%
    \edef#1{\detokenize\expandafter{#1}}%
  }%
\fi
%    \end{macrocode}
%    \end{macro}
%    \begin{macro}{\PE@sanitize}
%    Macro \cs{PE@sanitize} is only defined for
%    compatibility with version 1.4. Its use is deprecated.
%    \begin{macrocode}
\let\PE@sanitize\EdefSanitize
%    \end{macrocode}
%    \end{macro}
%
% \subsubsection{Space characters}
%
%    \begin{macro}{\PE@space@other}
%    \begin{macrocode}
\begingroup
  \catcode`\ =12\relax%
\def\x{\endgroup\def\PE@space@other{ }}\x\relax
%    \end{macrocode}
%    \end{macro}
%    \begin{macro}{\PE@space@space}
%    \begin{macrocode}
\def\PE@space@space{ }
%    \end{macrocode}
%    \end{macro}
%
% \subsubsection{Space normalization}
%
%    \begin{macro}{\PE@SanitizeSpaceOther}
%    \begin{macrocode}
\def\PE@SanitizeSpaceOther#1{%
  \edef#1{\expandafter\PE@SpaceToOther#1 \relax}%
}
%    \end{macrocode}
%    \end{macro}
%
%    \begin{macro}{\PE@SpaceToOther}
%    \begin{macrocode}
\def\PE@SpaceToOther#1 #2\relax{%
  #1%
  \ifx\\#2\\%
  \else
    \PE@space@other
    \ltx@ReturnAfterFi{%
      \PE@SpaceToOther#2\relax
    }%
  \fi
}
%    \end{macrocode}
%    \end{macro}
%
% \subsection{\cs{EdefUnescapeName}}
%
%    \begin{macro}{\EdefUnescapeName}
%    \begin{macrocode}
\def\EdefUnescapeName#1#2{%
  \EdefSanitize#1{#2}%
  \PE@SanitizeSpaceOther#1%
  \PE@UnescapeName#1%
  \PE@onelevel@sanitize#1%
}
%    \end{macrocode}
%    \end{macro}
%
%    \begin{macro}{\PE@UnescapeName}
%    \begin{macrocode}
\begingroup
  \catcode`\$=6  % hash
  \catcode`\#=12 % other
  \gdef\PE@UnescapeName$1{%
    \begingroup
      \PE@InitUccodeHexDigit
      \def\PE@result{}%
      \expandafter\PE@DeName$1#\relax\relax
    \expandafter\endgroup
    \expandafter\def\expandafter$1\expandafter{\PE@result}%
  }%
  \gdef\PE@DeName$1#$2$3{%
    \ifx\relax$2%
      \edef\PE@result{\PE@result$1}%
      \let\PE@next\relax
    \else
      \ifx\relax$3%
        % wrong escape sequence in input
        \edef\PE@result{\PE@result$1#}%
        \let\PE@next\relax
      \else
        \uppercase{%
          \def\PE@testA{$2}%
          \def\PE@testB{$3}%
        }%
        \ifcase\ifcase\expandafter\PE@TestUcHexDigit\PE@testA
                 \ifcase\expandafter\PE@TestUcHexDigit\PE@testB
                   \ltx@zero
                 \else
                   \ltx@one
                 \fi
               \else
                 \ltx@one
               \fi
          \uccode\ltx@zero="\PE@testA\PE@testB\relax
          \uppercase{%
            \def\PE@temp{^^@}%
          }%
          \uccode\ltx@zero=\ltx@zero
          \edef\PE@result{\PE@result$1\PE@temp}%
          \let\PE@next\PE@DeName
        \else
          % wrong escape sequence in input
          \edef\PE@result{\PE@result$1#}%
          \def\PE@next{\PE@DeName$2$3}%
        \fi
      \fi
    \fi
    \PE@next
  }%
\endgroup
%    \end{macrocode}
%    \end{macro}
%
%    \begin{macro}{\PE@InitUccodeHexDigit}
%    \begin{macrocode}
\def\PE@InitUccodeHexDigit{%
  \uccode`a=`A\relax
  \uccode`b=`B\relax
  \uccode`c=`C\relax
  \uccode`d=`D\relax
  \uccode`e=`E\relax
  \uccode`f=`F\relax
  \uccode`A=\ltx@zero
  \uccode`B=\ltx@zero
  \uccode`C=\ltx@zero
  \uccode`D=\ltx@zero
  \uccode`E=\ltx@zero
  \uccode`F=\ltx@zero
  \uccode`0=\ltx@zero
  \uccode`1=\ltx@zero
  \uccode`2=\ltx@zero
  \uccode`3=\ltx@zero
  \uccode`4=\ltx@zero
  \uccode`5=\ltx@zero
  \uccode`6=\ltx@zero
  \uccode`7=\ltx@zero
  \uccode`8=\ltx@zero
  \uccode`9=\ltx@zero
}
%    \end{macrocode}
%    \end{macro}
%
%    \begin{macro}{\PE@TestUcHexDigit}
%    \begin{macrocode}
\def\PE@TestUcHexDigit#1{%
  \ifnum`#1<48 % 0
    \ltx@one
  \else
    \ifnum`#1>70 % F
      \ltx@one
    \else
      \ifnum`#1>57 % 9
        \ifnum`#1<65 % A
          \ltx@one
        \else
          \ltx@zero
        \fi
      \else
        \ltx@zero
      \fi
    \fi
  \fi
}
%    \end{macrocode}
%    \end{macro}
%
% \subsection{\cs{EdefUnescapeString}}
%
%    \begin{macro}{\EdefUnescapeString}
%    \begin{macrocode}
\def\EdefUnescapeString#1#2{%
  \EdefSanitize#1{#2}%
  \PE@SanitizeSpaceOther#1%
  \PE@NormalizeLineEnd#1%
  \PE@UnescapeString#1%
  \PE@onelevel@sanitize#1%
}
%    \end{macrocode}
%    \end{macro}
%
%    \begin{macrocode}
\begingroup
  \uccode`\8=10 % lf
  \uccode`\9=13 % cr
\def\x#1#2{\endgroup
%    \end{macrocode}
%    \begin{macro}{\PE@NormalizeLineEnd}
%    \begin{macrocode}
  \def\PE@NormalizeLineEnd##1{%
    \def\PE@result{}%
    \expandafter\PE@@NormalizeLineEnd##1#2\relax
    \let##1\PE@result
  }%
%    \end{macrocode}
%    \end{macro}
%    \begin{macro}{\PE@@NormalizeLineEnd}
%    \begin{macrocode}
  \def\PE@@NormalizeLineEnd##1#2##2{%
    \ifx\relax##2%
      \edef\PE@result{\PE@result##1}%
      \let\PE@next\relax
    \else
      \edef\PE@result{\PE@result##1#1}%
      \ifx#1##2% lf
        \let\PE@next\PE@@NormalizeLineEnd
      \else
        \def\PE@next{\PE@@NormalizeLineEnd##2}%
      \fi
    \fi
    \PE@next
  }%
}%
\uppercase{%
  \x 89%
}
%    \end{macrocode}
%    \end{macro}
%
%    \begin{macrocode}
\begingroup
  \catcode`\|=0 %
  \catcode`\\=12 %
%    \end{macrocode}
%    \begin{macro}{\PE@UnescapeString}
%    \begin{macrocode}
  |gdef|PE@UnescapeString#1{%
    |begingroup
      |def|PE@result{}%
      |expandafter|PE@DeString#1\|relax
    |expandafter|endgroup
    |expandafter|def|expandafter#1|expandafter{|PE@result}%
  }%
%    \end{macrocode}
%    \end{macro}
%    \begin{macro}{\PE@DeString}
%    \begin{macrocode}
  |gdef|PE@DeString#1\#2{%
    |ifx|relax#2%
      |edef|PE@result{|PE@result#1}%
      |let|PE@next|relax
    |else
      |if n#2%
        |uccode|ltx@zero=10 %
      |else|if r#2%
        |uccode|ltx@zero=13 %
      |else|if t#2%
        |uccode|ltx@zero=9 %
      |else|if b#2%
        |uccode|ltx@zero=8 %
      |else|if f#2%
        |uccode|ltx@zero=12 %
      |else
        |uccode|ltx@zero=|ltx@zero
      |fi|fi|fi|fi|fi
      |ifnum|uccode|ltx@zero>|ltx@zero
        |uppercase{%
          |edef|PE@temp{^^@}%
        }%
        |edef|PE@result{|PE@result#1|PE@temp}%
        |let|PE@next|PE@DeString
      |else
        |if\#2% backslash
          |edef|PE@result{|PE@result#1}%
          |let|PE@next|PE@CheckEndBackslash
        |else
          |ifnum`#2=10 % linefeed
            |edef|PE@result{|PE@result#1}%
            |let|PE@next|PE@DeString
          |else
            |ifcase|PE@TestOctDigit#2%
              |edef|PE@result{|PE@result#1}%
              |def|PE@next{|PE@OctI#2}%
            |else
              |edef|PE@result{|PE@result#1#2}%
              |let|PE@next|PE@DeString
            |fi
          |fi
        |fi
      |fi
    |fi
    |PE@next
  }%
%    \end{macrocode}
%    \end{macro}
%    \begin{macro}{\PE@CheckEndBackslash}
%    \begin{macrocode}
  |gdef|PE@CheckEndBackslash#1{%
    |ifx|relax#1%
    |else
      |edef|PE@result{|PE@result\}%
      |expandafter|PE@DeString|expandafter#1%
    |fi
  }%
%    \end{macrocode}
%    \end{macro}
%    \begin{macrocode}
|endgroup
%    \end{macrocode}
%    \begin{macro}{\PE@TestOctDigit}
%    \begin{macrocode}
\def\PE@TestOctDigit#1{%
  \ifnum`#1<48 % 0
    \ltx@one
  \else
    \ifnum`#1>55 % 7
      \ltx@one
    \else
      \ltx@zero
    \fi
  \fi
}
%    \end{macrocode}
%    \end{macro}
%    \begin{macro}{\PE@OctI}
%    \begin{macrocode}
\def\PE@OctI#1#2{%
  \ifcase\PE@TestOctDigit#2%
    \def\PE@next{\PE@OctII{#1#2}}%
  \else
    \def\PE@next{\PE@OctAll#1#2}%
  \fi
  \PE@next
}
%    \end{macrocode}
%    \end{macro}
%    \begin{macro}{\PE@OctII}
%    \begin{macrocode}
\def\PE@OctII#1#2{%
  \ifcase\PE@TestOctDigit#2%
    \def\PE@next{\PE@OctIII#1#2}%
  \else
    \def\PE@next{\PE@OctAll{#1}#2}%
  \fi
  \PE@next
}
%    \end{macrocode}
%    \end{macro}
%    \begin{macrocode}
\ltx@IfUndefined{numexpr}{%
  \catcode`\$=9 %
  \catcode`\&=14 %
}{%
  \catcode`\$=14 %
  \catcode`\&=9 %
}
%    \end{macrocode}
%    \begin{macro}{\PE@OctIII}
%    \begin{macrocode}
\def\PE@OctIII#1#2#3{%
  \ifnum#1<4 %
    \def\PE@next{\PE@OctAll{#1#2#3}}%
  \else
$   \count\ltx@cclv#1 %
$   \advance\count\ltx@cclv -4 %
    \edef\PE@next{%
      \noexpand\PE@OctAll{%
$       \the\count\ltx@cclv
&       \the\numexpr#1-4\relax
        #2#3%
      }%
    }%
  \fi
  \PE@next
}
%    \end{macrocode}
%    \end{macro}
%    \begin{macro}{\PE@OctAll}
%    \begin{macrocode}
\def\PE@OctAll#1{%
  \uccode\ltx@zero='#1\relax
  \uppercase{%
    \edef\PE@result{\PE@result^^@}%
  }%
  \PE@DeString
}
%    \end{macrocode}
%    \end{macro}
%
% \subsection{User macros (\pdfTeX\space analogues)}
%
%    \begin{macrocode}
\begingroup\expandafter\expandafter\expandafter\endgroup
\expandafter\ifx\csname RequirePackage\endcsname\relax
  \def\TMP@RequirePackage#1[#2]{%
    \begingroup\expandafter\expandafter\expandafter\endgroup
    \expandafter\ifx\csname ver@#1.sty\endcsname\relax
      \input #1.sty\relax
    \fi
  }%
  \TMP@RequirePackage{pdftexcmds}[2007/11/11]%
\else
  \RequirePackage{pdftexcmds}[2007/11/11]%
\fi
%    \end{macrocode}
%
%    \begin{macrocode}
\begingroup\expandafter\expandafter\expandafter\endgroup
\expandafter\ifx\csname pdf@escapehex\endcsname\relax
%    \end{macrocode}
%
%    \begin{macro}{\EdefEscapeHex}
%    \begin{macrocode}
  \long\def\EdefEscapeHex#1#2{%
    \EdefSanitize#1{#2}%
    \PE@SanitizeSpaceOther#1%
    \PE@EscapeHex#1%
  }%
%    \end{macrocode}
%    \end{macro}
%    \begin{macro}{\EdefUnescapeHex}
%    \begin{macrocode}
  \def\EdefUnescapeHex#1#2{%
    \EdefSanitize#1{#2}%
    \PE@UnescapeHex#1%
  }%
%    \end{macrocode}
%    \end{macro}
%    \begin{macro}{\EdefEscapeName}
%    \begin{macrocode}
  \long\def\EdefEscapeName#1#2{%
    \EdefSanitize#1{#2}%
    \PE@SanitizeSpaceOther#1%
    \PE@EscapeName#1%
  }%
%    \end{macrocode}
%    \end{macro}
%    \begin{macro}{\EdefEscapeString}
%    \begin{macrocode}
  \long\def\EdefEscapeString#1#2{%
    \EdefSanitize#1{#2}%
    \PE@SanitizeSpaceOther#1%
    \PE@EscapeString#1%
  }%
%    \end{macrocode}
%    \end{macro}
%
%    \begin{macrocode}
\else
%    \end{macrocode}
%
%    \begin{macro}{\PE@edefbabel}
%    Help macro that adds support for \xpackage{babel}'s shorthand
%    characters.
%    \begin{macrocode}
  \long\def\PE@edefbabel#1#2#3{%
    \begingroup
      \csname @save@activestrue\endcsname
      \edef#1{#2{#3}}%
    \expandafter\endgroup
    \expandafter\def\expandafter#1\expandafter{#1}%
  }%
%    \end{macrocode}
%    \end{macro}
%
%    \begin{macro}{\EdefEscapeHex}
%    \begin{macrocode}
  \long\def\EdefEscapeHex#1#2{%
    \PE@edefbabel#1\pdf@escapehex{#2}%
  }%
%    \end{macrocode}
%    \end{macro}
%    \begin{macro}{\EdefUnescapeHex}
%    \begin{macrocode}
  \def\EdefUnescapeHex#1#2{%
    \PE@edefbabel#1\pdf@unescapehex{#2}%
  }%
%    \end{macrocode}
%    \end{macro}
%    \begin{macro}{\EdefEscapeName}
%    \begin{macrocode}
  \long\def\EdefEscapeName#1#2{%
    \PE@edefbabel#1\pdf@escapename{#2}%
  }%
%    \end{macrocode}
%    \end{macro}
%    \begin{macro}{\EdefEscapeString}
%    \begin{macrocode}
  \long\def\EdefEscapeString#1#2{%
    \PE@edefbabel#1\pdf@escapestring{#2}%
  }%
%    \end{macrocode}
%    \end{macro}
%
%    \begin{macrocode}
  \expandafter\PE@AtEnd
\fi%
%    \end{macrocode}
%
% \subsection{Help macros}
%
% \subsubsection{Characters}
%
%    Special characters with catcode 12 (other) are created and
%    stored in macros.
%    \begin{macro}{\PE@hash}
%    \begin{macrocode}
\edef\PE@hash{\string#}
%    \end{macrocode}
%    \end{macro}
%    \begin{macro}{\PE@backslash}
%    \begin{macrocode}
\begingroup
  \escapechar=-1 %
\edef\x{\endgroup
  \def\noexpand\PE@backslash{\string\\}%
}
\x
%    \end{macrocode}
%    \end{macro}
%
% \subsubsection{Switch for \eTeX}
%
%    \begin{macrocode}
\ltx@newif\ifPE@etex
\begingroup\expandafter\expandafter\expandafter\endgroup
\expandafter\ifx\csname numexpr\endcsname\relax
\else
  \PE@etextrue
\fi
%    \end{macrocode}
%
% \subsection{Conversions}
%
% \subsubsection{Conversion to hex string}
%
%    \begin{macro}{\PE@EscapeHex}
%    \begin{macrocode}
\ifPE@etex
  \def\PE@EscapeHex#1{%
    \edef#1{\expandafter\PE@ToHex#1\relax}%
  }%
\else
  \def\PE@EscapeHex#1{%
    \def\PE@result{}%
    \expandafter\PE@ToHex#1\relax
    \let#1\PE@result
  }%
\fi
%    \end{macrocode}
%    \end{macro}
%    \begin{macro}{\PE@ToHex}
%    \begin{macrocode}
\def\PE@ToHex#1{%
  \ifx\relax#1%
  \else
    \PE@HexChar{#1}%
    \expandafter\PE@ToHex
  \fi
}%
%    \end{macrocode}
%    \end{macro}
%    \begin{macro}{\PE@HexChar}
%    \begin{macrocode}
\ifPE@etex
  \def\PE@HexChar#1{%
    \PE@HexDigit{\numexpr\dimexpr.0625\dimexpr`#1sp\relax\relax\relax}%
    \PE@HexDigit{%
      \numexpr`#1-16*\dimexpr.0625\dimexpr`#1sp\relax\relax\relax
    }%
  }%
\else
  \def\PE@HexChar#1{%
    \dimen0=`#1sp%
    \dimen2=.0625\dimen0 %
    \advance\dimen0-16\dimen2 %
    \edef\PE@result{%
      \PE@result
      \PE@HexDigit{\dimen2 }%
      \PE@HexDigit{\dimen0 }%
    }%
  }%
\fi
%    \end{macrocode}
%    \end{macro}
%    \begin{macro}{\PE@HexDigit}
%    \begin{macrocode}
\def\PE@HexDigit#1{%
  \expandafter\string
  \ifcase#1%
    0\or 1\or 2\or 3\or 4\or 5\or 6\or 7\or 8\or 9\or
    A\or B\or C\or D\or E\or F%
  \fi
}
%    \end{macrocode}
%    \end{macro}
%
% \subsubsection{Character code to octal number}
%
%    \begin{macro}{\PE@OctChar}
%    \begin{macrocode}
\ifPE@etex
  \def\PE@OctChar#1{%
    \expandafter\PE@@OctChar
        \the\numexpr\dimexpr.015625\dimexpr`#1sp\relax\relax
            \expandafter\relax
        \expandafter\relax
        \the\numexpr\dimexpr.125\dimexpr`#1sp\relax\relax\relax
        \relax
        #1%
  }%
  \def\PE@@OctChar#1\relax#2\relax#3{%
    \PE@backslash
    #1%
    \the\numexpr#2-8*#1\relax
    \the\numexpr\dimexpr`#3sp\relax-8*#2\relax
  }%
\else
  \def\PE@OctChar#1{%
    \dimen0=`#1sp%
    \dimen2=.125\dimen0 %
    \dimen4=.125\dimen2 %
    \advance\dimen0-8\dimen2 %
    \advance\dimen2-8\dimen4 %
    \edef\PE@result{%
      \PE@result
      \PE@backslash
      \number\dimen4 %
      \number\dimen2 %
      \number\dimen0 %
    }%
  }%
\fi
%    \end{macrocode}
%    \end{macro}
%
% \subsubsection{Unpack hex string}
%
%    \begin{macro}{\PE@UnescapeHex}
%    \begin{macrocode}
\def\PE@UnescapeHex#1{%
  \begingroup
    \PE@InitUccodeHexDigit
    \def\PE@result{}%
    \expandafter\PE@DeHex#1\relax\relax
  \expandafter\endgroup
  \expandafter\def\expandafter#1\expandafter{\PE@result}%
}
%    \end{macrocode}
%    \end{macro}
%    \begin{macro}{\PE@DeHex}
%    \begin{macrocode}
\def\PE@DeHex#1#2{%
  \ifx#2\relax
    \ifx#1\relax
      \let\PE@next\relax
    \else
      \uppercase{%
        \def\PE@testA{#1}%
      }%
      \ifcase\expandafter\PE@TestUcHexDigit\PE@testA
        \def\PE@next{%
          \PE@DeHex#10\relax\relax
        }%
      \else
        \let\PE@next\relax
      \fi
    \fi
  \else
    \uppercase{%
      \def\PE@testA{#1}%
      \def\PE@testB{#2}%
    }%
    \ifcase\expandafter\PE@TestUcHexDigit\PE@testA
      \ifcase\expandafter\PE@TestUcHexDigit\PE@testB
        \uccode\ltx@zero="\PE@testA\PE@testB\relax
        \ifnum\uccode\ltx@zero=32 %
          \let\PE@temp\PE@space@space
        \else
          \uppercase{%
            \def\PE@temp{^^@}%
          }%
        \fi
        \edef\PE@result{\PE@result\PE@temp}%
        \let\PE@next\PE@DeHex
      \else
        % invalid input sequence
        \def\PE@next{%
          \PE@DeHex#1%
        }%
      \fi
    \else
      % invalid input sequence
      \def\PE@next{\PE@DeHex#2}%
    \fi
  \fi
  \PE@next
}
%    \end{macrocode}
%    \end{macro}
%
% \subsubsection{Conversion to PDF name}
%
%    \begin{macro}{\PE@EscapeName}
%    \begin{macrocode}
\ifPE@etex
  \def\PE@EscapeName#1{%
    \edef#1{\expandafter\PE@EscapeNameTokens#1\relax}%
  }%
\else
  \def\PE@EscapeName#1{%
    \def\PE@result{}%
    \expandafter\PE@EscapeNameTokens#1\relax
    \let#1\PE@result
  }%
\fi
%    \end{macrocode}
%    \end{macro}
%    \begin{macro}{\PE@EscapeNameTokens}
%    \begin{macrocode}
\def\PE@EscapeNameTokens#1{%
  \ifx\relax#1%
  \else
    \ifnum`#1<33 %
      \ifcase`#1 %
        % drop illegal zero
      \else
        \PE@EscapeNameAdd\PE@hash
        \PE@HexChar#1%
      \fi
    \else
      \ifnum`#1>126 %
        \PE@EscapeNameAdd\PE@hash
        \PE@HexChar#1%
      \else  \ifnum`#1=35  \PE@EscapeNameHashChar 23% #
        \else\ifnum`#1=37  \PE@EscapeNameHashChar 25% %
        \else\ifnum`#1=40  \PE@EscapeNameHashChar 28% (
        \else\ifnum`#1=41  \PE@EscapeNameHashChar 29% )
        \else\ifnum`#1=47  \PE@EscapeNameHashChar 2F% /
        \else\ifnum`#1=60  \PE@EscapeNameHashChar 3C% <
        \else\ifnum`#1=62  \PE@EscapeNameHashChar 3E% >
        \else\ifnum`#1=91  \PE@EscapeNameHashChar 5B% [
        \else\ifnum`#1=93  \PE@EscapeNameHashChar 5D% ]
        \else\ifnum`#1=123 \PE@EscapeNameHashChar 7B% {
        \else\ifnum`#1=125 \PE@EscapeNameHashChar 7D% }
        \else
          \PE@EscapeNameAdd{#1}%
        \fi\fi\fi\fi\fi\fi\fi\fi\fi\fi\fi
      \fi
    \fi
    \expandafter\PE@EscapeNameTokens
  \fi
}%
\def\PE@EscapeNameHashChar#1#2{%
  \PE@EscapeNameAdd{\PE@hash\string#1\string#2}%
}%
%    \end{macrocode}
%    \end{macro}
%    \begin{macro}{\PE@EscapeNameAdd}
%    \begin{macrocode}
\ifPE@etex
  \def\PE@EscapeNameAdd#1{#1}%
\else
  \def\PE@EscapeNameAdd#1{%
    \edef\PE@result{%
      \PE@result
      #1%
    }%
  }%
\fi
%    \end{macrocode}
%    \end{macro}
%
% \subsubsection{Conversion to PDF string}
%
%    \begin{macro}{\PE@EscapeString}
%    \begin{macrocode}
\ifPE@etex
  \def\PE@EscapeString#1{%
    \edef#1{\expandafter\PE@EscapeStringTokens#1\relax}%
  }%
\else
  \def\PE@EscapeString#1{%
    \begingroup
      \def\PE@result{}%
      \expandafter\PE@EscapeStringTokens#1\relax
    \expandafter\endgroup
    \expandafter\def\expandafter#1\expandafter{\PE@result}%
  }%
\fi
%    \end{macrocode}
%    \end{macro}
%    \begin{macro}{\PE@EscapeStringTokens}
%    \begin{macrocode}
\def\PE@EscapeStringTokens#1{%
  \ifx\relax#1%
  \else
    \ifnum`#1<33 %
      \PE@OctChar#1%
    \else
      \ifnum`#1>126 %
        \PE@OctChar#1%
      \else  \ifnum`#1=40 \PE@EscapeStringAdd{\string\(}% (
        \else\ifnum`#1=41 \PE@EscapeStringAdd{\string\)}% )
        \else\ifnum`#1=92 \PE@EscapeStringAdd{\string\\}% \
        \else
          \PE@EscapeStringAdd{#1}%
        \fi\fi\fi
      \fi
    \fi
    \expandafter\PE@EscapeStringTokens
  \fi
}%
%    \end{macrocode}
%    \end{macro}
%    \begin{macro}{\PE@EscapeStringAdd}
%    \begin{macrocode}
\ifPE@etex
  \def\PE@EscapeStringAdd#1{#1}%
\else
  \def\PE@EscapeStringAdd#1{%
    \edef\PE@result{%
      \PE@result
      #1%
    }%
  }%
\fi
%    \end{macrocode}
%    \end{macro}
%
%    \begin{macrocode}
\PE@AtEnd%
%    \end{macrocode}
%
%    \begin{macrocode}
%</package>
%    \end{macrocode}
%
% \section{Test}
%
% \subsection{Catcode checks for loading}
%
%    \begin{macrocode}
%<*test1>
%    \end{macrocode}
%    \begin{macrocode}
\catcode`\{=1 %
\catcode`\}=2 %
\catcode`\#=6 %
\catcode`\@=11 %
\expandafter\ifx\csname count@\endcsname\relax
  \countdef\count@=255 %
\fi
\expandafter\ifx\csname @gobble\endcsname\relax
  \long\def\@gobble#1{}%
\fi
\expandafter\ifx\csname @firstofone\endcsname\relax
  \long\def\@firstofone#1{#1}%
\fi
\expandafter\ifx\csname loop\endcsname\relax
  \expandafter\@firstofone
\else
  \expandafter\@gobble
\fi
{%
  \def\loop#1\repeat{%
    \def\body{#1}%
    \iterate
  }%
  \def\iterate{%
    \body
      \let\next\iterate
    \else
      \let\next\relax
    \fi
    \next
  }%
  \let\repeat=\fi
}%
\def\RestoreCatcodes{}
\count@=0 %
\loop
  \edef\RestoreCatcodes{%
    \RestoreCatcodes
    \catcode\the\count@=\the\catcode\count@\relax
  }%
\ifnum\count@<255 %
  \advance\count@ 1 %
\repeat

\def\RangeCatcodeInvalid#1#2{%
  \count@=#1\relax
  \loop
    \catcode\count@=15 %
  \ifnum\count@<#2\relax
    \advance\count@ 1 %
  \repeat
}
\def\RangeCatcodeCheck#1#2#3{%
  \count@=#1\relax
  \loop
    \ifnum#3=\catcode\count@
    \else
      \errmessage{%
        Character \the\count@\space
        with wrong catcode \the\catcode\count@\space
        instead of \number#3%
      }%
    \fi
  \ifnum\count@<#2\relax
    \advance\count@ 1 %
  \repeat
}
\def\space{ }
\expandafter\ifx\csname LoadCommand\endcsname\relax
  \def\LoadCommand{\input pdfescape.sty\relax}%
\fi
\def\Test{%
  \RangeCatcodeInvalid{0}{47}%
  \RangeCatcodeInvalid{58}{64}%
  \RangeCatcodeInvalid{91}{96}%
  \RangeCatcodeInvalid{123}{255}%
  \catcode`\@=12 %
  \catcode`\\=0 %
  \catcode`\%=14 %
  \LoadCommand
  \RangeCatcodeCheck{0}{36}{15}%
  \RangeCatcodeCheck{37}{37}{14}%
  \RangeCatcodeCheck{38}{47}{15}%
  \RangeCatcodeCheck{48}{57}{12}%
  \RangeCatcodeCheck{58}{63}{15}%
  \RangeCatcodeCheck{64}{64}{12}%
  \RangeCatcodeCheck{65}{90}{11}%
  \RangeCatcodeCheck{91}{91}{15}%
  \RangeCatcodeCheck{92}{92}{0}%
  \RangeCatcodeCheck{93}{96}{15}%
  \RangeCatcodeCheck{97}{122}{11}%
  \RangeCatcodeCheck{123}{255}{15}%
  \RestoreCatcodes
}
\Test
\csname @@end\endcsname
\end
%    \end{macrocode}
%    \begin{macrocode}
%</test1>
%    \end{macrocode}
%
% \subsection{Macro tests}
%
%    \begin{macrocode}
%<*test2|test3|test4|test5>
\NeedsTeXFormat{LaTeX2e}
\makeatletter
%    \end{macrocode}
%
% \subsection{Test with \cs{pdfescape...} commands}
%
%    \begin{macrocode}
%<*test2>
\ProvidesFile{pdfescape-test2.tex}%
    [2011/11/25 v1.13 Test with \string\pdfescape... commands]%
%</test2>
%    \end{macrocode}
%
% \subsection{Test without \cs{pdfescape...}, with \eTeX}
%
%    \begin{macrocode}
%<*test3>
\ProvidesFile{pdfescape-test3.tex}%
    [2011/11/25 v1.13 Test without \string\pdfescape..., with e-TeX]%
%</test3>
%    \end{macrocode}
%
% \subsection{Test without \cs{pdfescape...} and \eTeX}
%
%    \begin{macrocode}
%<*test4>
\ProvidesFile{pdfescape-test4.tex}%
    [2011/11/25 v1.13 Test without \string\pdfescape... and e-TeX]%
%</test4>
%    \end{macrocode}
%
% \subsection{Test with \LuaTeX}
%
%    \begin{macrocode}
%<*test5>
\ProvidesFile{pdfescape-test5.tex}%
    [2011/11/25 v1.13 Test with LuaTeX]%
%</test5>
%    \end{macrocode}
%
% \subsection{Check/ensure test preconditions}
%
% \subsubsection{Check \cs{pdfescape...}}
%
%    \begin{macrocode}
%<*test2>
\@ifundefined{pdfescapehex}{%
  \PackageError{pdfescape-test2}{%
    Missing \string\pdfescape... commands%
  }{Test aborted.}%
  \stop
}{}
%</test2>
%    \end{macrocode}
%    \begin{macrocode}
%<*test3|test4>
\let\pdfescapehex\@undefined
\let\pdfunescapehex\@undefined
\let\pdfescapename\@undefined
\let\pdfescapestring\@undefined
%</test3|test4>
%    \end{macrocode}
%
% \subsubsection{Check \eTeX}
%
%    \begin{macrocode}
%<*test3>
\@ifundefined{numexpr}{%
  \PackageError{pdfescape-test3}{%
    Missing \eTeX
  }{Test aborted.}%
  \stop
}{}
%</test3>
%    \end{macrocode}
% Package \xpackage{qstest} uses \eTeX, thus \eTeX's features
% can only be disabled later during loading of package \xpackage{pdfescape}.
%
% \subsubsection{Check \LuaTeX}
%
%    \begin{macrocode}
%<*test5>
\@ifundefined{directlua}{%
  \PackageError{pdfescape-test5}{%
    Missing LuaTeX%
  }{Test aborted.}%
  \stop
}{}
%</test5>
%    \end{macrocode}
%
% \subsection{Common part}
%
%    The files for testing uses the framework, provided by the
%    new package \xpackage{qstest} of David Kastrup.
%    \begin{macrocode}
\RequirePackage{qstest}
\IncludeTests{*}
\LogTests{log}{*}{*}

\newcommand*{\ExpectVar}[2]{%
  \ifx#1#2%
  \else
    \begingroup
      \@onelevel@sanitize#1%
      \@onelevel@sanitize#2%
      \typeout{[#1] <> [#2]}% hash-ok
    \endgroup
  \fi
  \Expect*{\ifx#1#2true\else false\fi}{true}%
}

\makeatletter
\begingroup
  \gdef\AllBytes{}%
  \count@=0 %
  \catcode0=12 %
  \@whilenum\count@<256 \do{%
    \lccode0=\count@
    \ifnum\count@=32 %
      \xdef\AllBytes{\AllBytes\space}%
    \else
      \lowercase{%
        \xdef\AllBytes{\AllBytes^^@}%
      }%
    \fi
    \advance\count@ by 1 %
  }%
\endgroup
\newcommand*{\AllBytesHex}{%
  000102030405060708090A0B0C0D0E0F%
  101112131415161718191A1B1C1D1E1F%
  202122232425262728292A2B2C2D2E2F%
  303132333435363738393A3B3C3D3E3F%
  404142434445464748494A4B4C4D4E4F%
  505152535455565758595A5B5C5D5E5F%
  606162636465666768696A6B6C6D6E6F%
  707172737475767778797A7B7C7D7E7F%
  808182838485868788898A8B8C8D8E8F%
  909192939495969798999A9B9C9D9E9F%
  A0A1A2A3A4A5A6A7A8A9AAABACADAEAF%
  B0B1B2B3B4B5B6B7B8B9BABBBCBDBEBF%
  C0C1C2C3C4C5C6C7C8C9CACBCCCDCECF%
  D0D1D2D3D4D5D6D7D8D9DADBDCDDDEDF%
  E0E1E2E3E4E5E6E7E8E9EAEBECEDEEEF%
  F0F1F2F3F4F5F6F7F8F9FAFBFCFDFEFF%
}
\@onelevel@sanitize\AllBytesHex
\expandafter\lowercase\expandafter{%
  \expandafter\newcommand\expandafter*\expandafter\AllBytesHexLC
      \expandafter{\AllBytesHex}%
}
\newcommand*{\AllBytesName}{}
\begingroup
  \catcode`\#=12 %
  \xdef\AllBytesName{%
    #01#02#03#04#05#06#07#08#09#0A#0B#0C#0D#0E#0F%
    #10#11#12#13#14#15#16#17#18#19#1A#1B#1C#1D#1E#1F%
    #20!"#23$#25&'#28#29*+,-.#2F%
    0123456789:;#3C=#3E?%
    @ABCDEFGHIJKLMNO%
    PQRSTUVWXYZ#5B\@backslashchar#5D^_%
    `abcdefghijklmno%
    pqrstuvwxyz#7B|#7D\string~#7F%
    #80#81#82#83#84#85#86#87#88#89#8A#8B#8C#8D#8E#8F%
    #90#91#92#93#94#95#96#97#98#99#9A#9B#9C#9D#9E#9F%
    #A0#A1#A2#A3#A4#A5#A6#A7#A8#A9#AA#AB#AC#AD#AE#AF%
    #B0#B1#B2#B3#B4#B5#B6#B7#B8#B9#BA#BB#BC#BD#BE#BF%
    #C0#C1#C2#C3#C4#C5#C6#C7#C8#C9#CA#CB#CC#CD#CE#CF%
    #D0#D1#D2#D3#D4#D5#D6#D7#D8#D9#DA#DB#DC#DD#DE#DF%
    #E0#E1#E2#E3#E4#E5#E6#E7#E8#E9#EA#EB#EC#ED#EE#EF%
    #F0#F1#F2#F3#F4#F5#F6#F7#F8#F9#FA#FB#FC#FD#FE#FF%
  }%
\endgroup
\@onelevel@sanitize\AllBytesName

\newcommand*{\AllBytesString}{}
\begingroup
  \def\|{|}%
  \edef\%{\@percentchar}%
  \catcode`\|=0 %
  \catcode`\#=12 %
  \catcode`\~=12 %
  \catcode`\\=12 %
  |xdef|AllBytesString{%
    \000\001\002\003\004\005\006\007\010\011\012\013\014\015\016\017%
    \020\021\022\023\024\025\026\027\030\031\032\033\034\035\036\037%
    \040!"#$|%&'\(\)*+,-./%
    0123456789:;<=>?%
    @ABCDEFGHIJKLMNO%
    PQRSTUVWXYZ[\\]^_%
    `abcdefghijklmno%
    pqrstuvwxyz{||}~\177%
    \200\201\202\203\204\205\206\207\210\211\212\213\214\215\216\217%
    \220\221\222\223\224\225\226\227\230\231\232\233\234\235\236\237%
    \240\241\242\243\244\245\246\247\250\251\252\253\254\255\256\257%
    \260\261\262\263\264\265\266\267\270\271\272\273\274\275\276\277%
    \300\301\302\303\304\305\306\307\310\311\312\313\314\315\316\317%
    \320\321\322\323\324\325\326\327\330\331\332\333\334\335\336\337%
    \340\341\342\343\344\345\346\347\350\351\352\353\354\355\356\357%
    \360\361\362\363\364\365\366\367\370\371\372\373\374\375\376\377%
  }%
|endgroup
\@onelevel@sanitize\AllBytesString

%<*test4>
\let\org@detokenize\detokenize
\let\detokenize\@undefined
\let\org@numexpr\numexpr
\let\numexpr\@undefined
%</test4>
\RequirePackage{pdfescape}
%<*test4>
\let\detokenize\org@detokenize
\let\numexpr\org@numexpr
%</test4>

\begin{qstest}{all-hex}{\AllBytes, escapehex}
  \EdefEscapeHex\x{\AllBytes}%
  \Expect*{\x}*{\AllBytesHex}%
  \ExpectVar\x\AllBytesHex
\end{qstest}

\begin{qstest}{all-unhex}{\AllBytesHex, unescapehex}
  \EdefUnescapeHex\x{\AllBytesHex}%
  \Expect*{\x}*{\AllBytes}%
  \ExpectVar\x\AllBytes
\end{qstest}

\begin{qstest}{all-unhex-lc}{\AllBytesHexLC, unescapehex, lowercase}
  \EdefUnescapeHex\x{\AllBytesHexLC}%
  \Expect*{\x}*{\AllBytes}%
  \ExpectVar\x\AllBytes
\end{qstest}

\begin{qstest}{unhex-incomplete}{unescapehex, incomplete}
  \EdefUnescapeHex\x{4}%
  \Expect*{\x}{@}%
\end{qstest}

\begin{qstest}{unhex-space}{unescapehex, space}
  \EdefUnescapeHex\x{20}%
  \Expect*{\x}{ }%
  \ExpectVar\x\space
\end{qstest}

\begin{qstest}{unhex-spaces}{unescapehex, spaces}
  \EdefUnescapeHex\x{204020204120}%
  \def\y#1{%
    \edef\z{#1\string @#1#1\string A#1}%
  }\y{ }%
  \Expect*{\x}*{\z}%
  \ExpectVar\x\z
\end{qstest}

\begin{qstest}{unhex-hash}{unescapehex, hash}
  \catcode`\#=12 %
  \EdefUnescapeHex\x{#20}%
  \ExpectVar\x\space
\end{qstest}

\begin{qstest}{unhex-invalid}{unescapehex, invalid}
  \def\test#1#2{%
    \EdefUnescapeHex\x{#1}%
    \edef\y{#2}%
    \@onelevel@sanitize\y
    \ExpectVar\x\y
  }%
%<*test2>
  \edef\x{\pdfunescapehex{4X}}%
  \edef\y{\string @}%
  \ifx\x\y
  \else
    \def~{\space}%
    \typeout{************************************}%
    \typeout{* Your pdfTeX contains bug 777.~~~~*}%
    \typeout{* This test is redefined as dummy, *}%
    \typeout{* because it triggers the bug.~~~~~*}%
    \typeout{************************************}%
    \def\test#1#2{}%
  \fi
%</test2>
  \test{X}{}%
  \test{XY}{}%
  \test{XYZ}{}%
  \test{A}{^^a0}%
  \test{AX}{^^a0}%
  \test{XA}{^^a0}%
  \test{XXAXX}{^^a0}%
\end{qstest}

\begin{qstest}{all-name}{\AllBytes, escapename}
  \EdefEscapeName\x{\AllBytes}%
  \Expect*{\x}*{\AllBytesName}%
  \ExpectVar\x\AllBytesName
\end{qstest}

\begin{qstest}{all-string}{\AllBytes, escapestring}
  \EdefEscapeString\x{\AllBytes}%
  \Expect*{\x}*{\AllBytesString}%
  \ExpectVar\x\AllBytesString
\end{qstest}

\begin{qstest}{uchexdigit}{unescape, uppercase hex digit}
  \catcode`\@=11 %
  \catcode0=12 %
  \def\test#1#2{%
    \uccode0=#1\relax
    \uppercase{%
      \def\x{^^@}%
    }%
    \Expect*{%
       \ifcase\expandafter\PE@TestUcHexDigit\x
         true%
       \else
         false%
       \fi
    }{#2}%
  }%
  \def\range#1#2#3{%
    \count0=#1\relax
    \loop
    \ifnum\count0<#2\relax
      \test{\count0}{#3}%
      \advance\count0 by 1 %
    \repeat
  }%
  \range{0}{47}{false}%
  \range{48}{57}{true}%
  \range{58}{64}{false}%
  \range{65}{70}{true}%
  \range{71}{255}{false}%
\end{qstest}

\begin{qstest}{unescapename}{unescapename}
  \def\test#1#2{%
    \EdefUnescapeName\x{#1}%
    \edef\y{#2}%
    \@onelevel@sanitize\y
    \ExpectVar\x\y
  }%
  \catcode`\#=12 %
  \catcode0=12 %
  \test{}{}%
  \test{x}{x}%
  \test{xy}{xy}%
  \test{#}{#}%
  \test{##}{##}%
  \test{###}{###}%
  \test{####}{####}%
  \test{#x}{#x}%
  \test{#xy}{#xy}%
  \test{#1}{#1}%
  \test{#40}{@}%
  \test{#400}{@0}%
  \test{#4x0}{#4x0}%
  \test{#ab}{^^ab}%
  \test{#00}{^^@}%
  \test{x#40y#40z}{x@y@z}%
  \test{#40#40#40#40}{@@@@}%
  \test{a#x}{a#x}%
  \test{a#xy}{a#xy}%
  \test{a#1}{a#1}%
  \test{a#40}{a@}%
  \test{a#400}{a@0}%
  \test{#20}{ }%
  \test{a#20}{a }%
  \test{a#20b}{a b}%
  \test{a#20#20#20b}{a \space\space b}%
\end{qstest}

\begin{qstest}{unescapestring}{unescapestring}
  \def\test#1#2{%
    \EdefUnescapeString\x{#1}%
    \edef\y{#2}%
    \@onelevel@sanitize\y
    \ExpectVar\x\y
  }%
  \catcode0=12 %
  \def\DefChar#1#2{%
    \begingroup
      \uccode0=#2\relax
    \uppercase{\endgroup
      \def#1{^^@}%
    }%
  }%
  \DefChar\nul{0}%
  \DefChar\one{1}%
  \DefChar\bel{8}%
  \DefChar\tab{9}%
  \DefChar\lf{10}%
  \DefChar\ff{12}%
  \DefChar\cr{13}%
  \DefChar\\{92}%
  \test{}{}%
  \test{a}{a}%
  \test{\\}{}%
  \test{\\\\}{\\}%
  \test{\\\\y}{\\y}%
  \test{\\000}{\nul}%
  \test{\\b}{\bel}%
  \test{\\t}{\tab}%
  \test{\\n}{\lf}%
  \test{\\f}{\ff}%
  \test{\\r}{\cr}%
  \test{\\(}{(}%
  \test{\\)}{)}%
  \test{\\040}{ }%
  \test{\\100}{@}%
  \test{\\40}{ }%
  \test{\\1}{\one}%
  \test{\\01}{\one}%
  \test{\\001}{\one}%
  \test{\\18}{\one8}%
  \test{\\018}{\one8}%
  \test{\\0018}{\one8}%
  \test{x\\}{x}%
  \test{x\\\\}{x\\}%
  \test{x\\\\y}{x\\y}%
  \test{x\\000}{x\nul}%
  \test{x\\b}{x\bel}%
  \test{x\\t}{x\tab}%
  \test{x\\n}{x\lf}%
  \test{x\\f}{x\ff}%
  \test{x\\r}{x\cr}%
  \test{x\\(}{x(}%
  \test{x\\)}{x)}%
  \test{x\\040}{x }%
  \test{x\\100}{x@}%
  \test{x\\40}{x }%
  \test{x\\1}{x\one}%
  \test{x\\01}{x\one}%
  \test{x\\001}{x\one}%
  \test{x\\18}{x\one8}%
  \test{x\\018}{x\one8}%
  \test{x\\0018}{x\one8}%
  \test{\\b\\t\\n\\f\\r\\(\\)\\\\\\000\\040}{%
    \bel\tab\lf\ff\cr()\\\nul\space
  }%
  \test{\\\lf}{}%
  \test{x\\\lf}{x}%
  \test{\cr}{\lf}%
  \test{\cr\lf}{\lf}%
  \test{\lf}{\lf}%
  \test{\lf\cr}{\lf\lf}%
  \test{x\cr}{x\lf}%
  \test{x\cr\lf}{x\lf}%
  \test{x\lf}{x\lf}%
  \test{x\lf\cr}{x\lf\lf}%
  \test{x\\\cr\lf y\cr}{xy\lf}%
%
  \test{\\409}{ 9}%
  \test{\\800}{800}%
  \test{\\900}{900}%
  \test{\\578}{/8}%
  \test{\\477}{?}%
  \test{\\377}{^^ff}%
  \test{\\777}{^^ff}%
  \test{\\7777}{^^ff7}%
\end{qstest}
\stop
%    \end{macrocode}
%
%    \begin{macrocode}
%</test2|test3|test4|test5>
%    \end{macrocode}
%
% \subsubsection{Test for \hologo{iniTeX}}
%
%    \begin{macrocode}
%<*test6>
\input pdfescape.sty\relax
\catcode`\{=1 %
\catcode`\}=2 %
\catcode`\#=6 %
\catcode`\^=7 %
\catcode`\@=11 %
%    \end{macrocode}
%    \begin{macrocode}
\begingroup
  \catcode`\@=11 %
  \countdef\count@=255 %
  \def\space{ }%
  \long\def\@whilenum#1\do #2{%
    \ifnum #1\relax
      #2\relax
      \@iwhilenum{#1\relax#2\relax}%
    \fi
  }%
  \long\def\@iwhilenum#1{%
    \ifnum #1%
      \expandafter\@iwhilenum
    \else
      \expandafter\ltx@gobble
    \fi
    {#1}%
  }%
  \gdef\AllBytes{}%
  \count@=0 %
  \catcode0=12 %
  \@whilenum\count@<256 \do{%
    \lccode0=\count@
    \ifnum\count@=32 %
      \xdef\AllBytes{\AllBytes\space}%
    \else
      \lowercase{%
        \xdef\AllBytes{\AllBytes^^@}%
      }%
    \fi
    \advance\count@ by 1 %
  }%
\endgroup
\def\AllBytesHex{%
  000102030405060708090A0B0C0D0E0F%
  101112131415161718191A1B1C1D1E1F%
  202122232425262728292A2B2C2D2E2F%
  303132333435363738393A3B3C3D3E3F%
  404142434445464748494A4B4C4D4E4F%
  505152535455565758595A5B5C5D5E5F%
  606162636465666768696A6B6C6D6E6F%
  707172737475767778797A7B7C7D7E7F%
  808182838485868788898A8B8C8D8E8F%
  909192939495969798999A9B9C9D9E9F%
  A0A1A2A3A4A5A6A7A8A9AAABACADAEAF%
  B0B1B2B3B4B5B6B7B8B9BABBBCBDBEBF%
  C0C1C2C3C4C5C6C7C8C9CACBCCCDCECF%
  D0D1D2D3D4D5D6D7D8D9DADBDCDDDEDF%
  E0E1E2E3E4E5E6E7E8E9EAEBECEDEEEF%
  F0F1F2F3F4F5F6F7F8F9FAFBFCFDFEFF%
}
\ltx@onelevel@sanitize\AllBytesHex
\expandafter\lowercase\expandafter{%
  \expandafter\def\expandafter\AllBytesHexLC
      \expandafter{\AllBytesHex}%
}
\begingroup
  \catcode`\#=12 %
  \xdef\AllBytesName{%
    #01#02#03#04#05#06#07#08#09#0A#0B#0C#0D#0E#0F%
    #10#11#12#13#14#15#16#17#18#19#1A#1B#1C#1D#1E#1F%
    #20!"#23$#25&'#28#29*+,-.#2F%
    0123456789:;#3C=#3E?%
    @ABCDEFGHIJKLMNO%
    PQRSTUVWXYZ#5B\ltx@backslashchar#5D^_%
    `abcdefghijklmno%
    pqrstuvwxyz#7B|#7D\string~#7F%
    #80#81#82#83#84#85#86#87#88#89#8A#8B#8C#8D#8E#8F%
    #90#91#92#93#94#95#96#97#98#99#9A#9B#9C#9D#9E#9F%
    #A0#A1#A2#A3#A4#A5#A6#A7#A8#A9#AA#AB#AC#AD#AE#AF%
    #B0#B1#B2#B3#B4#B5#B6#B7#B8#B9#BA#BB#BC#BD#BE#BF%
    #C0#C1#C2#C3#C4#C5#C6#C7#C8#C9#CA#CB#CC#CD#CE#CF%
    #D0#D1#D2#D3#D4#D5#D6#D7#D8#D9#DA#DB#DC#DD#DE#DF%
    #E0#E1#E2#E3#E4#E5#E6#E7#E8#E9#EA#EB#EC#ED#EE#EF%
    #F0#F1#F2#F3#F4#F5#F6#F7#F8#F9#FA#FB#FC#FD#FE#FF%
  }%
\endgroup
\ltx@onelevel@sanitize\AllBytesName
\edef\AllBytesFromName{\expandafter\ltx@gobble\AllBytes}

\begingroup
  \def\|{|}%
  \edef\%{\ltx@percentchar}%
  \catcode`\|=0 %
  \catcode`\#=12 %
  \catcode`\~=12 %
  \catcode`\\=12 %
  |xdef|AllBytesString{%
    \000\001\002\003\004\005\006\007\010\011\012\013\014\015\016\017%
    \020\021\022\023\024\025\026\027\030\031\032\033\034\035\036\037%
    \040!"#$|%&'\(\)*+,-./%
    0123456789:;<=>?%
    @ABCDEFGHIJKLMNO%
    PQRSTUVWXYZ[\\]^_%
    `abcdefghijklmno%
    pqrstuvwxyz{||}~\177%
    \200\201\202\203\204\205\206\207\210\211\212\213\214\215\216\217%
    \220\221\222\223\224\225\226\227\230\231\232\233\234\235\236\237%
    \240\241\242\243\244\245\246\247\250\251\252\253\254\255\256\257%
    \260\261\262\263\264\265\266\267\270\271\272\273\274\275\276\277%
    \300\301\302\303\304\305\306\307\310\311\312\313\314\315\316\317%
    \320\321\322\323\324\325\326\327\330\331\332\333\334\335\336\337%
    \340\341\342\343\344\345\346\347\350\351\352\353\354\355\356\357%
    \360\361\362\363\364\365\366\367\370\371\372\373\374\375\376\377%
  }%
|endgroup
\ltx@onelevel@sanitize\AllBytesString
%    \end{macrocode}
%    \begin{macrocode}
\def\msg#{\immediate\write16}
\def\Test#1#2#3{%
  \begingroup
    #1\TestResult{#2}%
    \ifx\TestResult#3%
    \else
      \newlinechar=10 %
      \msg{Expect:^^J#3}%
      \msg{Result:^^J\TestResult}%
      \errmessage{\string#2 -\string#1-> \string#3}%
    \fi
  \endgroup
}
\Test\EdefEscapeHex\AllBytes\AllBytesHex
\Test\EdefUnescapeHex\AllBytesHex\AllBytes
\Test\EdefEscapeName\AllBytes\AllBytesName
\Test\EdefUnescapeName\AllBytesName\AllBytesFromName
\Test\EdefEscapeString\AllBytes\AllBytesString
\Test\EdefUnescapeString\AllBytesString\AllBytes
%    \end{macrocode}
%    \begin{macrocode}
\csname @@end\endcsname\end
%</test6>
%    \end{macrocode}
%
% \section{Installation}
%
% \subsection{Download}
%
% \paragraph{Package.} This package is available on
% CTAN\footnote{\url{ftp://ftp.ctan.org/tex-archive/}}:
% \begin{description}
% \item[\CTAN{macros/latex/contrib/oberdiek/pdfescape.dtx}] The source file.
% \item[\CTAN{macros/latex/contrib/oberdiek/pdfescape.pdf}] Documentation.
% \end{description}
%
%
% \paragraph{Bundle.} All the packages of the bundle `oberdiek'
% are also available in a TDS compliant ZIP archive. There
% the packages are already unpacked and the documentation files
% are generated. The files and directories obey the TDS standard.
% \begin{description}
% \item[\CTAN{install/macros/latex/contrib/oberdiek.tds.zip}]
% \end{description}
% \emph{TDS} refers to the standard ``A Directory Structure
% for \TeX\ Files'' (\CTAN{tds/tds.pdf}). Directories
% with \xfile{texmf} in their name are usually organized this way.
%
% \subsection{Bundle installation}
%
% \paragraph{Unpacking.} Unpack the \xfile{oberdiek.tds.zip} in the
% TDS tree (also known as \xfile{texmf} tree) of your choice.
% Example (linux):
% \begin{quote}
%   |unzip oberdiek.tds.zip -d ~/texmf|
% \end{quote}
%
% \paragraph{Script installation.}
% Check the directory \xfile{TDS:scripts/oberdiek/} for
% scripts that need further installation steps.
% Package \xpackage{attachfile2} comes with the Perl script
% \xfile{pdfatfi.pl} that should be installed in such a way
% that it can be called as \texttt{pdfatfi}.
% Example (linux):
% \begin{quote}
%   |chmod +x scripts/oberdiek/pdfatfi.pl|\\
%   |cp scripts/oberdiek/pdfatfi.pl /usr/local/bin/|
% \end{quote}
%
% \subsection{Package installation}
%
% \paragraph{Unpacking.} The \xfile{.dtx} file is a self-extracting
% \docstrip\ archive. The files are extracted by running the
% \xfile{.dtx} through \plainTeX:
% \begin{quote}
%   \verb|tex pdfescape.dtx|
% \end{quote}
%
% \paragraph{TDS.} Now the different files must be moved into
% the different directories in your installation TDS tree
% (also known as \xfile{texmf} tree):
% \begin{quote}
% \def\t{^^A
% \begin{tabular}{@{}>{\ttfamily}l@{ $\rightarrow$ }>{\ttfamily}l@{}}
%   pdfescape.sty & tex/generic/oberdiek/pdfescape.sty\\
%   pdfescape.pdf & doc/latex/oberdiek/pdfescape.pdf\\
%   test/pdfescape-test1.tex & doc/latex/oberdiek/test/pdfescape-test1.tex\\
%   test/pdfescape-test2.tex & doc/latex/oberdiek/test/pdfescape-test2.tex\\
%   test/pdfescape-test3.tex & doc/latex/oberdiek/test/pdfescape-test3.tex\\
%   test/pdfescape-test4.tex & doc/latex/oberdiek/test/pdfescape-test4.tex\\
%   test/pdfescape-test5.tex & doc/latex/oberdiek/test/pdfescape-test5.tex\\
%   test/pdfescape-test6.tex & doc/latex/oberdiek/test/pdfescape-test6.tex\\
%   pdfescape.dtx & source/latex/oberdiek/pdfescape.dtx\\
% \end{tabular}^^A
% }^^A
% \sbox0{\t}^^A
% \ifdim\wd0>\linewidth
%   \begingroup
%     \advance\linewidth by\leftmargin
%     \advance\linewidth by\rightmargin
%   \edef\x{\endgroup
%     \def\noexpand\lw{\the\linewidth}^^A
%   }\x
%   \def\lwbox{^^A
%     \leavevmode
%     \hbox to \linewidth{^^A
%       \kern-\leftmargin\relax
%       \hss
%       \usebox0
%       \hss
%       \kern-\rightmargin\relax
%     }^^A
%   }^^A
%   \ifdim\wd0>\lw
%     \sbox0{\small\t}^^A
%     \ifdim\wd0>\linewidth
%       \ifdim\wd0>\lw
%         \sbox0{\footnotesize\t}^^A
%         \ifdim\wd0>\linewidth
%           \ifdim\wd0>\lw
%             \sbox0{\scriptsize\t}^^A
%             \ifdim\wd0>\linewidth
%               \ifdim\wd0>\lw
%                 \sbox0{\tiny\t}^^A
%                 \ifdim\wd0>\linewidth
%                   \lwbox
%                 \else
%                   \usebox0
%                 \fi
%               \else
%                 \lwbox
%               \fi
%             \else
%               \usebox0
%             \fi
%           \else
%             \lwbox
%           \fi
%         \else
%           \usebox0
%         \fi
%       \else
%         \lwbox
%       \fi
%     \else
%       \usebox0
%     \fi
%   \else
%     \lwbox
%   \fi
% \else
%   \usebox0
% \fi
% \end{quote}
% If you have a \xfile{docstrip.cfg} that configures and enables \docstrip's
% TDS installing feature, then some files can already be in the right
% place, see the documentation of \docstrip.
%
% \subsection{Refresh file name databases}
%
% If your \TeX~distribution
% (\teTeX, \mikTeX, \dots) relies on file name databases, you must refresh
% these. For example, \teTeX\ users run \verb|texhash| or
% \verb|mktexlsr|.
%
% \subsection{Some details for the interested}
%
% \paragraph{Attached source.}
%
% The PDF documentation on CTAN also includes the
% \xfile{.dtx} source file. It can be extracted by
% AcrobatReader 6 or higher. Another option is \textsf{pdftk},
% e.g. unpack the file into the current directory:
% \begin{quote}
%   \verb|pdftk pdfescape.pdf unpack_files output .|
% \end{quote}
%
% \paragraph{Unpacking with \LaTeX.}
% The \xfile{.dtx} chooses its action depending on the format:
% \begin{description}
% \item[\plainTeX:] Run \docstrip\ and extract the files.
% \item[\LaTeX:] Generate the documentation.
% \end{description}
% If you insist on using \LaTeX\ for \docstrip\ (really,
% \docstrip\ does not need \LaTeX), then inform the autodetect routine
% about your intention:
% \begin{quote}
%   \verb|latex \let\install=y\input{pdfescape.dtx}|
% \end{quote}
% Do not forget to quote the argument according to the demands
% of your shell.
%
% \paragraph{Generating the documentation.}
% You can use both the \xfile{.dtx} or the \xfile{.drv} to generate
% the documentation. The process can be configured by the
% configuration file \xfile{ltxdoc.cfg}. For instance, put this
% line into this file, if you want to have A4 as paper format:
% \begin{quote}
%   \verb|\PassOptionsToClass{a4paper}{article}|
% \end{quote}
% An example follows how to generate the
% documentation with pdf\LaTeX:
% \begin{quote}
%\begin{verbatim}
%pdflatex pdfescape.dtx
%makeindex -s gind.ist pdfescape.idx
%pdflatex pdfescape.dtx
%makeindex -s gind.ist pdfescape.idx
%pdflatex pdfescape.dtx
%\end{verbatim}
% \end{quote}
%
% \section{Catalogue}
%
% The following XML file can be used as source for the
% \href{http://mirror.ctan.org/help/Catalogue/catalogue.html}{\TeX\ Catalogue}.
% The elements \texttt{caption} and \texttt{description} are imported
% from the original XML file from the Catalogue.
% The name of the XML file in the Catalogue is \xfile{pdfescape.xml}.
%    \begin{macrocode}
%<*catalogue>
<?xml version='1.0' encoding='us-ascii'?>
<!DOCTYPE entry SYSTEM 'catalogue.dtd'>
<entry datestamp='$Date$' modifier='$Author$' id='pdfescape'>
  <name>pdfescape</name>
  <caption>Implements pdfTeX's escape features using TeX or e-TeX.</caption>
  <authorref id='auth:oberdiek'/>
  <copyright owner='Heiko Oberdiek' year='2007,2010,2011'/>
  <license type='lppl1.3'/>
  <version number='1.13'/>
  <description>
    This package implements <xref refid='pdftex'>pdfTeX</xref>'s
    escape features (<tt>\pdfescapehex</tt>, <tt>\pdfunescapehex</tt>,
    <tt>\pdfescapename</tt>, <tt>\pdfescapestring</tt>) using TeX or
    e-TeX.
    <p/>
    The package is part of the <xref refid='oberdiek'>oberdiek</xref>
    bundle.
  </description>
  <documentation details='Package documentation'
      href='ctan:/macros/latex/contrib/oberdiek/pdfescape.pdf'/>
  <ctan file='true' path='/macros/latex/contrib/oberdiek/pdfescape.dtx'/>
  <miktex location='oberdiek'/>
  <texlive location='oberdiek'/>
  <install path='/macros/latex/contrib/oberdiek/oberdiek.tds.zip'/>
</entry>
%</catalogue>
%    \end{macrocode}
%
% \begin{History}
%   \begin{Version}{2007/02/21 v1.0}
%   \item
%     First version.
%   \end{Version}
%   \begin{Version}{2007/02/25 v1.1}
%   \item
%     Test files added.
%   \item
%     \cs{EdefUnescapeHex} supports lowercase letters.
%   \item
%     Fix: \verb|\EdefEscapeName{^^@}|
%   \item
%     Fix: \verb|\EdefEscapeName{\string#}|
%   \item
%     Fix for \cs{EdefUnescapeHex} in case of incomplete hex string.
%   \item
%     Fix: \cs{EdefUnescapeHex} generates space tokens with catcode
%     10 (space) in all cases.
%   \item
%     Fix: \cs{EdefEscapeHex} and \cs{EdefEscapeName} now generate
%     tokens with catcode 12 (other) only.
%   \end{Version}
%   \begin{Version}{2007/03/20 v1.2}
%   \item
%     Fix: Wrong year in \cs{ProvidesPackage}.
%   \end{Version}
%   \begin{Version}{2007/04/11 v1.3}
%   \item
%     Line ends sanitized.
%   \end{Version}
%   \begin{Version}{2007/04/21 v1.4}
%   \item
%     \cs{EdefUnescapeName} and \cs{EdefUnescapeString} added.
%   \end{Version}
%   \begin{Version}{2007/08/27 v1.5}
%   \item
%     \cs{EdefSanitize} added (replaces \cs{PE@sanitize}).
%   \end{Version}
%   \begin{Version}{2007/09/09 v1.6}
%   \item
%     Fix in catcode setup.
%   \end{Version}
%   \begin{Version}{2007/10/27 v1.7}
%   \item
%     More efficient \cs{EdefSanitize}.
%   \end{Version}
%   \begin{Version}{2007/11/11 v1.8}
%   \item
%     Use of package \xpackage{pdftexcmds} for \hologo{LuaTeX} support.
%   \end{Version}
%   \begin{Version}{2010/03/01 v1.9}
%   \item
%     Compatibility with \hologo{iniTeX}.
%   \end{Version}
%   \begin{Version}{2010/11/12 v1.10}
%   \item
%     Use of package \xpackage{ltxcmds}.
%   \item
%     Fix for compatibility with \hologo{iniTeX}.
%   \end{Version}
%   \begin{Version}{2011/01/30 v1.11}
%   \item
%     Already loaded package files are not input in \hologo{plainTeX}.
%   \end{Version}
%   \begin{Version}{2011/04/04 v1.12}
%   \item
%     Further fixes for compatibility for \hologo{iniTeX}.
%   \item
%     Test file for \hologo{iniTeX} added.
%   \end{Version}
%   \begin{Version}{2011/11/25 v1.13}
%   \item
%     Higher order bit of octal sequences in \cs{EdefUnescapeString}
%     ignored according to the PDF specification
%     (Bug found by Bruno Le Floch).
%   \end{Version}
% \end{History}
%
% \PrintIndex
%
% \Finale
\endinput

%        (quote the arguments according to the demands of your shell)
%
% Documentation:
%    (a) If pdfescape.drv is present:
%           latex pdfescape.drv
%    (b) Without pdfescape.drv:
%           latex pdfescape.dtx; ...
%    The class ltxdoc loads the configuration file ltxdoc.cfg
%    if available. Here you can specify further options, e.g.
%    use A4 as paper format:
%       \PassOptionsToClass{a4paper}{article}
%
%    Programm calls to get the documentation (example):
%       pdflatex pdfescape.dtx
%       makeindex -s gind.ist pdfescape.idx
%       pdflatex pdfescape.dtx
%       makeindex -s gind.ist pdfescape.idx
%       pdflatex pdfescape.dtx
%
% Installation:
%    TDS:tex/generic/oberdiek/pdfescape.sty
%    TDS:doc/latex/oberdiek/pdfescape.pdf
%    TDS:doc/latex/oberdiek/test/pdfescape-test1.tex
%    TDS:doc/latex/oberdiek/test/pdfescape-test2.tex
%    TDS:doc/latex/oberdiek/test/pdfescape-test3.tex
%    TDS:doc/latex/oberdiek/test/pdfescape-test4.tex
%    TDS:doc/latex/oberdiek/test/pdfescape-test5.tex
%    TDS:doc/latex/oberdiek/test/pdfescape-test6.tex
%    TDS:source/latex/oberdiek/pdfescape.dtx
%
%<*ignore>
\begingroup
  \catcode123=1 %
  \catcode125=2 %
  \def\x{LaTeX2e}%
\expandafter\endgroup
\ifcase 0\ifx\install y1\fi\expandafter
         \ifx\csname processbatchFile\endcsname\relax\else1\fi
         \ifx\fmtname\x\else 1\fi\relax
\else\csname fi\endcsname
%</ignore>
%<*install>
\input docstrip.tex
\Msg{************************************************************************}
\Msg{* Installation}
\Msg{* Package: pdfescape 2011/11/25 v1.13 Implements pdfTeX's escape features (HO)}
\Msg{************************************************************************}

\keepsilent
\askforoverwritefalse

\let\MetaPrefix\relax
\preamble

This is a generated file.

Project: pdfescape
Version: 2011/11/25 v1.13

Copyright (C) 2007, 2010, 2011 by
   Heiko Oberdiek <heiko.oberdiek at googlemail.com>

This work may be distributed and/or modified under the
conditions of the LaTeX Project Public License, either
version 1.3c of this license or (at your option) any later
version. This version of this license is in
   http://www.latex-project.org/lppl/lppl-1-3c.txt
and the latest version of this license is in
   http://www.latex-project.org/lppl.txt
and version 1.3 or later is part of all distributions of
LaTeX version 2005/12/01 or later.

This work has the LPPL maintenance status "maintained".

This Current Maintainer of this work is Heiko Oberdiek.

The Base Interpreter refers to any `TeX-Format',
because some files are installed in TDS:tex/generic//.

This work consists of the main source file pdfescape.dtx
and the derived files
   pdfescape.sty, pdfescape.pdf, pdfescape.ins, pdfescape.drv,
   pdfescape-test1.tex, pdfescape-test2.tex,
   pdfescape-test3.tex, pdfescape-test4.tex,
   pdfescape-test5.tex, pdfescape-test6.tex.

\endpreamble
\let\MetaPrefix\DoubleperCent

\generate{%
  \file{pdfescape.ins}{\from{pdfescape.dtx}{install}}%
  \file{pdfescape.drv}{\from{pdfescape.dtx}{driver}}%
  \usedir{tex/generic/oberdiek}%
  \file{pdfescape.sty}{\from{pdfescape.dtx}{package}}%
  \usedir{doc/latex/oberdiek/test}%
  \file{pdfescape-test1.tex}{\from{pdfescape.dtx}{test1}}%
  \file{pdfescape-test2.tex}{\from{pdfescape.dtx}{test2}}%
  \file{pdfescape-test3.tex}{\from{pdfescape.dtx}{test3}}%
  \file{pdfescape-test4.tex}{\from{pdfescape.dtx}{test4}}%
  \file{pdfescape-test5.tex}{\from{pdfescape.dtx}{test5}}%
  \file{pdfescape-test6.tex}{\from{pdfescape.dtx}{test6}}%
  \nopreamble
  \nopostamble
  \usedir{source/latex/oberdiek/catalogue}%
  \file{pdfescape.xml}{\from{pdfescape.dtx}{catalogue}}%
}

\catcode32=13\relax% active space
\let =\space%
\Msg{************************************************************************}
\Msg{*}
\Msg{* To finish the installation you have to move the following}
\Msg{* file into a directory searched by TeX:}
\Msg{*}
\Msg{*     pdfescape.sty}
\Msg{*}
\Msg{* To produce the documentation run the file `pdfescape.drv'}
\Msg{* through LaTeX.}
\Msg{*}
\Msg{* Happy TeXing!}
\Msg{*}
\Msg{************************************************************************}

\endbatchfile
%</install>
%<*ignore>
\fi
%</ignore>
%<*driver>
\NeedsTeXFormat{LaTeX2e}
\ProvidesFile{pdfescape.drv}%
  [2011/11/25 v1.13 Implements pdfTeX's escape features (HO)]%
\documentclass{ltxdoc}
\usepackage{holtxdoc}[2011/11/22]
\begin{document}
  \DocInput{pdfescape.dtx}%
\end{document}
%</driver>
% \fi
%
% \CheckSum{2516}
%
% \CharacterTable
%  {Upper-case    \A\B\C\D\E\F\G\H\I\J\K\L\M\N\O\P\Q\R\S\T\U\V\W\X\Y\Z
%   Lower-case    \a\b\c\d\e\f\g\h\i\j\k\l\m\n\o\p\q\r\s\t\u\v\w\x\y\z
%   Digits        \0\1\2\3\4\5\6\7\8\9
%   Exclamation   \!     Double quote  \"     Hash (number) \#
%   Dollar        \$     Percent       \%     Ampersand     \&
%   Acute accent  \'     Left paren    \(     Right paren   \)
%   Asterisk      \*     Plus          \+     Comma         \,
%   Minus         \-     Point         \.     Solidus       \/
%   Colon         \:     Semicolon     \;     Less than     \<
%   Equals        \=     Greater than  \>     Question mark \?
%   Commercial at \@     Left bracket  \[     Backslash     \\
%   Right bracket \]     Circumflex    \^     Underscore    \_
%   Grave accent  \`     Left brace    \{     Vertical bar  \|
%   Right brace   \}     Tilde         \~}
%
% \GetFileInfo{pdfescape.drv}
%
% \title{The \xpackage{pdfescape} package}
% \date{2011/11/25 v1.13}
% \author{Heiko Oberdiek\\\xemail{heiko.oberdiek at googlemail.com}}
%
% \maketitle
%
% \begin{abstract}
% This package implements \pdfTeX's escape features
% (\cs{pdfescapehex}, \cs{pdfunescapehex}, \cs{pdfescapename},
% \cs{pdfescapestring}) using \TeX\ or \eTeX.
% \end{abstract}
%
% \tableofcontents
%
% \section{Documentation}
%
% \begin{declcs}{EdefEscapeHex} \M{cmd} \M{string}\\
%  \cs{EdefUnescapeHex} \M{cmd} \M{string}\\
%  \cs{EdefEscapeName} \M{cmd} \M{string}\\
%  \cs{EdefEscapeString} \M{cmd} \M{string}
% \end{declcs}
% These commands converts \meta{string} and stores the result
% in macro \meta{cmd}. The conversion result is the same
% as the conversion of the corresponding \pdfTeX's primitives.
% Note that the argument \meta{string} is expanded before
% the conversion.
%
% For example, if \pdfTeX\ \textgreater= 1.30 is present,
% then \cs{EdefEscapeHex}
% becomes to:
% \begin{quote}
%\begin{verbatim}
%\def\EdefEscapeHex#1#2{%
%  \edef#1{\pdfescapehex{#2}}%
%}
%\end{verbatim}
% \end{quote}
% The package provides implementations for the case that \pdfTeX\ is
% not present (or too old). Even \eTeX\ can be missing, however it is
% used if it is detected.
%
% \paragraph{Babel.}
% The input strings may contain shorthand characters of package
% \xpackage{babel}.
%
% \subsection{Additional unescape macros}
%
% \begin{declcs}{EdefUnescapeName} \M{cmd} \M{string}
% \end{declcs}
% Sequences of a hash sign with two hexadecimal digits are
% converted to the corresponding character (PDF-1.2).
% A hash sign that is not followed by two hexadecimal digits
% is left unchanged. The catcodes in the result string follow
% TeX's conventions. The space has catcode 10 (space) and
% the other characters have catcode 12 (other).
%
% \begin{declcs}{EdefUnescapeString} \M{cmd} \M{string}
% \end{declcs}
% Macro \meta{cmd} stores the unescaped string in \meta{string}.
% All the rules for literal strings are implemented, see
% PDF specification.
% The catcodes in the result string follow TeX's conventions.
%
% \subsection{Sanitizing macro}
%
% \begin{declcs}{EdefSanitize} \M{cmd} \M{string}
% \end{declcs}
% Argument \meta{string} is expanded, converted to a string
% of tokens with catcode 12 (other) and space tokens, and
% stored in macro \meta{cmd}.
%
% \StopEventually{
% }
%
% \section{Implementation}
%
%    \begin{macrocode}
%<*package>
%    \end{macrocode}
%
% \subsection{Reload check and package identification}
%    Reload check, especially if the package is not used with \LaTeX.
%    \begin{macrocode}
\begingroup\catcode61\catcode48\catcode32=10\relax%
  \catcode13=5 % ^^M
  \endlinechar=13 %
  \catcode35=6 % #
  \catcode39=12 % '
  \catcode44=12 % ,
  \catcode45=12 % -
  \catcode46=12 % .
  \catcode58=12 % :
  \catcode64=11 % @
  \catcode123=1 % {
  \catcode125=2 % }
  \expandafter\let\expandafter\x\csname ver@pdfescape.sty\endcsname
  \ifx\x\relax % plain-TeX, first loading
  \else
    \def\empty{}%
    \ifx\x\empty % LaTeX, first loading,
      % variable is initialized, but \ProvidesPackage not yet seen
    \else
      \expandafter\ifx\csname PackageInfo\endcsname\relax
        \def\x#1#2{%
          \immediate\write-1{Package #1 Info: #2.}%
        }%
      \else
        \def\x#1#2{\PackageInfo{#1}{#2, stopped}}%
      \fi
      \x{pdfescape}{The package is already loaded}%
      \aftergroup\endinput
    \fi
  \fi
\endgroup%
%    \end{macrocode}
%    Package identification:
%    \begin{macrocode}
\begingroup\catcode61\catcode48\catcode32=10\relax%
  \catcode13=5 % ^^M
  \endlinechar=13 %
  \catcode35=6 % #
  \catcode39=12 % '
  \catcode40=12 % (
  \catcode41=12 % )
  \catcode44=12 % ,
  \catcode45=12 % -
  \catcode46=12 % .
  \catcode47=12 % /
  \catcode58=12 % :
  \catcode64=11 % @
  \catcode91=12 % [
  \catcode93=12 % ]
  \catcode123=1 % {
  \catcode125=2 % }
  \expandafter\ifx\csname ProvidesPackage\endcsname\relax
    \def\x#1#2#3[#4]{\endgroup
      \immediate\write-1{Package: #3 #4}%
      \xdef#1{#4}%
    }%
  \else
    \def\x#1#2[#3]{\endgroup
      #2[{#3}]%
      \ifx#1\@undefined
        \xdef#1{#3}%
      \fi
      \ifx#1\relax
        \xdef#1{#3}%
      \fi
    }%
  \fi
\expandafter\x\csname ver@pdfescape.sty\endcsname
\ProvidesPackage{pdfescape}%
  [2011/11/25 v1.13 Implements pdfTeX's escape features (HO)]%
%    \end{macrocode}
%
% \subsection{Catcodes}
%
%    \begin{macrocode}
\begingroup\catcode61\catcode48\catcode32=10\relax%
  \catcode13=5 % ^^M
  \endlinechar=13 %
  \catcode123 1 % {
  \catcode125 2 % }
  \catcode64 11 %
  \def\x{\endgroup
    \expandafter\edef\csname PE@AtEnd\endcsname{%
      \endlinechar=\the\endlinechar\relax
      \catcode13=\the\catcode13\relax
      \catcode32=\the\catcode32\relax
      \catcode35=\the\catcode35\relax
      \catcode61=\the\catcode61\relax
      \catcode64=\the\catcode64\relax
      \catcode123=\the\catcode123\relax
      \catcode125=\the\catcode125\relax
    }%
  }%
\x\catcode61\catcode48\catcode32=10\relax%
\catcode13=5 % ^^M
\endlinechar=13 %
\catcode35=6 % #
\catcode64=11 % @
\catcode123=1 % {
\catcode125=2 % }
\def\TMP@EnsureCode#1#2#3{%
  \edef\PE@AtEnd{%
    \PE@AtEnd
    #1#2=\the#1#2\relax
  }%
  #1#2=#3\relax
}
\TMP@EnsureCode\catcode{0}{12}% ^^@
\TMP@EnsureCode\catcode{34}{12}% "
\TMP@EnsureCode\catcode{36}{3}% $
\TMP@EnsureCode\catcode{38}{4}% &
\TMP@EnsureCode\catcode{39}{12}% '
\TMP@EnsureCode\catcode{42}{12}% *
\TMP@EnsureCode\catcode{45}{12}% -
\TMP@EnsureCode\catcode{46}{12}% .
\TMP@EnsureCode\catcode{47}{12}% /
\TMP@EnsureCode\catcode{60}{12}% <
\TMP@EnsureCode\catcode{62}{12}% >
\TMP@EnsureCode\catcode{91}{12}% [
\TMP@EnsureCode\catcode{93}{12}% ]
\TMP@EnsureCode\catcode{94}{7}% ^
\TMP@EnsureCode\catcode{96}{12}% `
\TMP@EnsureCode\uccode{34}{0}% "
\TMP@EnsureCode\uccode{48}{0}% 0
\TMP@EnsureCode\uccode{61}{0}% =
\edef\PE@AtEnd{\PE@AtEnd\noexpand\endinput}
%    \end{macrocode}
%
% \subsection{Load package}
%
%    \begin{macrocode}
\begingroup\expandafter\expandafter\expandafter\endgroup
\expandafter\ifx\csname RequirePackage\endcsname\relax
  \def\TMP@RequirePackage#1[#2]{%
    \begingroup\expandafter\expandafter\expandafter\endgroup
    \expandafter\ifx\csname ver@#1.sty\endcsname\relax
      \input #1.sty\relax
    \fi
  }%
  \TMP@RequirePackage{ltxcmds}[2010/04/08]%
\else
  \RequirePackage{ltxcmds}[2010/04/08]%
\fi
%    \end{macrocode}
%
% \subsection{Sanitizing}
%
%    \begin{macro}{\EdefSanitize}
%    Macro \cs{EdefSanitize} takes |#2|, entirely converts
%    it to token with catcode 12 (other) and stores the
%    result in macro |#1|.
%    \begin{macrocode}
\begingroup\expandafter\expandafter\expandafter\endgroup
\expandafter\ifx\csname detokenize\endcsname\relax
  \long\def\EdefSanitize#1#2{%
    \begingroup
      \csname @safe@activestrue\endcsname
      \edef#1{#2}%
      \PE@onelevel@sanitize#1%
    \expandafter\endgroup
    \expandafter\def\expandafter#1\expandafter{#1}%
  }%
  \begingroup\expandafter\expandafter\expandafter\endgroup
  \expandafter\ifx\csname @onelevel@sanitize\endcsname\relax
    \def\PE@onelevel@sanitize#1{%
      \edef#1{\expandafter\PE@strip@prefix\meaning#1}%
    }%
    \def\PE@strip@prefix#1>{}%
  \else
    \let\PE@onelevel@sanitize\@onelevel@sanitize
  \fi
\else
  \long\def\EdefSanitize#1#2{%
    \begingroup
      \csname @safe@activestrue\endcsname
      \edef#1{#2}%
    \expandafter\endgroup
    \expandafter\def\expandafter#1\expandafter{%
      \detokenize\expandafter{#1}%
    }%
  }%
  \def\PE@onelevel@sanitize#1{%
    \edef#1{\detokenize\expandafter{#1}}%
  }%
\fi
%    \end{macrocode}
%    \end{macro}
%    \begin{macro}{\PE@sanitize}
%    Macro \cs{PE@sanitize} is only defined for
%    compatibility with version 1.4. Its use is deprecated.
%    \begin{macrocode}
\let\PE@sanitize\EdefSanitize
%    \end{macrocode}
%    \end{macro}
%
% \subsubsection{Space characters}
%
%    \begin{macro}{\PE@space@other}
%    \begin{macrocode}
\begingroup
  \catcode`\ =12\relax%
\def\x{\endgroup\def\PE@space@other{ }}\x\relax
%    \end{macrocode}
%    \end{macro}
%    \begin{macro}{\PE@space@space}
%    \begin{macrocode}
\def\PE@space@space{ }
%    \end{macrocode}
%    \end{macro}
%
% \subsubsection{Space normalization}
%
%    \begin{macro}{\PE@SanitizeSpaceOther}
%    \begin{macrocode}
\def\PE@SanitizeSpaceOther#1{%
  \edef#1{\expandafter\PE@SpaceToOther#1 \relax}%
}
%    \end{macrocode}
%    \end{macro}
%
%    \begin{macro}{\PE@SpaceToOther}
%    \begin{macrocode}
\def\PE@SpaceToOther#1 #2\relax{%
  #1%
  \ifx\\#2\\%
  \else
    \PE@space@other
    \ltx@ReturnAfterFi{%
      \PE@SpaceToOther#2\relax
    }%
  \fi
}
%    \end{macrocode}
%    \end{macro}
%
% \subsection{\cs{EdefUnescapeName}}
%
%    \begin{macro}{\EdefUnescapeName}
%    \begin{macrocode}
\def\EdefUnescapeName#1#2{%
  \EdefSanitize#1{#2}%
  \PE@SanitizeSpaceOther#1%
  \PE@UnescapeName#1%
  \PE@onelevel@sanitize#1%
}
%    \end{macrocode}
%    \end{macro}
%
%    \begin{macro}{\PE@UnescapeName}
%    \begin{macrocode}
\begingroup
  \catcode`\$=6  % hash
  \catcode`\#=12 % other
  \gdef\PE@UnescapeName$1{%
    \begingroup
      \PE@InitUccodeHexDigit
      \def\PE@result{}%
      \expandafter\PE@DeName$1#\relax\relax
    \expandafter\endgroup
    \expandafter\def\expandafter$1\expandafter{\PE@result}%
  }%
  \gdef\PE@DeName$1#$2$3{%
    \ifx\relax$2%
      \edef\PE@result{\PE@result$1}%
      \let\PE@next\relax
    \else
      \ifx\relax$3%
        % wrong escape sequence in input
        \edef\PE@result{\PE@result$1#}%
        \let\PE@next\relax
      \else
        \uppercase{%
          \def\PE@testA{$2}%
          \def\PE@testB{$3}%
        }%
        \ifcase\ifcase\expandafter\PE@TestUcHexDigit\PE@testA
                 \ifcase\expandafter\PE@TestUcHexDigit\PE@testB
                   \ltx@zero
                 \else
                   \ltx@one
                 \fi
               \else
                 \ltx@one
               \fi
          \uccode\ltx@zero="\PE@testA\PE@testB\relax
          \uppercase{%
            \def\PE@temp{^^@}%
          }%
          \uccode\ltx@zero=\ltx@zero
          \edef\PE@result{\PE@result$1\PE@temp}%
          \let\PE@next\PE@DeName
        \else
          % wrong escape sequence in input
          \edef\PE@result{\PE@result$1#}%
          \def\PE@next{\PE@DeName$2$3}%
        \fi
      \fi
    \fi
    \PE@next
  }%
\endgroup
%    \end{macrocode}
%    \end{macro}
%
%    \begin{macro}{\PE@InitUccodeHexDigit}
%    \begin{macrocode}
\def\PE@InitUccodeHexDigit{%
  \uccode`a=`A\relax
  \uccode`b=`B\relax
  \uccode`c=`C\relax
  \uccode`d=`D\relax
  \uccode`e=`E\relax
  \uccode`f=`F\relax
  \uccode`A=\ltx@zero
  \uccode`B=\ltx@zero
  \uccode`C=\ltx@zero
  \uccode`D=\ltx@zero
  \uccode`E=\ltx@zero
  \uccode`F=\ltx@zero
  \uccode`0=\ltx@zero
  \uccode`1=\ltx@zero
  \uccode`2=\ltx@zero
  \uccode`3=\ltx@zero
  \uccode`4=\ltx@zero
  \uccode`5=\ltx@zero
  \uccode`6=\ltx@zero
  \uccode`7=\ltx@zero
  \uccode`8=\ltx@zero
  \uccode`9=\ltx@zero
}
%    \end{macrocode}
%    \end{macro}
%
%    \begin{macro}{\PE@TestUcHexDigit}
%    \begin{macrocode}
\def\PE@TestUcHexDigit#1{%
  \ifnum`#1<48 % 0
    \ltx@one
  \else
    \ifnum`#1>70 % F
      \ltx@one
    \else
      \ifnum`#1>57 % 9
        \ifnum`#1<65 % A
          \ltx@one
        \else
          \ltx@zero
        \fi
      \else
        \ltx@zero
      \fi
    \fi
  \fi
}
%    \end{macrocode}
%    \end{macro}
%
% \subsection{\cs{EdefUnescapeString}}
%
%    \begin{macro}{\EdefUnescapeString}
%    \begin{macrocode}
\def\EdefUnescapeString#1#2{%
  \EdefSanitize#1{#2}%
  \PE@SanitizeSpaceOther#1%
  \PE@NormalizeLineEnd#1%
  \PE@UnescapeString#1%
  \PE@onelevel@sanitize#1%
}
%    \end{macrocode}
%    \end{macro}
%
%    \begin{macrocode}
\begingroup
  \uccode`\8=10 % lf
  \uccode`\9=13 % cr
\def\x#1#2{\endgroup
%    \end{macrocode}
%    \begin{macro}{\PE@NormalizeLineEnd}
%    \begin{macrocode}
  \def\PE@NormalizeLineEnd##1{%
    \def\PE@result{}%
    \expandafter\PE@@NormalizeLineEnd##1#2\relax
    \let##1\PE@result
  }%
%    \end{macrocode}
%    \end{macro}
%    \begin{macro}{\PE@@NormalizeLineEnd}
%    \begin{macrocode}
  \def\PE@@NormalizeLineEnd##1#2##2{%
    \ifx\relax##2%
      \edef\PE@result{\PE@result##1}%
      \let\PE@next\relax
    \else
      \edef\PE@result{\PE@result##1#1}%
      \ifx#1##2% lf
        \let\PE@next\PE@@NormalizeLineEnd
      \else
        \def\PE@next{\PE@@NormalizeLineEnd##2}%
      \fi
    \fi
    \PE@next
  }%
}%
\uppercase{%
  \x 89%
}
%    \end{macrocode}
%    \end{macro}
%
%    \begin{macrocode}
\begingroup
  \catcode`\|=0 %
  \catcode`\\=12 %
%    \end{macrocode}
%    \begin{macro}{\PE@UnescapeString}
%    \begin{macrocode}
  |gdef|PE@UnescapeString#1{%
    |begingroup
      |def|PE@result{}%
      |expandafter|PE@DeString#1\|relax
    |expandafter|endgroup
    |expandafter|def|expandafter#1|expandafter{|PE@result}%
  }%
%    \end{macrocode}
%    \end{macro}
%    \begin{macro}{\PE@DeString}
%    \begin{macrocode}
  |gdef|PE@DeString#1\#2{%
    |ifx|relax#2%
      |edef|PE@result{|PE@result#1}%
      |let|PE@next|relax
    |else
      |if n#2%
        |uccode|ltx@zero=10 %
      |else|if r#2%
        |uccode|ltx@zero=13 %
      |else|if t#2%
        |uccode|ltx@zero=9 %
      |else|if b#2%
        |uccode|ltx@zero=8 %
      |else|if f#2%
        |uccode|ltx@zero=12 %
      |else
        |uccode|ltx@zero=|ltx@zero
      |fi|fi|fi|fi|fi
      |ifnum|uccode|ltx@zero>|ltx@zero
        |uppercase{%
          |edef|PE@temp{^^@}%
        }%
        |edef|PE@result{|PE@result#1|PE@temp}%
        |let|PE@next|PE@DeString
      |else
        |if\#2% backslash
          |edef|PE@result{|PE@result#1}%
          |let|PE@next|PE@CheckEndBackslash
        |else
          |ifnum`#2=10 % linefeed
            |edef|PE@result{|PE@result#1}%
            |let|PE@next|PE@DeString
          |else
            |ifcase|PE@TestOctDigit#2%
              |edef|PE@result{|PE@result#1}%
              |def|PE@next{|PE@OctI#2}%
            |else
              |edef|PE@result{|PE@result#1#2}%
              |let|PE@next|PE@DeString
            |fi
          |fi
        |fi
      |fi
    |fi
    |PE@next
  }%
%    \end{macrocode}
%    \end{macro}
%    \begin{macro}{\PE@CheckEndBackslash}
%    \begin{macrocode}
  |gdef|PE@CheckEndBackslash#1{%
    |ifx|relax#1%
    |else
      |edef|PE@result{|PE@result\}%
      |expandafter|PE@DeString|expandafter#1%
    |fi
  }%
%    \end{macrocode}
%    \end{macro}
%    \begin{macrocode}
|endgroup
%    \end{macrocode}
%    \begin{macro}{\PE@TestOctDigit}
%    \begin{macrocode}
\def\PE@TestOctDigit#1{%
  \ifnum`#1<48 % 0
    \ltx@one
  \else
    \ifnum`#1>55 % 7
      \ltx@one
    \else
      \ltx@zero
    \fi
  \fi
}
%    \end{macrocode}
%    \end{macro}
%    \begin{macro}{\PE@OctI}
%    \begin{macrocode}
\def\PE@OctI#1#2{%
  \ifcase\PE@TestOctDigit#2%
    \def\PE@next{\PE@OctII{#1#2}}%
  \else
    \def\PE@next{\PE@OctAll#1#2}%
  \fi
  \PE@next
}
%    \end{macrocode}
%    \end{macro}
%    \begin{macro}{\PE@OctII}
%    \begin{macrocode}
\def\PE@OctII#1#2{%
  \ifcase\PE@TestOctDigit#2%
    \def\PE@next{\PE@OctIII#1#2}%
  \else
    \def\PE@next{\PE@OctAll{#1}#2}%
  \fi
  \PE@next
}
%    \end{macrocode}
%    \end{macro}
%    \begin{macrocode}
\ltx@IfUndefined{numexpr}{%
  \catcode`\$=9 %
  \catcode`\&=14 %
}{%
  \catcode`\$=14 %
  \catcode`\&=9 %
}
%    \end{macrocode}
%    \begin{macro}{\PE@OctIII}
%    \begin{macrocode}
\def\PE@OctIII#1#2#3{%
  \ifnum#1<4 %
    \def\PE@next{\PE@OctAll{#1#2#3}}%
  \else
$   \count\ltx@cclv#1 %
$   \advance\count\ltx@cclv -4 %
    \edef\PE@next{%
      \noexpand\PE@OctAll{%
$       \the\count\ltx@cclv
&       \the\numexpr#1-4\relax
        #2#3%
      }%
    }%
  \fi
  \PE@next
}
%    \end{macrocode}
%    \end{macro}
%    \begin{macro}{\PE@OctAll}
%    \begin{macrocode}
\def\PE@OctAll#1{%
  \uccode\ltx@zero='#1\relax
  \uppercase{%
    \edef\PE@result{\PE@result^^@}%
  }%
  \PE@DeString
}
%    \end{macrocode}
%    \end{macro}
%
% \subsection{User macros (\pdfTeX\space analogues)}
%
%    \begin{macrocode}
\begingroup\expandafter\expandafter\expandafter\endgroup
\expandafter\ifx\csname RequirePackage\endcsname\relax
  \def\TMP@RequirePackage#1[#2]{%
    \begingroup\expandafter\expandafter\expandafter\endgroup
    \expandafter\ifx\csname ver@#1.sty\endcsname\relax
      \input #1.sty\relax
    \fi
  }%
  \TMP@RequirePackage{pdftexcmds}[2007/11/11]%
\else
  \RequirePackage{pdftexcmds}[2007/11/11]%
\fi
%    \end{macrocode}
%
%    \begin{macrocode}
\begingroup\expandafter\expandafter\expandafter\endgroup
\expandafter\ifx\csname pdf@escapehex\endcsname\relax
%    \end{macrocode}
%
%    \begin{macro}{\EdefEscapeHex}
%    \begin{macrocode}
  \long\def\EdefEscapeHex#1#2{%
    \EdefSanitize#1{#2}%
    \PE@SanitizeSpaceOther#1%
    \PE@EscapeHex#1%
  }%
%    \end{macrocode}
%    \end{macro}
%    \begin{macro}{\EdefUnescapeHex}
%    \begin{macrocode}
  \def\EdefUnescapeHex#1#2{%
    \EdefSanitize#1{#2}%
    \PE@UnescapeHex#1%
  }%
%    \end{macrocode}
%    \end{macro}
%    \begin{macro}{\EdefEscapeName}
%    \begin{macrocode}
  \long\def\EdefEscapeName#1#2{%
    \EdefSanitize#1{#2}%
    \PE@SanitizeSpaceOther#1%
    \PE@EscapeName#1%
  }%
%    \end{macrocode}
%    \end{macro}
%    \begin{macro}{\EdefEscapeString}
%    \begin{macrocode}
  \long\def\EdefEscapeString#1#2{%
    \EdefSanitize#1{#2}%
    \PE@SanitizeSpaceOther#1%
    \PE@EscapeString#1%
  }%
%    \end{macrocode}
%    \end{macro}
%
%    \begin{macrocode}
\else
%    \end{macrocode}
%
%    \begin{macro}{\PE@edefbabel}
%    Help macro that adds support for \xpackage{babel}'s shorthand
%    characters.
%    \begin{macrocode}
  \long\def\PE@edefbabel#1#2#3{%
    \begingroup
      \csname @save@activestrue\endcsname
      \edef#1{#2{#3}}%
    \expandafter\endgroup
    \expandafter\def\expandafter#1\expandafter{#1}%
  }%
%    \end{macrocode}
%    \end{macro}
%
%    \begin{macro}{\EdefEscapeHex}
%    \begin{macrocode}
  \long\def\EdefEscapeHex#1#2{%
    \PE@edefbabel#1\pdf@escapehex{#2}%
  }%
%    \end{macrocode}
%    \end{macro}
%    \begin{macro}{\EdefUnescapeHex}
%    \begin{macrocode}
  \def\EdefUnescapeHex#1#2{%
    \PE@edefbabel#1\pdf@unescapehex{#2}%
  }%
%    \end{macrocode}
%    \end{macro}
%    \begin{macro}{\EdefEscapeName}
%    \begin{macrocode}
  \long\def\EdefEscapeName#1#2{%
    \PE@edefbabel#1\pdf@escapename{#2}%
  }%
%    \end{macrocode}
%    \end{macro}
%    \begin{macro}{\EdefEscapeString}
%    \begin{macrocode}
  \long\def\EdefEscapeString#1#2{%
    \PE@edefbabel#1\pdf@escapestring{#2}%
  }%
%    \end{macrocode}
%    \end{macro}
%
%    \begin{macrocode}
  \expandafter\PE@AtEnd
\fi%
%    \end{macrocode}
%
% \subsection{Help macros}
%
% \subsubsection{Characters}
%
%    Special characters with catcode 12 (other) are created and
%    stored in macros.
%    \begin{macro}{\PE@hash}
%    \begin{macrocode}
\edef\PE@hash{\string#}
%    \end{macrocode}
%    \end{macro}
%    \begin{macro}{\PE@backslash}
%    \begin{macrocode}
\begingroup
  \escapechar=-1 %
\edef\x{\endgroup
  \def\noexpand\PE@backslash{\string\\}%
}
\x
%    \end{macrocode}
%    \end{macro}
%
% \subsubsection{Switch for \eTeX}
%
%    \begin{macrocode}
\ltx@newif\ifPE@etex
\begingroup\expandafter\expandafter\expandafter\endgroup
\expandafter\ifx\csname numexpr\endcsname\relax
\else
  \PE@etextrue
\fi
%    \end{macrocode}
%
% \subsection{Conversions}
%
% \subsubsection{Conversion to hex string}
%
%    \begin{macro}{\PE@EscapeHex}
%    \begin{macrocode}
\ifPE@etex
  \def\PE@EscapeHex#1{%
    \edef#1{\expandafter\PE@ToHex#1\relax}%
  }%
\else
  \def\PE@EscapeHex#1{%
    \def\PE@result{}%
    \expandafter\PE@ToHex#1\relax
    \let#1\PE@result
  }%
\fi
%    \end{macrocode}
%    \end{macro}
%    \begin{macro}{\PE@ToHex}
%    \begin{macrocode}
\def\PE@ToHex#1{%
  \ifx\relax#1%
  \else
    \PE@HexChar{#1}%
    \expandafter\PE@ToHex
  \fi
}%
%    \end{macrocode}
%    \end{macro}
%    \begin{macro}{\PE@HexChar}
%    \begin{macrocode}
\ifPE@etex
  \def\PE@HexChar#1{%
    \PE@HexDigit{\numexpr\dimexpr.0625\dimexpr`#1sp\relax\relax\relax}%
    \PE@HexDigit{%
      \numexpr`#1-16*\dimexpr.0625\dimexpr`#1sp\relax\relax\relax
    }%
  }%
\else
  \def\PE@HexChar#1{%
    \dimen0=`#1sp%
    \dimen2=.0625\dimen0 %
    \advance\dimen0-16\dimen2 %
    \edef\PE@result{%
      \PE@result
      \PE@HexDigit{\dimen2 }%
      \PE@HexDigit{\dimen0 }%
    }%
  }%
\fi
%    \end{macrocode}
%    \end{macro}
%    \begin{macro}{\PE@HexDigit}
%    \begin{macrocode}
\def\PE@HexDigit#1{%
  \expandafter\string
  \ifcase#1%
    0\or 1\or 2\or 3\or 4\or 5\or 6\or 7\or 8\or 9\or
    A\or B\or C\or D\or E\or F%
  \fi
}
%    \end{macrocode}
%    \end{macro}
%
% \subsubsection{Character code to octal number}
%
%    \begin{macro}{\PE@OctChar}
%    \begin{macrocode}
\ifPE@etex
  \def\PE@OctChar#1{%
    \expandafter\PE@@OctChar
        \the\numexpr\dimexpr.015625\dimexpr`#1sp\relax\relax
            \expandafter\relax
        \expandafter\relax
        \the\numexpr\dimexpr.125\dimexpr`#1sp\relax\relax\relax
        \relax
        #1%
  }%
  \def\PE@@OctChar#1\relax#2\relax#3{%
    \PE@backslash
    #1%
    \the\numexpr#2-8*#1\relax
    \the\numexpr\dimexpr`#3sp\relax-8*#2\relax
  }%
\else
  \def\PE@OctChar#1{%
    \dimen0=`#1sp%
    \dimen2=.125\dimen0 %
    \dimen4=.125\dimen2 %
    \advance\dimen0-8\dimen2 %
    \advance\dimen2-8\dimen4 %
    \edef\PE@result{%
      \PE@result
      \PE@backslash
      \number\dimen4 %
      \number\dimen2 %
      \number\dimen0 %
    }%
  }%
\fi
%    \end{macrocode}
%    \end{macro}
%
% \subsubsection{Unpack hex string}
%
%    \begin{macro}{\PE@UnescapeHex}
%    \begin{macrocode}
\def\PE@UnescapeHex#1{%
  \begingroup
    \PE@InitUccodeHexDigit
    \def\PE@result{}%
    \expandafter\PE@DeHex#1\relax\relax
  \expandafter\endgroup
  \expandafter\def\expandafter#1\expandafter{\PE@result}%
}
%    \end{macrocode}
%    \end{macro}
%    \begin{macro}{\PE@DeHex}
%    \begin{macrocode}
\def\PE@DeHex#1#2{%
  \ifx#2\relax
    \ifx#1\relax
      \let\PE@next\relax
    \else
      \uppercase{%
        \def\PE@testA{#1}%
      }%
      \ifcase\expandafter\PE@TestUcHexDigit\PE@testA
        \def\PE@next{%
          \PE@DeHex#10\relax\relax
        }%
      \else
        \let\PE@next\relax
      \fi
    \fi
  \else
    \uppercase{%
      \def\PE@testA{#1}%
      \def\PE@testB{#2}%
    }%
    \ifcase\expandafter\PE@TestUcHexDigit\PE@testA
      \ifcase\expandafter\PE@TestUcHexDigit\PE@testB
        \uccode\ltx@zero="\PE@testA\PE@testB\relax
        \ifnum\uccode\ltx@zero=32 %
          \let\PE@temp\PE@space@space
        \else
          \uppercase{%
            \def\PE@temp{^^@}%
          }%
        \fi
        \edef\PE@result{\PE@result\PE@temp}%
        \let\PE@next\PE@DeHex
      \else
        % invalid input sequence
        \def\PE@next{%
          \PE@DeHex#1%
        }%
      \fi
    \else
      % invalid input sequence
      \def\PE@next{\PE@DeHex#2}%
    \fi
  \fi
  \PE@next
}
%    \end{macrocode}
%    \end{macro}
%
% \subsubsection{Conversion to PDF name}
%
%    \begin{macro}{\PE@EscapeName}
%    \begin{macrocode}
\ifPE@etex
  \def\PE@EscapeName#1{%
    \edef#1{\expandafter\PE@EscapeNameTokens#1\relax}%
  }%
\else
  \def\PE@EscapeName#1{%
    \def\PE@result{}%
    \expandafter\PE@EscapeNameTokens#1\relax
    \let#1\PE@result
  }%
\fi
%    \end{macrocode}
%    \end{macro}
%    \begin{macro}{\PE@EscapeNameTokens}
%    \begin{macrocode}
\def\PE@EscapeNameTokens#1{%
  \ifx\relax#1%
  \else
    \ifnum`#1<33 %
      \ifcase`#1 %
        % drop illegal zero
      \else
        \PE@EscapeNameAdd\PE@hash
        \PE@HexChar#1%
      \fi
    \else
      \ifnum`#1>126 %
        \PE@EscapeNameAdd\PE@hash
        \PE@HexChar#1%
      \else  \ifnum`#1=35  \PE@EscapeNameHashChar 23% #
        \else\ifnum`#1=37  \PE@EscapeNameHashChar 25% %
        \else\ifnum`#1=40  \PE@EscapeNameHashChar 28% (
        \else\ifnum`#1=41  \PE@EscapeNameHashChar 29% )
        \else\ifnum`#1=47  \PE@EscapeNameHashChar 2F% /
        \else\ifnum`#1=60  \PE@EscapeNameHashChar 3C% <
        \else\ifnum`#1=62  \PE@EscapeNameHashChar 3E% >
        \else\ifnum`#1=91  \PE@EscapeNameHashChar 5B% [
        \else\ifnum`#1=93  \PE@EscapeNameHashChar 5D% ]
        \else\ifnum`#1=123 \PE@EscapeNameHashChar 7B% {
        \else\ifnum`#1=125 \PE@EscapeNameHashChar 7D% }
        \else
          \PE@EscapeNameAdd{#1}%
        \fi\fi\fi\fi\fi\fi\fi\fi\fi\fi\fi
      \fi
    \fi
    \expandafter\PE@EscapeNameTokens
  \fi
}%
\def\PE@EscapeNameHashChar#1#2{%
  \PE@EscapeNameAdd{\PE@hash\string#1\string#2}%
}%
%    \end{macrocode}
%    \end{macro}
%    \begin{macro}{\PE@EscapeNameAdd}
%    \begin{macrocode}
\ifPE@etex
  \def\PE@EscapeNameAdd#1{#1}%
\else
  \def\PE@EscapeNameAdd#1{%
    \edef\PE@result{%
      \PE@result
      #1%
    }%
  }%
\fi
%    \end{macrocode}
%    \end{macro}
%
% \subsubsection{Conversion to PDF string}
%
%    \begin{macro}{\PE@EscapeString}
%    \begin{macrocode}
\ifPE@etex
  \def\PE@EscapeString#1{%
    \edef#1{\expandafter\PE@EscapeStringTokens#1\relax}%
  }%
\else
  \def\PE@EscapeString#1{%
    \begingroup
      \def\PE@result{}%
      \expandafter\PE@EscapeStringTokens#1\relax
    \expandafter\endgroup
    \expandafter\def\expandafter#1\expandafter{\PE@result}%
  }%
\fi
%    \end{macrocode}
%    \end{macro}
%    \begin{macro}{\PE@EscapeStringTokens}
%    \begin{macrocode}
\def\PE@EscapeStringTokens#1{%
  \ifx\relax#1%
  \else
    \ifnum`#1<33 %
      \PE@OctChar#1%
    \else
      \ifnum`#1>126 %
        \PE@OctChar#1%
      \else  \ifnum`#1=40 \PE@EscapeStringAdd{\string\(}% (
        \else\ifnum`#1=41 \PE@EscapeStringAdd{\string\)}% )
        \else\ifnum`#1=92 \PE@EscapeStringAdd{\string\\}% \
        \else
          \PE@EscapeStringAdd{#1}%
        \fi\fi\fi
      \fi
    \fi
    \expandafter\PE@EscapeStringTokens
  \fi
}%
%    \end{macrocode}
%    \end{macro}
%    \begin{macro}{\PE@EscapeStringAdd}
%    \begin{macrocode}
\ifPE@etex
  \def\PE@EscapeStringAdd#1{#1}%
\else
  \def\PE@EscapeStringAdd#1{%
    \edef\PE@result{%
      \PE@result
      #1%
    }%
  }%
\fi
%    \end{macrocode}
%    \end{macro}
%
%    \begin{macrocode}
\PE@AtEnd%
%    \end{macrocode}
%
%    \begin{macrocode}
%</package>
%    \end{macrocode}
%
% \section{Test}
%
% \subsection{Catcode checks for loading}
%
%    \begin{macrocode}
%<*test1>
%    \end{macrocode}
%    \begin{macrocode}
\catcode`\{=1 %
\catcode`\}=2 %
\catcode`\#=6 %
\catcode`\@=11 %
\expandafter\ifx\csname count@\endcsname\relax
  \countdef\count@=255 %
\fi
\expandafter\ifx\csname @gobble\endcsname\relax
  \long\def\@gobble#1{}%
\fi
\expandafter\ifx\csname @firstofone\endcsname\relax
  \long\def\@firstofone#1{#1}%
\fi
\expandafter\ifx\csname loop\endcsname\relax
  \expandafter\@firstofone
\else
  \expandafter\@gobble
\fi
{%
  \def\loop#1\repeat{%
    \def\body{#1}%
    \iterate
  }%
  \def\iterate{%
    \body
      \let\next\iterate
    \else
      \let\next\relax
    \fi
    \next
  }%
  \let\repeat=\fi
}%
\def\RestoreCatcodes{}
\count@=0 %
\loop
  \edef\RestoreCatcodes{%
    \RestoreCatcodes
    \catcode\the\count@=\the\catcode\count@\relax
  }%
\ifnum\count@<255 %
  \advance\count@ 1 %
\repeat

\def\RangeCatcodeInvalid#1#2{%
  \count@=#1\relax
  \loop
    \catcode\count@=15 %
  \ifnum\count@<#2\relax
    \advance\count@ 1 %
  \repeat
}
\def\RangeCatcodeCheck#1#2#3{%
  \count@=#1\relax
  \loop
    \ifnum#3=\catcode\count@
    \else
      \errmessage{%
        Character \the\count@\space
        with wrong catcode \the\catcode\count@\space
        instead of \number#3%
      }%
    \fi
  \ifnum\count@<#2\relax
    \advance\count@ 1 %
  \repeat
}
\def\space{ }
\expandafter\ifx\csname LoadCommand\endcsname\relax
  \def\LoadCommand{\input pdfescape.sty\relax}%
\fi
\def\Test{%
  \RangeCatcodeInvalid{0}{47}%
  \RangeCatcodeInvalid{58}{64}%
  \RangeCatcodeInvalid{91}{96}%
  \RangeCatcodeInvalid{123}{255}%
  \catcode`\@=12 %
  \catcode`\\=0 %
  \catcode`\%=14 %
  \LoadCommand
  \RangeCatcodeCheck{0}{36}{15}%
  \RangeCatcodeCheck{37}{37}{14}%
  \RangeCatcodeCheck{38}{47}{15}%
  \RangeCatcodeCheck{48}{57}{12}%
  \RangeCatcodeCheck{58}{63}{15}%
  \RangeCatcodeCheck{64}{64}{12}%
  \RangeCatcodeCheck{65}{90}{11}%
  \RangeCatcodeCheck{91}{91}{15}%
  \RangeCatcodeCheck{92}{92}{0}%
  \RangeCatcodeCheck{93}{96}{15}%
  \RangeCatcodeCheck{97}{122}{11}%
  \RangeCatcodeCheck{123}{255}{15}%
  \RestoreCatcodes
}
\Test
\csname @@end\endcsname
\end
%    \end{macrocode}
%    \begin{macrocode}
%</test1>
%    \end{macrocode}
%
% \subsection{Macro tests}
%
%    \begin{macrocode}
%<*test2|test3|test4|test5>
\NeedsTeXFormat{LaTeX2e}
\makeatletter
%    \end{macrocode}
%
% \subsection{Test with \cs{pdfescape...} commands}
%
%    \begin{macrocode}
%<*test2>
\ProvidesFile{pdfescape-test2.tex}%
    [2011/11/25 v1.13 Test with \string\pdfescape... commands]%
%</test2>
%    \end{macrocode}
%
% \subsection{Test without \cs{pdfescape...}, with \eTeX}
%
%    \begin{macrocode}
%<*test3>
\ProvidesFile{pdfescape-test3.tex}%
    [2011/11/25 v1.13 Test without \string\pdfescape..., with e-TeX]%
%</test3>
%    \end{macrocode}
%
% \subsection{Test without \cs{pdfescape...} and \eTeX}
%
%    \begin{macrocode}
%<*test4>
\ProvidesFile{pdfescape-test4.tex}%
    [2011/11/25 v1.13 Test without \string\pdfescape... and e-TeX]%
%</test4>
%    \end{macrocode}
%
% \subsection{Test with \LuaTeX}
%
%    \begin{macrocode}
%<*test5>
\ProvidesFile{pdfescape-test5.tex}%
    [2011/11/25 v1.13 Test with LuaTeX]%
%</test5>
%    \end{macrocode}
%
% \subsection{Check/ensure test preconditions}
%
% \subsubsection{Check \cs{pdfescape...}}
%
%    \begin{macrocode}
%<*test2>
\@ifundefined{pdfescapehex}{%
  \PackageError{pdfescape-test2}{%
    Missing \string\pdfescape... commands%
  }{Test aborted.}%
  \stop
}{}
%</test2>
%    \end{macrocode}
%    \begin{macrocode}
%<*test3|test4>
\let\pdfescapehex\@undefined
\let\pdfunescapehex\@undefined
\let\pdfescapename\@undefined
\let\pdfescapestring\@undefined
%</test3|test4>
%    \end{macrocode}
%
% \subsubsection{Check \eTeX}
%
%    \begin{macrocode}
%<*test3>
\@ifundefined{numexpr}{%
  \PackageError{pdfescape-test3}{%
    Missing \eTeX
  }{Test aborted.}%
  \stop
}{}
%</test3>
%    \end{macrocode}
% Package \xpackage{qstest} uses \eTeX, thus \eTeX's features
% can only be disabled later during loading of package \xpackage{pdfescape}.
%
% \subsubsection{Check \LuaTeX}
%
%    \begin{macrocode}
%<*test5>
\@ifundefined{directlua}{%
  \PackageError{pdfescape-test5}{%
    Missing LuaTeX%
  }{Test aborted.}%
  \stop
}{}
%</test5>
%    \end{macrocode}
%
% \subsection{Common part}
%
%    The files for testing uses the framework, provided by the
%    new package \xpackage{qstest} of David Kastrup.
%    \begin{macrocode}
\RequirePackage{qstest}
\IncludeTests{*}
\LogTests{log}{*}{*}

\newcommand*{\ExpectVar}[2]{%
  \ifx#1#2%
  \else
    \begingroup
      \@onelevel@sanitize#1%
      \@onelevel@sanitize#2%
      \typeout{[#1] <> [#2]}% hash-ok
    \endgroup
  \fi
  \Expect*{\ifx#1#2true\else false\fi}{true}%
}

\makeatletter
\begingroup
  \gdef\AllBytes{}%
  \count@=0 %
  \catcode0=12 %
  \@whilenum\count@<256 \do{%
    \lccode0=\count@
    \ifnum\count@=32 %
      \xdef\AllBytes{\AllBytes\space}%
    \else
      \lowercase{%
        \xdef\AllBytes{\AllBytes^^@}%
      }%
    \fi
    \advance\count@ by 1 %
  }%
\endgroup
\newcommand*{\AllBytesHex}{%
  000102030405060708090A0B0C0D0E0F%
  101112131415161718191A1B1C1D1E1F%
  202122232425262728292A2B2C2D2E2F%
  303132333435363738393A3B3C3D3E3F%
  404142434445464748494A4B4C4D4E4F%
  505152535455565758595A5B5C5D5E5F%
  606162636465666768696A6B6C6D6E6F%
  707172737475767778797A7B7C7D7E7F%
  808182838485868788898A8B8C8D8E8F%
  909192939495969798999A9B9C9D9E9F%
  A0A1A2A3A4A5A6A7A8A9AAABACADAEAF%
  B0B1B2B3B4B5B6B7B8B9BABBBCBDBEBF%
  C0C1C2C3C4C5C6C7C8C9CACBCCCDCECF%
  D0D1D2D3D4D5D6D7D8D9DADBDCDDDEDF%
  E0E1E2E3E4E5E6E7E8E9EAEBECEDEEEF%
  F0F1F2F3F4F5F6F7F8F9FAFBFCFDFEFF%
}
\@onelevel@sanitize\AllBytesHex
\expandafter\lowercase\expandafter{%
  \expandafter\newcommand\expandafter*\expandafter\AllBytesHexLC
      \expandafter{\AllBytesHex}%
}
\newcommand*{\AllBytesName}{}
\begingroup
  \catcode`\#=12 %
  \xdef\AllBytesName{%
    #01#02#03#04#05#06#07#08#09#0A#0B#0C#0D#0E#0F%
    #10#11#12#13#14#15#16#17#18#19#1A#1B#1C#1D#1E#1F%
    #20!"#23$#25&'#28#29*+,-.#2F%
    0123456789:;#3C=#3E?%
    @ABCDEFGHIJKLMNO%
    PQRSTUVWXYZ#5B\@backslashchar#5D^_%
    `abcdefghijklmno%
    pqrstuvwxyz#7B|#7D\string~#7F%
    #80#81#82#83#84#85#86#87#88#89#8A#8B#8C#8D#8E#8F%
    #90#91#92#93#94#95#96#97#98#99#9A#9B#9C#9D#9E#9F%
    #A0#A1#A2#A3#A4#A5#A6#A7#A8#A9#AA#AB#AC#AD#AE#AF%
    #B0#B1#B2#B3#B4#B5#B6#B7#B8#B9#BA#BB#BC#BD#BE#BF%
    #C0#C1#C2#C3#C4#C5#C6#C7#C8#C9#CA#CB#CC#CD#CE#CF%
    #D0#D1#D2#D3#D4#D5#D6#D7#D8#D9#DA#DB#DC#DD#DE#DF%
    #E0#E1#E2#E3#E4#E5#E6#E7#E8#E9#EA#EB#EC#ED#EE#EF%
    #F0#F1#F2#F3#F4#F5#F6#F7#F8#F9#FA#FB#FC#FD#FE#FF%
  }%
\endgroup
\@onelevel@sanitize\AllBytesName

\newcommand*{\AllBytesString}{}
\begingroup
  \def\|{|}%
  \edef\%{\@percentchar}%
  \catcode`\|=0 %
  \catcode`\#=12 %
  \catcode`\~=12 %
  \catcode`\\=12 %
  |xdef|AllBytesString{%
    \000\001\002\003\004\005\006\007\010\011\012\013\014\015\016\017%
    \020\021\022\023\024\025\026\027\030\031\032\033\034\035\036\037%
    \040!"#$|%&'\(\)*+,-./%
    0123456789:;<=>?%
    @ABCDEFGHIJKLMNO%
    PQRSTUVWXYZ[\\]^_%
    `abcdefghijklmno%
    pqrstuvwxyz{||}~\177%
    \200\201\202\203\204\205\206\207\210\211\212\213\214\215\216\217%
    \220\221\222\223\224\225\226\227\230\231\232\233\234\235\236\237%
    \240\241\242\243\244\245\246\247\250\251\252\253\254\255\256\257%
    \260\261\262\263\264\265\266\267\270\271\272\273\274\275\276\277%
    \300\301\302\303\304\305\306\307\310\311\312\313\314\315\316\317%
    \320\321\322\323\324\325\326\327\330\331\332\333\334\335\336\337%
    \340\341\342\343\344\345\346\347\350\351\352\353\354\355\356\357%
    \360\361\362\363\364\365\366\367\370\371\372\373\374\375\376\377%
  }%
|endgroup
\@onelevel@sanitize\AllBytesString

%<*test4>
\let\org@detokenize\detokenize
\let\detokenize\@undefined
\let\org@numexpr\numexpr
\let\numexpr\@undefined
%</test4>
\RequirePackage{pdfescape}
%<*test4>
\let\detokenize\org@detokenize
\let\numexpr\org@numexpr
%</test4>

\begin{qstest}{all-hex}{\AllBytes, escapehex}
  \EdefEscapeHex\x{\AllBytes}%
  \Expect*{\x}*{\AllBytesHex}%
  \ExpectVar\x\AllBytesHex
\end{qstest}

\begin{qstest}{all-unhex}{\AllBytesHex, unescapehex}
  \EdefUnescapeHex\x{\AllBytesHex}%
  \Expect*{\x}*{\AllBytes}%
  \ExpectVar\x\AllBytes
\end{qstest}

\begin{qstest}{all-unhex-lc}{\AllBytesHexLC, unescapehex, lowercase}
  \EdefUnescapeHex\x{\AllBytesHexLC}%
  \Expect*{\x}*{\AllBytes}%
  \ExpectVar\x\AllBytes
\end{qstest}

\begin{qstest}{unhex-incomplete}{unescapehex, incomplete}
  \EdefUnescapeHex\x{4}%
  \Expect*{\x}{@}%
\end{qstest}

\begin{qstest}{unhex-space}{unescapehex, space}
  \EdefUnescapeHex\x{20}%
  \Expect*{\x}{ }%
  \ExpectVar\x\space
\end{qstest}

\begin{qstest}{unhex-spaces}{unescapehex, spaces}
  \EdefUnescapeHex\x{204020204120}%
  \def\y#1{%
    \edef\z{#1\string @#1#1\string A#1}%
  }\y{ }%
  \Expect*{\x}*{\z}%
  \ExpectVar\x\z
\end{qstest}

\begin{qstest}{unhex-hash}{unescapehex, hash}
  \catcode`\#=12 %
  \EdefUnescapeHex\x{#20}%
  \ExpectVar\x\space
\end{qstest}

\begin{qstest}{unhex-invalid}{unescapehex, invalid}
  \def\test#1#2{%
    \EdefUnescapeHex\x{#1}%
    \edef\y{#2}%
    \@onelevel@sanitize\y
    \ExpectVar\x\y
  }%
%<*test2>
  \edef\x{\pdfunescapehex{4X}}%
  \edef\y{\string @}%
  \ifx\x\y
  \else
    \def~{\space}%
    \typeout{************************************}%
    \typeout{* Your pdfTeX contains bug 777.~~~~*}%
    \typeout{* This test is redefined as dummy, *}%
    \typeout{* because it triggers the bug.~~~~~*}%
    \typeout{************************************}%
    \def\test#1#2{}%
  \fi
%</test2>
  \test{X}{}%
  \test{XY}{}%
  \test{XYZ}{}%
  \test{A}{^^a0}%
  \test{AX}{^^a0}%
  \test{XA}{^^a0}%
  \test{XXAXX}{^^a0}%
\end{qstest}

\begin{qstest}{all-name}{\AllBytes, escapename}
  \EdefEscapeName\x{\AllBytes}%
  \Expect*{\x}*{\AllBytesName}%
  \ExpectVar\x\AllBytesName
\end{qstest}

\begin{qstest}{all-string}{\AllBytes, escapestring}
  \EdefEscapeString\x{\AllBytes}%
  \Expect*{\x}*{\AllBytesString}%
  \ExpectVar\x\AllBytesString
\end{qstest}

\begin{qstest}{uchexdigit}{unescape, uppercase hex digit}
  \catcode`\@=11 %
  \catcode0=12 %
  \def\test#1#2{%
    \uccode0=#1\relax
    \uppercase{%
      \def\x{^^@}%
    }%
    \Expect*{%
       \ifcase\expandafter\PE@TestUcHexDigit\x
         true%
       \else
         false%
       \fi
    }{#2}%
  }%
  \def\range#1#2#3{%
    \count0=#1\relax
    \loop
    \ifnum\count0<#2\relax
      \test{\count0}{#3}%
      \advance\count0 by 1 %
    \repeat
  }%
  \range{0}{47}{false}%
  \range{48}{57}{true}%
  \range{58}{64}{false}%
  \range{65}{70}{true}%
  \range{71}{255}{false}%
\end{qstest}

\begin{qstest}{unescapename}{unescapename}
  \def\test#1#2{%
    \EdefUnescapeName\x{#1}%
    \edef\y{#2}%
    \@onelevel@sanitize\y
    \ExpectVar\x\y
  }%
  \catcode`\#=12 %
  \catcode0=12 %
  \test{}{}%
  \test{x}{x}%
  \test{xy}{xy}%
  \test{#}{#}%
  \test{##}{##}%
  \test{###}{###}%
  \test{####}{####}%
  \test{#x}{#x}%
  \test{#xy}{#xy}%
  \test{#1}{#1}%
  \test{#40}{@}%
  \test{#400}{@0}%
  \test{#4x0}{#4x0}%
  \test{#ab}{^^ab}%
  \test{#00}{^^@}%
  \test{x#40y#40z}{x@y@z}%
  \test{#40#40#40#40}{@@@@}%
  \test{a#x}{a#x}%
  \test{a#xy}{a#xy}%
  \test{a#1}{a#1}%
  \test{a#40}{a@}%
  \test{a#400}{a@0}%
  \test{#20}{ }%
  \test{a#20}{a }%
  \test{a#20b}{a b}%
  \test{a#20#20#20b}{a \space\space b}%
\end{qstest}

\begin{qstest}{unescapestring}{unescapestring}
  \def\test#1#2{%
    \EdefUnescapeString\x{#1}%
    \edef\y{#2}%
    \@onelevel@sanitize\y
    \ExpectVar\x\y
  }%
  \catcode0=12 %
  \def\DefChar#1#2{%
    \begingroup
      \uccode0=#2\relax
    \uppercase{\endgroup
      \def#1{^^@}%
    }%
  }%
  \DefChar\nul{0}%
  \DefChar\one{1}%
  \DefChar\bel{8}%
  \DefChar\tab{9}%
  \DefChar\lf{10}%
  \DefChar\ff{12}%
  \DefChar\cr{13}%
  \DefChar\\{92}%
  \test{}{}%
  \test{a}{a}%
  \test{\\}{}%
  \test{\\\\}{\\}%
  \test{\\\\y}{\\y}%
  \test{\\000}{\nul}%
  \test{\\b}{\bel}%
  \test{\\t}{\tab}%
  \test{\\n}{\lf}%
  \test{\\f}{\ff}%
  \test{\\r}{\cr}%
  \test{\\(}{(}%
  \test{\\)}{)}%
  \test{\\040}{ }%
  \test{\\100}{@}%
  \test{\\40}{ }%
  \test{\\1}{\one}%
  \test{\\01}{\one}%
  \test{\\001}{\one}%
  \test{\\18}{\one8}%
  \test{\\018}{\one8}%
  \test{\\0018}{\one8}%
  \test{x\\}{x}%
  \test{x\\\\}{x\\}%
  \test{x\\\\y}{x\\y}%
  \test{x\\000}{x\nul}%
  \test{x\\b}{x\bel}%
  \test{x\\t}{x\tab}%
  \test{x\\n}{x\lf}%
  \test{x\\f}{x\ff}%
  \test{x\\r}{x\cr}%
  \test{x\\(}{x(}%
  \test{x\\)}{x)}%
  \test{x\\040}{x }%
  \test{x\\100}{x@}%
  \test{x\\40}{x }%
  \test{x\\1}{x\one}%
  \test{x\\01}{x\one}%
  \test{x\\001}{x\one}%
  \test{x\\18}{x\one8}%
  \test{x\\018}{x\one8}%
  \test{x\\0018}{x\one8}%
  \test{\\b\\t\\n\\f\\r\\(\\)\\\\\\000\\040}{%
    \bel\tab\lf\ff\cr()\\\nul\space
  }%
  \test{\\\lf}{}%
  \test{x\\\lf}{x}%
  \test{\cr}{\lf}%
  \test{\cr\lf}{\lf}%
  \test{\lf}{\lf}%
  \test{\lf\cr}{\lf\lf}%
  \test{x\cr}{x\lf}%
  \test{x\cr\lf}{x\lf}%
  \test{x\lf}{x\lf}%
  \test{x\lf\cr}{x\lf\lf}%
  \test{x\\\cr\lf y\cr}{xy\lf}%
%
  \test{\\409}{ 9}%
  \test{\\800}{800}%
  \test{\\900}{900}%
  \test{\\578}{/8}%
  \test{\\477}{?}%
  \test{\\377}{^^ff}%
  \test{\\777}{^^ff}%
  \test{\\7777}{^^ff7}%
\end{qstest}
\stop
%    \end{macrocode}
%
%    \begin{macrocode}
%</test2|test3|test4|test5>
%    \end{macrocode}
%
% \subsubsection{Test for \hologo{iniTeX}}
%
%    \begin{macrocode}
%<*test6>
\input pdfescape.sty\relax
\catcode`\{=1 %
\catcode`\}=2 %
\catcode`\#=6 %
\catcode`\^=7 %
\catcode`\@=11 %
%    \end{macrocode}
%    \begin{macrocode}
\begingroup
  \catcode`\@=11 %
  \countdef\count@=255 %
  \def\space{ }%
  \long\def\@whilenum#1\do #2{%
    \ifnum #1\relax
      #2\relax
      \@iwhilenum{#1\relax#2\relax}%
    \fi
  }%
  \long\def\@iwhilenum#1{%
    \ifnum #1%
      \expandafter\@iwhilenum
    \else
      \expandafter\ltx@gobble
    \fi
    {#1}%
  }%
  \gdef\AllBytes{}%
  \count@=0 %
  \catcode0=12 %
  \@whilenum\count@<256 \do{%
    \lccode0=\count@
    \ifnum\count@=32 %
      \xdef\AllBytes{\AllBytes\space}%
    \else
      \lowercase{%
        \xdef\AllBytes{\AllBytes^^@}%
      }%
    \fi
    \advance\count@ by 1 %
  }%
\endgroup
\def\AllBytesHex{%
  000102030405060708090A0B0C0D0E0F%
  101112131415161718191A1B1C1D1E1F%
  202122232425262728292A2B2C2D2E2F%
  303132333435363738393A3B3C3D3E3F%
  404142434445464748494A4B4C4D4E4F%
  505152535455565758595A5B5C5D5E5F%
  606162636465666768696A6B6C6D6E6F%
  707172737475767778797A7B7C7D7E7F%
  808182838485868788898A8B8C8D8E8F%
  909192939495969798999A9B9C9D9E9F%
  A0A1A2A3A4A5A6A7A8A9AAABACADAEAF%
  B0B1B2B3B4B5B6B7B8B9BABBBCBDBEBF%
  C0C1C2C3C4C5C6C7C8C9CACBCCCDCECF%
  D0D1D2D3D4D5D6D7D8D9DADBDCDDDEDF%
  E0E1E2E3E4E5E6E7E8E9EAEBECEDEEEF%
  F0F1F2F3F4F5F6F7F8F9FAFBFCFDFEFF%
}
\ltx@onelevel@sanitize\AllBytesHex
\expandafter\lowercase\expandafter{%
  \expandafter\def\expandafter\AllBytesHexLC
      \expandafter{\AllBytesHex}%
}
\begingroup
  \catcode`\#=12 %
  \xdef\AllBytesName{%
    #01#02#03#04#05#06#07#08#09#0A#0B#0C#0D#0E#0F%
    #10#11#12#13#14#15#16#17#18#19#1A#1B#1C#1D#1E#1F%
    #20!"#23$#25&'#28#29*+,-.#2F%
    0123456789:;#3C=#3E?%
    @ABCDEFGHIJKLMNO%
    PQRSTUVWXYZ#5B\ltx@backslashchar#5D^_%
    `abcdefghijklmno%
    pqrstuvwxyz#7B|#7D\string~#7F%
    #80#81#82#83#84#85#86#87#88#89#8A#8B#8C#8D#8E#8F%
    #90#91#92#93#94#95#96#97#98#99#9A#9B#9C#9D#9E#9F%
    #A0#A1#A2#A3#A4#A5#A6#A7#A8#A9#AA#AB#AC#AD#AE#AF%
    #B0#B1#B2#B3#B4#B5#B6#B7#B8#B9#BA#BB#BC#BD#BE#BF%
    #C0#C1#C2#C3#C4#C5#C6#C7#C8#C9#CA#CB#CC#CD#CE#CF%
    #D0#D1#D2#D3#D4#D5#D6#D7#D8#D9#DA#DB#DC#DD#DE#DF%
    #E0#E1#E2#E3#E4#E5#E6#E7#E8#E9#EA#EB#EC#ED#EE#EF%
    #F0#F1#F2#F3#F4#F5#F6#F7#F8#F9#FA#FB#FC#FD#FE#FF%
  }%
\endgroup
\ltx@onelevel@sanitize\AllBytesName
\edef\AllBytesFromName{\expandafter\ltx@gobble\AllBytes}

\begingroup
  \def\|{|}%
  \edef\%{\ltx@percentchar}%
  \catcode`\|=0 %
  \catcode`\#=12 %
  \catcode`\~=12 %
  \catcode`\\=12 %
  |xdef|AllBytesString{%
    \000\001\002\003\004\005\006\007\010\011\012\013\014\015\016\017%
    \020\021\022\023\024\025\026\027\030\031\032\033\034\035\036\037%
    \040!"#$|%&'\(\)*+,-./%
    0123456789:;<=>?%
    @ABCDEFGHIJKLMNO%
    PQRSTUVWXYZ[\\]^_%
    `abcdefghijklmno%
    pqrstuvwxyz{||}~\177%
    \200\201\202\203\204\205\206\207\210\211\212\213\214\215\216\217%
    \220\221\222\223\224\225\226\227\230\231\232\233\234\235\236\237%
    \240\241\242\243\244\245\246\247\250\251\252\253\254\255\256\257%
    \260\261\262\263\264\265\266\267\270\271\272\273\274\275\276\277%
    \300\301\302\303\304\305\306\307\310\311\312\313\314\315\316\317%
    \320\321\322\323\324\325\326\327\330\331\332\333\334\335\336\337%
    \340\341\342\343\344\345\346\347\350\351\352\353\354\355\356\357%
    \360\361\362\363\364\365\366\367\370\371\372\373\374\375\376\377%
  }%
|endgroup
\ltx@onelevel@sanitize\AllBytesString
%    \end{macrocode}
%    \begin{macrocode}
\def\msg#{\immediate\write16}
\def\Test#1#2#3{%
  \begingroup
    #1\TestResult{#2}%
    \ifx\TestResult#3%
    \else
      \newlinechar=10 %
      \msg{Expect:^^J#3}%
      \msg{Result:^^J\TestResult}%
      \errmessage{\string#2 -\string#1-> \string#3}%
    \fi
  \endgroup
}
\Test\EdefEscapeHex\AllBytes\AllBytesHex
\Test\EdefUnescapeHex\AllBytesHex\AllBytes
\Test\EdefEscapeName\AllBytes\AllBytesName
\Test\EdefUnescapeName\AllBytesName\AllBytesFromName
\Test\EdefEscapeString\AllBytes\AllBytesString
\Test\EdefUnescapeString\AllBytesString\AllBytes
%    \end{macrocode}
%    \begin{macrocode}
\csname @@end\endcsname\end
%</test6>
%    \end{macrocode}
%
% \section{Installation}
%
% \subsection{Download}
%
% \paragraph{Package.} This package is available on
% CTAN\footnote{\url{ftp://ftp.ctan.org/tex-archive/}}:
% \begin{description}
% \item[\CTAN{macros/latex/contrib/oberdiek/pdfescape.dtx}] The source file.
% \item[\CTAN{macros/latex/contrib/oberdiek/pdfescape.pdf}] Documentation.
% \end{description}
%
%
% \paragraph{Bundle.} All the packages of the bundle `oberdiek'
% are also available in a TDS compliant ZIP archive. There
% the packages are already unpacked and the documentation files
% are generated. The files and directories obey the TDS standard.
% \begin{description}
% \item[\CTAN{install/macros/latex/contrib/oberdiek.tds.zip}]
% \end{description}
% \emph{TDS} refers to the standard ``A Directory Structure
% for \TeX\ Files'' (\CTAN{tds/tds.pdf}). Directories
% with \xfile{texmf} in their name are usually organized this way.
%
% \subsection{Bundle installation}
%
% \paragraph{Unpacking.} Unpack the \xfile{oberdiek.tds.zip} in the
% TDS tree (also known as \xfile{texmf} tree) of your choice.
% Example (linux):
% \begin{quote}
%   |unzip oberdiek.tds.zip -d ~/texmf|
% \end{quote}
%
% \paragraph{Script installation.}
% Check the directory \xfile{TDS:scripts/oberdiek/} for
% scripts that need further installation steps.
% Package \xpackage{attachfile2} comes with the Perl script
% \xfile{pdfatfi.pl} that should be installed in such a way
% that it can be called as \texttt{pdfatfi}.
% Example (linux):
% \begin{quote}
%   |chmod +x scripts/oberdiek/pdfatfi.pl|\\
%   |cp scripts/oberdiek/pdfatfi.pl /usr/local/bin/|
% \end{quote}
%
% \subsection{Package installation}
%
% \paragraph{Unpacking.} The \xfile{.dtx} file is a self-extracting
% \docstrip\ archive. The files are extracted by running the
% \xfile{.dtx} through \plainTeX:
% \begin{quote}
%   \verb|tex pdfescape.dtx|
% \end{quote}
%
% \paragraph{TDS.} Now the different files must be moved into
% the different directories in your installation TDS tree
% (also known as \xfile{texmf} tree):
% \begin{quote}
% \def\t{^^A
% \begin{tabular}{@{}>{\ttfamily}l@{ $\rightarrow$ }>{\ttfamily}l@{}}
%   pdfescape.sty & tex/generic/oberdiek/pdfescape.sty\\
%   pdfescape.pdf & doc/latex/oberdiek/pdfescape.pdf\\
%   test/pdfescape-test1.tex & doc/latex/oberdiek/test/pdfescape-test1.tex\\
%   test/pdfescape-test2.tex & doc/latex/oberdiek/test/pdfescape-test2.tex\\
%   test/pdfescape-test3.tex & doc/latex/oberdiek/test/pdfescape-test3.tex\\
%   test/pdfescape-test4.tex & doc/latex/oberdiek/test/pdfescape-test4.tex\\
%   test/pdfescape-test5.tex & doc/latex/oberdiek/test/pdfescape-test5.tex\\
%   test/pdfescape-test6.tex & doc/latex/oberdiek/test/pdfescape-test6.tex\\
%   pdfescape.dtx & source/latex/oberdiek/pdfescape.dtx\\
% \end{tabular}^^A
% }^^A
% \sbox0{\t}^^A
% \ifdim\wd0>\linewidth
%   \begingroup
%     \advance\linewidth by\leftmargin
%     \advance\linewidth by\rightmargin
%   \edef\x{\endgroup
%     \def\noexpand\lw{\the\linewidth}^^A
%   }\x
%   \def\lwbox{^^A
%     \leavevmode
%     \hbox to \linewidth{^^A
%       \kern-\leftmargin\relax
%       \hss
%       \usebox0
%       \hss
%       \kern-\rightmargin\relax
%     }^^A
%   }^^A
%   \ifdim\wd0>\lw
%     \sbox0{\small\t}^^A
%     \ifdim\wd0>\linewidth
%       \ifdim\wd0>\lw
%         \sbox0{\footnotesize\t}^^A
%         \ifdim\wd0>\linewidth
%           \ifdim\wd0>\lw
%             \sbox0{\scriptsize\t}^^A
%             \ifdim\wd0>\linewidth
%               \ifdim\wd0>\lw
%                 \sbox0{\tiny\t}^^A
%                 \ifdim\wd0>\linewidth
%                   \lwbox
%                 \else
%                   \usebox0
%                 \fi
%               \else
%                 \lwbox
%               \fi
%             \else
%               \usebox0
%             \fi
%           \else
%             \lwbox
%           \fi
%         \else
%           \usebox0
%         \fi
%       \else
%         \lwbox
%       \fi
%     \else
%       \usebox0
%     \fi
%   \else
%     \lwbox
%   \fi
% \else
%   \usebox0
% \fi
% \end{quote}
% If you have a \xfile{docstrip.cfg} that configures and enables \docstrip's
% TDS installing feature, then some files can already be in the right
% place, see the documentation of \docstrip.
%
% \subsection{Refresh file name databases}
%
% If your \TeX~distribution
% (\teTeX, \mikTeX, \dots) relies on file name databases, you must refresh
% these. For example, \teTeX\ users run \verb|texhash| or
% \verb|mktexlsr|.
%
% \subsection{Some details for the interested}
%
% \paragraph{Attached source.}
%
% The PDF documentation on CTAN also includes the
% \xfile{.dtx} source file. It can be extracted by
% AcrobatReader 6 or higher. Another option is \textsf{pdftk},
% e.g. unpack the file into the current directory:
% \begin{quote}
%   \verb|pdftk pdfescape.pdf unpack_files output .|
% \end{quote}
%
% \paragraph{Unpacking with \LaTeX.}
% The \xfile{.dtx} chooses its action depending on the format:
% \begin{description}
% \item[\plainTeX:] Run \docstrip\ and extract the files.
% \item[\LaTeX:] Generate the documentation.
% \end{description}
% If you insist on using \LaTeX\ for \docstrip\ (really,
% \docstrip\ does not need \LaTeX), then inform the autodetect routine
% about your intention:
% \begin{quote}
%   \verb|latex \let\install=y% \iffalse meta-comment
%
% File: pdfescape.dtx
% Version: 2011/11/25 v1.13
% Info: Implements pdfTeX's escape features
%
% Copyright (C) 2007, 2010, 2011 by
%    Heiko Oberdiek <heiko.oberdiek at googlemail.com>
%
% This work may be distributed and/or modified under the
% conditions of the LaTeX Project Public License, either
% version 1.3c of this license or (at your option) any later
% version. This version of this license is in
%    http://www.latex-project.org/lppl/lppl-1-3c.txt
% and the latest version of this license is in
%    http://www.latex-project.org/lppl.txt
% and version 1.3 or later is part of all distributions of
% LaTeX version 2005/12/01 or later.
%
% This work has the LPPL maintenance status "maintained".
%
% This Current Maintainer of this work is Heiko Oberdiek.
%
% The Base Interpreter refers to any `TeX-Format',
% because some files are installed in TDS:tex/generic//.
%
% This work consists of the main source file pdfescape.dtx
% and the derived files
%    pdfescape.sty, pdfescape.pdf, pdfescape.ins, pdfescape.drv,
%    pdfescape-test1.tex, pdfescape-test2.tex,
%    pdfescape-test3.tex, pdfescape-test4.tex,
%    pdfescape-test5.tex, pdfescape-test6.tex.
%
% Distribution:
%    CTAN:macros/latex/contrib/oberdiek/pdfescape.dtx
%    CTAN:macros/latex/contrib/oberdiek/pdfescape.pdf
%
% Unpacking:
%    (a) If pdfescape.ins is present:
%           tex pdfescape.ins
%    (b) Without pdfescape.ins:
%           tex pdfescape.dtx
%    (c) If you insist on using LaTeX
%           latex \let\install=y\input{pdfescape.dtx}
%        (quote the arguments according to the demands of your shell)
%
% Documentation:
%    (a) If pdfescape.drv is present:
%           latex pdfescape.drv
%    (b) Without pdfescape.drv:
%           latex pdfescape.dtx; ...
%    The class ltxdoc loads the configuration file ltxdoc.cfg
%    if available. Here you can specify further options, e.g.
%    use A4 as paper format:
%       \PassOptionsToClass{a4paper}{article}
%
%    Programm calls to get the documentation (example):
%       pdflatex pdfescape.dtx
%       makeindex -s gind.ist pdfescape.idx
%       pdflatex pdfescape.dtx
%       makeindex -s gind.ist pdfescape.idx
%       pdflatex pdfescape.dtx
%
% Installation:
%    TDS:tex/generic/oberdiek/pdfescape.sty
%    TDS:doc/latex/oberdiek/pdfescape.pdf
%    TDS:doc/latex/oberdiek/test/pdfescape-test1.tex
%    TDS:doc/latex/oberdiek/test/pdfescape-test2.tex
%    TDS:doc/latex/oberdiek/test/pdfescape-test3.tex
%    TDS:doc/latex/oberdiek/test/pdfescape-test4.tex
%    TDS:doc/latex/oberdiek/test/pdfescape-test5.tex
%    TDS:doc/latex/oberdiek/test/pdfescape-test6.tex
%    TDS:source/latex/oberdiek/pdfescape.dtx
%
%<*ignore>
\begingroup
  \catcode123=1 %
  \catcode125=2 %
  \def\x{LaTeX2e}%
\expandafter\endgroup
\ifcase 0\ifx\install y1\fi\expandafter
         \ifx\csname processbatchFile\endcsname\relax\else1\fi
         \ifx\fmtname\x\else 1\fi\relax
\else\csname fi\endcsname
%</ignore>
%<*install>
\input docstrip.tex
\Msg{************************************************************************}
\Msg{* Installation}
\Msg{* Package: pdfescape 2011/11/25 v1.13 Implements pdfTeX's escape features (HO)}
\Msg{************************************************************************}

\keepsilent
\askforoverwritefalse

\let\MetaPrefix\relax
\preamble

This is a generated file.

Project: pdfescape
Version: 2011/11/25 v1.13

Copyright (C) 2007, 2010, 2011 by
   Heiko Oberdiek <heiko.oberdiek at googlemail.com>

This work may be distributed and/or modified under the
conditions of the LaTeX Project Public License, either
version 1.3c of this license or (at your option) any later
version. This version of this license is in
   http://www.latex-project.org/lppl/lppl-1-3c.txt
and the latest version of this license is in
   http://www.latex-project.org/lppl.txt
and version 1.3 or later is part of all distributions of
LaTeX version 2005/12/01 or later.

This work has the LPPL maintenance status "maintained".

This Current Maintainer of this work is Heiko Oberdiek.

The Base Interpreter refers to any `TeX-Format',
because some files are installed in TDS:tex/generic//.

This work consists of the main source file pdfescape.dtx
and the derived files
   pdfescape.sty, pdfescape.pdf, pdfescape.ins, pdfescape.drv,
   pdfescape-test1.tex, pdfescape-test2.tex,
   pdfescape-test3.tex, pdfescape-test4.tex,
   pdfescape-test5.tex, pdfescape-test6.tex.

\endpreamble
\let\MetaPrefix\DoubleperCent

\generate{%
  \file{pdfescape.ins}{\from{pdfescape.dtx}{install}}%
  \file{pdfescape.drv}{\from{pdfescape.dtx}{driver}}%
  \usedir{tex/generic/oberdiek}%
  \file{pdfescape.sty}{\from{pdfescape.dtx}{package}}%
  \usedir{doc/latex/oberdiek/test}%
  \file{pdfescape-test1.tex}{\from{pdfescape.dtx}{test1}}%
  \file{pdfescape-test2.tex}{\from{pdfescape.dtx}{test2}}%
  \file{pdfescape-test3.tex}{\from{pdfescape.dtx}{test3}}%
  \file{pdfescape-test4.tex}{\from{pdfescape.dtx}{test4}}%
  \file{pdfescape-test5.tex}{\from{pdfescape.dtx}{test5}}%
  \file{pdfescape-test6.tex}{\from{pdfescape.dtx}{test6}}%
  \nopreamble
  \nopostamble
  \usedir{source/latex/oberdiek/catalogue}%
  \file{pdfescape.xml}{\from{pdfescape.dtx}{catalogue}}%
}

\catcode32=13\relax% active space
\let =\space%
\Msg{************************************************************************}
\Msg{*}
\Msg{* To finish the installation you have to move the following}
\Msg{* file into a directory searched by TeX:}
\Msg{*}
\Msg{*     pdfescape.sty}
\Msg{*}
\Msg{* To produce the documentation run the file `pdfescape.drv'}
\Msg{* through LaTeX.}
\Msg{*}
\Msg{* Happy TeXing!}
\Msg{*}
\Msg{************************************************************************}

\endbatchfile
%</install>
%<*ignore>
\fi
%</ignore>
%<*driver>
\NeedsTeXFormat{LaTeX2e}
\ProvidesFile{pdfescape.drv}%
  [2011/11/25 v1.13 Implements pdfTeX's escape features (HO)]%
\documentclass{ltxdoc}
\usepackage{holtxdoc}[2011/11/22]
\begin{document}
  \DocInput{pdfescape.dtx}%
\end{document}
%</driver>
% \fi
%
% \CheckSum{2516}
%
% \CharacterTable
%  {Upper-case    \A\B\C\D\E\F\G\H\I\J\K\L\M\N\O\P\Q\R\S\T\U\V\W\X\Y\Z
%   Lower-case    \a\b\c\d\e\f\g\h\i\j\k\l\m\n\o\p\q\r\s\t\u\v\w\x\y\z
%   Digits        \0\1\2\3\4\5\6\7\8\9
%   Exclamation   \!     Double quote  \"     Hash (number) \#
%   Dollar        \$     Percent       \%     Ampersand     \&
%   Acute accent  \'     Left paren    \(     Right paren   \)
%   Asterisk      \*     Plus          \+     Comma         \,
%   Minus         \-     Point         \.     Solidus       \/
%   Colon         \:     Semicolon     \;     Less than     \<
%   Equals        \=     Greater than  \>     Question mark \?
%   Commercial at \@     Left bracket  \[     Backslash     \\
%   Right bracket \]     Circumflex    \^     Underscore    \_
%   Grave accent  \`     Left brace    \{     Vertical bar  \|
%   Right brace   \}     Tilde         \~}
%
% \GetFileInfo{pdfescape.drv}
%
% \title{The \xpackage{pdfescape} package}
% \date{2011/11/25 v1.13}
% \author{Heiko Oberdiek\\\xemail{heiko.oberdiek at googlemail.com}}
%
% \maketitle
%
% \begin{abstract}
% This package implements \pdfTeX's escape features
% (\cs{pdfescapehex}, \cs{pdfunescapehex}, \cs{pdfescapename},
% \cs{pdfescapestring}) using \TeX\ or \eTeX.
% \end{abstract}
%
% \tableofcontents
%
% \section{Documentation}
%
% \begin{declcs}{EdefEscapeHex} \M{cmd} \M{string}\\
%  \cs{EdefUnescapeHex} \M{cmd} \M{string}\\
%  \cs{EdefEscapeName} \M{cmd} \M{string}\\
%  \cs{EdefEscapeString} \M{cmd} \M{string}
% \end{declcs}
% These commands converts \meta{string} and stores the result
% in macro \meta{cmd}. The conversion result is the same
% as the conversion of the corresponding \pdfTeX's primitives.
% Note that the argument \meta{string} is expanded before
% the conversion.
%
% For example, if \pdfTeX\ \textgreater= 1.30 is present,
% then \cs{EdefEscapeHex}
% becomes to:
% \begin{quote}
%\begin{verbatim}
%\def\EdefEscapeHex#1#2{%
%  \edef#1{\pdfescapehex{#2}}%
%}
%\end{verbatim}
% \end{quote}
% The package provides implementations for the case that \pdfTeX\ is
% not present (or too old). Even \eTeX\ can be missing, however it is
% used if it is detected.
%
% \paragraph{Babel.}
% The input strings may contain shorthand characters of package
% \xpackage{babel}.
%
% \subsection{Additional unescape macros}
%
% \begin{declcs}{EdefUnescapeName} \M{cmd} \M{string}
% \end{declcs}
% Sequences of a hash sign with two hexadecimal digits are
% converted to the corresponding character (PDF-1.2).
% A hash sign that is not followed by two hexadecimal digits
% is left unchanged. The catcodes in the result string follow
% TeX's conventions. The space has catcode 10 (space) and
% the other characters have catcode 12 (other).
%
% \begin{declcs}{EdefUnescapeString} \M{cmd} \M{string}
% \end{declcs}
% Macro \meta{cmd} stores the unescaped string in \meta{string}.
% All the rules for literal strings are implemented, see
% PDF specification.
% The catcodes in the result string follow TeX's conventions.
%
% \subsection{Sanitizing macro}
%
% \begin{declcs}{EdefSanitize} \M{cmd} \M{string}
% \end{declcs}
% Argument \meta{string} is expanded, converted to a string
% of tokens with catcode 12 (other) and space tokens, and
% stored in macro \meta{cmd}.
%
% \StopEventually{
% }
%
% \section{Implementation}
%
%    \begin{macrocode}
%<*package>
%    \end{macrocode}
%
% \subsection{Reload check and package identification}
%    Reload check, especially if the package is not used with \LaTeX.
%    \begin{macrocode}
\begingroup\catcode61\catcode48\catcode32=10\relax%
  \catcode13=5 % ^^M
  \endlinechar=13 %
  \catcode35=6 % #
  \catcode39=12 % '
  \catcode44=12 % ,
  \catcode45=12 % -
  \catcode46=12 % .
  \catcode58=12 % :
  \catcode64=11 % @
  \catcode123=1 % {
  \catcode125=2 % }
  \expandafter\let\expandafter\x\csname ver@pdfescape.sty\endcsname
  \ifx\x\relax % plain-TeX, first loading
  \else
    \def\empty{}%
    \ifx\x\empty % LaTeX, first loading,
      % variable is initialized, but \ProvidesPackage not yet seen
    \else
      \expandafter\ifx\csname PackageInfo\endcsname\relax
        \def\x#1#2{%
          \immediate\write-1{Package #1 Info: #2.}%
        }%
      \else
        \def\x#1#2{\PackageInfo{#1}{#2, stopped}}%
      \fi
      \x{pdfescape}{The package is already loaded}%
      \aftergroup\endinput
    \fi
  \fi
\endgroup%
%    \end{macrocode}
%    Package identification:
%    \begin{macrocode}
\begingroup\catcode61\catcode48\catcode32=10\relax%
  \catcode13=5 % ^^M
  \endlinechar=13 %
  \catcode35=6 % #
  \catcode39=12 % '
  \catcode40=12 % (
  \catcode41=12 % )
  \catcode44=12 % ,
  \catcode45=12 % -
  \catcode46=12 % .
  \catcode47=12 % /
  \catcode58=12 % :
  \catcode64=11 % @
  \catcode91=12 % [
  \catcode93=12 % ]
  \catcode123=1 % {
  \catcode125=2 % }
  \expandafter\ifx\csname ProvidesPackage\endcsname\relax
    \def\x#1#2#3[#4]{\endgroup
      \immediate\write-1{Package: #3 #4}%
      \xdef#1{#4}%
    }%
  \else
    \def\x#1#2[#3]{\endgroup
      #2[{#3}]%
      \ifx#1\@undefined
        \xdef#1{#3}%
      \fi
      \ifx#1\relax
        \xdef#1{#3}%
      \fi
    }%
  \fi
\expandafter\x\csname ver@pdfescape.sty\endcsname
\ProvidesPackage{pdfescape}%
  [2011/11/25 v1.13 Implements pdfTeX's escape features (HO)]%
%    \end{macrocode}
%
% \subsection{Catcodes}
%
%    \begin{macrocode}
\begingroup\catcode61\catcode48\catcode32=10\relax%
  \catcode13=5 % ^^M
  \endlinechar=13 %
  \catcode123 1 % {
  \catcode125 2 % }
  \catcode64 11 %
  \def\x{\endgroup
    \expandafter\edef\csname PE@AtEnd\endcsname{%
      \endlinechar=\the\endlinechar\relax
      \catcode13=\the\catcode13\relax
      \catcode32=\the\catcode32\relax
      \catcode35=\the\catcode35\relax
      \catcode61=\the\catcode61\relax
      \catcode64=\the\catcode64\relax
      \catcode123=\the\catcode123\relax
      \catcode125=\the\catcode125\relax
    }%
  }%
\x\catcode61\catcode48\catcode32=10\relax%
\catcode13=5 % ^^M
\endlinechar=13 %
\catcode35=6 % #
\catcode64=11 % @
\catcode123=1 % {
\catcode125=2 % }
\def\TMP@EnsureCode#1#2#3{%
  \edef\PE@AtEnd{%
    \PE@AtEnd
    #1#2=\the#1#2\relax
  }%
  #1#2=#3\relax
}
\TMP@EnsureCode\catcode{0}{12}% ^^@
\TMP@EnsureCode\catcode{34}{12}% "
\TMP@EnsureCode\catcode{36}{3}% $
\TMP@EnsureCode\catcode{38}{4}% &
\TMP@EnsureCode\catcode{39}{12}% '
\TMP@EnsureCode\catcode{42}{12}% *
\TMP@EnsureCode\catcode{45}{12}% -
\TMP@EnsureCode\catcode{46}{12}% .
\TMP@EnsureCode\catcode{47}{12}% /
\TMP@EnsureCode\catcode{60}{12}% <
\TMP@EnsureCode\catcode{62}{12}% >
\TMP@EnsureCode\catcode{91}{12}% [
\TMP@EnsureCode\catcode{93}{12}% ]
\TMP@EnsureCode\catcode{94}{7}% ^
\TMP@EnsureCode\catcode{96}{12}% `
\TMP@EnsureCode\uccode{34}{0}% "
\TMP@EnsureCode\uccode{48}{0}% 0
\TMP@EnsureCode\uccode{61}{0}% =
\edef\PE@AtEnd{\PE@AtEnd\noexpand\endinput}
%    \end{macrocode}
%
% \subsection{Load package}
%
%    \begin{macrocode}
\begingroup\expandafter\expandafter\expandafter\endgroup
\expandafter\ifx\csname RequirePackage\endcsname\relax
  \def\TMP@RequirePackage#1[#2]{%
    \begingroup\expandafter\expandafter\expandafter\endgroup
    \expandafter\ifx\csname ver@#1.sty\endcsname\relax
      \input #1.sty\relax
    \fi
  }%
  \TMP@RequirePackage{ltxcmds}[2010/04/08]%
\else
  \RequirePackage{ltxcmds}[2010/04/08]%
\fi
%    \end{macrocode}
%
% \subsection{Sanitizing}
%
%    \begin{macro}{\EdefSanitize}
%    Macro \cs{EdefSanitize} takes |#2|, entirely converts
%    it to token with catcode 12 (other) and stores the
%    result in macro |#1|.
%    \begin{macrocode}
\begingroup\expandafter\expandafter\expandafter\endgroup
\expandafter\ifx\csname detokenize\endcsname\relax
  \long\def\EdefSanitize#1#2{%
    \begingroup
      \csname @safe@activestrue\endcsname
      \edef#1{#2}%
      \PE@onelevel@sanitize#1%
    \expandafter\endgroup
    \expandafter\def\expandafter#1\expandafter{#1}%
  }%
  \begingroup\expandafter\expandafter\expandafter\endgroup
  \expandafter\ifx\csname @onelevel@sanitize\endcsname\relax
    \def\PE@onelevel@sanitize#1{%
      \edef#1{\expandafter\PE@strip@prefix\meaning#1}%
    }%
    \def\PE@strip@prefix#1>{}%
  \else
    \let\PE@onelevel@sanitize\@onelevel@sanitize
  \fi
\else
  \long\def\EdefSanitize#1#2{%
    \begingroup
      \csname @safe@activestrue\endcsname
      \edef#1{#2}%
    \expandafter\endgroup
    \expandafter\def\expandafter#1\expandafter{%
      \detokenize\expandafter{#1}%
    }%
  }%
  \def\PE@onelevel@sanitize#1{%
    \edef#1{\detokenize\expandafter{#1}}%
  }%
\fi
%    \end{macrocode}
%    \end{macro}
%    \begin{macro}{\PE@sanitize}
%    Macro \cs{PE@sanitize} is only defined for
%    compatibility with version 1.4. Its use is deprecated.
%    \begin{macrocode}
\let\PE@sanitize\EdefSanitize
%    \end{macrocode}
%    \end{macro}
%
% \subsubsection{Space characters}
%
%    \begin{macro}{\PE@space@other}
%    \begin{macrocode}
\begingroup
  \catcode`\ =12\relax%
\def\x{\endgroup\def\PE@space@other{ }}\x\relax
%    \end{macrocode}
%    \end{macro}
%    \begin{macro}{\PE@space@space}
%    \begin{macrocode}
\def\PE@space@space{ }
%    \end{macrocode}
%    \end{macro}
%
% \subsubsection{Space normalization}
%
%    \begin{macro}{\PE@SanitizeSpaceOther}
%    \begin{macrocode}
\def\PE@SanitizeSpaceOther#1{%
  \edef#1{\expandafter\PE@SpaceToOther#1 \relax}%
}
%    \end{macrocode}
%    \end{macro}
%
%    \begin{macro}{\PE@SpaceToOther}
%    \begin{macrocode}
\def\PE@SpaceToOther#1 #2\relax{%
  #1%
  \ifx\\#2\\%
  \else
    \PE@space@other
    \ltx@ReturnAfterFi{%
      \PE@SpaceToOther#2\relax
    }%
  \fi
}
%    \end{macrocode}
%    \end{macro}
%
% \subsection{\cs{EdefUnescapeName}}
%
%    \begin{macro}{\EdefUnescapeName}
%    \begin{macrocode}
\def\EdefUnescapeName#1#2{%
  \EdefSanitize#1{#2}%
  \PE@SanitizeSpaceOther#1%
  \PE@UnescapeName#1%
  \PE@onelevel@sanitize#1%
}
%    \end{macrocode}
%    \end{macro}
%
%    \begin{macro}{\PE@UnescapeName}
%    \begin{macrocode}
\begingroup
  \catcode`\$=6  % hash
  \catcode`\#=12 % other
  \gdef\PE@UnescapeName$1{%
    \begingroup
      \PE@InitUccodeHexDigit
      \def\PE@result{}%
      \expandafter\PE@DeName$1#\relax\relax
    \expandafter\endgroup
    \expandafter\def\expandafter$1\expandafter{\PE@result}%
  }%
  \gdef\PE@DeName$1#$2$3{%
    \ifx\relax$2%
      \edef\PE@result{\PE@result$1}%
      \let\PE@next\relax
    \else
      \ifx\relax$3%
        % wrong escape sequence in input
        \edef\PE@result{\PE@result$1#}%
        \let\PE@next\relax
      \else
        \uppercase{%
          \def\PE@testA{$2}%
          \def\PE@testB{$3}%
        }%
        \ifcase\ifcase\expandafter\PE@TestUcHexDigit\PE@testA
                 \ifcase\expandafter\PE@TestUcHexDigit\PE@testB
                   \ltx@zero
                 \else
                   \ltx@one
                 \fi
               \else
                 \ltx@one
               \fi
          \uccode\ltx@zero="\PE@testA\PE@testB\relax
          \uppercase{%
            \def\PE@temp{^^@}%
          }%
          \uccode\ltx@zero=\ltx@zero
          \edef\PE@result{\PE@result$1\PE@temp}%
          \let\PE@next\PE@DeName
        \else
          % wrong escape sequence in input
          \edef\PE@result{\PE@result$1#}%
          \def\PE@next{\PE@DeName$2$3}%
        \fi
      \fi
    \fi
    \PE@next
  }%
\endgroup
%    \end{macrocode}
%    \end{macro}
%
%    \begin{macro}{\PE@InitUccodeHexDigit}
%    \begin{macrocode}
\def\PE@InitUccodeHexDigit{%
  \uccode`a=`A\relax
  \uccode`b=`B\relax
  \uccode`c=`C\relax
  \uccode`d=`D\relax
  \uccode`e=`E\relax
  \uccode`f=`F\relax
  \uccode`A=\ltx@zero
  \uccode`B=\ltx@zero
  \uccode`C=\ltx@zero
  \uccode`D=\ltx@zero
  \uccode`E=\ltx@zero
  \uccode`F=\ltx@zero
  \uccode`0=\ltx@zero
  \uccode`1=\ltx@zero
  \uccode`2=\ltx@zero
  \uccode`3=\ltx@zero
  \uccode`4=\ltx@zero
  \uccode`5=\ltx@zero
  \uccode`6=\ltx@zero
  \uccode`7=\ltx@zero
  \uccode`8=\ltx@zero
  \uccode`9=\ltx@zero
}
%    \end{macrocode}
%    \end{macro}
%
%    \begin{macro}{\PE@TestUcHexDigit}
%    \begin{macrocode}
\def\PE@TestUcHexDigit#1{%
  \ifnum`#1<48 % 0
    \ltx@one
  \else
    \ifnum`#1>70 % F
      \ltx@one
    \else
      \ifnum`#1>57 % 9
        \ifnum`#1<65 % A
          \ltx@one
        \else
          \ltx@zero
        \fi
      \else
        \ltx@zero
      \fi
    \fi
  \fi
}
%    \end{macrocode}
%    \end{macro}
%
% \subsection{\cs{EdefUnescapeString}}
%
%    \begin{macro}{\EdefUnescapeString}
%    \begin{macrocode}
\def\EdefUnescapeString#1#2{%
  \EdefSanitize#1{#2}%
  \PE@SanitizeSpaceOther#1%
  \PE@NormalizeLineEnd#1%
  \PE@UnescapeString#1%
  \PE@onelevel@sanitize#1%
}
%    \end{macrocode}
%    \end{macro}
%
%    \begin{macrocode}
\begingroup
  \uccode`\8=10 % lf
  \uccode`\9=13 % cr
\def\x#1#2{\endgroup
%    \end{macrocode}
%    \begin{macro}{\PE@NormalizeLineEnd}
%    \begin{macrocode}
  \def\PE@NormalizeLineEnd##1{%
    \def\PE@result{}%
    \expandafter\PE@@NormalizeLineEnd##1#2\relax
    \let##1\PE@result
  }%
%    \end{macrocode}
%    \end{macro}
%    \begin{macro}{\PE@@NormalizeLineEnd}
%    \begin{macrocode}
  \def\PE@@NormalizeLineEnd##1#2##2{%
    \ifx\relax##2%
      \edef\PE@result{\PE@result##1}%
      \let\PE@next\relax
    \else
      \edef\PE@result{\PE@result##1#1}%
      \ifx#1##2% lf
        \let\PE@next\PE@@NormalizeLineEnd
      \else
        \def\PE@next{\PE@@NormalizeLineEnd##2}%
      \fi
    \fi
    \PE@next
  }%
}%
\uppercase{%
  \x 89%
}
%    \end{macrocode}
%    \end{macro}
%
%    \begin{macrocode}
\begingroup
  \catcode`\|=0 %
  \catcode`\\=12 %
%    \end{macrocode}
%    \begin{macro}{\PE@UnescapeString}
%    \begin{macrocode}
  |gdef|PE@UnescapeString#1{%
    |begingroup
      |def|PE@result{}%
      |expandafter|PE@DeString#1\|relax
    |expandafter|endgroup
    |expandafter|def|expandafter#1|expandafter{|PE@result}%
  }%
%    \end{macrocode}
%    \end{macro}
%    \begin{macro}{\PE@DeString}
%    \begin{macrocode}
  |gdef|PE@DeString#1\#2{%
    |ifx|relax#2%
      |edef|PE@result{|PE@result#1}%
      |let|PE@next|relax
    |else
      |if n#2%
        |uccode|ltx@zero=10 %
      |else|if r#2%
        |uccode|ltx@zero=13 %
      |else|if t#2%
        |uccode|ltx@zero=9 %
      |else|if b#2%
        |uccode|ltx@zero=8 %
      |else|if f#2%
        |uccode|ltx@zero=12 %
      |else
        |uccode|ltx@zero=|ltx@zero
      |fi|fi|fi|fi|fi
      |ifnum|uccode|ltx@zero>|ltx@zero
        |uppercase{%
          |edef|PE@temp{^^@}%
        }%
        |edef|PE@result{|PE@result#1|PE@temp}%
        |let|PE@next|PE@DeString
      |else
        |if\#2% backslash
          |edef|PE@result{|PE@result#1}%
          |let|PE@next|PE@CheckEndBackslash
        |else
          |ifnum`#2=10 % linefeed
            |edef|PE@result{|PE@result#1}%
            |let|PE@next|PE@DeString
          |else
            |ifcase|PE@TestOctDigit#2%
              |edef|PE@result{|PE@result#1}%
              |def|PE@next{|PE@OctI#2}%
            |else
              |edef|PE@result{|PE@result#1#2}%
              |let|PE@next|PE@DeString
            |fi
          |fi
        |fi
      |fi
    |fi
    |PE@next
  }%
%    \end{macrocode}
%    \end{macro}
%    \begin{macro}{\PE@CheckEndBackslash}
%    \begin{macrocode}
  |gdef|PE@CheckEndBackslash#1{%
    |ifx|relax#1%
    |else
      |edef|PE@result{|PE@result\}%
      |expandafter|PE@DeString|expandafter#1%
    |fi
  }%
%    \end{macrocode}
%    \end{macro}
%    \begin{macrocode}
|endgroup
%    \end{macrocode}
%    \begin{macro}{\PE@TestOctDigit}
%    \begin{macrocode}
\def\PE@TestOctDigit#1{%
  \ifnum`#1<48 % 0
    \ltx@one
  \else
    \ifnum`#1>55 % 7
      \ltx@one
    \else
      \ltx@zero
    \fi
  \fi
}
%    \end{macrocode}
%    \end{macro}
%    \begin{macro}{\PE@OctI}
%    \begin{macrocode}
\def\PE@OctI#1#2{%
  \ifcase\PE@TestOctDigit#2%
    \def\PE@next{\PE@OctII{#1#2}}%
  \else
    \def\PE@next{\PE@OctAll#1#2}%
  \fi
  \PE@next
}
%    \end{macrocode}
%    \end{macro}
%    \begin{macro}{\PE@OctII}
%    \begin{macrocode}
\def\PE@OctII#1#2{%
  \ifcase\PE@TestOctDigit#2%
    \def\PE@next{\PE@OctIII#1#2}%
  \else
    \def\PE@next{\PE@OctAll{#1}#2}%
  \fi
  \PE@next
}
%    \end{macrocode}
%    \end{macro}
%    \begin{macrocode}
\ltx@IfUndefined{numexpr}{%
  \catcode`\$=9 %
  \catcode`\&=14 %
}{%
  \catcode`\$=14 %
  \catcode`\&=9 %
}
%    \end{macrocode}
%    \begin{macro}{\PE@OctIII}
%    \begin{macrocode}
\def\PE@OctIII#1#2#3{%
  \ifnum#1<4 %
    \def\PE@next{\PE@OctAll{#1#2#3}}%
  \else
$   \count\ltx@cclv#1 %
$   \advance\count\ltx@cclv -4 %
    \edef\PE@next{%
      \noexpand\PE@OctAll{%
$       \the\count\ltx@cclv
&       \the\numexpr#1-4\relax
        #2#3%
      }%
    }%
  \fi
  \PE@next
}
%    \end{macrocode}
%    \end{macro}
%    \begin{macro}{\PE@OctAll}
%    \begin{macrocode}
\def\PE@OctAll#1{%
  \uccode\ltx@zero='#1\relax
  \uppercase{%
    \edef\PE@result{\PE@result^^@}%
  }%
  \PE@DeString
}
%    \end{macrocode}
%    \end{macro}
%
% \subsection{User macros (\pdfTeX\space analogues)}
%
%    \begin{macrocode}
\begingroup\expandafter\expandafter\expandafter\endgroup
\expandafter\ifx\csname RequirePackage\endcsname\relax
  \def\TMP@RequirePackage#1[#2]{%
    \begingroup\expandafter\expandafter\expandafter\endgroup
    \expandafter\ifx\csname ver@#1.sty\endcsname\relax
      \input #1.sty\relax
    \fi
  }%
  \TMP@RequirePackage{pdftexcmds}[2007/11/11]%
\else
  \RequirePackage{pdftexcmds}[2007/11/11]%
\fi
%    \end{macrocode}
%
%    \begin{macrocode}
\begingroup\expandafter\expandafter\expandafter\endgroup
\expandafter\ifx\csname pdf@escapehex\endcsname\relax
%    \end{macrocode}
%
%    \begin{macro}{\EdefEscapeHex}
%    \begin{macrocode}
  \long\def\EdefEscapeHex#1#2{%
    \EdefSanitize#1{#2}%
    \PE@SanitizeSpaceOther#1%
    \PE@EscapeHex#1%
  }%
%    \end{macrocode}
%    \end{macro}
%    \begin{macro}{\EdefUnescapeHex}
%    \begin{macrocode}
  \def\EdefUnescapeHex#1#2{%
    \EdefSanitize#1{#2}%
    \PE@UnescapeHex#1%
  }%
%    \end{macrocode}
%    \end{macro}
%    \begin{macro}{\EdefEscapeName}
%    \begin{macrocode}
  \long\def\EdefEscapeName#1#2{%
    \EdefSanitize#1{#2}%
    \PE@SanitizeSpaceOther#1%
    \PE@EscapeName#1%
  }%
%    \end{macrocode}
%    \end{macro}
%    \begin{macro}{\EdefEscapeString}
%    \begin{macrocode}
  \long\def\EdefEscapeString#1#2{%
    \EdefSanitize#1{#2}%
    \PE@SanitizeSpaceOther#1%
    \PE@EscapeString#1%
  }%
%    \end{macrocode}
%    \end{macro}
%
%    \begin{macrocode}
\else
%    \end{macrocode}
%
%    \begin{macro}{\PE@edefbabel}
%    Help macro that adds support for \xpackage{babel}'s shorthand
%    characters.
%    \begin{macrocode}
  \long\def\PE@edefbabel#1#2#3{%
    \begingroup
      \csname @save@activestrue\endcsname
      \edef#1{#2{#3}}%
    \expandafter\endgroup
    \expandafter\def\expandafter#1\expandafter{#1}%
  }%
%    \end{macrocode}
%    \end{macro}
%
%    \begin{macro}{\EdefEscapeHex}
%    \begin{macrocode}
  \long\def\EdefEscapeHex#1#2{%
    \PE@edefbabel#1\pdf@escapehex{#2}%
  }%
%    \end{macrocode}
%    \end{macro}
%    \begin{macro}{\EdefUnescapeHex}
%    \begin{macrocode}
  \def\EdefUnescapeHex#1#2{%
    \PE@edefbabel#1\pdf@unescapehex{#2}%
  }%
%    \end{macrocode}
%    \end{macro}
%    \begin{macro}{\EdefEscapeName}
%    \begin{macrocode}
  \long\def\EdefEscapeName#1#2{%
    \PE@edefbabel#1\pdf@escapename{#2}%
  }%
%    \end{macrocode}
%    \end{macro}
%    \begin{macro}{\EdefEscapeString}
%    \begin{macrocode}
  \long\def\EdefEscapeString#1#2{%
    \PE@edefbabel#1\pdf@escapestring{#2}%
  }%
%    \end{macrocode}
%    \end{macro}
%
%    \begin{macrocode}
  \expandafter\PE@AtEnd
\fi%
%    \end{macrocode}
%
% \subsection{Help macros}
%
% \subsubsection{Characters}
%
%    Special characters with catcode 12 (other) are created and
%    stored in macros.
%    \begin{macro}{\PE@hash}
%    \begin{macrocode}
\edef\PE@hash{\string#}
%    \end{macrocode}
%    \end{macro}
%    \begin{macro}{\PE@backslash}
%    \begin{macrocode}
\begingroup
  \escapechar=-1 %
\edef\x{\endgroup
  \def\noexpand\PE@backslash{\string\\}%
}
\x
%    \end{macrocode}
%    \end{macro}
%
% \subsubsection{Switch for \eTeX}
%
%    \begin{macrocode}
\ltx@newif\ifPE@etex
\begingroup\expandafter\expandafter\expandafter\endgroup
\expandafter\ifx\csname numexpr\endcsname\relax
\else
  \PE@etextrue
\fi
%    \end{macrocode}
%
% \subsection{Conversions}
%
% \subsubsection{Conversion to hex string}
%
%    \begin{macro}{\PE@EscapeHex}
%    \begin{macrocode}
\ifPE@etex
  \def\PE@EscapeHex#1{%
    \edef#1{\expandafter\PE@ToHex#1\relax}%
  }%
\else
  \def\PE@EscapeHex#1{%
    \def\PE@result{}%
    \expandafter\PE@ToHex#1\relax
    \let#1\PE@result
  }%
\fi
%    \end{macrocode}
%    \end{macro}
%    \begin{macro}{\PE@ToHex}
%    \begin{macrocode}
\def\PE@ToHex#1{%
  \ifx\relax#1%
  \else
    \PE@HexChar{#1}%
    \expandafter\PE@ToHex
  \fi
}%
%    \end{macrocode}
%    \end{macro}
%    \begin{macro}{\PE@HexChar}
%    \begin{macrocode}
\ifPE@etex
  \def\PE@HexChar#1{%
    \PE@HexDigit{\numexpr\dimexpr.0625\dimexpr`#1sp\relax\relax\relax}%
    \PE@HexDigit{%
      \numexpr`#1-16*\dimexpr.0625\dimexpr`#1sp\relax\relax\relax
    }%
  }%
\else
  \def\PE@HexChar#1{%
    \dimen0=`#1sp%
    \dimen2=.0625\dimen0 %
    \advance\dimen0-16\dimen2 %
    \edef\PE@result{%
      \PE@result
      \PE@HexDigit{\dimen2 }%
      \PE@HexDigit{\dimen0 }%
    }%
  }%
\fi
%    \end{macrocode}
%    \end{macro}
%    \begin{macro}{\PE@HexDigit}
%    \begin{macrocode}
\def\PE@HexDigit#1{%
  \expandafter\string
  \ifcase#1%
    0\or 1\or 2\or 3\or 4\or 5\or 6\or 7\or 8\or 9\or
    A\or B\or C\or D\or E\or F%
  \fi
}
%    \end{macrocode}
%    \end{macro}
%
% \subsubsection{Character code to octal number}
%
%    \begin{macro}{\PE@OctChar}
%    \begin{macrocode}
\ifPE@etex
  \def\PE@OctChar#1{%
    \expandafter\PE@@OctChar
        \the\numexpr\dimexpr.015625\dimexpr`#1sp\relax\relax
            \expandafter\relax
        \expandafter\relax
        \the\numexpr\dimexpr.125\dimexpr`#1sp\relax\relax\relax
        \relax
        #1%
  }%
  \def\PE@@OctChar#1\relax#2\relax#3{%
    \PE@backslash
    #1%
    \the\numexpr#2-8*#1\relax
    \the\numexpr\dimexpr`#3sp\relax-8*#2\relax
  }%
\else
  \def\PE@OctChar#1{%
    \dimen0=`#1sp%
    \dimen2=.125\dimen0 %
    \dimen4=.125\dimen2 %
    \advance\dimen0-8\dimen2 %
    \advance\dimen2-8\dimen4 %
    \edef\PE@result{%
      \PE@result
      \PE@backslash
      \number\dimen4 %
      \number\dimen2 %
      \number\dimen0 %
    }%
  }%
\fi
%    \end{macrocode}
%    \end{macro}
%
% \subsubsection{Unpack hex string}
%
%    \begin{macro}{\PE@UnescapeHex}
%    \begin{macrocode}
\def\PE@UnescapeHex#1{%
  \begingroup
    \PE@InitUccodeHexDigit
    \def\PE@result{}%
    \expandafter\PE@DeHex#1\relax\relax
  \expandafter\endgroup
  \expandafter\def\expandafter#1\expandafter{\PE@result}%
}
%    \end{macrocode}
%    \end{macro}
%    \begin{macro}{\PE@DeHex}
%    \begin{macrocode}
\def\PE@DeHex#1#2{%
  \ifx#2\relax
    \ifx#1\relax
      \let\PE@next\relax
    \else
      \uppercase{%
        \def\PE@testA{#1}%
      }%
      \ifcase\expandafter\PE@TestUcHexDigit\PE@testA
        \def\PE@next{%
          \PE@DeHex#10\relax\relax
        }%
      \else
        \let\PE@next\relax
      \fi
    \fi
  \else
    \uppercase{%
      \def\PE@testA{#1}%
      \def\PE@testB{#2}%
    }%
    \ifcase\expandafter\PE@TestUcHexDigit\PE@testA
      \ifcase\expandafter\PE@TestUcHexDigit\PE@testB
        \uccode\ltx@zero="\PE@testA\PE@testB\relax
        \ifnum\uccode\ltx@zero=32 %
          \let\PE@temp\PE@space@space
        \else
          \uppercase{%
            \def\PE@temp{^^@}%
          }%
        \fi
        \edef\PE@result{\PE@result\PE@temp}%
        \let\PE@next\PE@DeHex
      \else
        % invalid input sequence
        \def\PE@next{%
          \PE@DeHex#1%
        }%
      \fi
    \else
      % invalid input sequence
      \def\PE@next{\PE@DeHex#2}%
    \fi
  \fi
  \PE@next
}
%    \end{macrocode}
%    \end{macro}
%
% \subsubsection{Conversion to PDF name}
%
%    \begin{macro}{\PE@EscapeName}
%    \begin{macrocode}
\ifPE@etex
  \def\PE@EscapeName#1{%
    \edef#1{\expandafter\PE@EscapeNameTokens#1\relax}%
  }%
\else
  \def\PE@EscapeName#1{%
    \def\PE@result{}%
    \expandafter\PE@EscapeNameTokens#1\relax
    \let#1\PE@result
  }%
\fi
%    \end{macrocode}
%    \end{macro}
%    \begin{macro}{\PE@EscapeNameTokens}
%    \begin{macrocode}
\def\PE@EscapeNameTokens#1{%
  \ifx\relax#1%
  \else
    \ifnum`#1<33 %
      \ifcase`#1 %
        % drop illegal zero
      \else
        \PE@EscapeNameAdd\PE@hash
        \PE@HexChar#1%
      \fi
    \else
      \ifnum`#1>126 %
        \PE@EscapeNameAdd\PE@hash
        \PE@HexChar#1%
      \else  \ifnum`#1=35  \PE@EscapeNameHashChar 23% #
        \else\ifnum`#1=37  \PE@EscapeNameHashChar 25% %
        \else\ifnum`#1=40  \PE@EscapeNameHashChar 28% (
        \else\ifnum`#1=41  \PE@EscapeNameHashChar 29% )
        \else\ifnum`#1=47  \PE@EscapeNameHashChar 2F% /
        \else\ifnum`#1=60  \PE@EscapeNameHashChar 3C% <
        \else\ifnum`#1=62  \PE@EscapeNameHashChar 3E% >
        \else\ifnum`#1=91  \PE@EscapeNameHashChar 5B% [
        \else\ifnum`#1=93  \PE@EscapeNameHashChar 5D% ]
        \else\ifnum`#1=123 \PE@EscapeNameHashChar 7B% {
        \else\ifnum`#1=125 \PE@EscapeNameHashChar 7D% }
        \else
          \PE@EscapeNameAdd{#1}%
        \fi\fi\fi\fi\fi\fi\fi\fi\fi\fi\fi
      \fi
    \fi
    \expandafter\PE@EscapeNameTokens
  \fi
}%
\def\PE@EscapeNameHashChar#1#2{%
  \PE@EscapeNameAdd{\PE@hash\string#1\string#2}%
}%
%    \end{macrocode}
%    \end{macro}
%    \begin{macro}{\PE@EscapeNameAdd}
%    \begin{macrocode}
\ifPE@etex
  \def\PE@EscapeNameAdd#1{#1}%
\else
  \def\PE@EscapeNameAdd#1{%
    \edef\PE@result{%
      \PE@result
      #1%
    }%
  }%
\fi
%    \end{macrocode}
%    \end{macro}
%
% \subsubsection{Conversion to PDF string}
%
%    \begin{macro}{\PE@EscapeString}
%    \begin{macrocode}
\ifPE@etex
  \def\PE@EscapeString#1{%
    \edef#1{\expandafter\PE@EscapeStringTokens#1\relax}%
  }%
\else
  \def\PE@EscapeString#1{%
    \begingroup
      \def\PE@result{}%
      \expandafter\PE@EscapeStringTokens#1\relax
    \expandafter\endgroup
    \expandafter\def\expandafter#1\expandafter{\PE@result}%
  }%
\fi
%    \end{macrocode}
%    \end{macro}
%    \begin{macro}{\PE@EscapeStringTokens}
%    \begin{macrocode}
\def\PE@EscapeStringTokens#1{%
  \ifx\relax#1%
  \else
    \ifnum`#1<33 %
      \PE@OctChar#1%
    \else
      \ifnum`#1>126 %
        \PE@OctChar#1%
      \else  \ifnum`#1=40 \PE@EscapeStringAdd{\string\(}% (
        \else\ifnum`#1=41 \PE@EscapeStringAdd{\string\)}% )
        \else\ifnum`#1=92 \PE@EscapeStringAdd{\string\\}% \
        \else
          \PE@EscapeStringAdd{#1}%
        \fi\fi\fi
      \fi
    \fi
    \expandafter\PE@EscapeStringTokens
  \fi
}%
%    \end{macrocode}
%    \end{macro}
%    \begin{macro}{\PE@EscapeStringAdd}
%    \begin{macrocode}
\ifPE@etex
  \def\PE@EscapeStringAdd#1{#1}%
\else
  \def\PE@EscapeStringAdd#1{%
    \edef\PE@result{%
      \PE@result
      #1%
    }%
  }%
\fi
%    \end{macrocode}
%    \end{macro}
%
%    \begin{macrocode}
\PE@AtEnd%
%    \end{macrocode}
%
%    \begin{macrocode}
%</package>
%    \end{macrocode}
%
% \section{Test}
%
% \subsection{Catcode checks for loading}
%
%    \begin{macrocode}
%<*test1>
%    \end{macrocode}
%    \begin{macrocode}
\catcode`\{=1 %
\catcode`\}=2 %
\catcode`\#=6 %
\catcode`\@=11 %
\expandafter\ifx\csname count@\endcsname\relax
  \countdef\count@=255 %
\fi
\expandafter\ifx\csname @gobble\endcsname\relax
  \long\def\@gobble#1{}%
\fi
\expandafter\ifx\csname @firstofone\endcsname\relax
  \long\def\@firstofone#1{#1}%
\fi
\expandafter\ifx\csname loop\endcsname\relax
  \expandafter\@firstofone
\else
  \expandafter\@gobble
\fi
{%
  \def\loop#1\repeat{%
    \def\body{#1}%
    \iterate
  }%
  \def\iterate{%
    \body
      \let\next\iterate
    \else
      \let\next\relax
    \fi
    \next
  }%
  \let\repeat=\fi
}%
\def\RestoreCatcodes{}
\count@=0 %
\loop
  \edef\RestoreCatcodes{%
    \RestoreCatcodes
    \catcode\the\count@=\the\catcode\count@\relax
  }%
\ifnum\count@<255 %
  \advance\count@ 1 %
\repeat

\def\RangeCatcodeInvalid#1#2{%
  \count@=#1\relax
  \loop
    \catcode\count@=15 %
  \ifnum\count@<#2\relax
    \advance\count@ 1 %
  \repeat
}
\def\RangeCatcodeCheck#1#2#3{%
  \count@=#1\relax
  \loop
    \ifnum#3=\catcode\count@
    \else
      \errmessage{%
        Character \the\count@\space
        with wrong catcode \the\catcode\count@\space
        instead of \number#3%
      }%
    \fi
  \ifnum\count@<#2\relax
    \advance\count@ 1 %
  \repeat
}
\def\space{ }
\expandafter\ifx\csname LoadCommand\endcsname\relax
  \def\LoadCommand{\input pdfescape.sty\relax}%
\fi
\def\Test{%
  \RangeCatcodeInvalid{0}{47}%
  \RangeCatcodeInvalid{58}{64}%
  \RangeCatcodeInvalid{91}{96}%
  \RangeCatcodeInvalid{123}{255}%
  \catcode`\@=12 %
  \catcode`\\=0 %
  \catcode`\%=14 %
  \LoadCommand
  \RangeCatcodeCheck{0}{36}{15}%
  \RangeCatcodeCheck{37}{37}{14}%
  \RangeCatcodeCheck{38}{47}{15}%
  \RangeCatcodeCheck{48}{57}{12}%
  \RangeCatcodeCheck{58}{63}{15}%
  \RangeCatcodeCheck{64}{64}{12}%
  \RangeCatcodeCheck{65}{90}{11}%
  \RangeCatcodeCheck{91}{91}{15}%
  \RangeCatcodeCheck{92}{92}{0}%
  \RangeCatcodeCheck{93}{96}{15}%
  \RangeCatcodeCheck{97}{122}{11}%
  \RangeCatcodeCheck{123}{255}{15}%
  \RestoreCatcodes
}
\Test
\csname @@end\endcsname
\end
%    \end{macrocode}
%    \begin{macrocode}
%</test1>
%    \end{macrocode}
%
% \subsection{Macro tests}
%
%    \begin{macrocode}
%<*test2|test3|test4|test5>
\NeedsTeXFormat{LaTeX2e}
\makeatletter
%    \end{macrocode}
%
% \subsection{Test with \cs{pdfescape...} commands}
%
%    \begin{macrocode}
%<*test2>
\ProvidesFile{pdfescape-test2.tex}%
    [2011/11/25 v1.13 Test with \string\pdfescape... commands]%
%</test2>
%    \end{macrocode}
%
% \subsection{Test without \cs{pdfescape...}, with \eTeX}
%
%    \begin{macrocode}
%<*test3>
\ProvidesFile{pdfescape-test3.tex}%
    [2011/11/25 v1.13 Test without \string\pdfescape..., with e-TeX]%
%</test3>
%    \end{macrocode}
%
% \subsection{Test without \cs{pdfescape...} and \eTeX}
%
%    \begin{macrocode}
%<*test4>
\ProvidesFile{pdfescape-test4.tex}%
    [2011/11/25 v1.13 Test without \string\pdfescape... and e-TeX]%
%</test4>
%    \end{macrocode}
%
% \subsection{Test with \LuaTeX}
%
%    \begin{macrocode}
%<*test5>
\ProvidesFile{pdfescape-test5.tex}%
    [2011/11/25 v1.13 Test with LuaTeX]%
%</test5>
%    \end{macrocode}
%
% \subsection{Check/ensure test preconditions}
%
% \subsubsection{Check \cs{pdfescape...}}
%
%    \begin{macrocode}
%<*test2>
\@ifundefined{pdfescapehex}{%
  \PackageError{pdfescape-test2}{%
    Missing \string\pdfescape... commands%
  }{Test aborted.}%
  \stop
}{}
%</test2>
%    \end{macrocode}
%    \begin{macrocode}
%<*test3|test4>
\let\pdfescapehex\@undefined
\let\pdfunescapehex\@undefined
\let\pdfescapename\@undefined
\let\pdfescapestring\@undefined
%</test3|test4>
%    \end{macrocode}
%
% \subsubsection{Check \eTeX}
%
%    \begin{macrocode}
%<*test3>
\@ifundefined{numexpr}{%
  \PackageError{pdfescape-test3}{%
    Missing \eTeX
  }{Test aborted.}%
  \stop
}{}
%</test3>
%    \end{macrocode}
% Package \xpackage{qstest} uses \eTeX, thus \eTeX's features
% can only be disabled later during loading of package \xpackage{pdfescape}.
%
% \subsubsection{Check \LuaTeX}
%
%    \begin{macrocode}
%<*test5>
\@ifundefined{directlua}{%
  \PackageError{pdfescape-test5}{%
    Missing LuaTeX%
  }{Test aborted.}%
  \stop
}{}
%</test5>
%    \end{macrocode}
%
% \subsection{Common part}
%
%    The files for testing uses the framework, provided by the
%    new package \xpackage{qstest} of David Kastrup.
%    \begin{macrocode}
\RequirePackage{qstest}
\IncludeTests{*}
\LogTests{log}{*}{*}

\newcommand*{\ExpectVar}[2]{%
  \ifx#1#2%
  \else
    \begingroup
      \@onelevel@sanitize#1%
      \@onelevel@sanitize#2%
      \typeout{[#1] <> [#2]}% hash-ok
    \endgroup
  \fi
  \Expect*{\ifx#1#2true\else false\fi}{true}%
}

\makeatletter
\begingroup
  \gdef\AllBytes{}%
  \count@=0 %
  \catcode0=12 %
  \@whilenum\count@<256 \do{%
    \lccode0=\count@
    \ifnum\count@=32 %
      \xdef\AllBytes{\AllBytes\space}%
    \else
      \lowercase{%
        \xdef\AllBytes{\AllBytes^^@}%
      }%
    \fi
    \advance\count@ by 1 %
  }%
\endgroup
\newcommand*{\AllBytesHex}{%
  000102030405060708090A0B0C0D0E0F%
  101112131415161718191A1B1C1D1E1F%
  202122232425262728292A2B2C2D2E2F%
  303132333435363738393A3B3C3D3E3F%
  404142434445464748494A4B4C4D4E4F%
  505152535455565758595A5B5C5D5E5F%
  606162636465666768696A6B6C6D6E6F%
  707172737475767778797A7B7C7D7E7F%
  808182838485868788898A8B8C8D8E8F%
  909192939495969798999A9B9C9D9E9F%
  A0A1A2A3A4A5A6A7A8A9AAABACADAEAF%
  B0B1B2B3B4B5B6B7B8B9BABBBCBDBEBF%
  C0C1C2C3C4C5C6C7C8C9CACBCCCDCECF%
  D0D1D2D3D4D5D6D7D8D9DADBDCDDDEDF%
  E0E1E2E3E4E5E6E7E8E9EAEBECEDEEEF%
  F0F1F2F3F4F5F6F7F8F9FAFBFCFDFEFF%
}
\@onelevel@sanitize\AllBytesHex
\expandafter\lowercase\expandafter{%
  \expandafter\newcommand\expandafter*\expandafter\AllBytesHexLC
      \expandafter{\AllBytesHex}%
}
\newcommand*{\AllBytesName}{}
\begingroup
  \catcode`\#=12 %
  \xdef\AllBytesName{%
    #01#02#03#04#05#06#07#08#09#0A#0B#0C#0D#0E#0F%
    #10#11#12#13#14#15#16#17#18#19#1A#1B#1C#1D#1E#1F%
    #20!"#23$#25&'#28#29*+,-.#2F%
    0123456789:;#3C=#3E?%
    @ABCDEFGHIJKLMNO%
    PQRSTUVWXYZ#5B\@backslashchar#5D^_%
    `abcdefghijklmno%
    pqrstuvwxyz#7B|#7D\string~#7F%
    #80#81#82#83#84#85#86#87#88#89#8A#8B#8C#8D#8E#8F%
    #90#91#92#93#94#95#96#97#98#99#9A#9B#9C#9D#9E#9F%
    #A0#A1#A2#A3#A4#A5#A6#A7#A8#A9#AA#AB#AC#AD#AE#AF%
    #B0#B1#B2#B3#B4#B5#B6#B7#B8#B9#BA#BB#BC#BD#BE#BF%
    #C0#C1#C2#C3#C4#C5#C6#C7#C8#C9#CA#CB#CC#CD#CE#CF%
    #D0#D1#D2#D3#D4#D5#D6#D7#D8#D9#DA#DB#DC#DD#DE#DF%
    #E0#E1#E2#E3#E4#E5#E6#E7#E8#E9#EA#EB#EC#ED#EE#EF%
    #F0#F1#F2#F3#F4#F5#F6#F7#F8#F9#FA#FB#FC#FD#FE#FF%
  }%
\endgroup
\@onelevel@sanitize\AllBytesName

\newcommand*{\AllBytesString}{}
\begingroup
  \def\|{|}%
  \edef\%{\@percentchar}%
  \catcode`\|=0 %
  \catcode`\#=12 %
  \catcode`\~=12 %
  \catcode`\\=12 %
  |xdef|AllBytesString{%
    \000\001\002\003\004\005\006\007\010\011\012\013\014\015\016\017%
    \020\021\022\023\024\025\026\027\030\031\032\033\034\035\036\037%
    \040!"#$|%&'\(\)*+,-./%
    0123456789:;<=>?%
    @ABCDEFGHIJKLMNO%
    PQRSTUVWXYZ[\\]^_%
    `abcdefghijklmno%
    pqrstuvwxyz{||}~\177%
    \200\201\202\203\204\205\206\207\210\211\212\213\214\215\216\217%
    \220\221\222\223\224\225\226\227\230\231\232\233\234\235\236\237%
    \240\241\242\243\244\245\246\247\250\251\252\253\254\255\256\257%
    \260\261\262\263\264\265\266\267\270\271\272\273\274\275\276\277%
    \300\301\302\303\304\305\306\307\310\311\312\313\314\315\316\317%
    \320\321\322\323\324\325\326\327\330\331\332\333\334\335\336\337%
    \340\341\342\343\344\345\346\347\350\351\352\353\354\355\356\357%
    \360\361\362\363\364\365\366\367\370\371\372\373\374\375\376\377%
  }%
|endgroup
\@onelevel@sanitize\AllBytesString

%<*test4>
\let\org@detokenize\detokenize
\let\detokenize\@undefined
\let\org@numexpr\numexpr
\let\numexpr\@undefined
%</test4>
\RequirePackage{pdfescape}
%<*test4>
\let\detokenize\org@detokenize
\let\numexpr\org@numexpr
%</test4>

\begin{qstest}{all-hex}{\AllBytes, escapehex}
  \EdefEscapeHex\x{\AllBytes}%
  \Expect*{\x}*{\AllBytesHex}%
  \ExpectVar\x\AllBytesHex
\end{qstest}

\begin{qstest}{all-unhex}{\AllBytesHex, unescapehex}
  \EdefUnescapeHex\x{\AllBytesHex}%
  \Expect*{\x}*{\AllBytes}%
  \ExpectVar\x\AllBytes
\end{qstest}

\begin{qstest}{all-unhex-lc}{\AllBytesHexLC, unescapehex, lowercase}
  \EdefUnescapeHex\x{\AllBytesHexLC}%
  \Expect*{\x}*{\AllBytes}%
  \ExpectVar\x\AllBytes
\end{qstest}

\begin{qstest}{unhex-incomplete}{unescapehex, incomplete}
  \EdefUnescapeHex\x{4}%
  \Expect*{\x}{@}%
\end{qstest}

\begin{qstest}{unhex-space}{unescapehex, space}
  \EdefUnescapeHex\x{20}%
  \Expect*{\x}{ }%
  \ExpectVar\x\space
\end{qstest}

\begin{qstest}{unhex-spaces}{unescapehex, spaces}
  \EdefUnescapeHex\x{204020204120}%
  \def\y#1{%
    \edef\z{#1\string @#1#1\string A#1}%
  }\y{ }%
  \Expect*{\x}*{\z}%
  \ExpectVar\x\z
\end{qstest}

\begin{qstest}{unhex-hash}{unescapehex, hash}
  \catcode`\#=12 %
  \EdefUnescapeHex\x{#20}%
  \ExpectVar\x\space
\end{qstest}

\begin{qstest}{unhex-invalid}{unescapehex, invalid}
  \def\test#1#2{%
    \EdefUnescapeHex\x{#1}%
    \edef\y{#2}%
    \@onelevel@sanitize\y
    \ExpectVar\x\y
  }%
%<*test2>
  \edef\x{\pdfunescapehex{4X}}%
  \edef\y{\string @}%
  \ifx\x\y
  \else
    \def~{\space}%
    \typeout{************************************}%
    \typeout{* Your pdfTeX contains bug 777.~~~~*}%
    \typeout{* This test is redefined as dummy, *}%
    \typeout{* because it triggers the bug.~~~~~*}%
    \typeout{************************************}%
    \def\test#1#2{}%
  \fi
%</test2>
  \test{X}{}%
  \test{XY}{}%
  \test{XYZ}{}%
  \test{A}{^^a0}%
  \test{AX}{^^a0}%
  \test{XA}{^^a0}%
  \test{XXAXX}{^^a0}%
\end{qstest}

\begin{qstest}{all-name}{\AllBytes, escapename}
  \EdefEscapeName\x{\AllBytes}%
  \Expect*{\x}*{\AllBytesName}%
  \ExpectVar\x\AllBytesName
\end{qstest}

\begin{qstest}{all-string}{\AllBytes, escapestring}
  \EdefEscapeString\x{\AllBytes}%
  \Expect*{\x}*{\AllBytesString}%
  \ExpectVar\x\AllBytesString
\end{qstest}

\begin{qstest}{uchexdigit}{unescape, uppercase hex digit}
  \catcode`\@=11 %
  \catcode0=12 %
  \def\test#1#2{%
    \uccode0=#1\relax
    \uppercase{%
      \def\x{^^@}%
    }%
    \Expect*{%
       \ifcase\expandafter\PE@TestUcHexDigit\x
         true%
       \else
         false%
       \fi
    }{#2}%
  }%
  \def\range#1#2#3{%
    \count0=#1\relax
    \loop
    \ifnum\count0<#2\relax
      \test{\count0}{#3}%
      \advance\count0 by 1 %
    \repeat
  }%
  \range{0}{47}{false}%
  \range{48}{57}{true}%
  \range{58}{64}{false}%
  \range{65}{70}{true}%
  \range{71}{255}{false}%
\end{qstest}

\begin{qstest}{unescapename}{unescapename}
  \def\test#1#2{%
    \EdefUnescapeName\x{#1}%
    \edef\y{#2}%
    \@onelevel@sanitize\y
    \ExpectVar\x\y
  }%
  \catcode`\#=12 %
  \catcode0=12 %
  \test{}{}%
  \test{x}{x}%
  \test{xy}{xy}%
  \test{#}{#}%
  \test{##}{##}%
  \test{###}{###}%
  \test{####}{####}%
  \test{#x}{#x}%
  \test{#xy}{#xy}%
  \test{#1}{#1}%
  \test{#40}{@}%
  \test{#400}{@0}%
  \test{#4x0}{#4x0}%
  \test{#ab}{^^ab}%
  \test{#00}{^^@}%
  \test{x#40y#40z}{x@y@z}%
  \test{#40#40#40#40}{@@@@}%
  \test{a#x}{a#x}%
  \test{a#xy}{a#xy}%
  \test{a#1}{a#1}%
  \test{a#40}{a@}%
  \test{a#400}{a@0}%
  \test{#20}{ }%
  \test{a#20}{a }%
  \test{a#20b}{a b}%
  \test{a#20#20#20b}{a \space\space b}%
\end{qstest}

\begin{qstest}{unescapestring}{unescapestring}
  \def\test#1#2{%
    \EdefUnescapeString\x{#1}%
    \edef\y{#2}%
    \@onelevel@sanitize\y
    \ExpectVar\x\y
  }%
  \catcode0=12 %
  \def\DefChar#1#2{%
    \begingroup
      \uccode0=#2\relax
    \uppercase{\endgroup
      \def#1{^^@}%
    }%
  }%
  \DefChar\nul{0}%
  \DefChar\one{1}%
  \DefChar\bel{8}%
  \DefChar\tab{9}%
  \DefChar\lf{10}%
  \DefChar\ff{12}%
  \DefChar\cr{13}%
  \DefChar\\{92}%
  \test{}{}%
  \test{a}{a}%
  \test{\\}{}%
  \test{\\\\}{\\}%
  \test{\\\\y}{\\y}%
  \test{\\000}{\nul}%
  \test{\\b}{\bel}%
  \test{\\t}{\tab}%
  \test{\\n}{\lf}%
  \test{\\f}{\ff}%
  \test{\\r}{\cr}%
  \test{\\(}{(}%
  \test{\\)}{)}%
  \test{\\040}{ }%
  \test{\\100}{@}%
  \test{\\40}{ }%
  \test{\\1}{\one}%
  \test{\\01}{\one}%
  \test{\\001}{\one}%
  \test{\\18}{\one8}%
  \test{\\018}{\one8}%
  \test{\\0018}{\one8}%
  \test{x\\}{x}%
  \test{x\\\\}{x\\}%
  \test{x\\\\y}{x\\y}%
  \test{x\\000}{x\nul}%
  \test{x\\b}{x\bel}%
  \test{x\\t}{x\tab}%
  \test{x\\n}{x\lf}%
  \test{x\\f}{x\ff}%
  \test{x\\r}{x\cr}%
  \test{x\\(}{x(}%
  \test{x\\)}{x)}%
  \test{x\\040}{x }%
  \test{x\\100}{x@}%
  \test{x\\40}{x }%
  \test{x\\1}{x\one}%
  \test{x\\01}{x\one}%
  \test{x\\001}{x\one}%
  \test{x\\18}{x\one8}%
  \test{x\\018}{x\one8}%
  \test{x\\0018}{x\one8}%
  \test{\\b\\t\\n\\f\\r\\(\\)\\\\\\000\\040}{%
    \bel\tab\lf\ff\cr()\\\nul\space
  }%
  \test{\\\lf}{}%
  \test{x\\\lf}{x}%
  \test{\cr}{\lf}%
  \test{\cr\lf}{\lf}%
  \test{\lf}{\lf}%
  \test{\lf\cr}{\lf\lf}%
  \test{x\cr}{x\lf}%
  \test{x\cr\lf}{x\lf}%
  \test{x\lf}{x\lf}%
  \test{x\lf\cr}{x\lf\lf}%
  \test{x\\\cr\lf y\cr}{xy\lf}%
%
  \test{\\409}{ 9}%
  \test{\\800}{800}%
  \test{\\900}{900}%
  \test{\\578}{/8}%
  \test{\\477}{?}%
  \test{\\377}{^^ff}%
  \test{\\777}{^^ff}%
  \test{\\7777}{^^ff7}%
\end{qstest}
\stop
%    \end{macrocode}
%
%    \begin{macrocode}
%</test2|test3|test4|test5>
%    \end{macrocode}
%
% \subsubsection{Test for \hologo{iniTeX}}
%
%    \begin{macrocode}
%<*test6>
\input pdfescape.sty\relax
\catcode`\{=1 %
\catcode`\}=2 %
\catcode`\#=6 %
\catcode`\^=7 %
\catcode`\@=11 %
%    \end{macrocode}
%    \begin{macrocode}
\begingroup
  \catcode`\@=11 %
  \countdef\count@=255 %
  \def\space{ }%
  \long\def\@whilenum#1\do #2{%
    \ifnum #1\relax
      #2\relax
      \@iwhilenum{#1\relax#2\relax}%
    \fi
  }%
  \long\def\@iwhilenum#1{%
    \ifnum #1%
      \expandafter\@iwhilenum
    \else
      \expandafter\ltx@gobble
    \fi
    {#1}%
  }%
  \gdef\AllBytes{}%
  \count@=0 %
  \catcode0=12 %
  \@whilenum\count@<256 \do{%
    \lccode0=\count@
    \ifnum\count@=32 %
      \xdef\AllBytes{\AllBytes\space}%
    \else
      \lowercase{%
        \xdef\AllBytes{\AllBytes^^@}%
      }%
    \fi
    \advance\count@ by 1 %
  }%
\endgroup
\def\AllBytesHex{%
  000102030405060708090A0B0C0D0E0F%
  101112131415161718191A1B1C1D1E1F%
  202122232425262728292A2B2C2D2E2F%
  303132333435363738393A3B3C3D3E3F%
  404142434445464748494A4B4C4D4E4F%
  505152535455565758595A5B5C5D5E5F%
  606162636465666768696A6B6C6D6E6F%
  707172737475767778797A7B7C7D7E7F%
  808182838485868788898A8B8C8D8E8F%
  909192939495969798999A9B9C9D9E9F%
  A0A1A2A3A4A5A6A7A8A9AAABACADAEAF%
  B0B1B2B3B4B5B6B7B8B9BABBBCBDBEBF%
  C0C1C2C3C4C5C6C7C8C9CACBCCCDCECF%
  D0D1D2D3D4D5D6D7D8D9DADBDCDDDEDF%
  E0E1E2E3E4E5E6E7E8E9EAEBECEDEEEF%
  F0F1F2F3F4F5F6F7F8F9FAFBFCFDFEFF%
}
\ltx@onelevel@sanitize\AllBytesHex
\expandafter\lowercase\expandafter{%
  \expandafter\def\expandafter\AllBytesHexLC
      \expandafter{\AllBytesHex}%
}
\begingroup
  \catcode`\#=12 %
  \xdef\AllBytesName{%
    #01#02#03#04#05#06#07#08#09#0A#0B#0C#0D#0E#0F%
    #10#11#12#13#14#15#16#17#18#19#1A#1B#1C#1D#1E#1F%
    #20!"#23$#25&'#28#29*+,-.#2F%
    0123456789:;#3C=#3E?%
    @ABCDEFGHIJKLMNO%
    PQRSTUVWXYZ#5B\ltx@backslashchar#5D^_%
    `abcdefghijklmno%
    pqrstuvwxyz#7B|#7D\string~#7F%
    #80#81#82#83#84#85#86#87#88#89#8A#8B#8C#8D#8E#8F%
    #90#91#92#93#94#95#96#97#98#99#9A#9B#9C#9D#9E#9F%
    #A0#A1#A2#A3#A4#A5#A6#A7#A8#A9#AA#AB#AC#AD#AE#AF%
    #B0#B1#B2#B3#B4#B5#B6#B7#B8#B9#BA#BB#BC#BD#BE#BF%
    #C0#C1#C2#C3#C4#C5#C6#C7#C8#C9#CA#CB#CC#CD#CE#CF%
    #D0#D1#D2#D3#D4#D5#D6#D7#D8#D9#DA#DB#DC#DD#DE#DF%
    #E0#E1#E2#E3#E4#E5#E6#E7#E8#E9#EA#EB#EC#ED#EE#EF%
    #F0#F1#F2#F3#F4#F5#F6#F7#F8#F9#FA#FB#FC#FD#FE#FF%
  }%
\endgroup
\ltx@onelevel@sanitize\AllBytesName
\edef\AllBytesFromName{\expandafter\ltx@gobble\AllBytes}

\begingroup
  \def\|{|}%
  \edef\%{\ltx@percentchar}%
  \catcode`\|=0 %
  \catcode`\#=12 %
  \catcode`\~=12 %
  \catcode`\\=12 %
  |xdef|AllBytesString{%
    \000\001\002\003\004\005\006\007\010\011\012\013\014\015\016\017%
    \020\021\022\023\024\025\026\027\030\031\032\033\034\035\036\037%
    \040!"#$|%&'\(\)*+,-./%
    0123456789:;<=>?%
    @ABCDEFGHIJKLMNO%
    PQRSTUVWXYZ[\\]^_%
    `abcdefghijklmno%
    pqrstuvwxyz{||}~\177%
    \200\201\202\203\204\205\206\207\210\211\212\213\214\215\216\217%
    \220\221\222\223\224\225\226\227\230\231\232\233\234\235\236\237%
    \240\241\242\243\244\245\246\247\250\251\252\253\254\255\256\257%
    \260\261\262\263\264\265\266\267\270\271\272\273\274\275\276\277%
    \300\301\302\303\304\305\306\307\310\311\312\313\314\315\316\317%
    \320\321\322\323\324\325\326\327\330\331\332\333\334\335\336\337%
    \340\341\342\343\344\345\346\347\350\351\352\353\354\355\356\357%
    \360\361\362\363\364\365\366\367\370\371\372\373\374\375\376\377%
  }%
|endgroup
\ltx@onelevel@sanitize\AllBytesString
%    \end{macrocode}
%    \begin{macrocode}
\def\msg#{\immediate\write16}
\def\Test#1#2#3{%
  \begingroup
    #1\TestResult{#2}%
    \ifx\TestResult#3%
    \else
      \newlinechar=10 %
      \msg{Expect:^^J#3}%
      \msg{Result:^^J\TestResult}%
      \errmessage{\string#2 -\string#1-> \string#3}%
    \fi
  \endgroup
}
\Test\EdefEscapeHex\AllBytes\AllBytesHex
\Test\EdefUnescapeHex\AllBytesHex\AllBytes
\Test\EdefEscapeName\AllBytes\AllBytesName
\Test\EdefUnescapeName\AllBytesName\AllBytesFromName
\Test\EdefEscapeString\AllBytes\AllBytesString
\Test\EdefUnescapeString\AllBytesString\AllBytes
%    \end{macrocode}
%    \begin{macrocode}
\csname @@end\endcsname\end
%</test6>
%    \end{macrocode}
%
% \section{Installation}
%
% \subsection{Download}
%
% \paragraph{Package.} This package is available on
% CTAN\footnote{\url{ftp://ftp.ctan.org/tex-archive/}}:
% \begin{description}
% \item[\CTAN{macros/latex/contrib/oberdiek/pdfescape.dtx}] The source file.
% \item[\CTAN{macros/latex/contrib/oberdiek/pdfescape.pdf}] Documentation.
% \end{description}
%
%
% \paragraph{Bundle.} All the packages of the bundle `oberdiek'
% are also available in a TDS compliant ZIP archive. There
% the packages are already unpacked and the documentation files
% are generated. The files and directories obey the TDS standard.
% \begin{description}
% \item[\CTAN{install/macros/latex/contrib/oberdiek.tds.zip}]
% \end{description}
% \emph{TDS} refers to the standard ``A Directory Structure
% for \TeX\ Files'' (\CTAN{tds/tds.pdf}). Directories
% with \xfile{texmf} in their name are usually organized this way.
%
% \subsection{Bundle installation}
%
% \paragraph{Unpacking.} Unpack the \xfile{oberdiek.tds.zip} in the
% TDS tree (also known as \xfile{texmf} tree) of your choice.
% Example (linux):
% \begin{quote}
%   |unzip oberdiek.tds.zip -d ~/texmf|
% \end{quote}
%
% \paragraph{Script installation.}
% Check the directory \xfile{TDS:scripts/oberdiek/} for
% scripts that need further installation steps.
% Package \xpackage{attachfile2} comes with the Perl script
% \xfile{pdfatfi.pl} that should be installed in such a way
% that it can be called as \texttt{pdfatfi}.
% Example (linux):
% \begin{quote}
%   |chmod +x scripts/oberdiek/pdfatfi.pl|\\
%   |cp scripts/oberdiek/pdfatfi.pl /usr/local/bin/|
% \end{quote}
%
% \subsection{Package installation}
%
% \paragraph{Unpacking.} The \xfile{.dtx} file is a self-extracting
% \docstrip\ archive. The files are extracted by running the
% \xfile{.dtx} through \plainTeX:
% \begin{quote}
%   \verb|tex pdfescape.dtx|
% \end{quote}
%
% \paragraph{TDS.} Now the different files must be moved into
% the different directories in your installation TDS tree
% (also known as \xfile{texmf} tree):
% \begin{quote}
% \def\t{^^A
% \begin{tabular}{@{}>{\ttfamily}l@{ $\rightarrow$ }>{\ttfamily}l@{}}
%   pdfescape.sty & tex/generic/oberdiek/pdfescape.sty\\
%   pdfescape.pdf & doc/latex/oberdiek/pdfescape.pdf\\
%   test/pdfescape-test1.tex & doc/latex/oberdiek/test/pdfescape-test1.tex\\
%   test/pdfescape-test2.tex & doc/latex/oberdiek/test/pdfescape-test2.tex\\
%   test/pdfescape-test3.tex & doc/latex/oberdiek/test/pdfescape-test3.tex\\
%   test/pdfescape-test4.tex & doc/latex/oberdiek/test/pdfescape-test4.tex\\
%   test/pdfescape-test5.tex & doc/latex/oberdiek/test/pdfescape-test5.tex\\
%   test/pdfescape-test6.tex & doc/latex/oberdiek/test/pdfescape-test6.tex\\
%   pdfescape.dtx & source/latex/oberdiek/pdfescape.dtx\\
% \end{tabular}^^A
% }^^A
% \sbox0{\t}^^A
% \ifdim\wd0>\linewidth
%   \begingroup
%     \advance\linewidth by\leftmargin
%     \advance\linewidth by\rightmargin
%   \edef\x{\endgroup
%     \def\noexpand\lw{\the\linewidth}^^A
%   }\x
%   \def\lwbox{^^A
%     \leavevmode
%     \hbox to \linewidth{^^A
%       \kern-\leftmargin\relax
%       \hss
%       \usebox0
%       \hss
%       \kern-\rightmargin\relax
%     }^^A
%   }^^A
%   \ifdim\wd0>\lw
%     \sbox0{\small\t}^^A
%     \ifdim\wd0>\linewidth
%       \ifdim\wd0>\lw
%         \sbox0{\footnotesize\t}^^A
%         \ifdim\wd0>\linewidth
%           \ifdim\wd0>\lw
%             \sbox0{\scriptsize\t}^^A
%             \ifdim\wd0>\linewidth
%               \ifdim\wd0>\lw
%                 \sbox0{\tiny\t}^^A
%                 \ifdim\wd0>\linewidth
%                   \lwbox
%                 \else
%                   \usebox0
%                 \fi
%               \else
%                 \lwbox
%               \fi
%             \else
%               \usebox0
%             \fi
%           \else
%             \lwbox
%           \fi
%         \else
%           \usebox0
%         \fi
%       \else
%         \lwbox
%       \fi
%     \else
%       \usebox0
%     \fi
%   \else
%     \lwbox
%   \fi
% \else
%   \usebox0
% \fi
% \end{quote}
% If you have a \xfile{docstrip.cfg} that configures and enables \docstrip's
% TDS installing feature, then some files can already be in the right
% place, see the documentation of \docstrip.
%
% \subsection{Refresh file name databases}
%
% If your \TeX~distribution
% (\teTeX, \mikTeX, \dots) relies on file name databases, you must refresh
% these. For example, \teTeX\ users run \verb|texhash| or
% \verb|mktexlsr|.
%
% \subsection{Some details for the interested}
%
% \paragraph{Attached source.}
%
% The PDF documentation on CTAN also includes the
% \xfile{.dtx} source file. It can be extracted by
% AcrobatReader 6 or higher. Another option is \textsf{pdftk},
% e.g. unpack the file into the current directory:
% \begin{quote}
%   \verb|pdftk pdfescape.pdf unpack_files output .|
% \end{quote}
%
% \paragraph{Unpacking with \LaTeX.}
% The \xfile{.dtx} chooses its action depending on the format:
% \begin{description}
% \item[\plainTeX:] Run \docstrip\ and extract the files.
% \item[\LaTeX:] Generate the documentation.
% \end{description}
% If you insist on using \LaTeX\ for \docstrip\ (really,
% \docstrip\ does not need \LaTeX), then inform the autodetect routine
% about your intention:
% \begin{quote}
%   \verb|latex \let\install=y\input{pdfescape.dtx}|
% \end{quote}
% Do not forget to quote the argument according to the demands
% of your shell.
%
% \paragraph{Generating the documentation.}
% You can use both the \xfile{.dtx} or the \xfile{.drv} to generate
% the documentation. The process can be configured by the
% configuration file \xfile{ltxdoc.cfg}. For instance, put this
% line into this file, if you want to have A4 as paper format:
% \begin{quote}
%   \verb|\PassOptionsToClass{a4paper}{article}|
% \end{quote}
% An example follows how to generate the
% documentation with pdf\LaTeX:
% \begin{quote}
%\begin{verbatim}
%pdflatex pdfescape.dtx
%makeindex -s gind.ist pdfescape.idx
%pdflatex pdfescape.dtx
%makeindex -s gind.ist pdfescape.idx
%pdflatex pdfescape.dtx
%\end{verbatim}
% \end{quote}
%
% \section{Catalogue}
%
% The following XML file can be used as source for the
% \href{http://mirror.ctan.org/help/Catalogue/catalogue.html}{\TeX\ Catalogue}.
% The elements \texttt{caption} and \texttt{description} are imported
% from the original XML file from the Catalogue.
% The name of the XML file in the Catalogue is \xfile{pdfescape.xml}.
%    \begin{macrocode}
%<*catalogue>
<?xml version='1.0' encoding='us-ascii'?>
<!DOCTYPE entry SYSTEM 'catalogue.dtd'>
<entry datestamp='$Date$' modifier='$Author$' id='pdfescape'>
  <name>pdfescape</name>
  <caption>Implements pdfTeX's escape features using TeX or e-TeX.</caption>
  <authorref id='auth:oberdiek'/>
  <copyright owner='Heiko Oberdiek' year='2007,2010,2011'/>
  <license type='lppl1.3'/>
  <version number='1.13'/>
  <description>
    This package implements <xref refid='pdftex'>pdfTeX</xref>'s
    escape features (<tt>\pdfescapehex</tt>, <tt>\pdfunescapehex</tt>,
    <tt>\pdfescapename</tt>, <tt>\pdfescapestring</tt>) using TeX or
    e-TeX.
    <p/>
    The package is part of the <xref refid='oberdiek'>oberdiek</xref>
    bundle.
  </description>
  <documentation details='Package documentation'
      href='ctan:/macros/latex/contrib/oberdiek/pdfescape.pdf'/>
  <ctan file='true' path='/macros/latex/contrib/oberdiek/pdfescape.dtx'/>
  <miktex location='oberdiek'/>
  <texlive location='oberdiek'/>
  <install path='/macros/latex/contrib/oberdiek/oberdiek.tds.zip'/>
</entry>
%</catalogue>
%    \end{macrocode}
%
% \begin{History}
%   \begin{Version}{2007/02/21 v1.0}
%   \item
%     First version.
%   \end{Version}
%   \begin{Version}{2007/02/25 v1.1}
%   \item
%     Test files added.
%   \item
%     \cs{EdefUnescapeHex} supports lowercase letters.
%   \item
%     Fix: \verb|\EdefEscapeName{^^@}|
%   \item
%     Fix: \verb|\EdefEscapeName{\string#}|
%   \item
%     Fix for \cs{EdefUnescapeHex} in case of incomplete hex string.
%   \item
%     Fix: \cs{EdefUnescapeHex} generates space tokens with catcode
%     10 (space) in all cases.
%   \item
%     Fix: \cs{EdefEscapeHex} and \cs{EdefEscapeName} now generate
%     tokens with catcode 12 (other) only.
%   \end{Version}
%   \begin{Version}{2007/03/20 v1.2}
%   \item
%     Fix: Wrong year in \cs{ProvidesPackage}.
%   \end{Version}
%   \begin{Version}{2007/04/11 v1.3}
%   \item
%     Line ends sanitized.
%   \end{Version}
%   \begin{Version}{2007/04/21 v1.4}
%   \item
%     \cs{EdefUnescapeName} and \cs{EdefUnescapeString} added.
%   \end{Version}
%   \begin{Version}{2007/08/27 v1.5}
%   \item
%     \cs{EdefSanitize} added (replaces \cs{PE@sanitize}).
%   \end{Version}
%   \begin{Version}{2007/09/09 v1.6}
%   \item
%     Fix in catcode setup.
%   \end{Version}
%   \begin{Version}{2007/10/27 v1.7}
%   \item
%     More efficient \cs{EdefSanitize}.
%   \end{Version}
%   \begin{Version}{2007/11/11 v1.8}
%   \item
%     Use of package \xpackage{pdftexcmds} for \hologo{LuaTeX} support.
%   \end{Version}
%   \begin{Version}{2010/03/01 v1.9}
%   \item
%     Compatibility with \hologo{iniTeX}.
%   \end{Version}
%   \begin{Version}{2010/11/12 v1.10}
%   \item
%     Use of package \xpackage{ltxcmds}.
%   \item
%     Fix for compatibility with \hologo{iniTeX}.
%   \end{Version}
%   \begin{Version}{2011/01/30 v1.11}
%   \item
%     Already loaded package files are not input in \hologo{plainTeX}.
%   \end{Version}
%   \begin{Version}{2011/04/04 v1.12}
%   \item
%     Further fixes for compatibility for \hologo{iniTeX}.
%   \item
%     Test file for \hologo{iniTeX} added.
%   \end{Version}
%   \begin{Version}{2011/11/25 v1.13}
%   \item
%     Higher order bit of octal sequences in \cs{EdefUnescapeString}
%     ignored according to the PDF specification
%     (Bug found by Bruno Le Floch).
%   \end{Version}
% \end{History}
%
% \PrintIndex
%
% \Finale
\endinput
|
% \end{quote}
% Do not forget to quote the argument according to the demands
% of your shell.
%
% \paragraph{Generating the documentation.}
% You can use both the \xfile{.dtx} or the \xfile{.drv} to generate
% the documentation. The process can be configured by the
% configuration file \xfile{ltxdoc.cfg}. For instance, put this
% line into this file, if you want to have A4 as paper format:
% \begin{quote}
%   \verb|\PassOptionsToClass{a4paper}{article}|
% \end{quote}
% An example follows how to generate the
% documentation with pdf\LaTeX:
% \begin{quote}
%\begin{verbatim}
%pdflatex pdfescape.dtx
%makeindex -s gind.ist pdfescape.idx
%pdflatex pdfescape.dtx
%makeindex -s gind.ist pdfescape.idx
%pdflatex pdfescape.dtx
%\end{verbatim}
% \end{quote}
%
% \section{Catalogue}
%
% The following XML file can be used as source for the
% \href{http://mirror.ctan.org/help/Catalogue/catalogue.html}{\TeX\ Catalogue}.
% The elements \texttt{caption} and \texttt{description} are imported
% from the original XML file from the Catalogue.
% The name of the XML file in the Catalogue is \xfile{pdfescape.xml}.
%    \begin{macrocode}
%<*catalogue>
<?xml version='1.0' encoding='us-ascii'?>
<!DOCTYPE entry SYSTEM 'catalogue.dtd'>
<entry datestamp='$Date$' modifier='$Author$' id='pdfescape'>
  <name>pdfescape</name>
  <caption>Implements pdfTeX's escape features using TeX or e-TeX.</caption>
  <authorref id='auth:oberdiek'/>
  <copyright owner='Heiko Oberdiek' year='2007,2010,2011'/>
  <license type='lppl1.3'/>
  <version number='1.13'/>
  <description>
    This package implements <xref refid='pdftex'>pdfTeX</xref>'s
    escape features (<tt>\pdfescapehex</tt>, <tt>\pdfunescapehex</tt>,
    <tt>\pdfescapename</tt>, <tt>\pdfescapestring</tt>) using TeX or
    e-TeX.
    <p/>
    The package is part of the <xref refid='oberdiek'>oberdiek</xref>
    bundle.
  </description>
  <documentation details='Package documentation'
      href='ctan:/macros/latex/contrib/oberdiek/pdfescape.pdf'/>
  <ctan file='true' path='/macros/latex/contrib/oberdiek/pdfescape.dtx'/>
  <miktex location='oberdiek'/>
  <texlive location='oberdiek'/>
  <install path='/macros/latex/contrib/oberdiek/oberdiek.tds.zip'/>
</entry>
%</catalogue>
%    \end{macrocode}
%
% \begin{History}
%   \begin{Version}{2007/02/21 v1.0}
%   \item
%     First version.
%   \end{Version}
%   \begin{Version}{2007/02/25 v1.1}
%   \item
%     Test files added.
%   \item
%     \cs{EdefUnescapeHex} supports lowercase letters.
%   \item
%     Fix: \verb|\EdefEscapeName{^^@}|
%   \item
%     Fix: \verb|\EdefEscapeName{\string#}|
%   \item
%     Fix for \cs{EdefUnescapeHex} in case of incomplete hex string.
%   \item
%     Fix: \cs{EdefUnescapeHex} generates space tokens with catcode
%     10 (space) in all cases.
%   \item
%     Fix: \cs{EdefEscapeHex} and \cs{EdefEscapeName} now generate
%     tokens with catcode 12 (other) only.
%   \end{Version}
%   \begin{Version}{2007/03/20 v1.2}
%   \item
%     Fix: Wrong year in \cs{ProvidesPackage}.
%   \end{Version}
%   \begin{Version}{2007/04/11 v1.3}
%   \item
%     Line ends sanitized.
%   \end{Version}
%   \begin{Version}{2007/04/21 v1.4}
%   \item
%     \cs{EdefUnescapeName} and \cs{EdefUnescapeString} added.
%   \end{Version}
%   \begin{Version}{2007/08/27 v1.5}
%   \item
%     \cs{EdefSanitize} added (replaces \cs{PE@sanitize}).
%   \end{Version}
%   \begin{Version}{2007/09/09 v1.6}
%   \item
%     Fix in catcode setup.
%   \end{Version}
%   \begin{Version}{2007/10/27 v1.7}
%   \item
%     More efficient \cs{EdefSanitize}.
%   \end{Version}
%   \begin{Version}{2007/11/11 v1.8}
%   \item
%     Use of package \xpackage{pdftexcmds} for \hologo{LuaTeX} support.
%   \end{Version}
%   \begin{Version}{2010/03/01 v1.9}
%   \item
%     Compatibility with \hologo{iniTeX}.
%   \end{Version}
%   \begin{Version}{2010/11/12 v1.10}
%   \item
%     Use of package \xpackage{ltxcmds}.
%   \item
%     Fix for compatibility with \hologo{iniTeX}.
%   \end{Version}
%   \begin{Version}{2011/01/30 v1.11}
%   \item
%     Already loaded package files are not input in \hologo{plainTeX}.
%   \end{Version}
%   \begin{Version}{2011/04/04 v1.12}
%   \item
%     Further fixes for compatibility for \hologo{iniTeX}.
%   \item
%     Test file for \hologo{iniTeX} added.
%   \end{Version}
%   \begin{Version}{2011/11/25 v1.13}
%   \item
%     Higher order bit of octal sequences in \cs{EdefUnescapeString}
%     ignored according to the PDF specification
%     (Bug found by Bruno Le Floch).
%   \end{Version}
% \end{History}
%
% \PrintIndex
%
% \Finale
\endinput
|
% \end{quote}
% Do not forget to quote the argument according to the demands
% of your shell.
%
% \paragraph{Generating the documentation.}
% You can use both the \xfile{.dtx} or the \xfile{.drv} to generate
% the documentation. The process can be configured by the
% configuration file \xfile{ltxdoc.cfg}. For instance, put this
% line into this file, if you want to have A4 as paper format:
% \begin{quote}
%   \verb|\PassOptionsToClass{a4paper}{article}|
% \end{quote}
% An example follows how to generate the
% documentation with pdf\LaTeX:
% \begin{quote}
%\begin{verbatim}
%pdflatex pdfescape.dtx
%makeindex -s gind.ist pdfescape.idx
%pdflatex pdfescape.dtx
%makeindex -s gind.ist pdfescape.idx
%pdflatex pdfescape.dtx
%\end{verbatim}
% \end{quote}
%
% \section{Catalogue}
%
% The following XML file can be used as source for the
% \href{http://mirror.ctan.org/help/Catalogue/catalogue.html}{\TeX\ Catalogue}.
% The elements \texttt{caption} and \texttt{description} are imported
% from the original XML file from the Catalogue.
% The name of the XML file in the Catalogue is \xfile{pdfescape.xml}.
%    \begin{macrocode}
%<*catalogue>
<?xml version='1.0' encoding='us-ascii'?>
<!DOCTYPE entry SYSTEM 'catalogue.dtd'>
<entry datestamp='$Date$' modifier='$Author$' id='pdfescape'>
  <name>pdfescape</name>
  <caption>Implements pdfTeX's escape features using TeX or e-TeX.</caption>
  <authorref id='auth:oberdiek'/>
  <copyright owner='Heiko Oberdiek' year='2007,2010,2011'/>
  <license type='lppl1.3'/>
  <version number='1.13'/>
  <description>
    This package implements <xref refid='pdftex'>pdfTeX</xref>'s
    escape features (<tt>\pdfescapehex</tt>, <tt>\pdfunescapehex</tt>,
    <tt>\pdfescapename</tt>, <tt>\pdfescapestring</tt>) using TeX or
    e-TeX.
    <p/>
    The package is part of the <xref refid='oberdiek'>oberdiek</xref>
    bundle.
  </description>
  <documentation details='Package documentation'
      href='ctan:/macros/latex/contrib/oberdiek/pdfescape.pdf'/>
  <ctan file='true' path='/macros/latex/contrib/oberdiek/pdfescape.dtx'/>
  <miktex location='oberdiek'/>
  <texlive location='oberdiek'/>
  <install path='/macros/latex/contrib/oberdiek/oberdiek.tds.zip'/>
</entry>
%</catalogue>
%    \end{macrocode}
%
% \begin{History}
%   \begin{Version}{2007/02/21 v1.0}
%   \item
%     First version.
%   \end{Version}
%   \begin{Version}{2007/02/25 v1.1}
%   \item
%     Test files added.
%   \item
%     \cs{EdefUnescapeHex} supports lowercase letters.
%   \item
%     Fix: \verb|\EdefEscapeName{^^@}|
%   \item
%     Fix: \verb|\EdefEscapeName{\string#}|
%   \item
%     Fix for \cs{EdefUnescapeHex} in case of incomplete hex string.
%   \item
%     Fix: \cs{EdefUnescapeHex} generates space tokens with catcode
%     10 (space) in all cases.
%   \item
%     Fix: \cs{EdefEscapeHex} and \cs{EdefEscapeName} now generate
%     tokens with catcode 12 (other) only.
%   \end{Version}
%   \begin{Version}{2007/03/20 v1.2}
%   \item
%     Fix: Wrong year in \cs{ProvidesPackage}.
%   \end{Version}
%   \begin{Version}{2007/04/11 v1.3}
%   \item
%     Line ends sanitized.
%   \end{Version}
%   \begin{Version}{2007/04/21 v1.4}
%   \item
%     \cs{EdefUnescapeName} and \cs{EdefUnescapeString} added.
%   \end{Version}
%   \begin{Version}{2007/08/27 v1.5}
%   \item
%     \cs{EdefSanitize} added (replaces \cs{PE@sanitize}).
%   \end{Version}
%   \begin{Version}{2007/09/09 v1.6}
%   \item
%     Fix in catcode setup.
%   \end{Version}
%   \begin{Version}{2007/10/27 v1.7}
%   \item
%     More efficient \cs{EdefSanitize}.
%   \end{Version}
%   \begin{Version}{2007/11/11 v1.8}
%   \item
%     Use of package \xpackage{pdftexcmds} for \hologo{LuaTeX} support.
%   \end{Version}
%   \begin{Version}{2010/03/01 v1.9}
%   \item
%     Compatibility with \hologo{iniTeX}.
%   \end{Version}
%   \begin{Version}{2010/11/12 v1.10}
%   \item
%     Use of package \xpackage{ltxcmds}.
%   \item
%     Fix for compatibility with \hologo{iniTeX}.
%   \end{Version}
%   \begin{Version}{2011/01/30 v1.11}
%   \item
%     Already loaded package files are not input in \hologo{plainTeX}.
%   \end{Version}
%   \begin{Version}{2011/04/04 v1.12}
%   \item
%     Further fixes for compatibility for \hologo{iniTeX}.
%   \item
%     Test file for \hologo{iniTeX} added.
%   \end{Version}
%   \begin{Version}{2011/11/25 v1.13}
%   \item
%     Higher order bit of octal sequences in \cs{EdefUnescapeString}
%     ignored according to the PDF specification
%     (Bug found by Bruno Le Floch).
%   \end{Version}
% \end{History}
%
% \PrintIndex
%
% \Finale
\endinput

%        (quote the arguments according to the demands of your shell)
%
% Documentation:
%    (a) If pdfescape.drv is present:
%           latex pdfescape.drv
%    (b) Without pdfescape.drv:
%           latex pdfescape.dtx; ...
%    The class ltxdoc loads the configuration file ltxdoc.cfg
%    if available. Here you can specify further options, e.g.
%    use A4 as paper format:
%       \PassOptionsToClass{a4paper}{article}
%
%    Programm calls to get the documentation (example):
%       pdflatex pdfescape.dtx
%       makeindex -s gind.ist pdfescape.idx
%       pdflatex pdfescape.dtx
%       makeindex -s gind.ist pdfescape.idx
%       pdflatex pdfescape.dtx
%
% Installation:
%    TDS:tex/generic/oberdiek/pdfescape.sty
%    TDS:doc/latex/oberdiek/pdfescape.pdf
%    TDS:doc/latex/oberdiek/test/pdfescape-test1.tex
%    TDS:doc/latex/oberdiek/test/pdfescape-test2.tex
%    TDS:doc/latex/oberdiek/test/pdfescape-test3.tex
%    TDS:doc/latex/oberdiek/test/pdfescape-test4.tex
%    TDS:doc/latex/oberdiek/test/pdfescape-test5.tex
%    TDS:doc/latex/oberdiek/test/pdfescape-test6.tex
%    TDS:source/latex/oberdiek/pdfescape.dtx
%
%<*ignore>
\begingroup
  \catcode123=1 %
  \catcode125=2 %
  \def\x{LaTeX2e}%
\expandafter\endgroup
\ifcase 0\ifx\install y1\fi\expandafter
         \ifx\csname processbatchFile\endcsname\relax\else1\fi
         \ifx\fmtname\x\else 1\fi\relax
\else\csname fi\endcsname
%</ignore>
%<*install>
\input docstrip.tex
\Msg{************************************************************************}
\Msg{* Installation}
\Msg{* Package: pdfescape 2011/11/25 v1.13 Implements pdfTeX's escape features (HO)}
\Msg{************************************************************************}

\keepsilent
\askforoverwritefalse

\let\MetaPrefix\relax
\preamble

This is a generated file.

Project: pdfescape
Version: 2011/11/25 v1.13

Copyright (C) 2007, 2010, 2011 by
   Heiko Oberdiek <heiko.oberdiek at googlemail.com>

This work may be distributed and/or modified under the
conditions of the LaTeX Project Public License, either
version 1.3c of this license or (at your option) any later
version. This version of this license is in
   http://www.latex-project.org/lppl/lppl-1-3c.txt
and the latest version of this license is in
   http://www.latex-project.org/lppl.txt
and version 1.3 or later is part of all distributions of
LaTeX version 2005/12/01 or later.

This work has the LPPL maintenance status "maintained".

This Current Maintainer of this work is Heiko Oberdiek.

The Base Interpreter refers to any `TeX-Format',
because some files are installed in TDS:tex/generic//.

This work consists of the main source file pdfescape.dtx
and the derived files
   pdfescape.sty, pdfescape.pdf, pdfescape.ins, pdfescape.drv,
   pdfescape-test1.tex, pdfescape-test2.tex,
   pdfescape-test3.tex, pdfescape-test4.tex,
   pdfescape-test5.tex, pdfescape-test6.tex.

\endpreamble
\let\MetaPrefix\DoubleperCent

\generate{%
  \file{pdfescape.ins}{\from{pdfescape.dtx}{install}}%
  \file{pdfescape.drv}{\from{pdfescape.dtx}{driver}}%
  \usedir{tex/generic/oberdiek}%
  \file{pdfescape.sty}{\from{pdfescape.dtx}{package}}%
  \usedir{doc/latex/oberdiek/test}%
  \file{pdfescape-test1.tex}{\from{pdfescape.dtx}{test1}}%
  \file{pdfescape-test2.tex}{\from{pdfescape.dtx}{test2}}%
  \file{pdfescape-test3.tex}{\from{pdfescape.dtx}{test3}}%
  \file{pdfescape-test4.tex}{\from{pdfescape.dtx}{test4}}%
  \file{pdfescape-test5.tex}{\from{pdfescape.dtx}{test5}}%
  \file{pdfescape-test6.tex}{\from{pdfescape.dtx}{test6}}%
  \nopreamble
  \nopostamble
  \usedir{source/latex/oberdiek/catalogue}%
  \file{pdfescape.xml}{\from{pdfescape.dtx}{catalogue}}%
}

\catcode32=13\relax% active space
\let =\space%
\Msg{************************************************************************}
\Msg{*}
\Msg{* To finish the installation you have to move the following}
\Msg{* file into a directory searched by TeX:}
\Msg{*}
\Msg{*     pdfescape.sty}
\Msg{*}
\Msg{* To produce the documentation run the file `pdfescape.drv'}
\Msg{* through LaTeX.}
\Msg{*}
\Msg{* Happy TeXing!}
\Msg{*}
\Msg{************************************************************************}

\endbatchfile
%</install>
%<*ignore>
\fi
%</ignore>
%<*driver>
\NeedsTeXFormat{LaTeX2e}
\ProvidesFile{pdfescape.drv}%
  [2011/11/25 v1.13 Implements pdfTeX's escape features (HO)]%
\documentclass{ltxdoc}
\usepackage{holtxdoc}[2011/11/22]
\begin{document}
  \DocInput{pdfescape.dtx}%
\end{document}
%</driver>
% \fi
%
% \CheckSum{2516}
%
% \CharacterTable
%  {Upper-case    \A\B\C\D\E\F\G\H\I\J\K\L\M\N\O\P\Q\R\S\T\U\V\W\X\Y\Z
%   Lower-case    \a\b\c\d\e\f\g\h\i\j\k\l\m\n\o\p\q\r\s\t\u\v\w\x\y\z
%   Digits        \0\1\2\3\4\5\6\7\8\9
%   Exclamation   \!     Double quote  \"     Hash (number) \#
%   Dollar        \$     Percent       \%     Ampersand     \&
%   Acute accent  \'     Left paren    \(     Right paren   \)
%   Asterisk      \*     Plus          \+     Comma         \,
%   Minus         \-     Point         \.     Solidus       \/
%   Colon         \:     Semicolon     \;     Less than     \<
%   Equals        \=     Greater than  \>     Question mark \?
%   Commercial at \@     Left bracket  \[     Backslash     \\
%   Right bracket \]     Circumflex    \^     Underscore    \_
%   Grave accent  \`     Left brace    \{     Vertical bar  \|
%   Right brace   \}     Tilde         \~}
%
% \GetFileInfo{pdfescape.drv}
%
% \title{The \xpackage{pdfescape} package}
% \date{2011/11/25 v1.13}
% \author{Heiko Oberdiek\\\xemail{heiko.oberdiek at googlemail.com}}
%
% \maketitle
%
% \begin{abstract}
% This package implements \pdfTeX's escape features
% (\cs{pdfescapehex}, \cs{pdfunescapehex}, \cs{pdfescapename},
% \cs{pdfescapestring}) using \TeX\ or \eTeX.
% \end{abstract}
%
% \tableofcontents
%
% \section{Documentation}
%
% \begin{declcs}{EdefEscapeHex} \M{cmd} \M{string}\\
%  \cs{EdefUnescapeHex} \M{cmd} \M{string}\\
%  \cs{EdefEscapeName} \M{cmd} \M{string}\\
%  \cs{EdefEscapeString} \M{cmd} \M{string}
% \end{declcs}
% These commands converts \meta{string} and stores the result
% in macro \meta{cmd}. The conversion result is the same
% as the conversion of the corresponding \pdfTeX's primitives.
% Note that the argument \meta{string} is expanded before
% the conversion.
%
% For example, if \pdfTeX\ \textgreater= 1.30 is present,
% then \cs{EdefEscapeHex}
% becomes to:
% \begin{quote}
%\begin{verbatim}
%\def\EdefEscapeHex#1#2{%
%  \edef#1{\pdfescapehex{#2}}%
%}
%\end{verbatim}
% \end{quote}
% The package provides implementations for the case that \pdfTeX\ is
% not present (or too old). Even \eTeX\ can be missing, however it is
% used if it is detected.
%
% \paragraph{Babel.}
% The input strings may contain shorthand characters of package
% \xpackage{babel}.
%
% \subsection{Additional unescape macros}
%
% \begin{declcs}{EdefUnescapeName} \M{cmd} \M{string}
% \end{declcs}
% Sequences of a hash sign with two hexadecimal digits are
% converted to the corresponding character (PDF-1.2).
% A hash sign that is not followed by two hexadecimal digits
% is left unchanged. The catcodes in the result string follow
% TeX's conventions. The space has catcode 10 (space) and
% the other characters have catcode 12 (other).
%
% \begin{declcs}{EdefUnescapeString} \M{cmd} \M{string}
% \end{declcs}
% Macro \meta{cmd} stores the unescaped string in \meta{string}.
% All the rules for literal strings are implemented, see
% PDF specification.
% The catcodes in the result string follow TeX's conventions.
%
% \subsection{Sanitizing macro}
%
% \begin{declcs}{EdefSanitize} \M{cmd} \M{string}
% \end{declcs}
% Argument \meta{string} is expanded, converted to a string
% of tokens with catcode 12 (other) and space tokens, and
% stored in macro \meta{cmd}.
%
% \StopEventually{
% }
%
% \section{Implementation}
%
%    \begin{macrocode}
%<*package>
%    \end{macrocode}
%
% \subsection{Reload check and package identification}
%    Reload check, especially if the package is not used with \LaTeX.
%    \begin{macrocode}
\begingroup\catcode61\catcode48\catcode32=10\relax%
  \catcode13=5 % ^^M
  \endlinechar=13 %
  \catcode35=6 % #
  \catcode39=12 % '
  \catcode44=12 % ,
  \catcode45=12 % -
  \catcode46=12 % .
  \catcode58=12 % :
  \catcode64=11 % @
  \catcode123=1 % {
  \catcode125=2 % }
  \expandafter\let\expandafter\x\csname ver@pdfescape.sty\endcsname
  \ifx\x\relax % plain-TeX, first loading
  \else
    \def\empty{}%
    \ifx\x\empty % LaTeX, first loading,
      % variable is initialized, but \ProvidesPackage not yet seen
    \else
      \expandafter\ifx\csname PackageInfo\endcsname\relax
        \def\x#1#2{%
          \immediate\write-1{Package #1 Info: #2.}%
        }%
      \else
        \def\x#1#2{\PackageInfo{#1}{#2, stopped}}%
      \fi
      \x{pdfescape}{The package is already loaded}%
      \aftergroup\endinput
    \fi
  \fi
\endgroup%
%    \end{macrocode}
%    Package identification:
%    \begin{macrocode}
\begingroup\catcode61\catcode48\catcode32=10\relax%
  \catcode13=5 % ^^M
  \endlinechar=13 %
  \catcode35=6 % #
  \catcode39=12 % '
  \catcode40=12 % (
  \catcode41=12 % )
  \catcode44=12 % ,
  \catcode45=12 % -
  \catcode46=12 % .
  \catcode47=12 % /
  \catcode58=12 % :
  \catcode64=11 % @
  \catcode91=12 % [
  \catcode93=12 % ]
  \catcode123=1 % {
  \catcode125=2 % }
  \expandafter\ifx\csname ProvidesPackage\endcsname\relax
    \def\x#1#2#3[#4]{\endgroup
      \immediate\write-1{Package: #3 #4}%
      \xdef#1{#4}%
    }%
  \else
    \def\x#1#2[#3]{\endgroup
      #2[{#3}]%
      \ifx#1\@undefined
        \xdef#1{#3}%
      \fi
      \ifx#1\relax
        \xdef#1{#3}%
      \fi
    }%
  \fi
\expandafter\x\csname ver@pdfescape.sty\endcsname
\ProvidesPackage{pdfescape}%
  [2011/11/25 v1.13 Implements pdfTeX's escape features (HO)]%
%    \end{macrocode}
%
% \subsection{Catcodes}
%
%    \begin{macrocode}
\begingroup\catcode61\catcode48\catcode32=10\relax%
  \catcode13=5 % ^^M
  \endlinechar=13 %
  \catcode123 1 % {
  \catcode125 2 % }
  \catcode64 11 %
  \def\x{\endgroup
    \expandafter\edef\csname PE@AtEnd\endcsname{%
      \endlinechar=\the\endlinechar\relax
      \catcode13=\the\catcode13\relax
      \catcode32=\the\catcode32\relax
      \catcode35=\the\catcode35\relax
      \catcode61=\the\catcode61\relax
      \catcode64=\the\catcode64\relax
      \catcode123=\the\catcode123\relax
      \catcode125=\the\catcode125\relax
    }%
  }%
\x\catcode61\catcode48\catcode32=10\relax%
\catcode13=5 % ^^M
\endlinechar=13 %
\catcode35=6 % #
\catcode64=11 % @
\catcode123=1 % {
\catcode125=2 % }
\def\TMP@EnsureCode#1#2#3{%
  \edef\PE@AtEnd{%
    \PE@AtEnd
    #1#2=\the#1#2\relax
  }%
  #1#2=#3\relax
}
\TMP@EnsureCode\catcode{0}{12}% ^^@
\TMP@EnsureCode\catcode{34}{12}% "
\TMP@EnsureCode\catcode{36}{3}% $
\TMP@EnsureCode\catcode{38}{4}% &
\TMP@EnsureCode\catcode{39}{12}% '
\TMP@EnsureCode\catcode{42}{12}% *
\TMP@EnsureCode\catcode{45}{12}% -
\TMP@EnsureCode\catcode{46}{12}% .
\TMP@EnsureCode\catcode{47}{12}% /
\TMP@EnsureCode\catcode{60}{12}% <
\TMP@EnsureCode\catcode{62}{12}% >
\TMP@EnsureCode\catcode{91}{12}% [
\TMP@EnsureCode\catcode{93}{12}% ]
\TMP@EnsureCode\catcode{94}{7}% ^
\TMP@EnsureCode\catcode{96}{12}% `
\TMP@EnsureCode\uccode{34}{0}% "
\TMP@EnsureCode\uccode{48}{0}% 0
\TMP@EnsureCode\uccode{61}{0}% =
\edef\PE@AtEnd{\PE@AtEnd\noexpand\endinput}
%    \end{macrocode}
%
% \subsection{Load package}
%
%    \begin{macrocode}
\begingroup\expandafter\expandafter\expandafter\endgroup
\expandafter\ifx\csname RequirePackage\endcsname\relax
  \def\TMP@RequirePackage#1[#2]{%
    \begingroup\expandafter\expandafter\expandafter\endgroup
    \expandafter\ifx\csname ver@#1.sty\endcsname\relax
      \input #1.sty\relax
    \fi
  }%
  \TMP@RequirePackage{ltxcmds}[2010/04/08]%
\else
  \RequirePackage{ltxcmds}[2010/04/08]%
\fi
%    \end{macrocode}
%
% \subsection{Sanitizing}
%
%    \begin{macro}{\EdefSanitize}
%    Macro \cs{EdefSanitize} takes |#2|, entirely converts
%    it to token with catcode 12 (other) and stores the
%    result in macro |#1|.
%    \begin{macrocode}
\begingroup\expandafter\expandafter\expandafter\endgroup
\expandafter\ifx\csname detokenize\endcsname\relax
  \long\def\EdefSanitize#1#2{%
    \begingroup
      \csname @safe@activestrue\endcsname
      \edef#1{#2}%
      \PE@onelevel@sanitize#1%
    \expandafter\endgroup
    \expandafter\def\expandafter#1\expandafter{#1}%
  }%
  \begingroup\expandafter\expandafter\expandafter\endgroup
  \expandafter\ifx\csname @onelevel@sanitize\endcsname\relax
    \def\PE@onelevel@sanitize#1{%
      \edef#1{\expandafter\PE@strip@prefix\meaning#1}%
    }%
    \def\PE@strip@prefix#1>{}%
  \else
    \let\PE@onelevel@sanitize\@onelevel@sanitize
  \fi
\else
  \long\def\EdefSanitize#1#2{%
    \begingroup
      \csname @safe@activestrue\endcsname
      \edef#1{#2}%
    \expandafter\endgroup
    \expandafter\def\expandafter#1\expandafter{%
      \detokenize\expandafter{#1}%
    }%
  }%
  \def\PE@onelevel@sanitize#1{%
    \edef#1{\detokenize\expandafter{#1}}%
  }%
\fi
%    \end{macrocode}
%    \end{macro}
%    \begin{macro}{\PE@sanitize}
%    Macro \cs{PE@sanitize} is only defined for
%    compatibility with version 1.4. Its use is deprecated.
%    \begin{macrocode}
\let\PE@sanitize\EdefSanitize
%    \end{macrocode}
%    \end{macro}
%
% \subsubsection{Space characters}
%
%    \begin{macro}{\PE@space@other}
%    \begin{macrocode}
\begingroup
  \catcode`\ =12\relax%
\def\x{\endgroup\def\PE@space@other{ }}\x\relax
%    \end{macrocode}
%    \end{macro}
%    \begin{macro}{\PE@space@space}
%    \begin{macrocode}
\def\PE@space@space{ }
%    \end{macrocode}
%    \end{macro}
%
% \subsubsection{Space normalization}
%
%    \begin{macro}{\PE@SanitizeSpaceOther}
%    \begin{macrocode}
\def\PE@SanitizeSpaceOther#1{%
  \edef#1{\expandafter\PE@SpaceToOther#1 \relax}%
}
%    \end{macrocode}
%    \end{macro}
%
%    \begin{macro}{\PE@SpaceToOther}
%    \begin{macrocode}
\def\PE@SpaceToOther#1 #2\relax{%
  #1%
  \ifx\\#2\\%
  \else
    \PE@space@other
    \ltx@ReturnAfterFi{%
      \PE@SpaceToOther#2\relax
    }%
  \fi
}
%    \end{macrocode}
%    \end{macro}
%
% \subsection{\cs{EdefUnescapeName}}
%
%    \begin{macro}{\EdefUnescapeName}
%    \begin{macrocode}
\def\EdefUnescapeName#1#2{%
  \EdefSanitize#1{#2}%
  \PE@SanitizeSpaceOther#1%
  \PE@UnescapeName#1%
  \PE@onelevel@sanitize#1%
}
%    \end{macrocode}
%    \end{macro}
%
%    \begin{macro}{\PE@UnescapeName}
%    \begin{macrocode}
\begingroup
  \catcode`\$=6  % hash
  \catcode`\#=12 % other
  \gdef\PE@UnescapeName$1{%
    \begingroup
      \PE@InitUccodeHexDigit
      \def\PE@result{}%
      \expandafter\PE@DeName$1#\relax\relax
    \expandafter\endgroup
    \expandafter\def\expandafter$1\expandafter{\PE@result}%
  }%
  \gdef\PE@DeName$1#$2$3{%
    \ifx\relax$2%
      \edef\PE@result{\PE@result$1}%
      \let\PE@next\relax
    \else
      \ifx\relax$3%
        % wrong escape sequence in input
        \edef\PE@result{\PE@result$1#}%
        \let\PE@next\relax
      \else
        \uppercase{%
          \def\PE@testA{$2}%
          \def\PE@testB{$3}%
        }%
        \ifcase\ifcase\expandafter\PE@TestUcHexDigit\PE@testA
                 \ifcase\expandafter\PE@TestUcHexDigit\PE@testB
                   \ltx@zero
                 \else
                   \ltx@one
                 \fi
               \else
                 \ltx@one
               \fi
          \uccode\ltx@zero="\PE@testA\PE@testB\relax
          \uppercase{%
            \def\PE@temp{^^@}%
          }%
          \uccode\ltx@zero=\ltx@zero
          \edef\PE@result{\PE@result$1\PE@temp}%
          \let\PE@next\PE@DeName
        \else
          % wrong escape sequence in input
          \edef\PE@result{\PE@result$1#}%
          \def\PE@next{\PE@DeName$2$3}%
        \fi
      \fi
    \fi
    \PE@next
  }%
\endgroup
%    \end{macrocode}
%    \end{macro}
%
%    \begin{macro}{\PE@InitUccodeHexDigit}
%    \begin{macrocode}
\def\PE@InitUccodeHexDigit{%
  \uccode`a=`A\relax
  \uccode`b=`B\relax
  \uccode`c=`C\relax
  \uccode`d=`D\relax
  \uccode`e=`E\relax
  \uccode`f=`F\relax
  \uccode`A=\ltx@zero
  \uccode`B=\ltx@zero
  \uccode`C=\ltx@zero
  \uccode`D=\ltx@zero
  \uccode`E=\ltx@zero
  \uccode`F=\ltx@zero
  \uccode`0=\ltx@zero
  \uccode`1=\ltx@zero
  \uccode`2=\ltx@zero
  \uccode`3=\ltx@zero
  \uccode`4=\ltx@zero
  \uccode`5=\ltx@zero
  \uccode`6=\ltx@zero
  \uccode`7=\ltx@zero
  \uccode`8=\ltx@zero
  \uccode`9=\ltx@zero
}
%    \end{macrocode}
%    \end{macro}
%
%    \begin{macro}{\PE@TestUcHexDigit}
%    \begin{macrocode}
\def\PE@TestUcHexDigit#1{%
  \ifnum`#1<48 % 0
    \ltx@one
  \else
    \ifnum`#1>70 % F
      \ltx@one
    \else
      \ifnum`#1>57 % 9
        \ifnum`#1<65 % A
          \ltx@one
        \else
          \ltx@zero
        \fi
      \else
        \ltx@zero
      \fi
    \fi
  \fi
}
%    \end{macrocode}
%    \end{macro}
%
% \subsection{\cs{EdefUnescapeString}}
%
%    \begin{macro}{\EdefUnescapeString}
%    \begin{macrocode}
\def\EdefUnescapeString#1#2{%
  \EdefSanitize#1{#2}%
  \PE@SanitizeSpaceOther#1%
  \PE@NormalizeLineEnd#1%
  \PE@UnescapeString#1%
  \PE@onelevel@sanitize#1%
}
%    \end{macrocode}
%    \end{macro}
%
%    \begin{macrocode}
\begingroup
  \uccode`\8=10 % lf
  \uccode`\9=13 % cr
\def\x#1#2{\endgroup
%    \end{macrocode}
%    \begin{macro}{\PE@NormalizeLineEnd}
%    \begin{macrocode}
  \def\PE@NormalizeLineEnd##1{%
    \def\PE@result{}%
    \expandafter\PE@@NormalizeLineEnd##1#2\relax
    \let##1\PE@result
  }%
%    \end{macrocode}
%    \end{macro}
%    \begin{macro}{\PE@@NormalizeLineEnd}
%    \begin{macrocode}
  \def\PE@@NormalizeLineEnd##1#2##2{%
    \ifx\relax##2%
      \edef\PE@result{\PE@result##1}%
      \let\PE@next\relax
    \else
      \edef\PE@result{\PE@result##1#1}%
      \ifx#1##2% lf
        \let\PE@next\PE@@NormalizeLineEnd
      \else
        \def\PE@next{\PE@@NormalizeLineEnd##2}%
      \fi
    \fi
    \PE@next
  }%
}%
\uppercase{%
  \x 89%
}
%    \end{macrocode}
%    \end{macro}
%
%    \begin{macrocode}
\begingroup
  \catcode`\|=0 %
  \catcode`\\=12 %
%    \end{macrocode}
%    \begin{macro}{\PE@UnescapeString}
%    \begin{macrocode}
  |gdef|PE@UnescapeString#1{%
    |begingroup
      |def|PE@result{}%
      |expandafter|PE@DeString#1\|relax
    |expandafter|endgroup
    |expandafter|def|expandafter#1|expandafter{|PE@result}%
  }%
%    \end{macrocode}
%    \end{macro}
%    \begin{macro}{\PE@DeString}
%    \begin{macrocode}
  |gdef|PE@DeString#1\#2{%
    |ifx|relax#2%
      |edef|PE@result{|PE@result#1}%
      |let|PE@next|relax
    |else
      |if n#2%
        |uccode|ltx@zero=10 %
      |else|if r#2%
        |uccode|ltx@zero=13 %
      |else|if t#2%
        |uccode|ltx@zero=9 %
      |else|if b#2%
        |uccode|ltx@zero=8 %
      |else|if f#2%
        |uccode|ltx@zero=12 %
      |else
        |uccode|ltx@zero=|ltx@zero
      |fi|fi|fi|fi|fi
      |ifnum|uccode|ltx@zero>|ltx@zero
        |uppercase{%
          |edef|PE@temp{^^@}%
        }%
        |edef|PE@result{|PE@result#1|PE@temp}%
        |let|PE@next|PE@DeString
      |else
        |if\#2% backslash
          |edef|PE@result{|PE@result#1}%
          |let|PE@next|PE@CheckEndBackslash
        |else
          |ifnum`#2=10 % linefeed
            |edef|PE@result{|PE@result#1}%
            |let|PE@next|PE@DeString
          |else
            |ifcase|PE@TestOctDigit#2%
              |edef|PE@result{|PE@result#1}%
              |def|PE@next{|PE@OctI#2}%
            |else
              |edef|PE@result{|PE@result#1#2}%
              |let|PE@next|PE@DeString
            |fi
          |fi
        |fi
      |fi
    |fi
    |PE@next
  }%
%    \end{macrocode}
%    \end{macro}
%    \begin{macro}{\PE@CheckEndBackslash}
%    \begin{macrocode}
  |gdef|PE@CheckEndBackslash#1{%
    |ifx|relax#1%
    |else
      |edef|PE@result{|PE@result\}%
      |expandafter|PE@DeString|expandafter#1%
    |fi
  }%
%    \end{macrocode}
%    \end{macro}
%    \begin{macrocode}
|endgroup
%    \end{macrocode}
%    \begin{macro}{\PE@TestOctDigit}
%    \begin{macrocode}
\def\PE@TestOctDigit#1{%
  \ifnum`#1<48 % 0
    \ltx@one
  \else
    \ifnum`#1>55 % 7
      \ltx@one
    \else
      \ltx@zero
    \fi
  \fi
}
%    \end{macrocode}
%    \end{macro}
%    \begin{macro}{\PE@OctI}
%    \begin{macrocode}
\def\PE@OctI#1#2{%
  \ifcase\PE@TestOctDigit#2%
    \def\PE@next{\PE@OctII{#1#2}}%
  \else
    \def\PE@next{\PE@OctAll#1#2}%
  \fi
  \PE@next
}
%    \end{macrocode}
%    \end{macro}
%    \begin{macro}{\PE@OctII}
%    \begin{macrocode}
\def\PE@OctII#1#2{%
  \ifcase\PE@TestOctDigit#2%
    \def\PE@next{\PE@OctIII#1#2}%
  \else
    \def\PE@next{\PE@OctAll{#1}#2}%
  \fi
  \PE@next
}
%    \end{macrocode}
%    \end{macro}
%    \begin{macrocode}
\ltx@IfUndefined{numexpr}{%
  \catcode`\$=9 %
  \catcode`\&=14 %
}{%
  \catcode`\$=14 %
  \catcode`\&=9 %
}
%    \end{macrocode}
%    \begin{macro}{\PE@OctIII}
%    \begin{macrocode}
\def\PE@OctIII#1#2#3{%
  \ifnum#1<4 %
    \def\PE@next{\PE@OctAll{#1#2#3}}%
  \else
$   \count\ltx@cclv#1 %
$   \advance\count\ltx@cclv -4 %
    \edef\PE@next{%
      \noexpand\PE@OctAll{%
$       \the\count\ltx@cclv
&       \the\numexpr#1-4\relax
        #2#3%
      }%
    }%
  \fi
  \PE@next
}
%    \end{macrocode}
%    \end{macro}
%    \begin{macro}{\PE@OctAll}
%    \begin{macrocode}
\def\PE@OctAll#1{%
  \uccode\ltx@zero='#1\relax
  \uppercase{%
    \edef\PE@result{\PE@result^^@}%
  }%
  \PE@DeString
}
%    \end{macrocode}
%    \end{macro}
%
% \subsection{User macros (\pdfTeX\space analogues)}
%
%    \begin{macrocode}
\begingroup\expandafter\expandafter\expandafter\endgroup
\expandafter\ifx\csname RequirePackage\endcsname\relax
  \def\TMP@RequirePackage#1[#2]{%
    \begingroup\expandafter\expandafter\expandafter\endgroup
    \expandafter\ifx\csname ver@#1.sty\endcsname\relax
      \input #1.sty\relax
    \fi
  }%
  \TMP@RequirePackage{pdftexcmds}[2007/11/11]%
\else
  \RequirePackage{pdftexcmds}[2007/11/11]%
\fi
%    \end{macrocode}
%
%    \begin{macrocode}
\begingroup\expandafter\expandafter\expandafter\endgroup
\expandafter\ifx\csname pdf@escapehex\endcsname\relax
%    \end{macrocode}
%
%    \begin{macro}{\EdefEscapeHex}
%    \begin{macrocode}
  \long\def\EdefEscapeHex#1#2{%
    \EdefSanitize#1{#2}%
    \PE@SanitizeSpaceOther#1%
    \PE@EscapeHex#1%
  }%
%    \end{macrocode}
%    \end{macro}
%    \begin{macro}{\EdefUnescapeHex}
%    \begin{macrocode}
  \def\EdefUnescapeHex#1#2{%
    \EdefSanitize#1{#2}%
    \PE@UnescapeHex#1%
  }%
%    \end{macrocode}
%    \end{macro}
%    \begin{macro}{\EdefEscapeName}
%    \begin{macrocode}
  \long\def\EdefEscapeName#1#2{%
    \EdefSanitize#1{#2}%
    \PE@SanitizeSpaceOther#1%
    \PE@EscapeName#1%
  }%
%    \end{macrocode}
%    \end{macro}
%    \begin{macro}{\EdefEscapeString}
%    \begin{macrocode}
  \long\def\EdefEscapeString#1#2{%
    \EdefSanitize#1{#2}%
    \PE@SanitizeSpaceOther#1%
    \PE@EscapeString#1%
  }%
%    \end{macrocode}
%    \end{macro}
%
%    \begin{macrocode}
\else
%    \end{macrocode}
%
%    \begin{macro}{\PE@edefbabel}
%    Help macro that adds support for \xpackage{babel}'s shorthand
%    characters.
%    \begin{macrocode}
  \long\def\PE@edefbabel#1#2#3{%
    \begingroup
      \csname @save@activestrue\endcsname
      \edef#1{#2{#3}}%
    \expandafter\endgroup
    \expandafter\def\expandafter#1\expandafter{#1}%
  }%
%    \end{macrocode}
%    \end{macro}
%
%    \begin{macro}{\EdefEscapeHex}
%    \begin{macrocode}
  \long\def\EdefEscapeHex#1#2{%
    \PE@edefbabel#1\pdf@escapehex{#2}%
  }%
%    \end{macrocode}
%    \end{macro}
%    \begin{macro}{\EdefUnescapeHex}
%    \begin{macrocode}
  \def\EdefUnescapeHex#1#2{%
    \PE@edefbabel#1\pdf@unescapehex{#2}%
  }%
%    \end{macrocode}
%    \end{macro}
%    \begin{macro}{\EdefEscapeName}
%    \begin{macrocode}
  \long\def\EdefEscapeName#1#2{%
    \PE@edefbabel#1\pdf@escapename{#2}%
  }%
%    \end{macrocode}
%    \end{macro}
%    \begin{macro}{\EdefEscapeString}
%    \begin{macrocode}
  \long\def\EdefEscapeString#1#2{%
    \PE@edefbabel#1\pdf@escapestring{#2}%
  }%
%    \end{macrocode}
%    \end{macro}
%
%    \begin{macrocode}
  \expandafter\PE@AtEnd
\fi%
%    \end{macrocode}
%
% \subsection{Help macros}
%
% \subsubsection{Characters}
%
%    Special characters with catcode 12 (other) are created and
%    stored in macros.
%    \begin{macro}{\PE@hash}
%    \begin{macrocode}
\edef\PE@hash{\string#}
%    \end{macrocode}
%    \end{macro}
%    \begin{macro}{\PE@backslash}
%    \begin{macrocode}
\begingroup
  \escapechar=-1 %
\edef\x{\endgroup
  \def\noexpand\PE@backslash{\string\\}%
}
\x
%    \end{macrocode}
%    \end{macro}
%
% \subsubsection{Switch for \eTeX}
%
%    \begin{macrocode}
\ltx@newif\ifPE@etex
\begingroup\expandafter\expandafter\expandafter\endgroup
\expandafter\ifx\csname numexpr\endcsname\relax
\else
  \PE@etextrue
\fi
%    \end{macrocode}
%
% \subsection{Conversions}
%
% \subsubsection{Conversion to hex string}
%
%    \begin{macro}{\PE@EscapeHex}
%    \begin{macrocode}
\ifPE@etex
  \def\PE@EscapeHex#1{%
    \edef#1{\expandafter\PE@ToHex#1\relax}%
  }%
\else
  \def\PE@EscapeHex#1{%
    \def\PE@result{}%
    \expandafter\PE@ToHex#1\relax
    \let#1\PE@result
  }%
\fi
%    \end{macrocode}
%    \end{macro}
%    \begin{macro}{\PE@ToHex}
%    \begin{macrocode}
\def\PE@ToHex#1{%
  \ifx\relax#1%
  \else
    \PE@HexChar{#1}%
    \expandafter\PE@ToHex
  \fi
}%
%    \end{macrocode}
%    \end{macro}
%    \begin{macro}{\PE@HexChar}
%    \begin{macrocode}
\ifPE@etex
  \def\PE@HexChar#1{%
    \PE@HexDigit{\numexpr\dimexpr.0625\dimexpr`#1sp\relax\relax\relax}%
    \PE@HexDigit{%
      \numexpr`#1-16*\dimexpr.0625\dimexpr`#1sp\relax\relax\relax
    }%
  }%
\else
  \def\PE@HexChar#1{%
    \dimen0=`#1sp%
    \dimen2=.0625\dimen0 %
    \advance\dimen0-16\dimen2 %
    \edef\PE@result{%
      \PE@result
      \PE@HexDigit{\dimen2 }%
      \PE@HexDigit{\dimen0 }%
    }%
  }%
\fi
%    \end{macrocode}
%    \end{macro}
%    \begin{macro}{\PE@HexDigit}
%    \begin{macrocode}
\def\PE@HexDigit#1{%
  \expandafter\string
  \ifcase#1%
    0\or 1\or 2\or 3\or 4\or 5\or 6\or 7\or 8\or 9\or
    A\or B\or C\or D\or E\or F%
  \fi
}
%    \end{macrocode}
%    \end{macro}
%
% \subsubsection{Character code to octal number}
%
%    \begin{macro}{\PE@OctChar}
%    \begin{macrocode}
\ifPE@etex
  \def\PE@OctChar#1{%
    \expandafter\PE@@OctChar
        \the\numexpr\dimexpr.015625\dimexpr`#1sp\relax\relax
            \expandafter\relax
        \expandafter\relax
        \the\numexpr\dimexpr.125\dimexpr`#1sp\relax\relax\relax
        \relax
        #1%
  }%
  \def\PE@@OctChar#1\relax#2\relax#3{%
    \PE@backslash
    #1%
    \the\numexpr#2-8*#1\relax
    \the\numexpr\dimexpr`#3sp\relax-8*#2\relax
  }%
\else
  \def\PE@OctChar#1{%
    \dimen0=`#1sp%
    \dimen2=.125\dimen0 %
    \dimen4=.125\dimen2 %
    \advance\dimen0-8\dimen2 %
    \advance\dimen2-8\dimen4 %
    \edef\PE@result{%
      \PE@result
      \PE@backslash
      \number\dimen4 %
      \number\dimen2 %
      \number\dimen0 %
    }%
  }%
\fi
%    \end{macrocode}
%    \end{macro}
%
% \subsubsection{Unpack hex string}
%
%    \begin{macro}{\PE@UnescapeHex}
%    \begin{macrocode}
\def\PE@UnescapeHex#1{%
  \begingroup
    \PE@InitUccodeHexDigit
    \def\PE@result{}%
    \expandafter\PE@DeHex#1\relax\relax
  \expandafter\endgroup
  \expandafter\def\expandafter#1\expandafter{\PE@result}%
}
%    \end{macrocode}
%    \end{macro}
%    \begin{macro}{\PE@DeHex}
%    \begin{macrocode}
\def\PE@DeHex#1#2{%
  \ifx#2\relax
    \ifx#1\relax
      \let\PE@next\relax
    \else
      \uppercase{%
        \def\PE@testA{#1}%
      }%
      \ifcase\expandafter\PE@TestUcHexDigit\PE@testA
        \def\PE@next{%
          \PE@DeHex#10\relax\relax
        }%
      \else
        \let\PE@next\relax
      \fi
    \fi
  \else
    \uppercase{%
      \def\PE@testA{#1}%
      \def\PE@testB{#2}%
    }%
    \ifcase\expandafter\PE@TestUcHexDigit\PE@testA
      \ifcase\expandafter\PE@TestUcHexDigit\PE@testB
        \uccode\ltx@zero="\PE@testA\PE@testB\relax
        \ifnum\uccode\ltx@zero=32 %
          \let\PE@temp\PE@space@space
        \else
          \uppercase{%
            \def\PE@temp{^^@}%
          }%
        \fi
        \edef\PE@result{\PE@result\PE@temp}%
        \let\PE@next\PE@DeHex
      \else
        % invalid input sequence
        \def\PE@next{%
          \PE@DeHex#1%
        }%
      \fi
    \else
      % invalid input sequence
      \def\PE@next{\PE@DeHex#2}%
    \fi
  \fi
  \PE@next
}
%    \end{macrocode}
%    \end{macro}
%
% \subsubsection{Conversion to PDF name}
%
%    \begin{macro}{\PE@EscapeName}
%    \begin{macrocode}
\ifPE@etex
  \def\PE@EscapeName#1{%
    \edef#1{\expandafter\PE@EscapeNameTokens#1\relax}%
  }%
\else
  \def\PE@EscapeName#1{%
    \def\PE@result{}%
    \expandafter\PE@EscapeNameTokens#1\relax
    \let#1\PE@result
  }%
\fi
%    \end{macrocode}
%    \end{macro}
%    \begin{macro}{\PE@EscapeNameTokens}
%    \begin{macrocode}
\def\PE@EscapeNameTokens#1{%
  \ifx\relax#1%
  \else
    \ifnum`#1<33 %
      \ifcase`#1 %
        % drop illegal zero
      \else
        \PE@EscapeNameAdd\PE@hash
        \PE@HexChar#1%
      \fi
    \else
      \ifnum`#1>126 %
        \PE@EscapeNameAdd\PE@hash
        \PE@HexChar#1%
      \else  \ifnum`#1=35  \PE@EscapeNameHashChar 23% #
        \else\ifnum`#1=37  \PE@EscapeNameHashChar 25% %
        \else\ifnum`#1=40  \PE@EscapeNameHashChar 28% (
        \else\ifnum`#1=41  \PE@EscapeNameHashChar 29% )
        \else\ifnum`#1=47  \PE@EscapeNameHashChar 2F% /
        \else\ifnum`#1=60  \PE@EscapeNameHashChar 3C% <
        \else\ifnum`#1=62  \PE@EscapeNameHashChar 3E% >
        \else\ifnum`#1=91  \PE@EscapeNameHashChar 5B% [
        \else\ifnum`#1=93  \PE@EscapeNameHashChar 5D% ]
        \else\ifnum`#1=123 \PE@EscapeNameHashChar 7B% {
        \else\ifnum`#1=125 \PE@EscapeNameHashChar 7D% }
        \else
          \PE@EscapeNameAdd{#1}%
        \fi\fi\fi\fi\fi\fi\fi\fi\fi\fi\fi
      \fi
    \fi
    \expandafter\PE@EscapeNameTokens
  \fi
}%
\def\PE@EscapeNameHashChar#1#2{%
  \PE@EscapeNameAdd{\PE@hash\string#1\string#2}%
}%
%    \end{macrocode}
%    \end{macro}
%    \begin{macro}{\PE@EscapeNameAdd}
%    \begin{macrocode}
\ifPE@etex
  \def\PE@EscapeNameAdd#1{#1}%
\else
  \def\PE@EscapeNameAdd#1{%
    \edef\PE@result{%
      \PE@result
      #1%
    }%
  }%
\fi
%    \end{macrocode}
%    \end{macro}
%
% \subsubsection{Conversion to PDF string}
%
%    \begin{macro}{\PE@EscapeString}
%    \begin{macrocode}
\ifPE@etex
  \def\PE@EscapeString#1{%
    \edef#1{\expandafter\PE@EscapeStringTokens#1\relax}%
  }%
\else
  \def\PE@EscapeString#1{%
    \begingroup
      \def\PE@result{}%
      \expandafter\PE@EscapeStringTokens#1\relax
    \expandafter\endgroup
    \expandafter\def\expandafter#1\expandafter{\PE@result}%
  }%
\fi
%    \end{macrocode}
%    \end{macro}
%    \begin{macro}{\PE@EscapeStringTokens}
%    \begin{macrocode}
\def\PE@EscapeStringTokens#1{%
  \ifx\relax#1%
  \else
    \ifnum`#1<33 %
      \PE@OctChar#1%
    \else
      \ifnum`#1>126 %
        \PE@OctChar#1%
      \else  \ifnum`#1=40 \PE@EscapeStringAdd{\string\(}% (
        \else\ifnum`#1=41 \PE@EscapeStringAdd{\string\)}% )
        \else\ifnum`#1=92 \PE@EscapeStringAdd{\string\\}% \
        \else
          \PE@EscapeStringAdd{#1}%
        \fi\fi\fi
      \fi
    \fi
    \expandafter\PE@EscapeStringTokens
  \fi
}%
%    \end{macrocode}
%    \end{macro}
%    \begin{macro}{\PE@EscapeStringAdd}
%    \begin{macrocode}
\ifPE@etex
  \def\PE@EscapeStringAdd#1{#1}%
\else
  \def\PE@EscapeStringAdd#1{%
    \edef\PE@result{%
      \PE@result
      #1%
    }%
  }%
\fi
%    \end{macrocode}
%    \end{macro}
%
%    \begin{macrocode}
\PE@AtEnd%
%    \end{macrocode}
%
%    \begin{macrocode}
%</package>
%    \end{macrocode}
%
% \section{Test}
%
% \subsection{Catcode checks for loading}
%
%    \begin{macrocode}
%<*test1>
%    \end{macrocode}
%    \begin{macrocode}
\catcode`\{=1 %
\catcode`\}=2 %
\catcode`\#=6 %
\catcode`\@=11 %
\expandafter\ifx\csname count@\endcsname\relax
  \countdef\count@=255 %
\fi
\expandafter\ifx\csname @gobble\endcsname\relax
  \long\def\@gobble#1{}%
\fi
\expandafter\ifx\csname @firstofone\endcsname\relax
  \long\def\@firstofone#1{#1}%
\fi
\expandafter\ifx\csname loop\endcsname\relax
  \expandafter\@firstofone
\else
  \expandafter\@gobble
\fi
{%
  \def\loop#1\repeat{%
    \def\body{#1}%
    \iterate
  }%
  \def\iterate{%
    \body
      \let\next\iterate
    \else
      \let\next\relax
    \fi
    \next
  }%
  \let\repeat=\fi
}%
\def\RestoreCatcodes{}
\count@=0 %
\loop
  \edef\RestoreCatcodes{%
    \RestoreCatcodes
    \catcode\the\count@=\the\catcode\count@\relax
  }%
\ifnum\count@<255 %
  \advance\count@ 1 %
\repeat

\def\RangeCatcodeInvalid#1#2{%
  \count@=#1\relax
  \loop
    \catcode\count@=15 %
  \ifnum\count@<#2\relax
    \advance\count@ 1 %
  \repeat
}
\def\RangeCatcodeCheck#1#2#3{%
  \count@=#1\relax
  \loop
    \ifnum#3=\catcode\count@
    \else
      \errmessage{%
        Character \the\count@\space
        with wrong catcode \the\catcode\count@\space
        instead of \number#3%
      }%
    \fi
  \ifnum\count@<#2\relax
    \advance\count@ 1 %
  \repeat
}
\def\space{ }
\expandafter\ifx\csname LoadCommand\endcsname\relax
  \def\LoadCommand{\input pdfescape.sty\relax}%
\fi
\def\Test{%
  \RangeCatcodeInvalid{0}{47}%
  \RangeCatcodeInvalid{58}{64}%
  \RangeCatcodeInvalid{91}{96}%
  \RangeCatcodeInvalid{123}{255}%
  \catcode`\@=12 %
  \catcode`\\=0 %
  \catcode`\%=14 %
  \LoadCommand
  \RangeCatcodeCheck{0}{36}{15}%
  \RangeCatcodeCheck{37}{37}{14}%
  \RangeCatcodeCheck{38}{47}{15}%
  \RangeCatcodeCheck{48}{57}{12}%
  \RangeCatcodeCheck{58}{63}{15}%
  \RangeCatcodeCheck{64}{64}{12}%
  \RangeCatcodeCheck{65}{90}{11}%
  \RangeCatcodeCheck{91}{91}{15}%
  \RangeCatcodeCheck{92}{92}{0}%
  \RangeCatcodeCheck{93}{96}{15}%
  \RangeCatcodeCheck{97}{122}{11}%
  \RangeCatcodeCheck{123}{255}{15}%
  \RestoreCatcodes
}
\Test
\csname @@end\endcsname
\end
%    \end{macrocode}
%    \begin{macrocode}
%</test1>
%    \end{macrocode}
%
% \subsection{Macro tests}
%
%    \begin{macrocode}
%<*test2|test3|test4|test5>
\NeedsTeXFormat{LaTeX2e}
\makeatletter
%    \end{macrocode}
%
% \subsection{Test with \cs{pdfescape...} commands}
%
%    \begin{macrocode}
%<*test2>
\ProvidesFile{pdfescape-test2.tex}%
    [2011/11/25 v1.13 Test with \string\pdfescape... commands]%
%</test2>
%    \end{macrocode}
%
% \subsection{Test without \cs{pdfescape...}, with \eTeX}
%
%    \begin{macrocode}
%<*test3>
\ProvidesFile{pdfescape-test3.tex}%
    [2011/11/25 v1.13 Test without \string\pdfescape..., with e-TeX]%
%</test3>
%    \end{macrocode}
%
% \subsection{Test without \cs{pdfescape...} and \eTeX}
%
%    \begin{macrocode}
%<*test4>
\ProvidesFile{pdfescape-test4.tex}%
    [2011/11/25 v1.13 Test without \string\pdfescape... and e-TeX]%
%</test4>
%    \end{macrocode}
%
% \subsection{Test with \LuaTeX}
%
%    \begin{macrocode}
%<*test5>
\ProvidesFile{pdfescape-test5.tex}%
    [2011/11/25 v1.13 Test with LuaTeX]%
%</test5>
%    \end{macrocode}
%
% \subsection{Check/ensure test preconditions}
%
% \subsubsection{Check \cs{pdfescape...}}
%
%    \begin{macrocode}
%<*test2>
\@ifundefined{pdfescapehex}{%
  \PackageError{pdfescape-test2}{%
    Missing \string\pdfescape... commands%
  }{Test aborted.}%
  \stop
}{}
%</test2>
%    \end{macrocode}
%    \begin{macrocode}
%<*test3|test4>
\let\pdfescapehex\@undefined
\let\pdfunescapehex\@undefined
\let\pdfescapename\@undefined
\let\pdfescapestring\@undefined
%</test3|test4>
%    \end{macrocode}
%
% \subsubsection{Check \eTeX}
%
%    \begin{macrocode}
%<*test3>
\@ifundefined{numexpr}{%
  \PackageError{pdfescape-test3}{%
    Missing \eTeX
  }{Test aborted.}%
  \stop
}{}
%</test3>
%    \end{macrocode}
% Package \xpackage{qstest} uses \eTeX, thus \eTeX's features
% can only be disabled later during loading of package \xpackage{pdfescape}.
%
% \subsubsection{Check \LuaTeX}
%
%    \begin{macrocode}
%<*test5>
\@ifundefined{directlua}{%
  \PackageError{pdfescape-test5}{%
    Missing LuaTeX%
  }{Test aborted.}%
  \stop
}{}
%</test5>
%    \end{macrocode}
%
% \subsection{Common part}
%
%    The files for testing uses the framework, provided by the
%    new package \xpackage{qstest} of David Kastrup.
%    \begin{macrocode}
\RequirePackage{qstest}
\IncludeTests{*}
\LogTests{log}{*}{*}

\newcommand*{\ExpectVar}[2]{%
  \ifx#1#2%
  \else
    \begingroup
      \@onelevel@sanitize#1%
      \@onelevel@sanitize#2%
      \typeout{[#1] <> [#2]}% hash-ok
    \endgroup
  \fi
  \Expect*{\ifx#1#2true\else false\fi}{true}%
}

\makeatletter
\begingroup
  \gdef\AllBytes{}%
  \count@=0 %
  \catcode0=12 %
  \@whilenum\count@<256 \do{%
    \lccode0=\count@
    \ifnum\count@=32 %
      \xdef\AllBytes{\AllBytes\space}%
    \else
      \lowercase{%
        \xdef\AllBytes{\AllBytes^^@}%
      }%
    \fi
    \advance\count@ by 1 %
  }%
\endgroup
\newcommand*{\AllBytesHex}{%
  000102030405060708090A0B0C0D0E0F%
  101112131415161718191A1B1C1D1E1F%
  202122232425262728292A2B2C2D2E2F%
  303132333435363738393A3B3C3D3E3F%
  404142434445464748494A4B4C4D4E4F%
  505152535455565758595A5B5C5D5E5F%
  606162636465666768696A6B6C6D6E6F%
  707172737475767778797A7B7C7D7E7F%
  808182838485868788898A8B8C8D8E8F%
  909192939495969798999A9B9C9D9E9F%
  A0A1A2A3A4A5A6A7A8A9AAABACADAEAF%
  B0B1B2B3B4B5B6B7B8B9BABBBCBDBEBF%
  C0C1C2C3C4C5C6C7C8C9CACBCCCDCECF%
  D0D1D2D3D4D5D6D7D8D9DADBDCDDDEDF%
  E0E1E2E3E4E5E6E7E8E9EAEBECEDEEEF%
  F0F1F2F3F4F5F6F7F8F9FAFBFCFDFEFF%
}
\@onelevel@sanitize\AllBytesHex
\expandafter\lowercase\expandafter{%
  \expandafter\newcommand\expandafter*\expandafter\AllBytesHexLC
      \expandafter{\AllBytesHex}%
}
\newcommand*{\AllBytesName}{}
\begingroup
  \catcode`\#=12 %
  \xdef\AllBytesName{%
    #01#02#03#04#05#06#07#08#09#0A#0B#0C#0D#0E#0F%
    #10#11#12#13#14#15#16#17#18#19#1A#1B#1C#1D#1E#1F%
    #20!"#23$#25&'#28#29*+,-.#2F%
    0123456789:;#3C=#3E?%
    @ABCDEFGHIJKLMNO%
    PQRSTUVWXYZ#5B\@backslashchar#5D^_%
    `abcdefghijklmno%
    pqrstuvwxyz#7B|#7D\string~#7F%
    #80#81#82#83#84#85#86#87#88#89#8A#8B#8C#8D#8E#8F%
    #90#91#92#93#94#95#96#97#98#99#9A#9B#9C#9D#9E#9F%
    #A0#A1#A2#A3#A4#A5#A6#A7#A8#A9#AA#AB#AC#AD#AE#AF%
    #B0#B1#B2#B3#B4#B5#B6#B7#B8#B9#BA#BB#BC#BD#BE#BF%
    #C0#C1#C2#C3#C4#C5#C6#C7#C8#C9#CA#CB#CC#CD#CE#CF%
    #D0#D1#D2#D3#D4#D5#D6#D7#D8#D9#DA#DB#DC#DD#DE#DF%
    #E0#E1#E2#E3#E4#E5#E6#E7#E8#E9#EA#EB#EC#ED#EE#EF%
    #F0#F1#F2#F3#F4#F5#F6#F7#F8#F9#FA#FB#FC#FD#FE#FF%
  }%
\endgroup
\@onelevel@sanitize\AllBytesName

\newcommand*{\AllBytesString}{}
\begingroup
  \def\|{|}%
  \edef\%{\@percentchar}%
  \catcode`\|=0 %
  \catcode`\#=12 %
  \catcode`\~=12 %
  \catcode`\\=12 %
  |xdef|AllBytesString{%
    \000\001\002\003\004\005\006\007\010\011\012\013\014\015\016\017%
    \020\021\022\023\024\025\026\027\030\031\032\033\034\035\036\037%
    \040!"#$|%&'\(\)*+,-./%
    0123456789:;<=>?%
    @ABCDEFGHIJKLMNO%
    PQRSTUVWXYZ[\\]^_%
    `abcdefghijklmno%
    pqrstuvwxyz{||}~\177%
    \200\201\202\203\204\205\206\207\210\211\212\213\214\215\216\217%
    \220\221\222\223\224\225\226\227\230\231\232\233\234\235\236\237%
    \240\241\242\243\244\245\246\247\250\251\252\253\254\255\256\257%
    \260\261\262\263\264\265\266\267\270\271\272\273\274\275\276\277%
    \300\301\302\303\304\305\306\307\310\311\312\313\314\315\316\317%
    \320\321\322\323\324\325\326\327\330\331\332\333\334\335\336\337%
    \340\341\342\343\344\345\346\347\350\351\352\353\354\355\356\357%
    \360\361\362\363\364\365\366\367\370\371\372\373\374\375\376\377%
  }%
|endgroup
\@onelevel@sanitize\AllBytesString

%<*test4>
\let\org@detokenize\detokenize
\let\detokenize\@undefined
\let\org@numexpr\numexpr
\let\numexpr\@undefined
%</test4>
\RequirePackage{pdfescape}
%<*test4>
\let\detokenize\org@detokenize
\let\numexpr\org@numexpr
%</test4>

\begin{qstest}{all-hex}{\AllBytes, escapehex}
  \EdefEscapeHex\x{\AllBytes}%
  \Expect*{\x}*{\AllBytesHex}%
  \ExpectVar\x\AllBytesHex
\end{qstest}

\begin{qstest}{all-unhex}{\AllBytesHex, unescapehex}
  \EdefUnescapeHex\x{\AllBytesHex}%
  \Expect*{\x}*{\AllBytes}%
  \ExpectVar\x\AllBytes
\end{qstest}

\begin{qstest}{all-unhex-lc}{\AllBytesHexLC, unescapehex, lowercase}
  \EdefUnescapeHex\x{\AllBytesHexLC}%
  \Expect*{\x}*{\AllBytes}%
  \ExpectVar\x\AllBytes
\end{qstest}

\begin{qstest}{unhex-incomplete}{unescapehex, incomplete}
  \EdefUnescapeHex\x{4}%
  \Expect*{\x}{@}%
\end{qstest}

\begin{qstest}{unhex-space}{unescapehex, space}
  \EdefUnescapeHex\x{20}%
  \Expect*{\x}{ }%
  \ExpectVar\x\space
\end{qstest}

\begin{qstest}{unhex-spaces}{unescapehex, spaces}
  \EdefUnescapeHex\x{204020204120}%
  \def\y#1{%
    \edef\z{#1\string @#1#1\string A#1}%
  }\y{ }%
  \Expect*{\x}*{\z}%
  \ExpectVar\x\z
\end{qstest}

\begin{qstest}{unhex-hash}{unescapehex, hash}
  \catcode`\#=12 %
  \EdefUnescapeHex\x{#20}%
  \ExpectVar\x\space
\end{qstest}

\begin{qstest}{unhex-invalid}{unescapehex, invalid}
  \def\test#1#2{%
    \EdefUnescapeHex\x{#1}%
    \edef\y{#2}%
    \@onelevel@sanitize\y
    \ExpectVar\x\y
  }%
%<*test2>
  \edef\x{\pdfunescapehex{4X}}%
  \edef\y{\string @}%
  \ifx\x\y
  \else
    \def~{\space}%
    \typeout{************************************}%
    \typeout{* Your pdfTeX contains bug 777.~~~~*}%
    \typeout{* This test is redefined as dummy, *}%
    \typeout{* because it triggers the bug.~~~~~*}%
    \typeout{************************************}%
    \def\test#1#2{}%
  \fi
%</test2>
  \test{X}{}%
  \test{XY}{}%
  \test{XYZ}{}%
  \test{A}{^^a0}%
  \test{AX}{^^a0}%
  \test{XA}{^^a0}%
  \test{XXAXX}{^^a0}%
\end{qstest}

\begin{qstest}{all-name}{\AllBytes, escapename}
  \EdefEscapeName\x{\AllBytes}%
  \Expect*{\x}*{\AllBytesName}%
  \ExpectVar\x\AllBytesName
\end{qstest}

\begin{qstest}{all-string}{\AllBytes, escapestring}
  \EdefEscapeString\x{\AllBytes}%
  \Expect*{\x}*{\AllBytesString}%
  \ExpectVar\x\AllBytesString
\end{qstest}

\begin{qstest}{uchexdigit}{unescape, uppercase hex digit}
  \catcode`\@=11 %
  \catcode0=12 %
  \def\test#1#2{%
    \uccode0=#1\relax
    \uppercase{%
      \def\x{^^@}%
    }%
    \Expect*{%
       \ifcase\expandafter\PE@TestUcHexDigit\x
         true%
       \else
         false%
       \fi
    }{#2}%
  }%
  \def\range#1#2#3{%
    \count0=#1\relax
    \loop
    \ifnum\count0<#2\relax
      \test{\count0}{#3}%
      \advance\count0 by 1 %
    \repeat
  }%
  \range{0}{47}{false}%
  \range{48}{57}{true}%
  \range{58}{64}{false}%
  \range{65}{70}{true}%
  \range{71}{255}{false}%
\end{qstest}

\begin{qstest}{unescapename}{unescapename}
  \def\test#1#2{%
    \EdefUnescapeName\x{#1}%
    \edef\y{#2}%
    \@onelevel@sanitize\y
    \ExpectVar\x\y
  }%
  \catcode`\#=12 %
  \catcode0=12 %
  \test{}{}%
  \test{x}{x}%
  \test{xy}{xy}%
  \test{#}{#}%
  \test{##}{##}%
  \test{###}{###}%
  \test{####}{####}%
  \test{#x}{#x}%
  \test{#xy}{#xy}%
  \test{#1}{#1}%
  \test{#40}{@}%
  \test{#400}{@0}%
  \test{#4x0}{#4x0}%
  \test{#ab}{^^ab}%
  \test{#00}{^^@}%
  \test{x#40y#40z}{x@y@z}%
  \test{#40#40#40#40}{@@@@}%
  \test{a#x}{a#x}%
  \test{a#xy}{a#xy}%
  \test{a#1}{a#1}%
  \test{a#40}{a@}%
  \test{a#400}{a@0}%
  \test{#20}{ }%
  \test{a#20}{a }%
  \test{a#20b}{a b}%
  \test{a#20#20#20b}{a \space\space b}%
\end{qstest}

\begin{qstest}{unescapestring}{unescapestring}
  \def\test#1#2{%
    \EdefUnescapeString\x{#1}%
    \edef\y{#2}%
    \@onelevel@sanitize\y
    \ExpectVar\x\y
  }%
  \catcode0=12 %
  \def\DefChar#1#2{%
    \begingroup
      \uccode0=#2\relax
    \uppercase{\endgroup
      \def#1{^^@}%
    }%
  }%
  \DefChar\nul{0}%
  \DefChar\one{1}%
  \DefChar\bel{8}%
  \DefChar\tab{9}%
  \DefChar\lf{10}%
  \DefChar\ff{12}%
  \DefChar\cr{13}%
  \DefChar\\{92}%
  \test{}{}%
  \test{a}{a}%
  \test{\\}{}%
  \test{\\\\}{\\}%
  \test{\\\\y}{\\y}%
  \test{\\000}{\nul}%
  \test{\\b}{\bel}%
  \test{\\t}{\tab}%
  \test{\\n}{\lf}%
  \test{\\f}{\ff}%
  \test{\\r}{\cr}%
  \test{\\(}{(}%
  \test{\\)}{)}%
  \test{\\040}{ }%
  \test{\\100}{@}%
  \test{\\40}{ }%
  \test{\\1}{\one}%
  \test{\\01}{\one}%
  \test{\\001}{\one}%
  \test{\\18}{\one8}%
  \test{\\018}{\one8}%
  \test{\\0018}{\one8}%
  \test{x\\}{x}%
  \test{x\\\\}{x\\}%
  \test{x\\\\y}{x\\y}%
  \test{x\\000}{x\nul}%
  \test{x\\b}{x\bel}%
  \test{x\\t}{x\tab}%
  \test{x\\n}{x\lf}%
  \test{x\\f}{x\ff}%
  \test{x\\r}{x\cr}%
  \test{x\\(}{x(}%
  \test{x\\)}{x)}%
  \test{x\\040}{x }%
  \test{x\\100}{x@}%
  \test{x\\40}{x }%
  \test{x\\1}{x\one}%
  \test{x\\01}{x\one}%
  \test{x\\001}{x\one}%
  \test{x\\18}{x\one8}%
  \test{x\\018}{x\one8}%
  \test{x\\0018}{x\one8}%
  \test{\\b\\t\\n\\f\\r\\(\\)\\\\\\000\\040}{%
    \bel\tab\lf\ff\cr()\\\nul\space
  }%
  \test{\\\lf}{}%
  \test{x\\\lf}{x}%
  \test{\cr}{\lf}%
  \test{\cr\lf}{\lf}%
  \test{\lf}{\lf}%
  \test{\lf\cr}{\lf\lf}%
  \test{x\cr}{x\lf}%
  \test{x\cr\lf}{x\lf}%
  \test{x\lf}{x\lf}%
  \test{x\lf\cr}{x\lf\lf}%
  \test{x\\\cr\lf y\cr}{xy\lf}%
%
  \test{\\409}{ 9}%
  \test{\\800}{800}%
  \test{\\900}{900}%
  \test{\\578}{/8}%
  \test{\\477}{?}%
  \test{\\377}{^^ff}%
  \test{\\777}{^^ff}%
  \test{\\7777}{^^ff7}%
\end{qstest}
\stop
%    \end{macrocode}
%
%    \begin{macrocode}
%</test2|test3|test4|test5>
%    \end{macrocode}
%
% \subsubsection{Test for \hologo{iniTeX}}
%
%    \begin{macrocode}
%<*test6>
\input pdfescape.sty\relax
\catcode`\{=1 %
\catcode`\}=2 %
\catcode`\#=6 %
\catcode`\^=7 %
\catcode`\@=11 %
%    \end{macrocode}
%    \begin{macrocode}
\begingroup
  \catcode`\@=11 %
  \countdef\count@=255 %
  \def\space{ }%
  \long\def\@whilenum#1\do #2{%
    \ifnum #1\relax
      #2\relax
      \@iwhilenum{#1\relax#2\relax}%
    \fi
  }%
  \long\def\@iwhilenum#1{%
    \ifnum #1%
      \expandafter\@iwhilenum
    \else
      \expandafter\ltx@gobble
    \fi
    {#1}%
  }%
  \gdef\AllBytes{}%
  \count@=0 %
  \catcode0=12 %
  \@whilenum\count@<256 \do{%
    \lccode0=\count@
    \ifnum\count@=32 %
      \xdef\AllBytes{\AllBytes\space}%
    \else
      \lowercase{%
        \xdef\AllBytes{\AllBytes^^@}%
      }%
    \fi
    \advance\count@ by 1 %
  }%
\endgroup
\def\AllBytesHex{%
  000102030405060708090A0B0C0D0E0F%
  101112131415161718191A1B1C1D1E1F%
  202122232425262728292A2B2C2D2E2F%
  303132333435363738393A3B3C3D3E3F%
  404142434445464748494A4B4C4D4E4F%
  505152535455565758595A5B5C5D5E5F%
  606162636465666768696A6B6C6D6E6F%
  707172737475767778797A7B7C7D7E7F%
  808182838485868788898A8B8C8D8E8F%
  909192939495969798999A9B9C9D9E9F%
  A0A1A2A3A4A5A6A7A8A9AAABACADAEAF%
  B0B1B2B3B4B5B6B7B8B9BABBBCBDBEBF%
  C0C1C2C3C4C5C6C7C8C9CACBCCCDCECF%
  D0D1D2D3D4D5D6D7D8D9DADBDCDDDEDF%
  E0E1E2E3E4E5E6E7E8E9EAEBECEDEEEF%
  F0F1F2F3F4F5F6F7F8F9FAFBFCFDFEFF%
}
\ltx@onelevel@sanitize\AllBytesHex
\expandafter\lowercase\expandafter{%
  \expandafter\def\expandafter\AllBytesHexLC
      \expandafter{\AllBytesHex}%
}
\begingroup
  \catcode`\#=12 %
  \xdef\AllBytesName{%
    #01#02#03#04#05#06#07#08#09#0A#0B#0C#0D#0E#0F%
    #10#11#12#13#14#15#16#17#18#19#1A#1B#1C#1D#1E#1F%
    #20!"#23$#25&'#28#29*+,-.#2F%
    0123456789:;#3C=#3E?%
    @ABCDEFGHIJKLMNO%
    PQRSTUVWXYZ#5B\ltx@backslashchar#5D^_%
    `abcdefghijklmno%
    pqrstuvwxyz#7B|#7D\string~#7F%
    #80#81#82#83#84#85#86#87#88#89#8A#8B#8C#8D#8E#8F%
    #90#91#92#93#94#95#96#97#98#99#9A#9B#9C#9D#9E#9F%
    #A0#A1#A2#A3#A4#A5#A6#A7#A8#A9#AA#AB#AC#AD#AE#AF%
    #B0#B1#B2#B3#B4#B5#B6#B7#B8#B9#BA#BB#BC#BD#BE#BF%
    #C0#C1#C2#C3#C4#C5#C6#C7#C8#C9#CA#CB#CC#CD#CE#CF%
    #D0#D1#D2#D3#D4#D5#D6#D7#D8#D9#DA#DB#DC#DD#DE#DF%
    #E0#E1#E2#E3#E4#E5#E6#E7#E8#E9#EA#EB#EC#ED#EE#EF%
    #F0#F1#F2#F3#F4#F5#F6#F7#F8#F9#FA#FB#FC#FD#FE#FF%
  }%
\endgroup
\ltx@onelevel@sanitize\AllBytesName
\edef\AllBytesFromName{\expandafter\ltx@gobble\AllBytes}

\begingroup
  \def\|{|}%
  \edef\%{\ltx@percentchar}%
  \catcode`\|=0 %
  \catcode`\#=12 %
  \catcode`\~=12 %
  \catcode`\\=12 %
  |xdef|AllBytesString{%
    \000\001\002\003\004\005\006\007\010\011\012\013\014\015\016\017%
    \020\021\022\023\024\025\026\027\030\031\032\033\034\035\036\037%
    \040!"#$|%&'\(\)*+,-./%
    0123456789:;<=>?%
    @ABCDEFGHIJKLMNO%
    PQRSTUVWXYZ[\\]^_%
    `abcdefghijklmno%
    pqrstuvwxyz{||}~\177%
    \200\201\202\203\204\205\206\207\210\211\212\213\214\215\216\217%
    \220\221\222\223\224\225\226\227\230\231\232\233\234\235\236\237%
    \240\241\242\243\244\245\246\247\250\251\252\253\254\255\256\257%
    \260\261\262\263\264\265\266\267\270\271\272\273\274\275\276\277%
    \300\301\302\303\304\305\306\307\310\311\312\313\314\315\316\317%
    \320\321\322\323\324\325\326\327\330\331\332\333\334\335\336\337%
    \340\341\342\343\344\345\346\347\350\351\352\353\354\355\356\357%
    \360\361\362\363\364\365\366\367\370\371\372\373\374\375\376\377%
  }%
|endgroup
\ltx@onelevel@sanitize\AllBytesString
%    \end{macrocode}
%    \begin{macrocode}
\def\msg#{\immediate\write16}
\def\Test#1#2#3{%
  \begingroup
    #1\TestResult{#2}%
    \ifx\TestResult#3%
    \else
      \newlinechar=10 %
      \msg{Expect:^^J#3}%
      \msg{Result:^^J\TestResult}%
      \errmessage{\string#2 -\string#1-> \string#3}%
    \fi
  \endgroup
}
\Test\EdefEscapeHex\AllBytes\AllBytesHex
\Test\EdefUnescapeHex\AllBytesHex\AllBytes
\Test\EdefEscapeName\AllBytes\AllBytesName
\Test\EdefUnescapeName\AllBytesName\AllBytesFromName
\Test\EdefEscapeString\AllBytes\AllBytesString
\Test\EdefUnescapeString\AllBytesString\AllBytes
%    \end{macrocode}
%    \begin{macrocode}
\csname @@end\endcsname\end
%</test6>
%    \end{macrocode}
%
% \section{Installation}
%
% \subsection{Download}
%
% \paragraph{Package.} This package is available on
% CTAN\footnote{\url{ftp://ftp.ctan.org/tex-archive/}}:
% \begin{description}
% \item[\CTAN{macros/latex/contrib/oberdiek/pdfescape.dtx}] The source file.
% \item[\CTAN{macros/latex/contrib/oberdiek/pdfescape.pdf}] Documentation.
% \end{description}
%
%
% \paragraph{Bundle.} All the packages of the bundle `oberdiek'
% are also available in a TDS compliant ZIP archive. There
% the packages are already unpacked and the documentation files
% are generated. The files and directories obey the TDS standard.
% \begin{description}
% \item[\CTAN{install/macros/latex/contrib/oberdiek.tds.zip}]
% \end{description}
% \emph{TDS} refers to the standard ``A Directory Structure
% for \TeX\ Files'' (\CTAN{tds/tds.pdf}). Directories
% with \xfile{texmf} in their name are usually organized this way.
%
% \subsection{Bundle installation}
%
% \paragraph{Unpacking.} Unpack the \xfile{oberdiek.tds.zip} in the
% TDS tree (also known as \xfile{texmf} tree) of your choice.
% Example (linux):
% \begin{quote}
%   |unzip oberdiek.tds.zip -d ~/texmf|
% \end{quote}
%
% \paragraph{Script installation.}
% Check the directory \xfile{TDS:scripts/oberdiek/} for
% scripts that need further installation steps.
% Package \xpackage{attachfile2} comes with the Perl script
% \xfile{pdfatfi.pl} that should be installed in such a way
% that it can be called as \texttt{pdfatfi}.
% Example (linux):
% \begin{quote}
%   |chmod +x scripts/oberdiek/pdfatfi.pl|\\
%   |cp scripts/oberdiek/pdfatfi.pl /usr/local/bin/|
% \end{quote}
%
% \subsection{Package installation}
%
% \paragraph{Unpacking.} The \xfile{.dtx} file is a self-extracting
% \docstrip\ archive. The files are extracted by running the
% \xfile{.dtx} through \plainTeX:
% \begin{quote}
%   \verb|tex pdfescape.dtx|
% \end{quote}
%
% \paragraph{TDS.} Now the different files must be moved into
% the different directories in your installation TDS tree
% (also known as \xfile{texmf} tree):
% \begin{quote}
% \def\t{^^A
% \begin{tabular}{@{}>{\ttfamily}l@{ $\rightarrow$ }>{\ttfamily}l@{}}
%   pdfescape.sty & tex/generic/oberdiek/pdfescape.sty\\
%   pdfescape.pdf & doc/latex/oberdiek/pdfescape.pdf\\
%   test/pdfescape-test1.tex & doc/latex/oberdiek/test/pdfescape-test1.tex\\
%   test/pdfescape-test2.tex & doc/latex/oberdiek/test/pdfescape-test2.tex\\
%   test/pdfescape-test3.tex & doc/latex/oberdiek/test/pdfescape-test3.tex\\
%   test/pdfescape-test4.tex & doc/latex/oberdiek/test/pdfescape-test4.tex\\
%   test/pdfescape-test5.tex & doc/latex/oberdiek/test/pdfescape-test5.tex\\
%   test/pdfescape-test6.tex & doc/latex/oberdiek/test/pdfescape-test6.tex\\
%   pdfescape.dtx & source/latex/oberdiek/pdfescape.dtx\\
% \end{tabular}^^A
% }^^A
% \sbox0{\t}^^A
% \ifdim\wd0>\linewidth
%   \begingroup
%     \advance\linewidth by\leftmargin
%     \advance\linewidth by\rightmargin
%   \edef\x{\endgroup
%     \def\noexpand\lw{\the\linewidth}^^A
%   }\x
%   \def\lwbox{^^A
%     \leavevmode
%     \hbox to \linewidth{^^A
%       \kern-\leftmargin\relax
%       \hss
%       \usebox0
%       \hss
%       \kern-\rightmargin\relax
%     }^^A
%   }^^A
%   \ifdim\wd0>\lw
%     \sbox0{\small\t}^^A
%     \ifdim\wd0>\linewidth
%       \ifdim\wd0>\lw
%         \sbox0{\footnotesize\t}^^A
%         \ifdim\wd0>\linewidth
%           \ifdim\wd0>\lw
%             \sbox0{\scriptsize\t}^^A
%             \ifdim\wd0>\linewidth
%               \ifdim\wd0>\lw
%                 \sbox0{\tiny\t}^^A
%                 \ifdim\wd0>\linewidth
%                   \lwbox
%                 \else
%                   \usebox0
%                 \fi
%               \else
%                 \lwbox
%               \fi
%             \else
%               \usebox0
%             \fi
%           \else
%             \lwbox
%           \fi
%         \else
%           \usebox0
%         \fi
%       \else
%         \lwbox
%       \fi
%     \else
%       \usebox0
%     \fi
%   \else
%     \lwbox
%   \fi
% \else
%   \usebox0
% \fi
% \end{quote}
% If you have a \xfile{docstrip.cfg} that configures and enables \docstrip's
% TDS installing feature, then some files can already be in the right
% place, see the documentation of \docstrip.
%
% \subsection{Refresh file name databases}
%
% If your \TeX~distribution
% (\teTeX, \mikTeX, \dots) relies on file name databases, you must refresh
% these. For example, \teTeX\ users run \verb|texhash| or
% \verb|mktexlsr|.
%
% \subsection{Some details for the interested}
%
% \paragraph{Attached source.}
%
% The PDF documentation on CTAN also includes the
% \xfile{.dtx} source file. It can be extracted by
% AcrobatReader 6 or higher. Another option is \textsf{pdftk},
% e.g. unpack the file into the current directory:
% \begin{quote}
%   \verb|pdftk pdfescape.pdf unpack_files output .|
% \end{quote}
%
% \paragraph{Unpacking with \LaTeX.}
% The \xfile{.dtx} chooses its action depending on the format:
% \begin{description}
% \item[\plainTeX:] Run \docstrip\ and extract the files.
% \item[\LaTeX:] Generate the documentation.
% \end{description}
% If you insist on using \LaTeX\ for \docstrip\ (really,
% \docstrip\ does not need \LaTeX), then inform the autodetect routine
% about your intention:
% \begin{quote}
%   \verb|latex \let\install=y% \iffalse meta-comment
%
% File: pdfescape.dtx
% Version: 2011/11/25 v1.13
% Info: Implements pdfTeX's escape features
%
% Copyright (C) 2007, 2010, 2011 by
%    Heiko Oberdiek <heiko.oberdiek at googlemail.com>
%
% This work may be distributed and/or modified under the
% conditions of the LaTeX Project Public License, either
% version 1.3c of this license or (at your option) any later
% version. This version of this license is in
%    http://www.latex-project.org/lppl/lppl-1-3c.txt
% and the latest version of this license is in
%    http://www.latex-project.org/lppl.txt
% and version 1.3 or later is part of all distributions of
% LaTeX version 2005/12/01 or later.
%
% This work has the LPPL maintenance status "maintained".
%
% This Current Maintainer of this work is Heiko Oberdiek.
%
% The Base Interpreter refers to any `TeX-Format',
% because some files are installed in TDS:tex/generic//.
%
% This work consists of the main source file pdfescape.dtx
% and the derived files
%    pdfescape.sty, pdfescape.pdf, pdfescape.ins, pdfescape.drv,
%    pdfescape-test1.tex, pdfescape-test2.tex,
%    pdfescape-test3.tex, pdfescape-test4.tex,
%    pdfescape-test5.tex, pdfescape-test6.tex.
%
% Distribution:
%    CTAN:macros/latex/contrib/oberdiek/pdfescape.dtx
%    CTAN:macros/latex/contrib/oberdiek/pdfescape.pdf
%
% Unpacking:
%    (a) If pdfescape.ins is present:
%           tex pdfescape.ins
%    (b) Without pdfescape.ins:
%           tex pdfescape.dtx
%    (c) If you insist on using LaTeX
%           latex \let\install=y% \iffalse meta-comment
%
% File: pdfescape.dtx
% Version: 2011/11/25 v1.13
% Info: Implements pdfTeX's escape features
%
% Copyright (C) 2007, 2010, 2011 by
%    Heiko Oberdiek <heiko.oberdiek at googlemail.com>
%
% This work may be distributed and/or modified under the
% conditions of the LaTeX Project Public License, either
% version 1.3c of this license or (at your option) any later
% version. This version of this license is in
%    http://www.latex-project.org/lppl/lppl-1-3c.txt
% and the latest version of this license is in
%    http://www.latex-project.org/lppl.txt
% and version 1.3 or later is part of all distributions of
% LaTeX version 2005/12/01 or later.
%
% This work has the LPPL maintenance status "maintained".
%
% This Current Maintainer of this work is Heiko Oberdiek.
%
% The Base Interpreter refers to any `TeX-Format',
% because some files are installed in TDS:tex/generic//.
%
% This work consists of the main source file pdfescape.dtx
% and the derived files
%    pdfescape.sty, pdfescape.pdf, pdfescape.ins, pdfescape.drv,
%    pdfescape-test1.tex, pdfescape-test2.tex,
%    pdfescape-test3.tex, pdfescape-test4.tex,
%    pdfescape-test5.tex, pdfescape-test6.tex.
%
% Distribution:
%    CTAN:macros/latex/contrib/oberdiek/pdfescape.dtx
%    CTAN:macros/latex/contrib/oberdiek/pdfescape.pdf
%
% Unpacking:
%    (a) If pdfescape.ins is present:
%           tex pdfescape.ins
%    (b) Without pdfescape.ins:
%           tex pdfescape.dtx
%    (c) If you insist on using LaTeX
%           latex \let\install=y% \iffalse meta-comment
%
% File: pdfescape.dtx
% Version: 2011/11/25 v1.13
% Info: Implements pdfTeX's escape features
%
% Copyright (C) 2007, 2010, 2011 by
%    Heiko Oberdiek <heiko.oberdiek at googlemail.com>
%
% This work may be distributed and/or modified under the
% conditions of the LaTeX Project Public License, either
% version 1.3c of this license or (at your option) any later
% version. This version of this license is in
%    http://www.latex-project.org/lppl/lppl-1-3c.txt
% and the latest version of this license is in
%    http://www.latex-project.org/lppl.txt
% and version 1.3 or later is part of all distributions of
% LaTeX version 2005/12/01 or later.
%
% This work has the LPPL maintenance status "maintained".
%
% This Current Maintainer of this work is Heiko Oberdiek.
%
% The Base Interpreter refers to any `TeX-Format',
% because some files are installed in TDS:tex/generic//.
%
% This work consists of the main source file pdfescape.dtx
% and the derived files
%    pdfescape.sty, pdfescape.pdf, pdfescape.ins, pdfescape.drv,
%    pdfescape-test1.tex, pdfescape-test2.tex,
%    pdfescape-test3.tex, pdfescape-test4.tex,
%    pdfescape-test5.tex, pdfescape-test6.tex.
%
% Distribution:
%    CTAN:macros/latex/contrib/oberdiek/pdfescape.dtx
%    CTAN:macros/latex/contrib/oberdiek/pdfescape.pdf
%
% Unpacking:
%    (a) If pdfescape.ins is present:
%           tex pdfescape.ins
%    (b) Without pdfescape.ins:
%           tex pdfescape.dtx
%    (c) If you insist on using LaTeX
%           latex \let\install=y\input{pdfescape.dtx}
%        (quote the arguments according to the demands of your shell)
%
% Documentation:
%    (a) If pdfescape.drv is present:
%           latex pdfescape.drv
%    (b) Without pdfescape.drv:
%           latex pdfescape.dtx; ...
%    The class ltxdoc loads the configuration file ltxdoc.cfg
%    if available. Here you can specify further options, e.g.
%    use A4 as paper format:
%       \PassOptionsToClass{a4paper}{article}
%
%    Programm calls to get the documentation (example):
%       pdflatex pdfescape.dtx
%       makeindex -s gind.ist pdfescape.idx
%       pdflatex pdfescape.dtx
%       makeindex -s gind.ist pdfescape.idx
%       pdflatex pdfescape.dtx
%
% Installation:
%    TDS:tex/generic/oberdiek/pdfescape.sty
%    TDS:doc/latex/oberdiek/pdfescape.pdf
%    TDS:doc/latex/oberdiek/test/pdfescape-test1.tex
%    TDS:doc/latex/oberdiek/test/pdfescape-test2.tex
%    TDS:doc/latex/oberdiek/test/pdfescape-test3.tex
%    TDS:doc/latex/oberdiek/test/pdfescape-test4.tex
%    TDS:doc/latex/oberdiek/test/pdfescape-test5.tex
%    TDS:doc/latex/oberdiek/test/pdfescape-test6.tex
%    TDS:source/latex/oberdiek/pdfescape.dtx
%
%<*ignore>
\begingroup
  \catcode123=1 %
  \catcode125=2 %
  \def\x{LaTeX2e}%
\expandafter\endgroup
\ifcase 0\ifx\install y1\fi\expandafter
         \ifx\csname processbatchFile\endcsname\relax\else1\fi
         \ifx\fmtname\x\else 1\fi\relax
\else\csname fi\endcsname
%</ignore>
%<*install>
\input docstrip.tex
\Msg{************************************************************************}
\Msg{* Installation}
\Msg{* Package: pdfescape 2011/11/25 v1.13 Implements pdfTeX's escape features (HO)}
\Msg{************************************************************************}

\keepsilent
\askforoverwritefalse

\let\MetaPrefix\relax
\preamble

This is a generated file.

Project: pdfescape
Version: 2011/11/25 v1.13

Copyright (C) 2007, 2010, 2011 by
   Heiko Oberdiek <heiko.oberdiek at googlemail.com>

This work may be distributed and/or modified under the
conditions of the LaTeX Project Public License, either
version 1.3c of this license or (at your option) any later
version. This version of this license is in
   http://www.latex-project.org/lppl/lppl-1-3c.txt
and the latest version of this license is in
   http://www.latex-project.org/lppl.txt
and version 1.3 or later is part of all distributions of
LaTeX version 2005/12/01 or later.

This work has the LPPL maintenance status "maintained".

This Current Maintainer of this work is Heiko Oberdiek.

The Base Interpreter refers to any `TeX-Format',
because some files are installed in TDS:tex/generic//.

This work consists of the main source file pdfescape.dtx
and the derived files
   pdfescape.sty, pdfescape.pdf, pdfescape.ins, pdfescape.drv,
   pdfescape-test1.tex, pdfescape-test2.tex,
   pdfescape-test3.tex, pdfescape-test4.tex,
   pdfescape-test5.tex, pdfescape-test6.tex.

\endpreamble
\let\MetaPrefix\DoubleperCent

\generate{%
  \file{pdfescape.ins}{\from{pdfescape.dtx}{install}}%
  \file{pdfescape.drv}{\from{pdfescape.dtx}{driver}}%
  \usedir{tex/generic/oberdiek}%
  \file{pdfescape.sty}{\from{pdfescape.dtx}{package}}%
  \usedir{doc/latex/oberdiek/test}%
  \file{pdfescape-test1.tex}{\from{pdfescape.dtx}{test1}}%
  \file{pdfescape-test2.tex}{\from{pdfescape.dtx}{test2}}%
  \file{pdfescape-test3.tex}{\from{pdfescape.dtx}{test3}}%
  \file{pdfescape-test4.tex}{\from{pdfescape.dtx}{test4}}%
  \file{pdfescape-test5.tex}{\from{pdfescape.dtx}{test5}}%
  \file{pdfescape-test6.tex}{\from{pdfescape.dtx}{test6}}%
  \nopreamble
  \nopostamble
  \usedir{source/latex/oberdiek/catalogue}%
  \file{pdfescape.xml}{\from{pdfescape.dtx}{catalogue}}%
}

\catcode32=13\relax% active space
\let =\space%
\Msg{************************************************************************}
\Msg{*}
\Msg{* To finish the installation you have to move the following}
\Msg{* file into a directory searched by TeX:}
\Msg{*}
\Msg{*     pdfescape.sty}
\Msg{*}
\Msg{* To produce the documentation run the file `pdfescape.drv'}
\Msg{* through LaTeX.}
\Msg{*}
\Msg{* Happy TeXing!}
\Msg{*}
\Msg{************************************************************************}

\endbatchfile
%</install>
%<*ignore>
\fi
%</ignore>
%<*driver>
\NeedsTeXFormat{LaTeX2e}
\ProvidesFile{pdfescape.drv}%
  [2011/11/25 v1.13 Implements pdfTeX's escape features (HO)]%
\documentclass{ltxdoc}
\usepackage{holtxdoc}[2011/11/22]
\begin{document}
  \DocInput{pdfescape.dtx}%
\end{document}
%</driver>
% \fi
%
% \CheckSum{2516}
%
% \CharacterTable
%  {Upper-case    \A\B\C\D\E\F\G\H\I\J\K\L\M\N\O\P\Q\R\S\T\U\V\W\X\Y\Z
%   Lower-case    \a\b\c\d\e\f\g\h\i\j\k\l\m\n\o\p\q\r\s\t\u\v\w\x\y\z
%   Digits        \0\1\2\3\4\5\6\7\8\9
%   Exclamation   \!     Double quote  \"     Hash (number) \#
%   Dollar        \$     Percent       \%     Ampersand     \&
%   Acute accent  \'     Left paren    \(     Right paren   \)
%   Asterisk      \*     Plus          \+     Comma         \,
%   Minus         \-     Point         \.     Solidus       \/
%   Colon         \:     Semicolon     \;     Less than     \<
%   Equals        \=     Greater than  \>     Question mark \?
%   Commercial at \@     Left bracket  \[     Backslash     \\
%   Right bracket \]     Circumflex    \^     Underscore    \_
%   Grave accent  \`     Left brace    \{     Vertical bar  \|
%   Right brace   \}     Tilde         \~}
%
% \GetFileInfo{pdfescape.drv}
%
% \title{The \xpackage{pdfescape} package}
% \date{2011/11/25 v1.13}
% \author{Heiko Oberdiek\\\xemail{heiko.oberdiek at googlemail.com}}
%
% \maketitle
%
% \begin{abstract}
% This package implements \pdfTeX's escape features
% (\cs{pdfescapehex}, \cs{pdfunescapehex}, \cs{pdfescapename},
% \cs{pdfescapestring}) using \TeX\ or \eTeX.
% \end{abstract}
%
% \tableofcontents
%
% \section{Documentation}
%
% \begin{declcs}{EdefEscapeHex} \M{cmd} \M{string}\\
%  \cs{EdefUnescapeHex} \M{cmd} \M{string}\\
%  \cs{EdefEscapeName} \M{cmd} \M{string}\\
%  \cs{EdefEscapeString} \M{cmd} \M{string}
% \end{declcs}
% These commands converts \meta{string} and stores the result
% in macro \meta{cmd}. The conversion result is the same
% as the conversion of the corresponding \pdfTeX's primitives.
% Note that the argument \meta{string} is expanded before
% the conversion.
%
% For example, if \pdfTeX\ \textgreater= 1.30 is present,
% then \cs{EdefEscapeHex}
% becomes to:
% \begin{quote}
%\begin{verbatim}
%\def\EdefEscapeHex#1#2{%
%  \edef#1{\pdfescapehex{#2}}%
%}
%\end{verbatim}
% \end{quote}
% The package provides implementations for the case that \pdfTeX\ is
% not present (or too old). Even \eTeX\ can be missing, however it is
% used if it is detected.
%
% \paragraph{Babel.}
% The input strings may contain shorthand characters of package
% \xpackage{babel}.
%
% \subsection{Additional unescape macros}
%
% \begin{declcs}{EdefUnescapeName} \M{cmd} \M{string}
% \end{declcs}
% Sequences of a hash sign with two hexadecimal digits are
% converted to the corresponding character (PDF-1.2).
% A hash sign that is not followed by two hexadecimal digits
% is left unchanged. The catcodes in the result string follow
% TeX's conventions. The space has catcode 10 (space) and
% the other characters have catcode 12 (other).
%
% \begin{declcs}{EdefUnescapeString} \M{cmd} \M{string}
% \end{declcs}
% Macro \meta{cmd} stores the unescaped string in \meta{string}.
% All the rules for literal strings are implemented, see
% PDF specification.
% The catcodes in the result string follow TeX's conventions.
%
% \subsection{Sanitizing macro}
%
% \begin{declcs}{EdefSanitize} \M{cmd} \M{string}
% \end{declcs}
% Argument \meta{string} is expanded, converted to a string
% of tokens with catcode 12 (other) and space tokens, and
% stored in macro \meta{cmd}.
%
% \StopEventually{
% }
%
% \section{Implementation}
%
%    \begin{macrocode}
%<*package>
%    \end{macrocode}
%
% \subsection{Reload check and package identification}
%    Reload check, especially if the package is not used with \LaTeX.
%    \begin{macrocode}
\begingroup\catcode61\catcode48\catcode32=10\relax%
  \catcode13=5 % ^^M
  \endlinechar=13 %
  \catcode35=6 % #
  \catcode39=12 % '
  \catcode44=12 % ,
  \catcode45=12 % -
  \catcode46=12 % .
  \catcode58=12 % :
  \catcode64=11 % @
  \catcode123=1 % {
  \catcode125=2 % }
  \expandafter\let\expandafter\x\csname ver@pdfescape.sty\endcsname
  \ifx\x\relax % plain-TeX, first loading
  \else
    \def\empty{}%
    \ifx\x\empty % LaTeX, first loading,
      % variable is initialized, but \ProvidesPackage not yet seen
    \else
      \expandafter\ifx\csname PackageInfo\endcsname\relax
        \def\x#1#2{%
          \immediate\write-1{Package #1 Info: #2.}%
        }%
      \else
        \def\x#1#2{\PackageInfo{#1}{#2, stopped}}%
      \fi
      \x{pdfescape}{The package is already loaded}%
      \aftergroup\endinput
    \fi
  \fi
\endgroup%
%    \end{macrocode}
%    Package identification:
%    \begin{macrocode}
\begingroup\catcode61\catcode48\catcode32=10\relax%
  \catcode13=5 % ^^M
  \endlinechar=13 %
  \catcode35=6 % #
  \catcode39=12 % '
  \catcode40=12 % (
  \catcode41=12 % )
  \catcode44=12 % ,
  \catcode45=12 % -
  \catcode46=12 % .
  \catcode47=12 % /
  \catcode58=12 % :
  \catcode64=11 % @
  \catcode91=12 % [
  \catcode93=12 % ]
  \catcode123=1 % {
  \catcode125=2 % }
  \expandafter\ifx\csname ProvidesPackage\endcsname\relax
    \def\x#1#2#3[#4]{\endgroup
      \immediate\write-1{Package: #3 #4}%
      \xdef#1{#4}%
    }%
  \else
    \def\x#1#2[#3]{\endgroup
      #2[{#3}]%
      \ifx#1\@undefined
        \xdef#1{#3}%
      \fi
      \ifx#1\relax
        \xdef#1{#3}%
      \fi
    }%
  \fi
\expandafter\x\csname ver@pdfescape.sty\endcsname
\ProvidesPackage{pdfescape}%
  [2011/11/25 v1.13 Implements pdfTeX's escape features (HO)]%
%    \end{macrocode}
%
% \subsection{Catcodes}
%
%    \begin{macrocode}
\begingroup\catcode61\catcode48\catcode32=10\relax%
  \catcode13=5 % ^^M
  \endlinechar=13 %
  \catcode123 1 % {
  \catcode125 2 % }
  \catcode64 11 %
  \def\x{\endgroup
    \expandafter\edef\csname PE@AtEnd\endcsname{%
      \endlinechar=\the\endlinechar\relax
      \catcode13=\the\catcode13\relax
      \catcode32=\the\catcode32\relax
      \catcode35=\the\catcode35\relax
      \catcode61=\the\catcode61\relax
      \catcode64=\the\catcode64\relax
      \catcode123=\the\catcode123\relax
      \catcode125=\the\catcode125\relax
    }%
  }%
\x\catcode61\catcode48\catcode32=10\relax%
\catcode13=5 % ^^M
\endlinechar=13 %
\catcode35=6 % #
\catcode64=11 % @
\catcode123=1 % {
\catcode125=2 % }
\def\TMP@EnsureCode#1#2#3{%
  \edef\PE@AtEnd{%
    \PE@AtEnd
    #1#2=\the#1#2\relax
  }%
  #1#2=#3\relax
}
\TMP@EnsureCode\catcode{0}{12}% ^^@
\TMP@EnsureCode\catcode{34}{12}% "
\TMP@EnsureCode\catcode{36}{3}% $
\TMP@EnsureCode\catcode{38}{4}% &
\TMP@EnsureCode\catcode{39}{12}% '
\TMP@EnsureCode\catcode{42}{12}% *
\TMP@EnsureCode\catcode{45}{12}% -
\TMP@EnsureCode\catcode{46}{12}% .
\TMP@EnsureCode\catcode{47}{12}% /
\TMP@EnsureCode\catcode{60}{12}% <
\TMP@EnsureCode\catcode{62}{12}% >
\TMP@EnsureCode\catcode{91}{12}% [
\TMP@EnsureCode\catcode{93}{12}% ]
\TMP@EnsureCode\catcode{94}{7}% ^
\TMP@EnsureCode\catcode{96}{12}% `
\TMP@EnsureCode\uccode{34}{0}% "
\TMP@EnsureCode\uccode{48}{0}% 0
\TMP@EnsureCode\uccode{61}{0}% =
\edef\PE@AtEnd{\PE@AtEnd\noexpand\endinput}
%    \end{macrocode}
%
% \subsection{Load package}
%
%    \begin{macrocode}
\begingroup\expandafter\expandafter\expandafter\endgroup
\expandafter\ifx\csname RequirePackage\endcsname\relax
  \def\TMP@RequirePackage#1[#2]{%
    \begingroup\expandafter\expandafter\expandafter\endgroup
    \expandafter\ifx\csname ver@#1.sty\endcsname\relax
      \input #1.sty\relax
    \fi
  }%
  \TMP@RequirePackage{ltxcmds}[2010/04/08]%
\else
  \RequirePackage{ltxcmds}[2010/04/08]%
\fi
%    \end{macrocode}
%
% \subsection{Sanitizing}
%
%    \begin{macro}{\EdefSanitize}
%    Macro \cs{EdefSanitize} takes |#2|, entirely converts
%    it to token with catcode 12 (other) and stores the
%    result in macro |#1|.
%    \begin{macrocode}
\begingroup\expandafter\expandafter\expandafter\endgroup
\expandafter\ifx\csname detokenize\endcsname\relax
  \long\def\EdefSanitize#1#2{%
    \begingroup
      \csname @safe@activestrue\endcsname
      \edef#1{#2}%
      \PE@onelevel@sanitize#1%
    \expandafter\endgroup
    \expandafter\def\expandafter#1\expandafter{#1}%
  }%
  \begingroup\expandafter\expandafter\expandafter\endgroup
  \expandafter\ifx\csname @onelevel@sanitize\endcsname\relax
    \def\PE@onelevel@sanitize#1{%
      \edef#1{\expandafter\PE@strip@prefix\meaning#1}%
    }%
    \def\PE@strip@prefix#1>{}%
  \else
    \let\PE@onelevel@sanitize\@onelevel@sanitize
  \fi
\else
  \long\def\EdefSanitize#1#2{%
    \begingroup
      \csname @safe@activestrue\endcsname
      \edef#1{#2}%
    \expandafter\endgroup
    \expandafter\def\expandafter#1\expandafter{%
      \detokenize\expandafter{#1}%
    }%
  }%
  \def\PE@onelevel@sanitize#1{%
    \edef#1{\detokenize\expandafter{#1}}%
  }%
\fi
%    \end{macrocode}
%    \end{macro}
%    \begin{macro}{\PE@sanitize}
%    Macro \cs{PE@sanitize} is only defined for
%    compatibility with version 1.4. Its use is deprecated.
%    \begin{macrocode}
\let\PE@sanitize\EdefSanitize
%    \end{macrocode}
%    \end{macro}
%
% \subsubsection{Space characters}
%
%    \begin{macro}{\PE@space@other}
%    \begin{macrocode}
\begingroup
  \catcode`\ =12\relax%
\def\x{\endgroup\def\PE@space@other{ }}\x\relax
%    \end{macrocode}
%    \end{macro}
%    \begin{macro}{\PE@space@space}
%    \begin{macrocode}
\def\PE@space@space{ }
%    \end{macrocode}
%    \end{macro}
%
% \subsubsection{Space normalization}
%
%    \begin{macro}{\PE@SanitizeSpaceOther}
%    \begin{macrocode}
\def\PE@SanitizeSpaceOther#1{%
  \edef#1{\expandafter\PE@SpaceToOther#1 \relax}%
}
%    \end{macrocode}
%    \end{macro}
%
%    \begin{macro}{\PE@SpaceToOther}
%    \begin{macrocode}
\def\PE@SpaceToOther#1 #2\relax{%
  #1%
  \ifx\\#2\\%
  \else
    \PE@space@other
    \ltx@ReturnAfterFi{%
      \PE@SpaceToOther#2\relax
    }%
  \fi
}
%    \end{macrocode}
%    \end{macro}
%
% \subsection{\cs{EdefUnescapeName}}
%
%    \begin{macro}{\EdefUnescapeName}
%    \begin{macrocode}
\def\EdefUnescapeName#1#2{%
  \EdefSanitize#1{#2}%
  \PE@SanitizeSpaceOther#1%
  \PE@UnescapeName#1%
  \PE@onelevel@sanitize#1%
}
%    \end{macrocode}
%    \end{macro}
%
%    \begin{macro}{\PE@UnescapeName}
%    \begin{macrocode}
\begingroup
  \catcode`\$=6  % hash
  \catcode`\#=12 % other
  \gdef\PE@UnescapeName$1{%
    \begingroup
      \PE@InitUccodeHexDigit
      \def\PE@result{}%
      \expandafter\PE@DeName$1#\relax\relax
    \expandafter\endgroup
    \expandafter\def\expandafter$1\expandafter{\PE@result}%
  }%
  \gdef\PE@DeName$1#$2$3{%
    \ifx\relax$2%
      \edef\PE@result{\PE@result$1}%
      \let\PE@next\relax
    \else
      \ifx\relax$3%
        % wrong escape sequence in input
        \edef\PE@result{\PE@result$1#}%
        \let\PE@next\relax
      \else
        \uppercase{%
          \def\PE@testA{$2}%
          \def\PE@testB{$3}%
        }%
        \ifcase\ifcase\expandafter\PE@TestUcHexDigit\PE@testA
                 \ifcase\expandafter\PE@TestUcHexDigit\PE@testB
                   \ltx@zero
                 \else
                   \ltx@one
                 \fi
               \else
                 \ltx@one
               \fi
          \uccode\ltx@zero="\PE@testA\PE@testB\relax
          \uppercase{%
            \def\PE@temp{^^@}%
          }%
          \uccode\ltx@zero=\ltx@zero
          \edef\PE@result{\PE@result$1\PE@temp}%
          \let\PE@next\PE@DeName
        \else
          % wrong escape sequence in input
          \edef\PE@result{\PE@result$1#}%
          \def\PE@next{\PE@DeName$2$3}%
        \fi
      \fi
    \fi
    \PE@next
  }%
\endgroup
%    \end{macrocode}
%    \end{macro}
%
%    \begin{macro}{\PE@InitUccodeHexDigit}
%    \begin{macrocode}
\def\PE@InitUccodeHexDigit{%
  \uccode`a=`A\relax
  \uccode`b=`B\relax
  \uccode`c=`C\relax
  \uccode`d=`D\relax
  \uccode`e=`E\relax
  \uccode`f=`F\relax
  \uccode`A=\ltx@zero
  \uccode`B=\ltx@zero
  \uccode`C=\ltx@zero
  \uccode`D=\ltx@zero
  \uccode`E=\ltx@zero
  \uccode`F=\ltx@zero
  \uccode`0=\ltx@zero
  \uccode`1=\ltx@zero
  \uccode`2=\ltx@zero
  \uccode`3=\ltx@zero
  \uccode`4=\ltx@zero
  \uccode`5=\ltx@zero
  \uccode`6=\ltx@zero
  \uccode`7=\ltx@zero
  \uccode`8=\ltx@zero
  \uccode`9=\ltx@zero
}
%    \end{macrocode}
%    \end{macro}
%
%    \begin{macro}{\PE@TestUcHexDigit}
%    \begin{macrocode}
\def\PE@TestUcHexDigit#1{%
  \ifnum`#1<48 % 0
    \ltx@one
  \else
    \ifnum`#1>70 % F
      \ltx@one
    \else
      \ifnum`#1>57 % 9
        \ifnum`#1<65 % A
          \ltx@one
        \else
          \ltx@zero
        \fi
      \else
        \ltx@zero
      \fi
    \fi
  \fi
}
%    \end{macrocode}
%    \end{macro}
%
% \subsection{\cs{EdefUnescapeString}}
%
%    \begin{macro}{\EdefUnescapeString}
%    \begin{macrocode}
\def\EdefUnescapeString#1#2{%
  \EdefSanitize#1{#2}%
  \PE@SanitizeSpaceOther#1%
  \PE@NormalizeLineEnd#1%
  \PE@UnescapeString#1%
  \PE@onelevel@sanitize#1%
}
%    \end{macrocode}
%    \end{macro}
%
%    \begin{macrocode}
\begingroup
  \uccode`\8=10 % lf
  \uccode`\9=13 % cr
\def\x#1#2{\endgroup
%    \end{macrocode}
%    \begin{macro}{\PE@NormalizeLineEnd}
%    \begin{macrocode}
  \def\PE@NormalizeLineEnd##1{%
    \def\PE@result{}%
    \expandafter\PE@@NormalizeLineEnd##1#2\relax
    \let##1\PE@result
  }%
%    \end{macrocode}
%    \end{macro}
%    \begin{macro}{\PE@@NormalizeLineEnd}
%    \begin{macrocode}
  \def\PE@@NormalizeLineEnd##1#2##2{%
    \ifx\relax##2%
      \edef\PE@result{\PE@result##1}%
      \let\PE@next\relax
    \else
      \edef\PE@result{\PE@result##1#1}%
      \ifx#1##2% lf
        \let\PE@next\PE@@NormalizeLineEnd
      \else
        \def\PE@next{\PE@@NormalizeLineEnd##2}%
      \fi
    \fi
    \PE@next
  }%
}%
\uppercase{%
  \x 89%
}
%    \end{macrocode}
%    \end{macro}
%
%    \begin{macrocode}
\begingroup
  \catcode`\|=0 %
  \catcode`\\=12 %
%    \end{macrocode}
%    \begin{macro}{\PE@UnescapeString}
%    \begin{macrocode}
  |gdef|PE@UnescapeString#1{%
    |begingroup
      |def|PE@result{}%
      |expandafter|PE@DeString#1\|relax
    |expandafter|endgroup
    |expandafter|def|expandafter#1|expandafter{|PE@result}%
  }%
%    \end{macrocode}
%    \end{macro}
%    \begin{macro}{\PE@DeString}
%    \begin{macrocode}
  |gdef|PE@DeString#1\#2{%
    |ifx|relax#2%
      |edef|PE@result{|PE@result#1}%
      |let|PE@next|relax
    |else
      |if n#2%
        |uccode|ltx@zero=10 %
      |else|if r#2%
        |uccode|ltx@zero=13 %
      |else|if t#2%
        |uccode|ltx@zero=9 %
      |else|if b#2%
        |uccode|ltx@zero=8 %
      |else|if f#2%
        |uccode|ltx@zero=12 %
      |else
        |uccode|ltx@zero=|ltx@zero
      |fi|fi|fi|fi|fi
      |ifnum|uccode|ltx@zero>|ltx@zero
        |uppercase{%
          |edef|PE@temp{^^@}%
        }%
        |edef|PE@result{|PE@result#1|PE@temp}%
        |let|PE@next|PE@DeString
      |else
        |if\#2% backslash
          |edef|PE@result{|PE@result#1}%
          |let|PE@next|PE@CheckEndBackslash
        |else
          |ifnum`#2=10 % linefeed
            |edef|PE@result{|PE@result#1}%
            |let|PE@next|PE@DeString
          |else
            |ifcase|PE@TestOctDigit#2%
              |edef|PE@result{|PE@result#1}%
              |def|PE@next{|PE@OctI#2}%
            |else
              |edef|PE@result{|PE@result#1#2}%
              |let|PE@next|PE@DeString
            |fi
          |fi
        |fi
      |fi
    |fi
    |PE@next
  }%
%    \end{macrocode}
%    \end{macro}
%    \begin{macro}{\PE@CheckEndBackslash}
%    \begin{macrocode}
  |gdef|PE@CheckEndBackslash#1{%
    |ifx|relax#1%
    |else
      |edef|PE@result{|PE@result\}%
      |expandafter|PE@DeString|expandafter#1%
    |fi
  }%
%    \end{macrocode}
%    \end{macro}
%    \begin{macrocode}
|endgroup
%    \end{macrocode}
%    \begin{macro}{\PE@TestOctDigit}
%    \begin{macrocode}
\def\PE@TestOctDigit#1{%
  \ifnum`#1<48 % 0
    \ltx@one
  \else
    \ifnum`#1>55 % 7
      \ltx@one
    \else
      \ltx@zero
    \fi
  \fi
}
%    \end{macrocode}
%    \end{macro}
%    \begin{macro}{\PE@OctI}
%    \begin{macrocode}
\def\PE@OctI#1#2{%
  \ifcase\PE@TestOctDigit#2%
    \def\PE@next{\PE@OctII{#1#2}}%
  \else
    \def\PE@next{\PE@OctAll#1#2}%
  \fi
  \PE@next
}
%    \end{macrocode}
%    \end{macro}
%    \begin{macro}{\PE@OctII}
%    \begin{macrocode}
\def\PE@OctII#1#2{%
  \ifcase\PE@TestOctDigit#2%
    \def\PE@next{\PE@OctIII#1#2}%
  \else
    \def\PE@next{\PE@OctAll{#1}#2}%
  \fi
  \PE@next
}
%    \end{macrocode}
%    \end{macro}
%    \begin{macrocode}
\ltx@IfUndefined{numexpr}{%
  \catcode`\$=9 %
  \catcode`\&=14 %
}{%
  \catcode`\$=14 %
  \catcode`\&=9 %
}
%    \end{macrocode}
%    \begin{macro}{\PE@OctIII}
%    \begin{macrocode}
\def\PE@OctIII#1#2#3{%
  \ifnum#1<4 %
    \def\PE@next{\PE@OctAll{#1#2#3}}%
  \else
$   \count\ltx@cclv#1 %
$   \advance\count\ltx@cclv -4 %
    \edef\PE@next{%
      \noexpand\PE@OctAll{%
$       \the\count\ltx@cclv
&       \the\numexpr#1-4\relax
        #2#3%
      }%
    }%
  \fi
  \PE@next
}
%    \end{macrocode}
%    \end{macro}
%    \begin{macro}{\PE@OctAll}
%    \begin{macrocode}
\def\PE@OctAll#1{%
  \uccode\ltx@zero='#1\relax
  \uppercase{%
    \edef\PE@result{\PE@result^^@}%
  }%
  \PE@DeString
}
%    \end{macrocode}
%    \end{macro}
%
% \subsection{User macros (\pdfTeX\space analogues)}
%
%    \begin{macrocode}
\begingroup\expandafter\expandafter\expandafter\endgroup
\expandafter\ifx\csname RequirePackage\endcsname\relax
  \def\TMP@RequirePackage#1[#2]{%
    \begingroup\expandafter\expandafter\expandafter\endgroup
    \expandafter\ifx\csname ver@#1.sty\endcsname\relax
      \input #1.sty\relax
    \fi
  }%
  \TMP@RequirePackage{pdftexcmds}[2007/11/11]%
\else
  \RequirePackage{pdftexcmds}[2007/11/11]%
\fi
%    \end{macrocode}
%
%    \begin{macrocode}
\begingroup\expandafter\expandafter\expandafter\endgroup
\expandafter\ifx\csname pdf@escapehex\endcsname\relax
%    \end{macrocode}
%
%    \begin{macro}{\EdefEscapeHex}
%    \begin{macrocode}
  \long\def\EdefEscapeHex#1#2{%
    \EdefSanitize#1{#2}%
    \PE@SanitizeSpaceOther#1%
    \PE@EscapeHex#1%
  }%
%    \end{macrocode}
%    \end{macro}
%    \begin{macro}{\EdefUnescapeHex}
%    \begin{macrocode}
  \def\EdefUnescapeHex#1#2{%
    \EdefSanitize#1{#2}%
    \PE@UnescapeHex#1%
  }%
%    \end{macrocode}
%    \end{macro}
%    \begin{macro}{\EdefEscapeName}
%    \begin{macrocode}
  \long\def\EdefEscapeName#1#2{%
    \EdefSanitize#1{#2}%
    \PE@SanitizeSpaceOther#1%
    \PE@EscapeName#1%
  }%
%    \end{macrocode}
%    \end{macro}
%    \begin{macro}{\EdefEscapeString}
%    \begin{macrocode}
  \long\def\EdefEscapeString#1#2{%
    \EdefSanitize#1{#2}%
    \PE@SanitizeSpaceOther#1%
    \PE@EscapeString#1%
  }%
%    \end{macrocode}
%    \end{macro}
%
%    \begin{macrocode}
\else
%    \end{macrocode}
%
%    \begin{macro}{\PE@edefbabel}
%    Help macro that adds support for \xpackage{babel}'s shorthand
%    characters.
%    \begin{macrocode}
  \long\def\PE@edefbabel#1#2#3{%
    \begingroup
      \csname @save@activestrue\endcsname
      \edef#1{#2{#3}}%
    \expandafter\endgroup
    \expandafter\def\expandafter#1\expandafter{#1}%
  }%
%    \end{macrocode}
%    \end{macro}
%
%    \begin{macro}{\EdefEscapeHex}
%    \begin{macrocode}
  \long\def\EdefEscapeHex#1#2{%
    \PE@edefbabel#1\pdf@escapehex{#2}%
  }%
%    \end{macrocode}
%    \end{macro}
%    \begin{macro}{\EdefUnescapeHex}
%    \begin{macrocode}
  \def\EdefUnescapeHex#1#2{%
    \PE@edefbabel#1\pdf@unescapehex{#2}%
  }%
%    \end{macrocode}
%    \end{macro}
%    \begin{macro}{\EdefEscapeName}
%    \begin{macrocode}
  \long\def\EdefEscapeName#1#2{%
    \PE@edefbabel#1\pdf@escapename{#2}%
  }%
%    \end{macrocode}
%    \end{macro}
%    \begin{macro}{\EdefEscapeString}
%    \begin{macrocode}
  \long\def\EdefEscapeString#1#2{%
    \PE@edefbabel#1\pdf@escapestring{#2}%
  }%
%    \end{macrocode}
%    \end{macro}
%
%    \begin{macrocode}
  \expandafter\PE@AtEnd
\fi%
%    \end{macrocode}
%
% \subsection{Help macros}
%
% \subsubsection{Characters}
%
%    Special characters with catcode 12 (other) are created and
%    stored in macros.
%    \begin{macro}{\PE@hash}
%    \begin{macrocode}
\edef\PE@hash{\string#}
%    \end{macrocode}
%    \end{macro}
%    \begin{macro}{\PE@backslash}
%    \begin{macrocode}
\begingroup
  \escapechar=-1 %
\edef\x{\endgroup
  \def\noexpand\PE@backslash{\string\\}%
}
\x
%    \end{macrocode}
%    \end{macro}
%
% \subsubsection{Switch for \eTeX}
%
%    \begin{macrocode}
\ltx@newif\ifPE@etex
\begingroup\expandafter\expandafter\expandafter\endgroup
\expandafter\ifx\csname numexpr\endcsname\relax
\else
  \PE@etextrue
\fi
%    \end{macrocode}
%
% \subsection{Conversions}
%
% \subsubsection{Conversion to hex string}
%
%    \begin{macro}{\PE@EscapeHex}
%    \begin{macrocode}
\ifPE@etex
  \def\PE@EscapeHex#1{%
    \edef#1{\expandafter\PE@ToHex#1\relax}%
  }%
\else
  \def\PE@EscapeHex#1{%
    \def\PE@result{}%
    \expandafter\PE@ToHex#1\relax
    \let#1\PE@result
  }%
\fi
%    \end{macrocode}
%    \end{macro}
%    \begin{macro}{\PE@ToHex}
%    \begin{macrocode}
\def\PE@ToHex#1{%
  \ifx\relax#1%
  \else
    \PE@HexChar{#1}%
    \expandafter\PE@ToHex
  \fi
}%
%    \end{macrocode}
%    \end{macro}
%    \begin{macro}{\PE@HexChar}
%    \begin{macrocode}
\ifPE@etex
  \def\PE@HexChar#1{%
    \PE@HexDigit{\numexpr\dimexpr.0625\dimexpr`#1sp\relax\relax\relax}%
    \PE@HexDigit{%
      \numexpr`#1-16*\dimexpr.0625\dimexpr`#1sp\relax\relax\relax
    }%
  }%
\else
  \def\PE@HexChar#1{%
    \dimen0=`#1sp%
    \dimen2=.0625\dimen0 %
    \advance\dimen0-16\dimen2 %
    \edef\PE@result{%
      \PE@result
      \PE@HexDigit{\dimen2 }%
      \PE@HexDigit{\dimen0 }%
    }%
  }%
\fi
%    \end{macrocode}
%    \end{macro}
%    \begin{macro}{\PE@HexDigit}
%    \begin{macrocode}
\def\PE@HexDigit#1{%
  \expandafter\string
  \ifcase#1%
    0\or 1\or 2\or 3\or 4\or 5\or 6\or 7\or 8\or 9\or
    A\or B\or C\or D\or E\or F%
  \fi
}
%    \end{macrocode}
%    \end{macro}
%
% \subsubsection{Character code to octal number}
%
%    \begin{macro}{\PE@OctChar}
%    \begin{macrocode}
\ifPE@etex
  \def\PE@OctChar#1{%
    \expandafter\PE@@OctChar
        \the\numexpr\dimexpr.015625\dimexpr`#1sp\relax\relax
            \expandafter\relax
        \expandafter\relax
        \the\numexpr\dimexpr.125\dimexpr`#1sp\relax\relax\relax
        \relax
        #1%
  }%
  \def\PE@@OctChar#1\relax#2\relax#3{%
    \PE@backslash
    #1%
    \the\numexpr#2-8*#1\relax
    \the\numexpr\dimexpr`#3sp\relax-8*#2\relax
  }%
\else
  \def\PE@OctChar#1{%
    \dimen0=`#1sp%
    \dimen2=.125\dimen0 %
    \dimen4=.125\dimen2 %
    \advance\dimen0-8\dimen2 %
    \advance\dimen2-8\dimen4 %
    \edef\PE@result{%
      \PE@result
      \PE@backslash
      \number\dimen4 %
      \number\dimen2 %
      \number\dimen0 %
    }%
  }%
\fi
%    \end{macrocode}
%    \end{macro}
%
% \subsubsection{Unpack hex string}
%
%    \begin{macro}{\PE@UnescapeHex}
%    \begin{macrocode}
\def\PE@UnescapeHex#1{%
  \begingroup
    \PE@InitUccodeHexDigit
    \def\PE@result{}%
    \expandafter\PE@DeHex#1\relax\relax
  \expandafter\endgroup
  \expandafter\def\expandafter#1\expandafter{\PE@result}%
}
%    \end{macrocode}
%    \end{macro}
%    \begin{macro}{\PE@DeHex}
%    \begin{macrocode}
\def\PE@DeHex#1#2{%
  \ifx#2\relax
    \ifx#1\relax
      \let\PE@next\relax
    \else
      \uppercase{%
        \def\PE@testA{#1}%
      }%
      \ifcase\expandafter\PE@TestUcHexDigit\PE@testA
        \def\PE@next{%
          \PE@DeHex#10\relax\relax
        }%
      \else
        \let\PE@next\relax
      \fi
    \fi
  \else
    \uppercase{%
      \def\PE@testA{#1}%
      \def\PE@testB{#2}%
    }%
    \ifcase\expandafter\PE@TestUcHexDigit\PE@testA
      \ifcase\expandafter\PE@TestUcHexDigit\PE@testB
        \uccode\ltx@zero="\PE@testA\PE@testB\relax
        \ifnum\uccode\ltx@zero=32 %
          \let\PE@temp\PE@space@space
        \else
          \uppercase{%
            \def\PE@temp{^^@}%
          }%
        \fi
        \edef\PE@result{\PE@result\PE@temp}%
        \let\PE@next\PE@DeHex
      \else
        % invalid input sequence
        \def\PE@next{%
          \PE@DeHex#1%
        }%
      \fi
    \else
      % invalid input sequence
      \def\PE@next{\PE@DeHex#2}%
    \fi
  \fi
  \PE@next
}
%    \end{macrocode}
%    \end{macro}
%
% \subsubsection{Conversion to PDF name}
%
%    \begin{macro}{\PE@EscapeName}
%    \begin{macrocode}
\ifPE@etex
  \def\PE@EscapeName#1{%
    \edef#1{\expandafter\PE@EscapeNameTokens#1\relax}%
  }%
\else
  \def\PE@EscapeName#1{%
    \def\PE@result{}%
    \expandafter\PE@EscapeNameTokens#1\relax
    \let#1\PE@result
  }%
\fi
%    \end{macrocode}
%    \end{macro}
%    \begin{macro}{\PE@EscapeNameTokens}
%    \begin{macrocode}
\def\PE@EscapeNameTokens#1{%
  \ifx\relax#1%
  \else
    \ifnum`#1<33 %
      \ifcase`#1 %
        % drop illegal zero
      \else
        \PE@EscapeNameAdd\PE@hash
        \PE@HexChar#1%
      \fi
    \else
      \ifnum`#1>126 %
        \PE@EscapeNameAdd\PE@hash
        \PE@HexChar#1%
      \else  \ifnum`#1=35  \PE@EscapeNameHashChar 23% #
        \else\ifnum`#1=37  \PE@EscapeNameHashChar 25% %
        \else\ifnum`#1=40  \PE@EscapeNameHashChar 28% (
        \else\ifnum`#1=41  \PE@EscapeNameHashChar 29% )
        \else\ifnum`#1=47  \PE@EscapeNameHashChar 2F% /
        \else\ifnum`#1=60  \PE@EscapeNameHashChar 3C% <
        \else\ifnum`#1=62  \PE@EscapeNameHashChar 3E% >
        \else\ifnum`#1=91  \PE@EscapeNameHashChar 5B% [
        \else\ifnum`#1=93  \PE@EscapeNameHashChar 5D% ]
        \else\ifnum`#1=123 \PE@EscapeNameHashChar 7B% {
        \else\ifnum`#1=125 \PE@EscapeNameHashChar 7D% }
        \else
          \PE@EscapeNameAdd{#1}%
        \fi\fi\fi\fi\fi\fi\fi\fi\fi\fi\fi
      \fi
    \fi
    \expandafter\PE@EscapeNameTokens
  \fi
}%
\def\PE@EscapeNameHashChar#1#2{%
  \PE@EscapeNameAdd{\PE@hash\string#1\string#2}%
}%
%    \end{macrocode}
%    \end{macro}
%    \begin{macro}{\PE@EscapeNameAdd}
%    \begin{macrocode}
\ifPE@etex
  \def\PE@EscapeNameAdd#1{#1}%
\else
  \def\PE@EscapeNameAdd#1{%
    \edef\PE@result{%
      \PE@result
      #1%
    }%
  }%
\fi
%    \end{macrocode}
%    \end{macro}
%
% \subsubsection{Conversion to PDF string}
%
%    \begin{macro}{\PE@EscapeString}
%    \begin{macrocode}
\ifPE@etex
  \def\PE@EscapeString#1{%
    \edef#1{\expandafter\PE@EscapeStringTokens#1\relax}%
  }%
\else
  \def\PE@EscapeString#1{%
    \begingroup
      \def\PE@result{}%
      \expandafter\PE@EscapeStringTokens#1\relax
    \expandafter\endgroup
    \expandafter\def\expandafter#1\expandafter{\PE@result}%
  }%
\fi
%    \end{macrocode}
%    \end{macro}
%    \begin{macro}{\PE@EscapeStringTokens}
%    \begin{macrocode}
\def\PE@EscapeStringTokens#1{%
  \ifx\relax#1%
  \else
    \ifnum`#1<33 %
      \PE@OctChar#1%
    \else
      \ifnum`#1>126 %
        \PE@OctChar#1%
      \else  \ifnum`#1=40 \PE@EscapeStringAdd{\string\(}% (
        \else\ifnum`#1=41 \PE@EscapeStringAdd{\string\)}% )
        \else\ifnum`#1=92 \PE@EscapeStringAdd{\string\\}% \
        \else
          \PE@EscapeStringAdd{#1}%
        \fi\fi\fi
      \fi
    \fi
    \expandafter\PE@EscapeStringTokens
  \fi
}%
%    \end{macrocode}
%    \end{macro}
%    \begin{macro}{\PE@EscapeStringAdd}
%    \begin{macrocode}
\ifPE@etex
  \def\PE@EscapeStringAdd#1{#1}%
\else
  \def\PE@EscapeStringAdd#1{%
    \edef\PE@result{%
      \PE@result
      #1%
    }%
  }%
\fi
%    \end{macrocode}
%    \end{macro}
%
%    \begin{macrocode}
\PE@AtEnd%
%    \end{macrocode}
%
%    \begin{macrocode}
%</package>
%    \end{macrocode}
%
% \section{Test}
%
% \subsection{Catcode checks for loading}
%
%    \begin{macrocode}
%<*test1>
%    \end{macrocode}
%    \begin{macrocode}
\catcode`\{=1 %
\catcode`\}=2 %
\catcode`\#=6 %
\catcode`\@=11 %
\expandafter\ifx\csname count@\endcsname\relax
  \countdef\count@=255 %
\fi
\expandafter\ifx\csname @gobble\endcsname\relax
  \long\def\@gobble#1{}%
\fi
\expandafter\ifx\csname @firstofone\endcsname\relax
  \long\def\@firstofone#1{#1}%
\fi
\expandafter\ifx\csname loop\endcsname\relax
  \expandafter\@firstofone
\else
  \expandafter\@gobble
\fi
{%
  \def\loop#1\repeat{%
    \def\body{#1}%
    \iterate
  }%
  \def\iterate{%
    \body
      \let\next\iterate
    \else
      \let\next\relax
    \fi
    \next
  }%
  \let\repeat=\fi
}%
\def\RestoreCatcodes{}
\count@=0 %
\loop
  \edef\RestoreCatcodes{%
    \RestoreCatcodes
    \catcode\the\count@=\the\catcode\count@\relax
  }%
\ifnum\count@<255 %
  \advance\count@ 1 %
\repeat

\def\RangeCatcodeInvalid#1#2{%
  \count@=#1\relax
  \loop
    \catcode\count@=15 %
  \ifnum\count@<#2\relax
    \advance\count@ 1 %
  \repeat
}
\def\RangeCatcodeCheck#1#2#3{%
  \count@=#1\relax
  \loop
    \ifnum#3=\catcode\count@
    \else
      \errmessage{%
        Character \the\count@\space
        with wrong catcode \the\catcode\count@\space
        instead of \number#3%
      }%
    \fi
  \ifnum\count@<#2\relax
    \advance\count@ 1 %
  \repeat
}
\def\space{ }
\expandafter\ifx\csname LoadCommand\endcsname\relax
  \def\LoadCommand{\input pdfescape.sty\relax}%
\fi
\def\Test{%
  \RangeCatcodeInvalid{0}{47}%
  \RangeCatcodeInvalid{58}{64}%
  \RangeCatcodeInvalid{91}{96}%
  \RangeCatcodeInvalid{123}{255}%
  \catcode`\@=12 %
  \catcode`\\=0 %
  \catcode`\%=14 %
  \LoadCommand
  \RangeCatcodeCheck{0}{36}{15}%
  \RangeCatcodeCheck{37}{37}{14}%
  \RangeCatcodeCheck{38}{47}{15}%
  \RangeCatcodeCheck{48}{57}{12}%
  \RangeCatcodeCheck{58}{63}{15}%
  \RangeCatcodeCheck{64}{64}{12}%
  \RangeCatcodeCheck{65}{90}{11}%
  \RangeCatcodeCheck{91}{91}{15}%
  \RangeCatcodeCheck{92}{92}{0}%
  \RangeCatcodeCheck{93}{96}{15}%
  \RangeCatcodeCheck{97}{122}{11}%
  \RangeCatcodeCheck{123}{255}{15}%
  \RestoreCatcodes
}
\Test
\csname @@end\endcsname
\end
%    \end{macrocode}
%    \begin{macrocode}
%</test1>
%    \end{macrocode}
%
% \subsection{Macro tests}
%
%    \begin{macrocode}
%<*test2|test3|test4|test5>
\NeedsTeXFormat{LaTeX2e}
\makeatletter
%    \end{macrocode}
%
% \subsection{Test with \cs{pdfescape...} commands}
%
%    \begin{macrocode}
%<*test2>
\ProvidesFile{pdfescape-test2.tex}%
    [2011/11/25 v1.13 Test with \string\pdfescape... commands]%
%</test2>
%    \end{macrocode}
%
% \subsection{Test without \cs{pdfescape...}, with \eTeX}
%
%    \begin{macrocode}
%<*test3>
\ProvidesFile{pdfescape-test3.tex}%
    [2011/11/25 v1.13 Test without \string\pdfescape..., with e-TeX]%
%</test3>
%    \end{macrocode}
%
% \subsection{Test without \cs{pdfescape...} and \eTeX}
%
%    \begin{macrocode}
%<*test4>
\ProvidesFile{pdfescape-test4.tex}%
    [2011/11/25 v1.13 Test without \string\pdfescape... and e-TeX]%
%</test4>
%    \end{macrocode}
%
% \subsection{Test with \LuaTeX}
%
%    \begin{macrocode}
%<*test5>
\ProvidesFile{pdfescape-test5.tex}%
    [2011/11/25 v1.13 Test with LuaTeX]%
%</test5>
%    \end{macrocode}
%
% \subsection{Check/ensure test preconditions}
%
% \subsubsection{Check \cs{pdfescape...}}
%
%    \begin{macrocode}
%<*test2>
\@ifundefined{pdfescapehex}{%
  \PackageError{pdfescape-test2}{%
    Missing \string\pdfescape... commands%
  }{Test aborted.}%
  \stop
}{}
%</test2>
%    \end{macrocode}
%    \begin{macrocode}
%<*test3|test4>
\let\pdfescapehex\@undefined
\let\pdfunescapehex\@undefined
\let\pdfescapename\@undefined
\let\pdfescapestring\@undefined
%</test3|test4>
%    \end{macrocode}
%
% \subsubsection{Check \eTeX}
%
%    \begin{macrocode}
%<*test3>
\@ifundefined{numexpr}{%
  \PackageError{pdfescape-test3}{%
    Missing \eTeX
  }{Test aborted.}%
  \stop
}{}
%</test3>
%    \end{macrocode}
% Package \xpackage{qstest} uses \eTeX, thus \eTeX's features
% can only be disabled later during loading of package \xpackage{pdfescape}.
%
% \subsubsection{Check \LuaTeX}
%
%    \begin{macrocode}
%<*test5>
\@ifundefined{directlua}{%
  \PackageError{pdfescape-test5}{%
    Missing LuaTeX%
  }{Test aborted.}%
  \stop
}{}
%</test5>
%    \end{macrocode}
%
% \subsection{Common part}
%
%    The files for testing uses the framework, provided by the
%    new package \xpackage{qstest} of David Kastrup.
%    \begin{macrocode}
\RequirePackage{qstest}
\IncludeTests{*}
\LogTests{log}{*}{*}

\newcommand*{\ExpectVar}[2]{%
  \ifx#1#2%
  \else
    \begingroup
      \@onelevel@sanitize#1%
      \@onelevel@sanitize#2%
      \typeout{[#1] <> [#2]}% hash-ok
    \endgroup
  \fi
  \Expect*{\ifx#1#2true\else false\fi}{true}%
}

\makeatletter
\begingroup
  \gdef\AllBytes{}%
  \count@=0 %
  \catcode0=12 %
  \@whilenum\count@<256 \do{%
    \lccode0=\count@
    \ifnum\count@=32 %
      \xdef\AllBytes{\AllBytes\space}%
    \else
      \lowercase{%
        \xdef\AllBytes{\AllBytes^^@}%
      }%
    \fi
    \advance\count@ by 1 %
  }%
\endgroup
\newcommand*{\AllBytesHex}{%
  000102030405060708090A0B0C0D0E0F%
  101112131415161718191A1B1C1D1E1F%
  202122232425262728292A2B2C2D2E2F%
  303132333435363738393A3B3C3D3E3F%
  404142434445464748494A4B4C4D4E4F%
  505152535455565758595A5B5C5D5E5F%
  606162636465666768696A6B6C6D6E6F%
  707172737475767778797A7B7C7D7E7F%
  808182838485868788898A8B8C8D8E8F%
  909192939495969798999A9B9C9D9E9F%
  A0A1A2A3A4A5A6A7A8A9AAABACADAEAF%
  B0B1B2B3B4B5B6B7B8B9BABBBCBDBEBF%
  C0C1C2C3C4C5C6C7C8C9CACBCCCDCECF%
  D0D1D2D3D4D5D6D7D8D9DADBDCDDDEDF%
  E0E1E2E3E4E5E6E7E8E9EAEBECEDEEEF%
  F0F1F2F3F4F5F6F7F8F9FAFBFCFDFEFF%
}
\@onelevel@sanitize\AllBytesHex
\expandafter\lowercase\expandafter{%
  \expandafter\newcommand\expandafter*\expandafter\AllBytesHexLC
      \expandafter{\AllBytesHex}%
}
\newcommand*{\AllBytesName}{}
\begingroup
  \catcode`\#=12 %
  \xdef\AllBytesName{%
    #01#02#03#04#05#06#07#08#09#0A#0B#0C#0D#0E#0F%
    #10#11#12#13#14#15#16#17#18#19#1A#1B#1C#1D#1E#1F%
    #20!"#23$#25&'#28#29*+,-.#2F%
    0123456789:;#3C=#3E?%
    @ABCDEFGHIJKLMNO%
    PQRSTUVWXYZ#5B\@backslashchar#5D^_%
    `abcdefghijklmno%
    pqrstuvwxyz#7B|#7D\string~#7F%
    #80#81#82#83#84#85#86#87#88#89#8A#8B#8C#8D#8E#8F%
    #90#91#92#93#94#95#96#97#98#99#9A#9B#9C#9D#9E#9F%
    #A0#A1#A2#A3#A4#A5#A6#A7#A8#A9#AA#AB#AC#AD#AE#AF%
    #B0#B1#B2#B3#B4#B5#B6#B7#B8#B9#BA#BB#BC#BD#BE#BF%
    #C0#C1#C2#C3#C4#C5#C6#C7#C8#C9#CA#CB#CC#CD#CE#CF%
    #D0#D1#D2#D3#D4#D5#D6#D7#D8#D9#DA#DB#DC#DD#DE#DF%
    #E0#E1#E2#E3#E4#E5#E6#E7#E8#E9#EA#EB#EC#ED#EE#EF%
    #F0#F1#F2#F3#F4#F5#F6#F7#F8#F9#FA#FB#FC#FD#FE#FF%
  }%
\endgroup
\@onelevel@sanitize\AllBytesName

\newcommand*{\AllBytesString}{}
\begingroup
  \def\|{|}%
  \edef\%{\@percentchar}%
  \catcode`\|=0 %
  \catcode`\#=12 %
  \catcode`\~=12 %
  \catcode`\\=12 %
  |xdef|AllBytesString{%
    \000\001\002\003\004\005\006\007\010\011\012\013\014\015\016\017%
    \020\021\022\023\024\025\026\027\030\031\032\033\034\035\036\037%
    \040!"#$|%&'\(\)*+,-./%
    0123456789:;<=>?%
    @ABCDEFGHIJKLMNO%
    PQRSTUVWXYZ[\\]^_%
    `abcdefghijklmno%
    pqrstuvwxyz{||}~\177%
    \200\201\202\203\204\205\206\207\210\211\212\213\214\215\216\217%
    \220\221\222\223\224\225\226\227\230\231\232\233\234\235\236\237%
    \240\241\242\243\244\245\246\247\250\251\252\253\254\255\256\257%
    \260\261\262\263\264\265\266\267\270\271\272\273\274\275\276\277%
    \300\301\302\303\304\305\306\307\310\311\312\313\314\315\316\317%
    \320\321\322\323\324\325\326\327\330\331\332\333\334\335\336\337%
    \340\341\342\343\344\345\346\347\350\351\352\353\354\355\356\357%
    \360\361\362\363\364\365\366\367\370\371\372\373\374\375\376\377%
  }%
|endgroup
\@onelevel@sanitize\AllBytesString

%<*test4>
\let\org@detokenize\detokenize
\let\detokenize\@undefined
\let\org@numexpr\numexpr
\let\numexpr\@undefined
%</test4>
\RequirePackage{pdfescape}
%<*test4>
\let\detokenize\org@detokenize
\let\numexpr\org@numexpr
%</test4>

\begin{qstest}{all-hex}{\AllBytes, escapehex}
  \EdefEscapeHex\x{\AllBytes}%
  \Expect*{\x}*{\AllBytesHex}%
  \ExpectVar\x\AllBytesHex
\end{qstest}

\begin{qstest}{all-unhex}{\AllBytesHex, unescapehex}
  \EdefUnescapeHex\x{\AllBytesHex}%
  \Expect*{\x}*{\AllBytes}%
  \ExpectVar\x\AllBytes
\end{qstest}

\begin{qstest}{all-unhex-lc}{\AllBytesHexLC, unescapehex, lowercase}
  \EdefUnescapeHex\x{\AllBytesHexLC}%
  \Expect*{\x}*{\AllBytes}%
  \ExpectVar\x\AllBytes
\end{qstest}

\begin{qstest}{unhex-incomplete}{unescapehex, incomplete}
  \EdefUnescapeHex\x{4}%
  \Expect*{\x}{@}%
\end{qstest}

\begin{qstest}{unhex-space}{unescapehex, space}
  \EdefUnescapeHex\x{20}%
  \Expect*{\x}{ }%
  \ExpectVar\x\space
\end{qstest}

\begin{qstest}{unhex-spaces}{unescapehex, spaces}
  \EdefUnescapeHex\x{204020204120}%
  \def\y#1{%
    \edef\z{#1\string @#1#1\string A#1}%
  }\y{ }%
  \Expect*{\x}*{\z}%
  \ExpectVar\x\z
\end{qstest}

\begin{qstest}{unhex-hash}{unescapehex, hash}
  \catcode`\#=12 %
  \EdefUnescapeHex\x{#20}%
  \ExpectVar\x\space
\end{qstest}

\begin{qstest}{unhex-invalid}{unescapehex, invalid}
  \def\test#1#2{%
    \EdefUnescapeHex\x{#1}%
    \edef\y{#2}%
    \@onelevel@sanitize\y
    \ExpectVar\x\y
  }%
%<*test2>
  \edef\x{\pdfunescapehex{4X}}%
  \edef\y{\string @}%
  \ifx\x\y
  \else
    \def~{\space}%
    \typeout{************************************}%
    \typeout{* Your pdfTeX contains bug 777.~~~~*}%
    \typeout{* This test is redefined as dummy, *}%
    \typeout{* because it triggers the bug.~~~~~*}%
    \typeout{************************************}%
    \def\test#1#2{}%
  \fi
%</test2>
  \test{X}{}%
  \test{XY}{}%
  \test{XYZ}{}%
  \test{A}{^^a0}%
  \test{AX}{^^a0}%
  \test{XA}{^^a0}%
  \test{XXAXX}{^^a0}%
\end{qstest}

\begin{qstest}{all-name}{\AllBytes, escapename}
  \EdefEscapeName\x{\AllBytes}%
  \Expect*{\x}*{\AllBytesName}%
  \ExpectVar\x\AllBytesName
\end{qstest}

\begin{qstest}{all-string}{\AllBytes, escapestring}
  \EdefEscapeString\x{\AllBytes}%
  \Expect*{\x}*{\AllBytesString}%
  \ExpectVar\x\AllBytesString
\end{qstest}

\begin{qstest}{uchexdigit}{unescape, uppercase hex digit}
  \catcode`\@=11 %
  \catcode0=12 %
  \def\test#1#2{%
    \uccode0=#1\relax
    \uppercase{%
      \def\x{^^@}%
    }%
    \Expect*{%
       \ifcase\expandafter\PE@TestUcHexDigit\x
         true%
       \else
         false%
       \fi
    }{#2}%
  }%
  \def\range#1#2#3{%
    \count0=#1\relax
    \loop
    \ifnum\count0<#2\relax
      \test{\count0}{#3}%
      \advance\count0 by 1 %
    \repeat
  }%
  \range{0}{47}{false}%
  \range{48}{57}{true}%
  \range{58}{64}{false}%
  \range{65}{70}{true}%
  \range{71}{255}{false}%
\end{qstest}

\begin{qstest}{unescapename}{unescapename}
  \def\test#1#2{%
    \EdefUnescapeName\x{#1}%
    \edef\y{#2}%
    \@onelevel@sanitize\y
    \ExpectVar\x\y
  }%
  \catcode`\#=12 %
  \catcode0=12 %
  \test{}{}%
  \test{x}{x}%
  \test{xy}{xy}%
  \test{#}{#}%
  \test{##}{##}%
  \test{###}{###}%
  \test{####}{####}%
  \test{#x}{#x}%
  \test{#xy}{#xy}%
  \test{#1}{#1}%
  \test{#40}{@}%
  \test{#400}{@0}%
  \test{#4x0}{#4x0}%
  \test{#ab}{^^ab}%
  \test{#00}{^^@}%
  \test{x#40y#40z}{x@y@z}%
  \test{#40#40#40#40}{@@@@}%
  \test{a#x}{a#x}%
  \test{a#xy}{a#xy}%
  \test{a#1}{a#1}%
  \test{a#40}{a@}%
  \test{a#400}{a@0}%
  \test{#20}{ }%
  \test{a#20}{a }%
  \test{a#20b}{a b}%
  \test{a#20#20#20b}{a \space\space b}%
\end{qstest}

\begin{qstest}{unescapestring}{unescapestring}
  \def\test#1#2{%
    \EdefUnescapeString\x{#1}%
    \edef\y{#2}%
    \@onelevel@sanitize\y
    \ExpectVar\x\y
  }%
  \catcode0=12 %
  \def\DefChar#1#2{%
    \begingroup
      \uccode0=#2\relax
    \uppercase{\endgroup
      \def#1{^^@}%
    }%
  }%
  \DefChar\nul{0}%
  \DefChar\one{1}%
  \DefChar\bel{8}%
  \DefChar\tab{9}%
  \DefChar\lf{10}%
  \DefChar\ff{12}%
  \DefChar\cr{13}%
  \DefChar\\{92}%
  \test{}{}%
  \test{a}{a}%
  \test{\\}{}%
  \test{\\\\}{\\}%
  \test{\\\\y}{\\y}%
  \test{\\000}{\nul}%
  \test{\\b}{\bel}%
  \test{\\t}{\tab}%
  \test{\\n}{\lf}%
  \test{\\f}{\ff}%
  \test{\\r}{\cr}%
  \test{\\(}{(}%
  \test{\\)}{)}%
  \test{\\040}{ }%
  \test{\\100}{@}%
  \test{\\40}{ }%
  \test{\\1}{\one}%
  \test{\\01}{\one}%
  \test{\\001}{\one}%
  \test{\\18}{\one8}%
  \test{\\018}{\one8}%
  \test{\\0018}{\one8}%
  \test{x\\}{x}%
  \test{x\\\\}{x\\}%
  \test{x\\\\y}{x\\y}%
  \test{x\\000}{x\nul}%
  \test{x\\b}{x\bel}%
  \test{x\\t}{x\tab}%
  \test{x\\n}{x\lf}%
  \test{x\\f}{x\ff}%
  \test{x\\r}{x\cr}%
  \test{x\\(}{x(}%
  \test{x\\)}{x)}%
  \test{x\\040}{x }%
  \test{x\\100}{x@}%
  \test{x\\40}{x }%
  \test{x\\1}{x\one}%
  \test{x\\01}{x\one}%
  \test{x\\001}{x\one}%
  \test{x\\18}{x\one8}%
  \test{x\\018}{x\one8}%
  \test{x\\0018}{x\one8}%
  \test{\\b\\t\\n\\f\\r\\(\\)\\\\\\000\\040}{%
    \bel\tab\lf\ff\cr()\\\nul\space
  }%
  \test{\\\lf}{}%
  \test{x\\\lf}{x}%
  \test{\cr}{\lf}%
  \test{\cr\lf}{\lf}%
  \test{\lf}{\lf}%
  \test{\lf\cr}{\lf\lf}%
  \test{x\cr}{x\lf}%
  \test{x\cr\lf}{x\lf}%
  \test{x\lf}{x\lf}%
  \test{x\lf\cr}{x\lf\lf}%
  \test{x\\\cr\lf y\cr}{xy\lf}%
%
  \test{\\409}{ 9}%
  \test{\\800}{800}%
  \test{\\900}{900}%
  \test{\\578}{/8}%
  \test{\\477}{?}%
  \test{\\377}{^^ff}%
  \test{\\777}{^^ff}%
  \test{\\7777}{^^ff7}%
\end{qstest}
\stop
%    \end{macrocode}
%
%    \begin{macrocode}
%</test2|test3|test4|test5>
%    \end{macrocode}
%
% \subsubsection{Test for \hologo{iniTeX}}
%
%    \begin{macrocode}
%<*test6>
\input pdfescape.sty\relax
\catcode`\{=1 %
\catcode`\}=2 %
\catcode`\#=6 %
\catcode`\^=7 %
\catcode`\@=11 %
%    \end{macrocode}
%    \begin{macrocode}
\begingroup
  \catcode`\@=11 %
  \countdef\count@=255 %
  \def\space{ }%
  \long\def\@whilenum#1\do #2{%
    \ifnum #1\relax
      #2\relax
      \@iwhilenum{#1\relax#2\relax}%
    \fi
  }%
  \long\def\@iwhilenum#1{%
    \ifnum #1%
      \expandafter\@iwhilenum
    \else
      \expandafter\ltx@gobble
    \fi
    {#1}%
  }%
  \gdef\AllBytes{}%
  \count@=0 %
  \catcode0=12 %
  \@whilenum\count@<256 \do{%
    \lccode0=\count@
    \ifnum\count@=32 %
      \xdef\AllBytes{\AllBytes\space}%
    \else
      \lowercase{%
        \xdef\AllBytes{\AllBytes^^@}%
      }%
    \fi
    \advance\count@ by 1 %
  }%
\endgroup
\def\AllBytesHex{%
  000102030405060708090A0B0C0D0E0F%
  101112131415161718191A1B1C1D1E1F%
  202122232425262728292A2B2C2D2E2F%
  303132333435363738393A3B3C3D3E3F%
  404142434445464748494A4B4C4D4E4F%
  505152535455565758595A5B5C5D5E5F%
  606162636465666768696A6B6C6D6E6F%
  707172737475767778797A7B7C7D7E7F%
  808182838485868788898A8B8C8D8E8F%
  909192939495969798999A9B9C9D9E9F%
  A0A1A2A3A4A5A6A7A8A9AAABACADAEAF%
  B0B1B2B3B4B5B6B7B8B9BABBBCBDBEBF%
  C0C1C2C3C4C5C6C7C8C9CACBCCCDCECF%
  D0D1D2D3D4D5D6D7D8D9DADBDCDDDEDF%
  E0E1E2E3E4E5E6E7E8E9EAEBECEDEEEF%
  F0F1F2F3F4F5F6F7F8F9FAFBFCFDFEFF%
}
\ltx@onelevel@sanitize\AllBytesHex
\expandafter\lowercase\expandafter{%
  \expandafter\def\expandafter\AllBytesHexLC
      \expandafter{\AllBytesHex}%
}
\begingroup
  \catcode`\#=12 %
  \xdef\AllBytesName{%
    #01#02#03#04#05#06#07#08#09#0A#0B#0C#0D#0E#0F%
    #10#11#12#13#14#15#16#17#18#19#1A#1B#1C#1D#1E#1F%
    #20!"#23$#25&'#28#29*+,-.#2F%
    0123456789:;#3C=#3E?%
    @ABCDEFGHIJKLMNO%
    PQRSTUVWXYZ#5B\ltx@backslashchar#5D^_%
    `abcdefghijklmno%
    pqrstuvwxyz#7B|#7D\string~#7F%
    #80#81#82#83#84#85#86#87#88#89#8A#8B#8C#8D#8E#8F%
    #90#91#92#93#94#95#96#97#98#99#9A#9B#9C#9D#9E#9F%
    #A0#A1#A2#A3#A4#A5#A6#A7#A8#A9#AA#AB#AC#AD#AE#AF%
    #B0#B1#B2#B3#B4#B5#B6#B7#B8#B9#BA#BB#BC#BD#BE#BF%
    #C0#C1#C2#C3#C4#C5#C6#C7#C8#C9#CA#CB#CC#CD#CE#CF%
    #D0#D1#D2#D3#D4#D5#D6#D7#D8#D9#DA#DB#DC#DD#DE#DF%
    #E0#E1#E2#E3#E4#E5#E6#E7#E8#E9#EA#EB#EC#ED#EE#EF%
    #F0#F1#F2#F3#F4#F5#F6#F7#F8#F9#FA#FB#FC#FD#FE#FF%
  }%
\endgroup
\ltx@onelevel@sanitize\AllBytesName
\edef\AllBytesFromName{\expandafter\ltx@gobble\AllBytes}

\begingroup
  \def\|{|}%
  \edef\%{\ltx@percentchar}%
  \catcode`\|=0 %
  \catcode`\#=12 %
  \catcode`\~=12 %
  \catcode`\\=12 %
  |xdef|AllBytesString{%
    \000\001\002\003\004\005\006\007\010\011\012\013\014\015\016\017%
    \020\021\022\023\024\025\026\027\030\031\032\033\034\035\036\037%
    \040!"#$|%&'\(\)*+,-./%
    0123456789:;<=>?%
    @ABCDEFGHIJKLMNO%
    PQRSTUVWXYZ[\\]^_%
    `abcdefghijklmno%
    pqrstuvwxyz{||}~\177%
    \200\201\202\203\204\205\206\207\210\211\212\213\214\215\216\217%
    \220\221\222\223\224\225\226\227\230\231\232\233\234\235\236\237%
    \240\241\242\243\244\245\246\247\250\251\252\253\254\255\256\257%
    \260\261\262\263\264\265\266\267\270\271\272\273\274\275\276\277%
    \300\301\302\303\304\305\306\307\310\311\312\313\314\315\316\317%
    \320\321\322\323\324\325\326\327\330\331\332\333\334\335\336\337%
    \340\341\342\343\344\345\346\347\350\351\352\353\354\355\356\357%
    \360\361\362\363\364\365\366\367\370\371\372\373\374\375\376\377%
  }%
|endgroup
\ltx@onelevel@sanitize\AllBytesString
%    \end{macrocode}
%    \begin{macrocode}
\def\msg#{\immediate\write16}
\def\Test#1#2#3{%
  \begingroup
    #1\TestResult{#2}%
    \ifx\TestResult#3%
    \else
      \newlinechar=10 %
      \msg{Expect:^^J#3}%
      \msg{Result:^^J\TestResult}%
      \errmessage{\string#2 -\string#1-> \string#3}%
    \fi
  \endgroup
}
\Test\EdefEscapeHex\AllBytes\AllBytesHex
\Test\EdefUnescapeHex\AllBytesHex\AllBytes
\Test\EdefEscapeName\AllBytes\AllBytesName
\Test\EdefUnescapeName\AllBytesName\AllBytesFromName
\Test\EdefEscapeString\AllBytes\AllBytesString
\Test\EdefUnescapeString\AllBytesString\AllBytes
%    \end{macrocode}
%    \begin{macrocode}
\csname @@end\endcsname\end
%</test6>
%    \end{macrocode}
%
% \section{Installation}
%
% \subsection{Download}
%
% \paragraph{Package.} This package is available on
% CTAN\footnote{\url{ftp://ftp.ctan.org/tex-archive/}}:
% \begin{description}
% \item[\CTAN{macros/latex/contrib/oberdiek/pdfescape.dtx}] The source file.
% \item[\CTAN{macros/latex/contrib/oberdiek/pdfescape.pdf}] Documentation.
% \end{description}
%
%
% \paragraph{Bundle.} All the packages of the bundle `oberdiek'
% are also available in a TDS compliant ZIP archive. There
% the packages are already unpacked and the documentation files
% are generated. The files and directories obey the TDS standard.
% \begin{description}
% \item[\CTAN{install/macros/latex/contrib/oberdiek.tds.zip}]
% \end{description}
% \emph{TDS} refers to the standard ``A Directory Structure
% for \TeX\ Files'' (\CTAN{tds/tds.pdf}). Directories
% with \xfile{texmf} in their name are usually organized this way.
%
% \subsection{Bundle installation}
%
% \paragraph{Unpacking.} Unpack the \xfile{oberdiek.tds.zip} in the
% TDS tree (also known as \xfile{texmf} tree) of your choice.
% Example (linux):
% \begin{quote}
%   |unzip oberdiek.tds.zip -d ~/texmf|
% \end{quote}
%
% \paragraph{Script installation.}
% Check the directory \xfile{TDS:scripts/oberdiek/} for
% scripts that need further installation steps.
% Package \xpackage{attachfile2} comes with the Perl script
% \xfile{pdfatfi.pl} that should be installed in such a way
% that it can be called as \texttt{pdfatfi}.
% Example (linux):
% \begin{quote}
%   |chmod +x scripts/oberdiek/pdfatfi.pl|\\
%   |cp scripts/oberdiek/pdfatfi.pl /usr/local/bin/|
% \end{quote}
%
% \subsection{Package installation}
%
% \paragraph{Unpacking.} The \xfile{.dtx} file is a self-extracting
% \docstrip\ archive. The files are extracted by running the
% \xfile{.dtx} through \plainTeX:
% \begin{quote}
%   \verb|tex pdfescape.dtx|
% \end{quote}
%
% \paragraph{TDS.} Now the different files must be moved into
% the different directories in your installation TDS tree
% (also known as \xfile{texmf} tree):
% \begin{quote}
% \def\t{^^A
% \begin{tabular}{@{}>{\ttfamily}l@{ $\rightarrow$ }>{\ttfamily}l@{}}
%   pdfescape.sty & tex/generic/oberdiek/pdfescape.sty\\
%   pdfescape.pdf & doc/latex/oberdiek/pdfescape.pdf\\
%   test/pdfescape-test1.tex & doc/latex/oberdiek/test/pdfescape-test1.tex\\
%   test/pdfescape-test2.tex & doc/latex/oberdiek/test/pdfescape-test2.tex\\
%   test/pdfescape-test3.tex & doc/latex/oberdiek/test/pdfescape-test3.tex\\
%   test/pdfescape-test4.tex & doc/latex/oberdiek/test/pdfescape-test4.tex\\
%   test/pdfescape-test5.tex & doc/latex/oberdiek/test/pdfescape-test5.tex\\
%   test/pdfescape-test6.tex & doc/latex/oberdiek/test/pdfescape-test6.tex\\
%   pdfescape.dtx & source/latex/oberdiek/pdfescape.dtx\\
% \end{tabular}^^A
% }^^A
% \sbox0{\t}^^A
% \ifdim\wd0>\linewidth
%   \begingroup
%     \advance\linewidth by\leftmargin
%     \advance\linewidth by\rightmargin
%   \edef\x{\endgroup
%     \def\noexpand\lw{\the\linewidth}^^A
%   }\x
%   \def\lwbox{^^A
%     \leavevmode
%     \hbox to \linewidth{^^A
%       \kern-\leftmargin\relax
%       \hss
%       \usebox0
%       \hss
%       \kern-\rightmargin\relax
%     }^^A
%   }^^A
%   \ifdim\wd0>\lw
%     \sbox0{\small\t}^^A
%     \ifdim\wd0>\linewidth
%       \ifdim\wd0>\lw
%         \sbox0{\footnotesize\t}^^A
%         \ifdim\wd0>\linewidth
%           \ifdim\wd0>\lw
%             \sbox0{\scriptsize\t}^^A
%             \ifdim\wd0>\linewidth
%               \ifdim\wd0>\lw
%                 \sbox0{\tiny\t}^^A
%                 \ifdim\wd0>\linewidth
%                   \lwbox
%                 \else
%                   \usebox0
%                 \fi
%               \else
%                 \lwbox
%               \fi
%             \else
%               \usebox0
%             \fi
%           \else
%             \lwbox
%           \fi
%         \else
%           \usebox0
%         \fi
%       \else
%         \lwbox
%       \fi
%     \else
%       \usebox0
%     \fi
%   \else
%     \lwbox
%   \fi
% \else
%   \usebox0
% \fi
% \end{quote}
% If you have a \xfile{docstrip.cfg} that configures and enables \docstrip's
% TDS installing feature, then some files can already be in the right
% place, see the documentation of \docstrip.
%
% \subsection{Refresh file name databases}
%
% If your \TeX~distribution
% (\teTeX, \mikTeX, \dots) relies on file name databases, you must refresh
% these. For example, \teTeX\ users run \verb|texhash| or
% \verb|mktexlsr|.
%
% \subsection{Some details for the interested}
%
% \paragraph{Attached source.}
%
% The PDF documentation on CTAN also includes the
% \xfile{.dtx} source file. It can be extracted by
% AcrobatReader 6 or higher. Another option is \textsf{pdftk},
% e.g. unpack the file into the current directory:
% \begin{quote}
%   \verb|pdftk pdfescape.pdf unpack_files output .|
% \end{quote}
%
% \paragraph{Unpacking with \LaTeX.}
% The \xfile{.dtx} chooses its action depending on the format:
% \begin{description}
% \item[\plainTeX:] Run \docstrip\ and extract the files.
% \item[\LaTeX:] Generate the documentation.
% \end{description}
% If you insist on using \LaTeX\ for \docstrip\ (really,
% \docstrip\ does not need \LaTeX), then inform the autodetect routine
% about your intention:
% \begin{quote}
%   \verb|latex \let\install=y\input{pdfescape.dtx}|
% \end{quote}
% Do not forget to quote the argument according to the demands
% of your shell.
%
% \paragraph{Generating the documentation.}
% You can use both the \xfile{.dtx} or the \xfile{.drv} to generate
% the documentation. The process can be configured by the
% configuration file \xfile{ltxdoc.cfg}. For instance, put this
% line into this file, if you want to have A4 as paper format:
% \begin{quote}
%   \verb|\PassOptionsToClass{a4paper}{article}|
% \end{quote}
% An example follows how to generate the
% documentation with pdf\LaTeX:
% \begin{quote}
%\begin{verbatim}
%pdflatex pdfescape.dtx
%makeindex -s gind.ist pdfescape.idx
%pdflatex pdfescape.dtx
%makeindex -s gind.ist pdfescape.idx
%pdflatex pdfescape.dtx
%\end{verbatim}
% \end{quote}
%
% \section{Catalogue}
%
% The following XML file can be used as source for the
% \href{http://mirror.ctan.org/help/Catalogue/catalogue.html}{\TeX\ Catalogue}.
% The elements \texttt{caption} and \texttt{description} are imported
% from the original XML file from the Catalogue.
% The name of the XML file in the Catalogue is \xfile{pdfescape.xml}.
%    \begin{macrocode}
%<*catalogue>
<?xml version='1.0' encoding='us-ascii'?>
<!DOCTYPE entry SYSTEM 'catalogue.dtd'>
<entry datestamp='$Date$' modifier='$Author$' id='pdfescape'>
  <name>pdfescape</name>
  <caption>Implements pdfTeX's escape features using TeX or e-TeX.</caption>
  <authorref id='auth:oberdiek'/>
  <copyright owner='Heiko Oberdiek' year='2007,2010,2011'/>
  <license type='lppl1.3'/>
  <version number='1.13'/>
  <description>
    This package implements <xref refid='pdftex'>pdfTeX</xref>'s
    escape features (<tt>\pdfescapehex</tt>, <tt>\pdfunescapehex</tt>,
    <tt>\pdfescapename</tt>, <tt>\pdfescapestring</tt>) using TeX or
    e-TeX.
    <p/>
    The package is part of the <xref refid='oberdiek'>oberdiek</xref>
    bundle.
  </description>
  <documentation details='Package documentation'
      href='ctan:/macros/latex/contrib/oberdiek/pdfescape.pdf'/>
  <ctan file='true' path='/macros/latex/contrib/oberdiek/pdfescape.dtx'/>
  <miktex location='oberdiek'/>
  <texlive location='oberdiek'/>
  <install path='/macros/latex/contrib/oberdiek/oberdiek.tds.zip'/>
</entry>
%</catalogue>
%    \end{macrocode}
%
% \begin{History}
%   \begin{Version}{2007/02/21 v1.0}
%   \item
%     First version.
%   \end{Version}
%   \begin{Version}{2007/02/25 v1.1}
%   \item
%     Test files added.
%   \item
%     \cs{EdefUnescapeHex} supports lowercase letters.
%   \item
%     Fix: \verb|\EdefEscapeName{^^@}|
%   \item
%     Fix: \verb|\EdefEscapeName{\string#}|
%   \item
%     Fix for \cs{EdefUnescapeHex} in case of incomplete hex string.
%   \item
%     Fix: \cs{EdefUnescapeHex} generates space tokens with catcode
%     10 (space) in all cases.
%   \item
%     Fix: \cs{EdefEscapeHex} and \cs{EdefEscapeName} now generate
%     tokens with catcode 12 (other) only.
%   \end{Version}
%   \begin{Version}{2007/03/20 v1.2}
%   \item
%     Fix: Wrong year in \cs{ProvidesPackage}.
%   \end{Version}
%   \begin{Version}{2007/04/11 v1.3}
%   \item
%     Line ends sanitized.
%   \end{Version}
%   \begin{Version}{2007/04/21 v1.4}
%   \item
%     \cs{EdefUnescapeName} and \cs{EdefUnescapeString} added.
%   \end{Version}
%   \begin{Version}{2007/08/27 v1.5}
%   \item
%     \cs{EdefSanitize} added (replaces \cs{PE@sanitize}).
%   \end{Version}
%   \begin{Version}{2007/09/09 v1.6}
%   \item
%     Fix in catcode setup.
%   \end{Version}
%   \begin{Version}{2007/10/27 v1.7}
%   \item
%     More efficient \cs{EdefSanitize}.
%   \end{Version}
%   \begin{Version}{2007/11/11 v1.8}
%   \item
%     Use of package \xpackage{pdftexcmds} for \hologo{LuaTeX} support.
%   \end{Version}
%   \begin{Version}{2010/03/01 v1.9}
%   \item
%     Compatibility with \hologo{iniTeX}.
%   \end{Version}
%   \begin{Version}{2010/11/12 v1.10}
%   \item
%     Use of package \xpackage{ltxcmds}.
%   \item
%     Fix for compatibility with \hologo{iniTeX}.
%   \end{Version}
%   \begin{Version}{2011/01/30 v1.11}
%   \item
%     Already loaded package files are not input in \hologo{plainTeX}.
%   \end{Version}
%   \begin{Version}{2011/04/04 v1.12}
%   \item
%     Further fixes for compatibility for \hologo{iniTeX}.
%   \item
%     Test file for \hologo{iniTeX} added.
%   \end{Version}
%   \begin{Version}{2011/11/25 v1.13}
%   \item
%     Higher order bit of octal sequences in \cs{EdefUnescapeString}
%     ignored according to the PDF specification
%     (Bug found by Bruno Le Floch).
%   \end{Version}
% \end{History}
%
% \PrintIndex
%
% \Finale
\endinput

%        (quote the arguments according to the demands of your shell)
%
% Documentation:
%    (a) If pdfescape.drv is present:
%           latex pdfescape.drv
%    (b) Without pdfescape.drv:
%           latex pdfescape.dtx; ...
%    The class ltxdoc loads the configuration file ltxdoc.cfg
%    if available. Here you can specify further options, e.g.
%    use A4 as paper format:
%       \PassOptionsToClass{a4paper}{article}
%
%    Programm calls to get the documentation (example):
%       pdflatex pdfescape.dtx
%       makeindex -s gind.ist pdfescape.idx
%       pdflatex pdfescape.dtx
%       makeindex -s gind.ist pdfescape.idx
%       pdflatex pdfescape.dtx
%
% Installation:
%    TDS:tex/generic/oberdiek/pdfescape.sty
%    TDS:doc/latex/oberdiek/pdfescape.pdf
%    TDS:doc/latex/oberdiek/test/pdfescape-test1.tex
%    TDS:doc/latex/oberdiek/test/pdfescape-test2.tex
%    TDS:doc/latex/oberdiek/test/pdfescape-test3.tex
%    TDS:doc/latex/oberdiek/test/pdfescape-test4.tex
%    TDS:doc/latex/oberdiek/test/pdfescape-test5.tex
%    TDS:doc/latex/oberdiek/test/pdfescape-test6.tex
%    TDS:source/latex/oberdiek/pdfescape.dtx
%
%<*ignore>
\begingroup
  \catcode123=1 %
  \catcode125=2 %
  \def\x{LaTeX2e}%
\expandafter\endgroup
\ifcase 0\ifx\install y1\fi\expandafter
         \ifx\csname processbatchFile\endcsname\relax\else1\fi
         \ifx\fmtname\x\else 1\fi\relax
\else\csname fi\endcsname
%</ignore>
%<*install>
\input docstrip.tex
\Msg{************************************************************************}
\Msg{* Installation}
\Msg{* Package: pdfescape 2011/11/25 v1.13 Implements pdfTeX's escape features (HO)}
\Msg{************************************************************************}

\keepsilent
\askforoverwritefalse

\let\MetaPrefix\relax
\preamble

This is a generated file.

Project: pdfescape
Version: 2011/11/25 v1.13

Copyright (C) 2007, 2010, 2011 by
   Heiko Oberdiek <heiko.oberdiek at googlemail.com>

This work may be distributed and/or modified under the
conditions of the LaTeX Project Public License, either
version 1.3c of this license or (at your option) any later
version. This version of this license is in
   http://www.latex-project.org/lppl/lppl-1-3c.txt
and the latest version of this license is in
   http://www.latex-project.org/lppl.txt
and version 1.3 or later is part of all distributions of
LaTeX version 2005/12/01 or later.

This work has the LPPL maintenance status "maintained".

This Current Maintainer of this work is Heiko Oberdiek.

The Base Interpreter refers to any `TeX-Format',
because some files are installed in TDS:tex/generic//.

This work consists of the main source file pdfescape.dtx
and the derived files
   pdfescape.sty, pdfescape.pdf, pdfescape.ins, pdfescape.drv,
   pdfescape-test1.tex, pdfescape-test2.tex,
   pdfescape-test3.tex, pdfescape-test4.tex,
   pdfescape-test5.tex, pdfescape-test6.tex.

\endpreamble
\let\MetaPrefix\DoubleperCent

\generate{%
  \file{pdfescape.ins}{\from{pdfescape.dtx}{install}}%
  \file{pdfescape.drv}{\from{pdfescape.dtx}{driver}}%
  \usedir{tex/generic/oberdiek}%
  \file{pdfescape.sty}{\from{pdfescape.dtx}{package}}%
  \usedir{doc/latex/oberdiek/test}%
  \file{pdfescape-test1.tex}{\from{pdfescape.dtx}{test1}}%
  \file{pdfescape-test2.tex}{\from{pdfescape.dtx}{test2}}%
  \file{pdfescape-test3.tex}{\from{pdfescape.dtx}{test3}}%
  \file{pdfescape-test4.tex}{\from{pdfescape.dtx}{test4}}%
  \file{pdfescape-test5.tex}{\from{pdfescape.dtx}{test5}}%
  \file{pdfescape-test6.tex}{\from{pdfescape.dtx}{test6}}%
  \nopreamble
  \nopostamble
  \usedir{source/latex/oberdiek/catalogue}%
  \file{pdfescape.xml}{\from{pdfescape.dtx}{catalogue}}%
}

\catcode32=13\relax% active space
\let =\space%
\Msg{************************************************************************}
\Msg{*}
\Msg{* To finish the installation you have to move the following}
\Msg{* file into a directory searched by TeX:}
\Msg{*}
\Msg{*     pdfescape.sty}
\Msg{*}
\Msg{* To produce the documentation run the file `pdfescape.drv'}
\Msg{* through LaTeX.}
\Msg{*}
\Msg{* Happy TeXing!}
\Msg{*}
\Msg{************************************************************************}

\endbatchfile
%</install>
%<*ignore>
\fi
%</ignore>
%<*driver>
\NeedsTeXFormat{LaTeX2e}
\ProvidesFile{pdfescape.drv}%
  [2011/11/25 v1.13 Implements pdfTeX's escape features (HO)]%
\documentclass{ltxdoc}
\usepackage{holtxdoc}[2011/11/22]
\begin{document}
  \DocInput{pdfescape.dtx}%
\end{document}
%</driver>
% \fi
%
% \CheckSum{2516}
%
% \CharacterTable
%  {Upper-case    \A\B\C\D\E\F\G\H\I\J\K\L\M\N\O\P\Q\R\S\T\U\V\W\X\Y\Z
%   Lower-case    \a\b\c\d\e\f\g\h\i\j\k\l\m\n\o\p\q\r\s\t\u\v\w\x\y\z
%   Digits        \0\1\2\3\4\5\6\7\8\9
%   Exclamation   \!     Double quote  \"     Hash (number) \#
%   Dollar        \$     Percent       \%     Ampersand     \&
%   Acute accent  \'     Left paren    \(     Right paren   \)
%   Asterisk      \*     Plus          \+     Comma         \,
%   Minus         \-     Point         \.     Solidus       \/
%   Colon         \:     Semicolon     \;     Less than     \<
%   Equals        \=     Greater than  \>     Question mark \?
%   Commercial at \@     Left bracket  \[     Backslash     \\
%   Right bracket \]     Circumflex    \^     Underscore    \_
%   Grave accent  \`     Left brace    \{     Vertical bar  \|
%   Right brace   \}     Tilde         \~}
%
% \GetFileInfo{pdfescape.drv}
%
% \title{The \xpackage{pdfescape} package}
% \date{2011/11/25 v1.13}
% \author{Heiko Oberdiek\\\xemail{heiko.oberdiek at googlemail.com}}
%
% \maketitle
%
% \begin{abstract}
% This package implements \pdfTeX's escape features
% (\cs{pdfescapehex}, \cs{pdfunescapehex}, \cs{pdfescapename},
% \cs{pdfescapestring}) using \TeX\ or \eTeX.
% \end{abstract}
%
% \tableofcontents
%
% \section{Documentation}
%
% \begin{declcs}{EdefEscapeHex} \M{cmd} \M{string}\\
%  \cs{EdefUnescapeHex} \M{cmd} \M{string}\\
%  \cs{EdefEscapeName} \M{cmd} \M{string}\\
%  \cs{EdefEscapeString} \M{cmd} \M{string}
% \end{declcs}
% These commands converts \meta{string} and stores the result
% in macro \meta{cmd}. The conversion result is the same
% as the conversion of the corresponding \pdfTeX's primitives.
% Note that the argument \meta{string} is expanded before
% the conversion.
%
% For example, if \pdfTeX\ \textgreater= 1.30 is present,
% then \cs{EdefEscapeHex}
% becomes to:
% \begin{quote}
%\begin{verbatim}
%\def\EdefEscapeHex#1#2{%
%  \edef#1{\pdfescapehex{#2}}%
%}
%\end{verbatim}
% \end{quote}
% The package provides implementations for the case that \pdfTeX\ is
% not present (or too old). Even \eTeX\ can be missing, however it is
% used if it is detected.
%
% \paragraph{Babel.}
% The input strings may contain shorthand characters of package
% \xpackage{babel}.
%
% \subsection{Additional unescape macros}
%
% \begin{declcs}{EdefUnescapeName} \M{cmd} \M{string}
% \end{declcs}
% Sequences of a hash sign with two hexadecimal digits are
% converted to the corresponding character (PDF-1.2).
% A hash sign that is not followed by two hexadecimal digits
% is left unchanged. The catcodes in the result string follow
% TeX's conventions. The space has catcode 10 (space) and
% the other characters have catcode 12 (other).
%
% \begin{declcs}{EdefUnescapeString} \M{cmd} \M{string}
% \end{declcs}
% Macro \meta{cmd} stores the unescaped string in \meta{string}.
% All the rules for literal strings are implemented, see
% PDF specification.
% The catcodes in the result string follow TeX's conventions.
%
% \subsection{Sanitizing macro}
%
% \begin{declcs}{EdefSanitize} \M{cmd} \M{string}
% \end{declcs}
% Argument \meta{string} is expanded, converted to a string
% of tokens with catcode 12 (other) and space tokens, and
% stored in macro \meta{cmd}.
%
% \StopEventually{
% }
%
% \section{Implementation}
%
%    \begin{macrocode}
%<*package>
%    \end{macrocode}
%
% \subsection{Reload check and package identification}
%    Reload check, especially if the package is not used with \LaTeX.
%    \begin{macrocode}
\begingroup\catcode61\catcode48\catcode32=10\relax%
  \catcode13=5 % ^^M
  \endlinechar=13 %
  \catcode35=6 % #
  \catcode39=12 % '
  \catcode44=12 % ,
  \catcode45=12 % -
  \catcode46=12 % .
  \catcode58=12 % :
  \catcode64=11 % @
  \catcode123=1 % {
  \catcode125=2 % }
  \expandafter\let\expandafter\x\csname ver@pdfescape.sty\endcsname
  \ifx\x\relax % plain-TeX, first loading
  \else
    \def\empty{}%
    \ifx\x\empty % LaTeX, first loading,
      % variable is initialized, but \ProvidesPackage not yet seen
    \else
      \expandafter\ifx\csname PackageInfo\endcsname\relax
        \def\x#1#2{%
          \immediate\write-1{Package #1 Info: #2.}%
        }%
      \else
        \def\x#1#2{\PackageInfo{#1}{#2, stopped}}%
      \fi
      \x{pdfescape}{The package is already loaded}%
      \aftergroup\endinput
    \fi
  \fi
\endgroup%
%    \end{macrocode}
%    Package identification:
%    \begin{macrocode}
\begingroup\catcode61\catcode48\catcode32=10\relax%
  \catcode13=5 % ^^M
  \endlinechar=13 %
  \catcode35=6 % #
  \catcode39=12 % '
  \catcode40=12 % (
  \catcode41=12 % )
  \catcode44=12 % ,
  \catcode45=12 % -
  \catcode46=12 % .
  \catcode47=12 % /
  \catcode58=12 % :
  \catcode64=11 % @
  \catcode91=12 % [
  \catcode93=12 % ]
  \catcode123=1 % {
  \catcode125=2 % }
  \expandafter\ifx\csname ProvidesPackage\endcsname\relax
    \def\x#1#2#3[#4]{\endgroup
      \immediate\write-1{Package: #3 #4}%
      \xdef#1{#4}%
    }%
  \else
    \def\x#1#2[#3]{\endgroup
      #2[{#3}]%
      \ifx#1\@undefined
        \xdef#1{#3}%
      \fi
      \ifx#1\relax
        \xdef#1{#3}%
      \fi
    }%
  \fi
\expandafter\x\csname ver@pdfescape.sty\endcsname
\ProvidesPackage{pdfescape}%
  [2011/11/25 v1.13 Implements pdfTeX's escape features (HO)]%
%    \end{macrocode}
%
% \subsection{Catcodes}
%
%    \begin{macrocode}
\begingroup\catcode61\catcode48\catcode32=10\relax%
  \catcode13=5 % ^^M
  \endlinechar=13 %
  \catcode123 1 % {
  \catcode125 2 % }
  \catcode64 11 %
  \def\x{\endgroup
    \expandafter\edef\csname PE@AtEnd\endcsname{%
      \endlinechar=\the\endlinechar\relax
      \catcode13=\the\catcode13\relax
      \catcode32=\the\catcode32\relax
      \catcode35=\the\catcode35\relax
      \catcode61=\the\catcode61\relax
      \catcode64=\the\catcode64\relax
      \catcode123=\the\catcode123\relax
      \catcode125=\the\catcode125\relax
    }%
  }%
\x\catcode61\catcode48\catcode32=10\relax%
\catcode13=5 % ^^M
\endlinechar=13 %
\catcode35=6 % #
\catcode64=11 % @
\catcode123=1 % {
\catcode125=2 % }
\def\TMP@EnsureCode#1#2#3{%
  \edef\PE@AtEnd{%
    \PE@AtEnd
    #1#2=\the#1#2\relax
  }%
  #1#2=#3\relax
}
\TMP@EnsureCode\catcode{0}{12}% ^^@
\TMP@EnsureCode\catcode{34}{12}% "
\TMP@EnsureCode\catcode{36}{3}% $
\TMP@EnsureCode\catcode{38}{4}% &
\TMP@EnsureCode\catcode{39}{12}% '
\TMP@EnsureCode\catcode{42}{12}% *
\TMP@EnsureCode\catcode{45}{12}% -
\TMP@EnsureCode\catcode{46}{12}% .
\TMP@EnsureCode\catcode{47}{12}% /
\TMP@EnsureCode\catcode{60}{12}% <
\TMP@EnsureCode\catcode{62}{12}% >
\TMP@EnsureCode\catcode{91}{12}% [
\TMP@EnsureCode\catcode{93}{12}% ]
\TMP@EnsureCode\catcode{94}{7}% ^
\TMP@EnsureCode\catcode{96}{12}% `
\TMP@EnsureCode\uccode{34}{0}% "
\TMP@EnsureCode\uccode{48}{0}% 0
\TMP@EnsureCode\uccode{61}{0}% =
\edef\PE@AtEnd{\PE@AtEnd\noexpand\endinput}
%    \end{macrocode}
%
% \subsection{Load package}
%
%    \begin{macrocode}
\begingroup\expandafter\expandafter\expandafter\endgroup
\expandafter\ifx\csname RequirePackage\endcsname\relax
  \def\TMP@RequirePackage#1[#2]{%
    \begingroup\expandafter\expandafter\expandafter\endgroup
    \expandafter\ifx\csname ver@#1.sty\endcsname\relax
      \input #1.sty\relax
    \fi
  }%
  \TMP@RequirePackage{ltxcmds}[2010/04/08]%
\else
  \RequirePackage{ltxcmds}[2010/04/08]%
\fi
%    \end{macrocode}
%
% \subsection{Sanitizing}
%
%    \begin{macro}{\EdefSanitize}
%    Macro \cs{EdefSanitize} takes |#2|, entirely converts
%    it to token with catcode 12 (other) and stores the
%    result in macro |#1|.
%    \begin{macrocode}
\begingroup\expandafter\expandafter\expandafter\endgroup
\expandafter\ifx\csname detokenize\endcsname\relax
  \long\def\EdefSanitize#1#2{%
    \begingroup
      \csname @safe@activestrue\endcsname
      \edef#1{#2}%
      \PE@onelevel@sanitize#1%
    \expandafter\endgroup
    \expandafter\def\expandafter#1\expandafter{#1}%
  }%
  \begingroup\expandafter\expandafter\expandafter\endgroup
  \expandafter\ifx\csname @onelevel@sanitize\endcsname\relax
    \def\PE@onelevel@sanitize#1{%
      \edef#1{\expandafter\PE@strip@prefix\meaning#1}%
    }%
    \def\PE@strip@prefix#1>{}%
  \else
    \let\PE@onelevel@sanitize\@onelevel@sanitize
  \fi
\else
  \long\def\EdefSanitize#1#2{%
    \begingroup
      \csname @safe@activestrue\endcsname
      \edef#1{#2}%
    \expandafter\endgroup
    \expandafter\def\expandafter#1\expandafter{%
      \detokenize\expandafter{#1}%
    }%
  }%
  \def\PE@onelevel@sanitize#1{%
    \edef#1{\detokenize\expandafter{#1}}%
  }%
\fi
%    \end{macrocode}
%    \end{macro}
%    \begin{macro}{\PE@sanitize}
%    Macro \cs{PE@sanitize} is only defined for
%    compatibility with version 1.4. Its use is deprecated.
%    \begin{macrocode}
\let\PE@sanitize\EdefSanitize
%    \end{macrocode}
%    \end{macro}
%
% \subsubsection{Space characters}
%
%    \begin{macro}{\PE@space@other}
%    \begin{macrocode}
\begingroup
  \catcode`\ =12\relax%
\def\x{\endgroup\def\PE@space@other{ }}\x\relax
%    \end{macrocode}
%    \end{macro}
%    \begin{macro}{\PE@space@space}
%    \begin{macrocode}
\def\PE@space@space{ }
%    \end{macrocode}
%    \end{macro}
%
% \subsubsection{Space normalization}
%
%    \begin{macro}{\PE@SanitizeSpaceOther}
%    \begin{macrocode}
\def\PE@SanitizeSpaceOther#1{%
  \edef#1{\expandafter\PE@SpaceToOther#1 \relax}%
}
%    \end{macrocode}
%    \end{macro}
%
%    \begin{macro}{\PE@SpaceToOther}
%    \begin{macrocode}
\def\PE@SpaceToOther#1 #2\relax{%
  #1%
  \ifx\\#2\\%
  \else
    \PE@space@other
    \ltx@ReturnAfterFi{%
      \PE@SpaceToOther#2\relax
    }%
  \fi
}
%    \end{macrocode}
%    \end{macro}
%
% \subsection{\cs{EdefUnescapeName}}
%
%    \begin{macro}{\EdefUnescapeName}
%    \begin{macrocode}
\def\EdefUnescapeName#1#2{%
  \EdefSanitize#1{#2}%
  \PE@SanitizeSpaceOther#1%
  \PE@UnescapeName#1%
  \PE@onelevel@sanitize#1%
}
%    \end{macrocode}
%    \end{macro}
%
%    \begin{macro}{\PE@UnescapeName}
%    \begin{macrocode}
\begingroup
  \catcode`\$=6  % hash
  \catcode`\#=12 % other
  \gdef\PE@UnescapeName$1{%
    \begingroup
      \PE@InitUccodeHexDigit
      \def\PE@result{}%
      \expandafter\PE@DeName$1#\relax\relax
    \expandafter\endgroup
    \expandafter\def\expandafter$1\expandafter{\PE@result}%
  }%
  \gdef\PE@DeName$1#$2$3{%
    \ifx\relax$2%
      \edef\PE@result{\PE@result$1}%
      \let\PE@next\relax
    \else
      \ifx\relax$3%
        % wrong escape sequence in input
        \edef\PE@result{\PE@result$1#}%
        \let\PE@next\relax
      \else
        \uppercase{%
          \def\PE@testA{$2}%
          \def\PE@testB{$3}%
        }%
        \ifcase\ifcase\expandafter\PE@TestUcHexDigit\PE@testA
                 \ifcase\expandafter\PE@TestUcHexDigit\PE@testB
                   \ltx@zero
                 \else
                   \ltx@one
                 \fi
               \else
                 \ltx@one
               \fi
          \uccode\ltx@zero="\PE@testA\PE@testB\relax
          \uppercase{%
            \def\PE@temp{^^@}%
          }%
          \uccode\ltx@zero=\ltx@zero
          \edef\PE@result{\PE@result$1\PE@temp}%
          \let\PE@next\PE@DeName
        \else
          % wrong escape sequence in input
          \edef\PE@result{\PE@result$1#}%
          \def\PE@next{\PE@DeName$2$3}%
        \fi
      \fi
    \fi
    \PE@next
  }%
\endgroup
%    \end{macrocode}
%    \end{macro}
%
%    \begin{macro}{\PE@InitUccodeHexDigit}
%    \begin{macrocode}
\def\PE@InitUccodeHexDigit{%
  \uccode`a=`A\relax
  \uccode`b=`B\relax
  \uccode`c=`C\relax
  \uccode`d=`D\relax
  \uccode`e=`E\relax
  \uccode`f=`F\relax
  \uccode`A=\ltx@zero
  \uccode`B=\ltx@zero
  \uccode`C=\ltx@zero
  \uccode`D=\ltx@zero
  \uccode`E=\ltx@zero
  \uccode`F=\ltx@zero
  \uccode`0=\ltx@zero
  \uccode`1=\ltx@zero
  \uccode`2=\ltx@zero
  \uccode`3=\ltx@zero
  \uccode`4=\ltx@zero
  \uccode`5=\ltx@zero
  \uccode`6=\ltx@zero
  \uccode`7=\ltx@zero
  \uccode`8=\ltx@zero
  \uccode`9=\ltx@zero
}
%    \end{macrocode}
%    \end{macro}
%
%    \begin{macro}{\PE@TestUcHexDigit}
%    \begin{macrocode}
\def\PE@TestUcHexDigit#1{%
  \ifnum`#1<48 % 0
    \ltx@one
  \else
    \ifnum`#1>70 % F
      \ltx@one
    \else
      \ifnum`#1>57 % 9
        \ifnum`#1<65 % A
          \ltx@one
        \else
          \ltx@zero
        \fi
      \else
        \ltx@zero
      \fi
    \fi
  \fi
}
%    \end{macrocode}
%    \end{macro}
%
% \subsection{\cs{EdefUnescapeString}}
%
%    \begin{macro}{\EdefUnescapeString}
%    \begin{macrocode}
\def\EdefUnescapeString#1#2{%
  \EdefSanitize#1{#2}%
  \PE@SanitizeSpaceOther#1%
  \PE@NormalizeLineEnd#1%
  \PE@UnescapeString#1%
  \PE@onelevel@sanitize#1%
}
%    \end{macrocode}
%    \end{macro}
%
%    \begin{macrocode}
\begingroup
  \uccode`\8=10 % lf
  \uccode`\9=13 % cr
\def\x#1#2{\endgroup
%    \end{macrocode}
%    \begin{macro}{\PE@NormalizeLineEnd}
%    \begin{macrocode}
  \def\PE@NormalizeLineEnd##1{%
    \def\PE@result{}%
    \expandafter\PE@@NormalizeLineEnd##1#2\relax
    \let##1\PE@result
  }%
%    \end{macrocode}
%    \end{macro}
%    \begin{macro}{\PE@@NormalizeLineEnd}
%    \begin{macrocode}
  \def\PE@@NormalizeLineEnd##1#2##2{%
    \ifx\relax##2%
      \edef\PE@result{\PE@result##1}%
      \let\PE@next\relax
    \else
      \edef\PE@result{\PE@result##1#1}%
      \ifx#1##2% lf
        \let\PE@next\PE@@NormalizeLineEnd
      \else
        \def\PE@next{\PE@@NormalizeLineEnd##2}%
      \fi
    \fi
    \PE@next
  }%
}%
\uppercase{%
  \x 89%
}
%    \end{macrocode}
%    \end{macro}
%
%    \begin{macrocode}
\begingroup
  \catcode`\|=0 %
  \catcode`\\=12 %
%    \end{macrocode}
%    \begin{macro}{\PE@UnescapeString}
%    \begin{macrocode}
  |gdef|PE@UnescapeString#1{%
    |begingroup
      |def|PE@result{}%
      |expandafter|PE@DeString#1\|relax
    |expandafter|endgroup
    |expandafter|def|expandafter#1|expandafter{|PE@result}%
  }%
%    \end{macrocode}
%    \end{macro}
%    \begin{macro}{\PE@DeString}
%    \begin{macrocode}
  |gdef|PE@DeString#1\#2{%
    |ifx|relax#2%
      |edef|PE@result{|PE@result#1}%
      |let|PE@next|relax
    |else
      |if n#2%
        |uccode|ltx@zero=10 %
      |else|if r#2%
        |uccode|ltx@zero=13 %
      |else|if t#2%
        |uccode|ltx@zero=9 %
      |else|if b#2%
        |uccode|ltx@zero=8 %
      |else|if f#2%
        |uccode|ltx@zero=12 %
      |else
        |uccode|ltx@zero=|ltx@zero
      |fi|fi|fi|fi|fi
      |ifnum|uccode|ltx@zero>|ltx@zero
        |uppercase{%
          |edef|PE@temp{^^@}%
        }%
        |edef|PE@result{|PE@result#1|PE@temp}%
        |let|PE@next|PE@DeString
      |else
        |if\#2% backslash
          |edef|PE@result{|PE@result#1}%
          |let|PE@next|PE@CheckEndBackslash
        |else
          |ifnum`#2=10 % linefeed
            |edef|PE@result{|PE@result#1}%
            |let|PE@next|PE@DeString
          |else
            |ifcase|PE@TestOctDigit#2%
              |edef|PE@result{|PE@result#1}%
              |def|PE@next{|PE@OctI#2}%
            |else
              |edef|PE@result{|PE@result#1#2}%
              |let|PE@next|PE@DeString
            |fi
          |fi
        |fi
      |fi
    |fi
    |PE@next
  }%
%    \end{macrocode}
%    \end{macro}
%    \begin{macro}{\PE@CheckEndBackslash}
%    \begin{macrocode}
  |gdef|PE@CheckEndBackslash#1{%
    |ifx|relax#1%
    |else
      |edef|PE@result{|PE@result\}%
      |expandafter|PE@DeString|expandafter#1%
    |fi
  }%
%    \end{macrocode}
%    \end{macro}
%    \begin{macrocode}
|endgroup
%    \end{macrocode}
%    \begin{macro}{\PE@TestOctDigit}
%    \begin{macrocode}
\def\PE@TestOctDigit#1{%
  \ifnum`#1<48 % 0
    \ltx@one
  \else
    \ifnum`#1>55 % 7
      \ltx@one
    \else
      \ltx@zero
    \fi
  \fi
}
%    \end{macrocode}
%    \end{macro}
%    \begin{macro}{\PE@OctI}
%    \begin{macrocode}
\def\PE@OctI#1#2{%
  \ifcase\PE@TestOctDigit#2%
    \def\PE@next{\PE@OctII{#1#2}}%
  \else
    \def\PE@next{\PE@OctAll#1#2}%
  \fi
  \PE@next
}
%    \end{macrocode}
%    \end{macro}
%    \begin{macro}{\PE@OctII}
%    \begin{macrocode}
\def\PE@OctII#1#2{%
  \ifcase\PE@TestOctDigit#2%
    \def\PE@next{\PE@OctIII#1#2}%
  \else
    \def\PE@next{\PE@OctAll{#1}#2}%
  \fi
  \PE@next
}
%    \end{macrocode}
%    \end{macro}
%    \begin{macrocode}
\ltx@IfUndefined{numexpr}{%
  \catcode`\$=9 %
  \catcode`\&=14 %
}{%
  \catcode`\$=14 %
  \catcode`\&=9 %
}
%    \end{macrocode}
%    \begin{macro}{\PE@OctIII}
%    \begin{macrocode}
\def\PE@OctIII#1#2#3{%
  \ifnum#1<4 %
    \def\PE@next{\PE@OctAll{#1#2#3}}%
  \else
$   \count\ltx@cclv#1 %
$   \advance\count\ltx@cclv -4 %
    \edef\PE@next{%
      \noexpand\PE@OctAll{%
$       \the\count\ltx@cclv
&       \the\numexpr#1-4\relax
        #2#3%
      }%
    }%
  \fi
  \PE@next
}
%    \end{macrocode}
%    \end{macro}
%    \begin{macro}{\PE@OctAll}
%    \begin{macrocode}
\def\PE@OctAll#1{%
  \uccode\ltx@zero='#1\relax
  \uppercase{%
    \edef\PE@result{\PE@result^^@}%
  }%
  \PE@DeString
}
%    \end{macrocode}
%    \end{macro}
%
% \subsection{User macros (\pdfTeX\space analogues)}
%
%    \begin{macrocode}
\begingroup\expandafter\expandafter\expandafter\endgroup
\expandafter\ifx\csname RequirePackage\endcsname\relax
  \def\TMP@RequirePackage#1[#2]{%
    \begingroup\expandafter\expandafter\expandafter\endgroup
    \expandafter\ifx\csname ver@#1.sty\endcsname\relax
      \input #1.sty\relax
    \fi
  }%
  \TMP@RequirePackage{pdftexcmds}[2007/11/11]%
\else
  \RequirePackage{pdftexcmds}[2007/11/11]%
\fi
%    \end{macrocode}
%
%    \begin{macrocode}
\begingroup\expandafter\expandafter\expandafter\endgroup
\expandafter\ifx\csname pdf@escapehex\endcsname\relax
%    \end{macrocode}
%
%    \begin{macro}{\EdefEscapeHex}
%    \begin{macrocode}
  \long\def\EdefEscapeHex#1#2{%
    \EdefSanitize#1{#2}%
    \PE@SanitizeSpaceOther#1%
    \PE@EscapeHex#1%
  }%
%    \end{macrocode}
%    \end{macro}
%    \begin{macro}{\EdefUnescapeHex}
%    \begin{macrocode}
  \def\EdefUnescapeHex#1#2{%
    \EdefSanitize#1{#2}%
    \PE@UnescapeHex#1%
  }%
%    \end{macrocode}
%    \end{macro}
%    \begin{macro}{\EdefEscapeName}
%    \begin{macrocode}
  \long\def\EdefEscapeName#1#2{%
    \EdefSanitize#1{#2}%
    \PE@SanitizeSpaceOther#1%
    \PE@EscapeName#1%
  }%
%    \end{macrocode}
%    \end{macro}
%    \begin{macro}{\EdefEscapeString}
%    \begin{macrocode}
  \long\def\EdefEscapeString#1#2{%
    \EdefSanitize#1{#2}%
    \PE@SanitizeSpaceOther#1%
    \PE@EscapeString#1%
  }%
%    \end{macrocode}
%    \end{macro}
%
%    \begin{macrocode}
\else
%    \end{macrocode}
%
%    \begin{macro}{\PE@edefbabel}
%    Help macro that adds support for \xpackage{babel}'s shorthand
%    characters.
%    \begin{macrocode}
  \long\def\PE@edefbabel#1#2#3{%
    \begingroup
      \csname @save@activestrue\endcsname
      \edef#1{#2{#3}}%
    \expandafter\endgroup
    \expandafter\def\expandafter#1\expandafter{#1}%
  }%
%    \end{macrocode}
%    \end{macro}
%
%    \begin{macro}{\EdefEscapeHex}
%    \begin{macrocode}
  \long\def\EdefEscapeHex#1#2{%
    \PE@edefbabel#1\pdf@escapehex{#2}%
  }%
%    \end{macrocode}
%    \end{macro}
%    \begin{macro}{\EdefUnescapeHex}
%    \begin{macrocode}
  \def\EdefUnescapeHex#1#2{%
    \PE@edefbabel#1\pdf@unescapehex{#2}%
  }%
%    \end{macrocode}
%    \end{macro}
%    \begin{macro}{\EdefEscapeName}
%    \begin{macrocode}
  \long\def\EdefEscapeName#1#2{%
    \PE@edefbabel#1\pdf@escapename{#2}%
  }%
%    \end{macrocode}
%    \end{macro}
%    \begin{macro}{\EdefEscapeString}
%    \begin{macrocode}
  \long\def\EdefEscapeString#1#2{%
    \PE@edefbabel#1\pdf@escapestring{#2}%
  }%
%    \end{macrocode}
%    \end{macro}
%
%    \begin{macrocode}
  \expandafter\PE@AtEnd
\fi%
%    \end{macrocode}
%
% \subsection{Help macros}
%
% \subsubsection{Characters}
%
%    Special characters with catcode 12 (other) are created and
%    stored in macros.
%    \begin{macro}{\PE@hash}
%    \begin{macrocode}
\edef\PE@hash{\string#}
%    \end{macrocode}
%    \end{macro}
%    \begin{macro}{\PE@backslash}
%    \begin{macrocode}
\begingroup
  \escapechar=-1 %
\edef\x{\endgroup
  \def\noexpand\PE@backslash{\string\\}%
}
\x
%    \end{macrocode}
%    \end{macro}
%
% \subsubsection{Switch for \eTeX}
%
%    \begin{macrocode}
\ltx@newif\ifPE@etex
\begingroup\expandafter\expandafter\expandafter\endgroup
\expandafter\ifx\csname numexpr\endcsname\relax
\else
  \PE@etextrue
\fi
%    \end{macrocode}
%
% \subsection{Conversions}
%
% \subsubsection{Conversion to hex string}
%
%    \begin{macro}{\PE@EscapeHex}
%    \begin{macrocode}
\ifPE@etex
  \def\PE@EscapeHex#1{%
    \edef#1{\expandafter\PE@ToHex#1\relax}%
  }%
\else
  \def\PE@EscapeHex#1{%
    \def\PE@result{}%
    \expandafter\PE@ToHex#1\relax
    \let#1\PE@result
  }%
\fi
%    \end{macrocode}
%    \end{macro}
%    \begin{macro}{\PE@ToHex}
%    \begin{macrocode}
\def\PE@ToHex#1{%
  \ifx\relax#1%
  \else
    \PE@HexChar{#1}%
    \expandafter\PE@ToHex
  \fi
}%
%    \end{macrocode}
%    \end{macro}
%    \begin{macro}{\PE@HexChar}
%    \begin{macrocode}
\ifPE@etex
  \def\PE@HexChar#1{%
    \PE@HexDigit{\numexpr\dimexpr.0625\dimexpr`#1sp\relax\relax\relax}%
    \PE@HexDigit{%
      \numexpr`#1-16*\dimexpr.0625\dimexpr`#1sp\relax\relax\relax
    }%
  }%
\else
  \def\PE@HexChar#1{%
    \dimen0=`#1sp%
    \dimen2=.0625\dimen0 %
    \advance\dimen0-16\dimen2 %
    \edef\PE@result{%
      \PE@result
      \PE@HexDigit{\dimen2 }%
      \PE@HexDigit{\dimen0 }%
    }%
  }%
\fi
%    \end{macrocode}
%    \end{macro}
%    \begin{macro}{\PE@HexDigit}
%    \begin{macrocode}
\def\PE@HexDigit#1{%
  \expandafter\string
  \ifcase#1%
    0\or 1\or 2\or 3\or 4\or 5\or 6\or 7\or 8\or 9\or
    A\or B\or C\or D\or E\or F%
  \fi
}
%    \end{macrocode}
%    \end{macro}
%
% \subsubsection{Character code to octal number}
%
%    \begin{macro}{\PE@OctChar}
%    \begin{macrocode}
\ifPE@etex
  \def\PE@OctChar#1{%
    \expandafter\PE@@OctChar
        \the\numexpr\dimexpr.015625\dimexpr`#1sp\relax\relax
            \expandafter\relax
        \expandafter\relax
        \the\numexpr\dimexpr.125\dimexpr`#1sp\relax\relax\relax
        \relax
        #1%
  }%
  \def\PE@@OctChar#1\relax#2\relax#3{%
    \PE@backslash
    #1%
    \the\numexpr#2-8*#1\relax
    \the\numexpr\dimexpr`#3sp\relax-8*#2\relax
  }%
\else
  \def\PE@OctChar#1{%
    \dimen0=`#1sp%
    \dimen2=.125\dimen0 %
    \dimen4=.125\dimen2 %
    \advance\dimen0-8\dimen2 %
    \advance\dimen2-8\dimen4 %
    \edef\PE@result{%
      \PE@result
      \PE@backslash
      \number\dimen4 %
      \number\dimen2 %
      \number\dimen0 %
    }%
  }%
\fi
%    \end{macrocode}
%    \end{macro}
%
% \subsubsection{Unpack hex string}
%
%    \begin{macro}{\PE@UnescapeHex}
%    \begin{macrocode}
\def\PE@UnescapeHex#1{%
  \begingroup
    \PE@InitUccodeHexDigit
    \def\PE@result{}%
    \expandafter\PE@DeHex#1\relax\relax
  \expandafter\endgroup
  \expandafter\def\expandafter#1\expandafter{\PE@result}%
}
%    \end{macrocode}
%    \end{macro}
%    \begin{macro}{\PE@DeHex}
%    \begin{macrocode}
\def\PE@DeHex#1#2{%
  \ifx#2\relax
    \ifx#1\relax
      \let\PE@next\relax
    \else
      \uppercase{%
        \def\PE@testA{#1}%
      }%
      \ifcase\expandafter\PE@TestUcHexDigit\PE@testA
        \def\PE@next{%
          \PE@DeHex#10\relax\relax
        }%
      \else
        \let\PE@next\relax
      \fi
    \fi
  \else
    \uppercase{%
      \def\PE@testA{#1}%
      \def\PE@testB{#2}%
    }%
    \ifcase\expandafter\PE@TestUcHexDigit\PE@testA
      \ifcase\expandafter\PE@TestUcHexDigit\PE@testB
        \uccode\ltx@zero="\PE@testA\PE@testB\relax
        \ifnum\uccode\ltx@zero=32 %
          \let\PE@temp\PE@space@space
        \else
          \uppercase{%
            \def\PE@temp{^^@}%
          }%
        \fi
        \edef\PE@result{\PE@result\PE@temp}%
        \let\PE@next\PE@DeHex
      \else
        % invalid input sequence
        \def\PE@next{%
          \PE@DeHex#1%
        }%
      \fi
    \else
      % invalid input sequence
      \def\PE@next{\PE@DeHex#2}%
    \fi
  \fi
  \PE@next
}
%    \end{macrocode}
%    \end{macro}
%
% \subsubsection{Conversion to PDF name}
%
%    \begin{macro}{\PE@EscapeName}
%    \begin{macrocode}
\ifPE@etex
  \def\PE@EscapeName#1{%
    \edef#1{\expandafter\PE@EscapeNameTokens#1\relax}%
  }%
\else
  \def\PE@EscapeName#1{%
    \def\PE@result{}%
    \expandafter\PE@EscapeNameTokens#1\relax
    \let#1\PE@result
  }%
\fi
%    \end{macrocode}
%    \end{macro}
%    \begin{macro}{\PE@EscapeNameTokens}
%    \begin{macrocode}
\def\PE@EscapeNameTokens#1{%
  \ifx\relax#1%
  \else
    \ifnum`#1<33 %
      \ifcase`#1 %
        % drop illegal zero
      \else
        \PE@EscapeNameAdd\PE@hash
        \PE@HexChar#1%
      \fi
    \else
      \ifnum`#1>126 %
        \PE@EscapeNameAdd\PE@hash
        \PE@HexChar#1%
      \else  \ifnum`#1=35  \PE@EscapeNameHashChar 23% #
        \else\ifnum`#1=37  \PE@EscapeNameHashChar 25% %
        \else\ifnum`#1=40  \PE@EscapeNameHashChar 28% (
        \else\ifnum`#1=41  \PE@EscapeNameHashChar 29% )
        \else\ifnum`#1=47  \PE@EscapeNameHashChar 2F% /
        \else\ifnum`#1=60  \PE@EscapeNameHashChar 3C% <
        \else\ifnum`#1=62  \PE@EscapeNameHashChar 3E% >
        \else\ifnum`#1=91  \PE@EscapeNameHashChar 5B% [
        \else\ifnum`#1=93  \PE@EscapeNameHashChar 5D% ]
        \else\ifnum`#1=123 \PE@EscapeNameHashChar 7B% {
        \else\ifnum`#1=125 \PE@EscapeNameHashChar 7D% }
        \else
          \PE@EscapeNameAdd{#1}%
        \fi\fi\fi\fi\fi\fi\fi\fi\fi\fi\fi
      \fi
    \fi
    \expandafter\PE@EscapeNameTokens
  \fi
}%
\def\PE@EscapeNameHashChar#1#2{%
  \PE@EscapeNameAdd{\PE@hash\string#1\string#2}%
}%
%    \end{macrocode}
%    \end{macro}
%    \begin{macro}{\PE@EscapeNameAdd}
%    \begin{macrocode}
\ifPE@etex
  \def\PE@EscapeNameAdd#1{#1}%
\else
  \def\PE@EscapeNameAdd#1{%
    \edef\PE@result{%
      \PE@result
      #1%
    }%
  }%
\fi
%    \end{macrocode}
%    \end{macro}
%
% \subsubsection{Conversion to PDF string}
%
%    \begin{macro}{\PE@EscapeString}
%    \begin{macrocode}
\ifPE@etex
  \def\PE@EscapeString#1{%
    \edef#1{\expandafter\PE@EscapeStringTokens#1\relax}%
  }%
\else
  \def\PE@EscapeString#1{%
    \begingroup
      \def\PE@result{}%
      \expandafter\PE@EscapeStringTokens#1\relax
    \expandafter\endgroup
    \expandafter\def\expandafter#1\expandafter{\PE@result}%
  }%
\fi
%    \end{macrocode}
%    \end{macro}
%    \begin{macro}{\PE@EscapeStringTokens}
%    \begin{macrocode}
\def\PE@EscapeStringTokens#1{%
  \ifx\relax#1%
  \else
    \ifnum`#1<33 %
      \PE@OctChar#1%
    \else
      \ifnum`#1>126 %
        \PE@OctChar#1%
      \else  \ifnum`#1=40 \PE@EscapeStringAdd{\string\(}% (
        \else\ifnum`#1=41 \PE@EscapeStringAdd{\string\)}% )
        \else\ifnum`#1=92 \PE@EscapeStringAdd{\string\\}% \
        \else
          \PE@EscapeStringAdd{#1}%
        \fi\fi\fi
      \fi
    \fi
    \expandafter\PE@EscapeStringTokens
  \fi
}%
%    \end{macrocode}
%    \end{macro}
%    \begin{macro}{\PE@EscapeStringAdd}
%    \begin{macrocode}
\ifPE@etex
  \def\PE@EscapeStringAdd#1{#1}%
\else
  \def\PE@EscapeStringAdd#1{%
    \edef\PE@result{%
      \PE@result
      #1%
    }%
  }%
\fi
%    \end{macrocode}
%    \end{macro}
%
%    \begin{macrocode}
\PE@AtEnd%
%    \end{macrocode}
%
%    \begin{macrocode}
%</package>
%    \end{macrocode}
%
% \section{Test}
%
% \subsection{Catcode checks for loading}
%
%    \begin{macrocode}
%<*test1>
%    \end{macrocode}
%    \begin{macrocode}
\catcode`\{=1 %
\catcode`\}=2 %
\catcode`\#=6 %
\catcode`\@=11 %
\expandafter\ifx\csname count@\endcsname\relax
  \countdef\count@=255 %
\fi
\expandafter\ifx\csname @gobble\endcsname\relax
  \long\def\@gobble#1{}%
\fi
\expandafter\ifx\csname @firstofone\endcsname\relax
  \long\def\@firstofone#1{#1}%
\fi
\expandafter\ifx\csname loop\endcsname\relax
  \expandafter\@firstofone
\else
  \expandafter\@gobble
\fi
{%
  \def\loop#1\repeat{%
    \def\body{#1}%
    \iterate
  }%
  \def\iterate{%
    \body
      \let\next\iterate
    \else
      \let\next\relax
    \fi
    \next
  }%
  \let\repeat=\fi
}%
\def\RestoreCatcodes{}
\count@=0 %
\loop
  \edef\RestoreCatcodes{%
    \RestoreCatcodes
    \catcode\the\count@=\the\catcode\count@\relax
  }%
\ifnum\count@<255 %
  \advance\count@ 1 %
\repeat

\def\RangeCatcodeInvalid#1#2{%
  \count@=#1\relax
  \loop
    \catcode\count@=15 %
  \ifnum\count@<#2\relax
    \advance\count@ 1 %
  \repeat
}
\def\RangeCatcodeCheck#1#2#3{%
  \count@=#1\relax
  \loop
    \ifnum#3=\catcode\count@
    \else
      \errmessage{%
        Character \the\count@\space
        with wrong catcode \the\catcode\count@\space
        instead of \number#3%
      }%
    \fi
  \ifnum\count@<#2\relax
    \advance\count@ 1 %
  \repeat
}
\def\space{ }
\expandafter\ifx\csname LoadCommand\endcsname\relax
  \def\LoadCommand{\input pdfescape.sty\relax}%
\fi
\def\Test{%
  \RangeCatcodeInvalid{0}{47}%
  \RangeCatcodeInvalid{58}{64}%
  \RangeCatcodeInvalid{91}{96}%
  \RangeCatcodeInvalid{123}{255}%
  \catcode`\@=12 %
  \catcode`\\=0 %
  \catcode`\%=14 %
  \LoadCommand
  \RangeCatcodeCheck{0}{36}{15}%
  \RangeCatcodeCheck{37}{37}{14}%
  \RangeCatcodeCheck{38}{47}{15}%
  \RangeCatcodeCheck{48}{57}{12}%
  \RangeCatcodeCheck{58}{63}{15}%
  \RangeCatcodeCheck{64}{64}{12}%
  \RangeCatcodeCheck{65}{90}{11}%
  \RangeCatcodeCheck{91}{91}{15}%
  \RangeCatcodeCheck{92}{92}{0}%
  \RangeCatcodeCheck{93}{96}{15}%
  \RangeCatcodeCheck{97}{122}{11}%
  \RangeCatcodeCheck{123}{255}{15}%
  \RestoreCatcodes
}
\Test
\csname @@end\endcsname
\end
%    \end{macrocode}
%    \begin{macrocode}
%</test1>
%    \end{macrocode}
%
% \subsection{Macro tests}
%
%    \begin{macrocode}
%<*test2|test3|test4|test5>
\NeedsTeXFormat{LaTeX2e}
\makeatletter
%    \end{macrocode}
%
% \subsection{Test with \cs{pdfescape...} commands}
%
%    \begin{macrocode}
%<*test2>
\ProvidesFile{pdfescape-test2.tex}%
    [2011/11/25 v1.13 Test with \string\pdfescape... commands]%
%</test2>
%    \end{macrocode}
%
% \subsection{Test without \cs{pdfescape...}, with \eTeX}
%
%    \begin{macrocode}
%<*test3>
\ProvidesFile{pdfescape-test3.tex}%
    [2011/11/25 v1.13 Test without \string\pdfescape..., with e-TeX]%
%</test3>
%    \end{macrocode}
%
% \subsection{Test without \cs{pdfescape...} and \eTeX}
%
%    \begin{macrocode}
%<*test4>
\ProvidesFile{pdfescape-test4.tex}%
    [2011/11/25 v1.13 Test without \string\pdfescape... and e-TeX]%
%</test4>
%    \end{macrocode}
%
% \subsection{Test with \LuaTeX}
%
%    \begin{macrocode}
%<*test5>
\ProvidesFile{pdfescape-test5.tex}%
    [2011/11/25 v1.13 Test with LuaTeX]%
%</test5>
%    \end{macrocode}
%
% \subsection{Check/ensure test preconditions}
%
% \subsubsection{Check \cs{pdfescape...}}
%
%    \begin{macrocode}
%<*test2>
\@ifundefined{pdfescapehex}{%
  \PackageError{pdfescape-test2}{%
    Missing \string\pdfescape... commands%
  }{Test aborted.}%
  \stop
}{}
%</test2>
%    \end{macrocode}
%    \begin{macrocode}
%<*test3|test4>
\let\pdfescapehex\@undefined
\let\pdfunescapehex\@undefined
\let\pdfescapename\@undefined
\let\pdfescapestring\@undefined
%</test3|test4>
%    \end{macrocode}
%
% \subsubsection{Check \eTeX}
%
%    \begin{macrocode}
%<*test3>
\@ifundefined{numexpr}{%
  \PackageError{pdfescape-test3}{%
    Missing \eTeX
  }{Test aborted.}%
  \stop
}{}
%</test3>
%    \end{macrocode}
% Package \xpackage{qstest} uses \eTeX, thus \eTeX's features
% can only be disabled later during loading of package \xpackage{pdfescape}.
%
% \subsubsection{Check \LuaTeX}
%
%    \begin{macrocode}
%<*test5>
\@ifundefined{directlua}{%
  \PackageError{pdfescape-test5}{%
    Missing LuaTeX%
  }{Test aborted.}%
  \stop
}{}
%</test5>
%    \end{macrocode}
%
% \subsection{Common part}
%
%    The files for testing uses the framework, provided by the
%    new package \xpackage{qstest} of David Kastrup.
%    \begin{macrocode}
\RequirePackage{qstest}
\IncludeTests{*}
\LogTests{log}{*}{*}

\newcommand*{\ExpectVar}[2]{%
  \ifx#1#2%
  \else
    \begingroup
      \@onelevel@sanitize#1%
      \@onelevel@sanitize#2%
      \typeout{[#1] <> [#2]}% hash-ok
    \endgroup
  \fi
  \Expect*{\ifx#1#2true\else false\fi}{true}%
}

\makeatletter
\begingroup
  \gdef\AllBytes{}%
  \count@=0 %
  \catcode0=12 %
  \@whilenum\count@<256 \do{%
    \lccode0=\count@
    \ifnum\count@=32 %
      \xdef\AllBytes{\AllBytes\space}%
    \else
      \lowercase{%
        \xdef\AllBytes{\AllBytes^^@}%
      }%
    \fi
    \advance\count@ by 1 %
  }%
\endgroup
\newcommand*{\AllBytesHex}{%
  000102030405060708090A0B0C0D0E0F%
  101112131415161718191A1B1C1D1E1F%
  202122232425262728292A2B2C2D2E2F%
  303132333435363738393A3B3C3D3E3F%
  404142434445464748494A4B4C4D4E4F%
  505152535455565758595A5B5C5D5E5F%
  606162636465666768696A6B6C6D6E6F%
  707172737475767778797A7B7C7D7E7F%
  808182838485868788898A8B8C8D8E8F%
  909192939495969798999A9B9C9D9E9F%
  A0A1A2A3A4A5A6A7A8A9AAABACADAEAF%
  B0B1B2B3B4B5B6B7B8B9BABBBCBDBEBF%
  C0C1C2C3C4C5C6C7C8C9CACBCCCDCECF%
  D0D1D2D3D4D5D6D7D8D9DADBDCDDDEDF%
  E0E1E2E3E4E5E6E7E8E9EAEBECEDEEEF%
  F0F1F2F3F4F5F6F7F8F9FAFBFCFDFEFF%
}
\@onelevel@sanitize\AllBytesHex
\expandafter\lowercase\expandafter{%
  \expandafter\newcommand\expandafter*\expandafter\AllBytesHexLC
      \expandafter{\AllBytesHex}%
}
\newcommand*{\AllBytesName}{}
\begingroup
  \catcode`\#=12 %
  \xdef\AllBytesName{%
    #01#02#03#04#05#06#07#08#09#0A#0B#0C#0D#0E#0F%
    #10#11#12#13#14#15#16#17#18#19#1A#1B#1C#1D#1E#1F%
    #20!"#23$#25&'#28#29*+,-.#2F%
    0123456789:;#3C=#3E?%
    @ABCDEFGHIJKLMNO%
    PQRSTUVWXYZ#5B\@backslashchar#5D^_%
    `abcdefghijklmno%
    pqrstuvwxyz#7B|#7D\string~#7F%
    #80#81#82#83#84#85#86#87#88#89#8A#8B#8C#8D#8E#8F%
    #90#91#92#93#94#95#96#97#98#99#9A#9B#9C#9D#9E#9F%
    #A0#A1#A2#A3#A4#A5#A6#A7#A8#A9#AA#AB#AC#AD#AE#AF%
    #B0#B1#B2#B3#B4#B5#B6#B7#B8#B9#BA#BB#BC#BD#BE#BF%
    #C0#C1#C2#C3#C4#C5#C6#C7#C8#C9#CA#CB#CC#CD#CE#CF%
    #D0#D1#D2#D3#D4#D5#D6#D7#D8#D9#DA#DB#DC#DD#DE#DF%
    #E0#E1#E2#E3#E4#E5#E6#E7#E8#E9#EA#EB#EC#ED#EE#EF%
    #F0#F1#F2#F3#F4#F5#F6#F7#F8#F9#FA#FB#FC#FD#FE#FF%
  }%
\endgroup
\@onelevel@sanitize\AllBytesName

\newcommand*{\AllBytesString}{}
\begingroup
  \def\|{|}%
  \edef\%{\@percentchar}%
  \catcode`\|=0 %
  \catcode`\#=12 %
  \catcode`\~=12 %
  \catcode`\\=12 %
  |xdef|AllBytesString{%
    \000\001\002\003\004\005\006\007\010\011\012\013\014\015\016\017%
    \020\021\022\023\024\025\026\027\030\031\032\033\034\035\036\037%
    \040!"#$|%&'\(\)*+,-./%
    0123456789:;<=>?%
    @ABCDEFGHIJKLMNO%
    PQRSTUVWXYZ[\\]^_%
    `abcdefghijklmno%
    pqrstuvwxyz{||}~\177%
    \200\201\202\203\204\205\206\207\210\211\212\213\214\215\216\217%
    \220\221\222\223\224\225\226\227\230\231\232\233\234\235\236\237%
    \240\241\242\243\244\245\246\247\250\251\252\253\254\255\256\257%
    \260\261\262\263\264\265\266\267\270\271\272\273\274\275\276\277%
    \300\301\302\303\304\305\306\307\310\311\312\313\314\315\316\317%
    \320\321\322\323\324\325\326\327\330\331\332\333\334\335\336\337%
    \340\341\342\343\344\345\346\347\350\351\352\353\354\355\356\357%
    \360\361\362\363\364\365\366\367\370\371\372\373\374\375\376\377%
  }%
|endgroup
\@onelevel@sanitize\AllBytesString

%<*test4>
\let\org@detokenize\detokenize
\let\detokenize\@undefined
\let\org@numexpr\numexpr
\let\numexpr\@undefined
%</test4>
\RequirePackage{pdfescape}
%<*test4>
\let\detokenize\org@detokenize
\let\numexpr\org@numexpr
%</test4>

\begin{qstest}{all-hex}{\AllBytes, escapehex}
  \EdefEscapeHex\x{\AllBytes}%
  \Expect*{\x}*{\AllBytesHex}%
  \ExpectVar\x\AllBytesHex
\end{qstest}

\begin{qstest}{all-unhex}{\AllBytesHex, unescapehex}
  \EdefUnescapeHex\x{\AllBytesHex}%
  \Expect*{\x}*{\AllBytes}%
  \ExpectVar\x\AllBytes
\end{qstest}

\begin{qstest}{all-unhex-lc}{\AllBytesHexLC, unescapehex, lowercase}
  \EdefUnescapeHex\x{\AllBytesHexLC}%
  \Expect*{\x}*{\AllBytes}%
  \ExpectVar\x\AllBytes
\end{qstest}

\begin{qstest}{unhex-incomplete}{unescapehex, incomplete}
  \EdefUnescapeHex\x{4}%
  \Expect*{\x}{@}%
\end{qstest}

\begin{qstest}{unhex-space}{unescapehex, space}
  \EdefUnescapeHex\x{20}%
  \Expect*{\x}{ }%
  \ExpectVar\x\space
\end{qstest}

\begin{qstest}{unhex-spaces}{unescapehex, spaces}
  \EdefUnescapeHex\x{204020204120}%
  \def\y#1{%
    \edef\z{#1\string @#1#1\string A#1}%
  }\y{ }%
  \Expect*{\x}*{\z}%
  \ExpectVar\x\z
\end{qstest}

\begin{qstest}{unhex-hash}{unescapehex, hash}
  \catcode`\#=12 %
  \EdefUnescapeHex\x{#20}%
  \ExpectVar\x\space
\end{qstest}

\begin{qstest}{unhex-invalid}{unescapehex, invalid}
  \def\test#1#2{%
    \EdefUnescapeHex\x{#1}%
    \edef\y{#2}%
    \@onelevel@sanitize\y
    \ExpectVar\x\y
  }%
%<*test2>
  \edef\x{\pdfunescapehex{4X}}%
  \edef\y{\string @}%
  \ifx\x\y
  \else
    \def~{\space}%
    \typeout{************************************}%
    \typeout{* Your pdfTeX contains bug 777.~~~~*}%
    \typeout{* This test is redefined as dummy, *}%
    \typeout{* because it triggers the bug.~~~~~*}%
    \typeout{************************************}%
    \def\test#1#2{}%
  \fi
%</test2>
  \test{X}{}%
  \test{XY}{}%
  \test{XYZ}{}%
  \test{A}{^^a0}%
  \test{AX}{^^a0}%
  \test{XA}{^^a0}%
  \test{XXAXX}{^^a0}%
\end{qstest}

\begin{qstest}{all-name}{\AllBytes, escapename}
  \EdefEscapeName\x{\AllBytes}%
  \Expect*{\x}*{\AllBytesName}%
  \ExpectVar\x\AllBytesName
\end{qstest}

\begin{qstest}{all-string}{\AllBytes, escapestring}
  \EdefEscapeString\x{\AllBytes}%
  \Expect*{\x}*{\AllBytesString}%
  \ExpectVar\x\AllBytesString
\end{qstest}

\begin{qstest}{uchexdigit}{unescape, uppercase hex digit}
  \catcode`\@=11 %
  \catcode0=12 %
  \def\test#1#2{%
    \uccode0=#1\relax
    \uppercase{%
      \def\x{^^@}%
    }%
    \Expect*{%
       \ifcase\expandafter\PE@TestUcHexDigit\x
         true%
       \else
         false%
       \fi
    }{#2}%
  }%
  \def\range#1#2#3{%
    \count0=#1\relax
    \loop
    \ifnum\count0<#2\relax
      \test{\count0}{#3}%
      \advance\count0 by 1 %
    \repeat
  }%
  \range{0}{47}{false}%
  \range{48}{57}{true}%
  \range{58}{64}{false}%
  \range{65}{70}{true}%
  \range{71}{255}{false}%
\end{qstest}

\begin{qstest}{unescapename}{unescapename}
  \def\test#1#2{%
    \EdefUnescapeName\x{#1}%
    \edef\y{#2}%
    \@onelevel@sanitize\y
    \ExpectVar\x\y
  }%
  \catcode`\#=12 %
  \catcode0=12 %
  \test{}{}%
  \test{x}{x}%
  \test{xy}{xy}%
  \test{#}{#}%
  \test{##}{##}%
  \test{###}{###}%
  \test{####}{####}%
  \test{#x}{#x}%
  \test{#xy}{#xy}%
  \test{#1}{#1}%
  \test{#40}{@}%
  \test{#400}{@0}%
  \test{#4x0}{#4x0}%
  \test{#ab}{^^ab}%
  \test{#00}{^^@}%
  \test{x#40y#40z}{x@y@z}%
  \test{#40#40#40#40}{@@@@}%
  \test{a#x}{a#x}%
  \test{a#xy}{a#xy}%
  \test{a#1}{a#1}%
  \test{a#40}{a@}%
  \test{a#400}{a@0}%
  \test{#20}{ }%
  \test{a#20}{a }%
  \test{a#20b}{a b}%
  \test{a#20#20#20b}{a \space\space b}%
\end{qstest}

\begin{qstest}{unescapestring}{unescapestring}
  \def\test#1#2{%
    \EdefUnescapeString\x{#1}%
    \edef\y{#2}%
    \@onelevel@sanitize\y
    \ExpectVar\x\y
  }%
  \catcode0=12 %
  \def\DefChar#1#2{%
    \begingroup
      \uccode0=#2\relax
    \uppercase{\endgroup
      \def#1{^^@}%
    }%
  }%
  \DefChar\nul{0}%
  \DefChar\one{1}%
  \DefChar\bel{8}%
  \DefChar\tab{9}%
  \DefChar\lf{10}%
  \DefChar\ff{12}%
  \DefChar\cr{13}%
  \DefChar\\{92}%
  \test{}{}%
  \test{a}{a}%
  \test{\\}{}%
  \test{\\\\}{\\}%
  \test{\\\\y}{\\y}%
  \test{\\000}{\nul}%
  \test{\\b}{\bel}%
  \test{\\t}{\tab}%
  \test{\\n}{\lf}%
  \test{\\f}{\ff}%
  \test{\\r}{\cr}%
  \test{\\(}{(}%
  \test{\\)}{)}%
  \test{\\040}{ }%
  \test{\\100}{@}%
  \test{\\40}{ }%
  \test{\\1}{\one}%
  \test{\\01}{\one}%
  \test{\\001}{\one}%
  \test{\\18}{\one8}%
  \test{\\018}{\one8}%
  \test{\\0018}{\one8}%
  \test{x\\}{x}%
  \test{x\\\\}{x\\}%
  \test{x\\\\y}{x\\y}%
  \test{x\\000}{x\nul}%
  \test{x\\b}{x\bel}%
  \test{x\\t}{x\tab}%
  \test{x\\n}{x\lf}%
  \test{x\\f}{x\ff}%
  \test{x\\r}{x\cr}%
  \test{x\\(}{x(}%
  \test{x\\)}{x)}%
  \test{x\\040}{x }%
  \test{x\\100}{x@}%
  \test{x\\40}{x }%
  \test{x\\1}{x\one}%
  \test{x\\01}{x\one}%
  \test{x\\001}{x\one}%
  \test{x\\18}{x\one8}%
  \test{x\\018}{x\one8}%
  \test{x\\0018}{x\one8}%
  \test{\\b\\t\\n\\f\\r\\(\\)\\\\\\000\\040}{%
    \bel\tab\lf\ff\cr()\\\nul\space
  }%
  \test{\\\lf}{}%
  \test{x\\\lf}{x}%
  \test{\cr}{\lf}%
  \test{\cr\lf}{\lf}%
  \test{\lf}{\lf}%
  \test{\lf\cr}{\lf\lf}%
  \test{x\cr}{x\lf}%
  \test{x\cr\lf}{x\lf}%
  \test{x\lf}{x\lf}%
  \test{x\lf\cr}{x\lf\lf}%
  \test{x\\\cr\lf y\cr}{xy\lf}%
%
  \test{\\409}{ 9}%
  \test{\\800}{800}%
  \test{\\900}{900}%
  \test{\\578}{/8}%
  \test{\\477}{?}%
  \test{\\377}{^^ff}%
  \test{\\777}{^^ff}%
  \test{\\7777}{^^ff7}%
\end{qstest}
\stop
%    \end{macrocode}
%
%    \begin{macrocode}
%</test2|test3|test4|test5>
%    \end{macrocode}
%
% \subsubsection{Test for \hologo{iniTeX}}
%
%    \begin{macrocode}
%<*test6>
\input pdfescape.sty\relax
\catcode`\{=1 %
\catcode`\}=2 %
\catcode`\#=6 %
\catcode`\^=7 %
\catcode`\@=11 %
%    \end{macrocode}
%    \begin{macrocode}
\begingroup
  \catcode`\@=11 %
  \countdef\count@=255 %
  \def\space{ }%
  \long\def\@whilenum#1\do #2{%
    \ifnum #1\relax
      #2\relax
      \@iwhilenum{#1\relax#2\relax}%
    \fi
  }%
  \long\def\@iwhilenum#1{%
    \ifnum #1%
      \expandafter\@iwhilenum
    \else
      \expandafter\ltx@gobble
    \fi
    {#1}%
  }%
  \gdef\AllBytes{}%
  \count@=0 %
  \catcode0=12 %
  \@whilenum\count@<256 \do{%
    \lccode0=\count@
    \ifnum\count@=32 %
      \xdef\AllBytes{\AllBytes\space}%
    \else
      \lowercase{%
        \xdef\AllBytes{\AllBytes^^@}%
      }%
    \fi
    \advance\count@ by 1 %
  }%
\endgroup
\def\AllBytesHex{%
  000102030405060708090A0B0C0D0E0F%
  101112131415161718191A1B1C1D1E1F%
  202122232425262728292A2B2C2D2E2F%
  303132333435363738393A3B3C3D3E3F%
  404142434445464748494A4B4C4D4E4F%
  505152535455565758595A5B5C5D5E5F%
  606162636465666768696A6B6C6D6E6F%
  707172737475767778797A7B7C7D7E7F%
  808182838485868788898A8B8C8D8E8F%
  909192939495969798999A9B9C9D9E9F%
  A0A1A2A3A4A5A6A7A8A9AAABACADAEAF%
  B0B1B2B3B4B5B6B7B8B9BABBBCBDBEBF%
  C0C1C2C3C4C5C6C7C8C9CACBCCCDCECF%
  D0D1D2D3D4D5D6D7D8D9DADBDCDDDEDF%
  E0E1E2E3E4E5E6E7E8E9EAEBECEDEEEF%
  F0F1F2F3F4F5F6F7F8F9FAFBFCFDFEFF%
}
\ltx@onelevel@sanitize\AllBytesHex
\expandafter\lowercase\expandafter{%
  \expandafter\def\expandafter\AllBytesHexLC
      \expandafter{\AllBytesHex}%
}
\begingroup
  \catcode`\#=12 %
  \xdef\AllBytesName{%
    #01#02#03#04#05#06#07#08#09#0A#0B#0C#0D#0E#0F%
    #10#11#12#13#14#15#16#17#18#19#1A#1B#1C#1D#1E#1F%
    #20!"#23$#25&'#28#29*+,-.#2F%
    0123456789:;#3C=#3E?%
    @ABCDEFGHIJKLMNO%
    PQRSTUVWXYZ#5B\ltx@backslashchar#5D^_%
    `abcdefghijklmno%
    pqrstuvwxyz#7B|#7D\string~#7F%
    #80#81#82#83#84#85#86#87#88#89#8A#8B#8C#8D#8E#8F%
    #90#91#92#93#94#95#96#97#98#99#9A#9B#9C#9D#9E#9F%
    #A0#A1#A2#A3#A4#A5#A6#A7#A8#A9#AA#AB#AC#AD#AE#AF%
    #B0#B1#B2#B3#B4#B5#B6#B7#B8#B9#BA#BB#BC#BD#BE#BF%
    #C0#C1#C2#C3#C4#C5#C6#C7#C8#C9#CA#CB#CC#CD#CE#CF%
    #D0#D1#D2#D3#D4#D5#D6#D7#D8#D9#DA#DB#DC#DD#DE#DF%
    #E0#E1#E2#E3#E4#E5#E6#E7#E8#E9#EA#EB#EC#ED#EE#EF%
    #F0#F1#F2#F3#F4#F5#F6#F7#F8#F9#FA#FB#FC#FD#FE#FF%
  }%
\endgroup
\ltx@onelevel@sanitize\AllBytesName
\edef\AllBytesFromName{\expandafter\ltx@gobble\AllBytes}

\begingroup
  \def\|{|}%
  \edef\%{\ltx@percentchar}%
  \catcode`\|=0 %
  \catcode`\#=12 %
  \catcode`\~=12 %
  \catcode`\\=12 %
  |xdef|AllBytesString{%
    \000\001\002\003\004\005\006\007\010\011\012\013\014\015\016\017%
    \020\021\022\023\024\025\026\027\030\031\032\033\034\035\036\037%
    \040!"#$|%&'\(\)*+,-./%
    0123456789:;<=>?%
    @ABCDEFGHIJKLMNO%
    PQRSTUVWXYZ[\\]^_%
    `abcdefghijklmno%
    pqrstuvwxyz{||}~\177%
    \200\201\202\203\204\205\206\207\210\211\212\213\214\215\216\217%
    \220\221\222\223\224\225\226\227\230\231\232\233\234\235\236\237%
    \240\241\242\243\244\245\246\247\250\251\252\253\254\255\256\257%
    \260\261\262\263\264\265\266\267\270\271\272\273\274\275\276\277%
    \300\301\302\303\304\305\306\307\310\311\312\313\314\315\316\317%
    \320\321\322\323\324\325\326\327\330\331\332\333\334\335\336\337%
    \340\341\342\343\344\345\346\347\350\351\352\353\354\355\356\357%
    \360\361\362\363\364\365\366\367\370\371\372\373\374\375\376\377%
  }%
|endgroup
\ltx@onelevel@sanitize\AllBytesString
%    \end{macrocode}
%    \begin{macrocode}
\def\msg#{\immediate\write16}
\def\Test#1#2#3{%
  \begingroup
    #1\TestResult{#2}%
    \ifx\TestResult#3%
    \else
      \newlinechar=10 %
      \msg{Expect:^^J#3}%
      \msg{Result:^^J\TestResult}%
      \errmessage{\string#2 -\string#1-> \string#3}%
    \fi
  \endgroup
}
\Test\EdefEscapeHex\AllBytes\AllBytesHex
\Test\EdefUnescapeHex\AllBytesHex\AllBytes
\Test\EdefEscapeName\AllBytes\AllBytesName
\Test\EdefUnescapeName\AllBytesName\AllBytesFromName
\Test\EdefEscapeString\AllBytes\AllBytesString
\Test\EdefUnescapeString\AllBytesString\AllBytes
%    \end{macrocode}
%    \begin{macrocode}
\csname @@end\endcsname\end
%</test6>
%    \end{macrocode}
%
% \section{Installation}
%
% \subsection{Download}
%
% \paragraph{Package.} This package is available on
% CTAN\footnote{\url{ftp://ftp.ctan.org/tex-archive/}}:
% \begin{description}
% \item[\CTAN{macros/latex/contrib/oberdiek/pdfescape.dtx}] The source file.
% \item[\CTAN{macros/latex/contrib/oberdiek/pdfescape.pdf}] Documentation.
% \end{description}
%
%
% \paragraph{Bundle.} All the packages of the bundle `oberdiek'
% are also available in a TDS compliant ZIP archive. There
% the packages are already unpacked and the documentation files
% are generated. The files and directories obey the TDS standard.
% \begin{description}
% \item[\CTAN{install/macros/latex/contrib/oberdiek.tds.zip}]
% \end{description}
% \emph{TDS} refers to the standard ``A Directory Structure
% for \TeX\ Files'' (\CTAN{tds/tds.pdf}). Directories
% with \xfile{texmf} in their name are usually organized this way.
%
% \subsection{Bundle installation}
%
% \paragraph{Unpacking.} Unpack the \xfile{oberdiek.tds.zip} in the
% TDS tree (also known as \xfile{texmf} tree) of your choice.
% Example (linux):
% \begin{quote}
%   |unzip oberdiek.tds.zip -d ~/texmf|
% \end{quote}
%
% \paragraph{Script installation.}
% Check the directory \xfile{TDS:scripts/oberdiek/} for
% scripts that need further installation steps.
% Package \xpackage{attachfile2} comes with the Perl script
% \xfile{pdfatfi.pl} that should be installed in such a way
% that it can be called as \texttt{pdfatfi}.
% Example (linux):
% \begin{quote}
%   |chmod +x scripts/oberdiek/pdfatfi.pl|\\
%   |cp scripts/oberdiek/pdfatfi.pl /usr/local/bin/|
% \end{quote}
%
% \subsection{Package installation}
%
% \paragraph{Unpacking.} The \xfile{.dtx} file is a self-extracting
% \docstrip\ archive. The files are extracted by running the
% \xfile{.dtx} through \plainTeX:
% \begin{quote}
%   \verb|tex pdfescape.dtx|
% \end{quote}
%
% \paragraph{TDS.} Now the different files must be moved into
% the different directories in your installation TDS tree
% (also known as \xfile{texmf} tree):
% \begin{quote}
% \def\t{^^A
% \begin{tabular}{@{}>{\ttfamily}l@{ $\rightarrow$ }>{\ttfamily}l@{}}
%   pdfescape.sty & tex/generic/oberdiek/pdfescape.sty\\
%   pdfescape.pdf & doc/latex/oberdiek/pdfescape.pdf\\
%   test/pdfescape-test1.tex & doc/latex/oberdiek/test/pdfescape-test1.tex\\
%   test/pdfescape-test2.tex & doc/latex/oberdiek/test/pdfescape-test2.tex\\
%   test/pdfescape-test3.tex & doc/latex/oberdiek/test/pdfescape-test3.tex\\
%   test/pdfescape-test4.tex & doc/latex/oberdiek/test/pdfescape-test4.tex\\
%   test/pdfescape-test5.tex & doc/latex/oberdiek/test/pdfescape-test5.tex\\
%   test/pdfescape-test6.tex & doc/latex/oberdiek/test/pdfescape-test6.tex\\
%   pdfescape.dtx & source/latex/oberdiek/pdfescape.dtx\\
% \end{tabular}^^A
% }^^A
% \sbox0{\t}^^A
% \ifdim\wd0>\linewidth
%   \begingroup
%     \advance\linewidth by\leftmargin
%     \advance\linewidth by\rightmargin
%   \edef\x{\endgroup
%     \def\noexpand\lw{\the\linewidth}^^A
%   }\x
%   \def\lwbox{^^A
%     \leavevmode
%     \hbox to \linewidth{^^A
%       \kern-\leftmargin\relax
%       \hss
%       \usebox0
%       \hss
%       \kern-\rightmargin\relax
%     }^^A
%   }^^A
%   \ifdim\wd0>\lw
%     \sbox0{\small\t}^^A
%     \ifdim\wd0>\linewidth
%       \ifdim\wd0>\lw
%         \sbox0{\footnotesize\t}^^A
%         \ifdim\wd0>\linewidth
%           \ifdim\wd0>\lw
%             \sbox0{\scriptsize\t}^^A
%             \ifdim\wd0>\linewidth
%               \ifdim\wd0>\lw
%                 \sbox0{\tiny\t}^^A
%                 \ifdim\wd0>\linewidth
%                   \lwbox
%                 \else
%                   \usebox0
%                 \fi
%               \else
%                 \lwbox
%               \fi
%             \else
%               \usebox0
%             \fi
%           \else
%             \lwbox
%           \fi
%         \else
%           \usebox0
%         \fi
%       \else
%         \lwbox
%       \fi
%     \else
%       \usebox0
%     \fi
%   \else
%     \lwbox
%   \fi
% \else
%   \usebox0
% \fi
% \end{quote}
% If you have a \xfile{docstrip.cfg} that configures and enables \docstrip's
% TDS installing feature, then some files can already be in the right
% place, see the documentation of \docstrip.
%
% \subsection{Refresh file name databases}
%
% If your \TeX~distribution
% (\teTeX, \mikTeX, \dots) relies on file name databases, you must refresh
% these. For example, \teTeX\ users run \verb|texhash| or
% \verb|mktexlsr|.
%
% \subsection{Some details for the interested}
%
% \paragraph{Attached source.}
%
% The PDF documentation on CTAN also includes the
% \xfile{.dtx} source file. It can be extracted by
% AcrobatReader 6 or higher. Another option is \textsf{pdftk},
% e.g. unpack the file into the current directory:
% \begin{quote}
%   \verb|pdftk pdfescape.pdf unpack_files output .|
% \end{quote}
%
% \paragraph{Unpacking with \LaTeX.}
% The \xfile{.dtx} chooses its action depending on the format:
% \begin{description}
% \item[\plainTeX:] Run \docstrip\ and extract the files.
% \item[\LaTeX:] Generate the documentation.
% \end{description}
% If you insist on using \LaTeX\ for \docstrip\ (really,
% \docstrip\ does not need \LaTeX), then inform the autodetect routine
% about your intention:
% \begin{quote}
%   \verb|latex \let\install=y% \iffalse meta-comment
%
% File: pdfescape.dtx
% Version: 2011/11/25 v1.13
% Info: Implements pdfTeX's escape features
%
% Copyright (C) 2007, 2010, 2011 by
%    Heiko Oberdiek <heiko.oberdiek at googlemail.com>
%
% This work may be distributed and/or modified under the
% conditions of the LaTeX Project Public License, either
% version 1.3c of this license or (at your option) any later
% version. This version of this license is in
%    http://www.latex-project.org/lppl/lppl-1-3c.txt
% and the latest version of this license is in
%    http://www.latex-project.org/lppl.txt
% and version 1.3 or later is part of all distributions of
% LaTeX version 2005/12/01 or later.
%
% This work has the LPPL maintenance status "maintained".
%
% This Current Maintainer of this work is Heiko Oberdiek.
%
% The Base Interpreter refers to any `TeX-Format',
% because some files are installed in TDS:tex/generic//.
%
% This work consists of the main source file pdfescape.dtx
% and the derived files
%    pdfescape.sty, pdfescape.pdf, pdfescape.ins, pdfescape.drv,
%    pdfescape-test1.tex, pdfescape-test2.tex,
%    pdfescape-test3.tex, pdfescape-test4.tex,
%    pdfescape-test5.tex, pdfescape-test6.tex.
%
% Distribution:
%    CTAN:macros/latex/contrib/oberdiek/pdfescape.dtx
%    CTAN:macros/latex/contrib/oberdiek/pdfescape.pdf
%
% Unpacking:
%    (a) If pdfescape.ins is present:
%           tex pdfescape.ins
%    (b) Without pdfescape.ins:
%           tex pdfescape.dtx
%    (c) If you insist on using LaTeX
%           latex \let\install=y\input{pdfescape.dtx}
%        (quote the arguments according to the demands of your shell)
%
% Documentation:
%    (a) If pdfescape.drv is present:
%           latex pdfescape.drv
%    (b) Without pdfescape.drv:
%           latex pdfescape.dtx; ...
%    The class ltxdoc loads the configuration file ltxdoc.cfg
%    if available. Here you can specify further options, e.g.
%    use A4 as paper format:
%       \PassOptionsToClass{a4paper}{article}
%
%    Programm calls to get the documentation (example):
%       pdflatex pdfescape.dtx
%       makeindex -s gind.ist pdfescape.idx
%       pdflatex pdfescape.dtx
%       makeindex -s gind.ist pdfescape.idx
%       pdflatex pdfescape.dtx
%
% Installation:
%    TDS:tex/generic/oberdiek/pdfescape.sty
%    TDS:doc/latex/oberdiek/pdfescape.pdf
%    TDS:doc/latex/oberdiek/test/pdfescape-test1.tex
%    TDS:doc/latex/oberdiek/test/pdfescape-test2.tex
%    TDS:doc/latex/oberdiek/test/pdfescape-test3.tex
%    TDS:doc/latex/oberdiek/test/pdfescape-test4.tex
%    TDS:doc/latex/oberdiek/test/pdfescape-test5.tex
%    TDS:doc/latex/oberdiek/test/pdfescape-test6.tex
%    TDS:source/latex/oberdiek/pdfescape.dtx
%
%<*ignore>
\begingroup
  \catcode123=1 %
  \catcode125=2 %
  \def\x{LaTeX2e}%
\expandafter\endgroup
\ifcase 0\ifx\install y1\fi\expandafter
         \ifx\csname processbatchFile\endcsname\relax\else1\fi
         \ifx\fmtname\x\else 1\fi\relax
\else\csname fi\endcsname
%</ignore>
%<*install>
\input docstrip.tex
\Msg{************************************************************************}
\Msg{* Installation}
\Msg{* Package: pdfescape 2011/11/25 v1.13 Implements pdfTeX's escape features (HO)}
\Msg{************************************************************************}

\keepsilent
\askforoverwritefalse

\let\MetaPrefix\relax
\preamble

This is a generated file.

Project: pdfescape
Version: 2011/11/25 v1.13

Copyright (C) 2007, 2010, 2011 by
   Heiko Oberdiek <heiko.oberdiek at googlemail.com>

This work may be distributed and/or modified under the
conditions of the LaTeX Project Public License, either
version 1.3c of this license or (at your option) any later
version. This version of this license is in
   http://www.latex-project.org/lppl/lppl-1-3c.txt
and the latest version of this license is in
   http://www.latex-project.org/lppl.txt
and version 1.3 or later is part of all distributions of
LaTeX version 2005/12/01 or later.

This work has the LPPL maintenance status "maintained".

This Current Maintainer of this work is Heiko Oberdiek.

The Base Interpreter refers to any `TeX-Format',
because some files are installed in TDS:tex/generic//.

This work consists of the main source file pdfescape.dtx
and the derived files
   pdfescape.sty, pdfescape.pdf, pdfescape.ins, pdfescape.drv,
   pdfescape-test1.tex, pdfescape-test2.tex,
   pdfescape-test3.tex, pdfescape-test4.tex,
   pdfescape-test5.tex, pdfescape-test6.tex.

\endpreamble
\let\MetaPrefix\DoubleperCent

\generate{%
  \file{pdfescape.ins}{\from{pdfescape.dtx}{install}}%
  \file{pdfescape.drv}{\from{pdfescape.dtx}{driver}}%
  \usedir{tex/generic/oberdiek}%
  \file{pdfescape.sty}{\from{pdfescape.dtx}{package}}%
  \usedir{doc/latex/oberdiek/test}%
  \file{pdfescape-test1.tex}{\from{pdfescape.dtx}{test1}}%
  \file{pdfescape-test2.tex}{\from{pdfescape.dtx}{test2}}%
  \file{pdfescape-test3.tex}{\from{pdfescape.dtx}{test3}}%
  \file{pdfescape-test4.tex}{\from{pdfescape.dtx}{test4}}%
  \file{pdfescape-test5.tex}{\from{pdfescape.dtx}{test5}}%
  \file{pdfescape-test6.tex}{\from{pdfescape.dtx}{test6}}%
  \nopreamble
  \nopostamble
  \usedir{source/latex/oberdiek/catalogue}%
  \file{pdfescape.xml}{\from{pdfescape.dtx}{catalogue}}%
}

\catcode32=13\relax% active space
\let =\space%
\Msg{************************************************************************}
\Msg{*}
\Msg{* To finish the installation you have to move the following}
\Msg{* file into a directory searched by TeX:}
\Msg{*}
\Msg{*     pdfescape.sty}
\Msg{*}
\Msg{* To produce the documentation run the file `pdfescape.drv'}
\Msg{* through LaTeX.}
\Msg{*}
\Msg{* Happy TeXing!}
\Msg{*}
\Msg{************************************************************************}

\endbatchfile
%</install>
%<*ignore>
\fi
%</ignore>
%<*driver>
\NeedsTeXFormat{LaTeX2e}
\ProvidesFile{pdfescape.drv}%
  [2011/11/25 v1.13 Implements pdfTeX's escape features (HO)]%
\documentclass{ltxdoc}
\usepackage{holtxdoc}[2011/11/22]
\begin{document}
  \DocInput{pdfescape.dtx}%
\end{document}
%</driver>
% \fi
%
% \CheckSum{2516}
%
% \CharacterTable
%  {Upper-case    \A\B\C\D\E\F\G\H\I\J\K\L\M\N\O\P\Q\R\S\T\U\V\W\X\Y\Z
%   Lower-case    \a\b\c\d\e\f\g\h\i\j\k\l\m\n\o\p\q\r\s\t\u\v\w\x\y\z
%   Digits        \0\1\2\3\4\5\6\7\8\9
%   Exclamation   \!     Double quote  \"     Hash (number) \#
%   Dollar        \$     Percent       \%     Ampersand     \&
%   Acute accent  \'     Left paren    \(     Right paren   \)
%   Asterisk      \*     Plus          \+     Comma         \,
%   Minus         \-     Point         \.     Solidus       \/
%   Colon         \:     Semicolon     \;     Less than     \<
%   Equals        \=     Greater than  \>     Question mark \?
%   Commercial at \@     Left bracket  \[     Backslash     \\
%   Right bracket \]     Circumflex    \^     Underscore    \_
%   Grave accent  \`     Left brace    \{     Vertical bar  \|
%   Right brace   \}     Tilde         \~}
%
% \GetFileInfo{pdfescape.drv}
%
% \title{The \xpackage{pdfescape} package}
% \date{2011/11/25 v1.13}
% \author{Heiko Oberdiek\\\xemail{heiko.oberdiek at googlemail.com}}
%
% \maketitle
%
% \begin{abstract}
% This package implements \pdfTeX's escape features
% (\cs{pdfescapehex}, \cs{pdfunescapehex}, \cs{pdfescapename},
% \cs{pdfescapestring}) using \TeX\ or \eTeX.
% \end{abstract}
%
% \tableofcontents
%
% \section{Documentation}
%
% \begin{declcs}{EdefEscapeHex} \M{cmd} \M{string}\\
%  \cs{EdefUnescapeHex} \M{cmd} \M{string}\\
%  \cs{EdefEscapeName} \M{cmd} \M{string}\\
%  \cs{EdefEscapeString} \M{cmd} \M{string}
% \end{declcs}
% These commands converts \meta{string} and stores the result
% in macro \meta{cmd}. The conversion result is the same
% as the conversion of the corresponding \pdfTeX's primitives.
% Note that the argument \meta{string} is expanded before
% the conversion.
%
% For example, if \pdfTeX\ \textgreater= 1.30 is present,
% then \cs{EdefEscapeHex}
% becomes to:
% \begin{quote}
%\begin{verbatim}
%\def\EdefEscapeHex#1#2{%
%  \edef#1{\pdfescapehex{#2}}%
%}
%\end{verbatim}
% \end{quote}
% The package provides implementations for the case that \pdfTeX\ is
% not present (or too old). Even \eTeX\ can be missing, however it is
% used if it is detected.
%
% \paragraph{Babel.}
% The input strings may contain shorthand characters of package
% \xpackage{babel}.
%
% \subsection{Additional unescape macros}
%
% \begin{declcs}{EdefUnescapeName} \M{cmd} \M{string}
% \end{declcs}
% Sequences of a hash sign with two hexadecimal digits are
% converted to the corresponding character (PDF-1.2).
% A hash sign that is not followed by two hexadecimal digits
% is left unchanged. The catcodes in the result string follow
% TeX's conventions. The space has catcode 10 (space) and
% the other characters have catcode 12 (other).
%
% \begin{declcs}{EdefUnescapeString} \M{cmd} \M{string}
% \end{declcs}
% Macro \meta{cmd} stores the unescaped string in \meta{string}.
% All the rules for literal strings are implemented, see
% PDF specification.
% The catcodes in the result string follow TeX's conventions.
%
% \subsection{Sanitizing macro}
%
% \begin{declcs}{EdefSanitize} \M{cmd} \M{string}
% \end{declcs}
% Argument \meta{string} is expanded, converted to a string
% of tokens with catcode 12 (other) and space tokens, and
% stored in macro \meta{cmd}.
%
% \StopEventually{
% }
%
% \section{Implementation}
%
%    \begin{macrocode}
%<*package>
%    \end{macrocode}
%
% \subsection{Reload check and package identification}
%    Reload check, especially if the package is not used with \LaTeX.
%    \begin{macrocode}
\begingroup\catcode61\catcode48\catcode32=10\relax%
  \catcode13=5 % ^^M
  \endlinechar=13 %
  \catcode35=6 % #
  \catcode39=12 % '
  \catcode44=12 % ,
  \catcode45=12 % -
  \catcode46=12 % .
  \catcode58=12 % :
  \catcode64=11 % @
  \catcode123=1 % {
  \catcode125=2 % }
  \expandafter\let\expandafter\x\csname ver@pdfescape.sty\endcsname
  \ifx\x\relax % plain-TeX, first loading
  \else
    \def\empty{}%
    \ifx\x\empty % LaTeX, first loading,
      % variable is initialized, but \ProvidesPackage not yet seen
    \else
      \expandafter\ifx\csname PackageInfo\endcsname\relax
        \def\x#1#2{%
          \immediate\write-1{Package #1 Info: #2.}%
        }%
      \else
        \def\x#1#2{\PackageInfo{#1}{#2, stopped}}%
      \fi
      \x{pdfescape}{The package is already loaded}%
      \aftergroup\endinput
    \fi
  \fi
\endgroup%
%    \end{macrocode}
%    Package identification:
%    \begin{macrocode}
\begingroup\catcode61\catcode48\catcode32=10\relax%
  \catcode13=5 % ^^M
  \endlinechar=13 %
  \catcode35=6 % #
  \catcode39=12 % '
  \catcode40=12 % (
  \catcode41=12 % )
  \catcode44=12 % ,
  \catcode45=12 % -
  \catcode46=12 % .
  \catcode47=12 % /
  \catcode58=12 % :
  \catcode64=11 % @
  \catcode91=12 % [
  \catcode93=12 % ]
  \catcode123=1 % {
  \catcode125=2 % }
  \expandafter\ifx\csname ProvidesPackage\endcsname\relax
    \def\x#1#2#3[#4]{\endgroup
      \immediate\write-1{Package: #3 #4}%
      \xdef#1{#4}%
    }%
  \else
    \def\x#1#2[#3]{\endgroup
      #2[{#3}]%
      \ifx#1\@undefined
        \xdef#1{#3}%
      \fi
      \ifx#1\relax
        \xdef#1{#3}%
      \fi
    }%
  \fi
\expandafter\x\csname ver@pdfescape.sty\endcsname
\ProvidesPackage{pdfescape}%
  [2011/11/25 v1.13 Implements pdfTeX's escape features (HO)]%
%    \end{macrocode}
%
% \subsection{Catcodes}
%
%    \begin{macrocode}
\begingroup\catcode61\catcode48\catcode32=10\relax%
  \catcode13=5 % ^^M
  \endlinechar=13 %
  \catcode123 1 % {
  \catcode125 2 % }
  \catcode64 11 %
  \def\x{\endgroup
    \expandafter\edef\csname PE@AtEnd\endcsname{%
      \endlinechar=\the\endlinechar\relax
      \catcode13=\the\catcode13\relax
      \catcode32=\the\catcode32\relax
      \catcode35=\the\catcode35\relax
      \catcode61=\the\catcode61\relax
      \catcode64=\the\catcode64\relax
      \catcode123=\the\catcode123\relax
      \catcode125=\the\catcode125\relax
    }%
  }%
\x\catcode61\catcode48\catcode32=10\relax%
\catcode13=5 % ^^M
\endlinechar=13 %
\catcode35=6 % #
\catcode64=11 % @
\catcode123=1 % {
\catcode125=2 % }
\def\TMP@EnsureCode#1#2#3{%
  \edef\PE@AtEnd{%
    \PE@AtEnd
    #1#2=\the#1#2\relax
  }%
  #1#2=#3\relax
}
\TMP@EnsureCode\catcode{0}{12}% ^^@
\TMP@EnsureCode\catcode{34}{12}% "
\TMP@EnsureCode\catcode{36}{3}% $
\TMP@EnsureCode\catcode{38}{4}% &
\TMP@EnsureCode\catcode{39}{12}% '
\TMP@EnsureCode\catcode{42}{12}% *
\TMP@EnsureCode\catcode{45}{12}% -
\TMP@EnsureCode\catcode{46}{12}% .
\TMP@EnsureCode\catcode{47}{12}% /
\TMP@EnsureCode\catcode{60}{12}% <
\TMP@EnsureCode\catcode{62}{12}% >
\TMP@EnsureCode\catcode{91}{12}% [
\TMP@EnsureCode\catcode{93}{12}% ]
\TMP@EnsureCode\catcode{94}{7}% ^
\TMP@EnsureCode\catcode{96}{12}% `
\TMP@EnsureCode\uccode{34}{0}% "
\TMP@EnsureCode\uccode{48}{0}% 0
\TMP@EnsureCode\uccode{61}{0}% =
\edef\PE@AtEnd{\PE@AtEnd\noexpand\endinput}
%    \end{macrocode}
%
% \subsection{Load package}
%
%    \begin{macrocode}
\begingroup\expandafter\expandafter\expandafter\endgroup
\expandafter\ifx\csname RequirePackage\endcsname\relax
  \def\TMP@RequirePackage#1[#2]{%
    \begingroup\expandafter\expandafter\expandafter\endgroup
    \expandafter\ifx\csname ver@#1.sty\endcsname\relax
      \input #1.sty\relax
    \fi
  }%
  \TMP@RequirePackage{ltxcmds}[2010/04/08]%
\else
  \RequirePackage{ltxcmds}[2010/04/08]%
\fi
%    \end{macrocode}
%
% \subsection{Sanitizing}
%
%    \begin{macro}{\EdefSanitize}
%    Macro \cs{EdefSanitize} takes |#2|, entirely converts
%    it to token with catcode 12 (other) and stores the
%    result in macro |#1|.
%    \begin{macrocode}
\begingroup\expandafter\expandafter\expandafter\endgroup
\expandafter\ifx\csname detokenize\endcsname\relax
  \long\def\EdefSanitize#1#2{%
    \begingroup
      \csname @safe@activestrue\endcsname
      \edef#1{#2}%
      \PE@onelevel@sanitize#1%
    \expandafter\endgroup
    \expandafter\def\expandafter#1\expandafter{#1}%
  }%
  \begingroup\expandafter\expandafter\expandafter\endgroup
  \expandafter\ifx\csname @onelevel@sanitize\endcsname\relax
    \def\PE@onelevel@sanitize#1{%
      \edef#1{\expandafter\PE@strip@prefix\meaning#1}%
    }%
    \def\PE@strip@prefix#1>{}%
  \else
    \let\PE@onelevel@sanitize\@onelevel@sanitize
  \fi
\else
  \long\def\EdefSanitize#1#2{%
    \begingroup
      \csname @safe@activestrue\endcsname
      \edef#1{#2}%
    \expandafter\endgroup
    \expandafter\def\expandafter#1\expandafter{%
      \detokenize\expandafter{#1}%
    }%
  }%
  \def\PE@onelevel@sanitize#1{%
    \edef#1{\detokenize\expandafter{#1}}%
  }%
\fi
%    \end{macrocode}
%    \end{macro}
%    \begin{macro}{\PE@sanitize}
%    Macro \cs{PE@sanitize} is only defined for
%    compatibility with version 1.4. Its use is deprecated.
%    \begin{macrocode}
\let\PE@sanitize\EdefSanitize
%    \end{macrocode}
%    \end{macro}
%
% \subsubsection{Space characters}
%
%    \begin{macro}{\PE@space@other}
%    \begin{macrocode}
\begingroup
  \catcode`\ =12\relax%
\def\x{\endgroup\def\PE@space@other{ }}\x\relax
%    \end{macrocode}
%    \end{macro}
%    \begin{macro}{\PE@space@space}
%    \begin{macrocode}
\def\PE@space@space{ }
%    \end{macrocode}
%    \end{macro}
%
% \subsubsection{Space normalization}
%
%    \begin{macro}{\PE@SanitizeSpaceOther}
%    \begin{macrocode}
\def\PE@SanitizeSpaceOther#1{%
  \edef#1{\expandafter\PE@SpaceToOther#1 \relax}%
}
%    \end{macrocode}
%    \end{macro}
%
%    \begin{macro}{\PE@SpaceToOther}
%    \begin{macrocode}
\def\PE@SpaceToOther#1 #2\relax{%
  #1%
  \ifx\\#2\\%
  \else
    \PE@space@other
    \ltx@ReturnAfterFi{%
      \PE@SpaceToOther#2\relax
    }%
  \fi
}
%    \end{macrocode}
%    \end{macro}
%
% \subsection{\cs{EdefUnescapeName}}
%
%    \begin{macro}{\EdefUnescapeName}
%    \begin{macrocode}
\def\EdefUnescapeName#1#2{%
  \EdefSanitize#1{#2}%
  \PE@SanitizeSpaceOther#1%
  \PE@UnescapeName#1%
  \PE@onelevel@sanitize#1%
}
%    \end{macrocode}
%    \end{macro}
%
%    \begin{macro}{\PE@UnescapeName}
%    \begin{macrocode}
\begingroup
  \catcode`\$=6  % hash
  \catcode`\#=12 % other
  \gdef\PE@UnescapeName$1{%
    \begingroup
      \PE@InitUccodeHexDigit
      \def\PE@result{}%
      \expandafter\PE@DeName$1#\relax\relax
    \expandafter\endgroup
    \expandafter\def\expandafter$1\expandafter{\PE@result}%
  }%
  \gdef\PE@DeName$1#$2$3{%
    \ifx\relax$2%
      \edef\PE@result{\PE@result$1}%
      \let\PE@next\relax
    \else
      \ifx\relax$3%
        % wrong escape sequence in input
        \edef\PE@result{\PE@result$1#}%
        \let\PE@next\relax
      \else
        \uppercase{%
          \def\PE@testA{$2}%
          \def\PE@testB{$3}%
        }%
        \ifcase\ifcase\expandafter\PE@TestUcHexDigit\PE@testA
                 \ifcase\expandafter\PE@TestUcHexDigit\PE@testB
                   \ltx@zero
                 \else
                   \ltx@one
                 \fi
               \else
                 \ltx@one
               \fi
          \uccode\ltx@zero="\PE@testA\PE@testB\relax
          \uppercase{%
            \def\PE@temp{^^@}%
          }%
          \uccode\ltx@zero=\ltx@zero
          \edef\PE@result{\PE@result$1\PE@temp}%
          \let\PE@next\PE@DeName
        \else
          % wrong escape sequence in input
          \edef\PE@result{\PE@result$1#}%
          \def\PE@next{\PE@DeName$2$3}%
        \fi
      \fi
    \fi
    \PE@next
  }%
\endgroup
%    \end{macrocode}
%    \end{macro}
%
%    \begin{macro}{\PE@InitUccodeHexDigit}
%    \begin{macrocode}
\def\PE@InitUccodeHexDigit{%
  \uccode`a=`A\relax
  \uccode`b=`B\relax
  \uccode`c=`C\relax
  \uccode`d=`D\relax
  \uccode`e=`E\relax
  \uccode`f=`F\relax
  \uccode`A=\ltx@zero
  \uccode`B=\ltx@zero
  \uccode`C=\ltx@zero
  \uccode`D=\ltx@zero
  \uccode`E=\ltx@zero
  \uccode`F=\ltx@zero
  \uccode`0=\ltx@zero
  \uccode`1=\ltx@zero
  \uccode`2=\ltx@zero
  \uccode`3=\ltx@zero
  \uccode`4=\ltx@zero
  \uccode`5=\ltx@zero
  \uccode`6=\ltx@zero
  \uccode`7=\ltx@zero
  \uccode`8=\ltx@zero
  \uccode`9=\ltx@zero
}
%    \end{macrocode}
%    \end{macro}
%
%    \begin{macro}{\PE@TestUcHexDigit}
%    \begin{macrocode}
\def\PE@TestUcHexDigit#1{%
  \ifnum`#1<48 % 0
    \ltx@one
  \else
    \ifnum`#1>70 % F
      \ltx@one
    \else
      \ifnum`#1>57 % 9
        \ifnum`#1<65 % A
          \ltx@one
        \else
          \ltx@zero
        \fi
      \else
        \ltx@zero
      \fi
    \fi
  \fi
}
%    \end{macrocode}
%    \end{macro}
%
% \subsection{\cs{EdefUnescapeString}}
%
%    \begin{macro}{\EdefUnescapeString}
%    \begin{macrocode}
\def\EdefUnescapeString#1#2{%
  \EdefSanitize#1{#2}%
  \PE@SanitizeSpaceOther#1%
  \PE@NormalizeLineEnd#1%
  \PE@UnescapeString#1%
  \PE@onelevel@sanitize#1%
}
%    \end{macrocode}
%    \end{macro}
%
%    \begin{macrocode}
\begingroup
  \uccode`\8=10 % lf
  \uccode`\9=13 % cr
\def\x#1#2{\endgroup
%    \end{macrocode}
%    \begin{macro}{\PE@NormalizeLineEnd}
%    \begin{macrocode}
  \def\PE@NormalizeLineEnd##1{%
    \def\PE@result{}%
    \expandafter\PE@@NormalizeLineEnd##1#2\relax
    \let##1\PE@result
  }%
%    \end{macrocode}
%    \end{macro}
%    \begin{macro}{\PE@@NormalizeLineEnd}
%    \begin{macrocode}
  \def\PE@@NormalizeLineEnd##1#2##2{%
    \ifx\relax##2%
      \edef\PE@result{\PE@result##1}%
      \let\PE@next\relax
    \else
      \edef\PE@result{\PE@result##1#1}%
      \ifx#1##2% lf
        \let\PE@next\PE@@NormalizeLineEnd
      \else
        \def\PE@next{\PE@@NormalizeLineEnd##2}%
      \fi
    \fi
    \PE@next
  }%
}%
\uppercase{%
  \x 89%
}
%    \end{macrocode}
%    \end{macro}
%
%    \begin{macrocode}
\begingroup
  \catcode`\|=0 %
  \catcode`\\=12 %
%    \end{macrocode}
%    \begin{macro}{\PE@UnescapeString}
%    \begin{macrocode}
  |gdef|PE@UnescapeString#1{%
    |begingroup
      |def|PE@result{}%
      |expandafter|PE@DeString#1\|relax
    |expandafter|endgroup
    |expandafter|def|expandafter#1|expandafter{|PE@result}%
  }%
%    \end{macrocode}
%    \end{macro}
%    \begin{macro}{\PE@DeString}
%    \begin{macrocode}
  |gdef|PE@DeString#1\#2{%
    |ifx|relax#2%
      |edef|PE@result{|PE@result#1}%
      |let|PE@next|relax
    |else
      |if n#2%
        |uccode|ltx@zero=10 %
      |else|if r#2%
        |uccode|ltx@zero=13 %
      |else|if t#2%
        |uccode|ltx@zero=9 %
      |else|if b#2%
        |uccode|ltx@zero=8 %
      |else|if f#2%
        |uccode|ltx@zero=12 %
      |else
        |uccode|ltx@zero=|ltx@zero
      |fi|fi|fi|fi|fi
      |ifnum|uccode|ltx@zero>|ltx@zero
        |uppercase{%
          |edef|PE@temp{^^@}%
        }%
        |edef|PE@result{|PE@result#1|PE@temp}%
        |let|PE@next|PE@DeString
      |else
        |if\#2% backslash
          |edef|PE@result{|PE@result#1}%
          |let|PE@next|PE@CheckEndBackslash
        |else
          |ifnum`#2=10 % linefeed
            |edef|PE@result{|PE@result#1}%
            |let|PE@next|PE@DeString
          |else
            |ifcase|PE@TestOctDigit#2%
              |edef|PE@result{|PE@result#1}%
              |def|PE@next{|PE@OctI#2}%
            |else
              |edef|PE@result{|PE@result#1#2}%
              |let|PE@next|PE@DeString
            |fi
          |fi
        |fi
      |fi
    |fi
    |PE@next
  }%
%    \end{macrocode}
%    \end{macro}
%    \begin{macro}{\PE@CheckEndBackslash}
%    \begin{macrocode}
  |gdef|PE@CheckEndBackslash#1{%
    |ifx|relax#1%
    |else
      |edef|PE@result{|PE@result\}%
      |expandafter|PE@DeString|expandafter#1%
    |fi
  }%
%    \end{macrocode}
%    \end{macro}
%    \begin{macrocode}
|endgroup
%    \end{macrocode}
%    \begin{macro}{\PE@TestOctDigit}
%    \begin{macrocode}
\def\PE@TestOctDigit#1{%
  \ifnum`#1<48 % 0
    \ltx@one
  \else
    \ifnum`#1>55 % 7
      \ltx@one
    \else
      \ltx@zero
    \fi
  \fi
}
%    \end{macrocode}
%    \end{macro}
%    \begin{macro}{\PE@OctI}
%    \begin{macrocode}
\def\PE@OctI#1#2{%
  \ifcase\PE@TestOctDigit#2%
    \def\PE@next{\PE@OctII{#1#2}}%
  \else
    \def\PE@next{\PE@OctAll#1#2}%
  \fi
  \PE@next
}
%    \end{macrocode}
%    \end{macro}
%    \begin{macro}{\PE@OctII}
%    \begin{macrocode}
\def\PE@OctII#1#2{%
  \ifcase\PE@TestOctDigit#2%
    \def\PE@next{\PE@OctIII#1#2}%
  \else
    \def\PE@next{\PE@OctAll{#1}#2}%
  \fi
  \PE@next
}
%    \end{macrocode}
%    \end{macro}
%    \begin{macrocode}
\ltx@IfUndefined{numexpr}{%
  \catcode`\$=9 %
  \catcode`\&=14 %
}{%
  \catcode`\$=14 %
  \catcode`\&=9 %
}
%    \end{macrocode}
%    \begin{macro}{\PE@OctIII}
%    \begin{macrocode}
\def\PE@OctIII#1#2#3{%
  \ifnum#1<4 %
    \def\PE@next{\PE@OctAll{#1#2#3}}%
  \else
$   \count\ltx@cclv#1 %
$   \advance\count\ltx@cclv -4 %
    \edef\PE@next{%
      \noexpand\PE@OctAll{%
$       \the\count\ltx@cclv
&       \the\numexpr#1-4\relax
        #2#3%
      }%
    }%
  \fi
  \PE@next
}
%    \end{macrocode}
%    \end{macro}
%    \begin{macro}{\PE@OctAll}
%    \begin{macrocode}
\def\PE@OctAll#1{%
  \uccode\ltx@zero='#1\relax
  \uppercase{%
    \edef\PE@result{\PE@result^^@}%
  }%
  \PE@DeString
}
%    \end{macrocode}
%    \end{macro}
%
% \subsection{User macros (\pdfTeX\space analogues)}
%
%    \begin{macrocode}
\begingroup\expandafter\expandafter\expandafter\endgroup
\expandafter\ifx\csname RequirePackage\endcsname\relax
  \def\TMP@RequirePackage#1[#2]{%
    \begingroup\expandafter\expandafter\expandafter\endgroup
    \expandafter\ifx\csname ver@#1.sty\endcsname\relax
      \input #1.sty\relax
    \fi
  }%
  \TMP@RequirePackage{pdftexcmds}[2007/11/11]%
\else
  \RequirePackage{pdftexcmds}[2007/11/11]%
\fi
%    \end{macrocode}
%
%    \begin{macrocode}
\begingroup\expandafter\expandafter\expandafter\endgroup
\expandafter\ifx\csname pdf@escapehex\endcsname\relax
%    \end{macrocode}
%
%    \begin{macro}{\EdefEscapeHex}
%    \begin{macrocode}
  \long\def\EdefEscapeHex#1#2{%
    \EdefSanitize#1{#2}%
    \PE@SanitizeSpaceOther#1%
    \PE@EscapeHex#1%
  }%
%    \end{macrocode}
%    \end{macro}
%    \begin{macro}{\EdefUnescapeHex}
%    \begin{macrocode}
  \def\EdefUnescapeHex#1#2{%
    \EdefSanitize#1{#2}%
    \PE@UnescapeHex#1%
  }%
%    \end{macrocode}
%    \end{macro}
%    \begin{macro}{\EdefEscapeName}
%    \begin{macrocode}
  \long\def\EdefEscapeName#1#2{%
    \EdefSanitize#1{#2}%
    \PE@SanitizeSpaceOther#1%
    \PE@EscapeName#1%
  }%
%    \end{macrocode}
%    \end{macro}
%    \begin{macro}{\EdefEscapeString}
%    \begin{macrocode}
  \long\def\EdefEscapeString#1#2{%
    \EdefSanitize#1{#2}%
    \PE@SanitizeSpaceOther#1%
    \PE@EscapeString#1%
  }%
%    \end{macrocode}
%    \end{macro}
%
%    \begin{macrocode}
\else
%    \end{macrocode}
%
%    \begin{macro}{\PE@edefbabel}
%    Help macro that adds support for \xpackage{babel}'s shorthand
%    characters.
%    \begin{macrocode}
  \long\def\PE@edefbabel#1#2#3{%
    \begingroup
      \csname @save@activestrue\endcsname
      \edef#1{#2{#3}}%
    \expandafter\endgroup
    \expandafter\def\expandafter#1\expandafter{#1}%
  }%
%    \end{macrocode}
%    \end{macro}
%
%    \begin{macro}{\EdefEscapeHex}
%    \begin{macrocode}
  \long\def\EdefEscapeHex#1#2{%
    \PE@edefbabel#1\pdf@escapehex{#2}%
  }%
%    \end{macrocode}
%    \end{macro}
%    \begin{macro}{\EdefUnescapeHex}
%    \begin{macrocode}
  \def\EdefUnescapeHex#1#2{%
    \PE@edefbabel#1\pdf@unescapehex{#2}%
  }%
%    \end{macrocode}
%    \end{macro}
%    \begin{macro}{\EdefEscapeName}
%    \begin{macrocode}
  \long\def\EdefEscapeName#1#2{%
    \PE@edefbabel#1\pdf@escapename{#2}%
  }%
%    \end{macrocode}
%    \end{macro}
%    \begin{macro}{\EdefEscapeString}
%    \begin{macrocode}
  \long\def\EdefEscapeString#1#2{%
    \PE@edefbabel#1\pdf@escapestring{#2}%
  }%
%    \end{macrocode}
%    \end{macro}
%
%    \begin{macrocode}
  \expandafter\PE@AtEnd
\fi%
%    \end{macrocode}
%
% \subsection{Help macros}
%
% \subsubsection{Characters}
%
%    Special characters with catcode 12 (other) are created and
%    stored in macros.
%    \begin{macro}{\PE@hash}
%    \begin{macrocode}
\edef\PE@hash{\string#}
%    \end{macrocode}
%    \end{macro}
%    \begin{macro}{\PE@backslash}
%    \begin{macrocode}
\begingroup
  \escapechar=-1 %
\edef\x{\endgroup
  \def\noexpand\PE@backslash{\string\\}%
}
\x
%    \end{macrocode}
%    \end{macro}
%
% \subsubsection{Switch for \eTeX}
%
%    \begin{macrocode}
\ltx@newif\ifPE@etex
\begingroup\expandafter\expandafter\expandafter\endgroup
\expandafter\ifx\csname numexpr\endcsname\relax
\else
  \PE@etextrue
\fi
%    \end{macrocode}
%
% \subsection{Conversions}
%
% \subsubsection{Conversion to hex string}
%
%    \begin{macro}{\PE@EscapeHex}
%    \begin{macrocode}
\ifPE@etex
  \def\PE@EscapeHex#1{%
    \edef#1{\expandafter\PE@ToHex#1\relax}%
  }%
\else
  \def\PE@EscapeHex#1{%
    \def\PE@result{}%
    \expandafter\PE@ToHex#1\relax
    \let#1\PE@result
  }%
\fi
%    \end{macrocode}
%    \end{macro}
%    \begin{macro}{\PE@ToHex}
%    \begin{macrocode}
\def\PE@ToHex#1{%
  \ifx\relax#1%
  \else
    \PE@HexChar{#1}%
    \expandafter\PE@ToHex
  \fi
}%
%    \end{macrocode}
%    \end{macro}
%    \begin{macro}{\PE@HexChar}
%    \begin{macrocode}
\ifPE@etex
  \def\PE@HexChar#1{%
    \PE@HexDigit{\numexpr\dimexpr.0625\dimexpr`#1sp\relax\relax\relax}%
    \PE@HexDigit{%
      \numexpr`#1-16*\dimexpr.0625\dimexpr`#1sp\relax\relax\relax
    }%
  }%
\else
  \def\PE@HexChar#1{%
    \dimen0=`#1sp%
    \dimen2=.0625\dimen0 %
    \advance\dimen0-16\dimen2 %
    \edef\PE@result{%
      \PE@result
      \PE@HexDigit{\dimen2 }%
      \PE@HexDigit{\dimen0 }%
    }%
  }%
\fi
%    \end{macrocode}
%    \end{macro}
%    \begin{macro}{\PE@HexDigit}
%    \begin{macrocode}
\def\PE@HexDigit#1{%
  \expandafter\string
  \ifcase#1%
    0\or 1\or 2\or 3\or 4\or 5\or 6\or 7\or 8\or 9\or
    A\or B\or C\or D\or E\or F%
  \fi
}
%    \end{macrocode}
%    \end{macro}
%
% \subsubsection{Character code to octal number}
%
%    \begin{macro}{\PE@OctChar}
%    \begin{macrocode}
\ifPE@etex
  \def\PE@OctChar#1{%
    \expandafter\PE@@OctChar
        \the\numexpr\dimexpr.015625\dimexpr`#1sp\relax\relax
            \expandafter\relax
        \expandafter\relax
        \the\numexpr\dimexpr.125\dimexpr`#1sp\relax\relax\relax
        \relax
        #1%
  }%
  \def\PE@@OctChar#1\relax#2\relax#3{%
    \PE@backslash
    #1%
    \the\numexpr#2-8*#1\relax
    \the\numexpr\dimexpr`#3sp\relax-8*#2\relax
  }%
\else
  \def\PE@OctChar#1{%
    \dimen0=`#1sp%
    \dimen2=.125\dimen0 %
    \dimen4=.125\dimen2 %
    \advance\dimen0-8\dimen2 %
    \advance\dimen2-8\dimen4 %
    \edef\PE@result{%
      \PE@result
      \PE@backslash
      \number\dimen4 %
      \number\dimen2 %
      \number\dimen0 %
    }%
  }%
\fi
%    \end{macrocode}
%    \end{macro}
%
% \subsubsection{Unpack hex string}
%
%    \begin{macro}{\PE@UnescapeHex}
%    \begin{macrocode}
\def\PE@UnescapeHex#1{%
  \begingroup
    \PE@InitUccodeHexDigit
    \def\PE@result{}%
    \expandafter\PE@DeHex#1\relax\relax
  \expandafter\endgroup
  \expandafter\def\expandafter#1\expandafter{\PE@result}%
}
%    \end{macrocode}
%    \end{macro}
%    \begin{macro}{\PE@DeHex}
%    \begin{macrocode}
\def\PE@DeHex#1#2{%
  \ifx#2\relax
    \ifx#1\relax
      \let\PE@next\relax
    \else
      \uppercase{%
        \def\PE@testA{#1}%
      }%
      \ifcase\expandafter\PE@TestUcHexDigit\PE@testA
        \def\PE@next{%
          \PE@DeHex#10\relax\relax
        }%
      \else
        \let\PE@next\relax
      \fi
    \fi
  \else
    \uppercase{%
      \def\PE@testA{#1}%
      \def\PE@testB{#2}%
    }%
    \ifcase\expandafter\PE@TestUcHexDigit\PE@testA
      \ifcase\expandafter\PE@TestUcHexDigit\PE@testB
        \uccode\ltx@zero="\PE@testA\PE@testB\relax
        \ifnum\uccode\ltx@zero=32 %
          \let\PE@temp\PE@space@space
        \else
          \uppercase{%
            \def\PE@temp{^^@}%
          }%
        \fi
        \edef\PE@result{\PE@result\PE@temp}%
        \let\PE@next\PE@DeHex
      \else
        % invalid input sequence
        \def\PE@next{%
          \PE@DeHex#1%
        }%
      \fi
    \else
      % invalid input sequence
      \def\PE@next{\PE@DeHex#2}%
    \fi
  \fi
  \PE@next
}
%    \end{macrocode}
%    \end{macro}
%
% \subsubsection{Conversion to PDF name}
%
%    \begin{macro}{\PE@EscapeName}
%    \begin{macrocode}
\ifPE@etex
  \def\PE@EscapeName#1{%
    \edef#1{\expandafter\PE@EscapeNameTokens#1\relax}%
  }%
\else
  \def\PE@EscapeName#1{%
    \def\PE@result{}%
    \expandafter\PE@EscapeNameTokens#1\relax
    \let#1\PE@result
  }%
\fi
%    \end{macrocode}
%    \end{macro}
%    \begin{macro}{\PE@EscapeNameTokens}
%    \begin{macrocode}
\def\PE@EscapeNameTokens#1{%
  \ifx\relax#1%
  \else
    \ifnum`#1<33 %
      \ifcase`#1 %
        % drop illegal zero
      \else
        \PE@EscapeNameAdd\PE@hash
        \PE@HexChar#1%
      \fi
    \else
      \ifnum`#1>126 %
        \PE@EscapeNameAdd\PE@hash
        \PE@HexChar#1%
      \else  \ifnum`#1=35  \PE@EscapeNameHashChar 23% #
        \else\ifnum`#1=37  \PE@EscapeNameHashChar 25% %
        \else\ifnum`#1=40  \PE@EscapeNameHashChar 28% (
        \else\ifnum`#1=41  \PE@EscapeNameHashChar 29% )
        \else\ifnum`#1=47  \PE@EscapeNameHashChar 2F% /
        \else\ifnum`#1=60  \PE@EscapeNameHashChar 3C% <
        \else\ifnum`#1=62  \PE@EscapeNameHashChar 3E% >
        \else\ifnum`#1=91  \PE@EscapeNameHashChar 5B% [
        \else\ifnum`#1=93  \PE@EscapeNameHashChar 5D% ]
        \else\ifnum`#1=123 \PE@EscapeNameHashChar 7B% {
        \else\ifnum`#1=125 \PE@EscapeNameHashChar 7D% }
        \else
          \PE@EscapeNameAdd{#1}%
        \fi\fi\fi\fi\fi\fi\fi\fi\fi\fi\fi
      \fi
    \fi
    \expandafter\PE@EscapeNameTokens
  \fi
}%
\def\PE@EscapeNameHashChar#1#2{%
  \PE@EscapeNameAdd{\PE@hash\string#1\string#2}%
}%
%    \end{macrocode}
%    \end{macro}
%    \begin{macro}{\PE@EscapeNameAdd}
%    \begin{macrocode}
\ifPE@etex
  \def\PE@EscapeNameAdd#1{#1}%
\else
  \def\PE@EscapeNameAdd#1{%
    \edef\PE@result{%
      \PE@result
      #1%
    }%
  }%
\fi
%    \end{macrocode}
%    \end{macro}
%
% \subsubsection{Conversion to PDF string}
%
%    \begin{macro}{\PE@EscapeString}
%    \begin{macrocode}
\ifPE@etex
  \def\PE@EscapeString#1{%
    \edef#1{\expandafter\PE@EscapeStringTokens#1\relax}%
  }%
\else
  \def\PE@EscapeString#1{%
    \begingroup
      \def\PE@result{}%
      \expandafter\PE@EscapeStringTokens#1\relax
    \expandafter\endgroup
    \expandafter\def\expandafter#1\expandafter{\PE@result}%
  }%
\fi
%    \end{macrocode}
%    \end{macro}
%    \begin{macro}{\PE@EscapeStringTokens}
%    \begin{macrocode}
\def\PE@EscapeStringTokens#1{%
  \ifx\relax#1%
  \else
    \ifnum`#1<33 %
      \PE@OctChar#1%
    \else
      \ifnum`#1>126 %
        \PE@OctChar#1%
      \else  \ifnum`#1=40 \PE@EscapeStringAdd{\string\(}% (
        \else\ifnum`#1=41 \PE@EscapeStringAdd{\string\)}% )
        \else\ifnum`#1=92 \PE@EscapeStringAdd{\string\\}% \
        \else
          \PE@EscapeStringAdd{#1}%
        \fi\fi\fi
      \fi
    \fi
    \expandafter\PE@EscapeStringTokens
  \fi
}%
%    \end{macrocode}
%    \end{macro}
%    \begin{macro}{\PE@EscapeStringAdd}
%    \begin{macrocode}
\ifPE@etex
  \def\PE@EscapeStringAdd#1{#1}%
\else
  \def\PE@EscapeStringAdd#1{%
    \edef\PE@result{%
      \PE@result
      #1%
    }%
  }%
\fi
%    \end{macrocode}
%    \end{macro}
%
%    \begin{macrocode}
\PE@AtEnd%
%    \end{macrocode}
%
%    \begin{macrocode}
%</package>
%    \end{macrocode}
%
% \section{Test}
%
% \subsection{Catcode checks for loading}
%
%    \begin{macrocode}
%<*test1>
%    \end{macrocode}
%    \begin{macrocode}
\catcode`\{=1 %
\catcode`\}=2 %
\catcode`\#=6 %
\catcode`\@=11 %
\expandafter\ifx\csname count@\endcsname\relax
  \countdef\count@=255 %
\fi
\expandafter\ifx\csname @gobble\endcsname\relax
  \long\def\@gobble#1{}%
\fi
\expandafter\ifx\csname @firstofone\endcsname\relax
  \long\def\@firstofone#1{#1}%
\fi
\expandafter\ifx\csname loop\endcsname\relax
  \expandafter\@firstofone
\else
  \expandafter\@gobble
\fi
{%
  \def\loop#1\repeat{%
    \def\body{#1}%
    \iterate
  }%
  \def\iterate{%
    \body
      \let\next\iterate
    \else
      \let\next\relax
    \fi
    \next
  }%
  \let\repeat=\fi
}%
\def\RestoreCatcodes{}
\count@=0 %
\loop
  \edef\RestoreCatcodes{%
    \RestoreCatcodes
    \catcode\the\count@=\the\catcode\count@\relax
  }%
\ifnum\count@<255 %
  \advance\count@ 1 %
\repeat

\def\RangeCatcodeInvalid#1#2{%
  \count@=#1\relax
  \loop
    \catcode\count@=15 %
  \ifnum\count@<#2\relax
    \advance\count@ 1 %
  \repeat
}
\def\RangeCatcodeCheck#1#2#3{%
  \count@=#1\relax
  \loop
    \ifnum#3=\catcode\count@
    \else
      \errmessage{%
        Character \the\count@\space
        with wrong catcode \the\catcode\count@\space
        instead of \number#3%
      }%
    \fi
  \ifnum\count@<#2\relax
    \advance\count@ 1 %
  \repeat
}
\def\space{ }
\expandafter\ifx\csname LoadCommand\endcsname\relax
  \def\LoadCommand{\input pdfescape.sty\relax}%
\fi
\def\Test{%
  \RangeCatcodeInvalid{0}{47}%
  \RangeCatcodeInvalid{58}{64}%
  \RangeCatcodeInvalid{91}{96}%
  \RangeCatcodeInvalid{123}{255}%
  \catcode`\@=12 %
  \catcode`\\=0 %
  \catcode`\%=14 %
  \LoadCommand
  \RangeCatcodeCheck{0}{36}{15}%
  \RangeCatcodeCheck{37}{37}{14}%
  \RangeCatcodeCheck{38}{47}{15}%
  \RangeCatcodeCheck{48}{57}{12}%
  \RangeCatcodeCheck{58}{63}{15}%
  \RangeCatcodeCheck{64}{64}{12}%
  \RangeCatcodeCheck{65}{90}{11}%
  \RangeCatcodeCheck{91}{91}{15}%
  \RangeCatcodeCheck{92}{92}{0}%
  \RangeCatcodeCheck{93}{96}{15}%
  \RangeCatcodeCheck{97}{122}{11}%
  \RangeCatcodeCheck{123}{255}{15}%
  \RestoreCatcodes
}
\Test
\csname @@end\endcsname
\end
%    \end{macrocode}
%    \begin{macrocode}
%</test1>
%    \end{macrocode}
%
% \subsection{Macro tests}
%
%    \begin{macrocode}
%<*test2|test3|test4|test5>
\NeedsTeXFormat{LaTeX2e}
\makeatletter
%    \end{macrocode}
%
% \subsection{Test with \cs{pdfescape...} commands}
%
%    \begin{macrocode}
%<*test2>
\ProvidesFile{pdfescape-test2.tex}%
    [2011/11/25 v1.13 Test with \string\pdfescape... commands]%
%</test2>
%    \end{macrocode}
%
% \subsection{Test without \cs{pdfescape...}, with \eTeX}
%
%    \begin{macrocode}
%<*test3>
\ProvidesFile{pdfescape-test3.tex}%
    [2011/11/25 v1.13 Test without \string\pdfescape..., with e-TeX]%
%</test3>
%    \end{macrocode}
%
% \subsection{Test without \cs{pdfescape...} and \eTeX}
%
%    \begin{macrocode}
%<*test4>
\ProvidesFile{pdfescape-test4.tex}%
    [2011/11/25 v1.13 Test without \string\pdfescape... and e-TeX]%
%</test4>
%    \end{macrocode}
%
% \subsection{Test with \LuaTeX}
%
%    \begin{macrocode}
%<*test5>
\ProvidesFile{pdfescape-test5.tex}%
    [2011/11/25 v1.13 Test with LuaTeX]%
%</test5>
%    \end{macrocode}
%
% \subsection{Check/ensure test preconditions}
%
% \subsubsection{Check \cs{pdfescape...}}
%
%    \begin{macrocode}
%<*test2>
\@ifundefined{pdfescapehex}{%
  \PackageError{pdfescape-test2}{%
    Missing \string\pdfescape... commands%
  }{Test aborted.}%
  \stop
}{}
%</test2>
%    \end{macrocode}
%    \begin{macrocode}
%<*test3|test4>
\let\pdfescapehex\@undefined
\let\pdfunescapehex\@undefined
\let\pdfescapename\@undefined
\let\pdfescapestring\@undefined
%</test3|test4>
%    \end{macrocode}
%
% \subsubsection{Check \eTeX}
%
%    \begin{macrocode}
%<*test3>
\@ifundefined{numexpr}{%
  \PackageError{pdfescape-test3}{%
    Missing \eTeX
  }{Test aborted.}%
  \stop
}{}
%</test3>
%    \end{macrocode}
% Package \xpackage{qstest} uses \eTeX, thus \eTeX's features
% can only be disabled later during loading of package \xpackage{pdfescape}.
%
% \subsubsection{Check \LuaTeX}
%
%    \begin{macrocode}
%<*test5>
\@ifundefined{directlua}{%
  \PackageError{pdfescape-test5}{%
    Missing LuaTeX%
  }{Test aborted.}%
  \stop
}{}
%</test5>
%    \end{macrocode}
%
% \subsection{Common part}
%
%    The files for testing uses the framework, provided by the
%    new package \xpackage{qstest} of David Kastrup.
%    \begin{macrocode}
\RequirePackage{qstest}
\IncludeTests{*}
\LogTests{log}{*}{*}

\newcommand*{\ExpectVar}[2]{%
  \ifx#1#2%
  \else
    \begingroup
      \@onelevel@sanitize#1%
      \@onelevel@sanitize#2%
      \typeout{[#1] <> [#2]}% hash-ok
    \endgroup
  \fi
  \Expect*{\ifx#1#2true\else false\fi}{true}%
}

\makeatletter
\begingroup
  \gdef\AllBytes{}%
  \count@=0 %
  \catcode0=12 %
  \@whilenum\count@<256 \do{%
    \lccode0=\count@
    \ifnum\count@=32 %
      \xdef\AllBytes{\AllBytes\space}%
    \else
      \lowercase{%
        \xdef\AllBytes{\AllBytes^^@}%
      }%
    \fi
    \advance\count@ by 1 %
  }%
\endgroup
\newcommand*{\AllBytesHex}{%
  000102030405060708090A0B0C0D0E0F%
  101112131415161718191A1B1C1D1E1F%
  202122232425262728292A2B2C2D2E2F%
  303132333435363738393A3B3C3D3E3F%
  404142434445464748494A4B4C4D4E4F%
  505152535455565758595A5B5C5D5E5F%
  606162636465666768696A6B6C6D6E6F%
  707172737475767778797A7B7C7D7E7F%
  808182838485868788898A8B8C8D8E8F%
  909192939495969798999A9B9C9D9E9F%
  A0A1A2A3A4A5A6A7A8A9AAABACADAEAF%
  B0B1B2B3B4B5B6B7B8B9BABBBCBDBEBF%
  C0C1C2C3C4C5C6C7C8C9CACBCCCDCECF%
  D0D1D2D3D4D5D6D7D8D9DADBDCDDDEDF%
  E0E1E2E3E4E5E6E7E8E9EAEBECEDEEEF%
  F0F1F2F3F4F5F6F7F8F9FAFBFCFDFEFF%
}
\@onelevel@sanitize\AllBytesHex
\expandafter\lowercase\expandafter{%
  \expandafter\newcommand\expandafter*\expandafter\AllBytesHexLC
      \expandafter{\AllBytesHex}%
}
\newcommand*{\AllBytesName}{}
\begingroup
  \catcode`\#=12 %
  \xdef\AllBytesName{%
    #01#02#03#04#05#06#07#08#09#0A#0B#0C#0D#0E#0F%
    #10#11#12#13#14#15#16#17#18#19#1A#1B#1C#1D#1E#1F%
    #20!"#23$#25&'#28#29*+,-.#2F%
    0123456789:;#3C=#3E?%
    @ABCDEFGHIJKLMNO%
    PQRSTUVWXYZ#5B\@backslashchar#5D^_%
    `abcdefghijklmno%
    pqrstuvwxyz#7B|#7D\string~#7F%
    #80#81#82#83#84#85#86#87#88#89#8A#8B#8C#8D#8E#8F%
    #90#91#92#93#94#95#96#97#98#99#9A#9B#9C#9D#9E#9F%
    #A0#A1#A2#A3#A4#A5#A6#A7#A8#A9#AA#AB#AC#AD#AE#AF%
    #B0#B1#B2#B3#B4#B5#B6#B7#B8#B9#BA#BB#BC#BD#BE#BF%
    #C0#C1#C2#C3#C4#C5#C6#C7#C8#C9#CA#CB#CC#CD#CE#CF%
    #D0#D1#D2#D3#D4#D5#D6#D7#D8#D9#DA#DB#DC#DD#DE#DF%
    #E0#E1#E2#E3#E4#E5#E6#E7#E8#E9#EA#EB#EC#ED#EE#EF%
    #F0#F1#F2#F3#F4#F5#F6#F7#F8#F9#FA#FB#FC#FD#FE#FF%
  }%
\endgroup
\@onelevel@sanitize\AllBytesName

\newcommand*{\AllBytesString}{}
\begingroup
  \def\|{|}%
  \edef\%{\@percentchar}%
  \catcode`\|=0 %
  \catcode`\#=12 %
  \catcode`\~=12 %
  \catcode`\\=12 %
  |xdef|AllBytesString{%
    \000\001\002\003\004\005\006\007\010\011\012\013\014\015\016\017%
    \020\021\022\023\024\025\026\027\030\031\032\033\034\035\036\037%
    \040!"#$|%&'\(\)*+,-./%
    0123456789:;<=>?%
    @ABCDEFGHIJKLMNO%
    PQRSTUVWXYZ[\\]^_%
    `abcdefghijklmno%
    pqrstuvwxyz{||}~\177%
    \200\201\202\203\204\205\206\207\210\211\212\213\214\215\216\217%
    \220\221\222\223\224\225\226\227\230\231\232\233\234\235\236\237%
    \240\241\242\243\244\245\246\247\250\251\252\253\254\255\256\257%
    \260\261\262\263\264\265\266\267\270\271\272\273\274\275\276\277%
    \300\301\302\303\304\305\306\307\310\311\312\313\314\315\316\317%
    \320\321\322\323\324\325\326\327\330\331\332\333\334\335\336\337%
    \340\341\342\343\344\345\346\347\350\351\352\353\354\355\356\357%
    \360\361\362\363\364\365\366\367\370\371\372\373\374\375\376\377%
  }%
|endgroup
\@onelevel@sanitize\AllBytesString

%<*test4>
\let\org@detokenize\detokenize
\let\detokenize\@undefined
\let\org@numexpr\numexpr
\let\numexpr\@undefined
%</test4>
\RequirePackage{pdfescape}
%<*test4>
\let\detokenize\org@detokenize
\let\numexpr\org@numexpr
%</test4>

\begin{qstest}{all-hex}{\AllBytes, escapehex}
  \EdefEscapeHex\x{\AllBytes}%
  \Expect*{\x}*{\AllBytesHex}%
  \ExpectVar\x\AllBytesHex
\end{qstest}

\begin{qstest}{all-unhex}{\AllBytesHex, unescapehex}
  \EdefUnescapeHex\x{\AllBytesHex}%
  \Expect*{\x}*{\AllBytes}%
  \ExpectVar\x\AllBytes
\end{qstest}

\begin{qstest}{all-unhex-lc}{\AllBytesHexLC, unescapehex, lowercase}
  \EdefUnescapeHex\x{\AllBytesHexLC}%
  \Expect*{\x}*{\AllBytes}%
  \ExpectVar\x\AllBytes
\end{qstest}

\begin{qstest}{unhex-incomplete}{unescapehex, incomplete}
  \EdefUnescapeHex\x{4}%
  \Expect*{\x}{@}%
\end{qstest}

\begin{qstest}{unhex-space}{unescapehex, space}
  \EdefUnescapeHex\x{20}%
  \Expect*{\x}{ }%
  \ExpectVar\x\space
\end{qstest}

\begin{qstest}{unhex-spaces}{unescapehex, spaces}
  \EdefUnescapeHex\x{204020204120}%
  \def\y#1{%
    \edef\z{#1\string @#1#1\string A#1}%
  }\y{ }%
  \Expect*{\x}*{\z}%
  \ExpectVar\x\z
\end{qstest}

\begin{qstest}{unhex-hash}{unescapehex, hash}
  \catcode`\#=12 %
  \EdefUnescapeHex\x{#20}%
  \ExpectVar\x\space
\end{qstest}

\begin{qstest}{unhex-invalid}{unescapehex, invalid}
  \def\test#1#2{%
    \EdefUnescapeHex\x{#1}%
    \edef\y{#2}%
    \@onelevel@sanitize\y
    \ExpectVar\x\y
  }%
%<*test2>
  \edef\x{\pdfunescapehex{4X}}%
  \edef\y{\string @}%
  \ifx\x\y
  \else
    \def~{\space}%
    \typeout{************************************}%
    \typeout{* Your pdfTeX contains bug 777.~~~~*}%
    \typeout{* This test is redefined as dummy, *}%
    \typeout{* because it triggers the bug.~~~~~*}%
    \typeout{************************************}%
    \def\test#1#2{}%
  \fi
%</test2>
  \test{X}{}%
  \test{XY}{}%
  \test{XYZ}{}%
  \test{A}{^^a0}%
  \test{AX}{^^a0}%
  \test{XA}{^^a0}%
  \test{XXAXX}{^^a0}%
\end{qstest}

\begin{qstest}{all-name}{\AllBytes, escapename}
  \EdefEscapeName\x{\AllBytes}%
  \Expect*{\x}*{\AllBytesName}%
  \ExpectVar\x\AllBytesName
\end{qstest}

\begin{qstest}{all-string}{\AllBytes, escapestring}
  \EdefEscapeString\x{\AllBytes}%
  \Expect*{\x}*{\AllBytesString}%
  \ExpectVar\x\AllBytesString
\end{qstest}

\begin{qstest}{uchexdigit}{unescape, uppercase hex digit}
  \catcode`\@=11 %
  \catcode0=12 %
  \def\test#1#2{%
    \uccode0=#1\relax
    \uppercase{%
      \def\x{^^@}%
    }%
    \Expect*{%
       \ifcase\expandafter\PE@TestUcHexDigit\x
         true%
       \else
         false%
       \fi
    }{#2}%
  }%
  \def\range#1#2#3{%
    \count0=#1\relax
    \loop
    \ifnum\count0<#2\relax
      \test{\count0}{#3}%
      \advance\count0 by 1 %
    \repeat
  }%
  \range{0}{47}{false}%
  \range{48}{57}{true}%
  \range{58}{64}{false}%
  \range{65}{70}{true}%
  \range{71}{255}{false}%
\end{qstest}

\begin{qstest}{unescapename}{unescapename}
  \def\test#1#2{%
    \EdefUnescapeName\x{#1}%
    \edef\y{#2}%
    \@onelevel@sanitize\y
    \ExpectVar\x\y
  }%
  \catcode`\#=12 %
  \catcode0=12 %
  \test{}{}%
  \test{x}{x}%
  \test{xy}{xy}%
  \test{#}{#}%
  \test{##}{##}%
  \test{###}{###}%
  \test{####}{####}%
  \test{#x}{#x}%
  \test{#xy}{#xy}%
  \test{#1}{#1}%
  \test{#40}{@}%
  \test{#400}{@0}%
  \test{#4x0}{#4x0}%
  \test{#ab}{^^ab}%
  \test{#00}{^^@}%
  \test{x#40y#40z}{x@y@z}%
  \test{#40#40#40#40}{@@@@}%
  \test{a#x}{a#x}%
  \test{a#xy}{a#xy}%
  \test{a#1}{a#1}%
  \test{a#40}{a@}%
  \test{a#400}{a@0}%
  \test{#20}{ }%
  \test{a#20}{a }%
  \test{a#20b}{a b}%
  \test{a#20#20#20b}{a \space\space b}%
\end{qstest}

\begin{qstest}{unescapestring}{unescapestring}
  \def\test#1#2{%
    \EdefUnescapeString\x{#1}%
    \edef\y{#2}%
    \@onelevel@sanitize\y
    \ExpectVar\x\y
  }%
  \catcode0=12 %
  \def\DefChar#1#2{%
    \begingroup
      \uccode0=#2\relax
    \uppercase{\endgroup
      \def#1{^^@}%
    }%
  }%
  \DefChar\nul{0}%
  \DefChar\one{1}%
  \DefChar\bel{8}%
  \DefChar\tab{9}%
  \DefChar\lf{10}%
  \DefChar\ff{12}%
  \DefChar\cr{13}%
  \DefChar\\{92}%
  \test{}{}%
  \test{a}{a}%
  \test{\\}{}%
  \test{\\\\}{\\}%
  \test{\\\\y}{\\y}%
  \test{\\000}{\nul}%
  \test{\\b}{\bel}%
  \test{\\t}{\tab}%
  \test{\\n}{\lf}%
  \test{\\f}{\ff}%
  \test{\\r}{\cr}%
  \test{\\(}{(}%
  \test{\\)}{)}%
  \test{\\040}{ }%
  \test{\\100}{@}%
  \test{\\40}{ }%
  \test{\\1}{\one}%
  \test{\\01}{\one}%
  \test{\\001}{\one}%
  \test{\\18}{\one8}%
  \test{\\018}{\one8}%
  \test{\\0018}{\one8}%
  \test{x\\}{x}%
  \test{x\\\\}{x\\}%
  \test{x\\\\y}{x\\y}%
  \test{x\\000}{x\nul}%
  \test{x\\b}{x\bel}%
  \test{x\\t}{x\tab}%
  \test{x\\n}{x\lf}%
  \test{x\\f}{x\ff}%
  \test{x\\r}{x\cr}%
  \test{x\\(}{x(}%
  \test{x\\)}{x)}%
  \test{x\\040}{x }%
  \test{x\\100}{x@}%
  \test{x\\40}{x }%
  \test{x\\1}{x\one}%
  \test{x\\01}{x\one}%
  \test{x\\001}{x\one}%
  \test{x\\18}{x\one8}%
  \test{x\\018}{x\one8}%
  \test{x\\0018}{x\one8}%
  \test{\\b\\t\\n\\f\\r\\(\\)\\\\\\000\\040}{%
    \bel\tab\lf\ff\cr()\\\nul\space
  }%
  \test{\\\lf}{}%
  \test{x\\\lf}{x}%
  \test{\cr}{\lf}%
  \test{\cr\lf}{\lf}%
  \test{\lf}{\lf}%
  \test{\lf\cr}{\lf\lf}%
  \test{x\cr}{x\lf}%
  \test{x\cr\lf}{x\lf}%
  \test{x\lf}{x\lf}%
  \test{x\lf\cr}{x\lf\lf}%
  \test{x\\\cr\lf y\cr}{xy\lf}%
%
  \test{\\409}{ 9}%
  \test{\\800}{800}%
  \test{\\900}{900}%
  \test{\\578}{/8}%
  \test{\\477}{?}%
  \test{\\377}{^^ff}%
  \test{\\777}{^^ff}%
  \test{\\7777}{^^ff7}%
\end{qstest}
\stop
%    \end{macrocode}
%
%    \begin{macrocode}
%</test2|test3|test4|test5>
%    \end{macrocode}
%
% \subsubsection{Test for \hologo{iniTeX}}
%
%    \begin{macrocode}
%<*test6>
\input pdfescape.sty\relax
\catcode`\{=1 %
\catcode`\}=2 %
\catcode`\#=6 %
\catcode`\^=7 %
\catcode`\@=11 %
%    \end{macrocode}
%    \begin{macrocode}
\begingroup
  \catcode`\@=11 %
  \countdef\count@=255 %
  \def\space{ }%
  \long\def\@whilenum#1\do #2{%
    \ifnum #1\relax
      #2\relax
      \@iwhilenum{#1\relax#2\relax}%
    \fi
  }%
  \long\def\@iwhilenum#1{%
    \ifnum #1%
      \expandafter\@iwhilenum
    \else
      \expandafter\ltx@gobble
    \fi
    {#1}%
  }%
  \gdef\AllBytes{}%
  \count@=0 %
  \catcode0=12 %
  \@whilenum\count@<256 \do{%
    \lccode0=\count@
    \ifnum\count@=32 %
      \xdef\AllBytes{\AllBytes\space}%
    \else
      \lowercase{%
        \xdef\AllBytes{\AllBytes^^@}%
      }%
    \fi
    \advance\count@ by 1 %
  }%
\endgroup
\def\AllBytesHex{%
  000102030405060708090A0B0C0D0E0F%
  101112131415161718191A1B1C1D1E1F%
  202122232425262728292A2B2C2D2E2F%
  303132333435363738393A3B3C3D3E3F%
  404142434445464748494A4B4C4D4E4F%
  505152535455565758595A5B5C5D5E5F%
  606162636465666768696A6B6C6D6E6F%
  707172737475767778797A7B7C7D7E7F%
  808182838485868788898A8B8C8D8E8F%
  909192939495969798999A9B9C9D9E9F%
  A0A1A2A3A4A5A6A7A8A9AAABACADAEAF%
  B0B1B2B3B4B5B6B7B8B9BABBBCBDBEBF%
  C0C1C2C3C4C5C6C7C8C9CACBCCCDCECF%
  D0D1D2D3D4D5D6D7D8D9DADBDCDDDEDF%
  E0E1E2E3E4E5E6E7E8E9EAEBECEDEEEF%
  F0F1F2F3F4F5F6F7F8F9FAFBFCFDFEFF%
}
\ltx@onelevel@sanitize\AllBytesHex
\expandafter\lowercase\expandafter{%
  \expandafter\def\expandafter\AllBytesHexLC
      \expandafter{\AllBytesHex}%
}
\begingroup
  \catcode`\#=12 %
  \xdef\AllBytesName{%
    #01#02#03#04#05#06#07#08#09#0A#0B#0C#0D#0E#0F%
    #10#11#12#13#14#15#16#17#18#19#1A#1B#1C#1D#1E#1F%
    #20!"#23$#25&'#28#29*+,-.#2F%
    0123456789:;#3C=#3E?%
    @ABCDEFGHIJKLMNO%
    PQRSTUVWXYZ#5B\ltx@backslashchar#5D^_%
    `abcdefghijklmno%
    pqrstuvwxyz#7B|#7D\string~#7F%
    #80#81#82#83#84#85#86#87#88#89#8A#8B#8C#8D#8E#8F%
    #90#91#92#93#94#95#96#97#98#99#9A#9B#9C#9D#9E#9F%
    #A0#A1#A2#A3#A4#A5#A6#A7#A8#A9#AA#AB#AC#AD#AE#AF%
    #B0#B1#B2#B3#B4#B5#B6#B7#B8#B9#BA#BB#BC#BD#BE#BF%
    #C0#C1#C2#C3#C4#C5#C6#C7#C8#C9#CA#CB#CC#CD#CE#CF%
    #D0#D1#D2#D3#D4#D5#D6#D7#D8#D9#DA#DB#DC#DD#DE#DF%
    #E0#E1#E2#E3#E4#E5#E6#E7#E8#E9#EA#EB#EC#ED#EE#EF%
    #F0#F1#F2#F3#F4#F5#F6#F7#F8#F9#FA#FB#FC#FD#FE#FF%
  }%
\endgroup
\ltx@onelevel@sanitize\AllBytesName
\edef\AllBytesFromName{\expandafter\ltx@gobble\AllBytes}

\begingroup
  \def\|{|}%
  \edef\%{\ltx@percentchar}%
  \catcode`\|=0 %
  \catcode`\#=12 %
  \catcode`\~=12 %
  \catcode`\\=12 %
  |xdef|AllBytesString{%
    \000\001\002\003\004\005\006\007\010\011\012\013\014\015\016\017%
    \020\021\022\023\024\025\026\027\030\031\032\033\034\035\036\037%
    \040!"#$|%&'\(\)*+,-./%
    0123456789:;<=>?%
    @ABCDEFGHIJKLMNO%
    PQRSTUVWXYZ[\\]^_%
    `abcdefghijklmno%
    pqrstuvwxyz{||}~\177%
    \200\201\202\203\204\205\206\207\210\211\212\213\214\215\216\217%
    \220\221\222\223\224\225\226\227\230\231\232\233\234\235\236\237%
    \240\241\242\243\244\245\246\247\250\251\252\253\254\255\256\257%
    \260\261\262\263\264\265\266\267\270\271\272\273\274\275\276\277%
    \300\301\302\303\304\305\306\307\310\311\312\313\314\315\316\317%
    \320\321\322\323\324\325\326\327\330\331\332\333\334\335\336\337%
    \340\341\342\343\344\345\346\347\350\351\352\353\354\355\356\357%
    \360\361\362\363\364\365\366\367\370\371\372\373\374\375\376\377%
  }%
|endgroup
\ltx@onelevel@sanitize\AllBytesString
%    \end{macrocode}
%    \begin{macrocode}
\def\msg#{\immediate\write16}
\def\Test#1#2#3{%
  \begingroup
    #1\TestResult{#2}%
    \ifx\TestResult#3%
    \else
      \newlinechar=10 %
      \msg{Expect:^^J#3}%
      \msg{Result:^^J\TestResult}%
      \errmessage{\string#2 -\string#1-> \string#3}%
    \fi
  \endgroup
}
\Test\EdefEscapeHex\AllBytes\AllBytesHex
\Test\EdefUnescapeHex\AllBytesHex\AllBytes
\Test\EdefEscapeName\AllBytes\AllBytesName
\Test\EdefUnescapeName\AllBytesName\AllBytesFromName
\Test\EdefEscapeString\AllBytes\AllBytesString
\Test\EdefUnescapeString\AllBytesString\AllBytes
%    \end{macrocode}
%    \begin{macrocode}
\csname @@end\endcsname\end
%</test6>
%    \end{macrocode}
%
% \section{Installation}
%
% \subsection{Download}
%
% \paragraph{Package.} This package is available on
% CTAN\footnote{\url{ftp://ftp.ctan.org/tex-archive/}}:
% \begin{description}
% \item[\CTAN{macros/latex/contrib/oberdiek/pdfescape.dtx}] The source file.
% \item[\CTAN{macros/latex/contrib/oberdiek/pdfescape.pdf}] Documentation.
% \end{description}
%
%
% \paragraph{Bundle.} All the packages of the bundle `oberdiek'
% are also available in a TDS compliant ZIP archive. There
% the packages are already unpacked and the documentation files
% are generated. The files and directories obey the TDS standard.
% \begin{description}
% \item[\CTAN{install/macros/latex/contrib/oberdiek.tds.zip}]
% \end{description}
% \emph{TDS} refers to the standard ``A Directory Structure
% for \TeX\ Files'' (\CTAN{tds/tds.pdf}). Directories
% with \xfile{texmf} in their name are usually organized this way.
%
% \subsection{Bundle installation}
%
% \paragraph{Unpacking.} Unpack the \xfile{oberdiek.tds.zip} in the
% TDS tree (also known as \xfile{texmf} tree) of your choice.
% Example (linux):
% \begin{quote}
%   |unzip oberdiek.tds.zip -d ~/texmf|
% \end{quote}
%
% \paragraph{Script installation.}
% Check the directory \xfile{TDS:scripts/oberdiek/} for
% scripts that need further installation steps.
% Package \xpackage{attachfile2} comes with the Perl script
% \xfile{pdfatfi.pl} that should be installed in such a way
% that it can be called as \texttt{pdfatfi}.
% Example (linux):
% \begin{quote}
%   |chmod +x scripts/oberdiek/pdfatfi.pl|\\
%   |cp scripts/oberdiek/pdfatfi.pl /usr/local/bin/|
% \end{quote}
%
% \subsection{Package installation}
%
% \paragraph{Unpacking.} The \xfile{.dtx} file is a self-extracting
% \docstrip\ archive. The files are extracted by running the
% \xfile{.dtx} through \plainTeX:
% \begin{quote}
%   \verb|tex pdfescape.dtx|
% \end{quote}
%
% \paragraph{TDS.} Now the different files must be moved into
% the different directories in your installation TDS tree
% (also known as \xfile{texmf} tree):
% \begin{quote}
% \def\t{^^A
% \begin{tabular}{@{}>{\ttfamily}l@{ $\rightarrow$ }>{\ttfamily}l@{}}
%   pdfescape.sty & tex/generic/oberdiek/pdfescape.sty\\
%   pdfescape.pdf & doc/latex/oberdiek/pdfescape.pdf\\
%   test/pdfescape-test1.tex & doc/latex/oberdiek/test/pdfescape-test1.tex\\
%   test/pdfescape-test2.tex & doc/latex/oberdiek/test/pdfescape-test2.tex\\
%   test/pdfescape-test3.tex & doc/latex/oberdiek/test/pdfescape-test3.tex\\
%   test/pdfescape-test4.tex & doc/latex/oberdiek/test/pdfescape-test4.tex\\
%   test/pdfescape-test5.tex & doc/latex/oberdiek/test/pdfescape-test5.tex\\
%   test/pdfescape-test6.tex & doc/latex/oberdiek/test/pdfescape-test6.tex\\
%   pdfescape.dtx & source/latex/oberdiek/pdfescape.dtx\\
% \end{tabular}^^A
% }^^A
% \sbox0{\t}^^A
% \ifdim\wd0>\linewidth
%   \begingroup
%     \advance\linewidth by\leftmargin
%     \advance\linewidth by\rightmargin
%   \edef\x{\endgroup
%     \def\noexpand\lw{\the\linewidth}^^A
%   }\x
%   \def\lwbox{^^A
%     \leavevmode
%     \hbox to \linewidth{^^A
%       \kern-\leftmargin\relax
%       \hss
%       \usebox0
%       \hss
%       \kern-\rightmargin\relax
%     }^^A
%   }^^A
%   \ifdim\wd0>\lw
%     \sbox0{\small\t}^^A
%     \ifdim\wd0>\linewidth
%       \ifdim\wd0>\lw
%         \sbox0{\footnotesize\t}^^A
%         \ifdim\wd0>\linewidth
%           \ifdim\wd0>\lw
%             \sbox0{\scriptsize\t}^^A
%             \ifdim\wd0>\linewidth
%               \ifdim\wd0>\lw
%                 \sbox0{\tiny\t}^^A
%                 \ifdim\wd0>\linewidth
%                   \lwbox
%                 \else
%                   \usebox0
%                 \fi
%               \else
%                 \lwbox
%               \fi
%             \else
%               \usebox0
%             \fi
%           \else
%             \lwbox
%           \fi
%         \else
%           \usebox0
%         \fi
%       \else
%         \lwbox
%       \fi
%     \else
%       \usebox0
%     \fi
%   \else
%     \lwbox
%   \fi
% \else
%   \usebox0
% \fi
% \end{quote}
% If you have a \xfile{docstrip.cfg} that configures and enables \docstrip's
% TDS installing feature, then some files can already be in the right
% place, see the documentation of \docstrip.
%
% \subsection{Refresh file name databases}
%
% If your \TeX~distribution
% (\teTeX, \mikTeX, \dots) relies on file name databases, you must refresh
% these. For example, \teTeX\ users run \verb|texhash| or
% \verb|mktexlsr|.
%
% \subsection{Some details for the interested}
%
% \paragraph{Attached source.}
%
% The PDF documentation on CTAN also includes the
% \xfile{.dtx} source file. It can be extracted by
% AcrobatReader 6 or higher. Another option is \textsf{pdftk},
% e.g. unpack the file into the current directory:
% \begin{quote}
%   \verb|pdftk pdfescape.pdf unpack_files output .|
% \end{quote}
%
% \paragraph{Unpacking with \LaTeX.}
% The \xfile{.dtx} chooses its action depending on the format:
% \begin{description}
% \item[\plainTeX:] Run \docstrip\ and extract the files.
% \item[\LaTeX:] Generate the documentation.
% \end{description}
% If you insist on using \LaTeX\ for \docstrip\ (really,
% \docstrip\ does not need \LaTeX), then inform the autodetect routine
% about your intention:
% \begin{quote}
%   \verb|latex \let\install=y\input{pdfescape.dtx}|
% \end{quote}
% Do not forget to quote the argument according to the demands
% of your shell.
%
% \paragraph{Generating the documentation.}
% You can use both the \xfile{.dtx} or the \xfile{.drv} to generate
% the documentation. The process can be configured by the
% configuration file \xfile{ltxdoc.cfg}. For instance, put this
% line into this file, if you want to have A4 as paper format:
% \begin{quote}
%   \verb|\PassOptionsToClass{a4paper}{article}|
% \end{quote}
% An example follows how to generate the
% documentation with pdf\LaTeX:
% \begin{quote}
%\begin{verbatim}
%pdflatex pdfescape.dtx
%makeindex -s gind.ist pdfescape.idx
%pdflatex pdfescape.dtx
%makeindex -s gind.ist pdfescape.idx
%pdflatex pdfescape.dtx
%\end{verbatim}
% \end{quote}
%
% \section{Catalogue}
%
% The following XML file can be used as source for the
% \href{http://mirror.ctan.org/help/Catalogue/catalogue.html}{\TeX\ Catalogue}.
% The elements \texttt{caption} and \texttt{description} are imported
% from the original XML file from the Catalogue.
% The name of the XML file in the Catalogue is \xfile{pdfescape.xml}.
%    \begin{macrocode}
%<*catalogue>
<?xml version='1.0' encoding='us-ascii'?>
<!DOCTYPE entry SYSTEM 'catalogue.dtd'>
<entry datestamp='$Date$' modifier='$Author$' id='pdfescape'>
  <name>pdfescape</name>
  <caption>Implements pdfTeX's escape features using TeX or e-TeX.</caption>
  <authorref id='auth:oberdiek'/>
  <copyright owner='Heiko Oberdiek' year='2007,2010,2011'/>
  <license type='lppl1.3'/>
  <version number='1.13'/>
  <description>
    This package implements <xref refid='pdftex'>pdfTeX</xref>'s
    escape features (<tt>\pdfescapehex</tt>, <tt>\pdfunescapehex</tt>,
    <tt>\pdfescapename</tt>, <tt>\pdfescapestring</tt>) using TeX or
    e-TeX.
    <p/>
    The package is part of the <xref refid='oberdiek'>oberdiek</xref>
    bundle.
  </description>
  <documentation details='Package documentation'
      href='ctan:/macros/latex/contrib/oberdiek/pdfescape.pdf'/>
  <ctan file='true' path='/macros/latex/contrib/oberdiek/pdfescape.dtx'/>
  <miktex location='oberdiek'/>
  <texlive location='oberdiek'/>
  <install path='/macros/latex/contrib/oberdiek/oberdiek.tds.zip'/>
</entry>
%</catalogue>
%    \end{macrocode}
%
% \begin{History}
%   \begin{Version}{2007/02/21 v1.0}
%   \item
%     First version.
%   \end{Version}
%   \begin{Version}{2007/02/25 v1.1}
%   \item
%     Test files added.
%   \item
%     \cs{EdefUnescapeHex} supports lowercase letters.
%   \item
%     Fix: \verb|\EdefEscapeName{^^@}|
%   \item
%     Fix: \verb|\EdefEscapeName{\string#}|
%   \item
%     Fix for \cs{EdefUnescapeHex} in case of incomplete hex string.
%   \item
%     Fix: \cs{EdefUnescapeHex} generates space tokens with catcode
%     10 (space) in all cases.
%   \item
%     Fix: \cs{EdefEscapeHex} and \cs{EdefEscapeName} now generate
%     tokens with catcode 12 (other) only.
%   \end{Version}
%   \begin{Version}{2007/03/20 v1.2}
%   \item
%     Fix: Wrong year in \cs{ProvidesPackage}.
%   \end{Version}
%   \begin{Version}{2007/04/11 v1.3}
%   \item
%     Line ends sanitized.
%   \end{Version}
%   \begin{Version}{2007/04/21 v1.4}
%   \item
%     \cs{EdefUnescapeName} and \cs{EdefUnescapeString} added.
%   \end{Version}
%   \begin{Version}{2007/08/27 v1.5}
%   \item
%     \cs{EdefSanitize} added (replaces \cs{PE@sanitize}).
%   \end{Version}
%   \begin{Version}{2007/09/09 v1.6}
%   \item
%     Fix in catcode setup.
%   \end{Version}
%   \begin{Version}{2007/10/27 v1.7}
%   \item
%     More efficient \cs{EdefSanitize}.
%   \end{Version}
%   \begin{Version}{2007/11/11 v1.8}
%   \item
%     Use of package \xpackage{pdftexcmds} for \hologo{LuaTeX} support.
%   \end{Version}
%   \begin{Version}{2010/03/01 v1.9}
%   \item
%     Compatibility with \hologo{iniTeX}.
%   \end{Version}
%   \begin{Version}{2010/11/12 v1.10}
%   \item
%     Use of package \xpackage{ltxcmds}.
%   \item
%     Fix for compatibility with \hologo{iniTeX}.
%   \end{Version}
%   \begin{Version}{2011/01/30 v1.11}
%   \item
%     Already loaded package files are not input in \hologo{plainTeX}.
%   \end{Version}
%   \begin{Version}{2011/04/04 v1.12}
%   \item
%     Further fixes for compatibility for \hologo{iniTeX}.
%   \item
%     Test file for \hologo{iniTeX} added.
%   \end{Version}
%   \begin{Version}{2011/11/25 v1.13}
%   \item
%     Higher order bit of octal sequences in \cs{EdefUnescapeString}
%     ignored according to the PDF specification
%     (Bug found by Bruno Le Floch).
%   \end{Version}
% \end{History}
%
% \PrintIndex
%
% \Finale
\endinput
|
% \end{quote}
% Do not forget to quote the argument according to the demands
% of your shell.
%
% \paragraph{Generating the documentation.}
% You can use both the \xfile{.dtx} or the \xfile{.drv} to generate
% the documentation. The process can be configured by the
% configuration file \xfile{ltxdoc.cfg}. For instance, put this
% line into this file, if you want to have A4 as paper format:
% \begin{quote}
%   \verb|\PassOptionsToClass{a4paper}{article}|
% \end{quote}
% An example follows how to generate the
% documentation with pdf\LaTeX:
% \begin{quote}
%\begin{verbatim}
%pdflatex pdfescape.dtx
%makeindex -s gind.ist pdfescape.idx
%pdflatex pdfescape.dtx
%makeindex -s gind.ist pdfescape.idx
%pdflatex pdfescape.dtx
%\end{verbatim}
% \end{quote}
%
% \section{Catalogue}
%
% The following XML file can be used as source for the
% \href{http://mirror.ctan.org/help/Catalogue/catalogue.html}{\TeX\ Catalogue}.
% The elements \texttt{caption} and \texttt{description} are imported
% from the original XML file from the Catalogue.
% The name of the XML file in the Catalogue is \xfile{pdfescape.xml}.
%    \begin{macrocode}
%<*catalogue>
<?xml version='1.0' encoding='us-ascii'?>
<!DOCTYPE entry SYSTEM 'catalogue.dtd'>
<entry datestamp='$Date$' modifier='$Author$' id='pdfescape'>
  <name>pdfescape</name>
  <caption>Implements pdfTeX's escape features using TeX or e-TeX.</caption>
  <authorref id='auth:oberdiek'/>
  <copyright owner='Heiko Oberdiek' year='2007,2010,2011'/>
  <license type='lppl1.3'/>
  <version number='1.13'/>
  <description>
    This package implements <xref refid='pdftex'>pdfTeX</xref>'s
    escape features (<tt>\pdfescapehex</tt>, <tt>\pdfunescapehex</tt>,
    <tt>\pdfescapename</tt>, <tt>\pdfescapestring</tt>) using TeX or
    e-TeX.
    <p/>
    The package is part of the <xref refid='oberdiek'>oberdiek</xref>
    bundle.
  </description>
  <documentation details='Package documentation'
      href='ctan:/macros/latex/contrib/oberdiek/pdfescape.pdf'/>
  <ctan file='true' path='/macros/latex/contrib/oberdiek/pdfescape.dtx'/>
  <miktex location='oberdiek'/>
  <texlive location='oberdiek'/>
  <install path='/macros/latex/contrib/oberdiek/oberdiek.tds.zip'/>
</entry>
%</catalogue>
%    \end{macrocode}
%
% \begin{History}
%   \begin{Version}{2007/02/21 v1.0}
%   \item
%     First version.
%   \end{Version}
%   \begin{Version}{2007/02/25 v1.1}
%   \item
%     Test files added.
%   \item
%     \cs{EdefUnescapeHex} supports lowercase letters.
%   \item
%     Fix: \verb|\EdefEscapeName{^^@}|
%   \item
%     Fix: \verb|\EdefEscapeName{\string#}|
%   \item
%     Fix for \cs{EdefUnescapeHex} in case of incomplete hex string.
%   \item
%     Fix: \cs{EdefUnescapeHex} generates space tokens with catcode
%     10 (space) in all cases.
%   \item
%     Fix: \cs{EdefEscapeHex} and \cs{EdefEscapeName} now generate
%     tokens with catcode 12 (other) only.
%   \end{Version}
%   \begin{Version}{2007/03/20 v1.2}
%   \item
%     Fix: Wrong year in \cs{ProvidesPackage}.
%   \end{Version}
%   \begin{Version}{2007/04/11 v1.3}
%   \item
%     Line ends sanitized.
%   \end{Version}
%   \begin{Version}{2007/04/21 v1.4}
%   \item
%     \cs{EdefUnescapeName} and \cs{EdefUnescapeString} added.
%   \end{Version}
%   \begin{Version}{2007/08/27 v1.5}
%   \item
%     \cs{EdefSanitize} added (replaces \cs{PE@sanitize}).
%   \end{Version}
%   \begin{Version}{2007/09/09 v1.6}
%   \item
%     Fix in catcode setup.
%   \end{Version}
%   \begin{Version}{2007/10/27 v1.7}
%   \item
%     More efficient \cs{EdefSanitize}.
%   \end{Version}
%   \begin{Version}{2007/11/11 v1.8}
%   \item
%     Use of package \xpackage{pdftexcmds} for \hologo{LuaTeX} support.
%   \end{Version}
%   \begin{Version}{2010/03/01 v1.9}
%   \item
%     Compatibility with \hologo{iniTeX}.
%   \end{Version}
%   \begin{Version}{2010/11/12 v1.10}
%   \item
%     Use of package \xpackage{ltxcmds}.
%   \item
%     Fix for compatibility with \hologo{iniTeX}.
%   \end{Version}
%   \begin{Version}{2011/01/30 v1.11}
%   \item
%     Already loaded package files are not input in \hologo{plainTeX}.
%   \end{Version}
%   \begin{Version}{2011/04/04 v1.12}
%   \item
%     Further fixes for compatibility for \hologo{iniTeX}.
%   \item
%     Test file for \hologo{iniTeX} added.
%   \end{Version}
%   \begin{Version}{2011/11/25 v1.13}
%   \item
%     Higher order bit of octal sequences in \cs{EdefUnescapeString}
%     ignored according to the PDF specification
%     (Bug found by Bruno Le Floch).
%   \end{Version}
% \end{History}
%
% \PrintIndex
%
% \Finale
\endinput

%        (quote the arguments according to the demands of your shell)
%
% Documentation:
%    (a) If pdfescape.drv is present:
%           latex pdfescape.drv
%    (b) Without pdfescape.drv:
%           latex pdfescape.dtx; ...
%    The class ltxdoc loads the configuration file ltxdoc.cfg
%    if available. Here you can specify further options, e.g.
%    use A4 as paper format:
%       \PassOptionsToClass{a4paper}{article}
%
%    Programm calls to get the documentation (example):
%       pdflatex pdfescape.dtx
%       makeindex -s gind.ist pdfescape.idx
%       pdflatex pdfescape.dtx
%       makeindex -s gind.ist pdfescape.idx
%       pdflatex pdfescape.dtx
%
% Installation:
%    TDS:tex/generic/oberdiek/pdfescape.sty
%    TDS:doc/latex/oberdiek/pdfescape.pdf
%    TDS:doc/latex/oberdiek/test/pdfescape-test1.tex
%    TDS:doc/latex/oberdiek/test/pdfescape-test2.tex
%    TDS:doc/latex/oberdiek/test/pdfescape-test3.tex
%    TDS:doc/latex/oberdiek/test/pdfescape-test4.tex
%    TDS:doc/latex/oberdiek/test/pdfescape-test5.tex
%    TDS:doc/latex/oberdiek/test/pdfescape-test6.tex
%    TDS:source/latex/oberdiek/pdfescape.dtx
%
%<*ignore>
\begingroup
  \catcode123=1 %
  \catcode125=2 %
  \def\x{LaTeX2e}%
\expandafter\endgroup
\ifcase 0\ifx\install y1\fi\expandafter
         \ifx\csname processbatchFile\endcsname\relax\else1\fi
         \ifx\fmtname\x\else 1\fi\relax
\else\csname fi\endcsname
%</ignore>
%<*install>
\input docstrip.tex
\Msg{************************************************************************}
\Msg{* Installation}
\Msg{* Package: pdfescape 2011/11/25 v1.13 Implements pdfTeX's escape features (HO)}
\Msg{************************************************************************}

\keepsilent
\askforoverwritefalse

\let\MetaPrefix\relax
\preamble

This is a generated file.

Project: pdfescape
Version: 2011/11/25 v1.13

Copyright (C) 2007, 2010, 2011 by
   Heiko Oberdiek <heiko.oberdiek at googlemail.com>

This work may be distributed and/or modified under the
conditions of the LaTeX Project Public License, either
version 1.3c of this license or (at your option) any later
version. This version of this license is in
   http://www.latex-project.org/lppl/lppl-1-3c.txt
and the latest version of this license is in
   http://www.latex-project.org/lppl.txt
and version 1.3 or later is part of all distributions of
LaTeX version 2005/12/01 or later.

This work has the LPPL maintenance status "maintained".

This Current Maintainer of this work is Heiko Oberdiek.

The Base Interpreter refers to any `TeX-Format',
because some files are installed in TDS:tex/generic//.

This work consists of the main source file pdfescape.dtx
and the derived files
   pdfescape.sty, pdfescape.pdf, pdfescape.ins, pdfescape.drv,
   pdfescape-test1.tex, pdfescape-test2.tex,
   pdfescape-test3.tex, pdfescape-test4.tex,
   pdfescape-test5.tex, pdfescape-test6.tex.

\endpreamble
\let\MetaPrefix\DoubleperCent

\generate{%
  \file{pdfescape.ins}{\from{pdfescape.dtx}{install}}%
  \file{pdfescape.drv}{\from{pdfescape.dtx}{driver}}%
  \usedir{tex/generic/oberdiek}%
  \file{pdfescape.sty}{\from{pdfescape.dtx}{package}}%
  \usedir{doc/latex/oberdiek/test}%
  \file{pdfescape-test1.tex}{\from{pdfescape.dtx}{test1}}%
  \file{pdfescape-test2.tex}{\from{pdfescape.dtx}{test2}}%
  \file{pdfescape-test3.tex}{\from{pdfescape.dtx}{test3}}%
  \file{pdfescape-test4.tex}{\from{pdfescape.dtx}{test4}}%
  \file{pdfescape-test5.tex}{\from{pdfescape.dtx}{test5}}%
  \file{pdfescape-test6.tex}{\from{pdfescape.dtx}{test6}}%
  \nopreamble
  \nopostamble
  \usedir{source/latex/oberdiek/catalogue}%
  \file{pdfescape.xml}{\from{pdfescape.dtx}{catalogue}}%
}

\catcode32=13\relax% active space
\let =\space%
\Msg{************************************************************************}
\Msg{*}
\Msg{* To finish the installation you have to move the following}
\Msg{* file into a directory searched by TeX:}
\Msg{*}
\Msg{*     pdfescape.sty}
\Msg{*}
\Msg{* To produce the documentation run the file `pdfescape.drv'}
\Msg{* through LaTeX.}
\Msg{*}
\Msg{* Happy TeXing!}
\Msg{*}
\Msg{************************************************************************}

\endbatchfile
%</install>
%<*ignore>
\fi
%</ignore>
%<*driver>
\NeedsTeXFormat{LaTeX2e}
\ProvidesFile{pdfescape.drv}%
  [2011/11/25 v1.13 Implements pdfTeX's escape features (HO)]%
\documentclass{ltxdoc}
\usepackage{holtxdoc}[2011/11/22]
\begin{document}
  \DocInput{pdfescape.dtx}%
\end{document}
%</driver>
% \fi
%
% \CheckSum{2516}
%
% \CharacterTable
%  {Upper-case    \A\B\C\D\E\F\G\H\I\J\K\L\M\N\O\P\Q\R\S\T\U\V\W\X\Y\Z
%   Lower-case    \a\b\c\d\e\f\g\h\i\j\k\l\m\n\o\p\q\r\s\t\u\v\w\x\y\z
%   Digits        \0\1\2\3\4\5\6\7\8\9
%   Exclamation   \!     Double quote  \"     Hash (number) \#
%   Dollar        \$     Percent       \%     Ampersand     \&
%   Acute accent  \'     Left paren    \(     Right paren   \)
%   Asterisk      \*     Plus          \+     Comma         \,
%   Minus         \-     Point         \.     Solidus       \/
%   Colon         \:     Semicolon     \;     Less than     \<
%   Equals        \=     Greater than  \>     Question mark \?
%   Commercial at \@     Left bracket  \[     Backslash     \\
%   Right bracket \]     Circumflex    \^     Underscore    \_
%   Grave accent  \`     Left brace    \{     Vertical bar  \|
%   Right brace   \}     Tilde         \~}
%
% \GetFileInfo{pdfescape.drv}
%
% \title{The \xpackage{pdfescape} package}
% \date{2011/11/25 v1.13}
% \author{Heiko Oberdiek\\\xemail{heiko.oberdiek at googlemail.com}}
%
% \maketitle
%
% \begin{abstract}
% This package implements \pdfTeX's escape features
% (\cs{pdfescapehex}, \cs{pdfunescapehex}, \cs{pdfescapename},
% \cs{pdfescapestring}) using \TeX\ or \eTeX.
% \end{abstract}
%
% \tableofcontents
%
% \section{Documentation}
%
% \begin{declcs}{EdefEscapeHex} \M{cmd} \M{string}\\
%  \cs{EdefUnescapeHex} \M{cmd} \M{string}\\
%  \cs{EdefEscapeName} \M{cmd} \M{string}\\
%  \cs{EdefEscapeString} \M{cmd} \M{string}
% \end{declcs}
% These commands converts \meta{string} and stores the result
% in macro \meta{cmd}. The conversion result is the same
% as the conversion of the corresponding \pdfTeX's primitives.
% Note that the argument \meta{string} is expanded before
% the conversion.
%
% For example, if \pdfTeX\ \textgreater= 1.30 is present,
% then \cs{EdefEscapeHex}
% becomes to:
% \begin{quote}
%\begin{verbatim}
%\def\EdefEscapeHex#1#2{%
%  \edef#1{\pdfescapehex{#2}}%
%}
%\end{verbatim}
% \end{quote}
% The package provides implementations for the case that \pdfTeX\ is
% not present (or too old). Even \eTeX\ can be missing, however it is
% used if it is detected.
%
% \paragraph{Babel.}
% The input strings may contain shorthand characters of package
% \xpackage{babel}.
%
% \subsection{Additional unescape macros}
%
% \begin{declcs}{EdefUnescapeName} \M{cmd} \M{string}
% \end{declcs}
% Sequences of a hash sign with two hexadecimal digits are
% converted to the corresponding character (PDF-1.2).
% A hash sign that is not followed by two hexadecimal digits
% is left unchanged. The catcodes in the result string follow
% TeX's conventions. The space has catcode 10 (space) and
% the other characters have catcode 12 (other).
%
% \begin{declcs}{EdefUnescapeString} \M{cmd} \M{string}
% \end{declcs}
% Macro \meta{cmd} stores the unescaped string in \meta{string}.
% All the rules for literal strings are implemented, see
% PDF specification.
% The catcodes in the result string follow TeX's conventions.
%
% \subsection{Sanitizing macro}
%
% \begin{declcs}{EdefSanitize} \M{cmd} \M{string}
% \end{declcs}
% Argument \meta{string} is expanded, converted to a string
% of tokens with catcode 12 (other) and space tokens, and
% stored in macro \meta{cmd}.
%
% \StopEventually{
% }
%
% \section{Implementation}
%
%    \begin{macrocode}
%<*package>
%    \end{macrocode}
%
% \subsection{Reload check and package identification}
%    Reload check, especially if the package is not used with \LaTeX.
%    \begin{macrocode}
\begingroup\catcode61\catcode48\catcode32=10\relax%
  \catcode13=5 % ^^M
  \endlinechar=13 %
  \catcode35=6 % #
  \catcode39=12 % '
  \catcode44=12 % ,
  \catcode45=12 % -
  \catcode46=12 % .
  \catcode58=12 % :
  \catcode64=11 % @
  \catcode123=1 % {
  \catcode125=2 % }
  \expandafter\let\expandafter\x\csname ver@pdfescape.sty\endcsname
  \ifx\x\relax % plain-TeX, first loading
  \else
    \def\empty{}%
    \ifx\x\empty % LaTeX, first loading,
      % variable is initialized, but \ProvidesPackage not yet seen
    \else
      \expandafter\ifx\csname PackageInfo\endcsname\relax
        \def\x#1#2{%
          \immediate\write-1{Package #1 Info: #2.}%
        }%
      \else
        \def\x#1#2{\PackageInfo{#1}{#2, stopped}}%
      \fi
      \x{pdfescape}{The package is already loaded}%
      \aftergroup\endinput
    \fi
  \fi
\endgroup%
%    \end{macrocode}
%    Package identification:
%    \begin{macrocode}
\begingroup\catcode61\catcode48\catcode32=10\relax%
  \catcode13=5 % ^^M
  \endlinechar=13 %
  \catcode35=6 % #
  \catcode39=12 % '
  \catcode40=12 % (
  \catcode41=12 % )
  \catcode44=12 % ,
  \catcode45=12 % -
  \catcode46=12 % .
  \catcode47=12 % /
  \catcode58=12 % :
  \catcode64=11 % @
  \catcode91=12 % [
  \catcode93=12 % ]
  \catcode123=1 % {
  \catcode125=2 % }
  \expandafter\ifx\csname ProvidesPackage\endcsname\relax
    \def\x#1#2#3[#4]{\endgroup
      \immediate\write-1{Package: #3 #4}%
      \xdef#1{#4}%
    }%
  \else
    \def\x#1#2[#3]{\endgroup
      #2[{#3}]%
      \ifx#1\@undefined
        \xdef#1{#3}%
      \fi
      \ifx#1\relax
        \xdef#1{#3}%
      \fi
    }%
  \fi
\expandafter\x\csname ver@pdfescape.sty\endcsname
\ProvidesPackage{pdfescape}%
  [2011/11/25 v1.13 Implements pdfTeX's escape features (HO)]%
%    \end{macrocode}
%
% \subsection{Catcodes}
%
%    \begin{macrocode}
\begingroup\catcode61\catcode48\catcode32=10\relax%
  \catcode13=5 % ^^M
  \endlinechar=13 %
  \catcode123 1 % {
  \catcode125 2 % }
  \catcode64 11 %
  \def\x{\endgroup
    \expandafter\edef\csname PE@AtEnd\endcsname{%
      \endlinechar=\the\endlinechar\relax
      \catcode13=\the\catcode13\relax
      \catcode32=\the\catcode32\relax
      \catcode35=\the\catcode35\relax
      \catcode61=\the\catcode61\relax
      \catcode64=\the\catcode64\relax
      \catcode123=\the\catcode123\relax
      \catcode125=\the\catcode125\relax
    }%
  }%
\x\catcode61\catcode48\catcode32=10\relax%
\catcode13=5 % ^^M
\endlinechar=13 %
\catcode35=6 % #
\catcode64=11 % @
\catcode123=1 % {
\catcode125=2 % }
\def\TMP@EnsureCode#1#2#3{%
  \edef\PE@AtEnd{%
    \PE@AtEnd
    #1#2=\the#1#2\relax
  }%
  #1#2=#3\relax
}
\TMP@EnsureCode\catcode{0}{12}% ^^@
\TMP@EnsureCode\catcode{34}{12}% "
\TMP@EnsureCode\catcode{36}{3}% $
\TMP@EnsureCode\catcode{38}{4}% &
\TMP@EnsureCode\catcode{39}{12}% '
\TMP@EnsureCode\catcode{42}{12}% *
\TMP@EnsureCode\catcode{45}{12}% -
\TMP@EnsureCode\catcode{46}{12}% .
\TMP@EnsureCode\catcode{47}{12}% /
\TMP@EnsureCode\catcode{60}{12}% <
\TMP@EnsureCode\catcode{62}{12}% >
\TMP@EnsureCode\catcode{91}{12}% [
\TMP@EnsureCode\catcode{93}{12}% ]
\TMP@EnsureCode\catcode{94}{7}% ^
\TMP@EnsureCode\catcode{96}{12}% `
\TMP@EnsureCode\uccode{34}{0}% "
\TMP@EnsureCode\uccode{48}{0}% 0
\TMP@EnsureCode\uccode{61}{0}% =
\edef\PE@AtEnd{\PE@AtEnd\noexpand\endinput}
%    \end{macrocode}
%
% \subsection{Load package}
%
%    \begin{macrocode}
\begingroup\expandafter\expandafter\expandafter\endgroup
\expandafter\ifx\csname RequirePackage\endcsname\relax
  \def\TMP@RequirePackage#1[#2]{%
    \begingroup\expandafter\expandafter\expandafter\endgroup
    \expandafter\ifx\csname ver@#1.sty\endcsname\relax
      \input #1.sty\relax
    \fi
  }%
  \TMP@RequirePackage{ltxcmds}[2010/04/08]%
\else
  \RequirePackage{ltxcmds}[2010/04/08]%
\fi
%    \end{macrocode}
%
% \subsection{Sanitizing}
%
%    \begin{macro}{\EdefSanitize}
%    Macro \cs{EdefSanitize} takes |#2|, entirely converts
%    it to token with catcode 12 (other) and stores the
%    result in macro |#1|.
%    \begin{macrocode}
\begingroup\expandafter\expandafter\expandafter\endgroup
\expandafter\ifx\csname detokenize\endcsname\relax
  \long\def\EdefSanitize#1#2{%
    \begingroup
      \csname @safe@activestrue\endcsname
      \edef#1{#2}%
      \PE@onelevel@sanitize#1%
    \expandafter\endgroup
    \expandafter\def\expandafter#1\expandafter{#1}%
  }%
  \begingroup\expandafter\expandafter\expandafter\endgroup
  \expandafter\ifx\csname @onelevel@sanitize\endcsname\relax
    \def\PE@onelevel@sanitize#1{%
      \edef#1{\expandafter\PE@strip@prefix\meaning#1}%
    }%
    \def\PE@strip@prefix#1>{}%
  \else
    \let\PE@onelevel@sanitize\@onelevel@sanitize
  \fi
\else
  \long\def\EdefSanitize#1#2{%
    \begingroup
      \csname @safe@activestrue\endcsname
      \edef#1{#2}%
    \expandafter\endgroup
    \expandafter\def\expandafter#1\expandafter{%
      \detokenize\expandafter{#1}%
    }%
  }%
  \def\PE@onelevel@sanitize#1{%
    \edef#1{\detokenize\expandafter{#1}}%
  }%
\fi
%    \end{macrocode}
%    \end{macro}
%    \begin{macro}{\PE@sanitize}
%    Macro \cs{PE@sanitize} is only defined for
%    compatibility with version 1.4. Its use is deprecated.
%    \begin{macrocode}
\let\PE@sanitize\EdefSanitize
%    \end{macrocode}
%    \end{macro}
%
% \subsubsection{Space characters}
%
%    \begin{macro}{\PE@space@other}
%    \begin{macrocode}
\begingroup
  \catcode`\ =12\relax%
\def\x{\endgroup\def\PE@space@other{ }}\x\relax
%    \end{macrocode}
%    \end{macro}
%    \begin{macro}{\PE@space@space}
%    \begin{macrocode}
\def\PE@space@space{ }
%    \end{macrocode}
%    \end{macro}
%
% \subsubsection{Space normalization}
%
%    \begin{macro}{\PE@SanitizeSpaceOther}
%    \begin{macrocode}
\def\PE@SanitizeSpaceOther#1{%
  \edef#1{\expandafter\PE@SpaceToOther#1 \relax}%
}
%    \end{macrocode}
%    \end{macro}
%
%    \begin{macro}{\PE@SpaceToOther}
%    \begin{macrocode}
\def\PE@SpaceToOther#1 #2\relax{%
  #1%
  \ifx\\#2\\%
  \else
    \PE@space@other
    \ltx@ReturnAfterFi{%
      \PE@SpaceToOther#2\relax
    }%
  \fi
}
%    \end{macrocode}
%    \end{macro}
%
% \subsection{\cs{EdefUnescapeName}}
%
%    \begin{macro}{\EdefUnescapeName}
%    \begin{macrocode}
\def\EdefUnescapeName#1#2{%
  \EdefSanitize#1{#2}%
  \PE@SanitizeSpaceOther#1%
  \PE@UnescapeName#1%
  \PE@onelevel@sanitize#1%
}
%    \end{macrocode}
%    \end{macro}
%
%    \begin{macro}{\PE@UnescapeName}
%    \begin{macrocode}
\begingroup
  \catcode`\$=6  % hash
  \catcode`\#=12 % other
  \gdef\PE@UnescapeName$1{%
    \begingroup
      \PE@InitUccodeHexDigit
      \def\PE@result{}%
      \expandafter\PE@DeName$1#\relax\relax
    \expandafter\endgroup
    \expandafter\def\expandafter$1\expandafter{\PE@result}%
  }%
  \gdef\PE@DeName$1#$2$3{%
    \ifx\relax$2%
      \edef\PE@result{\PE@result$1}%
      \let\PE@next\relax
    \else
      \ifx\relax$3%
        % wrong escape sequence in input
        \edef\PE@result{\PE@result$1#}%
        \let\PE@next\relax
      \else
        \uppercase{%
          \def\PE@testA{$2}%
          \def\PE@testB{$3}%
        }%
        \ifcase\ifcase\expandafter\PE@TestUcHexDigit\PE@testA
                 \ifcase\expandafter\PE@TestUcHexDigit\PE@testB
                   \ltx@zero
                 \else
                   \ltx@one
                 \fi
               \else
                 \ltx@one
               \fi
          \uccode\ltx@zero="\PE@testA\PE@testB\relax
          \uppercase{%
            \def\PE@temp{^^@}%
          }%
          \uccode\ltx@zero=\ltx@zero
          \edef\PE@result{\PE@result$1\PE@temp}%
          \let\PE@next\PE@DeName
        \else
          % wrong escape sequence in input
          \edef\PE@result{\PE@result$1#}%
          \def\PE@next{\PE@DeName$2$3}%
        \fi
      \fi
    \fi
    \PE@next
  }%
\endgroup
%    \end{macrocode}
%    \end{macro}
%
%    \begin{macro}{\PE@InitUccodeHexDigit}
%    \begin{macrocode}
\def\PE@InitUccodeHexDigit{%
  \uccode`a=`A\relax
  \uccode`b=`B\relax
  \uccode`c=`C\relax
  \uccode`d=`D\relax
  \uccode`e=`E\relax
  \uccode`f=`F\relax
  \uccode`A=\ltx@zero
  \uccode`B=\ltx@zero
  \uccode`C=\ltx@zero
  \uccode`D=\ltx@zero
  \uccode`E=\ltx@zero
  \uccode`F=\ltx@zero
  \uccode`0=\ltx@zero
  \uccode`1=\ltx@zero
  \uccode`2=\ltx@zero
  \uccode`3=\ltx@zero
  \uccode`4=\ltx@zero
  \uccode`5=\ltx@zero
  \uccode`6=\ltx@zero
  \uccode`7=\ltx@zero
  \uccode`8=\ltx@zero
  \uccode`9=\ltx@zero
}
%    \end{macrocode}
%    \end{macro}
%
%    \begin{macro}{\PE@TestUcHexDigit}
%    \begin{macrocode}
\def\PE@TestUcHexDigit#1{%
  \ifnum`#1<48 % 0
    \ltx@one
  \else
    \ifnum`#1>70 % F
      \ltx@one
    \else
      \ifnum`#1>57 % 9
        \ifnum`#1<65 % A
          \ltx@one
        \else
          \ltx@zero
        \fi
      \else
        \ltx@zero
      \fi
    \fi
  \fi
}
%    \end{macrocode}
%    \end{macro}
%
% \subsection{\cs{EdefUnescapeString}}
%
%    \begin{macro}{\EdefUnescapeString}
%    \begin{macrocode}
\def\EdefUnescapeString#1#2{%
  \EdefSanitize#1{#2}%
  \PE@SanitizeSpaceOther#1%
  \PE@NormalizeLineEnd#1%
  \PE@UnescapeString#1%
  \PE@onelevel@sanitize#1%
}
%    \end{macrocode}
%    \end{macro}
%
%    \begin{macrocode}
\begingroup
  \uccode`\8=10 % lf
  \uccode`\9=13 % cr
\def\x#1#2{\endgroup
%    \end{macrocode}
%    \begin{macro}{\PE@NormalizeLineEnd}
%    \begin{macrocode}
  \def\PE@NormalizeLineEnd##1{%
    \def\PE@result{}%
    \expandafter\PE@@NormalizeLineEnd##1#2\relax
    \let##1\PE@result
  }%
%    \end{macrocode}
%    \end{macro}
%    \begin{macro}{\PE@@NormalizeLineEnd}
%    \begin{macrocode}
  \def\PE@@NormalizeLineEnd##1#2##2{%
    \ifx\relax##2%
      \edef\PE@result{\PE@result##1}%
      \let\PE@next\relax
    \else
      \edef\PE@result{\PE@result##1#1}%
      \ifx#1##2% lf
        \let\PE@next\PE@@NormalizeLineEnd
      \else
        \def\PE@next{\PE@@NormalizeLineEnd##2}%
      \fi
    \fi
    \PE@next
  }%
}%
\uppercase{%
  \x 89%
}
%    \end{macrocode}
%    \end{macro}
%
%    \begin{macrocode}
\begingroup
  \catcode`\|=0 %
  \catcode`\\=12 %
%    \end{macrocode}
%    \begin{macro}{\PE@UnescapeString}
%    \begin{macrocode}
  |gdef|PE@UnescapeString#1{%
    |begingroup
      |def|PE@result{}%
      |expandafter|PE@DeString#1\|relax
    |expandafter|endgroup
    |expandafter|def|expandafter#1|expandafter{|PE@result}%
  }%
%    \end{macrocode}
%    \end{macro}
%    \begin{macro}{\PE@DeString}
%    \begin{macrocode}
  |gdef|PE@DeString#1\#2{%
    |ifx|relax#2%
      |edef|PE@result{|PE@result#1}%
      |let|PE@next|relax
    |else
      |if n#2%
        |uccode|ltx@zero=10 %
      |else|if r#2%
        |uccode|ltx@zero=13 %
      |else|if t#2%
        |uccode|ltx@zero=9 %
      |else|if b#2%
        |uccode|ltx@zero=8 %
      |else|if f#2%
        |uccode|ltx@zero=12 %
      |else
        |uccode|ltx@zero=|ltx@zero
      |fi|fi|fi|fi|fi
      |ifnum|uccode|ltx@zero>|ltx@zero
        |uppercase{%
          |edef|PE@temp{^^@}%
        }%
        |edef|PE@result{|PE@result#1|PE@temp}%
        |let|PE@next|PE@DeString
      |else
        |if\#2% backslash
          |edef|PE@result{|PE@result#1}%
          |let|PE@next|PE@CheckEndBackslash
        |else
          |ifnum`#2=10 % linefeed
            |edef|PE@result{|PE@result#1}%
            |let|PE@next|PE@DeString
          |else
            |ifcase|PE@TestOctDigit#2%
              |edef|PE@result{|PE@result#1}%
              |def|PE@next{|PE@OctI#2}%
            |else
              |edef|PE@result{|PE@result#1#2}%
              |let|PE@next|PE@DeString
            |fi
          |fi
        |fi
      |fi
    |fi
    |PE@next
  }%
%    \end{macrocode}
%    \end{macro}
%    \begin{macro}{\PE@CheckEndBackslash}
%    \begin{macrocode}
  |gdef|PE@CheckEndBackslash#1{%
    |ifx|relax#1%
    |else
      |edef|PE@result{|PE@result\}%
      |expandafter|PE@DeString|expandafter#1%
    |fi
  }%
%    \end{macrocode}
%    \end{macro}
%    \begin{macrocode}
|endgroup
%    \end{macrocode}
%    \begin{macro}{\PE@TestOctDigit}
%    \begin{macrocode}
\def\PE@TestOctDigit#1{%
  \ifnum`#1<48 % 0
    \ltx@one
  \else
    \ifnum`#1>55 % 7
      \ltx@one
    \else
      \ltx@zero
    \fi
  \fi
}
%    \end{macrocode}
%    \end{macro}
%    \begin{macro}{\PE@OctI}
%    \begin{macrocode}
\def\PE@OctI#1#2{%
  \ifcase\PE@TestOctDigit#2%
    \def\PE@next{\PE@OctII{#1#2}}%
  \else
    \def\PE@next{\PE@OctAll#1#2}%
  \fi
  \PE@next
}
%    \end{macrocode}
%    \end{macro}
%    \begin{macro}{\PE@OctII}
%    \begin{macrocode}
\def\PE@OctII#1#2{%
  \ifcase\PE@TestOctDigit#2%
    \def\PE@next{\PE@OctIII#1#2}%
  \else
    \def\PE@next{\PE@OctAll{#1}#2}%
  \fi
  \PE@next
}
%    \end{macrocode}
%    \end{macro}
%    \begin{macrocode}
\ltx@IfUndefined{numexpr}{%
  \catcode`\$=9 %
  \catcode`\&=14 %
}{%
  \catcode`\$=14 %
  \catcode`\&=9 %
}
%    \end{macrocode}
%    \begin{macro}{\PE@OctIII}
%    \begin{macrocode}
\def\PE@OctIII#1#2#3{%
  \ifnum#1<4 %
    \def\PE@next{\PE@OctAll{#1#2#3}}%
  \else
$   \count\ltx@cclv#1 %
$   \advance\count\ltx@cclv -4 %
    \edef\PE@next{%
      \noexpand\PE@OctAll{%
$       \the\count\ltx@cclv
&       \the\numexpr#1-4\relax
        #2#3%
      }%
    }%
  \fi
  \PE@next
}
%    \end{macrocode}
%    \end{macro}
%    \begin{macro}{\PE@OctAll}
%    \begin{macrocode}
\def\PE@OctAll#1{%
  \uccode\ltx@zero='#1\relax
  \uppercase{%
    \edef\PE@result{\PE@result^^@}%
  }%
  \PE@DeString
}
%    \end{macrocode}
%    \end{macro}
%
% \subsection{User macros (\pdfTeX\space analogues)}
%
%    \begin{macrocode}
\begingroup\expandafter\expandafter\expandafter\endgroup
\expandafter\ifx\csname RequirePackage\endcsname\relax
  \def\TMP@RequirePackage#1[#2]{%
    \begingroup\expandafter\expandafter\expandafter\endgroup
    \expandafter\ifx\csname ver@#1.sty\endcsname\relax
      \input #1.sty\relax
    \fi
  }%
  \TMP@RequirePackage{pdftexcmds}[2007/11/11]%
\else
  \RequirePackage{pdftexcmds}[2007/11/11]%
\fi
%    \end{macrocode}
%
%    \begin{macrocode}
\begingroup\expandafter\expandafter\expandafter\endgroup
\expandafter\ifx\csname pdf@escapehex\endcsname\relax
%    \end{macrocode}
%
%    \begin{macro}{\EdefEscapeHex}
%    \begin{macrocode}
  \long\def\EdefEscapeHex#1#2{%
    \EdefSanitize#1{#2}%
    \PE@SanitizeSpaceOther#1%
    \PE@EscapeHex#1%
  }%
%    \end{macrocode}
%    \end{macro}
%    \begin{macro}{\EdefUnescapeHex}
%    \begin{macrocode}
  \def\EdefUnescapeHex#1#2{%
    \EdefSanitize#1{#2}%
    \PE@UnescapeHex#1%
  }%
%    \end{macrocode}
%    \end{macro}
%    \begin{macro}{\EdefEscapeName}
%    \begin{macrocode}
  \long\def\EdefEscapeName#1#2{%
    \EdefSanitize#1{#2}%
    \PE@SanitizeSpaceOther#1%
    \PE@EscapeName#1%
  }%
%    \end{macrocode}
%    \end{macro}
%    \begin{macro}{\EdefEscapeString}
%    \begin{macrocode}
  \long\def\EdefEscapeString#1#2{%
    \EdefSanitize#1{#2}%
    \PE@SanitizeSpaceOther#1%
    \PE@EscapeString#1%
  }%
%    \end{macrocode}
%    \end{macro}
%
%    \begin{macrocode}
\else
%    \end{macrocode}
%
%    \begin{macro}{\PE@edefbabel}
%    Help macro that adds support for \xpackage{babel}'s shorthand
%    characters.
%    \begin{macrocode}
  \long\def\PE@edefbabel#1#2#3{%
    \begingroup
      \csname @save@activestrue\endcsname
      \edef#1{#2{#3}}%
    \expandafter\endgroup
    \expandafter\def\expandafter#1\expandafter{#1}%
  }%
%    \end{macrocode}
%    \end{macro}
%
%    \begin{macro}{\EdefEscapeHex}
%    \begin{macrocode}
  \long\def\EdefEscapeHex#1#2{%
    \PE@edefbabel#1\pdf@escapehex{#2}%
  }%
%    \end{macrocode}
%    \end{macro}
%    \begin{macro}{\EdefUnescapeHex}
%    \begin{macrocode}
  \def\EdefUnescapeHex#1#2{%
    \PE@edefbabel#1\pdf@unescapehex{#2}%
  }%
%    \end{macrocode}
%    \end{macro}
%    \begin{macro}{\EdefEscapeName}
%    \begin{macrocode}
  \long\def\EdefEscapeName#1#2{%
    \PE@edefbabel#1\pdf@escapename{#2}%
  }%
%    \end{macrocode}
%    \end{macro}
%    \begin{macro}{\EdefEscapeString}
%    \begin{macrocode}
  \long\def\EdefEscapeString#1#2{%
    \PE@edefbabel#1\pdf@escapestring{#2}%
  }%
%    \end{macrocode}
%    \end{macro}
%
%    \begin{macrocode}
  \expandafter\PE@AtEnd
\fi%
%    \end{macrocode}
%
% \subsection{Help macros}
%
% \subsubsection{Characters}
%
%    Special characters with catcode 12 (other) are created and
%    stored in macros.
%    \begin{macro}{\PE@hash}
%    \begin{macrocode}
\edef\PE@hash{\string#}
%    \end{macrocode}
%    \end{macro}
%    \begin{macro}{\PE@backslash}
%    \begin{macrocode}
\begingroup
  \escapechar=-1 %
\edef\x{\endgroup
  \def\noexpand\PE@backslash{\string\\}%
}
\x
%    \end{macrocode}
%    \end{macro}
%
% \subsubsection{Switch for \eTeX}
%
%    \begin{macrocode}
\ltx@newif\ifPE@etex
\begingroup\expandafter\expandafter\expandafter\endgroup
\expandafter\ifx\csname numexpr\endcsname\relax
\else
  \PE@etextrue
\fi
%    \end{macrocode}
%
% \subsection{Conversions}
%
% \subsubsection{Conversion to hex string}
%
%    \begin{macro}{\PE@EscapeHex}
%    \begin{macrocode}
\ifPE@etex
  \def\PE@EscapeHex#1{%
    \edef#1{\expandafter\PE@ToHex#1\relax}%
  }%
\else
  \def\PE@EscapeHex#1{%
    \def\PE@result{}%
    \expandafter\PE@ToHex#1\relax
    \let#1\PE@result
  }%
\fi
%    \end{macrocode}
%    \end{macro}
%    \begin{macro}{\PE@ToHex}
%    \begin{macrocode}
\def\PE@ToHex#1{%
  \ifx\relax#1%
  \else
    \PE@HexChar{#1}%
    \expandafter\PE@ToHex
  \fi
}%
%    \end{macrocode}
%    \end{macro}
%    \begin{macro}{\PE@HexChar}
%    \begin{macrocode}
\ifPE@etex
  \def\PE@HexChar#1{%
    \PE@HexDigit{\numexpr\dimexpr.0625\dimexpr`#1sp\relax\relax\relax}%
    \PE@HexDigit{%
      \numexpr`#1-16*\dimexpr.0625\dimexpr`#1sp\relax\relax\relax
    }%
  }%
\else
  \def\PE@HexChar#1{%
    \dimen0=`#1sp%
    \dimen2=.0625\dimen0 %
    \advance\dimen0-16\dimen2 %
    \edef\PE@result{%
      \PE@result
      \PE@HexDigit{\dimen2 }%
      \PE@HexDigit{\dimen0 }%
    }%
  }%
\fi
%    \end{macrocode}
%    \end{macro}
%    \begin{macro}{\PE@HexDigit}
%    \begin{macrocode}
\def\PE@HexDigit#1{%
  \expandafter\string
  \ifcase#1%
    0\or 1\or 2\or 3\or 4\or 5\or 6\or 7\or 8\or 9\or
    A\or B\or C\or D\or E\or F%
  \fi
}
%    \end{macrocode}
%    \end{macro}
%
% \subsubsection{Character code to octal number}
%
%    \begin{macro}{\PE@OctChar}
%    \begin{macrocode}
\ifPE@etex
  \def\PE@OctChar#1{%
    \expandafter\PE@@OctChar
        \the\numexpr\dimexpr.015625\dimexpr`#1sp\relax\relax
            \expandafter\relax
        \expandafter\relax
        \the\numexpr\dimexpr.125\dimexpr`#1sp\relax\relax\relax
        \relax
        #1%
  }%
  \def\PE@@OctChar#1\relax#2\relax#3{%
    \PE@backslash
    #1%
    \the\numexpr#2-8*#1\relax
    \the\numexpr\dimexpr`#3sp\relax-8*#2\relax
  }%
\else
  \def\PE@OctChar#1{%
    \dimen0=`#1sp%
    \dimen2=.125\dimen0 %
    \dimen4=.125\dimen2 %
    \advance\dimen0-8\dimen2 %
    \advance\dimen2-8\dimen4 %
    \edef\PE@result{%
      \PE@result
      \PE@backslash
      \number\dimen4 %
      \number\dimen2 %
      \number\dimen0 %
    }%
  }%
\fi
%    \end{macrocode}
%    \end{macro}
%
% \subsubsection{Unpack hex string}
%
%    \begin{macro}{\PE@UnescapeHex}
%    \begin{macrocode}
\def\PE@UnescapeHex#1{%
  \begingroup
    \PE@InitUccodeHexDigit
    \def\PE@result{}%
    \expandafter\PE@DeHex#1\relax\relax
  \expandafter\endgroup
  \expandafter\def\expandafter#1\expandafter{\PE@result}%
}
%    \end{macrocode}
%    \end{macro}
%    \begin{macro}{\PE@DeHex}
%    \begin{macrocode}
\def\PE@DeHex#1#2{%
  \ifx#2\relax
    \ifx#1\relax
      \let\PE@next\relax
    \else
      \uppercase{%
        \def\PE@testA{#1}%
      }%
      \ifcase\expandafter\PE@TestUcHexDigit\PE@testA
        \def\PE@next{%
          \PE@DeHex#10\relax\relax
        }%
      \else
        \let\PE@next\relax
      \fi
    \fi
  \else
    \uppercase{%
      \def\PE@testA{#1}%
      \def\PE@testB{#2}%
    }%
    \ifcase\expandafter\PE@TestUcHexDigit\PE@testA
      \ifcase\expandafter\PE@TestUcHexDigit\PE@testB
        \uccode\ltx@zero="\PE@testA\PE@testB\relax
        \ifnum\uccode\ltx@zero=32 %
          \let\PE@temp\PE@space@space
        \else
          \uppercase{%
            \def\PE@temp{^^@}%
          }%
        \fi
        \edef\PE@result{\PE@result\PE@temp}%
        \let\PE@next\PE@DeHex
      \else
        % invalid input sequence
        \def\PE@next{%
          \PE@DeHex#1%
        }%
      \fi
    \else
      % invalid input sequence
      \def\PE@next{\PE@DeHex#2}%
    \fi
  \fi
  \PE@next
}
%    \end{macrocode}
%    \end{macro}
%
% \subsubsection{Conversion to PDF name}
%
%    \begin{macro}{\PE@EscapeName}
%    \begin{macrocode}
\ifPE@etex
  \def\PE@EscapeName#1{%
    \edef#1{\expandafter\PE@EscapeNameTokens#1\relax}%
  }%
\else
  \def\PE@EscapeName#1{%
    \def\PE@result{}%
    \expandafter\PE@EscapeNameTokens#1\relax
    \let#1\PE@result
  }%
\fi
%    \end{macrocode}
%    \end{macro}
%    \begin{macro}{\PE@EscapeNameTokens}
%    \begin{macrocode}
\def\PE@EscapeNameTokens#1{%
  \ifx\relax#1%
  \else
    \ifnum`#1<33 %
      \ifcase`#1 %
        % drop illegal zero
      \else
        \PE@EscapeNameAdd\PE@hash
        \PE@HexChar#1%
      \fi
    \else
      \ifnum`#1>126 %
        \PE@EscapeNameAdd\PE@hash
        \PE@HexChar#1%
      \else  \ifnum`#1=35  \PE@EscapeNameHashChar 23% #
        \else\ifnum`#1=37  \PE@EscapeNameHashChar 25% %
        \else\ifnum`#1=40  \PE@EscapeNameHashChar 28% (
        \else\ifnum`#1=41  \PE@EscapeNameHashChar 29% )
        \else\ifnum`#1=47  \PE@EscapeNameHashChar 2F% /
        \else\ifnum`#1=60  \PE@EscapeNameHashChar 3C% <
        \else\ifnum`#1=62  \PE@EscapeNameHashChar 3E% >
        \else\ifnum`#1=91  \PE@EscapeNameHashChar 5B% [
        \else\ifnum`#1=93  \PE@EscapeNameHashChar 5D% ]
        \else\ifnum`#1=123 \PE@EscapeNameHashChar 7B% {
        \else\ifnum`#1=125 \PE@EscapeNameHashChar 7D% }
        \else
          \PE@EscapeNameAdd{#1}%
        \fi\fi\fi\fi\fi\fi\fi\fi\fi\fi\fi
      \fi
    \fi
    \expandafter\PE@EscapeNameTokens
  \fi
}%
\def\PE@EscapeNameHashChar#1#2{%
  \PE@EscapeNameAdd{\PE@hash\string#1\string#2}%
}%
%    \end{macrocode}
%    \end{macro}
%    \begin{macro}{\PE@EscapeNameAdd}
%    \begin{macrocode}
\ifPE@etex
  \def\PE@EscapeNameAdd#1{#1}%
\else
  \def\PE@EscapeNameAdd#1{%
    \edef\PE@result{%
      \PE@result
      #1%
    }%
  }%
\fi
%    \end{macrocode}
%    \end{macro}
%
% \subsubsection{Conversion to PDF string}
%
%    \begin{macro}{\PE@EscapeString}
%    \begin{macrocode}
\ifPE@etex
  \def\PE@EscapeString#1{%
    \edef#1{\expandafter\PE@EscapeStringTokens#1\relax}%
  }%
\else
  \def\PE@EscapeString#1{%
    \begingroup
      \def\PE@result{}%
      \expandafter\PE@EscapeStringTokens#1\relax
    \expandafter\endgroup
    \expandafter\def\expandafter#1\expandafter{\PE@result}%
  }%
\fi
%    \end{macrocode}
%    \end{macro}
%    \begin{macro}{\PE@EscapeStringTokens}
%    \begin{macrocode}
\def\PE@EscapeStringTokens#1{%
  \ifx\relax#1%
  \else
    \ifnum`#1<33 %
      \PE@OctChar#1%
    \else
      \ifnum`#1>126 %
        \PE@OctChar#1%
      \else  \ifnum`#1=40 \PE@EscapeStringAdd{\string\(}% (
        \else\ifnum`#1=41 \PE@EscapeStringAdd{\string\)}% )
        \else\ifnum`#1=92 \PE@EscapeStringAdd{\string\\}% \
        \else
          \PE@EscapeStringAdd{#1}%
        \fi\fi\fi
      \fi
    \fi
    \expandafter\PE@EscapeStringTokens
  \fi
}%
%    \end{macrocode}
%    \end{macro}
%    \begin{macro}{\PE@EscapeStringAdd}
%    \begin{macrocode}
\ifPE@etex
  \def\PE@EscapeStringAdd#1{#1}%
\else
  \def\PE@EscapeStringAdd#1{%
    \edef\PE@result{%
      \PE@result
      #1%
    }%
  }%
\fi
%    \end{macrocode}
%    \end{macro}
%
%    \begin{macrocode}
\PE@AtEnd%
%    \end{macrocode}
%
%    \begin{macrocode}
%</package>
%    \end{macrocode}
%
% \section{Test}
%
% \subsection{Catcode checks for loading}
%
%    \begin{macrocode}
%<*test1>
%    \end{macrocode}
%    \begin{macrocode}
\catcode`\{=1 %
\catcode`\}=2 %
\catcode`\#=6 %
\catcode`\@=11 %
\expandafter\ifx\csname count@\endcsname\relax
  \countdef\count@=255 %
\fi
\expandafter\ifx\csname @gobble\endcsname\relax
  \long\def\@gobble#1{}%
\fi
\expandafter\ifx\csname @firstofone\endcsname\relax
  \long\def\@firstofone#1{#1}%
\fi
\expandafter\ifx\csname loop\endcsname\relax
  \expandafter\@firstofone
\else
  \expandafter\@gobble
\fi
{%
  \def\loop#1\repeat{%
    \def\body{#1}%
    \iterate
  }%
  \def\iterate{%
    \body
      \let\next\iterate
    \else
      \let\next\relax
    \fi
    \next
  }%
  \let\repeat=\fi
}%
\def\RestoreCatcodes{}
\count@=0 %
\loop
  \edef\RestoreCatcodes{%
    \RestoreCatcodes
    \catcode\the\count@=\the\catcode\count@\relax
  }%
\ifnum\count@<255 %
  \advance\count@ 1 %
\repeat

\def\RangeCatcodeInvalid#1#2{%
  \count@=#1\relax
  \loop
    \catcode\count@=15 %
  \ifnum\count@<#2\relax
    \advance\count@ 1 %
  \repeat
}
\def\RangeCatcodeCheck#1#2#3{%
  \count@=#1\relax
  \loop
    \ifnum#3=\catcode\count@
    \else
      \errmessage{%
        Character \the\count@\space
        with wrong catcode \the\catcode\count@\space
        instead of \number#3%
      }%
    \fi
  \ifnum\count@<#2\relax
    \advance\count@ 1 %
  \repeat
}
\def\space{ }
\expandafter\ifx\csname LoadCommand\endcsname\relax
  \def\LoadCommand{\input pdfescape.sty\relax}%
\fi
\def\Test{%
  \RangeCatcodeInvalid{0}{47}%
  \RangeCatcodeInvalid{58}{64}%
  \RangeCatcodeInvalid{91}{96}%
  \RangeCatcodeInvalid{123}{255}%
  \catcode`\@=12 %
  \catcode`\\=0 %
  \catcode`\%=14 %
  \LoadCommand
  \RangeCatcodeCheck{0}{36}{15}%
  \RangeCatcodeCheck{37}{37}{14}%
  \RangeCatcodeCheck{38}{47}{15}%
  \RangeCatcodeCheck{48}{57}{12}%
  \RangeCatcodeCheck{58}{63}{15}%
  \RangeCatcodeCheck{64}{64}{12}%
  \RangeCatcodeCheck{65}{90}{11}%
  \RangeCatcodeCheck{91}{91}{15}%
  \RangeCatcodeCheck{92}{92}{0}%
  \RangeCatcodeCheck{93}{96}{15}%
  \RangeCatcodeCheck{97}{122}{11}%
  \RangeCatcodeCheck{123}{255}{15}%
  \RestoreCatcodes
}
\Test
\csname @@end\endcsname
\end
%    \end{macrocode}
%    \begin{macrocode}
%</test1>
%    \end{macrocode}
%
% \subsection{Macro tests}
%
%    \begin{macrocode}
%<*test2|test3|test4|test5>
\NeedsTeXFormat{LaTeX2e}
\makeatletter
%    \end{macrocode}
%
% \subsection{Test with \cs{pdfescape...} commands}
%
%    \begin{macrocode}
%<*test2>
\ProvidesFile{pdfescape-test2.tex}%
    [2011/11/25 v1.13 Test with \string\pdfescape... commands]%
%</test2>
%    \end{macrocode}
%
% \subsection{Test without \cs{pdfescape...}, with \eTeX}
%
%    \begin{macrocode}
%<*test3>
\ProvidesFile{pdfescape-test3.tex}%
    [2011/11/25 v1.13 Test without \string\pdfescape..., with e-TeX]%
%</test3>
%    \end{macrocode}
%
% \subsection{Test without \cs{pdfescape...} and \eTeX}
%
%    \begin{macrocode}
%<*test4>
\ProvidesFile{pdfescape-test4.tex}%
    [2011/11/25 v1.13 Test without \string\pdfescape... and e-TeX]%
%</test4>
%    \end{macrocode}
%
% \subsection{Test with \LuaTeX}
%
%    \begin{macrocode}
%<*test5>
\ProvidesFile{pdfescape-test5.tex}%
    [2011/11/25 v1.13 Test with LuaTeX]%
%</test5>
%    \end{macrocode}
%
% \subsection{Check/ensure test preconditions}
%
% \subsubsection{Check \cs{pdfescape...}}
%
%    \begin{macrocode}
%<*test2>
\@ifundefined{pdfescapehex}{%
  \PackageError{pdfescape-test2}{%
    Missing \string\pdfescape... commands%
  }{Test aborted.}%
  \stop
}{}
%</test2>
%    \end{macrocode}
%    \begin{macrocode}
%<*test3|test4>
\let\pdfescapehex\@undefined
\let\pdfunescapehex\@undefined
\let\pdfescapename\@undefined
\let\pdfescapestring\@undefined
%</test3|test4>
%    \end{macrocode}
%
% \subsubsection{Check \eTeX}
%
%    \begin{macrocode}
%<*test3>
\@ifundefined{numexpr}{%
  \PackageError{pdfescape-test3}{%
    Missing \eTeX
  }{Test aborted.}%
  \stop
}{}
%</test3>
%    \end{macrocode}
% Package \xpackage{qstest} uses \eTeX, thus \eTeX's features
% can only be disabled later during loading of package \xpackage{pdfescape}.
%
% \subsubsection{Check \LuaTeX}
%
%    \begin{macrocode}
%<*test5>
\@ifundefined{directlua}{%
  \PackageError{pdfescape-test5}{%
    Missing LuaTeX%
  }{Test aborted.}%
  \stop
}{}
%</test5>
%    \end{macrocode}
%
% \subsection{Common part}
%
%    The files for testing uses the framework, provided by the
%    new package \xpackage{qstest} of David Kastrup.
%    \begin{macrocode}
\RequirePackage{qstest}
\IncludeTests{*}
\LogTests{log}{*}{*}

\newcommand*{\ExpectVar}[2]{%
  \ifx#1#2%
  \else
    \begingroup
      \@onelevel@sanitize#1%
      \@onelevel@sanitize#2%
      \typeout{[#1] <> [#2]}% hash-ok
    \endgroup
  \fi
  \Expect*{\ifx#1#2true\else false\fi}{true}%
}

\makeatletter
\begingroup
  \gdef\AllBytes{}%
  \count@=0 %
  \catcode0=12 %
  \@whilenum\count@<256 \do{%
    \lccode0=\count@
    \ifnum\count@=32 %
      \xdef\AllBytes{\AllBytes\space}%
    \else
      \lowercase{%
        \xdef\AllBytes{\AllBytes^^@}%
      }%
    \fi
    \advance\count@ by 1 %
  }%
\endgroup
\newcommand*{\AllBytesHex}{%
  000102030405060708090A0B0C0D0E0F%
  101112131415161718191A1B1C1D1E1F%
  202122232425262728292A2B2C2D2E2F%
  303132333435363738393A3B3C3D3E3F%
  404142434445464748494A4B4C4D4E4F%
  505152535455565758595A5B5C5D5E5F%
  606162636465666768696A6B6C6D6E6F%
  707172737475767778797A7B7C7D7E7F%
  808182838485868788898A8B8C8D8E8F%
  909192939495969798999A9B9C9D9E9F%
  A0A1A2A3A4A5A6A7A8A9AAABACADAEAF%
  B0B1B2B3B4B5B6B7B8B9BABBBCBDBEBF%
  C0C1C2C3C4C5C6C7C8C9CACBCCCDCECF%
  D0D1D2D3D4D5D6D7D8D9DADBDCDDDEDF%
  E0E1E2E3E4E5E6E7E8E9EAEBECEDEEEF%
  F0F1F2F3F4F5F6F7F8F9FAFBFCFDFEFF%
}
\@onelevel@sanitize\AllBytesHex
\expandafter\lowercase\expandafter{%
  \expandafter\newcommand\expandafter*\expandafter\AllBytesHexLC
      \expandafter{\AllBytesHex}%
}
\newcommand*{\AllBytesName}{}
\begingroup
  \catcode`\#=12 %
  \xdef\AllBytesName{%
    #01#02#03#04#05#06#07#08#09#0A#0B#0C#0D#0E#0F%
    #10#11#12#13#14#15#16#17#18#19#1A#1B#1C#1D#1E#1F%
    #20!"#23$#25&'#28#29*+,-.#2F%
    0123456789:;#3C=#3E?%
    @ABCDEFGHIJKLMNO%
    PQRSTUVWXYZ#5B\@backslashchar#5D^_%
    `abcdefghijklmno%
    pqrstuvwxyz#7B|#7D\string~#7F%
    #80#81#82#83#84#85#86#87#88#89#8A#8B#8C#8D#8E#8F%
    #90#91#92#93#94#95#96#97#98#99#9A#9B#9C#9D#9E#9F%
    #A0#A1#A2#A3#A4#A5#A6#A7#A8#A9#AA#AB#AC#AD#AE#AF%
    #B0#B1#B2#B3#B4#B5#B6#B7#B8#B9#BA#BB#BC#BD#BE#BF%
    #C0#C1#C2#C3#C4#C5#C6#C7#C8#C9#CA#CB#CC#CD#CE#CF%
    #D0#D1#D2#D3#D4#D5#D6#D7#D8#D9#DA#DB#DC#DD#DE#DF%
    #E0#E1#E2#E3#E4#E5#E6#E7#E8#E9#EA#EB#EC#ED#EE#EF%
    #F0#F1#F2#F3#F4#F5#F6#F7#F8#F9#FA#FB#FC#FD#FE#FF%
  }%
\endgroup
\@onelevel@sanitize\AllBytesName

\newcommand*{\AllBytesString}{}
\begingroup
  \def\|{|}%
  \edef\%{\@percentchar}%
  \catcode`\|=0 %
  \catcode`\#=12 %
  \catcode`\~=12 %
  \catcode`\\=12 %
  |xdef|AllBytesString{%
    \000\001\002\003\004\005\006\007\010\011\012\013\014\015\016\017%
    \020\021\022\023\024\025\026\027\030\031\032\033\034\035\036\037%
    \040!"#$|%&'\(\)*+,-./%
    0123456789:;<=>?%
    @ABCDEFGHIJKLMNO%
    PQRSTUVWXYZ[\\]^_%
    `abcdefghijklmno%
    pqrstuvwxyz{||}~\177%
    \200\201\202\203\204\205\206\207\210\211\212\213\214\215\216\217%
    \220\221\222\223\224\225\226\227\230\231\232\233\234\235\236\237%
    \240\241\242\243\244\245\246\247\250\251\252\253\254\255\256\257%
    \260\261\262\263\264\265\266\267\270\271\272\273\274\275\276\277%
    \300\301\302\303\304\305\306\307\310\311\312\313\314\315\316\317%
    \320\321\322\323\324\325\326\327\330\331\332\333\334\335\336\337%
    \340\341\342\343\344\345\346\347\350\351\352\353\354\355\356\357%
    \360\361\362\363\364\365\366\367\370\371\372\373\374\375\376\377%
  }%
|endgroup
\@onelevel@sanitize\AllBytesString

%<*test4>
\let\org@detokenize\detokenize
\let\detokenize\@undefined
\let\org@numexpr\numexpr
\let\numexpr\@undefined
%</test4>
\RequirePackage{pdfescape}
%<*test4>
\let\detokenize\org@detokenize
\let\numexpr\org@numexpr
%</test4>

\begin{qstest}{all-hex}{\AllBytes, escapehex}
  \EdefEscapeHex\x{\AllBytes}%
  \Expect*{\x}*{\AllBytesHex}%
  \ExpectVar\x\AllBytesHex
\end{qstest}

\begin{qstest}{all-unhex}{\AllBytesHex, unescapehex}
  \EdefUnescapeHex\x{\AllBytesHex}%
  \Expect*{\x}*{\AllBytes}%
  \ExpectVar\x\AllBytes
\end{qstest}

\begin{qstest}{all-unhex-lc}{\AllBytesHexLC, unescapehex, lowercase}
  \EdefUnescapeHex\x{\AllBytesHexLC}%
  \Expect*{\x}*{\AllBytes}%
  \ExpectVar\x\AllBytes
\end{qstest}

\begin{qstest}{unhex-incomplete}{unescapehex, incomplete}
  \EdefUnescapeHex\x{4}%
  \Expect*{\x}{@}%
\end{qstest}

\begin{qstest}{unhex-space}{unescapehex, space}
  \EdefUnescapeHex\x{20}%
  \Expect*{\x}{ }%
  \ExpectVar\x\space
\end{qstest}

\begin{qstest}{unhex-spaces}{unescapehex, spaces}
  \EdefUnescapeHex\x{204020204120}%
  \def\y#1{%
    \edef\z{#1\string @#1#1\string A#1}%
  }\y{ }%
  \Expect*{\x}*{\z}%
  \ExpectVar\x\z
\end{qstest}

\begin{qstest}{unhex-hash}{unescapehex, hash}
  \catcode`\#=12 %
  \EdefUnescapeHex\x{#20}%
  \ExpectVar\x\space
\end{qstest}

\begin{qstest}{unhex-invalid}{unescapehex, invalid}
  \def\test#1#2{%
    \EdefUnescapeHex\x{#1}%
    \edef\y{#2}%
    \@onelevel@sanitize\y
    \ExpectVar\x\y
  }%
%<*test2>
  \edef\x{\pdfunescapehex{4X}}%
  \edef\y{\string @}%
  \ifx\x\y
  \else
    \def~{\space}%
    \typeout{************************************}%
    \typeout{* Your pdfTeX contains bug 777.~~~~*}%
    \typeout{* This test is redefined as dummy, *}%
    \typeout{* because it triggers the bug.~~~~~*}%
    \typeout{************************************}%
    \def\test#1#2{}%
  \fi
%</test2>
  \test{X}{}%
  \test{XY}{}%
  \test{XYZ}{}%
  \test{A}{^^a0}%
  \test{AX}{^^a0}%
  \test{XA}{^^a0}%
  \test{XXAXX}{^^a0}%
\end{qstest}

\begin{qstest}{all-name}{\AllBytes, escapename}
  \EdefEscapeName\x{\AllBytes}%
  \Expect*{\x}*{\AllBytesName}%
  \ExpectVar\x\AllBytesName
\end{qstest}

\begin{qstest}{all-string}{\AllBytes, escapestring}
  \EdefEscapeString\x{\AllBytes}%
  \Expect*{\x}*{\AllBytesString}%
  \ExpectVar\x\AllBytesString
\end{qstest}

\begin{qstest}{uchexdigit}{unescape, uppercase hex digit}
  \catcode`\@=11 %
  \catcode0=12 %
  \def\test#1#2{%
    \uccode0=#1\relax
    \uppercase{%
      \def\x{^^@}%
    }%
    \Expect*{%
       \ifcase\expandafter\PE@TestUcHexDigit\x
         true%
       \else
         false%
       \fi
    }{#2}%
  }%
  \def\range#1#2#3{%
    \count0=#1\relax
    \loop
    \ifnum\count0<#2\relax
      \test{\count0}{#3}%
      \advance\count0 by 1 %
    \repeat
  }%
  \range{0}{47}{false}%
  \range{48}{57}{true}%
  \range{58}{64}{false}%
  \range{65}{70}{true}%
  \range{71}{255}{false}%
\end{qstest}

\begin{qstest}{unescapename}{unescapename}
  \def\test#1#2{%
    \EdefUnescapeName\x{#1}%
    \edef\y{#2}%
    \@onelevel@sanitize\y
    \ExpectVar\x\y
  }%
  \catcode`\#=12 %
  \catcode0=12 %
  \test{}{}%
  \test{x}{x}%
  \test{xy}{xy}%
  \test{#}{#}%
  \test{##}{##}%
  \test{###}{###}%
  \test{####}{####}%
  \test{#x}{#x}%
  \test{#xy}{#xy}%
  \test{#1}{#1}%
  \test{#40}{@}%
  \test{#400}{@0}%
  \test{#4x0}{#4x0}%
  \test{#ab}{^^ab}%
  \test{#00}{^^@}%
  \test{x#40y#40z}{x@y@z}%
  \test{#40#40#40#40}{@@@@}%
  \test{a#x}{a#x}%
  \test{a#xy}{a#xy}%
  \test{a#1}{a#1}%
  \test{a#40}{a@}%
  \test{a#400}{a@0}%
  \test{#20}{ }%
  \test{a#20}{a }%
  \test{a#20b}{a b}%
  \test{a#20#20#20b}{a \space\space b}%
\end{qstest}

\begin{qstest}{unescapestring}{unescapestring}
  \def\test#1#2{%
    \EdefUnescapeString\x{#1}%
    \edef\y{#2}%
    \@onelevel@sanitize\y
    \ExpectVar\x\y
  }%
  \catcode0=12 %
  \def\DefChar#1#2{%
    \begingroup
      \uccode0=#2\relax
    \uppercase{\endgroup
      \def#1{^^@}%
    }%
  }%
  \DefChar\nul{0}%
  \DefChar\one{1}%
  \DefChar\bel{8}%
  \DefChar\tab{9}%
  \DefChar\lf{10}%
  \DefChar\ff{12}%
  \DefChar\cr{13}%
  \DefChar\\{92}%
  \test{}{}%
  \test{a}{a}%
  \test{\\}{}%
  \test{\\\\}{\\}%
  \test{\\\\y}{\\y}%
  \test{\\000}{\nul}%
  \test{\\b}{\bel}%
  \test{\\t}{\tab}%
  \test{\\n}{\lf}%
  \test{\\f}{\ff}%
  \test{\\r}{\cr}%
  \test{\\(}{(}%
  \test{\\)}{)}%
  \test{\\040}{ }%
  \test{\\100}{@}%
  \test{\\40}{ }%
  \test{\\1}{\one}%
  \test{\\01}{\one}%
  \test{\\001}{\one}%
  \test{\\18}{\one8}%
  \test{\\018}{\one8}%
  \test{\\0018}{\one8}%
  \test{x\\}{x}%
  \test{x\\\\}{x\\}%
  \test{x\\\\y}{x\\y}%
  \test{x\\000}{x\nul}%
  \test{x\\b}{x\bel}%
  \test{x\\t}{x\tab}%
  \test{x\\n}{x\lf}%
  \test{x\\f}{x\ff}%
  \test{x\\r}{x\cr}%
  \test{x\\(}{x(}%
  \test{x\\)}{x)}%
  \test{x\\040}{x }%
  \test{x\\100}{x@}%
  \test{x\\40}{x }%
  \test{x\\1}{x\one}%
  \test{x\\01}{x\one}%
  \test{x\\001}{x\one}%
  \test{x\\18}{x\one8}%
  \test{x\\018}{x\one8}%
  \test{x\\0018}{x\one8}%
  \test{\\b\\t\\n\\f\\r\\(\\)\\\\\\000\\040}{%
    \bel\tab\lf\ff\cr()\\\nul\space
  }%
  \test{\\\lf}{}%
  \test{x\\\lf}{x}%
  \test{\cr}{\lf}%
  \test{\cr\lf}{\lf}%
  \test{\lf}{\lf}%
  \test{\lf\cr}{\lf\lf}%
  \test{x\cr}{x\lf}%
  \test{x\cr\lf}{x\lf}%
  \test{x\lf}{x\lf}%
  \test{x\lf\cr}{x\lf\lf}%
  \test{x\\\cr\lf y\cr}{xy\lf}%
%
  \test{\\409}{ 9}%
  \test{\\800}{800}%
  \test{\\900}{900}%
  \test{\\578}{/8}%
  \test{\\477}{?}%
  \test{\\377}{^^ff}%
  \test{\\777}{^^ff}%
  \test{\\7777}{^^ff7}%
\end{qstest}
\stop
%    \end{macrocode}
%
%    \begin{macrocode}
%</test2|test3|test4|test5>
%    \end{macrocode}
%
% \subsubsection{Test for \hologo{iniTeX}}
%
%    \begin{macrocode}
%<*test6>
\input pdfescape.sty\relax
\catcode`\{=1 %
\catcode`\}=2 %
\catcode`\#=6 %
\catcode`\^=7 %
\catcode`\@=11 %
%    \end{macrocode}
%    \begin{macrocode}
\begingroup
  \catcode`\@=11 %
  \countdef\count@=255 %
  \def\space{ }%
  \long\def\@whilenum#1\do #2{%
    \ifnum #1\relax
      #2\relax
      \@iwhilenum{#1\relax#2\relax}%
    \fi
  }%
  \long\def\@iwhilenum#1{%
    \ifnum #1%
      \expandafter\@iwhilenum
    \else
      \expandafter\ltx@gobble
    \fi
    {#1}%
  }%
  \gdef\AllBytes{}%
  \count@=0 %
  \catcode0=12 %
  \@whilenum\count@<256 \do{%
    \lccode0=\count@
    \ifnum\count@=32 %
      \xdef\AllBytes{\AllBytes\space}%
    \else
      \lowercase{%
        \xdef\AllBytes{\AllBytes^^@}%
      }%
    \fi
    \advance\count@ by 1 %
  }%
\endgroup
\def\AllBytesHex{%
  000102030405060708090A0B0C0D0E0F%
  101112131415161718191A1B1C1D1E1F%
  202122232425262728292A2B2C2D2E2F%
  303132333435363738393A3B3C3D3E3F%
  404142434445464748494A4B4C4D4E4F%
  505152535455565758595A5B5C5D5E5F%
  606162636465666768696A6B6C6D6E6F%
  707172737475767778797A7B7C7D7E7F%
  808182838485868788898A8B8C8D8E8F%
  909192939495969798999A9B9C9D9E9F%
  A0A1A2A3A4A5A6A7A8A9AAABACADAEAF%
  B0B1B2B3B4B5B6B7B8B9BABBBCBDBEBF%
  C0C1C2C3C4C5C6C7C8C9CACBCCCDCECF%
  D0D1D2D3D4D5D6D7D8D9DADBDCDDDEDF%
  E0E1E2E3E4E5E6E7E8E9EAEBECEDEEEF%
  F0F1F2F3F4F5F6F7F8F9FAFBFCFDFEFF%
}
\ltx@onelevel@sanitize\AllBytesHex
\expandafter\lowercase\expandafter{%
  \expandafter\def\expandafter\AllBytesHexLC
      \expandafter{\AllBytesHex}%
}
\begingroup
  \catcode`\#=12 %
  \xdef\AllBytesName{%
    #01#02#03#04#05#06#07#08#09#0A#0B#0C#0D#0E#0F%
    #10#11#12#13#14#15#16#17#18#19#1A#1B#1C#1D#1E#1F%
    #20!"#23$#25&'#28#29*+,-.#2F%
    0123456789:;#3C=#3E?%
    @ABCDEFGHIJKLMNO%
    PQRSTUVWXYZ#5B\ltx@backslashchar#5D^_%
    `abcdefghijklmno%
    pqrstuvwxyz#7B|#7D\string~#7F%
    #80#81#82#83#84#85#86#87#88#89#8A#8B#8C#8D#8E#8F%
    #90#91#92#93#94#95#96#97#98#99#9A#9B#9C#9D#9E#9F%
    #A0#A1#A2#A3#A4#A5#A6#A7#A8#A9#AA#AB#AC#AD#AE#AF%
    #B0#B1#B2#B3#B4#B5#B6#B7#B8#B9#BA#BB#BC#BD#BE#BF%
    #C0#C1#C2#C3#C4#C5#C6#C7#C8#C9#CA#CB#CC#CD#CE#CF%
    #D0#D1#D2#D3#D4#D5#D6#D7#D8#D9#DA#DB#DC#DD#DE#DF%
    #E0#E1#E2#E3#E4#E5#E6#E7#E8#E9#EA#EB#EC#ED#EE#EF%
    #F0#F1#F2#F3#F4#F5#F6#F7#F8#F9#FA#FB#FC#FD#FE#FF%
  }%
\endgroup
\ltx@onelevel@sanitize\AllBytesName
\edef\AllBytesFromName{\expandafter\ltx@gobble\AllBytes}

\begingroup
  \def\|{|}%
  \edef\%{\ltx@percentchar}%
  \catcode`\|=0 %
  \catcode`\#=12 %
  \catcode`\~=12 %
  \catcode`\\=12 %
  |xdef|AllBytesString{%
    \000\001\002\003\004\005\006\007\010\011\012\013\014\015\016\017%
    \020\021\022\023\024\025\026\027\030\031\032\033\034\035\036\037%
    \040!"#$|%&'\(\)*+,-./%
    0123456789:;<=>?%
    @ABCDEFGHIJKLMNO%
    PQRSTUVWXYZ[\\]^_%
    `abcdefghijklmno%
    pqrstuvwxyz{||}~\177%
    \200\201\202\203\204\205\206\207\210\211\212\213\214\215\216\217%
    \220\221\222\223\224\225\226\227\230\231\232\233\234\235\236\237%
    \240\241\242\243\244\245\246\247\250\251\252\253\254\255\256\257%
    \260\261\262\263\264\265\266\267\270\271\272\273\274\275\276\277%
    \300\301\302\303\304\305\306\307\310\311\312\313\314\315\316\317%
    \320\321\322\323\324\325\326\327\330\331\332\333\334\335\336\337%
    \340\341\342\343\344\345\346\347\350\351\352\353\354\355\356\357%
    \360\361\362\363\364\365\366\367\370\371\372\373\374\375\376\377%
  }%
|endgroup
\ltx@onelevel@sanitize\AllBytesString
%    \end{macrocode}
%    \begin{macrocode}
\def\msg#{\immediate\write16}
\def\Test#1#2#3{%
  \begingroup
    #1\TestResult{#2}%
    \ifx\TestResult#3%
    \else
      \newlinechar=10 %
      \msg{Expect:^^J#3}%
      \msg{Result:^^J\TestResult}%
      \errmessage{\string#2 -\string#1-> \string#3}%
    \fi
  \endgroup
}
\Test\EdefEscapeHex\AllBytes\AllBytesHex
\Test\EdefUnescapeHex\AllBytesHex\AllBytes
\Test\EdefEscapeName\AllBytes\AllBytesName
\Test\EdefUnescapeName\AllBytesName\AllBytesFromName
\Test\EdefEscapeString\AllBytes\AllBytesString
\Test\EdefUnescapeString\AllBytesString\AllBytes
%    \end{macrocode}
%    \begin{macrocode}
\csname @@end\endcsname\end
%</test6>
%    \end{macrocode}
%
% \section{Installation}
%
% \subsection{Download}
%
% \paragraph{Package.} This package is available on
% CTAN\footnote{\url{ftp://ftp.ctan.org/tex-archive/}}:
% \begin{description}
% \item[\CTAN{macros/latex/contrib/oberdiek/pdfescape.dtx}] The source file.
% \item[\CTAN{macros/latex/contrib/oberdiek/pdfescape.pdf}] Documentation.
% \end{description}
%
%
% \paragraph{Bundle.} All the packages of the bundle `oberdiek'
% are also available in a TDS compliant ZIP archive. There
% the packages are already unpacked and the documentation files
% are generated. The files and directories obey the TDS standard.
% \begin{description}
% \item[\CTAN{install/macros/latex/contrib/oberdiek.tds.zip}]
% \end{description}
% \emph{TDS} refers to the standard ``A Directory Structure
% for \TeX\ Files'' (\CTAN{tds/tds.pdf}). Directories
% with \xfile{texmf} in their name are usually organized this way.
%
% \subsection{Bundle installation}
%
% \paragraph{Unpacking.} Unpack the \xfile{oberdiek.tds.zip} in the
% TDS tree (also known as \xfile{texmf} tree) of your choice.
% Example (linux):
% \begin{quote}
%   |unzip oberdiek.tds.zip -d ~/texmf|
% \end{quote}
%
% \paragraph{Script installation.}
% Check the directory \xfile{TDS:scripts/oberdiek/} for
% scripts that need further installation steps.
% Package \xpackage{attachfile2} comes with the Perl script
% \xfile{pdfatfi.pl} that should be installed in such a way
% that it can be called as \texttt{pdfatfi}.
% Example (linux):
% \begin{quote}
%   |chmod +x scripts/oberdiek/pdfatfi.pl|\\
%   |cp scripts/oberdiek/pdfatfi.pl /usr/local/bin/|
% \end{quote}
%
% \subsection{Package installation}
%
% \paragraph{Unpacking.} The \xfile{.dtx} file is a self-extracting
% \docstrip\ archive. The files are extracted by running the
% \xfile{.dtx} through \plainTeX:
% \begin{quote}
%   \verb|tex pdfescape.dtx|
% \end{quote}
%
% \paragraph{TDS.} Now the different files must be moved into
% the different directories in your installation TDS tree
% (also known as \xfile{texmf} tree):
% \begin{quote}
% \def\t{^^A
% \begin{tabular}{@{}>{\ttfamily}l@{ $\rightarrow$ }>{\ttfamily}l@{}}
%   pdfescape.sty & tex/generic/oberdiek/pdfescape.sty\\
%   pdfescape.pdf & doc/latex/oberdiek/pdfescape.pdf\\
%   test/pdfescape-test1.tex & doc/latex/oberdiek/test/pdfescape-test1.tex\\
%   test/pdfescape-test2.tex & doc/latex/oberdiek/test/pdfescape-test2.tex\\
%   test/pdfescape-test3.tex & doc/latex/oberdiek/test/pdfescape-test3.tex\\
%   test/pdfescape-test4.tex & doc/latex/oberdiek/test/pdfescape-test4.tex\\
%   test/pdfescape-test5.tex & doc/latex/oberdiek/test/pdfescape-test5.tex\\
%   test/pdfescape-test6.tex & doc/latex/oberdiek/test/pdfescape-test6.tex\\
%   pdfescape.dtx & source/latex/oberdiek/pdfescape.dtx\\
% \end{tabular}^^A
% }^^A
% \sbox0{\t}^^A
% \ifdim\wd0>\linewidth
%   \begingroup
%     \advance\linewidth by\leftmargin
%     \advance\linewidth by\rightmargin
%   \edef\x{\endgroup
%     \def\noexpand\lw{\the\linewidth}^^A
%   }\x
%   \def\lwbox{^^A
%     \leavevmode
%     \hbox to \linewidth{^^A
%       \kern-\leftmargin\relax
%       \hss
%       \usebox0
%       \hss
%       \kern-\rightmargin\relax
%     }^^A
%   }^^A
%   \ifdim\wd0>\lw
%     \sbox0{\small\t}^^A
%     \ifdim\wd0>\linewidth
%       \ifdim\wd0>\lw
%         \sbox0{\footnotesize\t}^^A
%         \ifdim\wd0>\linewidth
%           \ifdim\wd0>\lw
%             \sbox0{\scriptsize\t}^^A
%             \ifdim\wd0>\linewidth
%               \ifdim\wd0>\lw
%                 \sbox0{\tiny\t}^^A
%                 \ifdim\wd0>\linewidth
%                   \lwbox
%                 \else
%                   \usebox0
%                 \fi
%               \else
%                 \lwbox
%               \fi
%             \else
%               \usebox0
%             \fi
%           \else
%             \lwbox
%           \fi
%         \else
%           \usebox0
%         \fi
%       \else
%         \lwbox
%       \fi
%     \else
%       \usebox0
%     \fi
%   \else
%     \lwbox
%   \fi
% \else
%   \usebox0
% \fi
% \end{quote}
% If you have a \xfile{docstrip.cfg} that configures and enables \docstrip's
% TDS installing feature, then some files can already be in the right
% place, see the documentation of \docstrip.
%
% \subsection{Refresh file name databases}
%
% If your \TeX~distribution
% (\teTeX, \mikTeX, \dots) relies on file name databases, you must refresh
% these. For example, \teTeX\ users run \verb|texhash| or
% \verb|mktexlsr|.
%
% \subsection{Some details for the interested}
%
% \paragraph{Attached source.}
%
% The PDF documentation on CTAN also includes the
% \xfile{.dtx} source file. It can be extracted by
% AcrobatReader 6 or higher. Another option is \textsf{pdftk},
% e.g. unpack the file into the current directory:
% \begin{quote}
%   \verb|pdftk pdfescape.pdf unpack_files output .|
% \end{quote}
%
% \paragraph{Unpacking with \LaTeX.}
% The \xfile{.dtx} chooses its action depending on the format:
% \begin{description}
% \item[\plainTeX:] Run \docstrip\ and extract the files.
% \item[\LaTeX:] Generate the documentation.
% \end{description}
% If you insist on using \LaTeX\ for \docstrip\ (really,
% \docstrip\ does not need \LaTeX), then inform the autodetect routine
% about your intention:
% \begin{quote}
%   \verb|latex \let\install=y% \iffalse meta-comment
%
% File: pdfescape.dtx
% Version: 2011/11/25 v1.13
% Info: Implements pdfTeX's escape features
%
% Copyright (C) 2007, 2010, 2011 by
%    Heiko Oberdiek <heiko.oberdiek at googlemail.com>
%
% This work may be distributed and/or modified under the
% conditions of the LaTeX Project Public License, either
% version 1.3c of this license or (at your option) any later
% version. This version of this license is in
%    http://www.latex-project.org/lppl/lppl-1-3c.txt
% and the latest version of this license is in
%    http://www.latex-project.org/lppl.txt
% and version 1.3 or later is part of all distributions of
% LaTeX version 2005/12/01 or later.
%
% This work has the LPPL maintenance status "maintained".
%
% This Current Maintainer of this work is Heiko Oberdiek.
%
% The Base Interpreter refers to any `TeX-Format',
% because some files are installed in TDS:tex/generic//.
%
% This work consists of the main source file pdfescape.dtx
% and the derived files
%    pdfescape.sty, pdfescape.pdf, pdfescape.ins, pdfescape.drv,
%    pdfescape-test1.tex, pdfescape-test2.tex,
%    pdfescape-test3.tex, pdfescape-test4.tex,
%    pdfescape-test5.tex, pdfescape-test6.tex.
%
% Distribution:
%    CTAN:macros/latex/contrib/oberdiek/pdfescape.dtx
%    CTAN:macros/latex/contrib/oberdiek/pdfescape.pdf
%
% Unpacking:
%    (a) If pdfescape.ins is present:
%           tex pdfescape.ins
%    (b) Without pdfescape.ins:
%           tex pdfescape.dtx
%    (c) If you insist on using LaTeX
%           latex \let\install=y% \iffalse meta-comment
%
% File: pdfescape.dtx
% Version: 2011/11/25 v1.13
% Info: Implements pdfTeX's escape features
%
% Copyright (C) 2007, 2010, 2011 by
%    Heiko Oberdiek <heiko.oberdiek at googlemail.com>
%
% This work may be distributed and/or modified under the
% conditions of the LaTeX Project Public License, either
% version 1.3c of this license or (at your option) any later
% version. This version of this license is in
%    http://www.latex-project.org/lppl/lppl-1-3c.txt
% and the latest version of this license is in
%    http://www.latex-project.org/lppl.txt
% and version 1.3 or later is part of all distributions of
% LaTeX version 2005/12/01 or later.
%
% This work has the LPPL maintenance status "maintained".
%
% This Current Maintainer of this work is Heiko Oberdiek.
%
% The Base Interpreter refers to any `TeX-Format',
% because some files are installed in TDS:tex/generic//.
%
% This work consists of the main source file pdfescape.dtx
% and the derived files
%    pdfescape.sty, pdfescape.pdf, pdfescape.ins, pdfescape.drv,
%    pdfescape-test1.tex, pdfescape-test2.tex,
%    pdfescape-test3.tex, pdfescape-test4.tex,
%    pdfescape-test5.tex, pdfescape-test6.tex.
%
% Distribution:
%    CTAN:macros/latex/contrib/oberdiek/pdfescape.dtx
%    CTAN:macros/latex/contrib/oberdiek/pdfescape.pdf
%
% Unpacking:
%    (a) If pdfescape.ins is present:
%           tex pdfescape.ins
%    (b) Without pdfescape.ins:
%           tex pdfescape.dtx
%    (c) If you insist on using LaTeX
%           latex \let\install=y\input{pdfescape.dtx}
%        (quote the arguments according to the demands of your shell)
%
% Documentation:
%    (a) If pdfescape.drv is present:
%           latex pdfescape.drv
%    (b) Without pdfescape.drv:
%           latex pdfescape.dtx; ...
%    The class ltxdoc loads the configuration file ltxdoc.cfg
%    if available. Here you can specify further options, e.g.
%    use A4 as paper format:
%       \PassOptionsToClass{a4paper}{article}
%
%    Programm calls to get the documentation (example):
%       pdflatex pdfescape.dtx
%       makeindex -s gind.ist pdfescape.idx
%       pdflatex pdfescape.dtx
%       makeindex -s gind.ist pdfescape.idx
%       pdflatex pdfescape.dtx
%
% Installation:
%    TDS:tex/generic/oberdiek/pdfescape.sty
%    TDS:doc/latex/oberdiek/pdfescape.pdf
%    TDS:doc/latex/oberdiek/test/pdfescape-test1.tex
%    TDS:doc/latex/oberdiek/test/pdfescape-test2.tex
%    TDS:doc/latex/oberdiek/test/pdfescape-test3.tex
%    TDS:doc/latex/oberdiek/test/pdfescape-test4.tex
%    TDS:doc/latex/oberdiek/test/pdfescape-test5.tex
%    TDS:doc/latex/oberdiek/test/pdfescape-test6.tex
%    TDS:source/latex/oberdiek/pdfescape.dtx
%
%<*ignore>
\begingroup
  \catcode123=1 %
  \catcode125=2 %
  \def\x{LaTeX2e}%
\expandafter\endgroup
\ifcase 0\ifx\install y1\fi\expandafter
         \ifx\csname processbatchFile\endcsname\relax\else1\fi
         \ifx\fmtname\x\else 1\fi\relax
\else\csname fi\endcsname
%</ignore>
%<*install>
\input docstrip.tex
\Msg{************************************************************************}
\Msg{* Installation}
\Msg{* Package: pdfescape 2011/11/25 v1.13 Implements pdfTeX's escape features (HO)}
\Msg{************************************************************************}

\keepsilent
\askforoverwritefalse

\let\MetaPrefix\relax
\preamble

This is a generated file.

Project: pdfescape
Version: 2011/11/25 v1.13

Copyright (C) 2007, 2010, 2011 by
   Heiko Oberdiek <heiko.oberdiek at googlemail.com>

This work may be distributed and/or modified under the
conditions of the LaTeX Project Public License, either
version 1.3c of this license or (at your option) any later
version. This version of this license is in
   http://www.latex-project.org/lppl/lppl-1-3c.txt
and the latest version of this license is in
   http://www.latex-project.org/lppl.txt
and version 1.3 or later is part of all distributions of
LaTeX version 2005/12/01 or later.

This work has the LPPL maintenance status "maintained".

This Current Maintainer of this work is Heiko Oberdiek.

The Base Interpreter refers to any `TeX-Format',
because some files are installed in TDS:tex/generic//.

This work consists of the main source file pdfescape.dtx
and the derived files
   pdfescape.sty, pdfescape.pdf, pdfescape.ins, pdfescape.drv,
   pdfescape-test1.tex, pdfescape-test2.tex,
   pdfescape-test3.tex, pdfescape-test4.tex,
   pdfescape-test5.tex, pdfescape-test6.tex.

\endpreamble
\let\MetaPrefix\DoubleperCent

\generate{%
  \file{pdfescape.ins}{\from{pdfescape.dtx}{install}}%
  \file{pdfescape.drv}{\from{pdfescape.dtx}{driver}}%
  \usedir{tex/generic/oberdiek}%
  \file{pdfescape.sty}{\from{pdfescape.dtx}{package}}%
  \usedir{doc/latex/oberdiek/test}%
  \file{pdfescape-test1.tex}{\from{pdfescape.dtx}{test1}}%
  \file{pdfescape-test2.tex}{\from{pdfescape.dtx}{test2}}%
  \file{pdfescape-test3.tex}{\from{pdfescape.dtx}{test3}}%
  \file{pdfescape-test4.tex}{\from{pdfescape.dtx}{test4}}%
  \file{pdfescape-test5.tex}{\from{pdfescape.dtx}{test5}}%
  \file{pdfescape-test6.tex}{\from{pdfescape.dtx}{test6}}%
  \nopreamble
  \nopostamble
  \usedir{source/latex/oberdiek/catalogue}%
  \file{pdfescape.xml}{\from{pdfescape.dtx}{catalogue}}%
}

\catcode32=13\relax% active space
\let =\space%
\Msg{************************************************************************}
\Msg{*}
\Msg{* To finish the installation you have to move the following}
\Msg{* file into a directory searched by TeX:}
\Msg{*}
\Msg{*     pdfescape.sty}
\Msg{*}
\Msg{* To produce the documentation run the file `pdfescape.drv'}
\Msg{* through LaTeX.}
\Msg{*}
\Msg{* Happy TeXing!}
\Msg{*}
\Msg{************************************************************************}

\endbatchfile
%</install>
%<*ignore>
\fi
%</ignore>
%<*driver>
\NeedsTeXFormat{LaTeX2e}
\ProvidesFile{pdfescape.drv}%
  [2011/11/25 v1.13 Implements pdfTeX's escape features (HO)]%
\documentclass{ltxdoc}
\usepackage{holtxdoc}[2011/11/22]
\begin{document}
  \DocInput{pdfescape.dtx}%
\end{document}
%</driver>
% \fi
%
% \CheckSum{2516}
%
% \CharacterTable
%  {Upper-case    \A\B\C\D\E\F\G\H\I\J\K\L\M\N\O\P\Q\R\S\T\U\V\W\X\Y\Z
%   Lower-case    \a\b\c\d\e\f\g\h\i\j\k\l\m\n\o\p\q\r\s\t\u\v\w\x\y\z
%   Digits        \0\1\2\3\4\5\6\7\8\9
%   Exclamation   \!     Double quote  \"     Hash (number) \#
%   Dollar        \$     Percent       \%     Ampersand     \&
%   Acute accent  \'     Left paren    \(     Right paren   \)
%   Asterisk      \*     Plus          \+     Comma         \,
%   Minus         \-     Point         \.     Solidus       \/
%   Colon         \:     Semicolon     \;     Less than     \<
%   Equals        \=     Greater than  \>     Question mark \?
%   Commercial at \@     Left bracket  \[     Backslash     \\
%   Right bracket \]     Circumflex    \^     Underscore    \_
%   Grave accent  \`     Left brace    \{     Vertical bar  \|
%   Right brace   \}     Tilde         \~}
%
% \GetFileInfo{pdfescape.drv}
%
% \title{The \xpackage{pdfescape} package}
% \date{2011/11/25 v1.13}
% \author{Heiko Oberdiek\\\xemail{heiko.oberdiek at googlemail.com}}
%
% \maketitle
%
% \begin{abstract}
% This package implements \pdfTeX's escape features
% (\cs{pdfescapehex}, \cs{pdfunescapehex}, \cs{pdfescapename},
% \cs{pdfescapestring}) using \TeX\ or \eTeX.
% \end{abstract}
%
% \tableofcontents
%
% \section{Documentation}
%
% \begin{declcs}{EdefEscapeHex} \M{cmd} \M{string}\\
%  \cs{EdefUnescapeHex} \M{cmd} \M{string}\\
%  \cs{EdefEscapeName} \M{cmd} \M{string}\\
%  \cs{EdefEscapeString} \M{cmd} \M{string}
% \end{declcs}
% These commands converts \meta{string} and stores the result
% in macro \meta{cmd}. The conversion result is the same
% as the conversion of the corresponding \pdfTeX's primitives.
% Note that the argument \meta{string} is expanded before
% the conversion.
%
% For example, if \pdfTeX\ \textgreater= 1.30 is present,
% then \cs{EdefEscapeHex}
% becomes to:
% \begin{quote}
%\begin{verbatim}
%\def\EdefEscapeHex#1#2{%
%  \edef#1{\pdfescapehex{#2}}%
%}
%\end{verbatim}
% \end{quote}
% The package provides implementations for the case that \pdfTeX\ is
% not present (or too old). Even \eTeX\ can be missing, however it is
% used if it is detected.
%
% \paragraph{Babel.}
% The input strings may contain shorthand characters of package
% \xpackage{babel}.
%
% \subsection{Additional unescape macros}
%
% \begin{declcs}{EdefUnescapeName} \M{cmd} \M{string}
% \end{declcs}
% Sequences of a hash sign with two hexadecimal digits are
% converted to the corresponding character (PDF-1.2).
% A hash sign that is not followed by two hexadecimal digits
% is left unchanged. The catcodes in the result string follow
% TeX's conventions. The space has catcode 10 (space) and
% the other characters have catcode 12 (other).
%
% \begin{declcs}{EdefUnescapeString} \M{cmd} \M{string}
% \end{declcs}
% Macro \meta{cmd} stores the unescaped string in \meta{string}.
% All the rules for literal strings are implemented, see
% PDF specification.
% The catcodes in the result string follow TeX's conventions.
%
% \subsection{Sanitizing macro}
%
% \begin{declcs}{EdefSanitize} \M{cmd} \M{string}
% \end{declcs}
% Argument \meta{string} is expanded, converted to a string
% of tokens with catcode 12 (other) and space tokens, and
% stored in macro \meta{cmd}.
%
% \StopEventually{
% }
%
% \section{Implementation}
%
%    \begin{macrocode}
%<*package>
%    \end{macrocode}
%
% \subsection{Reload check and package identification}
%    Reload check, especially if the package is not used with \LaTeX.
%    \begin{macrocode}
\begingroup\catcode61\catcode48\catcode32=10\relax%
  \catcode13=5 % ^^M
  \endlinechar=13 %
  \catcode35=6 % #
  \catcode39=12 % '
  \catcode44=12 % ,
  \catcode45=12 % -
  \catcode46=12 % .
  \catcode58=12 % :
  \catcode64=11 % @
  \catcode123=1 % {
  \catcode125=2 % }
  \expandafter\let\expandafter\x\csname ver@pdfescape.sty\endcsname
  \ifx\x\relax % plain-TeX, first loading
  \else
    \def\empty{}%
    \ifx\x\empty % LaTeX, first loading,
      % variable is initialized, but \ProvidesPackage not yet seen
    \else
      \expandafter\ifx\csname PackageInfo\endcsname\relax
        \def\x#1#2{%
          \immediate\write-1{Package #1 Info: #2.}%
        }%
      \else
        \def\x#1#2{\PackageInfo{#1}{#2, stopped}}%
      \fi
      \x{pdfescape}{The package is already loaded}%
      \aftergroup\endinput
    \fi
  \fi
\endgroup%
%    \end{macrocode}
%    Package identification:
%    \begin{macrocode}
\begingroup\catcode61\catcode48\catcode32=10\relax%
  \catcode13=5 % ^^M
  \endlinechar=13 %
  \catcode35=6 % #
  \catcode39=12 % '
  \catcode40=12 % (
  \catcode41=12 % )
  \catcode44=12 % ,
  \catcode45=12 % -
  \catcode46=12 % .
  \catcode47=12 % /
  \catcode58=12 % :
  \catcode64=11 % @
  \catcode91=12 % [
  \catcode93=12 % ]
  \catcode123=1 % {
  \catcode125=2 % }
  \expandafter\ifx\csname ProvidesPackage\endcsname\relax
    \def\x#1#2#3[#4]{\endgroup
      \immediate\write-1{Package: #3 #4}%
      \xdef#1{#4}%
    }%
  \else
    \def\x#1#2[#3]{\endgroup
      #2[{#3}]%
      \ifx#1\@undefined
        \xdef#1{#3}%
      \fi
      \ifx#1\relax
        \xdef#1{#3}%
      \fi
    }%
  \fi
\expandafter\x\csname ver@pdfescape.sty\endcsname
\ProvidesPackage{pdfescape}%
  [2011/11/25 v1.13 Implements pdfTeX's escape features (HO)]%
%    \end{macrocode}
%
% \subsection{Catcodes}
%
%    \begin{macrocode}
\begingroup\catcode61\catcode48\catcode32=10\relax%
  \catcode13=5 % ^^M
  \endlinechar=13 %
  \catcode123 1 % {
  \catcode125 2 % }
  \catcode64 11 %
  \def\x{\endgroup
    \expandafter\edef\csname PE@AtEnd\endcsname{%
      \endlinechar=\the\endlinechar\relax
      \catcode13=\the\catcode13\relax
      \catcode32=\the\catcode32\relax
      \catcode35=\the\catcode35\relax
      \catcode61=\the\catcode61\relax
      \catcode64=\the\catcode64\relax
      \catcode123=\the\catcode123\relax
      \catcode125=\the\catcode125\relax
    }%
  }%
\x\catcode61\catcode48\catcode32=10\relax%
\catcode13=5 % ^^M
\endlinechar=13 %
\catcode35=6 % #
\catcode64=11 % @
\catcode123=1 % {
\catcode125=2 % }
\def\TMP@EnsureCode#1#2#3{%
  \edef\PE@AtEnd{%
    \PE@AtEnd
    #1#2=\the#1#2\relax
  }%
  #1#2=#3\relax
}
\TMP@EnsureCode\catcode{0}{12}% ^^@
\TMP@EnsureCode\catcode{34}{12}% "
\TMP@EnsureCode\catcode{36}{3}% $
\TMP@EnsureCode\catcode{38}{4}% &
\TMP@EnsureCode\catcode{39}{12}% '
\TMP@EnsureCode\catcode{42}{12}% *
\TMP@EnsureCode\catcode{45}{12}% -
\TMP@EnsureCode\catcode{46}{12}% .
\TMP@EnsureCode\catcode{47}{12}% /
\TMP@EnsureCode\catcode{60}{12}% <
\TMP@EnsureCode\catcode{62}{12}% >
\TMP@EnsureCode\catcode{91}{12}% [
\TMP@EnsureCode\catcode{93}{12}% ]
\TMP@EnsureCode\catcode{94}{7}% ^
\TMP@EnsureCode\catcode{96}{12}% `
\TMP@EnsureCode\uccode{34}{0}% "
\TMP@EnsureCode\uccode{48}{0}% 0
\TMP@EnsureCode\uccode{61}{0}% =
\edef\PE@AtEnd{\PE@AtEnd\noexpand\endinput}
%    \end{macrocode}
%
% \subsection{Load package}
%
%    \begin{macrocode}
\begingroup\expandafter\expandafter\expandafter\endgroup
\expandafter\ifx\csname RequirePackage\endcsname\relax
  \def\TMP@RequirePackage#1[#2]{%
    \begingroup\expandafter\expandafter\expandafter\endgroup
    \expandafter\ifx\csname ver@#1.sty\endcsname\relax
      \input #1.sty\relax
    \fi
  }%
  \TMP@RequirePackage{ltxcmds}[2010/04/08]%
\else
  \RequirePackage{ltxcmds}[2010/04/08]%
\fi
%    \end{macrocode}
%
% \subsection{Sanitizing}
%
%    \begin{macro}{\EdefSanitize}
%    Macro \cs{EdefSanitize} takes |#2|, entirely converts
%    it to token with catcode 12 (other) and stores the
%    result in macro |#1|.
%    \begin{macrocode}
\begingroup\expandafter\expandafter\expandafter\endgroup
\expandafter\ifx\csname detokenize\endcsname\relax
  \long\def\EdefSanitize#1#2{%
    \begingroup
      \csname @safe@activestrue\endcsname
      \edef#1{#2}%
      \PE@onelevel@sanitize#1%
    \expandafter\endgroup
    \expandafter\def\expandafter#1\expandafter{#1}%
  }%
  \begingroup\expandafter\expandafter\expandafter\endgroup
  \expandafter\ifx\csname @onelevel@sanitize\endcsname\relax
    \def\PE@onelevel@sanitize#1{%
      \edef#1{\expandafter\PE@strip@prefix\meaning#1}%
    }%
    \def\PE@strip@prefix#1>{}%
  \else
    \let\PE@onelevel@sanitize\@onelevel@sanitize
  \fi
\else
  \long\def\EdefSanitize#1#2{%
    \begingroup
      \csname @safe@activestrue\endcsname
      \edef#1{#2}%
    \expandafter\endgroup
    \expandafter\def\expandafter#1\expandafter{%
      \detokenize\expandafter{#1}%
    }%
  }%
  \def\PE@onelevel@sanitize#1{%
    \edef#1{\detokenize\expandafter{#1}}%
  }%
\fi
%    \end{macrocode}
%    \end{macro}
%    \begin{macro}{\PE@sanitize}
%    Macro \cs{PE@sanitize} is only defined for
%    compatibility with version 1.4. Its use is deprecated.
%    \begin{macrocode}
\let\PE@sanitize\EdefSanitize
%    \end{macrocode}
%    \end{macro}
%
% \subsubsection{Space characters}
%
%    \begin{macro}{\PE@space@other}
%    \begin{macrocode}
\begingroup
  \catcode`\ =12\relax%
\def\x{\endgroup\def\PE@space@other{ }}\x\relax
%    \end{macrocode}
%    \end{macro}
%    \begin{macro}{\PE@space@space}
%    \begin{macrocode}
\def\PE@space@space{ }
%    \end{macrocode}
%    \end{macro}
%
% \subsubsection{Space normalization}
%
%    \begin{macro}{\PE@SanitizeSpaceOther}
%    \begin{macrocode}
\def\PE@SanitizeSpaceOther#1{%
  \edef#1{\expandafter\PE@SpaceToOther#1 \relax}%
}
%    \end{macrocode}
%    \end{macro}
%
%    \begin{macro}{\PE@SpaceToOther}
%    \begin{macrocode}
\def\PE@SpaceToOther#1 #2\relax{%
  #1%
  \ifx\\#2\\%
  \else
    \PE@space@other
    \ltx@ReturnAfterFi{%
      \PE@SpaceToOther#2\relax
    }%
  \fi
}
%    \end{macrocode}
%    \end{macro}
%
% \subsection{\cs{EdefUnescapeName}}
%
%    \begin{macro}{\EdefUnescapeName}
%    \begin{macrocode}
\def\EdefUnescapeName#1#2{%
  \EdefSanitize#1{#2}%
  \PE@SanitizeSpaceOther#1%
  \PE@UnescapeName#1%
  \PE@onelevel@sanitize#1%
}
%    \end{macrocode}
%    \end{macro}
%
%    \begin{macro}{\PE@UnescapeName}
%    \begin{macrocode}
\begingroup
  \catcode`\$=6  % hash
  \catcode`\#=12 % other
  \gdef\PE@UnescapeName$1{%
    \begingroup
      \PE@InitUccodeHexDigit
      \def\PE@result{}%
      \expandafter\PE@DeName$1#\relax\relax
    \expandafter\endgroup
    \expandafter\def\expandafter$1\expandafter{\PE@result}%
  }%
  \gdef\PE@DeName$1#$2$3{%
    \ifx\relax$2%
      \edef\PE@result{\PE@result$1}%
      \let\PE@next\relax
    \else
      \ifx\relax$3%
        % wrong escape sequence in input
        \edef\PE@result{\PE@result$1#}%
        \let\PE@next\relax
      \else
        \uppercase{%
          \def\PE@testA{$2}%
          \def\PE@testB{$3}%
        }%
        \ifcase\ifcase\expandafter\PE@TestUcHexDigit\PE@testA
                 \ifcase\expandafter\PE@TestUcHexDigit\PE@testB
                   \ltx@zero
                 \else
                   \ltx@one
                 \fi
               \else
                 \ltx@one
               \fi
          \uccode\ltx@zero="\PE@testA\PE@testB\relax
          \uppercase{%
            \def\PE@temp{^^@}%
          }%
          \uccode\ltx@zero=\ltx@zero
          \edef\PE@result{\PE@result$1\PE@temp}%
          \let\PE@next\PE@DeName
        \else
          % wrong escape sequence in input
          \edef\PE@result{\PE@result$1#}%
          \def\PE@next{\PE@DeName$2$3}%
        \fi
      \fi
    \fi
    \PE@next
  }%
\endgroup
%    \end{macrocode}
%    \end{macro}
%
%    \begin{macro}{\PE@InitUccodeHexDigit}
%    \begin{macrocode}
\def\PE@InitUccodeHexDigit{%
  \uccode`a=`A\relax
  \uccode`b=`B\relax
  \uccode`c=`C\relax
  \uccode`d=`D\relax
  \uccode`e=`E\relax
  \uccode`f=`F\relax
  \uccode`A=\ltx@zero
  \uccode`B=\ltx@zero
  \uccode`C=\ltx@zero
  \uccode`D=\ltx@zero
  \uccode`E=\ltx@zero
  \uccode`F=\ltx@zero
  \uccode`0=\ltx@zero
  \uccode`1=\ltx@zero
  \uccode`2=\ltx@zero
  \uccode`3=\ltx@zero
  \uccode`4=\ltx@zero
  \uccode`5=\ltx@zero
  \uccode`6=\ltx@zero
  \uccode`7=\ltx@zero
  \uccode`8=\ltx@zero
  \uccode`9=\ltx@zero
}
%    \end{macrocode}
%    \end{macro}
%
%    \begin{macro}{\PE@TestUcHexDigit}
%    \begin{macrocode}
\def\PE@TestUcHexDigit#1{%
  \ifnum`#1<48 % 0
    \ltx@one
  \else
    \ifnum`#1>70 % F
      \ltx@one
    \else
      \ifnum`#1>57 % 9
        \ifnum`#1<65 % A
          \ltx@one
        \else
          \ltx@zero
        \fi
      \else
        \ltx@zero
      \fi
    \fi
  \fi
}
%    \end{macrocode}
%    \end{macro}
%
% \subsection{\cs{EdefUnescapeString}}
%
%    \begin{macro}{\EdefUnescapeString}
%    \begin{macrocode}
\def\EdefUnescapeString#1#2{%
  \EdefSanitize#1{#2}%
  \PE@SanitizeSpaceOther#1%
  \PE@NormalizeLineEnd#1%
  \PE@UnescapeString#1%
  \PE@onelevel@sanitize#1%
}
%    \end{macrocode}
%    \end{macro}
%
%    \begin{macrocode}
\begingroup
  \uccode`\8=10 % lf
  \uccode`\9=13 % cr
\def\x#1#2{\endgroup
%    \end{macrocode}
%    \begin{macro}{\PE@NormalizeLineEnd}
%    \begin{macrocode}
  \def\PE@NormalizeLineEnd##1{%
    \def\PE@result{}%
    \expandafter\PE@@NormalizeLineEnd##1#2\relax
    \let##1\PE@result
  }%
%    \end{macrocode}
%    \end{macro}
%    \begin{macro}{\PE@@NormalizeLineEnd}
%    \begin{macrocode}
  \def\PE@@NormalizeLineEnd##1#2##2{%
    \ifx\relax##2%
      \edef\PE@result{\PE@result##1}%
      \let\PE@next\relax
    \else
      \edef\PE@result{\PE@result##1#1}%
      \ifx#1##2% lf
        \let\PE@next\PE@@NormalizeLineEnd
      \else
        \def\PE@next{\PE@@NormalizeLineEnd##2}%
      \fi
    \fi
    \PE@next
  }%
}%
\uppercase{%
  \x 89%
}
%    \end{macrocode}
%    \end{macro}
%
%    \begin{macrocode}
\begingroup
  \catcode`\|=0 %
  \catcode`\\=12 %
%    \end{macrocode}
%    \begin{macro}{\PE@UnescapeString}
%    \begin{macrocode}
  |gdef|PE@UnescapeString#1{%
    |begingroup
      |def|PE@result{}%
      |expandafter|PE@DeString#1\|relax
    |expandafter|endgroup
    |expandafter|def|expandafter#1|expandafter{|PE@result}%
  }%
%    \end{macrocode}
%    \end{macro}
%    \begin{macro}{\PE@DeString}
%    \begin{macrocode}
  |gdef|PE@DeString#1\#2{%
    |ifx|relax#2%
      |edef|PE@result{|PE@result#1}%
      |let|PE@next|relax
    |else
      |if n#2%
        |uccode|ltx@zero=10 %
      |else|if r#2%
        |uccode|ltx@zero=13 %
      |else|if t#2%
        |uccode|ltx@zero=9 %
      |else|if b#2%
        |uccode|ltx@zero=8 %
      |else|if f#2%
        |uccode|ltx@zero=12 %
      |else
        |uccode|ltx@zero=|ltx@zero
      |fi|fi|fi|fi|fi
      |ifnum|uccode|ltx@zero>|ltx@zero
        |uppercase{%
          |edef|PE@temp{^^@}%
        }%
        |edef|PE@result{|PE@result#1|PE@temp}%
        |let|PE@next|PE@DeString
      |else
        |if\#2% backslash
          |edef|PE@result{|PE@result#1}%
          |let|PE@next|PE@CheckEndBackslash
        |else
          |ifnum`#2=10 % linefeed
            |edef|PE@result{|PE@result#1}%
            |let|PE@next|PE@DeString
          |else
            |ifcase|PE@TestOctDigit#2%
              |edef|PE@result{|PE@result#1}%
              |def|PE@next{|PE@OctI#2}%
            |else
              |edef|PE@result{|PE@result#1#2}%
              |let|PE@next|PE@DeString
            |fi
          |fi
        |fi
      |fi
    |fi
    |PE@next
  }%
%    \end{macrocode}
%    \end{macro}
%    \begin{macro}{\PE@CheckEndBackslash}
%    \begin{macrocode}
  |gdef|PE@CheckEndBackslash#1{%
    |ifx|relax#1%
    |else
      |edef|PE@result{|PE@result\}%
      |expandafter|PE@DeString|expandafter#1%
    |fi
  }%
%    \end{macrocode}
%    \end{macro}
%    \begin{macrocode}
|endgroup
%    \end{macrocode}
%    \begin{macro}{\PE@TestOctDigit}
%    \begin{macrocode}
\def\PE@TestOctDigit#1{%
  \ifnum`#1<48 % 0
    \ltx@one
  \else
    \ifnum`#1>55 % 7
      \ltx@one
    \else
      \ltx@zero
    \fi
  \fi
}
%    \end{macrocode}
%    \end{macro}
%    \begin{macro}{\PE@OctI}
%    \begin{macrocode}
\def\PE@OctI#1#2{%
  \ifcase\PE@TestOctDigit#2%
    \def\PE@next{\PE@OctII{#1#2}}%
  \else
    \def\PE@next{\PE@OctAll#1#2}%
  \fi
  \PE@next
}
%    \end{macrocode}
%    \end{macro}
%    \begin{macro}{\PE@OctII}
%    \begin{macrocode}
\def\PE@OctII#1#2{%
  \ifcase\PE@TestOctDigit#2%
    \def\PE@next{\PE@OctIII#1#2}%
  \else
    \def\PE@next{\PE@OctAll{#1}#2}%
  \fi
  \PE@next
}
%    \end{macrocode}
%    \end{macro}
%    \begin{macrocode}
\ltx@IfUndefined{numexpr}{%
  \catcode`\$=9 %
  \catcode`\&=14 %
}{%
  \catcode`\$=14 %
  \catcode`\&=9 %
}
%    \end{macrocode}
%    \begin{macro}{\PE@OctIII}
%    \begin{macrocode}
\def\PE@OctIII#1#2#3{%
  \ifnum#1<4 %
    \def\PE@next{\PE@OctAll{#1#2#3}}%
  \else
$   \count\ltx@cclv#1 %
$   \advance\count\ltx@cclv -4 %
    \edef\PE@next{%
      \noexpand\PE@OctAll{%
$       \the\count\ltx@cclv
&       \the\numexpr#1-4\relax
        #2#3%
      }%
    }%
  \fi
  \PE@next
}
%    \end{macrocode}
%    \end{macro}
%    \begin{macro}{\PE@OctAll}
%    \begin{macrocode}
\def\PE@OctAll#1{%
  \uccode\ltx@zero='#1\relax
  \uppercase{%
    \edef\PE@result{\PE@result^^@}%
  }%
  \PE@DeString
}
%    \end{macrocode}
%    \end{macro}
%
% \subsection{User macros (\pdfTeX\space analogues)}
%
%    \begin{macrocode}
\begingroup\expandafter\expandafter\expandafter\endgroup
\expandafter\ifx\csname RequirePackage\endcsname\relax
  \def\TMP@RequirePackage#1[#2]{%
    \begingroup\expandafter\expandafter\expandafter\endgroup
    \expandafter\ifx\csname ver@#1.sty\endcsname\relax
      \input #1.sty\relax
    \fi
  }%
  \TMP@RequirePackage{pdftexcmds}[2007/11/11]%
\else
  \RequirePackage{pdftexcmds}[2007/11/11]%
\fi
%    \end{macrocode}
%
%    \begin{macrocode}
\begingroup\expandafter\expandafter\expandafter\endgroup
\expandafter\ifx\csname pdf@escapehex\endcsname\relax
%    \end{macrocode}
%
%    \begin{macro}{\EdefEscapeHex}
%    \begin{macrocode}
  \long\def\EdefEscapeHex#1#2{%
    \EdefSanitize#1{#2}%
    \PE@SanitizeSpaceOther#1%
    \PE@EscapeHex#1%
  }%
%    \end{macrocode}
%    \end{macro}
%    \begin{macro}{\EdefUnescapeHex}
%    \begin{macrocode}
  \def\EdefUnescapeHex#1#2{%
    \EdefSanitize#1{#2}%
    \PE@UnescapeHex#1%
  }%
%    \end{macrocode}
%    \end{macro}
%    \begin{macro}{\EdefEscapeName}
%    \begin{macrocode}
  \long\def\EdefEscapeName#1#2{%
    \EdefSanitize#1{#2}%
    \PE@SanitizeSpaceOther#1%
    \PE@EscapeName#1%
  }%
%    \end{macrocode}
%    \end{macro}
%    \begin{macro}{\EdefEscapeString}
%    \begin{macrocode}
  \long\def\EdefEscapeString#1#2{%
    \EdefSanitize#1{#2}%
    \PE@SanitizeSpaceOther#1%
    \PE@EscapeString#1%
  }%
%    \end{macrocode}
%    \end{macro}
%
%    \begin{macrocode}
\else
%    \end{macrocode}
%
%    \begin{macro}{\PE@edefbabel}
%    Help macro that adds support for \xpackage{babel}'s shorthand
%    characters.
%    \begin{macrocode}
  \long\def\PE@edefbabel#1#2#3{%
    \begingroup
      \csname @save@activestrue\endcsname
      \edef#1{#2{#3}}%
    \expandafter\endgroup
    \expandafter\def\expandafter#1\expandafter{#1}%
  }%
%    \end{macrocode}
%    \end{macro}
%
%    \begin{macro}{\EdefEscapeHex}
%    \begin{macrocode}
  \long\def\EdefEscapeHex#1#2{%
    \PE@edefbabel#1\pdf@escapehex{#2}%
  }%
%    \end{macrocode}
%    \end{macro}
%    \begin{macro}{\EdefUnescapeHex}
%    \begin{macrocode}
  \def\EdefUnescapeHex#1#2{%
    \PE@edefbabel#1\pdf@unescapehex{#2}%
  }%
%    \end{macrocode}
%    \end{macro}
%    \begin{macro}{\EdefEscapeName}
%    \begin{macrocode}
  \long\def\EdefEscapeName#1#2{%
    \PE@edefbabel#1\pdf@escapename{#2}%
  }%
%    \end{macrocode}
%    \end{macro}
%    \begin{macro}{\EdefEscapeString}
%    \begin{macrocode}
  \long\def\EdefEscapeString#1#2{%
    \PE@edefbabel#1\pdf@escapestring{#2}%
  }%
%    \end{macrocode}
%    \end{macro}
%
%    \begin{macrocode}
  \expandafter\PE@AtEnd
\fi%
%    \end{macrocode}
%
% \subsection{Help macros}
%
% \subsubsection{Characters}
%
%    Special characters with catcode 12 (other) are created and
%    stored in macros.
%    \begin{macro}{\PE@hash}
%    \begin{macrocode}
\edef\PE@hash{\string#}
%    \end{macrocode}
%    \end{macro}
%    \begin{macro}{\PE@backslash}
%    \begin{macrocode}
\begingroup
  \escapechar=-1 %
\edef\x{\endgroup
  \def\noexpand\PE@backslash{\string\\}%
}
\x
%    \end{macrocode}
%    \end{macro}
%
% \subsubsection{Switch for \eTeX}
%
%    \begin{macrocode}
\ltx@newif\ifPE@etex
\begingroup\expandafter\expandafter\expandafter\endgroup
\expandafter\ifx\csname numexpr\endcsname\relax
\else
  \PE@etextrue
\fi
%    \end{macrocode}
%
% \subsection{Conversions}
%
% \subsubsection{Conversion to hex string}
%
%    \begin{macro}{\PE@EscapeHex}
%    \begin{macrocode}
\ifPE@etex
  \def\PE@EscapeHex#1{%
    \edef#1{\expandafter\PE@ToHex#1\relax}%
  }%
\else
  \def\PE@EscapeHex#1{%
    \def\PE@result{}%
    \expandafter\PE@ToHex#1\relax
    \let#1\PE@result
  }%
\fi
%    \end{macrocode}
%    \end{macro}
%    \begin{macro}{\PE@ToHex}
%    \begin{macrocode}
\def\PE@ToHex#1{%
  \ifx\relax#1%
  \else
    \PE@HexChar{#1}%
    \expandafter\PE@ToHex
  \fi
}%
%    \end{macrocode}
%    \end{macro}
%    \begin{macro}{\PE@HexChar}
%    \begin{macrocode}
\ifPE@etex
  \def\PE@HexChar#1{%
    \PE@HexDigit{\numexpr\dimexpr.0625\dimexpr`#1sp\relax\relax\relax}%
    \PE@HexDigit{%
      \numexpr`#1-16*\dimexpr.0625\dimexpr`#1sp\relax\relax\relax
    }%
  }%
\else
  \def\PE@HexChar#1{%
    \dimen0=`#1sp%
    \dimen2=.0625\dimen0 %
    \advance\dimen0-16\dimen2 %
    \edef\PE@result{%
      \PE@result
      \PE@HexDigit{\dimen2 }%
      \PE@HexDigit{\dimen0 }%
    }%
  }%
\fi
%    \end{macrocode}
%    \end{macro}
%    \begin{macro}{\PE@HexDigit}
%    \begin{macrocode}
\def\PE@HexDigit#1{%
  \expandafter\string
  \ifcase#1%
    0\or 1\or 2\or 3\or 4\or 5\or 6\or 7\or 8\or 9\or
    A\or B\or C\or D\or E\or F%
  \fi
}
%    \end{macrocode}
%    \end{macro}
%
% \subsubsection{Character code to octal number}
%
%    \begin{macro}{\PE@OctChar}
%    \begin{macrocode}
\ifPE@etex
  \def\PE@OctChar#1{%
    \expandafter\PE@@OctChar
        \the\numexpr\dimexpr.015625\dimexpr`#1sp\relax\relax
            \expandafter\relax
        \expandafter\relax
        \the\numexpr\dimexpr.125\dimexpr`#1sp\relax\relax\relax
        \relax
        #1%
  }%
  \def\PE@@OctChar#1\relax#2\relax#3{%
    \PE@backslash
    #1%
    \the\numexpr#2-8*#1\relax
    \the\numexpr\dimexpr`#3sp\relax-8*#2\relax
  }%
\else
  \def\PE@OctChar#1{%
    \dimen0=`#1sp%
    \dimen2=.125\dimen0 %
    \dimen4=.125\dimen2 %
    \advance\dimen0-8\dimen2 %
    \advance\dimen2-8\dimen4 %
    \edef\PE@result{%
      \PE@result
      \PE@backslash
      \number\dimen4 %
      \number\dimen2 %
      \number\dimen0 %
    }%
  }%
\fi
%    \end{macrocode}
%    \end{macro}
%
% \subsubsection{Unpack hex string}
%
%    \begin{macro}{\PE@UnescapeHex}
%    \begin{macrocode}
\def\PE@UnescapeHex#1{%
  \begingroup
    \PE@InitUccodeHexDigit
    \def\PE@result{}%
    \expandafter\PE@DeHex#1\relax\relax
  \expandafter\endgroup
  \expandafter\def\expandafter#1\expandafter{\PE@result}%
}
%    \end{macrocode}
%    \end{macro}
%    \begin{macro}{\PE@DeHex}
%    \begin{macrocode}
\def\PE@DeHex#1#2{%
  \ifx#2\relax
    \ifx#1\relax
      \let\PE@next\relax
    \else
      \uppercase{%
        \def\PE@testA{#1}%
      }%
      \ifcase\expandafter\PE@TestUcHexDigit\PE@testA
        \def\PE@next{%
          \PE@DeHex#10\relax\relax
        }%
      \else
        \let\PE@next\relax
      \fi
    \fi
  \else
    \uppercase{%
      \def\PE@testA{#1}%
      \def\PE@testB{#2}%
    }%
    \ifcase\expandafter\PE@TestUcHexDigit\PE@testA
      \ifcase\expandafter\PE@TestUcHexDigit\PE@testB
        \uccode\ltx@zero="\PE@testA\PE@testB\relax
        \ifnum\uccode\ltx@zero=32 %
          \let\PE@temp\PE@space@space
        \else
          \uppercase{%
            \def\PE@temp{^^@}%
          }%
        \fi
        \edef\PE@result{\PE@result\PE@temp}%
        \let\PE@next\PE@DeHex
      \else
        % invalid input sequence
        \def\PE@next{%
          \PE@DeHex#1%
        }%
      \fi
    \else
      % invalid input sequence
      \def\PE@next{\PE@DeHex#2}%
    \fi
  \fi
  \PE@next
}
%    \end{macrocode}
%    \end{macro}
%
% \subsubsection{Conversion to PDF name}
%
%    \begin{macro}{\PE@EscapeName}
%    \begin{macrocode}
\ifPE@etex
  \def\PE@EscapeName#1{%
    \edef#1{\expandafter\PE@EscapeNameTokens#1\relax}%
  }%
\else
  \def\PE@EscapeName#1{%
    \def\PE@result{}%
    \expandafter\PE@EscapeNameTokens#1\relax
    \let#1\PE@result
  }%
\fi
%    \end{macrocode}
%    \end{macro}
%    \begin{macro}{\PE@EscapeNameTokens}
%    \begin{macrocode}
\def\PE@EscapeNameTokens#1{%
  \ifx\relax#1%
  \else
    \ifnum`#1<33 %
      \ifcase`#1 %
        % drop illegal zero
      \else
        \PE@EscapeNameAdd\PE@hash
        \PE@HexChar#1%
      \fi
    \else
      \ifnum`#1>126 %
        \PE@EscapeNameAdd\PE@hash
        \PE@HexChar#1%
      \else  \ifnum`#1=35  \PE@EscapeNameHashChar 23% #
        \else\ifnum`#1=37  \PE@EscapeNameHashChar 25% %
        \else\ifnum`#1=40  \PE@EscapeNameHashChar 28% (
        \else\ifnum`#1=41  \PE@EscapeNameHashChar 29% )
        \else\ifnum`#1=47  \PE@EscapeNameHashChar 2F% /
        \else\ifnum`#1=60  \PE@EscapeNameHashChar 3C% <
        \else\ifnum`#1=62  \PE@EscapeNameHashChar 3E% >
        \else\ifnum`#1=91  \PE@EscapeNameHashChar 5B% [
        \else\ifnum`#1=93  \PE@EscapeNameHashChar 5D% ]
        \else\ifnum`#1=123 \PE@EscapeNameHashChar 7B% {
        \else\ifnum`#1=125 \PE@EscapeNameHashChar 7D% }
        \else
          \PE@EscapeNameAdd{#1}%
        \fi\fi\fi\fi\fi\fi\fi\fi\fi\fi\fi
      \fi
    \fi
    \expandafter\PE@EscapeNameTokens
  \fi
}%
\def\PE@EscapeNameHashChar#1#2{%
  \PE@EscapeNameAdd{\PE@hash\string#1\string#2}%
}%
%    \end{macrocode}
%    \end{macro}
%    \begin{macro}{\PE@EscapeNameAdd}
%    \begin{macrocode}
\ifPE@etex
  \def\PE@EscapeNameAdd#1{#1}%
\else
  \def\PE@EscapeNameAdd#1{%
    \edef\PE@result{%
      \PE@result
      #1%
    }%
  }%
\fi
%    \end{macrocode}
%    \end{macro}
%
% \subsubsection{Conversion to PDF string}
%
%    \begin{macro}{\PE@EscapeString}
%    \begin{macrocode}
\ifPE@etex
  \def\PE@EscapeString#1{%
    \edef#1{\expandafter\PE@EscapeStringTokens#1\relax}%
  }%
\else
  \def\PE@EscapeString#1{%
    \begingroup
      \def\PE@result{}%
      \expandafter\PE@EscapeStringTokens#1\relax
    \expandafter\endgroup
    \expandafter\def\expandafter#1\expandafter{\PE@result}%
  }%
\fi
%    \end{macrocode}
%    \end{macro}
%    \begin{macro}{\PE@EscapeStringTokens}
%    \begin{macrocode}
\def\PE@EscapeStringTokens#1{%
  \ifx\relax#1%
  \else
    \ifnum`#1<33 %
      \PE@OctChar#1%
    \else
      \ifnum`#1>126 %
        \PE@OctChar#1%
      \else  \ifnum`#1=40 \PE@EscapeStringAdd{\string\(}% (
        \else\ifnum`#1=41 \PE@EscapeStringAdd{\string\)}% )
        \else\ifnum`#1=92 \PE@EscapeStringAdd{\string\\}% \
        \else
          \PE@EscapeStringAdd{#1}%
        \fi\fi\fi
      \fi
    \fi
    \expandafter\PE@EscapeStringTokens
  \fi
}%
%    \end{macrocode}
%    \end{macro}
%    \begin{macro}{\PE@EscapeStringAdd}
%    \begin{macrocode}
\ifPE@etex
  \def\PE@EscapeStringAdd#1{#1}%
\else
  \def\PE@EscapeStringAdd#1{%
    \edef\PE@result{%
      \PE@result
      #1%
    }%
  }%
\fi
%    \end{macrocode}
%    \end{macro}
%
%    \begin{macrocode}
\PE@AtEnd%
%    \end{macrocode}
%
%    \begin{macrocode}
%</package>
%    \end{macrocode}
%
% \section{Test}
%
% \subsection{Catcode checks for loading}
%
%    \begin{macrocode}
%<*test1>
%    \end{macrocode}
%    \begin{macrocode}
\catcode`\{=1 %
\catcode`\}=2 %
\catcode`\#=6 %
\catcode`\@=11 %
\expandafter\ifx\csname count@\endcsname\relax
  \countdef\count@=255 %
\fi
\expandafter\ifx\csname @gobble\endcsname\relax
  \long\def\@gobble#1{}%
\fi
\expandafter\ifx\csname @firstofone\endcsname\relax
  \long\def\@firstofone#1{#1}%
\fi
\expandafter\ifx\csname loop\endcsname\relax
  \expandafter\@firstofone
\else
  \expandafter\@gobble
\fi
{%
  \def\loop#1\repeat{%
    \def\body{#1}%
    \iterate
  }%
  \def\iterate{%
    \body
      \let\next\iterate
    \else
      \let\next\relax
    \fi
    \next
  }%
  \let\repeat=\fi
}%
\def\RestoreCatcodes{}
\count@=0 %
\loop
  \edef\RestoreCatcodes{%
    \RestoreCatcodes
    \catcode\the\count@=\the\catcode\count@\relax
  }%
\ifnum\count@<255 %
  \advance\count@ 1 %
\repeat

\def\RangeCatcodeInvalid#1#2{%
  \count@=#1\relax
  \loop
    \catcode\count@=15 %
  \ifnum\count@<#2\relax
    \advance\count@ 1 %
  \repeat
}
\def\RangeCatcodeCheck#1#2#3{%
  \count@=#1\relax
  \loop
    \ifnum#3=\catcode\count@
    \else
      \errmessage{%
        Character \the\count@\space
        with wrong catcode \the\catcode\count@\space
        instead of \number#3%
      }%
    \fi
  \ifnum\count@<#2\relax
    \advance\count@ 1 %
  \repeat
}
\def\space{ }
\expandafter\ifx\csname LoadCommand\endcsname\relax
  \def\LoadCommand{\input pdfescape.sty\relax}%
\fi
\def\Test{%
  \RangeCatcodeInvalid{0}{47}%
  \RangeCatcodeInvalid{58}{64}%
  \RangeCatcodeInvalid{91}{96}%
  \RangeCatcodeInvalid{123}{255}%
  \catcode`\@=12 %
  \catcode`\\=0 %
  \catcode`\%=14 %
  \LoadCommand
  \RangeCatcodeCheck{0}{36}{15}%
  \RangeCatcodeCheck{37}{37}{14}%
  \RangeCatcodeCheck{38}{47}{15}%
  \RangeCatcodeCheck{48}{57}{12}%
  \RangeCatcodeCheck{58}{63}{15}%
  \RangeCatcodeCheck{64}{64}{12}%
  \RangeCatcodeCheck{65}{90}{11}%
  \RangeCatcodeCheck{91}{91}{15}%
  \RangeCatcodeCheck{92}{92}{0}%
  \RangeCatcodeCheck{93}{96}{15}%
  \RangeCatcodeCheck{97}{122}{11}%
  \RangeCatcodeCheck{123}{255}{15}%
  \RestoreCatcodes
}
\Test
\csname @@end\endcsname
\end
%    \end{macrocode}
%    \begin{macrocode}
%</test1>
%    \end{macrocode}
%
% \subsection{Macro tests}
%
%    \begin{macrocode}
%<*test2|test3|test4|test5>
\NeedsTeXFormat{LaTeX2e}
\makeatletter
%    \end{macrocode}
%
% \subsection{Test with \cs{pdfescape...} commands}
%
%    \begin{macrocode}
%<*test2>
\ProvidesFile{pdfescape-test2.tex}%
    [2011/11/25 v1.13 Test with \string\pdfescape... commands]%
%</test2>
%    \end{macrocode}
%
% \subsection{Test without \cs{pdfescape...}, with \eTeX}
%
%    \begin{macrocode}
%<*test3>
\ProvidesFile{pdfescape-test3.tex}%
    [2011/11/25 v1.13 Test without \string\pdfescape..., with e-TeX]%
%</test3>
%    \end{macrocode}
%
% \subsection{Test without \cs{pdfescape...} and \eTeX}
%
%    \begin{macrocode}
%<*test4>
\ProvidesFile{pdfescape-test4.tex}%
    [2011/11/25 v1.13 Test without \string\pdfescape... and e-TeX]%
%</test4>
%    \end{macrocode}
%
% \subsection{Test with \LuaTeX}
%
%    \begin{macrocode}
%<*test5>
\ProvidesFile{pdfescape-test5.tex}%
    [2011/11/25 v1.13 Test with LuaTeX]%
%</test5>
%    \end{macrocode}
%
% \subsection{Check/ensure test preconditions}
%
% \subsubsection{Check \cs{pdfescape...}}
%
%    \begin{macrocode}
%<*test2>
\@ifundefined{pdfescapehex}{%
  \PackageError{pdfescape-test2}{%
    Missing \string\pdfescape... commands%
  }{Test aborted.}%
  \stop
}{}
%</test2>
%    \end{macrocode}
%    \begin{macrocode}
%<*test3|test4>
\let\pdfescapehex\@undefined
\let\pdfunescapehex\@undefined
\let\pdfescapename\@undefined
\let\pdfescapestring\@undefined
%</test3|test4>
%    \end{macrocode}
%
% \subsubsection{Check \eTeX}
%
%    \begin{macrocode}
%<*test3>
\@ifundefined{numexpr}{%
  \PackageError{pdfescape-test3}{%
    Missing \eTeX
  }{Test aborted.}%
  \stop
}{}
%</test3>
%    \end{macrocode}
% Package \xpackage{qstest} uses \eTeX, thus \eTeX's features
% can only be disabled later during loading of package \xpackage{pdfescape}.
%
% \subsubsection{Check \LuaTeX}
%
%    \begin{macrocode}
%<*test5>
\@ifundefined{directlua}{%
  \PackageError{pdfescape-test5}{%
    Missing LuaTeX%
  }{Test aborted.}%
  \stop
}{}
%</test5>
%    \end{macrocode}
%
% \subsection{Common part}
%
%    The files for testing uses the framework, provided by the
%    new package \xpackage{qstest} of David Kastrup.
%    \begin{macrocode}
\RequirePackage{qstest}
\IncludeTests{*}
\LogTests{log}{*}{*}

\newcommand*{\ExpectVar}[2]{%
  \ifx#1#2%
  \else
    \begingroup
      \@onelevel@sanitize#1%
      \@onelevel@sanitize#2%
      \typeout{[#1] <> [#2]}% hash-ok
    \endgroup
  \fi
  \Expect*{\ifx#1#2true\else false\fi}{true}%
}

\makeatletter
\begingroup
  \gdef\AllBytes{}%
  \count@=0 %
  \catcode0=12 %
  \@whilenum\count@<256 \do{%
    \lccode0=\count@
    \ifnum\count@=32 %
      \xdef\AllBytes{\AllBytes\space}%
    \else
      \lowercase{%
        \xdef\AllBytes{\AllBytes^^@}%
      }%
    \fi
    \advance\count@ by 1 %
  }%
\endgroup
\newcommand*{\AllBytesHex}{%
  000102030405060708090A0B0C0D0E0F%
  101112131415161718191A1B1C1D1E1F%
  202122232425262728292A2B2C2D2E2F%
  303132333435363738393A3B3C3D3E3F%
  404142434445464748494A4B4C4D4E4F%
  505152535455565758595A5B5C5D5E5F%
  606162636465666768696A6B6C6D6E6F%
  707172737475767778797A7B7C7D7E7F%
  808182838485868788898A8B8C8D8E8F%
  909192939495969798999A9B9C9D9E9F%
  A0A1A2A3A4A5A6A7A8A9AAABACADAEAF%
  B0B1B2B3B4B5B6B7B8B9BABBBCBDBEBF%
  C0C1C2C3C4C5C6C7C8C9CACBCCCDCECF%
  D0D1D2D3D4D5D6D7D8D9DADBDCDDDEDF%
  E0E1E2E3E4E5E6E7E8E9EAEBECEDEEEF%
  F0F1F2F3F4F5F6F7F8F9FAFBFCFDFEFF%
}
\@onelevel@sanitize\AllBytesHex
\expandafter\lowercase\expandafter{%
  \expandafter\newcommand\expandafter*\expandafter\AllBytesHexLC
      \expandafter{\AllBytesHex}%
}
\newcommand*{\AllBytesName}{}
\begingroup
  \catcode`\#=12 %
  \xdef\AllBytesName{%
    #01#02#03#04#05#06#07#08#09#0A#0B#0C#0D#0E#0F%
    #10#11#12#13#14#15#16#17#18#19#1A#1B#1C#1D#1E#1F%
    #20!"#23$#25&'#28#29*+,-.#2F%
    0123456789:;#3C=#3E?%
    @ABCDEFGHIJKLMNO%
    PQRSTUVWXYZ#5B\@backslashchar#5D^_%
    `abcdefghijklmno%
    pqrstuvwxyz#7B|#7D\string~#7F%
    #80#81#82#83#84#85#86#87#88#89#8A#8B#8C#8D#8E#8F%
    #90#91#92#93#94#95#96#97#98#99#9A#9B#9C#9D#9E#9F%
    #A0#A1#A2#A3#A4#A5#A6#A7#A8#A9#AA#AB#AC#AD#AE#AF%
    #B0#B1#B2#B3#B4#B5#B6#B7#B8#B9#BA#BB#BC#BD#BE#BF%
    #C0#C1#C2#C3#C4#C5#C6#C7#C8#C9#CA#CB#CC#CD#CE#CF%
    #D0#D1#D2#D3#D4#D5#D6#D7#D8#D9#DA#DB#DC#DD#DE#DF%
    #E0#E1#E2#E3#E4#E5#E6#E7#E8#E9#EA#EB#EC#ED#EE#EF%
    #F0#F1#F2#F3#F4#F5#F6#F7#F8#F9#FA#FB#FC#FD#FE#FF%
  }%
\endgroup
\@onelevel@sanitize\AllBytesName

\newcommand*{\AllBytesString}{}
\begingroup
  \def\|{|}%
  \edef\%{\@percentchar}%
  \catcode`\|=0 %
  \catcode`\#=12 %
  \catcode`\~=12 %
  \catcode`\\=12 %
  |xdef|AllBytesString{%
    \000\001\002\003\004\005\006\007\010\011\012\013\014\015\016\017%
    \020\021\022\023\024\025\026\027\030\031\032\033\034\035\036\037%
    \040!"#$|%&'\(\)*+,-./%
    0123456789:;<=>?%
    @ABCDEFGHIJKLMNO%
    PQRSTUVWXYZ[\\]^_%
    `abcdefghijklmno%
    pqrstuvwxyz{||}~\177%
    \200\201\202\203\204\205\206\207\210\211\212\213\214\215\216\217%
    \220\221\222\223\224\225\226\227\230\231\232\233\234\235\236\237%
    \240\241\242\243\244\245\246\247\250\251\252\253\254\255\256\257%
    \260\261\262\263\264\265\266\267\270\271\272\273\274\275\276\277%
    \300\301\302\303\304\305\306\307\310\311\312\313\314\315\316\317%
    \320\321\322\323\324\325\326\327\330\331\332\333\334\335\336\337%
    \340\341\342\343\344\345\346\347\350\351\352\353\354\355\356\357%
    \360\361\362\363\364\365\366\367\370\371\372\373\374\375\376\377%
  }%
|endgroup
\@onelevel@sanitize\AllBytesString

%<*test4>
\let\org@detokenize\detokenize
\let\detokenize\@undefined
\let\org@numexpr\numexpr
\let\numexpr\@undefined
%</test4>
\RequirePackage{pdfescape}
%<*test4>
\let\detokenize\org@detokenize
\let\numexpr\org@numexpr
%</test4>

\begin{qstest}{all-hex}{\AllBytes, escapehex}
  \EdefEscapeHex\x{\AllBytes}%
  \Expect*{\x}*{\AllBytesHex}%
  \ExpectVar\x\AllBytesHex
\end{qstest}

\begin{qstest}{all-unhex}{\AllBytesHex, unescapehex}
  \EdefUnescapeHex\x{\AllBytesHex}%
  \Expect*{\x}*{\AllBytes}%
  \ExpectVar\x\AllBytes
\end{qstest}

\begin{qstest}{all-unhex-lc}{\AllBytesHexLC, unescapehex, lowercase}
  \EdefUnescapeHex\x{\AllBytesHexLC}%
  \Expect*{\x}*{\AllBytes}%
  \ExpectVar\x\AllBytes
\end{qstest}

\begin{qstest}{unhex-incomplete}{unescapehex, incomplete}
  \EdefUnescapeHex\x{4}%
  \Expect*{\x}{@}%
\end{qstest}

\begin{qstest}{unhex-space}{unescapehex, space}
  \EdefUnescapeHex\x{20}%
  \Expect*{\x}{ }%
  \ExpectVar\x\space
\end{qstest}

\begin{qstest}{unhex-spaces}{unescapehex, spaces}
  \EdefUnescapeHex\x{204020204120}%
  \def\y#1{%
    \edef\z{#1\string @#1#1\string A#1}%
  }\y{ }%
  \Expect*{\x}*{\z}%
  \ExpectVar\x\z
\end{qstest}

\begin{qstest}{unhex-hash}{unescapehex, hash}
  \catcode`\#=12 %
  \EdefUnescapeHex\x{#20}%
  \ExpectVar\x\space
\end{qstest}

\begin{qstest}{unhex-invalid}{unescapehex, invalid}
  \def\test#1#2{%
    \EdefUnescapeHex\x{#1}%
    \edef\y{#2}%
    \@onelevel@sanitize\y
    \ExpectVar\x\y
  }%
%<*test2>
  \edef\x{\pdfunescapehex{4X}}%
  \edef\y{\string @}%
  \ifx\x\y
  \else
    \def~{\space}%
    \typeout{************************************}%
    \typeout{* Your pdfTeX contains bug 777.~~~~*}%
    \typeout{* This test is redefined as dummy, *}%
    \typeout{* because it triggers the bug.~~~~~*}%
    \typeout{************************************}%
    \def\test#1#2{}%
  \fi
%</test2>
  \test{X}{}%
  \test{XY}{}%
  \test{XYZ}{}%
  \test{A}{^^a0}%
  \test{AX}{^^a0}%
  \test{XA}{^^a0}%
  \test{XXAXX}{^^a0}%
\end{qstest}

\begin{qstest}{all-name}{\AllBytes, escapename}
  \EdefEscapeName\x{\AllBytes}%
  \Expect*{\x}*{\AllBytesName}%
  \ExpectVar\x\AllBytesName
\end{qstest}

\begin{qstest}{all-string}{\AllBytes, escapestring}
  \EdefEscapeString\x{\AllBytes}%
  \Expect*{\x}*{\AllBytesString}%
  \ExpectVar\x\AllBytesString
\end{qstest}

\begin{qstest}{uchexdigit}{unescape, uppercase hex digit}
  \catcode`\@=11 %
  \catcode0=12 %
  \def\test#1#2{%
    \uccode0=#1\relax
    \uppercase{%
      \def\x{^^@}%
    }%
    \Expect*{%
       \ifcase\expandafter\PE@TestUcHexDigit\x
         true%
       \else
         false%
       \fi
    }{#2}%
  }%
  \def\range#1#2#3{%
    \count0=#1\relax
    \loop
    \ifnum\count0<#2\relax
      \test{\count0}{#3}%
      \advance\count0 by 1 %
    \repeat
  }%
  \range{0}{47}{false}%
  \range{48}{57}{true}%
  \range{58}{64}{false}%
  \range{65}{70}{true}%
  \range{71}{255}{false}%
\end{qstest}

\begin{qstest}{unescapename}{unescapename}
  \def\test#1#2{%
    \EdefUnescapeName\x{#1}%
    \edef\y{#2}%
    \@onelevel@sanitize\y
    \ExpectVar\x\y
  }%
  \catcode`\#=12 %
  \catcode0=12 %
  \test{}{}%
  \test{x}{x}%
  \test{xy}{xy}%
  \test{#}{#}%
  \test{##}{##}%
  \test{###}{###}%
  \test{####}{####}%
  \test{#x}{#x}%
  \test{#xy}{#xy}%
  \test{#1}{#1}%
  \test{#40}{@}%
  \test{#400}{@0}%
  \test{#4x0}{#4x0}%
  \test{#ab}{^^ab}%
  \test{#00}{^^@}%
  \test{x#40y#40z}{x@y@z}%
  \test{#40#40#40#40}{@@@@}%
  \test{a#x}{a#x}%
  \test{a#xy}{a#xy}%
  \test{a#1}{a#1}%
  \test{a#40}{a@}%
  \test{a#400}{a@0}%
  \test{#20}{ }%
  \test{a#20}{a }%
  \test{a#20b}{a b}%
  \test{a#20#20#20b}{a \space\space b}%
\end{qstest}

\begin{qstest}{unescapestring}{unescapestring}
  \def\test#1#2{%
    \EdefUnescapeString\x{#1}%
    \edef\y{#2}%
    \@onelevel@sanitize\y
    \ExpectVar\x\y
  }%
  \catcode0=12 %
  \def\DefChar#1#2{%
    \begingroup
      \uccode0=#2\relax
    \uppercase{\endgroup
      \def#1{^^@}%
    }%
  }%
  \DefChar\nul{0}%
  \DefChar\one{1}%
  \DefChar\bel{8}%
  \DefChar\tab{9}%
  \DefChar\lf{10}%
  \DefChar\ff{12}%
  \DefChar\cr{13}%
  \DefChar\\{92}%
  \test{}{}%
  \test{a}{a}%
  \test{\\}{}%
  \test{\\\\}{\\}%
  \test{\\\\y}{\\y}%
  \test{\\000}{\nul}%
  \test{\\b}{\bel}%
  \test{\\t}{\tab}%
  \test{\\n}{\lf}%
  \test{\\f}{\ff}%
  \test{\\r}{\cr}%
  \test{\\(}{(}%
  \test{\\)}{)}%
  \test{\\040}{ }%
  \test{\\100}{@}%
  \test{\\40}{ }%
  \test{\\1}{\one}%
  \test{\\01}{\one}%
  \test{\\001}{\one}%
  \test{\\18}{\one8}%
  \test{\\018}{\one8}%
  \test{\\0018}{\one8}%
  \test{x\\}{x}%
  \test{x\\\\}{x\\}%
  \test{x\\\\y}{x\\y}%
  \test{x\\000}{x\nul}%
  \test{x\\b}{x\bel}%
  \test{x\\t}{x\tab}%
  \test{x\\n}{x\lf}%
  \test{x\\f}{x\ff}%
  \test{x\\r}{x\cr}%
  \test{x\\(}{x(}%
  \test{x\\)}{x)}%
  \test{x\\040}{x }%
  \test{x\\100}{x@}%
  \test{x\\40}{x }%
  \test{x\\1}{x\one}%
  \test{x\\01}{x\one}%
  \test{x\\001}{x\one}%
  \test{x\\18}{x\one8}%
  \test{x\\018}{x\one8}%
  \test{x\\0018}{x\one8}%
  \test{\\b\\t\\n\\f\\r\\(\\)\\\\\\000\\040}{%
    \bel\tab\lf\ff\cr()\\\nul\space
  }%
  \test{\\\lf}{}%
  \test{x\\\lf}{x}%
  \test{\cr}{\lf}%
  \test{\cr\lf}{\lf}%
  \test{\lf}{\lf}%
  \test{\lf\cr}{\lf\lf}%
  \test{x\cr}{x\lf}%
  \test{x\cr\lf}{x\lf}%
  \test{x\lf}{x\lf}%
  \test{x\lf\cr}{x\lf\lf}%
  \test{x\\\cr\lf y\cr}{xy\lf}%
%
  \test{\\409}{ 9}%
  \test{\\800}{800}%
  \test{\\900}{900}%
  \test{\\578}{/8}%
  \test{\\477}{?}%
  \test{\\377}{^^ff}%
  \test{\\777}{^^ff}%
  \test{\\7777}{^^ff7}%
\end{qstest}
\stop
%    \end{macrocode}
%
%    \begin{macrocode}
%</test2|test3|test4|test5>
%    \end{macrocode}
%
% \subsubsection{Test for \hologo{iniTeX}}
%
%    \begin{macrocode}
%<*test6>
\input pdfescape.sty\relax
\catcode`\{=1 %
\catcode`\}=2 %
\catcode`\#=6 %
\catcode`\^=7 %
\catcode`\@=11 %
%    \end{macrocode}
%    \begin{macrocode}
\begingroup
  \catcode`\@=11 %
  \countdef\count@=255 %
  \def\space{ }%
  \long\def\@whilenum#1\do #2{%
    \ifnum #1\relax
      #2\relax
      \@iwhilenum{#1\relax#2\relax}%
    \fi
  }%
  \long\def\@iwhilenum#1{%
    \ifnum #1%
      \expandafter\@iwhilenum
    \else
      \expandafter\ltx@gobble
    \fi
    {#1}%
  }%
  \gdef\AllBytes{}%
  \count@=0 %
  \catcode0=12 %
  \@whilenum\count@<256 \do{%
    \lccode0=\count@
    \ifnum\count@=32 %
      \xdef\AllBytes{\AllBytes\space}%
    \else
      \lowercase{%
        \xdef\AllBytes{\AllBytes^^@}%
      }%
    \fi
    \advance\count@ by 1 %
  }%
\endgroup
\def\AllBytesHex{%
  000102030405060708090A0B0C0D0E0F%
  101112131415161718191A1B1C1D1E1F%
  202122232425262728292A2B2C2D2E2F%
  303132333435363738393A3B3C3D3E3F%
  404142434445464748494A4B4C4D4E4F%
  505152535455565758595A5B5C5D5E5F%
  606162636465666768696A6B6C6D6E6F%
  707172737475767778797A7B7C7D7E7F%
  808182838485868788898A8B8C8D8E8F%
  909192939495969798999A9B9C9D9E9F%
  A0A1A2A3A4A5A6A7A8A9AAABACADAEAF%
  B0B1B2B3B4B5B6B7B8B9BABBBCBDBEBF%
  C0C1C2C3C4C5C6C7C8C9CACBCCCDCECF%
  D0D1D2D3D4D5D6D7D8D9DADBDCDDDEDF%
  E0E1E2E3E4E5E6E7E8E9EAEBECEDEEEF%
  F0F1F2F3F4F5F6F7F8F9FAFBFCFDFEFF%
}
\ltx@onelevel@sanitize\AllBytesHex
\expandafter\lowercase\expandafter{%
  \expandafter\def\expandafter\AllBytesHexLC
      \expandafter{\AllBytesHex}%
}
\begingroup
  \catcode`\#=12 %
  \xdef\AllBytesName{%
    #01#02#03#04#05#06#07#08#09#0A#0B#0C#0D#0E#0F%
    #10#11#12#13#14#15#16#17#18#19#1A#1B#1C#1D#1E#1F%
    #20!"#23$#25&'#28#29*+,-.#2F%
    0123456789:;#3C=#3E?%
    @ABCDEFGHIJKLMNO%
    PQRSTUVWXYZ#5B\ltx@backslashchar#5D^_%
    `abcdefghijklmno%
    pqrstuvwxyz#7B|#7D\string~#7F%
    #80#81#82#83#84#85#86#87#88#89#8A#8B#8C#8D#8E#8F%
    #90#91#92#93#94#95#96#97#98#99#9A#9B#9C#9D#9E#9F%
    #A0#A1#A2#A3#A4#A5#A6#A7#A8#A9#AA#AB#AC#AD#AE#AF%
    #B0#B1#B2#B3#B4#B5#B6#B7#B8#B9#BA#BB#BC#BD#BE#BF%
    #C0#C1#C2#C3#C4#C5#C6#C7#C8#C9#CA#CB#CC#CD#CE#CF%
    #D0#D1#D2#D3#D4#D5#D6#D7#D8#D9#DA#DB#DC#DD#DE#DF%
    #E0#E1#E2#E3#E4#E5#E6#E7#E8#E9#EA#EB#EC#ED#EE#EF%
    #F0#F1#F2#F3#F4#F5#F6#F7#F8#F9#FA#FB#FC#FD#FE#FF%
  }%
\endgroup
\ltx@onelevel@sanitize\AllBytesName
\edef\AllBytesFromName{\expandafter\ltx@gobble\AllBytes}

\begingroup
  \def\|{|}%
  \edef\%{\ltx@percentchar}%
  \catcode`\|=0 %
  \catcode`\#=12 %
  \catcode`\~=12 %
  \catcode`\\=12 %
  |xdef|AllBytesString{%
    \000\001\002\003\004\005\006\007\010\011\012\013\014\015\016\017%
    \020\021\022\023\024\025\026\027\030\031\032\033\034\035\036\037%
    \040!"#$|%&'\(\)*+,-./%
    0123456789:;<=>?%
    @ABCDEFGHIJKLMNO%
    PQRSTUVWXYZ[\\]^_%
    `abcdefghijklmno%
    pqrstuvwxyz{||}~\177%
    \200\201\202\203\204\205\206\207\210\211\212\213\214\215\216\217%
    \220\221\222\223\224\225\226\227\230\231\232\233\234\235\236\237%
    \240\241\242\243\244\245\246\247\250\251\252\253\254\255\256\257%
    \260\261\262\263\264\265\266\267\270\271\272\273\274\275\276\277%
    \300\301\302\303\304\305\306\307\310\311\312\313\314\315\316\317%
    \320\321\322\323\324\325\326\327\330\331\332\333\334\335\336\337%
    \340\341\342\343\344\345\346\347\350\351\352\353\354\355\356\357%
    \360\361\362\363\364\365\366\367\370\371\372\373\374\375\376\377%
  }%
|endgroup
\ltx@onelevel@sanitize\AllBytesString
%    \end{macrocode}
%    \begin{macrocode}
\def\msg#{\immediate\write16}
\def\Test#1#2#3{%
  \begingroup
    #1\TestResult{#2}%
    \ifx\TestResult#3%
    \else
      \newlinechar=10 %
      \msg{Expect:^^J#3}%
      \msg{Result:^^J\TestResult}%
      \errmessage{\string#2 -\string#1-> \string#3}%
    \fi
  \endgroup
}
\Test\EdefEscapeHex\AllBytes\AllBytesHex
\Test\EdefUnescapeHex\AllBytesHex\AllBytes
\Test\EdefEscapeName\AllBytes\AllBytesName
\Test\EdefUnescapeName\AllBytesName\AllBytesFromName
\Test\EdefEscapeString\AllBytes\AllBytesString
\Test\EdefUnescapeString\AllBytesString\AllBytes
%    \end{macrocode}
%    \begin{macrocode}
\csname @@end\endcsname\end
%</test6>
%    \end{macrocode}
%
% \section{Installation}
%
% \subsection{Download}
%
% \paragraph{Package.} This package is available on
% CTAN\footnote{\url{ftp://ftp.ctan.org/tex-archive/}}:
% \begin{description}
% \item[\CTAN{macros/latex/contrib/oberdiek/pdfescape.dtx}] The source file.
% \item[\CTAN{macros/latex/contrib/oberdiek/pdfescape.pdf}] Documentation.
% \end{description}
%
%
% \paragraph{Bundle.} All the packages of the bundle `oberdiek'
% are also available in a TDS compliant ZIP archive. There
% the packages are already unpacked and the documentation files
% are generated. The files and directories obey the TDS standard.
% \begin{description}
% \item[\CTAN{install/macros/latex/contrib/oberdiek.tds.zip}]
% \end{description}
% \emph{TDS} refers to the standard ``A Directory Structure
% for \TeX\ Files'' (\CTAN{tds/tds.pdf}). Directories
% with \xfile{texmf} in their name are usually organized this way.
%
% \subsection{Bundle installation}
%
% \paragraph{Unpacking.} Unpack the \xfile{oberdiek.tds.zip} in the
% TDS tree (also known as \xfile{texmf} tree) of your choice.
% Example (linux):
% \begin{quote}
%   |unzip oberdiek.tds.zip -d ~/texmf|
% \end{quote}
%
% \paragraph{Script installation.}
% Check the directory \xfile{TDS:scripts/oberdiek/} for
% scripts that need further installation steps.
% Package \xpackage{attachfile2} comes with the Perl script
% \xfile{pdfatfi.pl} that should be installed in such a way
% that it can be called as \texttt{pdfatfi}.
% Example (linux):
% \begin{quote}
%   |chmod +x scripts/oberdiek/pdfatfi.pl|\\
%   |cp scripts/oberdiek/pdfatfi.pl /usr/local/bin/|
% \end{quote}
%
% \subsection{Package installation}
%
% \paragraph{Unpacking.} The \xfile{.dtx} file is a self-extracting
% \docstrip\ archive. The files are extracted by running the
% \xfile{.dtx} through \plainTeX:
% \begin{quote}
%   \verb|tex pdfescape.dtx|
% \end{quote}
%
% \paragraph{TDS.} Now the different files must be moved into
% the different directories in your installation TDS tree
% (also known as \xfile{texmf} tree):
% \begin{quote}
% \def\t{^^A
% \begin{tabular}{@{}>{\ttfamily}l@{ $\rightarrow$ }>{\ttfamily}l@{}}
%   pdfescape.sty & tex/generic/oberdiek/pdfescape.sty\\
%   pdfescape.pdf & doc/latex/oberdiek/pdfescape.pdf\\
%   test/pdfescape-test1.tex & doc/latex/oberdiek/test/pdfescape-test1.tex\\
%   test/pdfescape-test2.tex & doc/latex/oberdiek/test/pdfescape-test2.tex\\
%   test/pdfescape-test3.tex & doc/latex/oberdiek/test/pdfescape-test3.tex\\
%   test/pdfescape-test4.tex & doc/latex/oberdiek/test/pdfescape-test4.tex\\
%   test/pdfescape-test5.tex & doc/latex/oberdiek/test/pdfescape-test5.tex\\
%   test/pdfescape-test6.tex & doc/latex/oberdiek/test/pdfescape-test6.tex\\
%   pdfescape.dtx & source/latex/oberdiek/pdfescape.dtx\\
% \end{tabular}^^A
% }^^A
% \sbox0{\t}^^A
% \ifdim\wd0>\linewidth
%   \begingroup
%     \advance\linewidth by\leftmargin
%     \advance\linewidth by\rightmargin
%   \edef\x{\endgroup
%     \def\noexpand\lw{\the\linewidth}^^A
%   }\x
%   \def\lwbox{^^A
%     \leavevmode
%     \hbox to \linewidth{^^A
%       \kern-\leftmargin\relax
%       \hss
%       \usebox0
%       \hss
%       \kern-\rightmargin\relax
%     }^^A
%   }^^A
%   \ifdim\wd0>\lw
%     \sbox0{\small\t}^^A
%     \ifdim\wd0>\linewidth
%       \ifdim\wd0>\lw
%         \sbox0{\footnotesize\t}^^A
%         \ifdim\wd0>\linewidth
%           \ifdim\wd0>\lw
%             \sbox0{\scriptsize\t}^^A
%             \ifdim\wd0>\linewidth
%               \ifdim\wd0>\lw
%                 \sbox0{\tiny\t}^^A
%                 \ifdim\wd0>\linewidth
%                   \lwbox
%                 \else
%                   \usebox0
%                 \fi
%               \else
%                 \lwbox
%               \fi
%             \else
%               \usebox0
%             \fi
%           \else
%             \lwbox
%           \fi
%         \else
%           \usebox0
%         \fi
%       \else
%         \lwbox
%       \fi
%     \else
%       \usebox0
%     \fi
%   \else
%     \lwbox
%   \fi
% \else
%   \usebox0
% \fi
% \end{quote}
% If you have a \xfile{docstrip.cfg} that configures and enables \docstrip's
% TDS installing feature, then some files can already be in the right
% place, see the documentation of \docstrip.
%
% \subsection{Refresh file name databases}
%
% If your \TeX~distribution
% (\teTeX, \mikTeX, \dots) relies on file name databases, you must refresh
% these. For example, \teTeX\ users run \verb|texhash| or
% \verb|mktexlsr|.
%
% \subsection{Some details for the interested}
%
% \paragraph{Attached source.}
%
% The PDF documentation on CTAN also includes the
% \xfile{.dtx} source file. It can be extracted by
% AcrobatReader 6 or higher. Another option is \textsf{pdftk},
% e.g. unpack the file into the current directory:
% \begin{quote}
%   \verb|pdftk pdfescape.pdf unpack_files output .|
% \end{quote}
%
% \paragraph{Unpacking with \LaTeX.}
% The \xfile{.dtx} chooses its action depending on the format:
% \begin{description}
% \item[\plainTeX:] Run \docstrip\ and extract the files.
% \item[\LaTeX:] Generate the documentation.
% \end{description}
% If you insist on using \LaTeX\ for \docstrip\ (really,
% \docstrip\ does not need \LaTeX), then inform the autodetect routine
% about your intention:
% \begin{quote}
%   \verb|latex \let\install=y\input{pdfescape.dtx}|
% \end{quote}
% Do not forget to quote the argument according to the demands
% of your shell.
%
% \paragraph{Generating the documentation.}
% You can use both the \xfile{.dtx} or the \xfile{.drv} to generate
% the documentation. The process can be configured by the
% configuration file \xfile{ltxdoc.cfg}. For instance, put this
% line into this file, if you want to have A4 as paper format:
% \begin{quote}
%   \verb|\PassOptionsToClass{a4paper}{article}|
% \end{quote}
% An example follows how to generate the
% documentation with pdf\LaTeX:
% \begin{quote}
%\begin{verbatim}
%pdflatex pdfescape.dtx
%makeindex -s gind.ist pdfescape.idx
%pdflatex pdfescape.dtx
%makeindex -s gind.ist pdfescape.idx
%pdflatex pdfescape.dtx
%\end{verbatim}
% \end{quote}
%
% \section{Catalogue}
%
% The following XML file can be used as source for the
% \href{http://mirror.ctan.org/help/Catalogue/catalogue.html}{\TeX\ Catalogue}.
% The elements \texttt{caption} and \texttt{description} are imported
% from the original XML file from the Catalogue.
% The name of the XML file in the Catalogue is \xfile{pdfescape.xml}.
%    \begin{macrocode}
%<*catalogue>
<?xml version='1.0' encoding='us-ascii'?>
<!DOCTYPE entry SYSTEM 'catalogue.dtd'>
<entry datestamp='$Date$' modifier='$Author$' id='pdfescape'>
  <name>pdfescape</name>
  <caption>Implements pdfTeX's escape features using TeX or e-TeX.</caption>
  <authorref id='auth:oberdiek'/>
  <copyright owner='Heiko Oberdiek' year='2007,2010,2011'/>
  <license type='lppl1.3'/>
  <version number='1.13'/>
  <description>
    This package implements <xref refid='pdftex'>pdfTeX</xref>'s
    escape features (<tt>\pdfescapehex</tt>, <tt>\pdfunescapehex</tt>,
    <tt>\pdfescapename</tt>, <tt>\pdfescapestring</tt>) using TeX or
    e-TeX.
    <p/>
    The package is part of the <xref refid='oberdiek'>oberdiek</xref>
    bundle.
  </description>
  <documentation details='Package documentation'
      href='ctan:/macros/latex/contrib/oberdiek/pdfescape.pdf'/>
  <ctan file='true' path='/macros/latex/contrib/oberdiek/pdfescape.dtx'/>
  <miktex location='oberdiek'/>
  <texlive location='oberdiek'/>
  <install path='/macros/latex/contrib/oberdiek/oberdiek.tds.zip'/>
</entry>
%</catalogue>
%    \end{macrocode}
%
% \begin{History}
%   \begin{Version}{2007/02/21 v1.0}
%   \item
%     First version.
%   \end{Version}
%   \begin{Version}{2007/02/25 v1.1}
%   \item
%     Test files added.
%   \item
%     \cs{EdefUnescapeHex} supports lowercase letters.
%   \item
%     Fix: \verb|\EdefEscapeName{^^@}|
%   \item
%     Fix: \verb|\EdefEscapeName{\string#}|
%   \item
%     Fix for \cs{EdefUnescapeHex} in case of incomplete hex string.
%   \item
%     Fix: \cs{EdefUnescapeHex} generates space tokens with catcode
%     10 (space) in all cases.
%   \item
%     Fix: \cs{EdefEscapeHex} and \cs{EdefEscapeName} now generate
%     tokens with catcode 12 (other) only.
%   \end{Version}
%   \begin{Version}{2007/03/20 v1.2}
%   \item
%     Fix: Wrong year in \cs{ProvidesPackage}.
%   \end{Version}
%   \begin{Version}{2007/04/11 v1.3}
%   \item
%     Line ends sanitized.
%   \end{Version}
%   \begin{Version}{2007/04/21 v1.4}
%   \item
%     \cs{EdefUnescapeName} and \cs{EdefUnescapeString} added.
%   \end{Version}
%   \begin{Version}{2007/08/27 v1.5}
%   \item
%     \cs{EdefSanitize} added (replaces \cs{PE@sanitize}).
%   \end{Version}
%   \begin{Version}{2007/09/09 v1.6}
%   \item
%     Fix in catcode setup.
%   \end{Version}
%   \begin{Version}{2007/10/27 v1.7}
%   \item
%     More efficient \cs{EdefSanitize}.
%   \end{Version}
%   \begin{Version}{2007/11/11 v1.8}
%   \item
%     Use of package \xpackage{pdftexcmds} for \hologo{LuaTeX} support.
%   \end{Version}
%   \begin{Version}{2010/03/01 v1.9}
%   \item
%     Compatibility with \hologo{iniTeX}.
%   \end{Version}
%   \begin{Version}{2010/11/12 v1.10}
%   \item
%     Use of package \xpackage{ltxcmds}.
%   \item
%     Fix for compatibility with \hologo{iniTeX}.
%   \end{Version}
%   \begin{Version}{2011/01/30 v1.11}
%   \item
%     Already loaded package files are not input in \hologo{plainTeX}.
%   \end{Version}
%   \begin{Version}{2011/04/04 v1.12}
%   \item
%     Further fixes for compatibility for \hologo{iniTeX}.
%   \item
%     Test file for \hologo{iniTeX} added.
%   \end{Version}
%   \begin{Version}{2011/11/25 v1.13}
%   \item
%     Higher order bit of octal sequences in \cs{EdefUnescapeString}
%     ignored according to the PDF specification
%     (Bug found by Bruno Le Floch).
%   \end{Version}
% \end{History}
%
% \PrintIndex
%
% \Finale
\endinput

%        (quote the arguments according to the demands of your shell)
%
% Documentation:
%    (a) If pdfescape.drv is present:
%           latex pdfescape.drv
%    (b) Without pdfescape.drv:
%           latex pdfescape.dtx; ...
%    The class ltxdoc loads the configuration file ltxdoc.cfg
%    if available. Here you can specify further options, e.g.
%    use A4 as paper format:
%       \PassOptionsToClass{a4paper}{article}
%
%    Programm calls to get the documentation (example):
%       pdflatex pdfescape.dtx
%       makeindex -s gind.ist pdfescape.idx
%       pdflatex pdfescape.dtx
%       makeindex -s gind.ist pdfescape.idx
%       pdflatex pdfescape.dtx
%
% Installation:
%    TDS:tex/generic/oberdiek/pdfescape.sty
%    TDS:doc/latex/oberdiek/pdfescape.pdf
%    TDS:doc/latex/oberdiek/test/pdfescape-test1.tex
%    TDS:doc/latex/oberdiek/test/pdfescape-test2.tex
%    TDS:doc/latex/oberdiek/test/pdfescape-test3.tex
%    TDS:doc/latex/oberdiek/test/pdfescape-test4.tex
%    TDS:doc/latex/oberdiek/test/pdfescape-test5.tex
%    TDS:doc/latex/oberdiek/test/pdfescape-test6.tex
%    TDS:source/latex/oberdiek/pdfescape.dtx
%
%<*ignore>
\begingroup
  \catcode123=1 %
  \catcode125=2 %
  \def\x{LaTeX2e}%
\expandafter\endgroup
\ifcase 0\ifx\install y1\fi\expandafter
         \ifx\csname processbatchFile\endcsname\relax\else1\fi
         \ifx\fmtname\x\else 1\fi\relax
\else\csname fi\endcsname
%</ignore>
%<*install>
\input docstrip.tex
\Msg{************************************************************************}
\Msg{* Installation}
\Msg{* Package: pdfescape 2011/11/25 v1.13 Implements pdfTeX's escape features (HO)}
\Msg{************************************************************************}

\keepsilent
\askforoverwritefalse

\let\MetaPrefix\relax
\preamble

This is a generated file.

Project: pdfescape
Version: 2011/11/25 v1.13

Copyright (C) 2007, 2010, 2011 by
   Heiko Oberdiek <heiko.oberdiek at googlemail.com>

This work may be distributed and/or modified under the
conditions of the LaTeX Project Public License, either
version 1.3c of this license or (at your option) any later
version. This version of this license is in
   http://www.latex-project.org/lppl/lppl-1-3c.txt
and the latest version of this license is in
   http://www.latex-project.org/lppl.txt
and version 1.3 or later is part of all distributions of
LaTeX version 2005/12/01 or later.

This work has the LPPL maintenance status "maintained".

This Current Maintainer of this work is Heiko Oberdiek.

The Base Interpreter refers to any `TeX-Format',
because some files are installed in TDS:tex/generic//.

This work consists of the main source file pdfescape.dtx
and the derived files
   pdfescape.sty, pdfescape.pdf, pdfescape.ins, pdfescape.drv,
   pdfescape-test1.tex, pdfescape-test2.tex,
   pdfescape-test3.tex, pdfescape-test4.tex,
   pdfescape-test5.tex, pdfescape-test6.tex.

\endpreamble
\let\MetaPrefix\DoubleperCent

\generate{%
  \file{pdfescape.ins}{\from{pdfescape.dtx}{install}}%
  \file{pdfescape.drv}{\from{pdfescape.dtx}{driver}}%
  \usedir{tex/generic/oberdiek}%
  \file{pdfescape.sty}{\from{pdfescape.dtx}{package}}%
  \usedir{doc/latex/oberdiek/test}%
  \file{pdfescape-test1.tex}{\from{pdfescape.dtx}{test1}}%
  \file{pdfescape-test2.tex}{\from{pdfescape.dtx}{test2}}%
  \file{pdfescape-test3.tex}{\from{pdfescape.dtx}{test3}}%
  \file{pdfescape-test4.tex}{\from{pdfescape.dtx}{test4}}%
  \file{pdfescape-test5.tex}{\from{pdfescape.dtx}{test5}}%
  \file{pdfescape-test6.tex}{\from{pdfescape.dtx}{test6}}%
  \nopreamble
  \nopostamble
  \usedir{source/latex/oberdiek/catalogue}%
  \file{pdfescape.xml}{\from{pdfescape.dtx}{catalogue}}%
}

\catcode32=13\relax% active space
\let =\space%
\Msg{************************************************************************}
\Msg{*}
\Msg{* To finish the installation you have to move the following}
\Msg{* file into a directory searched by TeX:}
\Msg{*}
\Msg{*     pdfescape.sty}
\Msg{*}
\Msg{* To produce the documentation run the file `pdfescape.drv'}
\Msg{* through LaTeX.}
\Msg{*}
\Msg{* Happy TeXing!}
\Msg{*}
\Msg{************************************************************************}

\endbatchfile
%</install>
%<*ignore>
\fi
%</ignore>
%<*driver>
\NeedsTeXFormat{LaTeX2e}
\ProvidesFile{pdfescape.drv}%
  [2011/11/25 v1.13 Implements pdfTeX's escape features (HO)]%
\documentclass{ltxdoc}
\usepackage{holtxdoc}[2011/11/22]
\begin{document}
  \DocInput{pdfescape.dtx}%
\end{document}
%</driver>
% \fi
%
% \CheckSum{2516}
%
% \CharacterTable
%  {Upper-case    \A\B\C\D\E\F\G\H\I\J\K\L\M\N\O\P\Q\R\S\T\U\V\W\X\Y\Z
%   Lower-case    \a\b\c\d\e\f\g\h\i\j\k\l\m\n\o\p\q\r\s\t\u\v\w\x\y\z
%   Digits        \0\1\2\3\4\5\6\7\8\9
%   Exclamation   \!     Double quote  \"     Hash (number) \#
%   Dollar        \$     Percent       \%     Ampersand     \&
%   Acute accent  \'     Left paren    \(     Right paren   \)
%   Asterisk      \*     Plus          \+     Comma         \,
%   Minus         \-     Point         \.     Solidus       \/
%   Colon         \:     Semicolon     \;     Less than     \<
%   Equals        \=     Greater than  \>     Question mark \?
%   Commercial at \@     Left bracket  \[     Backslash     \\
%   Right bracket \]     Circumflex    \^     Underscore    \_
%   Grave accent  \`     Left brace    \{     Vertical bar  \|
%   Right brace   \}     Tilde         \~}
%
% \GetFileInfo{pdfescape.drv}
%
% \title{The \xpackage{pdfescape} package}
% \date{2011/11/25 v1.13}
% \author{Heiko Oberdiek\\\xemail{heiko.oberdiek at googlemail.com}}
%
% \maketitle
%
% \begin{abstract}
% This package implements \pdfTeX's escape features
% (\cs{pdfescapehex}, \cs{pdfunescapehex}, \cs{pdfescapename},
% \cs{pdfescapestring}) using \TeX\ or \eTeX.
% \end{abstract}
%
% \tableofcontents
%
% \section{Documentation}
%
% \begin{declcs}{EdefEscapeHex} \M{cmd} \M{string}\\
%  \cs{EdefUnescapeHex} \M{cmd} \M{string}\\
%  \cs{EdefEscapeName} \M{cmd} \M{string}\\
%  \cs{EdefEscapeString} \M{cmd} \M{string}
% \end{declcs}
% These commands converts \meta{string} and stores the result
% in macro \meta{cmd}. The conversion result is the same
% as the conversion of the corresponding \pdfTeX's primitives.
% Note that the argument \meta{string} is expanded before
% the conversion.
%
% For example, if \pdfTeX\ \textgreater= 1.30 is present,
% then \cs{EdefEscapeHex}
% becomes to:
% \begin{quote}
%\begin{verbatim}
%\def\EdefEscapeHex#1#2{%
%  \edef#1{\pdfescapehex{#2}}%
%}
%\end{verbatim}
% \end{quote}
% The package provides implementations for the case that \pdfTeX\ is
% not present (or too old). Even \eTeX\ can be missing, however it is
% used if it is detected.
%
% \paragraph{Babel.}
% The input strings may contain shorthand characters of package
% \xpackage{babel}.
%
% \subsection{Additional unescape macros}
%
% \begin{declcs}{EdefUnescapeName} \M{cmd} \M{string}
% \end{declcs}
% Sequences of a hash sign with two hexadecimal digits are
% converted to the corresponding character (PDF-1.2).
% A hash sign that is not followed by two hexadecimal digits
% is left unchanged. The catcodes in the result string follow
% TeX's conventions. The space has catcode 10 (space) and
% the other characters have catcode 12 (other).
%
% \begin{declcs}{EdefUnescapeString} \M{cmd} \M{string}
% \end{declcs}
% Macro \meta{cmd} stores the unescaped string in \meta{string}.
% All the rules for literal strings are implemented, see
% PDF specification.
% The catcodes in the result string follow TeX's conventions.
%
% \subsection{Sanitizing macro}
%
% \begin{declcs}{EdefSanitize} \M{cmd} \M{string}
% \end{declcs}
% Argument \meta{string} is expanded, converted to a string
% of tokens with catcode 12 (other) and space tokens, and
% stored in macro \meta{cmd}.
%
% \StopEventually{
% }
%
% \section{Implementation}
%
%    \begin{macrocode}
%<*package>
%    \end{macrocode}
%
% \subsection{Reload check and package identification}
%    Reload check, especially if the package is not used with \LaTeX.
%    \begin{macrocode}
\begingroup\catcode61\catcode48\catcode32=10\relax%
  \catcode13=5 % ^^M
  \endlinechar=13 %
  \catcode35=6 % #
  \catcode39=12 % '
  \catcode44=12 % ,
  \catcode45=12 % -
  \catcode46=12 % .
  \catcode58=12 % :
  \catcode64=11 % @
  \catcode123=1 % {
  \catcode125=2 % }
  \expandafter\let\expandafter\x\csname ver@pdfescape.sty\endcsname
  \ifx\x\relax % plain-TeX, first loading
  \else
    \def\empty{}%
    \ifx\x\empty % LaTeX, first loading,
      % variable is initialized, but \ProvidesPackage not yet seen
    \else
      \expandafter\ifx\csname PackageInfo\endcsname\relax
        \def\x#1#2{%
          \immediate\write-1{Package #1 Info: #2.}%
        }%
      \else
        \def\x#1#2{\PackageInfo{#1}{#2, stopped}}%
      \fi
      \x{pdfescape}{The package is already loaded}%
      \aftergroup\endinput
    \fi
  \fi
\endgroup%
%    \end{macrocode}
%    Package identification:
%    \begin{macrocode}
\begingroup\catcode61\catcode48\catcode32=10\relax%
  \catcode13=5 % ^^M
  \endlinechar=13 %
  \catcode35=6 % #
  \catcode39=12 % '
  \catcode40=12 % (
  \catcode41=12 % )
  \catcode44=12 % ,
  \catcode45=12 % -
  \catcode46=12 % .
  \catcode47=12 % /
  \catcode58=12 % :
  \catcode64=11 % @
  \catcode91=12 % [
  \catcode93=12 % ]
  \catcode123=1 % {
  \catcode125=2 % }
  \expandafter\ifx\csname ProvidesPackage\endcsname\relax
    \def\x#1#2#3[#4]{\endgroup
      \immediate\write-1{Package: #3 #4}%
      \xdef#1{#4}%
    }%
  \else
    \def\x#1#2[#3]{\endgroup
      #2[{#3}]%
      \ifx#1\@undefined
        \xdef#1{#3}%
      \fi
      \ifx#1\relax
        \xdef#1{#3}%
      \fi
    }%
  \fi
\expandafter\x\csname ver@pdfescape.sty\endcsname
\ProvidesPackage{pdfescape}%
  [2011/11/25 v1.13 Implements pdfTeX's escape features (HO)]%
%    \end{macrocode}
%
% \subsection{Catcodes}
%
%    \begin{macrocode}
\begingroup\catcode61\catcode48\catcode32=10\relax%
  \catcode13=5 % ^^M
  \endlinechar=13 %
  \catcode123 1 % {
  \catcode125 2 % }
  \catcode64 11 %
  \def\x{\endgroup
    \expandafter\edef\csname PE@AtEnd\endcsname{%
      \endlinechar=\the\endlinechar\relax
      \catcode13=\the\catcode13\relax
      \catcode32=\the\catcode32\relax
      \catcode35=\the\catcode35\relax
      \catcode61=\the\catcode61\relax
      \catcode64=\the\catcode64\relax
      \catcode123=\the\catcode123\relax
      \catcode125=\the\catcode125\relax
    }%
  }%
\x\catcode61\catcode48\catcode32=10\relax%
\catcode13=5 % ^^M
\endlinechar=13 %
\catcode35=6 % #
\catcode64=11 % @
\catcode123=1 % {
\catcode125=2 % }
\def\TMP@EnsureCode#1#2#3{%
  \edef\PE@AtEnd{%
    \PE@AtEnd
    #1#2=\the#1#2\relax
  }%
  #1#2=#3\relax
}
\TMP@EnsureCode\catcode{0}{12}% ^^@
\TMP@EnsureCode\catcode{34}{12}% "
\TMP@EnsureCode\catcode{36}{3}% $
\TMP@EnsureCode\catcode{38}{4}% &
\TMP@EnsureCode\catcode{39}{12}% '
\TMP@EnsureCode\catcode{42}{12}% *
\TMP@EnsureCode\catcode{45}{12}% -
\TMP@EnsureCode\catcode{46}{12}% .
\TMP@EnsureCode\catcode{47}{12}% /
\TMP@EnsureCode\catcode{60}{12}% <
\TMP@EnsureCode\catcode{62}{12}% >
\TMP@EnsureCode\catcode{91}{12}% [
\TMP@EnsureCode\catcode{93}{12}% ]
\TMP@EnsureCode\catcode{94}{7}% ^
\TMP@EnsureCode\catcode{96}{12}% `
\TMP@EnsureCode\uccode{34}{0}% "
\TMP@EnsureCode\uccode{48}{0}% 0
\TMP@EnsureCode\uccode{61}{0}% =
\edef\PE@AtEnd{\PE@AtEnd\noexpand\endinput}
%    \end{macrocode}
%
% \subsection{Load package}
%
%    \begin{macrocode}
\begingroup\expandafter\expandafter\expandafter\endgroup
\expandafter\ifx\csname RequirePackage\endcsname\relax
  \def\TMP@RequirePackage#1[#2]{%
    \begingroup\expandafter\expandafter\expandafter\endgroup
    \expandafter\ifx\csname ver@#1.sty\endcsname\relax
      \input #1.sty\relax
    \fi
  }%
  \TMP@RequirePackage{ltxcmds}[2010/04/08]%
\else
  \RequirePackage{ltxcmds}[2010/04/08]%
\fi
%    \end{macrocode}
%
% \subsection{Sanitizing}
%
%    \begin{macro}{\EdefSanitize}
%    Macro \cs{EdefSanitize} takes |#2|, entirely converts
%    it to token with catcode 12 (other) and stores the
%    result in macro |#1|.
%    \begin{macrocode}
\begingroup\expandafter\expandafter\expandafter\endgroup
\expandafter\ifx\csname detokenize\endcsname\relax
  \long\def\EdefSanitize#1#2{%
    \begingroup
      \csname @safe@activestrue\endcsname
      \edef#1{#2}%
      \PE@onelevel@sanitize#1%
    \expandafter\endgroup
    \expandafter\def\expandafter#1\expandafter{#1}%
  }%
  \begingroup\expandafter\expandafter\expandafter\endgroup
  \expandafter\ifx\csname @onelevel@sanitize\endcsname\relax
    \def\PE@onelevel@sanitize#1{%
      \edef#1{\expandafter\PE@strip@prefix\meaning#1}%
    }%
    \def\PE@strip@prefix#1>{}%
  \else
    \let\PE@onelevel@sanitize\@onelevel@sanitize
  \fi
\else
  \long\def\EdefSanitize#1#2{%
    \begingroup
      \csname @safe@activestrue\endcsname
      \edef#1{#2}%
    \expandafter\endgroup
    \expandafter\def\expandafter#1\expandafter{%
      \detokenize\expandafter{#1}%
    }%
  }%
  \def\PE@onelevel@sanitize#1{%
    \edef#1{\detokenize\expandafter{#1}}%
  }%
\fi
%    \end{macrocode}
%    \end{macro}
%    \begin{macro}{\PE@sanitize}
%    Macro \cs{PE@sanitize} is only defined for
%    compatibility with version 1.4. Its use is deprecated.
%    \begin{macrocode}
\let\PE@sanitize\EdefSanitize
%    \end{macrocode}
%    \end{macro}
%
% \subsubsection{Space characters}
%
%    \begin{macro}{\PE@space@other}
%    \begin{macrocode}
\begingroup
  \catcode`\ =12\relax%
\def\x{\endgroup\def\PE@space@other{ }}\x\relax
%    \end{macrocode}
%    \end{macro}
%    \begin{macro}{\PE@space@space}
%    \begin{macrocode}
\def\PE@space@space{ }
%    \end{macrocode}
%    \end{macro}
%
% \subsubsection{Space normalization}
%
%    \begin{macro}{\PE@SanitizeSpaceOther}
%    \begin{macrocode}
\def\PE@SanitizeSpaceOther#1{%
  \edef#1{\expandafter\PE@SpaceToOther#1 \relax}%
}
%    \end{macrocode}
%    \end{macro}
%
%    \begin{macro}{\PE@SpaceToOther}
%    \begin{macrocode}
\def\PE@SpaceToOther#1 #2\relax{%
  #1%
  \ifx\\#2\\%
  \else
    \PE@space@other
    \ltx@ReturnAfterFi{%
      \PE@SpaceToOther#2\relax
    }%
  \fi
}
%    \end{macrocode}
%    \end{macro}
%
% \subsection{\cs{EdefUnescapeName}}
%
%    \begin{macro}{\EdefUnescapeName}
%    \begin{macrocode}
\def\EdefUnescapeName#1#2{%
  \EdefSanitize#1{#2}%
  \PE@SanitizeSpaceOther#1%
  \PE@UnescapeName#1%
  \PE@onelevel@sanitize#1%
}
%    \end{macrocode}
%    \end{macro}
%
%    \begin{macro}{\PE@UnescapeName}
%    \begin{macrocode}
\begingroup
  \catcode`\$=6  % hash
  \catcode`\#=12 % other
  \gdef\PE@UnescapeName$1{%
    \begingroup
      \PE@InitUccodeHexDigit
      \def\PE@result{}%
      \expandafter\PE@DeName$1#\relax\relax
    \expandafter\endgroup
    \expandafter\def\expandafter$1\expandafter{\PE@result}%
  }%
  \gdef\PE@DeName$1#$2$3{%
    \ifx\relax$2%
      \edef\PE@result{\PE@result$1}%
      \let\PE@next\relax
    \else
      \ifx\relax$3%
        % wrong escape sequence in input
        \edef\PE@result{\PE@result$1#}%
        \let\PE@next\relax
      \else
        \uppercase{%
          \def\PE@testA{$2}%
          \def\PE@testB{$3}%
        }%
        \ifcase\ifcase\expandafter\PE@TestUcHexDigit\PE@testA
                 \ifcase\expandafter\PE@TestUcHexDigit\PE@testB
                   \ltx@zero
                 \else
                   \ltx@one
                 \fi
               \else
                 \ltx@one
               \fi
          \uccode\ltx@zero="\PE@testA\PE@testB\relax
          \uppercase{%
            \def\PE@temp{^^@}%
          }%
          \uccode\ltx@zero=\ltx@zero
          \edef\PE@result{\PE@result$1\PE@temp}%
          \let\PE@next\PE@DeName
        \else
          % wrong escape sequence in input
          \edef\PE@result{\PE@result$1#}%
          \def\PE@next{\PE@DeName$2$3}%
        \fi
      \fi
    \fi
    \PE@next
  }%
\endgroup
%    \end{macrocode}
%    \end{macro}
%
%    \begin{macro}{\PE@InitUccodeHexDigit}
%    \begin{macrocode}
\def\PE@InitUccodeHexDigit{%
  \uccode`a=`A\relax
  \uccode`b=`B\relax
  \uccode`c=`C\relax
  \uccode`d=`D\relax
  \uccode`e=`E\relax
  \uccode`f=`F\relax
  \uccode`A=\ltx@zero
  \uccode`B=\ltx@zero
  \uccode`C=\ltx@zero
  \uccode`D=\ltx@zero
  \uccode`E=\ltx@zero
  \uccode`F=\ltx@zero
  \uccode`0=\ltx@zero
  \uccode`1=\ltx@zero
  \uccode`2=\ltx@zero
  \uccode`3=\ltx@zero
  \uccode`4=\ltx@zero
  \uccode`5=\ltx@zero
  \uccode`6=\ltx@zero
  \uccode`7=\ltx@zero
  \uccode`8=\ltx@zero
  \uccode`9=\ltx@zero
}
%    \end{macrocode}
%    \end{macro}
%
%    \begin{macro}{\PE@TestUcHexDigit}
%    \begin{macrocode}
\def\PE@TestUcHexDigit#1{%
  \ifnum`#1<48 % 0
    \ltx@one
  \else
    \ifnum`#1>70 % F
      \ltx@one
    \else
      \ifnum`#1>57 % 9
        \ifnum`#1<65 % A
          \ltx@one
        \else
          \ltx@zero
        \fi
      \else
        \ltx@zero
      \fi
    \fi
  \fi
}
%    \end{macrocode}
%    \end{macro}
%
% \subsection{\cs{EdefUnescapeString}}
%
%    \begin{macro}{\EdefUnescapeString}
%    \begin{macrocode}
\def\EdefUnescapeString#1#2{%
  \EdefSanitize#1{#2}%
  \PE@SanitizeSpaceOther#1%
  \PE@NormalizeLineEnd#1%
  \PE@UnescapeString#1%
  \PE@onelevel@sanitize#1%
}
%    \end{macrocode}
%    \end{macro}
%
%    \begin{macrocode}
\begingroup
  \uccode`\8=10 % lf
  \uccode`\9=13 % cr
\def\x#1#2{\endgroup
%    \end{macrocode}
%    \begin{macro}{\PE@NormalizeLineEnd}
%    \begin{macrocode}
  \def\PE@NormalizeLineEnd##1{%
    \def\PE@result{}%
    \expandafter\PE@@NormalizeLineEnd##1#2\relax
    \let##1\PE@result
  }%
%    \end{macrocode}
%    \end{macro}
%    \begin{macro}{\PE@@NormalizeLineEnd}
%    \begin{macrocode}
  \def\PE@@NormalizeLineEnd##1#2##2{%
    \ifx\relax##2%
      \edef\PE@result{\PE@result##1}%
      \let\PE@next\relax
    \else
      \edef\PE@result{\PE@result##1#1}%
      \ifx#1##2% lf
        \let\PE@next\PE@@NormalizeLineEnd
      \else
        \def\PE@next{\PE@@NormalizeLineEnd##2}%
      \fi
    \fi
    \PE@next
  }%
}%
\uppercase{%
  \x 89%
}
%    \end{macrocode}
%    \end{macro}
%
%    \begin{macrocode}
\begingroup
  \catcode`\|=0 %
  \catcode`\\=12 %
%    \end{macrocode}
%    \begin{macro}{\PE@UnescapeString}
%    \begin{macrocode}
  |gdef|PE@UnescapeString#1{%
    |begingroup
      |def|PE@result{}%
      |expandafter|PE@DeString#1\|relax
    |expandafter|endgroup
    |expandafter|def|expandafter#1|expandafter{|PE@result}%
  }%
%    \end{macrocode}
%    \end{macro}
%    \begin{macro}{\PE@DeString}
%    \begin{macrocode}
  |gdef|PE@DeString#1\#2{%
    |ifx|relax#2%
      |edef|PE@result{|PE@result#1}%
      |let|PE@next|relax
    |else
      |if n#2%
        |uccode|ltx@zero=10 %
      |else|if r#2%
        |uccode|ltx@zero=13 %
      |else|if t#2%
        |uccode|ltx@zero=9 %
      |else|if b#2%
        |uccode|ltx@zero=8 %
      |else|if f#2%
        |uccode|ltx@zero=12 %
      |else
        |uccode|ltx@zero=|ltx@zero
      |fi|fi|fi|fi|fi
      |ifnum|uccode|ltx@zero>|ltx@zero
        |uppercase{%
          |edef|PE@temp{^^@}%
        }%
        |edef|PE@result{|PE@result#1|PE@temp}%
        |let|PE@next|PE@DeString
      |else
        |if\#2% backslash
          |edef|PE@result{|PE@result#1}%
          |let|PE@next|PE@CheckEndBackslash
        |else
          |ifnum`#2=10 % linefeed
            |edef|PE@result{|PE@result#1}%
            |let|PE@next|PE@DeString
          |else
            |ifcase|PE@TestOctDigit#2%
              |edef|PE@result{|PE@result#1}%
              |def|PE@next{|PE@OctI#2}%
            |else
              |edef|PE@result{|PE@result#1#2}%
              |let|PE@next|PE@DeString
            |fi
          |fi
        |fi
      |fi
    |fi
    |PE@next
  }%
%    \end{macrocode}
%    \end{macro}
%    \begin{macro}{\PE@CheckEndBackslash}
%    \begin{macrocode}
  |gdef|PE@CheckEndBackslash#1{%
    |ifx|relax#1%
    |else
      |edef|PE@result{|PE@result\}%
      |expandafter|PE@DeString|expandafter#1%
    |fi
  }%
%    \end{macrocode}
%    \end{macro}
%    \begin{macrocode}
|endgroup
%    \end{macrocode}
%    \begin{macro}{\PE@TestOctDigit}
%    \begin{macrocode}
\def\PE@TestOctDigit#1{%
  \ifnum`#1<48 % 0
    \ltx@one
  \else
    \ifnum`#1>55 % 7
      \ltx@one
    \else
      \ltx@zero
    \fi
  \fi
}
%    \end{macrocode}
%    \end{macro}
%    \begin{macro}{\PE@OctI}
%    \begin{macrocode}
\def\PE@OctI#1#2{%
  \ifcase\PE@TestOctDigit#2%
    \def\PE@next{\PE@OctII{#1#2}}%
  \else
    \def\PE@next{\PE@OctAll#1#2}%
  \fi
  \PE@next
}
%    \end{macrocode}
%    \end{macro}
%    \begin{macro}{\PE@OctII}
%    \begin{macrocode}
\def\PE@OctII#1#2{%
  \ifcase\PE@TestOctDigit#2%
    \def\PE@next{\PE@OctIII#1#2}%
  \else
    \def\PE@next{\PE@OctAll{#1}#2}%
  \fi
  \PE@next
}
%    \end{macrocode}
%    \end{macro}
%    \begin{macrocode}
\ltx@IfUndefined{numexpr}{%
  \catcode`\$=9 %
  \catcode`\&=14 %
}{%
  \catcode`\$=14 %
  \catcode`\&=9 %
}
%    \end{macrocode}
%    \begin{macro}{\PE@OctIII}
%    \begin{macrocode}
\def\PE@OctIII#1#2#3{%
  \ifnum#1<4 %
    \def\PE@next{\PE@OctAll{#1#2#3}}%
  \else
$   \count\ltx@cclv#1 %
$   \advance\count\ltx@cclv -4 %
    \edef\PE@next{%
      \noexpand\PE@OctAll{%
$       \the\count\ltx@cclv
&       \the\numexpr#1-4\relax
        #2#3%
      }%
    }%
  \fi
  \PE@next
}
%    \end{macrocode}
%    \end{macro}
%    \begin{macro}{\PE@OctAll}
%    \begin{macrocode}
\def\PE@OctAll#1{%
  \uccode\ltx@zero='#1\relax
  \uppercase{%
    \edef\PE@result{\PE@result^^@}%
  }%
  \PE@DeString
}
%    \end{macrocode}
%    \end{macro}
%
% \subsection{User macros (\pdfTeX\space analogues)}
%
%    \begin{macrocode}
\begingroup\expandafter\expandafter\expandafter\endgroup
\expandafter\ifx\csname RequirePackage\endcsname\relax
  \def\TMP@RequirePackage#1[#2]{%
    \begingroup\expandafter\expandafter\expandafter\endgroup
    \expandafter\ifx\csname ver@#1.sty\endcsname\relax
      \input #1.sty\relax
    \fi
  }%
  \TMP@RequirePackage{pdftexcmds}[2007/11/11]%
\else
  \RequirePackage{pdftexcmds}[2007/11/11]%
\fi
%    \end{macrocode}
%
%    \begin{macrocode}
\begingroup\expandafter\expandafter\expandafter\endgroup
\expandafter\ifx\csname pdf@escapehex\endcsname\relax
%    \end{macrocode}
%
%    \begin{macro}{\EdefEscapeHex}
%    \begin{macrocode}
  \long\def\EdefEscapeHex#1#2{%
    \EdefSanitize#1{#2}%
    \PE@SanitizeSpaceOther#1%
    \PE@EscapeHex#1%
  }%
%    \end{macrocode}
%    \end{macro}
%    \begin{macro}{\EdefUnescapeHex}
%    \begin{macrocode}
  \def\EdefUnescapeHex#1#2{%
    \EdefSanitize#1{#2}%
    \PE@UnescapeHex#1%
  }%
%    \end{macrocode}
%    \end{macro}
%    \begin{macro}{\EdefEscapeName}
%    \begin{macrocode}
  \long\def\EdefEscapeName#1#2{%
    \EdefSanitize#1{#2}%
    \PE@SanitizeSpaceOther#1%
    \PE@EscapeName#1%
  }%
%    \end{macrocode}
%    \end{macro}
%    \begin{macro}{\EdefEscapeString}
%    \begin{macrocode}
  \long\def\EdefEscapeString#1#2{%
    \EdefSanitize#1{#2}%
    \PE@SanitizeSpaceOther#1%
    \PE@EscapeString#1%
  }%
%    \end{macrocode}
%    \end{macro}
%
%    \begin{macrocode}
\else
%    \end{macrocode}
%
%    \begin{macro}{\PE@edefbabel}
%    Help macro that adds support for \xpackage{babel}'s shorthand
%    characters.
%    \begin{macrocode}
  \long\def\PE@edefbabel#1#2#3{%
    \begingroup
      \csname @save@activestrue\endcsname
      \edef#1{#2{#3}}%
    \expandafter\endgroup
    \expandafter\def\expandafter#1\expandafter{#1}%
  }%
%    \end{macrocode}
%    \end{macro}
%
%    \begin{macro}{\EdefEscapeHex}
%    \begin{macrocode}
  \long\def\EdefEscapeHex#1#2{%
    \PE@edefbabel#1\pdf@escapehex{#2}%
  }%
%    \end{macrocode}
%    \end{macro}
%    \begin{macro}{\EdefUnescapeHex}
%    \begin{macrocode}
  \def\EdefUnescapeHex#1#2{%
    \PE@edefbabel#1\pdf@unescapehex{#2}%
  }%
%    \end{macrocode}
%    \end{macro}
%    \begin{macro}{\EdefEscapeName}
%    \begin{macrocode}
  \long\def\EdefEscapeName#1#2{%
    \PE@edefbabel#1\pdf@escapename{#2}%
  }%
%    \end{macrocode}
%    \end{macro}
%    \begin{macro}{\EdefEscapeString}
%    \begin{macrocode}
  \long\def\EdefEscapeString#1#2{%
    \PE@edefbabel#1\pdf@escapestring{#2}%
  }%
%    \end{macrocode}
%    \end{macro}
%
%    \begin{macrocode}
  \expandafter\PE@AtEnd
\fi%
%    \end{macrocode}
%
% \subsection{Help macros}
%
% \subsubsection{Characters}
%
%    Special characters with catcode 12 (other) are created and
%    stored in macros.
%    \begin{macro}{\PE@hash}
%    \begin{macrocode}
\edef\PE@hash{\string#}
%    \end{macrocode}
%    \end{macro}
%    \begin{macro}{\PE@backslash}
%    \begin{macrocode}
\begingroup
  \escapechar=-1 %
\edef\x{\endgroup
  \def\noexpand\PE@backslash{\string\\}%
}
\x
%    \end{macrocode}
%    \end{macro}
%
% \subsubsection{Switch for \eTeX}
%
%    \begin{macrocode}
\ltx@newif\ifPE@etex
\begingroup\expandafter\expandafter\expandafter\endgroup
\expandafter\ifx\csname numexpr\endcsname\relax
\else
  \PE@etextrue
\fi
%    \end{macrocode}
%
% \subsection{Conversions}
%
% \subsubsection{Conversion to hex string}
%
%    \begin{macro}{\PE@EscapeHex}
%    \begin{macrocode}
\ifPE@etex
  \def\PE@EscapeHex#1{%
    \edef#1{\expandafter\PE@ToHex#1\relax}%
  }%
\else
  \def\PE@EscapeHex#1{%
    \def\PE@result{}%
    \expandafter\PE@ToHex#1\relax
    \let#1\PE@result
  }%
\fi
%    \end{macrocode}
%    \end{macro}
%    \begin{macro}{\PE@ToHex}
%    \begin{macrocode}
\def\PE@ToHex#1{%
  \ifx\relax#1%
  \else
    \PE@HexChar{#1}%
    \expandafter\PE@ToHex
  \fi
}%
%    \end{macrocode}
%    \end{macro}
%    \begin{macro}{\PE@HexChar}
%    \begin{macrocode}
\ifPE@etex
  \def\PE@HexChar#1{%
    \PE@HexDigit{\numexpr\dimexpr.0625\dimexpr`#1sp\relax\relax\relax}%
    \PE@HexDigit{%
      \numexpr`#1-16*\dimexpr.0625\dimexpr`#1sp\relax\relax\relax
    }%
  }%
\else
  \def\PE@HexChar#1{%
    \dimen0=`#1sp%
    \dimen2=.0625\dimen0 %
    \advance\dimen0-16\dimen2 %
    \edef\PE@result{%
      \PE@result
      \PE@HexDigit{\dimen2 }%
      \PE@HexDigit{\dimen0 }%
    }%
  }%
\fi
%    \end{macrocode}
%    \end{macro}
%    \begin{macro}{\PE@HexDigit}
%    \begin{macrocode}
\def\PE@HexDigit#1{%
  \expandafter\string
  \ifcase#1%
    0\or 1\or 2\or 3\or 4\or 5\or 6\or 7\or 8\or 9\or
    A\or B\or C\or D\or E\or F%
  \fi
}
%    \end{macrocode}
%    \end{macro}
%
% \subsubsection{Character code to octal number}
%
%    \begin{macro}{\PE@OctChar}
%    \begin{macrocode}
\ifPE@etex
  \def\PE@OctChar#1{%
    \expandafter\PE@@OctChar
        \the\numexpr\dimexpr.015625\dimexpr`#1sp\relax\relax
            \expandafter\relax
        \expandafter\relax
        \the\numexpr\dimexpr.125\dimexpr`#1sp\relax\relax\relax
        \relax
        #1%
  }%
  \def\PE@@OctChar#1\relax#2\relax#3{%
    \PE@backslash
    #1%
    \the\numexpr#2-8*#1\relax
    \the\numexpr\dimexpr`#3sp\relax-8*#2\relax
  }%
\else
  \def\PE@OctChar#1{%
    \dimen0=`#1sp%
    \dimen2=.125\dimen0 %
    \dimen4=.125\dimen2 %
    \advance\dimen0-8\dimen2 %
    \advance\dimen2-8\dimen4 %
    \edef\PE@result{%
      \PE@result
      \PE@backslash
      \number\dimen4 %
      \number\dimen2 %
      \number\dimen0 %
    }%
  }%
\fi
%    \end{macrocode}
%    \end{macro}
%
% \subsubsection{Unpack hex string}
%
%    \begin{macro}{\PE@UnescapeHex}
%    \begin{macrocode}
\def\PE@UnescapeHex#1{%
  \begingroup
    \PE@InitUccodeHexDigit
    \def\PE@result{}%
    \expandafter\PE@DeHex#1\relax\relax
  \expandafter\endgroup
  \expandafter\def\expandafter#1\expandafter{\PE@result}%
}
%    \end{macrocode}
%    \end{macro}
%    \begin{macro}{\PE@DeHex}
%    \begin{macrocode}
\def\PE@DeHex#1#2{%
  \ifx#2\relax
    \ifx#1\relax
      \let\PE@next\relax
    \else
      \uppercase{%
        \def\PE@testA{#1}%
      }%
      \ifcase\expandafter\PE@TestUcHexDigit\PE@testA
        \def\PE@next{%
          \PE@DeHex#10\relax\relax
        }%
      \else
        \let\PE@next\relax
      \fi
    \fi
  \else
    \uppercase{%
      \def\PE@testA{#1}%
      \def\PE@testB{#2}%
    }%
    \ifcase\expandafter\PE@TestUcHexDigit\PE@testA
      \ifcase\expandafter\PE@TestUcHexDigit\PE@testB
        \uccode\ltx@zero="\PE@testA\PE@testB\relax
        \ifnum\uccode\ltx@zero=32 %
          \let\PE@temp\PE@space@space
        \else
          \uppercase{%
            \def\PE@temp{^^@}%
          }%
        \fi
        \edef\PE@result{\PE@result\PE@temp}%
        \let\PE@next\PE@DeHex
      \else
        % invalid input sequence
        \def\PE@next{%
          \PE@DeHex#1%
        }%
      \fi
    \else
      % invalid input sequence
      \def\PE@next{\PE@DeHex#2}%
    \fi
  \fi
  \PE@next
}
%    \end{macrocode}
%    \end{macro}
%
% \subsubsection{Conversion to PDF name}
%
%    \begin{macro}{\PE@EscapeName}
%    \begin{macrocode}
\ifPE@etex
  \def\PE@EscapeName#1{%
    \edef#1{\expandafter\PE@EscapeNameTokens#1\relax}%
  }%
\else
  \def\PE@EscapeName#1{%
    \def\PE@result{}%
    \expandafter\PE@EscapeNameTokens#1\relax
    \let#1\PE@result
  }%
\fi
%    \end{macrocode}
%    \end{macro}
%    \begin{macro}{\PE@EscapeNameTokens}
%    \begin{macrocode}
\def\PE@EscapeNameTokens#1{%
  \ifx\relax#1%
  \else
    \ifnum`#1<33 %
      \ifcase`#1 %
        % drop illegal zero
      \else
        \PE@EscapeNameAdd\PE@hash
        \PE@HexChar#1%
      \fi
    \else
      \ifnum`#1>126 %
        \PE@EscapeNameAdd\PE@hash
        \PE@HexChar#1%
      \else  \ifnum`#1=35  \PE@EscapeNameHashChar 23% #
        \else\ifnum`#1=37  \PE@EscapeNameHashChar 25% %
        \else\ifnum`#1=40  \PE@EscapeNameHashChar 28% (
        \else\ifnum`#1=41  \PE@EscapeNameHashChar 29% )
        \else\ifnum`#1=47  \PE@EscapeNameHashChar 2F% /
        \else\ifnum`#1=60  \PE@EscapeNameHashChar 3C% <
        \else\ifnum`#1=62  \PE@EscapeNameHashChar 3E% >
        \else\ifnum`#1=91  \PE@EscapeNameHashChar 5B% [
        \else\ifnum`#1=93  \PE@EscapeNameHashChar 5D% ]
        \else\ifnum`#1=123 \PE@EscapeNameHashChar 7B% {
        \else\ifnum`#1=125 \PE@EscapeNameHashChar 7D% }
        \else
          \PE@EscapeNameAdd{#1}%
        \fi\fi\fi\fi\fi\fi\fi\fi\fi\fi\fi
      \fi
    \fi
    \expandafter\PE@EscapeNameTokens
  \fi
}%
\def\PE@EscapeNameHashChar#1#2{%
  \PE@EscapeNameAdd{\PE@hash\string#1\string#2}%
}%
%    \end{macrocode}
%    \end{macro}
%    \begin{macro}{\PE@EscapeNameAdd}
%    \begin{macrocode}
\ifPE@etex
  \def\PE@EscapeNameAdd#1{#1}%
\else
  \def\PE@EscapeNameAdd#1{%
    \edef\PE@result{%
      \PE@result
      #1%
    }%
  }%
\fi
%    \end{macrocode}
%    \end{macro}
%
% \subsubsection{Conversion to PDF string}
%
%    \begin{macro}{\PE@EscapeString}
%    \begin{macrocode}
\ifPE@etex
  \def\PE@EscapeString#1{%
    \edef#1{\expandafter\PE@EscapeStringTokens#1\relax}%
  }%
\else
  \def\PE@EscapeString#1{%
    \begingroup
      \def\PE@result{}%
      \expandafter\PE@EscapeStringTokens#1\relax
    \expandafter\endgroup
    \expandafter\def\expandafter#1\expandafter{\PE@result}%
  }%
\fi
%    \end{macrocode}
%    \end{macro}
%    \begin{macro}{\PE@EscapeStringTokens}
%    \begin{macrocode}
\def\PE@EscapeStringTokens#1{%
  \ifx\relax#1%
  \else
    \ifnum`#1<33 %
      \PE@OctChar#1%
    \else
      \ifnum`#1>126 %
        \PE@OctChar#1%
      \else  \ifnum`#1=40 \PE@EscapeStringAdd{\string\(}% (
        \else\ifnum`#1=41 \PE@EscapeStringAdd{\string\)}% )
        \else\ifnum`#1=92 \PE@EscapeStringAdd{\string\\}% \
        \else
          \PE@EscapeStringAdd{#1}%
        \fi\fi\fi
      \fi
    \fi
    \expandafter\PE@EscapeStringTokens
  \fi
}%
%    \end{macrocode}
%    \end{macro}
%    \begin{macro}{\PE@EscapeStringAdd}
%    \begin{macrocode}
\ifPE@etex
  \def\PE@EscapeStringAdd#1{#1}%
\else
  \def\PE@EscapeStringAdd#1{%
    \edef\PE@result{%
      \PE@result
      #1%
    }%
  }%
\fi
%    \end{macrocode}
%    \end{macro}
%
%    \begin{macrocode}
\PE@AtEnd%
%    \end{macrocode}
%
%    \begin{macrocode}
%</package>
%    \end{macrocode}
%
% \section{Test}
%
% \subsection{Catcode checks for loading}
%
%    \begin{macrocode}
%<*test1>
%    \end{macrocode}
%    \begin{macrocode}
\catcode`\{=1 %
\catcode`\}=2 %
\catcode`\#=6 %
\catcode`\@=11 %
\expandafter\ifx\csname count@\endcsname\relax
  \countdef\count@=255 %
\fi
\expandafter\ifx\csname @gobble\endcsname\relax
  \long\def\@gobble#1{}%
\fi
\expandafter\ifx\csname @firstofone\endcsname\relax
  \long\def\@firstofone#1{#1}%
\fi
\expandafter\ifx\csname loop\endcsname\relax
  \expandafter\@firstofone
\else
  \expandafter\@gobble
\fi
{%
  \def\loop#1\repeat{%
    \def\body{#1}%
    \iterate
  }%
  \def\iterate{%
    \body
      \let\next\iterate
    \else
      \let\next\relax
    \fi
    \next
  }%
  \let\repeat=\fi
}%
\def\RestoreCatcodes{}
\count@=0 %
\loop
  \edef\RestoreCatcodes{%
    \RestoreCatcodes
    \catcode\the\count@=\the\catcode\count@\relax
  }%
\ifnum\count@<255 %
  \advance\count@ 1 %
\repeat

\def\RangeCatcodeInvalid#1#2{%
  \count@=#1\relax
  \loop
    \catcode\count@=15 %
  \ifnum\count@<#2\relax
    \advance\count@ 1 %
  \repeat
}
\def\RangeCatcodeCheck#1#2#3{%
  \count@=#1\relax
  \loop
    \ifnum#3=\catcode\count@
    \else
      \errmessage{%
        Character \the\count@\space
        with wrong catcode \the\catcode\count@\space
        instead of \number#3%
      }%
    \fi
  \ifnum\count@<#2\relax
    \advance\count@ 1 %
  \repeat
}
\def\space{ }
\expandafter\ifx\csname LoadCommand\endcsname\relax
  \def\LoadCommand{\input pdfescape.sty\relax}%
\fi
\def\Test{%
  \RangeCatcodeInvalid{0}{47}%
  \RangeCatcodeInvalid{58}{64}%
  \RangeCatcodeInvalid{91}{96}%
  \RangeCatcodeInvalid{123}{255}%
  \catcode`\@=12 %
  \catcode`\\=0 %
  \catcode`\%=14 %
  \LoadCommand
  \RangeCatcodeCheck{0}{36}{15}%
  \RangeCatcodeCheck{37}{37}{14}%
  \RangeCatcodeCheck{38}{47}{15}%
  \RangeCatcodeCheck{48}{57}{12}%
  \RangeCatcodeCheck{58}{63}{15}%
  \RangeCatcodeCheck{64}{64}{12}%
  \RangeCatcodeCheck{65}{90}{11}%
  \RangeCatcodeCheck{91}{91}{15}%
  \RangeCatcodeCheck{92}{92}{0}%
  \RangeCatcodeCheck{93}{96}{15}%
  \RangeCatcodeCheck{97}{122}{11}%
  \RangeCatcodeCheck{123}{255}{15}%
  \RestoreCatcodes
}
\Test
\csname @@end\endcsname
\end
%    \end{macrocode}
%    \begin{macrocode}
%</test1>
%    \end{macrocode}
%
% \subsection{Macro tests}
%
%    \begin{macrocode}
%<*test2|test3|test4|test5>
\NeedsTeXFormat{LaTeX2e}
\makeatletter
%    \end{macrocode}
%
% \subsection{Test with \cs{pdfescape...} commands}
%
%    \begin{macrocode}
%<*test2>
\ProvidesFile{pdfescape-test2.tex}%
    [2011/11/25 v1.13 Test with \string\pdfescape... commands]%
%</test2>
%    \end{macrocode}
%
% \subsection{Test without \cs{pdfescape...}, with \eTeX}
%
%    \begin{macrocode}
%<*test3>
\ProvidesFile{pdfescape-test3.tex}%
    [2011/11/25 v1.13 Test without \string\pdfescape..., with e-TeX]%
%</test3>
%    \end{macrocode}
%
% \subsection{Test without \cs{pdfescape...} and \eTeX}
%
%    \begin{macrocode}
%<*test4>
\ProvidesFile{pdfescape-test4.tex}%
    [2011/11/25 v1.13 Test without \string\pdfescape... and e-TeX]%
%</test4>
%    \end{macrocode}
%
% \subsection{Test with \LuaTeX}
%
%    \begin{macrocode}
%<*test5>
\ProvidesFile{pdfescape-test5.tex}%
    [2011/11/25 v1.13 Test with LuaTeX]%
%</test5>
%    \end{macrocode}
%
% \subsection{Check/ensure test preconditions}
%
% \subsubsection{Check \cs{pdfescape...}}
%
%    \begin{macrocode}
%<*test2>
\@ifundefined{pdfescapehex}{%
  \PackageError{pdfescape-test2}{%
    Missing \string\pdfescape... commands%
  }{Test aborted.}%
  \stop
}{}
%</test2>
%    \end{macrocode}
%    \begin{macrocode}
%<*test3|test4>
\let\pdfescapehex\@undefined
\let\pdfunescapehex\@undefined
\let\pdfescapename\@undefined
\let\pdfescapestring\@undefined
%</test3|test4>
%    \end{macrocode}
%
% \subsubsection{Check \eTeX}
%
%    \begin{macrocode}
%<*test3>
\@ifundefined{numexpr}{%
  \PackageError{pdfescape-test3}{%
    Missing \eTeX
  }{Test aborted.}%
  \stop
}{}
%</test3>
%    \end{macrocode}
% Package \xpackage{qstest} uses \eTeX, thus \eTeX's features
% can only be disabled later during loading of package \xpackage{pdfescape}.
%
% \subsubsection{Check \LuaTeX}
%
%    \begin{macrocode}
%<*test5>
\@ifundefined{directlua}{%
  \PackageError{pdfescape-test5}{%
    Missing LuaTeX%
  }{Test aborted.}%
  \stop
}{}
%</test5>
%    \end{macrocode}
%
% \subsection{Common part}
%
%    The files for testing uses the framework, provided by the
%    new package \xpackage{qstest} of David Kastrup.
%    \begin{macrocode}
\RequirePackage{qstest}
\IncludeTests{*}
\LogTests{log}{*}{*}

\newcommand*{\ExpectVar}[2]{%
  \ifx#1#2%
  \else
    \begingroup
      \@onelevel@sanitize#1%
      \@onelevel@sanitize#2%
      \typeout{[#1] <> [#2]}% hash-ok
    \endgroup
  \fi
  \Expect*{\ifx#1#2true\else false\fi}{true}%
}

\makeatletter
\begingroup
  \gdef\AllBytes{}%
  \count@=0 %
  \catcode0=12 %
  \@whilenum\count@<256 \do{%
    \lccode0=\count@
    \ifnum\count@=32 %
      \xdef\AllBytes{\AllBytes\space}%
    \else
      \lowercase{%
        \xdef\AllBytes{\AllBytes^^@}%
      }%
    \fi
    \advance\count@ by 1 %
  }%
\endgroup
\newcommand*{\AllBytesHex}{%
  000102030405060708090A0B0C0D0E0F%
  101112131415161718191A1B1C1D1E1F%
  202122232425262728292A2B2C2D2E2F%
  303132333435363738393A3B3C3D3E3F%
  404142434445464748494A4B4C4D4E4F%
  505152535455565758595A5B5C5D5E5F%
  606162636465666768696A6B6C6D6E6F%
  707172737475767778797A7B7C7D7E7F%
  808182838485868788898A8B8C8D8E8F%
  909192939495969798999A9B9C9D9E9F%
  A0A1A2A3A4A5A6A7A8A9AAABACADAEAF%
  B0B1B2B3B4B5B6B7B8B9BABBBCBDBEBF%
  C0C1C2C3C4C5C6C7C8C9CACBCCCDCECF%
  D0D1D2D3D4D5D6D7D8D9DADBDCDDDEDF%
  E0E1E2E3E4E5E6E7E8E9EAEBECEDEEEF%
  F0F1F2F3F4F5F6F7F8F9FAFBFCFDFEFF%
}
\@onelevel@sanitize\AllBytesHex
\expandafter\lowercase\expandafter{%
  \expandafter\newcommand\expandafter*\expandafter\AllBytesHexLC
      \expandafter{\AllBytesHex}%
}
\newcommand*{\AllBytesName}{}
\begingroup
  \catcode`\#=12 %
  \xdef\AllBytesName{%
    #01#02#03#04#05#06#07#08#09#0A#0B#0C#0D#0E#0F%
    #10#11#12#13#14#15#16#17#18#19#1A#1B#1C#1D#1E#1F%
    #20!"#23$#25&'#28#29*+,-.#2F%
    0123456789:;#3C=#3E?%
    @ABCDEFGHIJKLMNO%
    PQRSTUVWXYZ#5B\@backslashchar#5D^_%
    `abcdefghijklmno%
    pqrstuvwxyz#7B|#7D\string~#7F%
    #80#81#82#83#84#85#86#87#88#89#8A#8B#8C#8D#8E#8F%
    #90#91#92#93#94#95#96#97#98#99#9A#9B#9C#9D#9E#9F%
    #A0#A1#A2#A3#A4#A5#A6#A7#A8#A9#AA#AB#AC#AD#AE#AF%
    #B0#B1#B2#B3#B4#B5#B6#B7#B8#B9#BA#BB#BC#BD#BE#BF%
    #C0#C1#C2#C3#C4#C5#C6#C7#C8#C9#CA#CB#CC#CD#CE#CF%
    #D0#D1#D2#D3#D4#D5#D6#D7#D8#D9#DA#DB#DC#DD#DE#DF%
    #E0#E1#E2#E3#E4#E5#E6#E7#E8#E9#EA#EB#EC#ED#EE#EF%
    #F0#F1#F2#F3#F4#F5#F6#F7#F8#F9#FA#FB#FC#FD#FE#FF%
  }%
\endgroup
\@onelevel@sanitize\AllBytesName

\newcommand*{\AllBytesString}{}
\begingroup
  \def\|{|}%
  \edef\%{\@percentchar}%
  \catcode`\|=0 %
  \catcode`\#=12 %
  \catcode`\~=12 %
  \catcode`\\=12 %
  |xdef|AllBytesString{%
    \000\001\002\003\004\005\006\007\010\011\012\013\014\015\016\017%
    \020\021\022\023\024\025\026\027\030\031\032\033\034\035\036\037%
    \040!"#$|%&'\(\)*+,-./%
    0123456789:;<=>?%
    @ABCDEFGHIJKLMNO%
    PQRSTUVWXYZ[\\]^_%
    `abcdefghijklmno%
    pqrstuvwxyz{||}~\177%
    \200\201\202\203\204\205\206\207\210\211\212\213\214\215\216\217%
    \220\221\222\223\224\225\226\227\230\231\232\233\234\235\236\237%
    \240\241\242\243\244\245\246\247\250\251\252\253\254\255\256\257%
    \260\261\262\263\264\265\266\267\270\271\272\273\274\275\276\277%
    \300\301\302\303\304\305\306\307\310\311\312\313\314\315\316\317%
    \320\321\322\323\324\325\326\327\330\331\332\333\334\335\336\337%
    \340\341\342\343\344\345\346\347\350\351\352\353\354\355\356\357%
    \360\361\362\363\364\365\366\367\370\371\372\373\374\375\376\377%
  }%
|endgroup
\@onelevel@sanitize\AllBytesString

%<*test4>
\let\org@detokenize\detokenize
\let\detokenize\@undefined
\let\org@numexpr\numexpr
\let\numexpr\@undefined
%</test4>
\RequirePackage{pdfescape}
%<*test4>
\let\detokenize\org@detokenize
\let\numexpr\org@numexpr
%</test4>

\begin{qstest}{all-hex}{\AllBytes, escapehex}
  \EdefEscapeHex\x{\AllBytes}%
  \Expect*{\x}*{\AllBytesHex}%
  \ExpectVar\x\AllBytesHex
\end{qstest}

\begin{qstest}{all-unhex}{\AllBytesHex, unescapehex}
  \EdefUnescapeHex\x{\AllBytesHex}%
  \Expect*{\x}*{\AllBytes}%
  \ExpectVar\x\AllBytes
\end{qstest}

\begin{qstest}{all-unhex-lc}{\AllBytesHexLC, unescapehex, lowercase}
  \EdefUnescapeHex\x{\AllBytesHexLC}%
  \Expect*{\x}*{\AllBytes}%
  \ExpectVar\x\AllBytes
\end{qstest}

\begin{qstest}{unhex-incomplete}{unescapehex, incomplete}
  \EdefUnescapeHex\x{4}%
  \Expect*{\x}{@}%
\end{qstest}

\begin{qstest}{unhex-space}{unescapehex, space}
  \EdefUnescapeHex\x{20}%
  \Expect*{\x}{ }%
  \ExpectVar\x\space
\end{qstest}

\begin{qstest}{unhex-spaces}{unescapehex, spaces}
  \EdefUnescapeHex\x{204020204120}%
  \def\y#1{%
    \edef\z{#1\string @#1#1\string A#1}%
  }\y{ }%
  \Expect*{\x}*{\z}%
  \ExpectVar\x\z
\end{qstest}

\begin{qstest}{unhex-hash}{unescapehex, hash}
  \catcode`\#=12 %
  \EdefUnescapeHex\x{#20}%
  \ExpectVar\x\space
\end{qstest}

\begin{qstest}{unhex-invalid}{unescapehex, invalid}
  \def\test#1#2{%
    \EdefUnescapeHex\x{#1}%
    \edef\y{#2}%
    \@onelevel@sanitize\y
    \ExpectVar\x\y
  }%
%<*test2>
  \edef\x{\pdfunescapehex{4X}}%
  \edef\y{\string @}%
  \ifx\x\y
  \else
    \def~{\space}%
    \typeout{************************************}%
    \typeout{* Your pdfTeX contains bug 777.~~~~*}%
    \typeout{* This test is redefined as dummy, *}%
    \typeout{* because it triggers the bug.~~~~~*}%
    \typeout{************************************}%
    \def\test#1#2{}%
  \fi
%</test2>
  \test{X}{}%
  \test{XY}{}%
  \test{XYZ}{}%
  \test{A}{^^a0}%
  \test{AX}{^^a0}%
  \test{XA}{^^a0}%
  \test{XXAXX}{^^a0}%
\end{qstest}

\begin{qstest}{all-name}{\AllBytes, escapename}
  \EdefEscapeName\x{\AllBytes}%
  \Expect*{\x}*{\AllBytesName}%
  \ExpectVar\x\AllBytesName
\end{qstest}

\begin{qstest}{all-string}{\AllBytes, escapestring}
  \EdefEscapeString\x{\AllBytes}%
  \Expect*{\x}*{\AllBytesString}%
  \ExpectVar\x\AllBytesString
\end{qstest}

\begin{qstest}{uchexdigit}{unescape, uppercase hex digit}
  \catcode`\@=11 %
  \catcode0=12 %
  \def\test#1#2{%
    \uccode0=#1\relax
    \uppercase{%
      \def\x{^^@}%
    }%
    \Expect*{%
       \ifcase\expandafter\PE@TestUcHexDigit\x
         true%
       \else
         false%
       \fi
    }{#2}%
  }%
  \def\range#1#2#3{%
    \count0=#1\relax
    \loop
    \ifnum\count0<#2\relax
      \test{\count0}{#3}%
      \advance\count0 by 1 %
    \repeat
  }%
  \range{0}{47}{false}%
  \range{48}{57}{true}%
  \range{58}{64}{false}%
  \range{65}{70}{true}%
  \range{71}{255}{false}%
\end{qstest}

\begin{qstest}{unescapename}{unescapename}
  \def\test#1#2{%
    \EdefUnescapeName\x{#1}%
    \edef\y{#2}%
    \@onelevel@sanitize\y
    \ExpectVar\x\y
  }%
  \catcode`\#=12 %
  \catcode0=12 %
  \test{}{}%
  \test{x}{x}%
  \test{xy}{xy}%
  \test{#}{#}%
  \test{##}{##}%
  \test{###}{###}%
  \test{####}{####}%
  \test{#x}{#x}%
  \test{#xy}{#xy}%
  \test{#1}{#1}%
  \test{#40}{@}%
  \test{#400}{@0}%
  \test{#4x0}{#4x0}%
  \test{#ab}{^^ab}%
  \test{#00}{^^@}%
  \test{x#40y#40z}{x@y@z}%
  \test{#40#40#40#40}{@@@@}%
  \test{a#x}{a#x}%
  \test{a#xy}{a#xy}%
  \test{a#1}{a#1}%
  \test{a#40}{a@}%
  \test{a#400}{a@0}%
  \test{#20}{ }%
  \test{a#20}{a }%
  \test{a#20b}{a b}%
  \test{a#20#20#20b}{a \space\space b}%
\end{qstest}

\begin{qstest}{unescapestring}{unescapestring}
  \def\test#1#2{%
    \EdefUnescapeString\x{#1}%
    \edef\y{#2}%
    \@onelevel@sanitize\y
    \ExpectVar\x\y
  }%
  \catcode0=12 %
  \def\DefChar#1#2{%
    \begingroup
      \uccode0=#2\relax
    \uppercase{\endgroup
      \def#1{^^@}%
    }%
  }%
  \DefChar\nul{0}%
  \DefChar\one{1}%
  \DefChar\bel{8}%
  \DefChar\tab{9}%
  \DefChar\lf{10}%
  \DefChar\ff{12}%
  \DefChar\cr{13}%
  \DefChar\\{92}%
  \test{}{}%
  \test{a}{a}%
  \test{\\}{}%
  \test{\\\\}{\\}%
  \test{\\\\y}{\\y}%
  \test{\\000}{\nul}%
  \test{\\b}{\bel}%
  \test{\\t}{\tab}%
  \test{\\n}{\lf}%
  \test{\\f}{\ff}%
  \test{\\r}{\cr}%
  \test{\\(}{(}%
  \test{\\)}{)}%
  \test{\\040}{ }%
  \test{\\100}{@}%
  \test{\\40}{ }%
  \test{\\1}{\one}%
  \test{\\01}{\one}%
  \test{\\001}{\one}%
  \test{\\18}{\one8}%
  \test{\\018}{\one8}%
  \test{\\0018}{\one8}%
  \test{x\\}{x}%
  \test{x\\\\}{x\\}%
  \test{x\\\\y}{x\\y}%
  \test{x\\000}{x\nul}%
  \test{x\\b}{x\bel}%
  \test{x\\t}{x\tab}%
  \test{x\\n}{x\lf}%
  \test{x\\f}{x\ff}%
  \test{x\\r}{x\cr}%
  \test{x\\(}{x(}%
  \test{x\\)}{x)}%
  \test{x\\040}{x }%
  \test{x\\100}{x@}%
  \test{x\\40}{x }%
  \test{x\\1}{x\one}%
  \test{x\\01}{x\one}%
  \test{x\\001}{x\one}%
  \test{x\\18}{x\one8}%
  \test{x\\018}{x\one8}%
  \test{x\\0018}{x\one8}%
  \test{\\b\\t\\n\\f\\r\\(\\)\\\\\\000\\040}{%
    \bel\tab\lf\ff\cr()\\\nul\space
  }%
  \test{\\\lf}{}%
  \test{x\\\lf}{x}%
  \test{\cr}{\lf}%
  \test{\cr\lf}{\lf}%
  \test{\lf}{\lf}%
  \test{\lf\cr}{\lf\lf}%
  \test{x\cr}{x\lf}%
  \test{x\cr\lf}{x\lf}%
  \test{x\lf}{x\lf}%
  \test{x\lf\cr}{x\lf\lf}%
  \test{x\\\cr\lf y\cr}{xy\lf}%
%
  \test{\\409}{ 9}%
  \test{\\800}{800}%
  \test{\\900}{900}%
  \test{\\578}{/8}%
  \test{\\477}{?}%
  \test{\\377}{^^ff}%
  \test{\\777}{^^ff}%
  \test{\\7777}{^^ff7}%
\end{qstest}
\stop
%    \end{macrocode}
%
%    \begin{macrocode}
%</test2|test3|test4|test5>
%    \end{macrocode}
%
% \subsubsection{Test for \hologo{iniTeX}}
%
%    \begin{macrocode}
%<*test6>
\input pdfescape.sty\relax
\catcode`\{=1 %
\catcode`\}=2 %
\catcode`\#=6 %
\catcode`\^=7 %
\catcode`\@=11 %
%    \end{macrocode}
%    \begin{macrocode}
\begingroup
  \catcode`\@=11 %
  \countdef\count@=255 %
  \def\space{ }%
  \long\def\@whilenum#1\do #2{%
    \ifnum #1\relax
      #2\relax
      \@iwhilenum{#1\relax#2\relax}%
    \fi
  }%
  \long\def\@iwhilenum#1{%
    \ifnum #1%
      \expandafter\@iwhilenum
    \else
      \expandafter\ltx@gobble
    \fi
    {#1}%
  }%
  \gdef\AllBytes{}%
  \count@=0 %
  \catcode0=12 %
  \@whilenum\count@<256 \do{%
    \lccode0=\count@
    \ifnum\count@=32 %
      \xdef\AllBytes{\AllBytes\space}%
    \else
      \lowercase{%
        \xdef\AllBytes{\AllBytes^^@}%
      }%
    \fi
    \advance\count@ by 1 %
  }%
\endgroup
\def\AllBytesHex{%
  000102030405060708090A0B0C0D0E0F%
  101112131415161718191A1B1C1D1E1F%
  202122232425262728292A2B2C2D2E2F%
  303132333435363738393A3B3C3D3E3F%
  404142434445464748494A4B4C4D4E4F%
  505152535455565758595A5B5C5D5E5F%
  606162636465666768696A6B6C6D6E6F%
  707172737475767778797A7B7C7D7E7F%
  808182838485868788898A8B8C8D8E8F%
  909192939495969798999A9B9C9D9E9F%
  A0A1A2A3A4A5A6A7A8A9AAABACADAEAF%
  B0B1B2B3B4B5B6B7B8B9BABBBCBDBEBF%
  C0C1C2C3C4C5C6C7C8C9CACBCCCDCECF%
  D0D1D2D3D4D5D6D7D8D9DADBDCDDDEDF%
  E0E1E2E3E4E5E6E7E8E9EAEBECEDEEEF%
  F0F1F2F3F4F5F6F7F8F9FAFBFCFDFEFF%
}
\ltx@onelevel@sanitize\AllBytesHex
\expandafter\lowercase\expandafter{%
  \expandafter\def\expandafter\AllBytesHexLC
      \expandafter{\AllBytesHex}%
}
\begingroup
  \catcode`\#=12 %
  \xdef\AllBytesName{%
    #01#02#03#04#05#06#07#08#09#0A#0B#0C#0D#0E#0F%
    #10#11#12#13#14#15#16#17#18#19#1A#1B#1C#1D#1E#1F%
    #20!"#23$#25&'#28#29*+,-.#2F%
    0123456789:;#3C=#3E?%
    @ABCDEFGHIJKLMNO%
    PQRSTUVWXYZ#5B\ltx@backslashchar#5D^_%
    `abcdefghijklmno%
    pqrstuvwxyz#7B|#7D\string~#7F%
    #80#81#82#83#84#85#86#87#88#89#8A#8B#8C#8D#8E#8F%
    #90#91#92#93#94#95#96#97#98#99#9A#9B#9C#9D#9E#9F%
    #A0#A1#A2#A3#A4#A5#A6#A7#A8#A9#AA#AB#AC#AD#AE#AF%
    #B0#B1#B2#B3#B4#B5#B6#B7#B8#B9#BA#BB#BC#BD#BE#BF%
    #C0#C1#C2#C3#C4#C5#C6#C7#C8#C9#CA#CB#CC#CD#CE#CF%
    #D0#D1#D2#D3#D4#D5#D6#D7#D8#D9#DA#DB#DC#DD#DE#DF%
    #E0#E1#E2#E3#E4#E5#E6#E7#E8#E9#EA#EB#EC#ED#EE#EF%
    #F0#F1#F2#F3#F4#F5#F6#F7#F8#F9#FA#FB#FC#FD#FE#FF%
  }%
\endgroup
\ltx@onelevel@sanitize\AllBytesName
\edef\AllBytesFromName{\expandafter\ltx@gobble\AllBytes}

\begingroup
  \def\|{|}%
  \edef\%{\ltx@percentchar}%
  \catcode`\|=0 %
  \catcode`\#=12 %
  \catcode`\~=12 %
  \catcode`\\=12 %
  |xdef|AllBytesString{%
    \000\001\002\003\004\005\006\007\010\011\012\013\014\015\016\017%
    \020\021\022\023\024\025\026\027\030\031\032\033\034\035\036\037%
    \040!"#$|%&'\(\)*+,-./%
    0123456789:;<=>?%
    @ABCDEFGHIJKLMNO%
    PQRSTUVWXYZ[\\]^_%
    `abcdefghijklmno%
    pqrstuvwxyz{||}~\177%
    \200\201\202\203\204\205\206\207\210\211\212\213\214\215\216\217%
    \220\221\222\223\224\225\226\227\230\231\232\233\234\235\236\237%
    \240\241\242\243\244\245\246\247\250\251\252\253\254\255\256\257%
    \260\261\262\263\264\265\266\267\270\271\272\273\274\275\276\277%
    \300\301\302\303\304\305\306\307\310\311\312\313\314\315\316\317%
    \320\321\322\323\324\325\326\327\330\331\332\333\334\335\336\337%
    \340\341\342\343\344\345\346\347\350\351\352\353\354\355\356\357%
    \360\361\362\363\364\365\366\367\370\371\372\373\374\375\376\377%
  }%
|endgroup
\ltx@onelevel@sanitize\AllBytesString
%    \end{macrocode}
%    \begin{macrocode}
\def\msg#{\immediate\write16}
\def\Test#1#2#3{%
  \begingroup
    #1\TestResult{#2}%
    \ifx\TestResult#3%
    \else
      \newlinechar=10 %
      \msg{Expect:^^J#3}%
      \msg{Result:^^J\TestResult}%
      \errmessage{\string#2 -\string#1-> \string#3}%
    \fi
  \endgroup
}
\Test\EdefEscapeHex\AllBytes\AllBytesHex
\Test\EdefUnescapeHex\AllBytesHex\AllBytes
\Test\EdefEscapeName\AllBytes\AllBytesName
\Test\EdefUnescapeName\AllBytesName\AllBytesFromName
\Test\EdefEscapeString\AllBytes\AllBytesString
\Test\EdefUnescapeString\AllBytesString\AllBytes
%    \end{macrocode}
%    \begin{macrocode}
\csname @@end\endcsname\end
%</test6>
%    \end{macrocode}
%
% \section{Installation}
%
% \subsection{Download}
%
% \paragraph{Package.} This package is available on
% CTAN\footnote{\url{ftp://ftp.ctan.org/tex-archive/}}:
% \begin{description}
% \item[\CTAN{macros/latex/contrib/oberdiek/pdfescape.dtx}] The source file.
% \item[\CTAN{macros/latex/contrib/oberdiek/pdfescape.pdf}] Documentation.
% \end{description}
%
%
% \paragraph{Bundle.} All the packages of the bundle `oberdiek'
% are also available in a TDS compliant ZIP archive. There
% the packages are already unpacked and the documentation files
% are generated. The files and directories obey the TDS standard.
% \begin{description}
% \item[\CTAN{install/macros/latex/contrib/oberdiek.tds.zip}]
% \end{description}
% \emph{TDS} refers to the standard ``A Directory Structure
% for \TeX\ Files'' (\CTAN{tds/tds.pdf}). Directories
% with \xfile{texmf} in their name are usually organized this way.
%
% \subsection{Bundle installation}
%
% \paragraph{Unpacking.} Unpack the \xfile{oberdiek.tds.zip} in the
% TDS tree (also known as \xfile{texmf} tree) of your choice.
% Example (linux):
% \begin{quote}
%   |unzip oberdiek.tds.zip -d ~/texmf|
% \end{quote}
%
% \paragraph{Script installation.}
% Check the directory \xfile{TDS:scripts/oberdiek/} for
% scripts that need further installation steps.
% Package \xpackage{attachfile2} comes with the Perl script
% \xfile{pdfatfi.pl} that should be installed in such a way
% that it can be called as \texttt{pdfatfi}.
% Example (linux):
% \begin{quote}
%   |chmod +x scripts/oberdiek/pdfatfi.pl|\\
%   |cp scripts/oberdiek/pdfatfi.pl /usr/local/bin/|
% \end{quote}
%
% \subsection{Package installation}
%
% \paragraph{Unpacking.} The \xfile{.dtx} file is a self-extracting
% \docstrip\ archive. The files are extracted by running the
% \xfile{.dtx} through \plainTeX:
% \begin{quote}
%   \verb|tex pdfescape.dtx|
% \end{quote}
%
% \paragraph{TDS.} Now the different files must be moved into
% the different directories in your installation TDS tree
% (also known as \xfile{texmf} tree):
% \begin{quote}
% \def\t{^^A
% \begin{tabular}{@{}>{\ttfamily}l@{ $\rightarrow$ }>{\ttfamily}l@{}}
%   pdfescape.sty & tex/generic/oberdiek/pdfescape.sty\\
%   pdfescape.pdf & doc/latex/oberdiek/pdfescape.pdf\\
%   test/pdfescape-test1.tex & doc/latex/oberdiek/test/pdfescape-test1.tex\\
%   test/pdfescape-test2.tex & doc/latex/oberdiek/test/pdfescape-test2.tex\\
%   test/pdfescape-test3.tex & doc/latex/oberdiek/test/pdfescape-test3.tex\\
%   test/pdfescape-test4.tex & doc/latex/oberdiek/test/pdfescape-test4.tex\\
%   test/pdfescape-test5.tex & doc/latex/oberdiek/test/pdfescape-test5.tex\\
%   test/pdfescape-test6.tex & doc/latex/oberdiek/test/pdfescape-test6.tex\\
%   pdfescape.dtx & source/latex/oberdiek/pdfescape.dtx\\
% \end{tabular}^^A
% }^^A
% \sbox0{\t}^^A
% \ifdim\wd0>\linewidth
%   \begingroup
%     \advance\linewidth by\leftmargin
%     \advance\linewidth by\rightmargin
%   \edef\x{\endgroup
%     \def\noexpand\lw{\the\linewidth}^^A
%   }\x
%   \def\lwbox{^^A
%     \leavevmode
%     \hbox to \linewidth{^^A
%       \kern-\leftmargin\relax
%       \hss
%       \usebox0
%       \hss
%       \kern-\rightmargin\relax
%     }^^A
%   }^^A
%   \ifdim\wd0>\lw
%     \sbox0{\small\t}^^A
%     \ifdim\wd0>\linewidth
%       \ifdim\wd0>\lw
%         \sbox0{\footnotesize\t}^^A
%         \ifdim\wd0>\linewidth
%           \ifdim\wd0>\lw
%             \sbox0{\scriptsize\t}^^A
%             \ifdim\wd0>\linewidth
%               \ifdim\wd0>\lw
%                 \sbox0{\tiny\t}^^A
%                 \ifdim\wd0>\linewidth
%                   \lwbox
%                 \else
%                   \usebox0
%                 \fi
%               \else
%                 \lwbox
%               \fi
%             \else
%               \usebox0
%             \fi
%           \else
%             \lwbox
%           \fi
%         \else
%           \usebox0
%         \fi
%       \else
%         \lwbox
%       \fi
%     \else
%       \usebox0
%     \fi
%   \else
%     \lwbox
%   \fi
% \else
%   \usebox0
% \fi
% \end{quote}
% If you have a \xfile{docstrip.cfg} that configures and enables \docstrip's
% TDS installing feature, then some files can already be in the right
% place, see the documentation of \docstrip.
%
% \subsection{Refresh file name databases}
%
% If your \TeX~distribution
% (\teTeX, \mikTeX, \dots) relies on file name databases, you must refresh
% these. For example, \teTeX\ users run \verb|texhash| or
% \verb|mktexlsr|.
%
% \subsection{Some details for the interested}
%
% \paragraph{Attached source.}
%
% The PDF documentation on CTAN also includes the
% \xfile{.dtx} source file. It can be extracted by
% AcrobatReader 6 or higher. Another option is \textsf{pdftk},
% e.g. unpack the file into the current directory:
% \begin{quote}
%   \verb|pdftk pdfescape.pdf unpack_files output .|
% \end{quote}
%
% \paragraph{Unpacking with \LaTeX.}
% The \xfile{.dtx} chooses its action depending on the format:
% \begin{description}
% \item[\plainTeX:] Run \docstrip\ and extract the files.
% \item[\LaTeX:] Generate the documentation.
% \end{description}
% If you insist on using \LaTeX\ for \docstrip\ (really,
% \docstrip\ does not need \LaTeX), then inform the autodetect routine
% about your intention:
% \begin{quote}
%   \verb|latex \let\install=y% \iffalse meta-comment
%
% File: pdfescape.dtx
% Version: 2011/11/25 v1.13
% Info: Implements pdfTeX's escape features
%
% Copyright (C) 2007, 2010, 2011 by
%    Heiko Oberdiek <heiko.oberdiek at googlemail.com>
%
% This work may be distributed and/or modified under the
% conditions of the LaTeX Project Public License, either
% version 1.3c of this license or (at your option) any later
% version. This version of this license is in
%    http://www.latex-project.org/lppl/lppl-1-3c.txt
% and the latest version of this license is in
%    http://www.latex-project.org/lppl.txt
% and version 1.3 or later is part of all distributions of
% LaTeX version 2005/12/01 or later.
%
% This work has the LPPL maintenance status "maintained".
%
% This Current Maintainer of this work is Heiko Oberdiek.
%
% The Base Interpreter refers to any `TeX-Format',
% because some files are installed in TDS:tex/generic//.
%
% This work consists of the main source file pdfescape.dtx
% and the derived files
%    pdfescape.sty, pdfescape.pdf, pdfescape.ins, pdfescape.drv,
%    pdfescape-test1.tex, pdfescape-test2.tex,
%    pdfescape-test3.tex, pdfescape-test4.tex,
%    pdfescape-test5.tex, pdfescape-test6.tex.
%
% Distribution:
%    CTAN:macros/latex/contrib/oberdiek/pdfescape.dtx
%    CTAN:macros/latex/contrib/oberdiek/pdfescape.pdf
%
% Unpacking:
%    (a) If pdfescape.ins is present:
%           tex pdfescape.ins
%    (b) Without pdfescape.ins:
%           tex pdfescape.dtx
%    (c) If you insist on using LaTeX
%           latex \let\install=y\input{pdfescape.dtx}
%        (quote the arguments according to the demands of your shell)
%
% Documentation:
%    (a) If pdfescape.drv is present:
%           latex pdfescape.drv
%    (b) Without pdfescape.drv:
%           latex pdfescape.dtx; ...
%    The class ltxdoc loads the configuration file ltxdoc.cfg
%    if available. Here you can specify further options, e.g.
%    use A4 as paper format:
%       \PassOptionsToClass{a4paper}{article}
%
%    Programm calls to get the documentation (example):
%       pdflatex pdfescape.dtx
%       makeindex -s gind.ist pdfescape.idx
%       pdflatex pdfescape.dtx
%       makeindex -s gind.ist pdfescape.idx
%       pdflatex pdfescape.dtx
%
% Installation:
%    TDS:tex/generic/oberdiek/pdfescape.sty
%    TDS:doc/latex/oberdiek/pdfescape.pdf
%    TDS:doc/latex/oberdiek/test/pdfescape-test1.tex
%    TDS:doc/latex/oberdiek/test/pdfescape-test2.tex
%    TDS:doc/latex/oberdiek/test/pdfescape-test3.tex
%    TDS:doc/latex/oberdiek/test/pdfescape-test4.tex
%    TDS:doc/latex/oberdiek/test/pdfescape-test5.tex
%    TDS:doc/latex/oberdiek/test/pdfescape-test6.tex
%    TDS:source/latex/oberdiek/pdfescape.dtx
%
%<*ignore>
\begingroup
  \catcode123=1 %
  \catcode125=2 %
  \def\x{LaTeX2e}%
\expandafter\endgroup
\ifcase 0\ifx\install y1\fi\expandafter
         \ifx\csname processbatchFile\endcsname\relax\else1\fi
         \ifx\fmtname\x\else 1\fi\relax
\else\csname fi\endcsname
%</ignore>
%<*install>
\input docstrip.tex
\Msg{************************************************************************}
\Msg{* Installation}
\Msg{* Package: pdfescape 2011/11/25 v1.13 Implements pdfTeX's escape features (HO)}
\Msg{************************************************************************}

\keepsilent
\askforoverwritefalse

\let\MetaPrefix\relax
\preamble

This is a generated file.

Project: pdfescape
Version: 2011/11/25 v1.13

Copyright (C) 2007, 2010, 2011 by
   Heiko Oberdiek <heiko.oberdiek at googlemail.com>

This work may be distributed and/or modified under the
conditions of the LaTeX Project Public License, either
version 1.3c of this license or (at your option) any later
version. This version of this license is in
   http://www.latex-project.org/lppl/lppl-1-3c.txt
and the latest version of this license is in
   http://www.latex-project.org/lppl.txt
and version 1.3 or later is part of all distributions of
LaTeX version 2005/12/01 or later.

This work has the LPPL maintenance status "maintained".

This Current Maintainer of this work is Heiko Oberdiek.

The Base Interpreter refers to any `TeX-Format',
because some files are installed in TDS:tex/generic//.

This work consists of the main source file pdfescape.dtx
and the derived files
   pdfescape.sty, pdfescape.pdf, pdfescape.ins, pdfescape.drv,
   pdfescape-test1.tex, pdfescape-test2.tex,
   pdfescape-test3.tex, pdfescape-test4.tex,
   pdfescape-test5.tex, pdfescape-test6.tex.

\endpreamble
\let\MetaPrefix\DoubleperCent

\generate{%
  \file{pdfescape.ins}{\from{pdfescape.dtx}{install}}%
  \file{pdfescape.drv}{\from{pdfescape.dtx}{driver}}%
  \usedir{tex/generic/oberdiek}%
  \file{pdfescape.sty}{\from{pdfescape.dtx}{package}}%
  \usedir{doc/latex/oberdiek/test}%
  \file{pdfescape-test1.tex}{\from{pdfescape.dtx}{test1}}%
  \file{pdfescape-test2.tex}{\from{pdfescape.dtx}{test2}}%
  \file{pdfescape-test3.tex}{\from{pdfescape.dtx}{test3}}%
  \file{pdfescape-test4.tex}{\from{pdfescape.dtx}{test4}}%
  \file{pdfescape-test5.tex}{\from{pdfescape.dtx}{test5}}%
  \file{pdfescape-test6.tex}{\from{pdfescape.dtx}{test6}}%
  \nopreamble
  \nopostamble
  \usedir{source/latex/oberdiek/catalogue}%
  \file{pdfescape.xml}{\from{pdfescape.dtx}{catalogue}}%
}

\catcode32=13\relax% active space
\let =\space%
\Msg{************************************************************************}
\Msg{*}
\Msg{* To finish the installation you have to move the following}
\Msg{* file into a directory searched by TeX:}
\Msg{*}
\Msg{*     pdfescape.sty}
\Msg{*}
\Msg{* To produce the documentation run the file `pdfescape.drv'}
\Msg{* through LaTeX.}
\Msg{*}
\Msg{* Happy TeXing!}
\Msg{*}
\Msg{************************************************************************}

\endbatchfile
%</install>
%<*ignore>
\fi
%</ignore>
%<*driver>
\NeedsTeXFormat{LaTeX2e}
\ProvidesFile{pdfescape.drv}%
  [2011/11/25 v1.13 Implements pdfTeX's escape features (HO)]%
\documentclass{ltxdoc}
\usepackage{holtxdoc}[2011/11/22]
\begin{document}
  \DocInput{pdfescape.dtx}%
\end{document}
%</driver>
% \fi
%
% \CheckSum{2516}
%
% \CharacterTable
%  {Upper-case    \A\B\C\D\E\F\G\H\I\J\K\L\M\N\O\P\Q\R\S\T\U\V\W\X\Y\Z
%   Lower-case    \a\b\c\d\e\f\g\h\i\j\k\l\m\n\o\p\q\r\s\t\u\v\w\x\y\z
%   Digits        \0\1\2\3\4\5\6\7\8\9
%   Exclamation   \!     Double quote  \"     Hash (number) \#
%   Dollar        \$     Percent       \%     Ampersand     \&
%   Acute accent  \'     Left paren    \(     Right paren   \)
%   Asterisk      \*     Plus          \+     Comma         \,
%   Minus         \-     Point         \.     Solidus       \/
%   Colon         \:     Semicolon     \;     Less than     \<
%   Equals        \=     Greater than  \>     Question mark \?
%   Commercial at \@     Left bracket  \[     Backslash     \\
%   Right bracket \]     Circumflex    \^     Underscore    \_
%   Grave accent  \`     Left brace    \{     Vertical bar  \|
%   Right brace   \}     Tilde         \~}
%
% \GetFileInfo{pdfescape.drv}
%
% \title{The \xpackage{pdfescape} package}
% \date{2011/11/25 v1.13}
% \author{Heiko Oberdiek\\\xemail{heiko.oberdiek at googlemail.com}}
%
% \maketitle
%
% \begin{abstract}
% This package implements \pdfTeX's escape features
% (\cs{pdfescapehex}, \cs{pdfunescapehex}, \cs{pdfescapename},
% \cs{pdfescapestring}) using \TeX\ or \eTeX.
% \end{abstract}
%
% \tableofcontents
%
% \section{Documentation}
%
% \begin{declcs}{EdefEscapeHex} \M{cmd} \M{string}\\
%  \cs{EdefUnescapeHex} \M{cmd} \M{string}\\
%  \cs{EdefEscapeName} \M{cmd} \M{string}\\
%  \cs{EdefEscapeString} \M{cmd} \M{string}
% \end{declcs}
% These commands converts \meta{string} and stores the result
% in macro \meta{cmd}. The conversion result is the same
% as the conversion of the corresponding \pdfTeX's primitives.
% Note that the argument \meta{string} is expanded before
% the conversion.
%
% For example, if \pdfTeX\ \textgreater= 1.30 is present,
% then \cs{EdefEscapeHex}
% becomes to:
% \begin{quote}
%\begin{verbatim}
%\def\EdefEscapeHex#1#2{%
%  \edef#1{\pdfescapehex{#2}}%
%}
%\end{verbatim}
% \end{quote}
% The package provides implementations for the case that \pdfTeX\ is
% not present (or too old). Even \eTeX\ can be missing, however it is
% used if it is detected.
%
% \paragraph{Babel.}
% The input strings may contain shorthand characters of package
% \xpackage{babel}.
%
% \subsection{Additional unescape macros}
%
% \begin{declcs}{EdefUnescapeName} \M{cmd} \M{string}
% \end{declcs}
% Sequences of a hash sign with two hexadecimal digits are
% converted to the corresponding character (PDF-1.2).
% A hash sign that is not followed by two hexadecimal digits
% is left unchanged. The catcodes in the result string follow
% TeX's conventions. The space has catcode 10 (space) and
% the other characters have catcode 12 (other).
%
% \begin{declcs}{EdefUnescapeString} \M{cmd} \M{string}
% \end{declcs}
% Macro \meta{cmd} stores the unescaped string in \meta{string}.
% All the rules for literal strings are implemented, see
% PDF specification.
% The catcodes in the result string follow TeX's conventions.
%
% \subsection{Sanitizing macro}
%
% \begin{declcs}{EdefSanitize} \M{cmd} \M{string}
% \end{declcs}
% Argument \meta{string} is expanded, converted to a string
% of tokens with catcode 12 (other) and space tokens, and
% stored in macro \meta{cmd}.
%
% \StopEventually{
% }
%
% \section{Implementation}
%
%    \begin{macrocode}
%<*package>
%    \end{macrocode}
%
% \subsection{Reload check and package identification}
%    Reload check, especially if the package is not used with \LaTeX.
%    \begin{macrocode}
\begingroup\catcode61\catcode48\catcode32=10\relax%
  \catcode13=5 % ^^M
  \endlinechar=13 %
  \catcode35=6 % #
  \catcode39=12 % '
  \catcode44=12 % ,
  \catcode45=12 % -
  \catcode46=12 % .
  \catcode58=12 % :
  \catcode64=11 % @
  \catcode123=1 % {
  \catcode125=2 % }
  \expandafter\let\expandafter\x\csname ver@pdfescape.sty\endcsname
  \ifx\x\relax % plain-TeX, first loading
  \else
    \def\empty{}%
    \ifx\x\empty % LaTeX, first loading,
      % variable is initialized, but \ProvidesPackage not yet seen
    \else
      \expandafter\ifx\csname PackageInfo\endcsname\relax
        \def\x#1#2{%
          \immediate\write-1{Package #1 Info: #2.}%
        }%
      \else
        \def\x#1#2{\PackageInfo{#1}{#2, stopped}}%
      \fi
      \x{pdfescape}{The package is already loaded}%
      \aftergroup\endinput
    \fi
  \fi
\endgroup%
%    \end{macrocode}
%    Package identification:
%    \begin{macrocode}
\begingroup\catcode61\catcode48\catcode32=10\relax%
  \catcode13=5 % ^^M
  \endlinechar=13 %
  \catcode35=6 % #
  \catcode39=12 % '
  \catcode40=12 % (
  \catcode41=12 % )
  \catcode44=12 % ,
  \catcode45=12 % -
  \catcode46=12 % .
  \catcode47=12 % /
  \catcode58=12 % :
  \catcode64=11 % @
  \catcode91=12 % [
  \catcode93=12 % ]
  \catcode123=1 % {
  \catcode125=2 % }
  \expandafter\ifx\csname ProvidesPackage\endcsname\relax
    \def\x#1#2#3[#4]{\endgroup
      \immediate\write-1{Package: #3 #4}%
      \xdef#1{#4}%
    }%
  \else
    \def\x#1#2[#3]{\endgroup
      #2[{#3}]%
      \ifx#1\@undefined
        \xdef#1{#3}%
      \fi
      \ifx#1\relax
        \xdef#1{#3}%
      \fi
    }%
  \fi
\expandafter\x\csname ver@pdfescape.sty\endcsname
\ProvidesPackage{pdfescape}%
  [2011/11/25 v1.13 Implements pdfTeX's escape features (HO)]%
%    \end{macrocode}
%
% \subsection{Catcodes}
%
%    \begin{macrocode}
\begingroup\catcode61\catcode48\catcode32=10\relax%
  \catcode13=5 % ^^M
  \endlinechar=13 %
  \catcode123 1 % {
  \catcode125 2 % }
  \catcode64 11 %
  \def\x{\endgroup
    \expandafter\edef\csname PE@AtEnd\endcsname{%
      \endlinechar=\the\endlinechar\relax
      \catcode13=\the\catcode13\relax
      \catcode32=\the\catcode32\relax
      \catcode35=\the\catcode35\relax
      \catcode61=\the\catcode61\relax
      \catcode64=\the\catcode64\relax
      \catcode123=\the\catcode123\relax
      \catcode125=\the\catcode125\relax
    }%
  }%
\x\catcode61\catcode48\catcode32=10\relax%
\catcode13=5 % ^^M
\endlinechar=13 %
\catcode35=6 % #
\catcode64=11 % @
\catcode123=1 % {
\catcode125=2 % }
\def\TMP@EnsureCode#1#2#3{%
  \edef\PE@AtEnd{%
    \PE@AtEnd
    #1#2=\the#1#2\relax
  }%
  #1#2=#3\relax
}
\TMP@EnsureCode\catcode{0}{12}% ^^@
\TMP@EnsureCode\catcode{34}{12}% "
\TMP@EnsureCode\catcode{36}{3}% $
\TMP@EnsureCode\catcode{38}{4}% &
\TMP@EnsureCode\catcode{39}{12}% '
\TMP@EnsureCode\catcode{42}{12}% *
\TMP@EnsureCode\catcode{45}{12}% -
\TMP@EnsureCode\catcode{46}{12}% .
\TMP@EnsureCode\catcode{47}{12}% /
\TMP@EnsureCode\catcode{60}{12}% <
\TMP@EnsureCode\catcode{62}{12}% >
\TMP@EnsureCode\catcode{91}{12}% [
\TMP@EnsureCode\catcode{93}{12}% ]
\TMP@EnsureCode\catcode{94}{7}% ^
\TMP@EnsureCode\catcode{96}{12}% `
\TMP@EnsureCode\uccode{34}{0}% "
\TMP@EnsureCode\uccode{48}{0}% 0
\TMP@EnsureCode\uccode{61}{0}% =
\edef\PE@AtEnd{\PE@AtEnd\noexpand\endinput}
%    \end{macrocode}
%
% \subsection{Load package}
%
%    \begin{macrocode}
\begingroup\expandafter\expandafter\expandafter\endgroup
\expandafter\ifx\csname RequirePackage\endcsname\relax
  \def\TMP@RequirePackage#1[#2]{%
    \begingroup\expandafter\expandafter\expandafter\endgroup
    \expandafter\ifx\csname ver@#1.sty\endcsname\relax
      \input #1.sty\relax
    \fi
  }%
  \TMP@RequirePackage{ltxcmds}[2010/04/08]%
\else
  \RequirePackage{ltxcmds}[2010/04/08]%
\fi
%    \end{macrocode}
%
% \subsection{Sanitizing}
%
%    \begin{macro}{\EdefSanitize}
%    Macro \cs{EdefSanitize} takes |#2|, entirely converts
%    it to token with catcode 12 (other) and stores the
%    result in macro |#1|.
%    \begin{macrocode}
\begingroup\expandafter\expandafter\expandafter\endgroup
\expandafter\ifx\csname detokenize\endcsname\relax
  \long\def\EdefSanitize#1#2{%
    \begingroup
      \csname @safe@activestrue\endcsname
      \edef#1{#2}%
      \PE@onelevel@sanitize#1%
    \expandafter\endgroup
    \expandafter\def\expandafter#1\expandafter{#1}%
  }%
  \begingroup\expandafter\expandafter\expandafter\endgroup
  \expandafter\ifx\csname @onelevel@sanitize\endcsname\relax
    \def\PE@onelevel@sanitize#1{%
      \edef#1{\expandafter\PE@strip@prefix\meaning#1}%
    }%
    \def\PE@strip@prefix#1>{}%
  \else
    \let\PE@onelevel@sanitize\@onelevel@sanitize
  \fi
\else
  \long\def\EdefSanitize#1#2{%
    \begingroup
      \csname @safe@activestrue\endcsname
      \edef#1{#2}%
    \expandafter\endgroup
    \expandafter\def\expandafter#1\expandafter{%
      \detokenize\expandafter{#1}%
    }%
  }%
  \def\PE@onelevel@sanitize#1{%
    \edef#1{\detokenize\expandafter{#1}}%
  }%
\fi
%    \end{macrocode}
%    \end{macro}
%    \begin{macro}{\PE@sanitize}
%    Macro \cs{PE@sanitize} is only defined for
%    compatibility with version 1.4. Its use is deprecated.
%    \begin{macrocode}
\let\PE@sanitize\EdefSanitize
%    \end{macrocode}
%    \end{macro}
%
% \subsubsection{Space characters}
%
%    \begin{macro}{\PE@space@other}
%    \begin{macrocode}
\begingroup
  \catcode`\ =12\relax%
\def\x{\endgroup\def\PE@space@other{ }}\x\relax
%    \end{macrocode}
%    \end{macro}
%    \begin{macro}{\PE@space@space}
%    \begin{macrocode}
\def\PE@space@space{ }
%    \end{macrocode}
%    \end{macro}
%
% \subsubsection{Space normalization}
%
%    \begin{macro}{\PE@SanitizeSpaceOther}
%    \begin{macrocode}
\def\PE@SanitizeSpaceOther#1{%
  \edef#1{\expandafter\PE@SpaceToOther#1 \relax}%
}
%    \end{macrocode}
%    \end{macro}
%
%    \begin{macro}{\PE@SpaceToOther}
%    \begin{macrocode}
\def\PE@SpaceToOther#1 #2\relax{%
  #1%
  \ifx\\#2\\%
  \else
    \PE@space@other
    \ltx@ReturnAfterFi{%
      \PE@SpaceToOther#2\relax
    }%
  \fi
}
%    \end{macrocode}
%    \end{macro}
%
% \subsection{\cs{EdefUnescapeName}}
%
%    \begin{macro}{\EdefUnescapeName}
%    \begin{macrocode}
\def\EdefUnescapeName#1#2{%
  \EdefSanitize#1{#2}%
  \PE@SanitizeSpaceOther#1%
  \PE@UnescapeName#1%
  \PE@onelevel@sanitize#1%
}
%    \end{macrocode}
%    \end{macro}
%
%    \begin{macro}{\PE@UnescapeName}
%    \begin{macrocode}
\begingroup
  \catcode`\$=6  % hash
  \catcode`\#=12 % other
  \gdef\PE@UnescapeName$1{%
    \begingroup
      \PE@InitUccodeHexDigit
      \def\PE@result{}%
      \expandafter\PE@DeName$1#\relax\relax
    \expandafter\endgroup
    \expandafter\def\expandafter$1\expandafter{\PE@result}%
  }%
  \gdef\PE@DeName$1#$2$3{%
    \ifx\relax$2%
      \edef\PE@result{\PE@result$1}%
      \let\PE@next\relax
    \else
      \ifx\relax$3%
        % wrong escape sequence in input
        \edef\PE@result{\PE@result$1#}%
        \let\PE@next\relax
      \else
        \uppercase{%
          \def\PE@testA{$2}%
          \def\PE@testB{$3}%
        }%
        \ifcase\ifcase\expandafter\PE@TestUcHexDigit\PE@testA
                 \ifcase\expandafter\PE@TestUcHexDigit\PE@testB
                   \ltx@zero
                 \else
                   \ltx@one
                 \fi
               \else
                 \ltx@one
               \fi
          \uccode\ltx@zero="\PE@testA\PE@testB\relax
          \uppercase{%
            \def\PE@temp{^^@}%
          }%
          \uccode\ltx@zero=\ltx@zero
          \edef\PE@result{\PE@result$1\PE@temp}%
          \let\PE@next\PE@DeName
        \else
          % wrong escape sequence in input
          \edef\PE@result{\PE@result$1#}%
          \def\PE@next{\PE@DeName$2$3}%
        \fi
      \fi
    \fi
    \PE@next
  }%
\endgroup
%    \end{macrocode}
%    \end{macro}
%
%    \begin{macro}{\PE@InitUccodeHexDigit}
%    \begin{macrocode}
\def\PE@InitUccodeHexDigit{%
  \uccode`a=`A\relax
  \uccode`b=`B\relax
  \uccode`c=`C\relax
  \uccode`d=`D\relax
  \uccode`e=`E\relax
  \uccode`f=`F\relax
  \uccode`A=\ltx@zero
  \uccode`B=\ltx@zero
  \uccode`C=\ltx@zero
  \uccode`D=\ltx@zero
  \uccode`E=\ltx@zero
  \uccode`F=\ltx@zero
  \uccode`0=\ltx@zero
  \uccode`1=\ltx@zero
  \uccode`2=\ltx@zero
  \uccode`3=\ltx@zero
  \uccode`4=\ltx@zero
  \uccode`5=\ltx@zero
  \uccode`6=\ltx@zero
  \uccode`7=\ltx@zero
  \uccode`8=\ltx@zero
  \uccode`9=\ltx@zero
}
%    \end{macrocode}
%    \end{macro}
%
%    \begin{macro}{\PE@TestUcHexDigit}
%    \begin{macrocode}
\def\PE@TestUcHexDigit#1{%
  \ifnum`#1<48 % 0
    \ltx@one
  \else
    \ifnum`#1>70 % F
      \ltx@one
    \else
      \ifnum`#1>57 % 9
        \ifnum`#1<65 % A
          \ltx@one
        \else
          \ltx@zero
        \fi
      \else
        \ltx@zero
      \fi
    \fi
  \fi
}
%    \end{macrocode}
%    \end{macro}
%
% \subsection{\cs{EdefUnescapeString}}
%
%    \begin{macro}{\EdefUnescapeString}
%    \begin{macrocode}
\def\EdefUnescapeString#1#2{%
  \EdefSanitize#1{#2}%
  \PE@SanitizeSpaceOther#1%
  \PE@NormalizeLineEnd#1%
  \PE@UnescapeString#1%
  \PE@onelevel@sanitize#1%
}
%    \end{macrocode}
%    \end{macro}
%
%    \begin{macrocode}
\begingroup
  \uccode`\8=10 % lf
  \uccode`\9=13 % cr
\def\x#1#2{\endgroup
%    \end{macrocode}
%    \begin{macro}{\PE@NormalizeLineEnd}
%    \begin{macrocode}
  \def\PE@NormalizeLineEnd##1{%
    \def\PE@result{}%
    \expandafter\PE@@NormalizeLineEnd##1#2\relax
    \let##1\PE@result
  }%
%    \end{macrocode}
%    \end{macro}
%    \begin{macro}{\PE@@NormalizeLineEnd}
%    \begin{macrocode}
  \def\PE@@NormalizeLineEnd##1#2##2{%
    \ifx\relax##2%
      \edef\PE@result{\PE@result##1}%
      \let\PE@next\relax
    \else
      \edef\PE@result{\PE@result##1#1}%
      \ifx#1##2% lf
        \let\PE@next\PE@@NormalizeLineEnd
      \else
        \def\PE@next{\PE@@NormalizeLineEnd##2}%
      \fi
    \fi
    \PE@next
  }%
}%
\uppercase{%
  \x 89%
}
%    \end{macrocode}
%    \end{macro}
%
%    \begin{macrocode}
\begingroup
  \catcode`\|=0 %
  \catcode`\\=12 %
%    \end{macrocode}
%    \begin{macro}{\PE@UnescapeString}
%    \begin{macrocode}
  |gdef|PE@UnescapeString#1{%
    |begingroup
      |def|PE@result{}%
      |expandafter|PE@DeString#1\|relax
    |expandafter|endgroup
    |expandafter|def|expandafter#1|expandafter{|PE@result}%
  }%
%    \end{macrocode}
%    \end{macro}
%    \begin{macro}{\PE@DeString}
%    \begin{macrocode}
  |gdef|PE@DeString#1\#2{%
    |ifx|relax#2%
      |edef|PE@result{|PE@result#1}%
      |let|PE@next|relax
    |else
      |if n#2%
        |uccode|ltx@zero=10 %
      |else|if r#2%
        |uccode|ltx@zero=13 %
      |else|if t#2%
        |uccode|ltx@zero=9 %
      |else|if b#2%
        |uccode|ltx@zero=8 %
      |else|if f#2%
        |uccode|ltx@zero=12 %
      |else
        |uccode|ltx@zero=|ltx@zero
      |fi|fi|fi|fi|fi
      |ifnum|uccode|ltx@zero>|ltx@zero
        |uppercase{%
          |edef|PE@temp{^^@}%
        }%
        |edef|PE@result{|PE@result#1|PE@temp}%
        |let|PE@next|PE@DeString
      |else
        |if\#2% backslash
          |edef|PE@result{|PE@result#1}%
          |let|PE@next|PE@CheckEndBackslash
        |else
          |ifnum`#2=10 % linefeed
            |edef|PE@result{|PE@result#1}%
            |let|PE@next|PE@DeString
          |else
            |ifcase|PE@TestOctDigit#2%
              |edef|PE@result{|PE@result#1}%
              |def|PE@next{|PE@OctI#2}%
            |else
              |edef|PE@result{|PE@result#1#2}%
              |let|PE@next|PE@DeString
            |fi
          |fi
        |fi
      |fi
    |fi
    |PE@next
  }%
%    \end{macrocode}
%    \end{macro}
%    \begin{macro}{\PE@CheckEndBackslash}
%    \begin{macrocode}
  |gdef|PE@CheckEndBackslash#1{%
    |ifx|relax#1%
    |else
      |edef|PE@result{|PE@result\}%
      |expandafter|PE@DeString|expandafter#1%
    |fi
  }%
%    \end{macrocode}
%    \end{macro}
%    \begin{macrocode}
|endgroup
%    \end{macrocode}
%    \begin{macro}{\PE@TestOctDigit}
%    \begin{macrocode}
\def\PE@TestOctDigit#1{%
  \ifnum`#1<48 % 0
    \ltx@one
  \else
    \ifnum`#1>55 % 7
      \ltx@one
    \else
      \ltx@zero
    \fi
  \fi
}
%    \end{macrocode}
%    \end{macro}
%    \begin{macro}{\PE@OctI}
%    \begin{macrocode}
\def\PE@OctI#1#2{%
  \ifcase\PE@TestOctDigit#2%
    \def\PE@next{\PE@OctII{#1#2}}%
  \else
    \def\PE@next{\PE@OctAll#1#2}%
  \fi
  \PE@next
}
%    \end{macrocode}
%    \end{macro}
%    \begin{macro}{\PE@OctII}
%    \begin{macrocode}
\def\PE@OctII#1#2{%
  \ifcase\PE@TestOctDigit#2%
    \def\PE@next{\PE@OctIII#1#2}%
  \else
    \def\PE@next{\PE@OctAll{#1}#2}%
  \fi
  \PE@next
}
%    \end{macrocode}
%    \end{macro}
%    \begin{macrocode}
\ltx@IfUndefined{numexpr}{%
  \catcode`\$=9 %
  \catcode`\&=14 %
}{%
  \catcode`\$=14 %
  \catcode`\&=9 %
}
%    \end{macrocode}
%    \begin{macro}{\PE@OctIII}
%    \begin{macrocode}
\def\PE@OctIII#1#2#3{%
  \ifnum#1<4 %
    \def\PE@next{\PE@OctAll{#1#2#3}}%
  \else
$   \count\ltx@cclv#1 %
$   \advance\count\ltx@cclv -4 %
    \edef\PE@next{%
      \noexpand\PE@OctAll{%
$       \the\count\ltx@cclv
&       \the\numexpr#1-4\relax
        #2#3%
      }%
    }%
  \fi
  \PE@next
}
%    \end{macrocode}
%    \end{macro}
%    \begin{macro}{\PE@OctAll}
%    \begin{macrocode}
\def\PE@OctAll#1{%
  \uccode\ltx@zero='#1\relax
  \uppercase{%
    \edef\PE@result{\PE@result^^@}%
  }%
  \PE@DeString
}
%    \end{macrocode}
%    \end{macro}
%
% \subsection{User macros (\pdfTeX\space analogues)}
%
%    \begin{macrocode}
\begingroup\expandafter\expandafter\expandafter\endgroup
\expandafter\ifx\csname RequirePackage\endcsname\relax
  \def\TMP@RequirePackage#1[#2]{%
    \begingroup\expandafter\expandafter\expandafter\endgroup
    \expandafter\ifx\csname ver@#1.sty\endcsname\relax
      \input #1.sty\relax
    \fi
  }%
  \TMP@RequirePackage{pdftexcmds}[2007/11/11]%
\else
  \RequirePackage{pdftexcmds}[2007/11/11]%
\fi
%    \end{macrocode}
%
%    \begin{macrocode}
\begingroup\expandafter\expandafter\expandafter\endgroup
\expandafter\ifx\csname pdf@escapehex\endcsname\relax
%    \end{macrocode}
%
%    \begin{macro}{\EdefEscapeHex}
%    \begin{macrocode}
  \long\def\EdefEscapeHex#1#2{%
    \EdefSanitize#1{#2}%
    \PE@SanitizeSpaceOther#1%
    \PE@EscapeHex#1%
  }%
%    \end{macrocode}
%    \end{macro}
%    \begin{macro}{\EdefUnescapeHex}
%    \begin{macrocode}
  \def\EdefUnescapeHex#1#2{%
    \EdefSanitize#1{#2}%
    \PE@UnescapeHex#1%
  }%
%    \end{macrocode}
%    \end{macro}
%    \begin{macro}{\EdefEscapeName}
%    \begin{macrocode}
  \long\def\EdefEscapeName#1#2{%
    \EdefSanitize#1{#2}%
    \PE@SanitizeSpaceOther#1%
    \PE@EscapeName#1%
  }%
%    \end{macrocode}
%    \end{macro}
%    \begin{macro}{\EdefEscapeString}
%    \begin{macrocode}
  \long\def\EdefEscapeString#1#2{%
    \EdefSanitize#1{#2}%
    \PE@SanitizeSpaceOther#1%
    \PE@EscapeString#1%
  }%
%    \end{macrocode}
%    \end{macro}
%
%    \begin{macrocode}
\else
%    \end{macrocode}
%
%    \begin{macro}{\PE@edefbabel}
%    Help macro that adds support for \xpackage{babel}'s shorthand
%    characters.
%    \begin{macrocode}
  \long\def\PE@edefbabel#1#2#3{%
    \begingroup
      \csname @save@activestrue\endcsname
      \edef#1{#2{#3}}%
    \expandafter\endgroup
    \expandafter\def\expandafter#1\expandafter{#1}%
  }%
%    \end{macrocode}
%    \end{macro}
%
%    \begin{macro}{\EdefEscapeHex}
%    \begin{macrocode}
  \long\def\EdefEscapeHex#1#2{%
    \PE@edefbabel#1\pdf@escapehex{#2}%
  }%
%    \end{macrocode}
%    \end{macro}
%    \begin{macro}{\EdefUnescapeHex}
%    \begin{macrocode}
  \def\EdefUnescapeHex#1#2{%
    \PE@edefbabel#1\pdf@unescapehex{#2}%
  }%
%    \end{macrocode}
%    \end{macro}
%    \begin{macro}{\EdefEscapeName}
%    \begin{macrocode}
  \long\def\EdefEscapeName#1#2{%
    \PE@edefbabel#1\pdf@escapename{#2}%
  }%
%    \end{macrocode}
%    \end{macro}
%    \begin{macro}{\EdefEscapeString}
%    \begin{macrocode}
  \long\def\EdefEscapeString#1#2{%
    \PE@edefbabel#1\pdf@escapestring{#2}%
  }%
%    \end{macrocode}
%    \end{macro}
%
%    \begin{macrocode}
  \expandafter\PE@AtEnd
\fi%
%    \end{macrocode}
%
% \subsection{Help macros}
%
% \subsubsection{Characters}
%
%    Special characters with catcode 12 (other) are created and
%    stored in macros.
%    \begin{macro}{\PE@hash}
%    \begin{macrocode}
\edef\PE@hash{\string#}
%    \end{macrocode}
%    \end{macro}
%    \begin{macro}{\PE@backslash}
%    \begin{macrocode}
\begingroup
  \escapechar=-1 %
\edef\x{\endgroup
  \def\noexpand\PE@backslash{\string\\}%
}
\x
%    \end{macrocode}
%    \end{macro}
%
% \subsubsection{Switch for \eTeX}
%
%    \begin{macrocode}
\ltx@newif\ifPE@etex
\begingroup\expandafter\expandafter\expandafter\endgroup
\expandafter\ifx\csname numexpr\endcsname\relax
\else
  \PE@etextrue
\fi
%    \end{macrocode}
%
% \subsection{Conversions}
%
% \subsubsection{Conversion to hex string}
%
%    \begin{macro}{\PE@EscapeHex}
%    \begin{macrocode}
\ifPE@etex
  \def\PE@EscapeHex#1{%
    \edef#1{\expandafter\PE@ToHex#1\relax}%
  }%
\else
  \def\PE@EscapeHex#1{%
    \def\PE@result{}%
    \expandafter\PE@ToHex#1\relax
    \let#1\PE@result
  }%
\fi
%    \end{macrocode}
%    \end{macro}
%    \begin{macro}{\PE@ToHex}
%    \begin{macrocode}
\def\PE@ToHex#1{%
  \ifx\relax#1%
  \else
    \PE@HexChar{#1}%
    \expandafter\PE@ToHex
  \fi
}%
%    \end{macrocode}
%    \end{macro}
%    \begin{macro}{\PE@HexChar}
%    \begin{macrocode}
\ifPE@etex
  \def\PE@HexChar#1{%
    \PE@HexDigit{\numexpr\dimexpr.0625\dimexpr`#1sp\relax\relax\relax}%
    \PE@HexDigit{%
      \numexpr`#1-16*\dimexpr.0625\dimexpr`#1sp\relax\relax\relax
    }%
  }%
\else
  \def\PE@HexChar#1{%
    \dimen0=`#1sp%
    \dimen2=.0625\dimen0 %
    \advance\dimen0-16\dimen2 %
    \edef\PE@result{%
      \PE@result
      \PE@HexDigit{\dimen2 }%
      \PE@HexDigit{\dimen0 }%
    }%
  }%
\fi
%    \end{macrocode}
%    \end{macro}
%    \begin{macro}{\PE@HexDigit}
%    \begin{macrocode}
\def\PE@HexDigit#1{%
  \expandafter\string
  \ifcase#1%
    0\or 1\or 2\or 3\or 4\or 5\or 6\or 7\or 8\or 9\or
    A\or B\or C\or D\or E\or F%
  \fi
}
%    \end{macrocode}
%    \end{macro}
%
% \subsubsection{Character code to octal number}
%
%    \begin{macro}{\PE@OctChar}
%    \begin{macrocode}
\ifPE@etex
  \def\PE@OctChar#1{%
    \expandafter\PE@@OctChar
        \the\numexpr\dimexpr.015625\dimexpr`#1sp\relax\relax
            \expandafter\relax
        \expandafter\relax
        \the\numexpr\dimexpr.125\dimexpr`#1sp\relax\relax\relax
        \relax
        #1%
  }%
  \def\PE@@OctChar#1\relax#2\relax#3{%
    \PE@backslash
    #1%
    \the\numexpr#2-8*#1\relax
    \the\numexpr\dimexpr`#3sp\relax-8*#2\relax
  }%
\else
  \def\PE@OctChar#1{%
    \dimen0=`#1sp%
    \dimen2=.125\dimen0 %
    \dimen4=.125\dimen2 %
    \advance\dimen0-8\dimen2 %
    \advance\dimen2-8\dimen4 %
    \edef\PE@result{%
      \PE@result
      \PE@backslash
      \number\dimen4 %
      \number\dimen2 %
      \number\dimen0 %
    }%
  }%
\fi
%    \end{macrocode}
%    \end{macro}
%
% \subsubsection{Unpack hex string}
%
%    \begin{macro}{\PE@UnescapeHex}
%    \begin{macrocode}
\def\PE@UnescapeHex#1{%
  \begingroup
    \PE@InitUccodeHexDigit
    \def\PE@result{}%
    \expandafter\PE@DeHex#1\relax\relax
  \expandafter\endgroup
  \expandafter\def\expandafter#1\expandafter{\PE@result}%
}
%    \end{macrocode}
%    \end{macro}
%    \begin{macro}{\PE@DeHex}
%    \begin{macrocode}
\def\PE@DeHex#1#2{%
  \ifx#2\relax
    \ifx#1\relax
      \let\PE@next\relax
    \else
      \uppercase{%
        \def\PE@testA{#1}%
      }%
      \ifcase\expandafter\PE@TestUcHexDigit\PE@testA
        \def\PE@next{%
          \PE@DeHex#10\relax\relax
        }%
      \else
        \let\PE@next\relax
      \fi
    \fi
  \else
    \uppercase{%
      \def\PE@testA{#1}%
      \def\PE@testB{#2}%
    }%
    \ifcase\expandafter\PE@TestUcHexDigit\PE@testA
      \ifcase\expandafter\PE@TestUcHexDigit\PE@testB
        \uccode\ltx@zero="\PE@testA\PE@testB\relax
        \ifnum\uccode\ltx@zero=32 %
          \let\PE@temp\PE@space@space
        \else
          \uppercase{%
            \def\PE@temp{^^@}%
          }%
        \fi
        \edef\PE@result{\PE@result\PE@temp}%
        \let\PE@next\PE@DeHex
      \else
        % invalid input sequence
        \def\PE@next{%
          \PE@DeHex#1%
        }%
      \fi
    \else
      % invalid input sequence
      \def\PE@next{\PE@DeHex#2}%
    \fi
  \fi
  \PE@next
}
%    \end{macrocode}
%    \end{macro}
%
% \subsubsection{Conversion to PDF name}
%
%    \begin{macro}{\PE@EscapeName}
%    \begin{macrocode}
\ifPE@etex
  \def\PE@EscapeName#1{%
    \edef#1{\expandafter\PE@EscapeNameTokens#1\relax}%
  }%
\else
  \def\PE@EscapeName#1{%
    \def\PE@result{}%
    \expandafter\PE@EscapeNameTokens#1\relax
    \let#1\PE@result
  }%
\fi
%    \end{macrocode}
%    \end{macro}
%    \begin{macro}{\PE@EscapeNameTokens}
%    \begin{macrocode}
\def\PE@EscapeNameTokens#1{%
  \ifx\relax#1%
  \else
    \ifnum`#1<33 %
      \ifcase`#1 %
        % drop illegal zero
      \else
        \PE@EscapeNameAdd\PE@hash
        \PE@HexChar#1%
      \fi
    \else
      \ifnum`#1>126 %
        \PE@EscapeNameAdd\PE@hash
        \PE@HexChar#1%
      \else  \ifnum`#1=35  \PE@EscapeNameHashChar 23% #
        \else\ifnum`#1=37  \PE@EscapeNameHashChar 25% %
        \else\ifnum`#1=40  \PE@EscapeNameHashChar 28% (
        \else\ifnum`#1=41  \PE@EscapeNameHashChar 29% )
        \else\ifnum`#1=47  \PE@EscapeNameHashChar 2F% /
        \else\ifnum`#1=60  \PE@EscapeNameHashChar 3C% <
        \else\ifnum`#1=62  \PE@EscapeNameHashChar 3E% >
        \else\ifnum`#1=91  \PE@EscapeNameHashChar 5B% [
        \else\ifnum`#1=93  \PE@EscapeNameHashChar 5D% ]
        \else\ifnum`#1=123 \PE@EscapeNameHashChar 7B% {
        \else\ifnum`#1=125 \PE@EscapeNameHashChar 7D% }
        \else
          \PE@EscapeNameAdd{#1}%
        \fi\fi\fi\fi\fi\fi\fi\fi\fi\fi\fi
      \fi
    \fi
    \expandafter\PE@EscapeNameTokens
  \fi
}%
\def\PE@EscapeNameHashChar#1#2{%
  \PE@EscapeNameAdd{\PE@hash\string#1\string#2}%
}%
%    \end{macrocode}
%    \end{macro}
%    \begin{macro}{\PE@EscapeNameAdd}
%    \begin{macrocode}
\ifPE@etex
  \def\PE@EscapeNameAdd#1{#1}%
\else
  \def\PE@EscapeNameAdd#1{%
    \edef\PE@result{%
      \PE@result
      #1%
    }%
  }%
\fi
%    \end{macrocode}
%    \end{macro}
%
% \subsubsection{Conversion to PDF string}
%
%    \begin{macro}{\PE@EscapeString}
%    \begin{macrocode}
\ifPE@etex
  \def\PE@EscapeString#1{%
    \edef#1{\expandafter\PE@EscapeStringTokens#1\relax}%
  }%
\else
  \def\PE@EscapeString#1{%
    \begingroup
      \def\PE@result{}%
      \expandafter\PE@EscapeStringTokens#1\relax
    \expandafter\endgroup
    \expandafter\def\expandafter#1\expandafter{\PE@result}%
  }%
\fi
%    \end{macrocode}
%    \end{macro}
%    \begin{macro}{\PE@EscapeStringTokens}
%    \begin{macrocode}
\def\PE@EscapeStringTokens#1{%
  \ifx\relax#1%
  \else
    \ifnum`#1<33 %
      \PE@OctChar#1%
    \else
      \ifnum`#1>126 %
        \PE@OctChar#1%
      \else  \ifnum`#1=40 \PE@EscapeStringAdd{\string\(}% (
        \else\ifnum`#1=41 \PE@EscapeStringAdd{\string\)}% )
        \else\ifnum`#1=92 \PE@EscapeStringAdd{\string\\}% \
        \else
          \PE@EscapeStringAdd{#1}%
        \fi\fi\fi
      \fi
    \fi
    \expandafter\PE@EscapeStringTokens
  \fi
}%
%    \end{macrocode}
%    \end{macro}
%    \begin{macro}{\PE@EscapeStringAdd}
%    \begin{macrocode}
\ifPE@etex
  \def\PE@EscapeStringAdd#1{#1}%
\else
  \def\PE@EscapeStringAdd#1{%
    \edef\PE@result{%
      \PE@result
      #1%
    }%
  }%
\fi
%    \end{macrocode}
%    \end{macro}
%
%    \begin{macrocode}
\PE@AtEnd%
%    \end{macrocode}
%
%    \begin{macrocode}
%</package>
%    \end{macrocode}
%
% \section{Test}
%
% \subsection{Catcode checks for loading}
%
%    \begin{macrocode}
%<*test1>
%    \end{macrocode}
%    \begin{macrocode}
\catcode`\{=1 %
\catcode`\}=2 %
\catcode`\#=6 %
\catcode`\@=11 %
\expandafter\ifx\csname count@\endcsname\relax
  \countdef\count@=255 %
\fi
\expandafter\ifx\csname @gobble\endcsname\relax
  \long\def\@gobble#1{}%
\fi
\expandafter\ifx\csname @firstofone\endcsname\relax
  \long\def\@firstofone#1{#1}%
\fi
\expandafter\ifx\csname loop\endcsname\relax
  \expandafter\@firstofone
\else
  \expandafter\@gobble
\fi
{%
  \def\loop#1\repeat{%
    \def\body{#1}%
    \iterate
  }%
  \def\iterate{%
    \body
      \let\next\iterate
    \else
      \let\next\relax
    \fi
    \next
  }%
  \let\repeat=\fi
}%
\def\RestoreCatcodes{}
\count@=0 %
\loop
  \edef\RestoreCatcodes{%
    \RestoreCatcodes
    \catcode\the\count@=\the\catcode\count@\relax
  }%
\ifnum\count@<255 %
  \advance\count@ 1 %
\repeat

\def\RangeCatcodeInvalid#1#2{%
  \count@=#1\relax
  \loop
    \catcode\count@=15 %
  \ifnum\count@<#2\relax
    \advance\count@ 1 %
  \repeat
}
\def\RangeCatcodeCheck#1#2#3{%
  \count@=#1\relax
  \loop
    \ifnum#3=\catcode\count@
    \else
      \errmessage{%
        Character \the\count@\space
        with wrong catcode \the\catcode\count@\space
        instead of \number#3%
      }%
    \fi
  \ifnum\count@<#2\relax
    \advance\count@ 1 %
  \repeat
}
\def\space{ }
\expandafter\ifx\csname LoadCommand\endcsname\relax
  \def\LoadCommand{\input pdfescape.sty\relax}%
\fi
\def\Test{%
  \RangeCatcodeInvalid{0}{47}%
  \RangeCatcodeInvalid{58}{64}%
  \RangeCatcodeInvalid{91}{96}%
  \RangeCatcodeInvalid{123}{255}%
  \catcode`\@=12 %
  \catcode`\\=0 %
  \catcode`\%=14 %
  \LoadCommand
  \RangeCatcodeCheck{0}{36}{15}%
  \RangeCatcodeCheck{37}{37}{14}%
  \RangeCatcodeCheck{38}{47}{15}%
  \RangeCatcodeCheck{48}{57}{12}%
  \RangeCatcodeCheck{58}{63}{15}%
  \RangeCatcodeCheck{64}{64}{12}%
  \RangeCatcodeCheck{65}{90}{11}%
  \RangeCatcodeCheck{91}{91}{15}%
  \RangeCatcodeCheck{92}{92}{0}%
  \RangeCatcodeCheck{93}{96}{15}%
  \RangeCatcodeCheck{97}{122}{11}%
  \RangeCatcodeCheck{123}{255}{15}%
  \RestoreCatcodes
}
\Test
\csname @@end\endcsname
\end
%    \end{macrocode}
%    \begin{macrocode}
%</test1>
%    \end{macrocode}
%
% \subsection{Macro tests}
%
%    \begin{macrocode}
%<*test2|test3|test4|test5>
\NeedsTeXFormat{LaTeX2e}
\makeatletter
%    \end{macrocode}
%
% \subsection{Test with \cs{pdfescape...} commands}
%
%    \begin{macrocode}
%<*test2>
\ProvidesFile{pdfescape-test2.tex}%
    [2011/11/25 v1.13 Test with \string\pdfescape... commands]%
%</test2>
%    \end{macrocode}
%
% \subsection{Test without \cs{pdfescape...}, with \eTeX}
%
%    \begin{macrocode}
%<*test3>
\ProvidesFile{pdfescape-test3.tex}%
    [2011/11/25 v1.13 Test without \string\pdfescape..., with e-TeX]%
%</test3>
%    \end{macrocode}
%
% \subsection{Test without \cs{pdfescape...} and \eTeX}
%
%    \begin{macrocode}
%<*test4>
\ProvidesFile{pdfescape-test4.tex}%
    [2011/11/25 v1.13 Test without \string\pdfescape... and e-TeX]%
%</test4>
%    \end{macrocode}
%
% \subsection{Test with \LuaTeX}
%
%    \begin{macrocode}
%<*test5>
\ProvidesFile{pdfescape-test5.tex}%
    [2011/11/25 v1.13 Test with LuaTeX]%
%</test5>
%    \end{macrocode}
%
% \subsection{Check/ensure test preconditions}
%
% \subsubsection{Check \cs{pdfescape...}}
%
%    \begin{macrocode}
%<*test2>
\@ifundefined{pdfescapehex}{%
  \PackageError{pdfescape-test2}{%
    Missing \string\pdfescape... commands%
  }{Test aborted.}%
  \stop
}{}
%</test2>
%    \end{macrocode}
%    \begin{macrocode}
%<*test3|test4>
\let\pdfescapehex\@undefined
\let\pdfunescapehex\@undefined
\let\pdfescapename\@undefined
\let\pdfescapestring\@undefined
%</test3|test4>
%    \end{macrocode}
%
% \subsubsection{Check \eTeX}
%
%    \begin{macrocode}
%<*test3>
\@ifundefined{numexpr}{%
  \PackageError{pdfescape-test3}{%
    Missing \eTeX
  }{Test aborted.}%
  \stop
}{}
%</test3>
%    \end{macrocode}
% Package \xpackage{qstest} uses \eTeX, thus \eTeX's features
% can only be disabled later during loading of package \xpackage{pdfescape}.
%
% \subsubsection{Check \LuaTeX}
%
%    \begin{macrocode}
%<*test5>
\@ifundefined{directlua}{%
  \PackageError{pdfescape-test5}{%
    Missing LuaTeX%
  }{Test aborted.}%
  \stop
}{}
%</test5>
%    \end{macrocode}
%
% \subsection{Common part}
%
%    The files for testing uses the framework, provided by the
%    new package \xpackage{qstest} of David Kastrup.
%    \begin{macrocode}
\RequirePackage{qstest}
\IncludeTests{*}
\LogTests{log}{*}{*}

\newcommand*{\ExpectVar}[2]{%
  \ifx#1#2%
  \else
    \begingroup
      \@onelevel@sanitize#1%
      \@onelevel@sanitize#2%
      \typeout{[#1] <> [#2]}% hash-ok
    \endgroup
  \fi
  \Expect*{\ifx#1#2true\else false\fi}{true}%
}

\makeatletter
\begingroup
  \gdef\AllBytes{}%
  \count@=0 %
  \catcode0=12 %
  \@whilenum\count@<256 \do{%
    \lccode0=\count@
    \ifnum\count@=32 %
      \xdef\AllBytes{\AllBytes\space}%
    \else
      \lowercase{%
        \xdef\AllBytes{\AllBytes^^@}%
      }%
    \fi
    \advance\count@ by 1 %
  }%
\endgroup
\newcommand*{\AllBytesHex}{%
  000102030405060708090A0B0C0D0E0F%
  101112131415161718191A1B1C1D1E1F%
  202122232425262728292A2B2C2D2E2F%
  303132333435363738393A3B3C3D3E3F%
  404142434445464748494A4B4C4D4E4F%
  505152535455565758595A5B5C5D5E5F%
  606162636465666768696A6B6C6D6E6F%
  707172737475767778797A7B7C7D7E7F%
  808182838485868788898A8B8C8D8E8F%
  909192939495969798999A9B9C9D9E9F%
  A0A1A2A3A4A5A6A7A8A9AAABACADAEAF%
  B0B1B2B3B4B5B6B7B8B9BABBBCBDBEBF%
  C0C1C2C3C4C5C6C7C8C9CACBCCCDCECF%
  D0D1D2D3D4D5D6D7D8D9DADBDCDDDEDF%
  E0E1E2E3E4E5E6E7E8E9EAEBECEDEEEF%
  F0F1F2F3F4F5F6F7F8F9FAFBFCFDFEFF%
}
\@onelevel@sanitize\AllBytesHex
\expandafter\lowercase\expandafter{%
  \expandafter\newcommand\expandafter*\expandafter\AllBytesHexLC
      \expandafter{\AllBytesHex}%
}
\newcommand*{\AllBytesName}{}
\begingroup
  \catcode`\#=12 %
  \xdef\AllBytesName{%
    #01#02#03#04#05#06#07#08#09#0A#0B#0C#0D#0E#0F%
    #10#11#12#13#14#15#16#17#18#19#1A#1B#1C#1D#1E#1F%
    #20!"#23$#25&'#28#29*+,-.#2F%
    0123456789:;#3C=#3E?%
    @ABCDEFGHIJKLMNO%
    PQRSTUVWXYZ#5B\@backslashchar#5D^_%
    `abcdefghijklmno%
    pqrstuvwxyz#7B|#7D\string~#7F%
    #80#81#82#83#84#85#86#87#88#89#8A#8B#8C#8D#8E#8F%
    #90#91#92#93#94#95#96#97#98#99#9A#9B#9C#9D#9E#9F%
    #A0#A1#A2#A3#A4#A5#A6#A7#A8#A9#AA#AB#AC#AD#AE#AF%
    #B0#B1#B2#B3#B4#B5#B6#B7#B8#B9#BA#BB#BC#BD#BE#BF%
    #C0#C1#C2#C3#C4#C5#C6#C7#C8#C9#CA#CB#CC#CD#CE#CF%
    #D0#D1#D2#D3#D4#D5#D6#D7#D8#D9#DA#DB#DC#DD#DE#DF%
    #E0#E1#E2#E3#E4#E5#E6#E7#E8#E9#EA#EB#EC#ED#EE#EF%
    #F0#F1#F2#F3#F4#F5#F6#F7#F8#F9#FA#FB#FC#FD#FE#FF%
  }%
\endgroup
\@onelevel@sanitize\AllBytesName

\newcommand*{\AllBytesString}{}
\begingroup
  \def\|{|}%
  \edef\%{\@percentchar}%
  \catcode`\|=0 %
  \catcode`\#=12 %
  \catcode`\~=12 %
  \catcode`\\=12 %
  |xdef|AllBytesString{%
    \000\001\002\003\004\005\006\007\010\011\012\013\014\015\016\017%
    \020\021\022\023\024\025\026\027\030\031\032\033\034\035\036\037%
    \040!"#$|%&'\(\)*+,-./%
    0123456789:;<=>?%
    @ABCDEFGHIJKLMNO%
    PQRSTUVWXYZ[\\]^_%
    `abcdefghijklmno%
    pqrstuvwxyz{||}~\177%
    \200\201\202\203\204\205\206\207\210\211\212\213\214\215\216\217%
    \220\221\222\223\224\225\226\227\230\231\232\233\234\235\236\237%
    \240\241\242\243\244\245\246\247\250\251\252\253\254\255\256\257%
    \260\261\262\263\264\265\266\267\270\271\272\273\274\275\276\277%
    \300\301\302\303\304\305\306\307\310\311\312\313\314\315\316\317%
    \320\321\322\323\324\325\326\327\330\331\332\333\334\335\336\337%
    \340\341\342\343\344\345\346\347\350\351\352\353\354\355\356\357%
    \360\361\362\363\364\365\366\367\370\371\372\373\374\375\376\377%
  }%
|endgroup
\@onelevel@sanitize\AllBytesString

%<*test4>
\let\org@detokenize\detokenize
\let\detokenize\@undefined
\let\org@numexpr\numexpr
\let\numexpr\@undefined
%</test4>
\RequirePackage{pdfescape}
%<*test4>
\let\detokenize\org@detokenize
\let\numexpr\org@numexpr
%</test4>

\begin{qstest}{all-hex}{\AllBytes, escapehex}
  \EdefEscapeHex\x{\AllBytes}%
  \Expect*{\x}*{\AllBytesHex}%
  \ExpectVar\x\AllBytesHex
\end{qstest}

\begin{qstest}{all-unhex}{\AllBytesHex, unescapehex}
  \EdefUnescapeHex\x{\AllBytesHex}%
  \Expect*{\x}*{\AllBytes}%
  \ExpectVar\x\AllBytes
\end{qstest}

\begin{qstest}{all-unhex-lc}{\AllBytesHexLC, unescapehex, lowercase}
  \EdefUnescapeHex\x{\AllBytesHexLC}%
  \Expect*{\x}*{\AllBytes}%
  \ExpectVar\x\AllBytes
\end{qstest}

\begin{qstest}{unhex-incomplete}{unescapehex, incomplete}
  \EdefUnescapeHex\x{4}%
  \Expect*{\x}{@}%
\end{qstest}

\begin{qstest}{unhex-space}{unescapehex, space}
  \EdefUnescapeHex\x{20}%
  \Expect*{\x}{ }%
  \ExpectVar\x\space
\end{qstest}

\begin{qstest}{unhex-spaces}{unescapehex, spaces}
  \EdefUnescapeHex\x{204020204120}%
  \def\y#1{%
    \edef\z{#1\string @#1#1\string A#1}%
  }\y{ }%
  \Expect*{\x}*{\z}%
  \ExpectVar\x\z
\end{qstest}

\begin{qstest}{unhex-hash}{unescapehex, hash}
  \catcode`\#=12 %
  \EdefUnescapeHex\x{#20}%
  \ExpectVar\x\space
\end{qstest}

\begin{qstest}{unhex-invalid}{unescapehex, invalid}
  \def\test#1#2{%
    \EdefUnescapeHex\x{#1}%
    \edef\y{#2}%
    \@onelevel@sanitize\y
    \ExpectVar\x\y
  }%
%<*test2>
  \edef\x{\pdfunescapehex{4X}}%
  \edef\y{\string @}%
  \ifx\x\y
  \else
    \def~{\space}%
    \typeout{************************************}%
    \typeout{* Your pdfTeX contains bug 777.~~~~*}%
    \typeout{* This test is redefined as dummy, *}%
    \typeout{* because it triggers the bug.~~~~~*}%
    \typeout{************************************}%
    \def\test#1#2{}%
  \fi
%</test2>
  \test{X}{}%
  \test{XY}{}%
  \test{XYZ}{}%
  \test{A}{^^a0}%
  \test{AX}{^^a0}%
  \test{XA}{^^a0}%
  \test{XXAXX}{^^a0}%
\end{qstest}

\begin{qstest}{all-name}{\AllBytes, escapename}
  \EdefEscapeName\x{\AllBytes}%
  \Expect*{\x}*{\AllBytesName}%
  \ExpectVar\x\AllBytesName
\end{qstest}

\begin{qstest}{all-string}{\AllBytes, escapestring}
  \EdefEscapeString\x{\AllBytes}%
  \Expect*{\x}*{\AllBytesString}%
  \ExpectVar\x\AllBytesString
\end{qstest}

\begin{qstest}{uchexdigit}{unescape, uppercase hex digit}
  \catcode`\@=11 %
  \catcode0=12 %
  \def\test#1#2{%
    \uccode0=#1\relax
    \uppercase{%
      \def\x{^^@}%
    }%
    \Expect*{%
       \ifcase\expandafter\PE@TestUcHexDigit\x
         true%
       \else
         false%
       \fi
    }{#2}%
  }%
  \def\range#1#2#3{%
    \count0=#1\relax
    \loop
    \ifnum\count0<#2\relax
      \test{\count0}{#3}%
      \advance\count0 by 1 %
    \repeat
  }%
  \range{0}{47}{false}%
  \range{48}{57}{true}%
  \range{58}{64}{false}%
  \range{65}{70}{true}%
  \range{71}{255}{false}%
\end{qstest}

\begin{qstest}{unescapename}{unescapename}
  \def\test#1#2{%
    \EdefUnescapeName\x{#1}%
    \edef\y{#2}%
    \@onelevel@sanitize\y
    \ExpectVar\x\y
  }%
  \catcode`\#=12 %
  \catcode0=12 %
  \test{}{}%
  \test{x}{x}%
  \test{xy}{xy}%
  \test{#}{#}%
  \test{##}{##}%
  \test{###}{###}%
  \test{####}{####}%
  \test{#x}{#x}%
  \test{#xy}{#xy}%
  \test{#1}{#1}%
  \test{#40}{@}%
  \test{#400}{@0}%
  \test{#4x0}{#4x0}%
  \test{#ab}{^^ab}%
  \test{#00}{^^@}%
  \test{x#40y#40z}{x@y@z}%
  \test{#40#40#40#40}{@@@@}%
  \test{a#x}{a#x}%
  \test{a#xy}{a#xy}%
  \test{a#1}{a#1}%
  \test{a#40}{a@}%
  \test{a#400}{a@0}%
  \test{#20}{ }%
  \test{a#20}{a }%
  \test{a#20b}{a b}%
  \test{a#20#20#20b}{a \space\space b}%
\end{qstest}

\begin{qstest}{unescapestring}{unescapestring}
  \def\test#1#2{%
    \EdefUnescapeString\x{#1}%
    \edef\y{#2}%
    \@onelevel@sanitize\y
    \ExpectVar\x\y
  }%
  \catcode0=12 %
  \def\DefChar#1#2{%
    \begingroup
      \uccode0=#2\relax
    \uppercase{\endgroup
      \def#1{^^@}%
    }%
  }%
  \DefChar\nul{0}%
  \DefChar\one{1}%
  \DefChar\bel{8}%
  \DefChar\tab{9}%
  \DefChar\lf{10}%
  \DefChar\ff{12}%
  \DefChar\cr{13}%
  \DefChar\\{92}%
  \test{}{}%
  \test{a}{a}%
  \test{\\}{}%
  \test{\\\\}{\\}%
  \test{\\\\y}{\\y}%
  \test{\\000}{\nul}%
  \test{\\b}{\bel}%
  \test{\\t}{\tab}%
  \test{\\n}{\lf}%
  \test{\\f}{\ff}%
  \test{\\r}{\cr}%
  \test{\\(}{(}%
  \test{\\)}{)}%
  \test{\\040}{ }%
  \test{\\100}{@}%
  \test{\\40}{ }%
  \test{\\1}{\one}%
  \test{\\01}{\one}%
  \test{\\001}{\one}%
  \test{\\18}{\one8}%
  \test{\\018}{\one8}%
  \test{\\0018}{\one8}%
  \test{x\\}{x}%
  \test{x\\\\}{x\\}%
  \test{x\\\\y}{x\\y}%
  \test{x\\000}{x\nul}%
  \test{x\\b}{x\bel}%
  \test{x\\t}{x\tab}%
  \test{x\\n}{x\lf}%
  \test{x\\f}{x\ff}%
  \test{x\\r}{x\cr}%
  \test{x\\(}{x(}%
  \test{x\\)}{x)}%
  \test{x\\040}{x }%
  \test{x\\100}{x@}%
  \test{x\\40}{x }%
  \test{x\\1}{x\one}%
  \test{x\\01}{x\one}%
  \test{x\\001}{x\one}%
  \test{x\\18}{x\one8}%
  \test{x\\018}{x\one8}%
  \test{x\\0018}{x\one8}%
  \test{\\b\\t\\n\\f\\r\\(\\)\\\\\\000\\040}{%
    \bel\tab\lf\ff\cr()\\\nul\space
  }%
  \test{\\\lf}{}%
  \test{x\\\lf}{x}%
  \test{\cr}{\lf}%
  \test{\cr\lf}{\lf}%
  \test{\lf}{\lf}%
  \test{\lf\cr}{\lf\lf}%
  \test{x\cr}{x\lf}%
  \test{x\cr\lf}{x\lf}%
  \test{x\lf}{x\lf}%
  \test{x\lf\cr}{x\lf\lf}%
  \test{x\\\cr\lf y\cr}{xy\lf}%
%
  \test{\\409}{ 9}%
  \test{\\800}{800}%
  \test{\\900}{900}%
  \test{\\578}{/8}%
  \test{\\477}{?}%
  \test{\\377}{^^ff}%
  \test{\\777}{^^ff}%
  \test{\\7777}{^^ff7}%
\end{qstest}
\stop
%    \end{macrocode}
%
%    \begin{macrocode}
%</test2|test3|test4|test5>
%    \end{macrocode}
%
% \subsubsection{Test for \hologo{iniTeX}}
%
%    \begin{macrocode}
%<*test6>
\input pdfescape.sty\relax
\catcode`\{=1 %
\catcode`\}=2 %
\catcode`\#=6 %
\catcode`\^=7 %
\catcode`\@=11 %
%    \end{macrocode}
%    \begin{macrocode}
\begingroup
  \catcode`\@=11 %
  \countdef\count@=255 %
  \def\space{ }%
  \long\def\@whilenum#1\do #2{%
    \ifnum #1\relax
      #2\relax
      \@iwhilenum{#1\relax#2\relax}%
    \fi
  }%
  \long\def\@iwhilenum#1{%
    \ifnum #1%
      \expandafter\@iwhilenum
    \else
      \expandafter\ltx@gobble
    \fi
    {#1}%
  }%
  \gdef\AllBytes{}%
  \count@=0 %
  \catcode0=12 %
  \@whilenum\count@<256 \do{%
    \lccode0=\count@
    \ifnum\count@=32 %
      \xdef\AllBytes{\AllBytes\space}%
    \else
      \lowercase{%
        \xdef\AllBytes{\AllBytes^^@}%
      }%
    \fi
    \advance\count@ by 1 %
  }%
\endgroup
\def\AllBytesHex{%
  000102030405060708090A0B0C0D0E0F%
  101112131415161718191A1B1C1D1E1F%
  202122232425262728292A2B2C2D2E2F%
  303132333435363738393A3B3C3D3E3F%
  404142434445464748494A4B4C4D4E4F%
  505152535455565758595A5B5C5D5E5F%
  606162636465666768696A6B6C6D6E6F%
  707172737475767778797A7B7C7D7E7F%
  808182838485868788898A8B8C8D8E8F%
  909192939495969798999A9B9C9D9E9F%
  A0A1A2A3A4A5A6A7A8A9AAABACADAEAF%
  B0B1B2B3B4B5B6B7B8B9BABBBCBDBEBF%
  C0C1C2C3C4C5C6C7C8C9CACBCCCDCECF%
  D0D1D2D3D4D5D6D7D8D9DADBDCDDDEDF%
  E0E1E2E3E4E5E6E7E8E9EAEBECEDEEEF%
  F0F1F2F3F4F5F6F7F8F9FAFBFCFDFEFF%
}
\ltx@onelevel@sanitize\AllBytesHex
\expandafter\lowercase\expandafter{%
  \expandafter\def\expandafter\AllBytesHexLC
      \expandafter{\AllBytesHex}%
}
\begingroup
  \catcode`\#=12 %
  \xdef\AllBytesName{%
    #01#02#03#04#05#06#07#08#09#0A#0B#0C#0D#0E#0F%
    #10#11#12#13#14#15#16#17#18#19#1A#1B#1C#1D#1E#1F%
    #20!"#23$#25&'#28#29*+,-.#2F%
    0123456789:;#3C=#3E?%
    @ABCDEFGHIJKLMNO%
    PQRSTUVWXYZ#5B\ltx@backslashchar#5D^_%
    `abcdefghijklmno%
    pqrstuvwxyz#7B|#7D\string~#7F%
    #80#81#82#83#84#85#86#87#88#89#8A#8B#8C#8D#8E#8F%
    #90#91#92#93#94#95#96#97#98#99#9A#9B#9C#9D#9E#9F%
    #A0#A1#A2#A3#A4#A5#A6#A7#A8#A9#AA#AB#AC#AD#AE#AF%
    #B0#B1#B2#B3#B4#B5#B6#B7#B8#B9#BA#BB#BC#BD#BE#BF%
    #C0#C1#C2#C3#C4#C5#C6#C7#C8#C9#CA#CB#CC#CD#CE#CF%
    #D0#D1#D2#D3#D4#D5#D6#D7#D8#D9#DA#DB#DC#DD#DE#DF%
    #E0#E1#E2#E3#E4#E5#E6#E7#E8#E9#EA#EB#EC#ED#EE#EF%
    #F0#F1#F2#F3#F4#F5#F6#F7#F8#F9#FA#FB#FC#FD#FE#FF%
  }%
\endgroup
\ltx@onelevel@sanitize\AllBytesName
\edef\AllBytesFromName{\expandafter\ltx@gobble\AllBytes}

\begingroup
  \def\|{|}%
  \edef\%{\ltx@percentchar}%
  \catcode`\|=0 %
  \catcode`\#=12 %
  \catcode`\~=12 %
  \catcode`\\=12 %
  |xdef|AllBytesString{%
    \000\001\002\003\004\005\006\007\010\011\012\013\014\015\016\017%
    \020\021\022\023\024\025\026\027\030\031\032\033\034\035\036\037%
    \040!"#$|%&'\(\)*+,-./%
    0123456789:;<=>?%
    @ABCDEFGHIJKLMNO%
    PQRSTUVWXYZ[\\]^_%
    `abcdefghijklmno%
    pqrstuvwxyz{||}~\177%
    \200\201\202\203\204\205\206\207\210\211\212\213\214\215\216\217%
    \220\221\222\223\224\225\226\227\230\231\232\233\234\235\236\237%
    \240\241\242\243\244\245\246\247\250\251\252\253\254\255\256\257%
    \260\261\262\263\264\265\266\267\270\271\272\273\274\275\276\277%
    \300\301\302\303\304\305\306\307\310\311\312\313\314\315\316\317%
    \320\321\322\323\324\325\326\327\330\331\332\333\334\335\336\337%
    \340\341\342\343\344\345\346\347\350\351\352\353\354\355\356\357%
    \360\361\362\363\364\365\366\367\370\371\372\373\374\375\376\377%
  }%
|endgroup
\ltx@onelevel@sanitize\AllBytesString
%    \end{macrocode}
%    \begin{macrocode}
\def\msg#{\immediate\write16}
\def\Test#1#2#3{%
  \begingroup
    #1\TestResult{#2}%
    \ifx\TestResult#3%
    \else
      \newlinechar=10 %
      \msg{Expect:^^J#3}%
      \msg{Result:^^J\TestResult}%
      \errmessage{\string#2 -\string#1-> \string#3}%
    \fi
  \endgroup
}
\Test\EdefEscapeHex\AllBytes\AllBytesHex
\Test\EdefUnescapeHex\AllBytesHex\AllBytes
\Test\EdefEscapeName\AllBytes\AllBytesName
\Test\EdefUnescapeName\AllBytesName\AllBytesFromName
\Test\EdefEscapeString\AllBytes\AllBytesString
\Test\EdefUnescapeString\AllBytesString\AllBytes
%    \end{macrocode}
%    \begin{macrocode}
\csname @@end\endcsname\end
%</test6>
%    \end{macrocode}
%
% \section{Installation}
%
% \subsection{Download}
%
% \paragraph{Package.} This package is available on
% CTAN\footnote{\url{ftp://ftp.ctan.org/tex-archive/}}:
% \begin{description}
% \item[\CTAN{macros/latex/contrib/oberdiek/pdfescape.dtx}] The source file.
% \item[\CTAN{macros/latex/contrib/oberdiek/pdfescape.pdf}] Documentation.
% \end{description}
%
%
% \paragraph{Bundle.} All the packages of the bundle `oberdiek'
% are also available in a TDS compliant ZIP archive. There
% the packages are already unpacked and the documentation files
% are generated. The files and directories obey the TDS standard.
% \begin{description}
% \item[\CTAN{install/macros/latex/contrib/oberdiek.tds.zip}]
% \end{description}
% \emph{TDS} refers to the standard ``A Directory Structure
% for \TeX\ Files'' (\CTAN{tds/tds.pdf}). Directories
% with \xfile{texmf} in their name are usually organized this way.
%
% \subsection{Bundle installation}
%
% \paragraph{Unpacking.} Unpack the \xfile{oberdiek.tds.zip} in the
% TDS tree (also known as \xfile{texmf} tree) of your choice.
% Example (linux):
% \begin{quote}
%   |unzip oberdiek.tds.zip -d ~/texmf|
% \end{quote}
%
% \paragraph{Script installation.}
% Check the directory \xfile{TDS:scripts/oberdiek/} for
% scripts that need further installation steps.
% Package \xpackage{attachfile2} comes with the Perl script
% \xfile{pdfatfi.pl} that should be installed in such a way
% that it can be called as \texttt{pdfatfi}.
% Example (linux):
% \begin{quote}
%   |chmod +x scripts/oberdiek/pdfatfi.pl|\\
%   |cp scripts/oberdiek/pdfatfi.pl /usr/local/bin/|
% \end{quote}
%
% \subsection{Package installation}
%
% \paragraph{Unpacking.} The \xfile{.dtx} file is a self-extracting
% \docstrip\ archive. The files are extracted by running the
% \xfile{.dtx} through \plainTeX:
% \begin{quote}
%   \verb|tex pdfescape.dtx|
% \end{quote}
%
% \paragraph{TDS.} Now the different files must be moved into
% the different directories in your installation TDS tree
% (also known as \xfile{texmf} tree):
% \begin{quote}
% \def\t{^^A
% \begin{tabular}{@{}>{\ttfamily}l@{ $\rightarrow$ }>{\ttfamily}l@{}}
%   pdfescape.sty & tex/generic/oberdiek/pdfescape.sty\\
%   pdfescape.pdf & doc/latex/oberdiek/pdfescape.pdf\\
%   test/pdfescape-test1.tex & doc/latex/oberdiek/test/pdfescape-test1.tex\\
%   test/pdfescape-test2.tex & doc/latex/oberdiek/test/pdfescape-test2.tex\\
%   test/pdfescape-test3.tex & doc/latex/oberdiek/test/pdfescape-test3.tex\\
%   test/pdfescape-test4.tex & doc/latex/oberdiek/test/pdfescape-test4.tex\\
%   test/pdfescape-test5.tex & doc/latex/oberdiek/test/pdfescape-test5.tex\\
%   test/pdfescape-test6.tex & doc/latex/oberdiek/test/pdfescape-test6.tex\\
%   pdfescape.dtx & source/latex/oberdiek/pdfescape.dtx\\
% \end{tabular}^^A
% }^^A
% \sbox0{\t}^^A
% \ifdim\wd0>\linewidth
%   \begingroup
%     \advance\linewidth by\leftmargin
%     \advance\linewidth by\rightmargin
%   \edef\x{\endgroup
%     \def\noexpand\lw{\the\linewidth}^^A
%   }\x
%   \def\lwbox{^^A
%     \leavevmode
%     \hbox to \linewidth{^^A
%       \kern-\leftmargin\relax
%       \hss
%       \usebox0
%       \hss
%       \kern-\rightmargin\relax
%     }^^A
%   }^^A
%   \ifdim\wd0>\lw
%     \sbox0{\small\t}^^A
%     \ifdim\wd0>\linewidth
%       \ifdim\wd0>\lw
%         \sbox0{\footnotesize\t}^^A
%         \ifdim\wd0>\linewidth
%           \ifdim\wd0>\lw
%             \sbox0{\scriptsize\t}^^A
%             \ifdim\wd0>\linewidth
%               \ifdim\wd0>\lw
%                 \sbox0{\tiny\t}^^A
%                 \ifdim\wd0>\linewidth
%                   \lwbox
%                 \else
%                   \usebox0
%                 \fi
%               \else
%                 \lwbox
%               \fi
%             \else
%               \usebox0
%             \fi
%           \else
%             \lwbox
%           \fi
%         \else
%           \usebox0
%         \fi
%       \else
%         \lwbox
%       \fi
%     \else
%       \usebox0
%     \fi
%   \else
%     \lwbox
%   \fi
% \else
%   \usebox0
% \fi
% \end{quote}
% If you have a \xfile{docstrip.cfg} that configures and enables \docstrip's
% TDS installing feature, then some files can already be in the right
% place, see the documentation of \docstrip.
%
% \subsection{Refresh file name databases}
%
% If your \TeX~distribution
% (\teTeX, \mikTeX, \dots) relies on file name databases, you must refresh
% these. For example, \teTeX\ users run \verb|texhash| or
% \verb|mktexlsr|.
%
% \subsection{Some details for the interested}
%
% \paragraph{Attached source.}
%
% The PDF documentation on CTAN also includes the
% \xfile{.dtx} source file. It can be extracted by
% AcrobatReader 6 or higher. Another option is \textsf{pdftk},
% e.g. unpack the file into the current directory:
% \begin{quote}
%   \verb|pdftk pdfescape.pdf unpack_files output .|
% \end{quote}
%
% \paragraph{Unpacking with \LaTeX.}
% The \xfile{.dtx} chooses its action depending on the format:
% \begin{description}
% \item[\plainTeX:] Run \docstrip\ and extract the files.
% \item[\LaTeX:] Generate the documentation.
% \end{description}
% If you insist on using \LaTeX\ for \docstrip\ (really,
% \docstrip\ does not need \LaTeX), then inform the autodetect routine
% about your intention:
% \begin{quote}
%   \verb|latex \let\install=y\input{pdfescape.dtx}|
% \end{quote}
% Do not forget to quote the argument according to the demands
% of your shell.
%
% \paragraph{Generating the documentation.}
% You can use both the \xfile{.dtx} or the \xfile{.drv} to generate
% the documentation. The process can be configured by the
% configuration file \xfile{ltxdoc.cfg}. For instance, put this
% line into this file, if you want to have A4 as paper format:
% \begin{quote}
%   \verb|\PassOptionsToClass{a4paper}{article}|
% \end{quote}
% An example follows how to generate the
% documentation with pdf\LaTeX:
% \begin{quote}
%\begin{verbatim}
%pdflatex pdfescape.dtx
%makeindex -s gind.ist pdfescape.idx
%pdflatex pdfescape.dtx
%makeindex -s gind.ist pdfescape.idx
%pdflatex pdfescape.dtx
%\end{verbatim}
% \end{quote}
%
% \section{Catalogue}
%
% The following XML file can be used as source for the
% \href{http://mirror.ctan.org/help/Catalogue/catalogue.html}{\TeX\ Catalogue}.
% The elements \texttt{caption} and \texttt{description} are imported
% from the original XML file from the Catalogue.
% The name of the XML file in the Catalogue is \xfile{pdfescape.xml}.
%    \begin{macrocode}
%<*catalogue>
<?xml version='1.0' encoding='us-ascii'?>
<!DOCTYPE entry SYSTEM 'catalogue.dtd'>
<entry datestamp='$Date$' modifier='$Author$' id='pdfescape'>
  <name>pdfescape</name>
  <caption>Implements pdfTeX's escape features using TeX or e-TeX.</caption>
  <authorref id='auth:oberdiek'/>
  <copyright owner='Heiko Oberdiek' year='2007,2010,2011'/>
  <license type='lppl1.3'/>
  <version number='1.13'/>
  <description>
    This package implements <xref refid='pdftex'>pdfTeX</xref>'s
    escape features (<tt>\pdfescapehex</tt>, <tt>\pdfunescapehex</tt>,
    <tt>\pdfescapename</tt>, <tt>\pdfescapestring</tt>) using TeX or
    e-TeX.
    <p/>
    The package is part of the <xref refid='oberdiek'>oberdiek</xref>
    bundle.
  </description>
  <documentation details='Package documentation'
      href='ctan:/macros/latex/contrib/oberdiek/pdfescape.pdf'/>
  <ctan file='true' path='/macros/latex/contrib/oberdiek/pdfescape.dtx'/>
  <miktex location='oberdiek'/>
  <texlive location='oberdiek'/>
  <install path='/macros/latex/contrib/oberdiek/oberdiek.tds.zip'/>
</entry>
%</catalogue>
%    \end{macrocode}
%
% \begin{History}
%   \begin{Version}{2007/02/21 v1.0}
%   \item
%     First version.
%   \end{Version}
%   \begin{Version}{2007/02/25 v1.1}
%   \item
%     Test files added.
%   \item
%     \cs{EdefUnescapeHex} supports lowercase letters.
%   \item
%     Fix: \verb|\EdefEscapeName{^^@}|
%   \item
%     Fix: \verb|\EdefEscapeName{\string#}|
%   \item
%     Fix for \cs{EdefUnescapeHex} in case of incomplete hex string.
%   \item
%     Fix: \cs{EdefUnescapeHex} generates space tokens with catcode
%     10 (space) in all cases.
%   \item
%     Fix: \cs{EdefEscapeHex} and \cs{EdefEscapeName} now generate
%     tokens with catcode 12 (other) only.
%   \end{Version}
%   \begin{Version}{2007/03/20 v1.2}
%   \item
%     Fix: Wrong year in \cs{ProvidesPackage}.
%   \end{Version}
%   \begin{Version}{2007/04/11 v1.3}
%   \item
%     Line ends sanitized.
%   \end{Version}
%   \begin{Version}{2007/04/21 v1.4}
%   \item
%     \cs{EdefUnescapeName} and \cs{EdefUnescapeString} added.
%   \end{Version}
%   \begin{Version}{2007/08/27 v1.5}
%   \item
%     \cs{EdefSanitize} added (replaces \cs{PE@sanitize}).
%   \end{Version}
%   \begin{Version}{2007/09/09 v1.6}
%   \item
%     Fix in catcode setup.
%   \end{Version}
%   \begin{Version}{2007/10/27 v1.7}
%   \item
%     More efficient \cs{EdefSanitize}.
%   \end{Version}
%   \begin{Version}{2007/11/11 v1.8}
%   \item
%     Use of package \xpackage{pdftexcmds} for \hologo{LuaTeX} support.
%   \end{Version}
%   \begin{Version}{2010/03/01 v1.9}
%   \item
%     Compatibility with \hologo{iniTeX}.
%   \end{Version}
%   \begin{Version}{2010/11/12 v1.10}
%   \item
%     Use of package \xpackage{ltxcmds}.
%   \item
%     Fix for compatibility with \hologo{iniTeX}.
%   \end{Version}
%   \begin{Version}{2011/01/30 v1.11}
%   \item
%     Already loaded package files are not input in \hologo{plainTeX}.
%   \end{Version}
%   \begin{Version}{2011/04/04 v1.12}
%   \item
%     Further fixes for compatibility for \hologo{iniTeX}.
%   \item
%     Test file for \hologo{iniTeX} added.
%   \end{Version}
%   \begin{Version}{2011/11/25 v1.13}
%   \item
%     Higher order bit of octal sequences in \cs{EdefUnescapeString}
%     ignored according to the PDF specification
%     (Bug found by Bruno Le Floch).
%   \end{Version}
% \end{History}
%
% \PrintIndex
%
% \Finale
\endinput
|
% \end{quote}
% Do not forget to quote the argument according to the demands
% of your shell.
%
% \paragraph{Generating the documentation.}
% You can use both the \xfile{.dtx} or the \xfile{.drv} to generate
% the documentation. The process can be configured by the
% configuration file \xfile{ltxdoc.cfg}. For instance, put this
% line into this file, if you want to have A4 as paper format:
% \begin{quote}
%   \verb|\PassOptionsToClass{a4paper}{article}|
% \end{quote}
% An example follows how to generate the
% documentation with pdf\LaTeX:
% \begin{quote}
%\begin{verbatim}
%pdflatex pdfescape.dtx
%makeindex -s gind.ist pdfescape.idx
%pdflatex pdfescape.dtx
%makeindex -s gind.ist pdfescape.idx
%pdflatex pdfescape.dtx
%\end{verbatim}
% \end{quote}
%
% \section{Catalogue}
%
% The following XML file can be used as source for the
% \href{http://mirror.ctan.org/help/Catalogue/catalogue.html}{\TeX\ Catalogue}.
% The elements \texttt{caption} and \texttt{description} are imported
% from the original XML file from the Catalogue.
% The name of the XML file in the Catalogue is \xfile{pdfescape.xml}.
%    \begin{macrocode}
%<*catalogue>
<?xml version='1.0' encoding='us-ascii'?>
<!DOCTYPE entry SYSTEM 'catalogue.dtd'>
<entry datestamp='$Date$' modifier='$Author$' id='pdfescape'>
  <name>pdfescape</name>
  <caption>Implements pdfTeX's escape features using TeX or e-TeX.</caption>
  <authorref id='auth:oberdiek'/>
  <copyright owner='Heiko Oberdiek' year='2007,2010,2011'/>
  <license type='lppl1.3'/>
  <version number='1.13'/>
  <description>
    This package implements <xref refid='pdftex'>pdfTeX</xref>'s
    escape features (<tt>\pdfescapehex</tt>, <tt>\pdfunescapehex</tt>,
    <tt>\pdfescapename</tt>, <tt>\pdfescapestring</tt>) using TeX or
    e-TeX.
    <p/>
    The package is part of the <xref refid='oberdiek'>oberdiek</xref>
    bundle.
  </description>
  <documentation details='Package documentation'
      href='ctan:/macros/latex/contrib/oberdiek/pdfescape.pdf'/>
  <ctan file='true' path='/macros/latex/contrib/oberdiek/pdfescape.dtx'/>
  <miktex location='oberdiek'/>
  <texlive location='oberdiek'/>
  <install path='/macros/latex/contrib/oberdiek/oberdiek.tds.zip'/>
</entry>
%</catalogue>
%    \end{macrocode}
%
% \begin{History}
%   \begin{Version}{2007/02/21 v1.0}
%   \item
%     First version.
%   \end{Version}
%   \begin{Version}{2007/02/25 v1.1}
%   \item
%     Test files added.
%   \item
%     \cs{EdefUnescapeHex} supports lowercase letters.
%   \item
%     Fix: \verb|\EdefEscapeName{^^@}|
%   \item
%     Fix: \verb|\EdefEscapeName{\string#}|
%   \item
%     Fix for \cs{EdefUnescapeHex} in case of incomplete hex string.
%   \item
%     Fix: \cs{EdefUnescapeHex} generates space tokens with catcode
%     10 (space) in all cases.
%   \item
%     Fix: \cs{EdefEscapeHex} and \cs{EdefEscapeName} now generate
%     tokens with catcode 12 (other) only.
%   \end{Version}
%   \begin{Version}{2007/03/20 v1.2}
%   \item
%     Fix: Wrong year in \cs{ProvidesPackage}.
%   \end{Version}
%   \begin{Version}{2007/04/11 v1.3}
%   \item
%     Line ends sanitized.
%   \end{Version}
%   \begin{Version}{2007/04/21 v1.4}
%   \item
%     \cs{EdefUnescapeName} and \cs{EdefUnescapeString} added.
%   \end{Version}
%   \begin{Version}{2007/08/27 v1.5}
%   \item
%     \cs{EdefSanitize} added (replaces \cs{PE@sanitize}).
%   \end{Version}
%   \begin{Version}{2007/09/09 v1.6}
%   \item
%     Fix in catcode setup.
%   \end{Version}
%   \begin{Version}{2007/10/27 v1.7}
%   \item
%     More efficient \cs{EdefSanitize}.
%   \end{Version}
%   \begin{Version}{2007/11/11 v1.8}
%   \item
%     Use of package \xpackage{pdftexcmds} for \hologo{LuaTeX} support.
%   \end{Version}
%   \begin{Version}{2010/03/01 v1.9}
%   \item
%     Compatibility with \hologo{iniTeX}.
%   \end{Version}
%   \begin{Version}{2010/11/12 v1.10}
%   \item
%     Use of package \xpackage{ltxcmds}.
%   \item
%     Fix for compatibility with \hologo{iniTeX}.
%   \end{Version}
%   \begin{Version}{2011/01/30 v1.11}
%   \item
%     Already loaded package files are not input in \hologo{plainTeX}.
%   \end{Version}
%   \begin{Version}{2011/04/04 v1.12}
%   \item
%     Further fixes for compatibility for \hologo{iniTeX}.
%   \item
%     Test file for \hologo{iniTeX} added.
%   \end{Version}
%   \begin{Version}{2011/11/25 v1.13}
%   \item
%     Higher order bit of octal sequences in \cs{EdefUnescapeString}
%     ignored according to the PDF specification
%     (Bug found by Bruno Le Floch).
%   \end{Version}
% \end{History}
%
% \PrintIndex
%
% \Finale
\endinput
|
% \end{quote}
% Do not forget to quote the argument according to the demands
% of your shell.
%
% \paragraph{Generating the documentation.}
% You can use both the \xfile{.dtx} or the \xfile{.drv} to generate
% the documentation. The process can be configured by the
% configuration file \xfile{ltxdoc.cfg}. For instance, put this
% line into this file, if you want to have A4 as paper format:
% \begin{quote}
%   \verb|\PassOptionsToClass{a4paper}{article}|
% \end{quote}
% An example follows how to generate the
% documentation with pdf\LaTeX:
% \begin{quote}
%\begin{verbatim}
%pdflatex pdfescape.dtx
%makeindex -s gind.ist pdfescape.idx
%pdflatex pdfescape.dtx
%makeindex -s gind.ist pdfescape.idx
%pdflatex pdfescape.dtx
%\end{verbatim}
% \end{quote}
%
% \section{Catalogue}
%
% The following XML file can be used as source for the
% \href{http://mirror.ctan.org/help/Catalogue/catalogue.html}{\TeX\ Catalogue}.
% The elements \texttt{caption} and \texttt{description} are imported
% from the original XML file from the Catalogue.
% The name of the XML file in the Catalogue is \xfile{pdfescape.xml}.
%    \begin{macrocode}
%<*catalogue>
<?xml version='1.0' encoding='us-ascii'?>
<!DOCTYPE entry SYSTEM 'catalogue.dtd'>
<entry datestamp='$Date$' modifier='$Author$' id='pdfescape'>
  <name>pdfescape</name>
  <caption>Implements pdfTeX's escape features using TeX or e-TeX.</caption>
  <authorref id='auth:oberdiek'/>
  <copyright owner='Heiko Oberdiek' year='2007,2010,2011'/>
  <license type='lppl1.3'/>
  <version number='1.13'/>
  <description>
    This package implements <xref refid='pdftex'>pdfTeX</xref>'s
    escape features (<tt>\pdfescapehex</tt>, <tt>\pdfunescapehex</tt>,
    <tt>\pdfescapename</tt>, <tt>\pdfescapestring</tt>) using TeX or
    e-TeX.
    <p/>
    The package is part of the <xref refid='oberdiek'>oberdiek</xref>
    bundle.
  </description>
  <documentation details='Package documentation'
      href='ctan:/macros/latex/contrib/oberdiek/pdfescape.pdf'/>
  <ctan file='true' path='/macros/latex/contrib/oberdiek/pdfescape.dtx'/>
  <miktex location='oberdiek'/>
  <texlive location='oberdiek'/>
  <install path='/macros/latex/contrib/oberdiek/oberdiek.tds.zip'/>
</entry>
%</catalogue>
%    \end{macrocode}
%
% \begin{History}
%   \begin{Version}{2007/02/21 v1.0}
%   \item
%     First version.
%   \end{Version}
%   \begin{Version}{2007/02/25 v1.1}
%   \item
%     Test files added.
%   \item
%     \cs{EdefUnescapeHex} supports lowercase letters.
%   \item
%     Fix: \verb|\EdefEscapeName{^^@}|
%   \item
%     Fix: \verb|\EdefEscapeName{\string#}|
%   \item
%     Fix for \cs{EdefUnescapeHex} in case of incomplete hex string.
%   \item
%     Fix: \cs{EdefUnescapeHex} generates space tokens with catcode
%     10 (space) in all cases.
%   \item
%     Fix: \cs{EdefEscapeHex} and \cs{EdefEscapeName} now generate
%     tokens with catcode 12 (other) only.
%   \end{Version}
%   \begin{Version}{2007/03/20 v1.2}
%   \item
%     Fix: Wrong year in \cs{ProvidesPackage}.
%   \end{Version}
%   \begin{Version}{2007/04/11 v1.3}
%   \item
%     Line ends sanitized.
%   \end{Version}
%   \begin{Version}{2007/04/21 v1.4}
%   \item
%     \cs{EdefUnescapeName} and \cs{EdefUnescapeString} added.
%   \end{Version}
%   \begin{Version}{2007/08/27 v1.5}
%   \item
%     \cs{EdefSanitize} added (replaces \cs{PE@sanitize}).
%   \end{Version}
%   \begin{Version}{2007/09/09 v1.6}
%   \item
%     Fix in catcode setup.
%   \end{Version}
%   \begin{Version}{2007/10/27 v1.7}
%   \item
%     More efficient \cs{EdefSanitize}.
%   \end{Version}
%   \begin{Version}{2007/11/11 v1.8}
%   \item
%     Use of package \xpackage{pdftexcmds} for \hologo{LuaTeX} support.
%   \end{Version}
%   \begin{Version}{2010/03/01 v1.9}
%   \item
%     Compatibility with \hologo{iniTeX}.
%   \end{Version}
%   \begin{Version}{2010/11/12 v1.10}
%   \item
%     Use of package \xpackage{ltxcmds}.
%   \item
%     Fix for compatibility with \hologo{iniTeX}.
%   \end{Version}
%   \begin{Version}{2011/01/30 v1.11}
%   \item
%     Already loaded package files are not input in \hologo{plainTeX}.
%   \end{Version}
%   \begin{Version}{2011/04/04 v1.12}
%   \item
%     Further fixes for compatibility for \hologo{iniTeX}.
%   \item
%     Test file for \hologo{iniTeX} added.
%   \end{Version}
%   \begin{Version}{2011/11/25 v1.13}
%   \item
%     Higher order bit of octal sequences in \cs{EdefUnescapeString}
%     ignored according to the PDF specification
%     (Bug found by Bruno Le Floch).
%   \end{Version}
% \end{History}
%
% \PrintIndex
%
% \Finale
\endinput
|
% \end{quote}
% Do not forget to quote the argument according to the demands
% of your shell.
%
% \paragraph{Generating the documentation.}
% You can use both the \xfile{.dtx} or the \xfile{.drv} to generate
% the documentation. The process can be configured by the
% configuration file \xfile{ltxdoc.cfg}. For instance, put this
% line into this file, if you want to have A4 as paper format:
% \begin{quote}
%   \verb|\PassOptionsToClass{a4paper}{article}|
% \end{quote}
% An example follows how to generate the
% documentation with pdf\LaTeX:
% \begin{quote}
%\begin{verbatim}
%pdflatex pdfescape.dtx
%makeindex -s gind.ist pdfescape.idx
%pdflatex pdfescape.dtx
%makeindex -s gind.ist pdfescape.idx
%pdflatex pdfescape.dtx
%\end{verbatim}
% \end{quote}
%
% \section{Catalogue}
%
% The following XML file can be used as source for the
% \href{http://mirror.ctan.org/help/Catalogue/catalogue.html}{\TeX\ Catalogue}.
% The elements \texttt{caption} and \texttt{description} are imported
% from the original XML file from the Catalogue.
% The name of the XML file in the Catalogue is \xfile{pdfescape.xml}.
%    \begin{macrocode}
%<*catalogue>
<?xml version='1.0' encoding='us-ascii'?>
<!DOCTYPE entry SYSTEM 'catalogue.dtd'>
<entry datestamp='$Date$' modifier='$Author$' id='pdfescape'>
  <name>pdfescape</name>
  <caption>Implements pdfTeX's escape features using TeX or e-TeX.</caption>
  <authorref id='auth:oberdiek'/>
  <copyright owner='Heiko Oberdiek' year='2007,2010,2011'/>
  <license type='lppl1.3'/>
  <version number='1.13'/>
  <description>
    This package implements <xref refid='pdftex'>pdfTeX</xref>'s
    escape features (<tt>\pdfescapehex</tt>, <tt>\pdfunescapehex</tt>,
    <tt>\pdfescapename</tt>, <tt>\pdfescapestring</tt>) using TeX or
    e-TeX.
    <p/>
    The package is part of the <xref refid='oberdiek'>oberdiek</xref>
    bundle.
  </description>
  <documentation details='Package documentation'
      href='ctan:/macros/latex/contrib/oberdiek/pdfescape.pdf'/>
  <ctan file='true' path='/macros/latex/contrib/oberdiek/pdfescape.dtx'/>
  <miktex location='oberdiek'/>
  <texlive location='oberdiek'/>
  <install path='/macros/latex/contrib/oberdiek/oberdiek.tds.zip'/>
</entry>
%</catalogue>
%    \end{macrocode}
%
% \begin{History}
%   \begin{Version}{2007/02/21 v1.0}
%   \item
%     First version.
%   \end{Version}
%   \begin{Version}{2007/02/25 v1.1}
%   \item
%     Test files added.
%   \item
%     \cs{EdefUnescapeHex} supports lowercase letters.
%   \item
%     Fix: \verb|\EdefEscapeName{^^@}|
%   \item
%     Fix: \verb|\EdefEscapeName{\string#}|
%   \item
%     Fix for \cs{EdefUnescapeHex} in case of incomplete hex string.
%   \item
%     Fix: \cs{EdefUnescapeHex} generates space tokens with catcode
%     10 (space) in all cases.
%   \item
%     Fix: \cs{EdefEscapeHex} and \cs{EdefEscapeName} now generate
%     tokens with catcode 12 (other) only.
%   \end{Version}
%   \begin{Version}{2007/03/20 v1.2}
%   \item
%     Fix: Wrong year in \cs{ProvidesPackage}.
%   \end{Version}
%   \begin{Version}{2007/04/11 v1.3}
%   \item
%     Line ends sanitized.
%   \end{Version}
%   \begin{Version}{2007/04/21 v1.4}
%   \item
%     \cs{EdefUnescapeName} and \cs{EdefUnescapeString} added.
%   \end{Version}
%   \begin{Version}{2007/08/27 v1.5}
%   \item
%     \cs{EdefSanitize} added (replaces \cs{PE@sanitize}).
%   \end{Version}
%   \begin{Version}{2007/09/09 v1.6}
%   \item
%     Fix in catcode setup.
%   \end{Version}
%   \begin{Version}{2007/10/27 v1.7}
%   \item
%     More efficient \cs{EdefSanitize}.
%   \end{Version}
%   \begin{Version}{2007/11/11 v1.8}
%   \item
%     Use of package \xpackage{pdftexcmds} for \hologo{LuaTeX} support.
%   \end{Version}
%   \begin{Version}{2010/03/01 v1.9}
%   \item
%     Compatibility with \hologo{iniTeX}.
%   \end{Version}
%   \begin{Version}{2010/11/12 v1.10}
%   \item
%     Use of package \xpackage{ltxcmds}.
%   \item
%     Fix for compatibility with \hologo{iniTeX}.
%   \end{Version}
%   \begin{Version}{2011/01/30 v1.11}
%   \item
%     Already loaded package files are not input in \hologo{plainTeX}.
%   \end{Version}
%   \begin{Version}{2011/04/04 v1.12}
%   \item
%     Further fixes for compatibility for \hologo{iniTeX}.
%   \item
%     Test file for \hologo{iniTeX} added.
%   \end{Version}
%   \begin{Version}{2011/11/25 v1.13}
%   \item
%     Higher order bit of octal sequences in \cs{EdefUnescapeString}
%     ignored according to the PDF specification
%     (Bug found by Bruno Le Floch).
%   \end{Version}
% \end{History}
%
% \PrintIndex
%
% \Finale
\endinput
