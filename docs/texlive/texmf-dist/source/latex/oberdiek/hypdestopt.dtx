% \iffalse meta-comment
%
% File: hypdestopt.dtx
% Version: 2011/05/13 v2.3
% Info: Hyperref destination optimizer
%
% Copyright (C) 2006-2008, 2011 by
%    Heiko Oberdiek <heiko.oberdiek at googlemail.com>
%
% This work may be distributed and/or modified under the
% conditions of the LaTeX Project Public License, either
% version 1.3c of this license or (at your option) any later
% version. This version of this license is in
%    http://www.latex-project.org/lppl/lppl-1-3c.txt
% and the latest version of this license is in
%    http://www.latex-project.org/lppl.txt
% and version 1.3 or later is part of all distributions of
% LaTeX version 2005/12/01 or later.
%
% This work has the LPPL maintenance status "maintained".
%
% This Current Maintainer of this work is Heiko Oberdiek.
%
% This work consists of the main source file hypdestopt.dtx
% and the derived files
%    hypdestopt.sty, hypdestopt.pdf, hypdestopt.ins, hypdestopt.drv.
%
% Distribution:
%    CTAN:macros/latex/contrib/oberdiek/hypdestopt.dtx
%    CTAN:macros/latex/contrib/oberdiek/hypdestopt.pdf
%
% Unpacking:
%    (a) If hypdestopt.ins is present:
%           tex hypdestopt.ins
%    (b) Without hypdestopt.ins:
%           tex hypdestopt.dtx
%    (c) If you insist on using LaTeX
%           latex \let\install=y% \iffalse meta-comment
%
% File: hypdestopt.dtx
% Version: 2011/05/13 v2.3
% Info: Hyperref destination optimizer
%
% Copyright (C) 2006-2008, 2011 by
%    Heiko Oberdiek <heiko.oberdiek at googlemail.com>
%
% This work may be distributed and/or modified under the
% conditions of the LaTeX Project Public License, either
% version 1.3c of this license or (at your option) any later
% version. This version of this license is in
%    http://www.latex-project.org/lppl/lppl-1-3c.txt
% and the latest version of this license is in
%    http://www.latex-project.org/lppl.txt
% and version 1.3 or later is part of all distributions of
% LaTeX version 2005/12/01 or later.
%
% This work has the LPPL maintenance status "maintained".
%
% This Current Maintainer of this work is Heiko Oberdiek.
%
% This work consists of the main source file hypdestopt.dtx
% and the derived files
%    hypdestopt.sty, hypdestopt.pdf, hypdestopt.ins, hypdestopt.drv.
%
% Distribution:
%    CTAN:macros/latex/contrib/oberdiek/hypdestopt.dtx
%    CTAN:macros/latex/contrib/oberdiek/hypdestopt.pdf
%
% Unpacking:
%    (a) If hypdestopt.ins is present:
%           tex hypdestopt.ins
%    (b) Without hypdestopt.ins:
%           tex hypdestopt.dtx
%    (c) If you insist on using LaTeX
%           latex \let\install=y% \iffalse meta-comment
%
% File: hypdestopt.dtx
% Version: 2011/05/13 v2.3
% Info: Hyperref destination optimizer
%
% Copyright (C) 2006-2008, 2011 by
%    Heiko Oberdiek <heiko.oberdiek at googlemail.com>
%
% This work may be distributed and/or modified under the
% conditions of the LaTeX Project Public License, either
% version 1.3c of this license or (at your option) any later
% version. This version of this license is in
%    http://www.latex-project.org/lppl/lppl-1-3c.txt
% and the latest version of this license is in
%    http://www.latex-project.org/lppl.txt
% and version 1.3 or later is part of all distributions of
% LaTeX version 2005/12/01 or later.
%
% This work has the LPPL maintenance status "maintained".
%
% This Current Maintainer of this work is Heiko Oberdiek.
%
% This work consists of the main source file hypdestopt.dtx
% and the derived files
%    hypdestopt.sty, hypdestopt.pdf, hypdestopt.ins, hypdestopt.drv.
%
% Distribution:
%    CTAN:macros/latex/contrib/oberdiek/hypdestopt.dtx
%    CTAN:macros/latex/contrib/oberdiek/hypdestopt.pdf
%
% Unpacking:
%    (a) If hypdestopt.ins is present:
%           tex hypdestopt.ins
%    (b) Without hypdestopt.ins:
%           tex hypdestopt.dtx
%    (c) If you insist on using LaTeX
%           latex \let\install=y% \iffalse meta-comment
%
% File: hypdestopt.dtx
% Version: 2011/05/13 v2.3
% Info: Hyperref destination optimizer
%
% Copyright (C) 2006-2008, 2011 by
%    Heiko Oberdiek <heiko.oberdiek at googlemail.com>
%
% This work may be distributed and/or modified under the
% conditions of the LaTeX Project Public License, either
% version 1.3c of this license or (at your option) any later
% version. This version of this license is in
%    http://www.latex-project.org/lppl/lppl-1-3c.txt
% and the latest version of this license is in
%    http://www.latex-project.org/lppl.txt
% and version 1.3 or later is part of all distributions of
% LaTeX version 2005/12/01 or later.
%
% This work has the LPPL maintenance status "maintained".
%
% This Current Maintainer of this work is Heiko Oberdiek.
%
% This work consists of the main source file hypdestopt.dtx
% and the derived files
%    hypdestopt.sty, hypdestopt.pdf, hypdestopt.ins, hypdestopt.drv.
%
% Distribution:
%    CTAN:macros/latex/contrib/oberdiek/hypdestopt.dtx
%    CTAN:macros/latex/contrib/oberdiek/hypdestopt.pdf
%
% Unpacking:
%    (a) If hypdestopt.ins is present:
%           tex hypdestopt.ins
%    (b) Without hypdestopt.ins:
%           tex hypdestopt.dtx
%    (c) If you insist on using LaTeX
%           latex \let\install=y\input{hypdestopt.dtx}
%        (quote the arguments according to the demands of your shell)
%
% Documentation:
%    (a) If hypdestopt.drv is present:
%           latex hypdestopt.drv
%    (b) Without hypdestopt.drv:
%           latex hypdestopt.dtx; ...
%    The class ltxdoc loads the configuration file ltxdoc.cfg
%    if available. Here you can specify further options, e.g.
%    use A4 as paper format:
%       \PassOptionsToClass{a4paper}{article}
%
%    Programm calls to get the documentation (example):
%       pdflatex hypdestopt.dtx
%       makeindex -s gind.ist hypdestopt.idx
%       pdflatex hypdestopt.dtx
%       makeindex -s gind.ist hypdestopt.idx
%       pdflatex hypdestopt.dtx
%
% Installation:
%    TDS:tex/latex/oberdiek/hypdestopt.sty
%    TDS:doc/latex/oberdiek/hypdestopt.pdf
%    TDS:source/latex/oberdiek/hypdestopt.dtx
%
%<*ignore>
\begingroup
  \catcode123=1 %
  \catcode125=2 %
  \def\x{LaTeX2e}%
\expandafter\endgroup
\ifcase 0\ifx\install y1\fi\expandafter
         \ifx\csname processbatchFile\endcsname\relax\else1\fi
         \ifx\fmtname\x\else 1\fi\relax
\else\csname fi\endcsname
%</ignore>
%<*install>
\input docstrip.tex
\Msg{************************************************************************}
\Msg{* Installation}
\Msg{* Package: hypdestopt 2011/05/13 v2.3 Hyperref destination optimizer (HO)}
\Msg{************************************************************************}

\keepsilent
\askforoverwritefalse

\let\MetaPrefix\relax
\preamble

This is a generated file.

Project: hypdestopt
Version: 2011/05/13 v2.3

Copyright (C) 2006-2008, 2011 by
   Heiko Oberdiek <heiko.oberdiek at googlemail.com>

This work may be distributed and/or modified under the
conditions of the LaTeX Project Public License, either
version 1.3c of this license or (at your option) any later
version. This version of this license is in
   http://www.latex-project.org/lppl/lppl-1-3c.txt
and the latest version of this license is in
   http://www.latex-project.org/lppl.txt
and version 1.3 or later is part of all distributions of
LaTeX version 2005/12/01 or later.

This work has the LPPL maintenance status "maintained".

This Current Maintainer of this work is Heiko Oberdiek.

This work consists of the main source file hypdestopt.dtx
and the derived files
   hypdestopt.sty, hypdestopt.pdf, hypdestopt.ins, hypdestopt.drv.

\endpreamble
\let\MetaPrefix\DoubleperCent

\generate{%
  \file{hypdestopt.ins}{\from{hypdestopt.dtx}{install}}%
  \file{hypdestopt.drv}{\from{hypdestopt.dtx}{driver}}%
  \usedir{tex/latex/oberdiek}%
  \file{hypdestopt.sty}{\from{hypdestopt.dtx}{package}}%
  \nopreamble
  \nopostamble
  \usedir{source/latex/oberdiek/catalogue}%
  \file{hypdestopt.xml}{\from{hypdestopt.dtx}{catalogue}}%
}

\catcode32=13\relax% active space
\let =\space%
\Msg{************************************************************************}
\Msg{*}
\Msg{* To finish the installation you have to move the following}
\Msg{* file into a directory searched by TeX:}
\Msg{*}
\Msg{*     hypdestopt.sty}
\Msg{*}
\Msg{* To produce the documentation run the file `hypdestopt.drv'}
\Msg{* through LaTeX.}
\Msg{*}
\Msg{* Happy TeXing!}
\Msg{*}
\Msg{************************************************************************}

\endbatchfile
%</install>
%<*ignore>
\fi
%</ignore>
%<*driver>
\NeedsTeXFormat{LaTeX2e}
\ProvidesFile{hypdestopt.drv}%
  [2011/05/13 v2.3 Hyperref destination optimizer (HO)]%
\documentclass{ltxdoc}
\usepackage{holtxdoc}[2011/11/22]
\begin{document}
  \DocInput{hypdestopt.dtx}%
\end{document}
%</driver>
% \fi
%
% \CheckSum{565}
%
% \CharacterTable
%  {Upper-case    \A\B\C\D\E\F\G\H\I\J\K\L\M\N\O\P\Q\R\S\T\U\V\W\X\Y\Z
%   Lower-case    \a\b\c\d\e\f\g\h\i\j\k\l\m\n\o\p\q\r\s\t\u\v\w\x\y\z
%   Digits        \0\1\2\3\4\5\6\7\8\9
%   Exclamation   \!     Double quote  \"     Hash (number) \#
%   Dollar        \$     Percent       \%     Ampersand     \&
%   Acute accent  \'     Left paren    \(     Right paren   \)
%   Asterisk      \*     Plus          \+     Comma         \,
%   Minus         \-     Point         \.     Solidus       \/
%   Colon         \:     Semicolon     \;     Less than     \<
%   Equals        \=     Greater than  \>     Question mark \?
%   Commercial at \@     Left bracket  \[     Backslash     \\
%   Right bracket \]     Circumflex    \^     Underscore    \_
%   Grave accent  \`     Left brace    \{     Vertical bar  \|
%   Right brace   \}     Tilde         \~}
%
% \GetFileInfo{hypdestopt.drv}
%
% \title{The \xpackage{hypdestopt} package}
% \date{2011/05/13 v2.3}
% \author{Heiko Oberdiek\\\xemail{heiko.oberdiek at googlemail.com}}
%
% \maketitle
%
% \begin{abstract}
% Package \xpackage{hypdestopt} supports \xpackage{hyperref}'s
% \xoption{pdftex} driver. It removes unnecessary destinations
% and shortens the destination names or uses numbered destinations
% to get smaller PDF files.
% \end{abstract}
%
% \tableofcontents
%
% \section{User interface}
%
% \subsection{Introduction}
%
% Before PDF-1.5 annotations and destinations cannot be compressed.
% If the destination names are not needed for external use, the
% file size can be decreased by the following means:
% \begin{itemize}
% \item Unused destinations are removed.
% \item The destination names are shortened (option \xoption{name}).
% \item Using numbered destinations (option \xoption{num}).
% \end{itemize}
%
% \subsection{Requirements}
%
% \begin{itemize}
% \item Package \xpackage{hyperref} 2006/06/01 v6.75a or newer
%       (\cite{hyperref}).
% \item Package \xpackage{alphalph} 2006/05/30 v1.4 or newer
%       (\cite{alphalph}), if option \xoption{name} is used.
% \item Package \xpackage{ifpdf} (\cite{ifpdf}).
% \item \pdfTeX\ 1.30.0 or newer.
% \item \pdfTeX\ in PDF mode.
% \item \eTeX\ extensions enabled.
% \item Probably an additional compile run of \pdfLaTeX\ is necessary.
% \end{itemize}
%
% In the first compile runs you can get warnings such as:
%\begin{quote}
%\begin{verbatim}
%! pdfTeX warning (dest): name{...} has been referenced ...
%\end{verbatim}
%\end{quote}
% These warnings should vanish in later compile runs.
% However these warnings also can occur without this package.
% The package does not cure them, thus these warnings will remain,
% but the destination name can be different. In such cases test
% without package, too.
%
% \subsection{Use}
%
% If the requirements are met, load the package:
%\begin{quote}
%\verb|\usepackage{hypdestopt}|
%\end{quote}
%
% The following options are supported:
% \begin{description}
% \item[\xoption{verbose}:] Verbose debug output is enabled and written
%   in the protocol file.
% \item[\xoption{num}:] Numbered destinations are used. The file size
%   is smaller, because names are no longer used.
%   This is the default.
% \item[\xoption{name}:] Destinations are identified by names.
% \end{description}
%
% \subsection{Limitations}
%
% \begin{itemize}
% \item Forget this package, if you need preserved destination names.
% \item Destination name strings use all bytes (0..255) except
%       the carriage return (13), left parenthesis (40), right
%       parenthesis (41), and backslash (92), because they
%       must be quoted in general and therefore occupy two bytes
%       instead of one.
%
%       Further the zero byte (0) is avoided for programs
%       that implement strings using zero terminated C strings.
%       And 255 (0xFF) is avoided to get rid of a possible
%       unicode marker at the begin.
%
%       So far I have not seen problems with:
%       \begin{itemize}
%       \item AcrobatReader 5.08/Linux
%       \item AcrobatReader 7.0/Linux
%       \item xpdf 3.00
%       \item Ghostscript 8.50
%       \item gv 3.5.8
%       \item GSview 4.6
%       \end{itemize}
%       But I have not tested all and all possible PDF viewers.
% \item Use of named destinations (\cs{pdfdest}, \cs{pdfoutline},
%       \cs{pdfstartlink}, \dots) that are not supported by this
%       package.
% \item Currently only \xpackage{hyperref} with \pdfTeX\ in PDF
%       mode is supported.
% \end{itemize}
%
% \subsection{Future}
%
% A more general approach is a PDF postprocessor that takes
% a PDF file, performs some transformations and writes the
% result in a more optimized PDF file. Then it does not depend,
% how the original PDF file was generated and further improvements
% are easier to apply. For example, the destination names could be sorted:
% often used destination names would then be shorter than seldom used ones.
%
% \StopEventually{
% }
%
% \section{Implementation}
%
% \subsection{Identification}
%
%    \begin{macrocode}
%<*package>
\NeedsTeXFormat{LaTeX2e}
\ProvidesPackage{hypdestopt}%
  [2011/05/13 v2.3 Hyperref destination optimizer (HO)]%
%    \end{macrocode}
%
% \subsection{Options}
%
% \subsubsection{Option \xoption{verbose}}
%
%    \begin{macrocode}
\newif\ifHypDest@Verbose
\DeclareOption{verbose}{\HypDest@Verbosetrue}
%    \end{macrocode}
%
%    \begin{macro}{\HypDest@VerboseInfo}
%    Wrapper for verbose messages.
%    \begin{macrocode}
\def\HypDest@VerboseInfo#1{%
  \ifHypDest@Verbose
    \PackageInfo{hypdestopt}{#1}%
  \fi
}
%    \end{macrocode}
%    \end{macro}
%
% \subsubsection{Options \xoption{num} and \xoption{name}}
%
%    The options \xoption{num} or \xoption{name} specify
%    the method, how destinations are referenced (by name or
%    number). Default is option \xoption{num}.
%    \begin{macrocode}
\newif\ifHypDest@name
\DeclareOption{num}{\HypDest@namefalse}
\DeclareOption{name}{\HypDest@nametrue}
%    \end{macrocode}
%
%    \begin{macrocode}
\ProcessOptions*\relax
%    \end{macrocode}
%
% \subsection{Check requirements}
%
%    First \pdfTeX\ must running in PDF mode.
%    \begin{macrocode}
\RequirePackage{ifpdf}[2007/09/09]
\RequirePackage{pdftexcmds}[2007/11/11]
\ifpdf
\else
  \PackageError{hypdestopt}{%
    This package requires pdfTeX in PDF mode%
  }\@ehc
  \expandafter\endinput
\fi
%    \end{macrocode}
%    The version of \pdfTeX\ must not be too old, because
%    \cs{pdfescapehex} and \cs{pdfunescapehex} are used.
%    \begin{macrocode}
\begingroup\expandafter\expandafter\expandafter\endgroup
\expandafter\ifx\csname pdf@escapehex\endcsname\relax
  \PackageError{hypdestopt}{%
    This pdfTeX is too old, at least 1.30.0 is required%
  }\@ehc
  \expandafter\endinput
\fi
%    \end{macrocode}
%    Features of \eTeX\ are used, e.g. \cs{numexpr}.
%    \begin{macrocode}
\begingroup\expandafter\expandafter\expandafter\endgroup
\expandafter\ifx\csname numexpr\endcsname\relax
  \PackageError{hypdestopt}{%
    e-TeX features are missing%
  }\@ehc
  \expandafter\endinput
\fi
%    \end{macrocode}
%    Package \xpackage{alphalph} provides \cs{newalphalph} since
%    version 2006/05/30 v1.4.
%    \begin{macrocode}
\ifHypDest@name
  \RequirePackage{alphalph}[2006/05/30]%
\fi
%    \end{macrocode}
%    \begin{macrocode}
\RequirePackage{auxhook}[2009/12/14]
\RequirePackage{pdfescape}[2007/04/21]
%    \end{macrocode}
%
% \subsection{Preamble for auxiliary file}
%
%    Provide dummy definitions for the macros that are used in the
%    auxiliary files. If the package is used no longer, then these
%    commands will not generate errors.
%
%    \begin{macro}{\HypDest@PrependDocument}
%    We add our stuff in front of the \cs{AtBeginDocument} hook
%    to ensure that we are before \xpackage{hyperref}'s stuff.
%    \begin{macrocode}
\long\def\HypDest@PrependDocument#1{%
  \begingroup
    \toks\z@{#1}%
    \toks\tw@\expandafter{\@begindocumenthook}%
    \xdef\@begindocumenthook{\the\toks\z@\the\toks\tw@}%
  \endgroup
}
%    \end{macrocode}
%    \end{macro}
%    \begin{macrocode}
\AddLineBeginAux{%
  \string\providecommand{\string\HypDest@Use}[1]{}%
}
%    \end{macrocode}
%
% \subsection{Generation of destination names}
%
%    Counter |HypDest| is used for identifying destinations.
%    \begin{macrocode}
\newcounter{HypDest}
%    \end{macrocode}
%
%    \begin{macrocode}
\ifHypDest@name
%    \end{macrocode}
%
%    \begin{macro}{\HypDest@HexChar}
%    Destination names are generated by automatically
%    numbering with the help of package \xpackage{alphalph}.
%    \cs{HypDest@HexChar} converts a number of the range 1 until 252
%    into the hexadecimal representation of the string character.
%    \begin{macrocode}
  \def\HypDest@HexChar#1{%
    \ifcase#1\or
%    \end{macrocode}
%    Avoid zero byte because of C strings in PDF viewer
%    applications.
%    \begin{macrocode}
      01\or 02\or 03\or 04\or 05\or 06\or 07\or
%    \end{macrocode}
%    Omit carriage return (13/\verb|^^0d|).
%    It needs quoting, otherwise it would be converted
%    to line feed (10/\verb|^^0a|).
%    \begin{macrocode}
      08\or 09\or 0A\or 0B\or 0C\or 0E\or 0F\or
      10\or 11\or 12\or 13\or 14\or 15\or 16\or 17\or
      18\or 19\or 1A\or 1B\or 1C\or 1D\or 1E\or 1F\or
      20\or 21\or 22\or 23\or 24\or 25\or 26\or 27\or
%    \end{macrocode}
%    Omit left and right parentheses (40/\verb|^^28|, 41/\verb|^^39|),
%    they need quoting in general.
%    \begin{macrocode}
      2A\or 2B\or 2C\or 2D\or 2E\or 2F\or
      30\or 31\or 32\or 33\or 34\or 35\or 36\or 37\or
      38\or 39\or 3A\or 3B\or 3C\or 3D\or 3E\or 3F\or
      40\or 41\or 42\or 43\or 44\or 45\or 46\or 47\or
      48\or 49\or 4A\or 4B\or 4C\or 4D\or 4E\or 4F\or
      50\or 51\or 52\or 53\or 54\or 55\or 56\or 57\or
%    \end{macrocode}
%    Omit backslash (92/\verb|^^5C|), it needs quoting.
%    \begin{macrocode}
      58\or 59\or 5A\or 5B\or 5D\or 5E\or 5F\or
      60\or 61\or 62\or 63\or 64\or 65\or 66\or 67\or
      68\or 69\or 6A\or 6B\or 6C\or 6D\or 6E\or 6F\or
      70\or 71\or 72\or 73\or 74\or 75\or 76\or 77\or
      78\or 79\or 7A\or 7B\or 7C\or 7D\or 7E\or 7F\or
      80\or 81\or 82\or 83\or 84\or 85\or 86\or 87\or
      88\or 89\or 8A\or 8B\or 8C\or 8D\or 8E\or 8F\or
      90\or 91\or 92\or 93\or 94\or 95\or 96\or 97\or
      98\or 99\or 9A\or 9B\or 9C\or 9D\or 9E\or 9F\or
      A0\or A1\or A2\or A3\or A4\or A5\or A6\or A7\or
      A8\or A9\or AA\or AB\or AC\or AD\or AE\or AF\or
      B0\or B1\or B2\or B3\or B4\or B5\or B6\or B7\or
      B8\or B9\or BA\or BB\or BC\or BD\or BE\or BF\or
      C0\or C1\or C2\or C3\or C4\or C5\or C6\or C7\or
      C8\or C9\or CA\or CB\or CC\or CD\or CE\or CF\or
      D0\or D1\or D2\or D3\or D4\or D5\or D6\or D7\or
      D8\or D9\or DA\or DB\or DC\or DD\or DE\or DF\or
      E0\or E1\or E2\or E3\or E4\or E5\or E6\or E7\or
      E8\or E9\or EA\or EB\or EC\or ED\or EE\or EF\or
      F0\or F1\or F2\or F3\or F4\or F5\or F6\or F7\or
%    \end{macrocode}
%    Avoid 255 (0xFF) to get rid of a possible unicode
%    marker at the begin of the string.
%    \begin{macrocode}
      F8\or F9\or FA\or FB\or FC\or FD\or FE%
    \fi
  }%
%    \end{macrocode}
%    \end{macro}
%    \begin{macro}{HypDest@HexString}
%    Now package \xpackage{alphalph} comes into play.
%    \cs{HypDest@HexString} is defined and converts
%    a positive number into a string, given in hexadecimal
%    representation.
%    \begin{macrocode}
  \newalphalph\HypDest@HexString\HypDest@HexChar{250}%
%    \end{macrocode}
%    \end{macro}
%    \begin{macro}{\theHypDest}
%    For use, the hexadecimal string is converted back.
%    \begin{macrocode}
  \renewcommand*{\theHypDest}{%
    \pdf@unescapehex{\HypDest@HexString{\value{HypDest}}}%
  }%
%    \end{macrocode}
%    \end{macro}
%
%    With option \xoption{num} we use the number directly.
%    \begin{macrocode}
\else
  \renewcommand*{\theHypDest}{%
    \number\value{HypDest}%
  }%
\fi
%    \end{macrocode}
%
% \subsection{Assign destination names}
%
%    \begin{macro}{\HypDest@Prefix}
%    The new destination names are remembered in macros whose names
%    start with prefix \cs{HypDest@Prefix}.
%    \begin{macrocode}
\edef\HypDest@Prefix{HypDest\string:}
%    \end{macrocode}
%    \end{macro}
%
%    \begin{macro}{\HypDest@Use}
%    During the first read of the auxiliary files, the used destinations
%    get fresh generated short destination names. Also for the old
%    destination names we use the hexadecimal representation. That
%    avoid problems with arbitrary names.
%    \begin{macrocode}
\def\HypDest@Use#1{%
  \begingroup
    \edef\x{%
      \expandafter\noexpand
      \csname\HypDest@Prefix\pdf@unescapehex{#1}\endcsname
    }%
    \expandafter\ifx\x\relax
      \stepcounter{HypDest}%
      \expandafter\xdef\x{\theHypDest}%
      \let\on@line\@empty
      \ifHypDest@name
        \HypDest@VerboseInfo{%
          Use: (\pdf@unescapehex{#1}) -\string> %
          0x\pdf@escapehex{\x} (\number\value{HypDest})%
        }%
      \else
        \HypDest@VerboseInfo{%
          Use: (\pdf@unescapehex{#1}) -\string> num \x
        }%
      \fi
    \fi
  \endgroup
}
%    \end{macrocode}
%    \end{macro}
%
%    After the first \xfile{.aux} file processing the destination names
%    are assigned and we can disable \cs{HypDest@Use}.
%    \begin{macrocode}
\AtBeginDocument{%
  \let\HypDest@Use\@gobble
}
%    \end{macrocode}
%
%    \begin{macro}{\HypDest@MarkUsed}
%    Destinations that are actually used are marked by \cs{HypDest@MarkUsed}.
%    \cs{nofiles} is respected.
%    \begin{macrocode}
\def\HypDest@MarkUsed#1{%
  \HypDest@VerboseInfo{%
    MarkUsed: (#1)%
  }%
  \if@filesw
    \immediate\write\@auxout{%
      \string\HypDest@Use{\pdf@escapehex{#1}}%
    }%
  \fi
}%
%    \end{macrocode}
%    \end{macro}
%
% \subsection{Redefinition of \xpackage{hyperref}'s hooks}
%
%    Package \xpackage{hyperref} can be loaded later, therefore
%    we redefine \xpackage{hyperref}'s macros at |\begin{document}|.
%    \begin{macrocode}
\HypDest@PrependDocument{%
%    \end{macrocode}
%
%    Check hyperref version.
%    \begin{macrocode}
  \@ifpackagelater{hyperref}{2006/06/01}{}{%
    \PackageError{hypdestopt}{%
      hyperref 2006/06/01 v6.75a or later is required%
    }\@ehc
  }%
%    \end{macrocode}
%
% \subsubsection{Destination setting}
%
%    \begin{macrocode}
  \ifHypDest@name
    \let\HypDest@Org@DestName\Hy@DestName
    \renewcommand*{\Hy@DestName}[2]{%
      \EdefUnescapeString\HypDest@temp{#1}%
      \@ifundefined{\HypDest@Prefix\HypDest@temp}{%
        \HypDest@VerboseInfo{%
          DestName: (\HypDest@temp) unused%
        }%
      }{%
        \HypDest@Org@DestName{%
          \csname\HypDest@Prefix\HypDest@temp\endcsname
        }{#2}%
        \HypDest@VerboseInfo{%
          DestName: (\HypDest@temp) %
          0x\pdf@escapehex{%
            \csname\HypDest@Prefix\HypDest@temp\endcsname
          }%
        }%
      }%
    }%
  \else
    \renewcommand*{\Hy@DestName}[2]{%
      \EdefUnescapeString\HypDest@temp{#1}%
      \@ifundefined{\HypDest@Prefix\HypDest@temp}{%
        \HypDest@VerboseInfo{%
          DestName: (\HypDest@temp) unused%
        }%
      }{%
        \pdfdest num%
        \csname\HypDest@Prefix\HypDest@temp\endcsname#2\relax
        \HypDest@VerboseInfo{%
          DestName: (\HypDest@temp) %
          num \csname\HypDest@Prefix\HypDest@temp\endcsname
        }%
      }%
    }%
  \fi
%    \end{macrocode}
%
% \subsubsection{Links}
%
%    \begin{macrocode}
  \let\HypDest@Org@StartlinkName\Hy@StartlinkName
  \ifHypDest@name
    \renewcommand*{\Hy@StartlinkName}[2]{%
      \HypDest@MarkUsed{#2}%
      \HypDest@Org@StartlinkName{#1}{%
        \@ifundefined{\HypDest@Prefix#2}{%
          #2%
        }{%
          \csname\HypDest@Prefix#2\endcsname
        }%
      }%
    }%
  \else
    \renewcommand*{\Hy@StartlinkName}[2]{%
      \HypDest@MarkUsed{#2}%
      \@ifundefined{\HypDest@Prefix#2}{%
        \HypDest@Org@StartlinkName{#1}{#2}%
      }{%
        \pdfstartlink attr{#1}%
                      goto num\csname\HypDest@Prefix#2\endcsname
        \relax
      }%
    }%
  \fi
%    \end{macrocode}
%
% \subsubsection{Outlines of package \xpackage{hyperref}}
%
%    \begin{macrocode}
  \let\HypDest@Org@OutlineName\Hy@OutlineName
  \ifHypDest@name
    \renewcommand*{\Hy@OutlineName}[4]{%
      \HypDest@Org@OutlineName{#1}{%
        \@ifundefined{\HypDest@Prefix#2}{%
          #2%
        }{%
          \csname\HypDest@Prefix#2\endcsname
        }%
      }{#3}{#4}%
    }%
  \else
    \renewcommand*{\Hy@OutlineName}[4]{%
      \@ifundefined{\HypDest@Prefix#2}{%
        \HypDest@Org@OutlineName{#1}{#2}{#3}{#4}%
      }{%
        \pdfoutline goto num\csname\HypDest@Prefix#2\endcsname
                    count#3{#4}%
      }%
    }%
  \fi
%    \end{macrocode}
%    Because \cs{Hy@OutlineName} is called after the \xfile{.out} file
%    is written in the previous run. Therefore we mark the destination
%    earlier in \cs{@@writetorep}.
%    \begin{macrocode}
  \let\HypDest@Org@@writetorep\@@writetorep
  \renewcommand*{\@@writetorep}[5]{%
    \begingroup
      \edef\Hy@tempa{#5}%
      \ifx\Hy@tempa\Hy@bookmarkstype
        \HypDest@MarkUsed{#3}%
      \fi
    \endgroup
    \HypDest@Org@@writetorep{#1}{#2}{#3}{#4}{#5}%
  }%
%    \end{macrocode}
%
% \subsubsection{Outlines of package \xpackage{bookmark}}
%
%    \begin{macrocode}
  \@ifpackageloaded{bookmark}{%
    \@ifpackagelater{bookmark}{2008/08/08}{%
      \renewcommand*{\BKM@DefGotoNameAction}[2]{%
        \@ifundefined{\HypDest@Prefix#2}{%
          \edef#1{goto name{hypdestopt\string :unknown}}%
        }{%
          \ifHypDest@name
            \edef#1{goto name{\csname\HypDest@Prefix#2\endcsname}}%
          \else
            \edef#1{goto num\csname\HypDest@Prefix#2\endcsname}%
          \fi
        }%
      }%
      \def\BKM@HypDestOptHook{%
        \ifx\BKM@dest\@empty
        \else
          \ifx\BKM@gotor\@empty
            \HypDest@MarkUsed\BKM@dest
          \fi
        \fi
      }%
    }{%
      \@PackageError{hypdestopt}{%
        Package `bookmark' is too old.\MessageBreak
        Version 2008/08/08 or later is needed%
      }\@ehc
    }%
  }{}%
%    \end{macrocode}
%
%    \begin{macrocode}
}
%    \end{macrocode}
%
%
%    \begin{macrocode}
%</package>
%    \end{macrocode}
%
% \section{Installation}
%
% \subsection{Download}
%
% \paragraph{Package.} This package is available on
% CTAN\footnote{\url{ftp://ftp.ctan.org/tex-archive/}}:
% \begin{description}
% \item[\CTAN{macros/latex/contrib/oberdiek/hypdestopt.dtx}] The source file.
% \item[\CTAN{macros/latex/contrib/oberdiek/hypdestopt.pdf}] Documentation.
% \end{description}
%
%
% \paragraph{Bundle.} All the packages of the bundle `oberdiek'
% are also available in a TDS compliant ZIP archive. There
% the packages are already unpacked and the documentation files
% are generated. The files and directories obey the TDS standard.
% \begin{description}
% \item[\CTAN{install/macros/latex/contrib/oberdiek.tds.zip}]
% \end{description}
% \emph{TDS} refers to the standard ``A Directory Structure
% for \TeX\ Files'' (\CTAN{tds/tds.pdf}). Directories
% with \xfile{texmf} in their name are usually organized this way.
%
% \subsection{Bundle installation}
%
% \paragraph{Unpacking.} Unpack the \xfile{oberdiek.tds.zip} in the
% TDS tree (also known as \xfile{texmf} tree) of your choice.
% Example (linux):
% \begin{quote}
%   |unzip oberdiek.tds.zip -d ~/texmf|
% \end{quote}
%
% \paragraph{Script installation.}
% Check the directory \xfile{TDS:scripts/oberdiek/} for
% scripts that need further installation steps.
% Package \xpackage{attachfile2} comes with the Perl script
% \xfile{pdfatfi.pl} that should be installed in such a way
% that it can be called as \texttt{pdfatfi}.
% Example (linux):
% \begin{quote}
%   |chmod +x scripts/oberdiek/pdfatfi.pl|\\
%   |cp scripts/oberdiek/pdfatfi.pl /usr/local/bin/|
% \end{quote}
%
% \subsection{Package installation}
%
% \paragraph{Unpacking.} The \xfile{.dtx} file is a self-extracting
% \docstrip\ archive. The files are extracted by running the
% \xfile{.dtx} through \plainTeX:
% \begin{quote}
%   \verb|tex hypdestopt.dtx|
% \end{quote}
%
% \paragraph{TDS.} Now the different files must be moved into
% the different directories in your installation TDS tree
% (also known as \xfile{texmf} tree):
% \begin{quote}
% \def\t{^^A
% \begin{tabular}{@{}>{\ttfamily}l@{ $\rightarrow$ }>{\ttfamily}l@{}}
%   hypdestopt.sty & tex/latex/oberdiek/hypdestopt.sty\\
%   hypdestopt.pdf & doc/latex/oberdiek/hypdestopt.pdf\\
%   hypdestopt.dtx & source/latex/oberdiek/hypdestopt.dtx\\
% \end{tabular}^^A
% }^^A
% \sbox0{\t}^^A
% \ifdim\wd0>\linewidth
%   \begingroup
%     \advance\linewidth by\leftmargin
%     \advance\linewidth by\rightmargin
%   \edef\x{\endgroup
%     \def\noexpand\lw{\the\linewidth}^^A
%   }\x
%   \def\lwbox{^^A
%     \leavevmode
%     \hbox to \linewidth{^^A
%       \kern-\leftmargin\relax
%       \hss
%       \usebox0
%       \hss
%       \kern-\rightmargin\relax
%     }^^A
%   }^^A
%   \ifdim\wd0>\lw
%     \sbox0{\small\t}^^A
%     \ifdim\wd0>\linewidth
%       \ifdim\wd0>\lw
%         \sbox0{\footnotesize\t}^^A
%         \ifdim\wd0>\linewidth
%           \ifdim\wd0>\lw
%             \sbox0{\scriptsize\t}^^A
%             \ifdim\wd0>\linewidth
%               \ifdim\wd0>\lw
%                 \sbox0{\tiny\t}^^A
%                 \ifdim\wd0>\linewidth
%                   \lwbox
%                 \else
%                   \usebox0
%                 \fi
%               \else
%                 \lwbox
%               \fi
%             \else
%               \usebox0
%             \fi
%           \else
%             \lwbox
%           \fi
%         \else
%           \usebox0
%         \fi
%       \else
%         \lwbox
%       \fi
%     \else
%       \usebox0
%     \fi
%   \else
%     \lwbox
%   \fi
% \else
%   \usebox0
% \fi
% \end{quote}
% If you have a \xfile{docstrip.cfg} that configures and enables \docstrip's
% TDS installing feature, then some files can already be in the right
% place, see the documentation of \docstrip.
%
% \subsection{Refresh file name databases}
%
% If your \TeX~distribution
% (\teTeX, \mikTeX, \dots) relies on file name databases, you must refresh
% these. For example, \teTeX\ users run \verb|texhash| or
% \verb|mktexlsr|.
%
% \subsection{Some details for the interested}
%
% \paragraph{Attached source.}
%
% The PDF documentation on CTAN also includes the
% \xfile{.dtx} source file. It can be extracted by
% AcrobatReader 6 or higher. Another option is \textsf{pdftk},
% e.g. unpack the file into the current directory:
% \begin{quote}
%   \verb|pdftk hypdestopt.pdf unpack_files output .|
% \end{quote}
%
% \paragraph{Unpacking with \LaTeX.}
% The \xfile{.dtx} chooses its action depending on the format:
% \begin{description}
% \item[\plainTeX:] Run \docstrip\ and extract the files.
% \item[\LaTeX:] Generate the documentation.
% \end{description}
% If you insist on using \LaTeX\ for \docstrip\ (really,
% \docstrip\ does not need \LaTeX), then inform the autodetect routine
% about your intention:
% \begin{quote}
%   \verb|latex \let\install=y\input{hypdestopt.dtx}|
% \end{quote}
% Do not forget to quote the argument according to the demands
% of your shell.
%
% \paragraph{Generating the documentation.}
% You can use both the \xfile{.dtx} or the \xfile{.drv} to generate
% the documentation. The process can be configured by the
% configuration file \xfile{ltxdoc.cfg}. For instance, put this
% line into this file, if you want to have A4 as paper format:
% \begin{quote}
%   \verb|\PassOptionsToClass{a4paper}{article}|
% \end{quote}
% An example follows how to generate the
% documentation with pdf\LaTeX:
% \begin{quote}
%\begin{verbatim}
%pdflatex hypdestopt.dtx
%makeindex -s gind.ist hypdestopt.idx
%pdflatex hypdestopt.dtx
%makeindex -s gind.ist hypdestopt.idx
%pdflatex hypdestopt.dtx
%\end{verbatim}
% \end{quote}
%
% \section{Catalogue}
%
% The following XML file can be used as source for the
% \href{http://mirror.ctan.org/help/Catalogue/catalogue.html}{\TeX\ Catalogue}.
% The elements \texttt{caption} and \texttt{description} are imported
% from the original XML file from the Catalogue.
% The name of the XML file in the Catalogue is \xfile{hypdestopt.xml}.
%    \begin{macrocode}
%<*catalogue>
<?xml version='1.0' encoding='us-ascii'?>
<!DOCTYPE entry SYSTEM 'catalogue.dtd'>
<entry datestamp='$Date$' modifier='$Author$' id='hypdestopt'>
  <name>hypdestopt</name>
  <caption>Hyperref destination optimizer.</caption>
  <authorref id='auth:oberdiek'/>
  <copyright owner='Heiko Oberdiek' year='2006-2008,2011'/>
  <license type='lppl1.3'/>
  <version number='2.3'/>
  <description>
    This package supports <xref refid='hyperref'>hyperref</xref>'s
    pdftex driver. It removes unnecessary destinations
    and shortens the destination names or uses numbered destinations
    to get smaller PDF files.
    <p/>
    The package is part of the <xref refid='oberdiek'>oberdiek</xref>
    bundle.
  </description>
  <documentation details='Package documentation'
      href='ctan:/macros/latex/contrib/oberdiek/hypdestopt.pdf'/>
  <ctan file='true' path='/macros/latex/contrib/oberdiek/hypdestopt.dtx'/>
  <miktex location='oberdiek'/>
  <texlive location='oberdiek'/>
  <install path='/macros/latex/contrib/oberdiek/oberdiek.tds.zip'/>
</entry>
%</catalogue>
%    \end{macrocode}
%
% \begin{thebibliography}{9}
%
% \bibitem{alphalph}
%   Heiko Oberdiek: \textit{The \xpackage{alphalph} package};
%   2006/05/30 v1.4;
%   \CTAN{macros/latex/contrib/oberdiek/alphalph.pdf}.
%
% \bibitem{hyperref}
%   Sebastian Rahtz, Heiko Oberdiek:
%   \textit{The \xpackage{hyperref} package};
%   2006/06/01 v6.75a;
%   \CTAN{macros/latex/contrib/hyperref/}.
%
% \bibitem{ifpdf}
%   Heiko Oberdiek: \textit{The \xpackage{ifpdf} package};
%   2006/02/20 v1.4;
%   \CTAN{macros/latex/contrib/oberdiek/ifpdf.pdf}.
%
% \end{thebibliography}
%
% \begin{History}
%   \begin{Version}{2006/06/01 v1.0}
%   \item
%     First version.
%   \end{Version}
%   \begin{Version}{2006/06/01 v2.0}
%   \item
%     New method for referencing destinations by number; an idea
%     proposed by Lars Hellstr\"om in the mailing list LATEX-L.
%   \item
%     Options \xoption{name} and \xoption{num} added.
%   \end{Version}
%   \begin{Version}{2007/11/11 v2.1}
%   \item
%     Use of package \xpackage{pdftexcmds} for \LuaTeX\ support.
%   \end{Version}
%   \begin{Version}{2008/08/08 v2.2}
%   \item
%     Support for package \xpackage{bookmark} added.
%   \end{Version}
%   \begin{Version}{2011/05/13 v2.3}
%   \item
%     Fix for \cs{Hy@DestName} if the destination name contains
%     special characters.
%   \item
%     Fix for option \xoption{name} and package \xpackage{bookmark}.
%   \end{Version}
% \end{History}
%
% \PrintIndex
%
% \Finale
\endinput

%        (quote the arguments according to the demands of your shell)
%
% Documentation:
%    (a) If hypdestopt.drv is present:
%           latex hypdestopt.drv
%    (b) Without hypdestopt.drv:
%           latex hypdestopt.dtx; ...
%    The class ltxdoc loads the configuration file ltxdoc.cfg
%    if available. Here you can specify further options, e.g.
%    use A4 as paper format:
%       \PassOptionsToClass{a4paper}{article}
%
%    Programm calls to get the documentation (example):
%       pdflatex hypdestopt.dtx
%       makeindex -s gind.ist hypdestopt.idx
%       pdflatex hypdestopt.dtx
%       makeindex -s gind.ist hypdestopt.idx
%       pdflatex hypdestopt.dtx
%
% Installation:
%    TDS:tex/latex/oberdiek/hypdestopt.sty
%    TDS:doc/latex/oberdiek/hypdestopt.pdf
%    TDS:source/latex/oberdiek/hypdestopt.dtx
%
%<*ignore>
\begingroup
  \catcode123=1 %
  \catcode125=2 %
  \def\x{LaTeX2e}%
\expandafter\endgroup
\ifcase 0\ifx\install y1\fi\expandafter
         \ifx\csname processbatchFile\endcsname\relax\else1\fi
         \ifx\fmtname\x\else 1\fi\relax
\else\csname fi\endcsname
%</ignore>
%<*install>
\input docstrip.tex
\Msg{************************************************************************}
\Msg{* Installation}
\Msg{* Package: hypdestopt 2011/05/13 v2.3 Hyperref destination optimizer (HO)}
\Msg{************************************************************************}

\keepsilent
\askforoverwritefalse

\let\MetaPrefix\relax
\preamble

This is a generated file.

Project: hypdestopt
Version: 2011/05/13 v2.3

Copyright (C) 2006-2008, 2011 by
   Heiko Oberdiek <heiko.oberdiek at googlemail.com>

This work may be distributed and/or modified under the
conditions of the LaTeX Project Public License, either
version 1.3c of this license or (at your option) any later
version. This version of this license is in
   http://www.latex-project.org/lppl/lppl-1-3c.txt
and the latest version of this license is in
   http://www.latex-project.org/lppl.txt
and version 1.3 or later is part of all distributions of
LaTeX version 2005/12/01 or later.

This work has the LPPL maintenance status "maintained".

This Current Maintainer of this work is Heiko Oberdiek.

This work consists of the main source file hypdestopt.dtx
and the derived files
   hypdestopt.sty, hypdestopt.pdf, hypdestopt.ins, hypdestopt.drv.

\endpreamble
\let\MetaPrefix\DoubleperCent

\generate{%
  \file{hypdestopt.ins}{\from{hypdestopt.dtx}{install}}%
  \file{hypdestopt.drv}{\from{hypdestopt.dtx}{driver}}%
  \usedir{tex/latex/oberdiek}%
  \file{hypdestopt.sty}{\from{hypdestopt.dtx}{package}}%
  \nopreamble
  \nopostamble
  \usedir{source/latex/oberdiek/catalogue}%
  \file{hypdestopt.xml}{\from{hypdestopt.dtx}{catalogue}}%
}

\catcode32=13\relax% active space
\let =\space%
\Msg{************************************************************************}
\Msg{*}
\Msg{* To finish the installation you have to move the following}
\Msg{* file into a directory searched by TeX:}
\Msg{*}
\Msg{*     hypdestopt.sty}
\Msg{*}
\Msg{* To produce the documentation run the file `hypdestopt.drv'}
\Msg{* through LaTeX.}
\Msg{*}
\Msg{* Happy TeXing!}
\Msg{*}
\Msg{************************************************************************}

\endbatchfile
%</install>
%<*ignore>
\fi
%</ignore>
%<*driver>
\NeedsTeXFormat{LaTeX2e}
\ProvidesFile{hypdestopt.drv}%
  [2011/05/13 v2.3 Hyperref destination optimizer (HO)]%
\documentclass{ltxdoc}
\usepackage{holtxdoc}[2011/11/22]
\begin{document}
  \DocInput{hypdestopt.dtx}%
\end{document}
%</driver>
% \fi
%
% \CheckSum{565}
%
% \CharacterTable
%  {Upper-case    \A\B\C\D\E\F\G\H\I\J\K\L\M\N\O\P\Q\R\S\T\U\V\W\X\Y\Z
%   Lower-case    \a\b\c\d\e\f\g\h\i\j\k\l\m\n\o\p\q\r\s\t\u\v\w\x\y\z
%   Digits        \0\1\2\3\4\5\6\7\8\9
%   Exclamation   \!     Double quote  \"     Hash (number) \#
%   Dollar        \$     Percent       \%     Ampersand     \&
%   Acute accent  \'     Left paren    \(     Right paren   \)
%   Asterisk      \*     Plus          \+     Comma         \,
%   Minus         \-     Point         \.     Solidus       \/
%   Colon         \:     Semicolon     \;     Less than     \<
%   Equals        \=     Greater than  \>     Question mark \?
%   Commercial at \@     Left bracket  \[     Backslash     \\
%   Right bracket \]     Circumflex    \^     Underscore    \_
%   Grave accent  \`     Left brace    \{     Vertical bar  \|
%   Right brace   \}     Tilde         \~}
%
% \GetFileInfo{hypdestopt.drv}
%
% \title{The \xpackage{hypdestopt} package}
% \date{2011/05/13 v2.3}
% \author{Heiko Oberdiek\\\xemail{heiko.oberdiek at googlemail.com}}
%
% \maketitle
%
% \begin{abstract}
% Package \xpackage{hypdestopt} supports \xpackage{hyperref}'s
% \xoption{pdftex} driver. It removes unnecessary destinations
% and shortens the destination names or uses numbered destinations
% to get smaller PDF files.
% \end{abstract}
%
% \tableofcontents
%
% \section{User interface}
%
% \subsection{Introduction}
%
% Before PDF-1.5 annotations and destinations cannot be compressed.
% If the destination names are not needed for external use, the
% file size can be decreased by the following means:
% \begin{itemize}
% \item Unused destinations are removed.
% \item The destination names are shortened (option \xoption{name}).
% \item Using numbered destinations (option \xoption{num}).
% \end{itemize}
%
% \subsection{Requirements}
%
% \begin{itemize}
% \item Package \xpackage{hyperref} 2006/06/01 v6.75a or newer
%       (\cite{hyperref}).
% \item Package \xpackage{alphalph} 2006/05/30 v1.4 or newer
%       (\cite{alphalph}), if option \xoption{name} is used.
% \item Package \xpackage{ifpdf} (\cite{ifpdf}).
% \item \pdfTeX\ 1.30.0 or newer.
% \item \pdfTeX\ in PDF mode.
% \item \eTeX\ extensions enabled.
% \item Probably an additional compile run of \pdfLaTeX\ is necessary.
% \end{itemize}
%
% In the first compile runs you can get warnings such as:
%\begin{quote}
%\begin{verbatim}
%! pdfTeX warning (dest): name{...} has been referenced ...
%\end{verbatim}
%\end{quote}
% These warnings should vanish in later compile runs.
% However these warnings also can occur without this package.
% The package does not cure them, thus these warnings will remain,
% but the destination name can be different. In such cases test
% without package, too.
%
% \subsection{Use}
%
% If the requirements are met, load the package:
%\begin{quote}
%\verb|\usepackage{hypdestopt}|
%\end{quote}
%
% The following options are supported:
% \begin{description}
% \item[\xoption{verbose}:] Verbose debug output is enabled and written
%   in the protocol file.
% \item[\xoption{num}:] Numbered destinations are used. The file size
%   is smaller, because names are no longer used.
%   This is the default.
% \item[\xoption{name}:] Destinations are identified by names.
% \end{description}
%
% \subsection{Limitations}
%
% \begin{itemize}
% \item Forget this package, if you need preserved destination names.
% \item Destination name strings use all bytes (0..255) except
%       the carriage return (13), left parenthesis (40), right
%       parenthesis (41), and backslash (92), because they
%       must be quoted in general and therefore occupy two bytes
%       instead of one.
%
%       Further the zero byte (0) is avoided for programs
%       that implement strings using zero terminated C strings.
%       And 255 (0xFF) is avoided to get rid of a possible
%       unicode marker at the begin.
%
%       So far I have not seen problems with:
%       \begin{itemize}
%       \item AcrobatReader 5.08/Linux
%       \item AcrobatReader 7.0/Linux
%       \item xpdf 3.00
%       \item Ghostscript 8.50
%       \item gv 3.5.8
%       \item GSview 4.6
%       \end{itemize}
%       But I have not tested all and all possible PDF viewers.
% \item Use of named destinations (\cs{pdfdest}, \cs{pdfoutline},
%       \cs{pdfstartlink}, \dots) that are not supported by this
%       package.
% \item Currently only \xpackage{hyperref} with \pdfTeX\ in PDF
%       mode is supported.
% \end{itemize}
%
% \subsection{Future}
%
% A more general approach is a PDF postprocessor that takes
% a PDF file, performs some transformations and writes the
% result in a more optimized PDF file. Then it does not depend,
% how the original PDF file was generated and further improvements
% are easier to apply. For example, the destination names could be sorted:
% often used destination names would then be shorter than seldom used ones.
%
% \StopEventually{
% }
%
% \section{Implementation}
%
% \subsection{Identification}
%
%    \begin{macrocode}
%<*package>
\NeedsTeXFormat{LaTeX2e}
\ProvidesPackage{hypdestopt}%
  [2011/05/13 v2.3 Hyperref destination optimizer (HO)]%
%    \end{macrocode}
%
% \subsection{Options}
%
% \subsubsection{Option \xoption{verbose}}
%
%    \begin{macrocode}
\newif\ifHypDest@Verbose
\DeclareOption{verbose}{\HypDest@Verbosetrue}
%    \end{macrocode}
%
%    \begin{macro}{\HypDest@VerboseInfo}
%    Wrapper for verbose messages.
%    \begin{macrocode}
\def\HypDest@VerboseInfo#1{%
  \ifHypDest@Verbose
    \PackageInfo{hypdestopt}{#1}%
  \fi
}
%    \end{macrocode}
%    \end{macro}
%
% \subsubsection{Options \xoption{num} and \xoption{name}}
%
%    The options \xoption{num} or \xoption{name} specify
%    the method, how destinations are referenced (by name or
%    number). Default is option \xoption{num}.
%    \begin{macrocode}
\newif\ifHypDest@name
\DeclareOption{num}{\HypDest@namefalse}
\DeclareOption{name}{\HypDest@nametrue}
%    \end{macrocode}
%
%    \begin{macrocode}
\ProcessOptions*\relax
%    \end{macrocode}
%
% \subsection{Check requirements}
%
%    First \pdfTeX\ must running in PDF mode.
%    \begin{macrocode}
\RequirePackage{ifpdf}[2007/09/09]
\RequirePackage{pdftexcmds}[2007/11/11]
\ifpdf
\else
  \PackageError{hypdestopt}{%
    This package requires pdfTeX in PDF mode%
  }\@ehc
  \expandafter\endinput
\fi
%    \end{macrocode}
%    The version of \pdfTeX\ must not be too old, because
%    \cs{pdfescapehex} and \cs{pdfunescapehex} are used.
%    \begin{macrocode}
\begingroup\expandafter\expandafter\expandafter\endgroup
\expandafter\ifx\csname pdf@escapehex\endcsname\relax
  \PackageError{hypdestopt}{%
    This pdfTeX is too old, at least 1.30.0 is required%
  }\@ehc
  \expandafter\endinput
\fi
%    \end{macrocode}
%    Features of \eTeX\ are used, e.g. \cs{numexpr}.
%    \begin{macrocode}
\begingroup\expandafter\expandafter\expandafter\endgroup
\expandafter\ifx\csname numexpr\endcsname\relax
  \PackageError{hypdestopt}{%
    e-TeX features are missing%
  }\@ehc
  \expandafter\endinput
\fi
%    \end{macrocode}
%    Package \xpackage{alphalph} provides \cs{newalphalph} since
%    version 2006/05/30 v1.4.
%    \begin{macrocode}
\ifHypDest@name
  \RequirePackage{alphalph}[2006/05/30]%
\fi
%    \end{macrocode}
%    \begin{macrocode}
\RequirePackage{auxhook}[2009/12/14]
\RequirePackage{pdfescape}[2007/04/21]
%    \end{macrocode}
%
% \subsection{Preamble for auxiliary file}
%
%    Provide dummy definitions for the macros that are used in the
%    auxiliary files. If the package is used no longer, then these
%    commands will not generate errors.
%
%    \begin{macro}{\HypDest@PrependDocument}
%    We add our stuff in front of the \cs{AtBeginDocument} hook
%    to ensure that we are before \xpackage{hyperref}'s stuff.
%    \begin{macrocode}
\long\def\HypDest@PrependDocument#1{%
  \begingroup
    \toks\z@{#1}%
    \toks\tw@\expandafter{\@begindocumenthook}%
    \xdef\@begindocumenthook{\the\toks\z@\the\toks\tw@}%
  \endgroup
}
%    \end{macrocode}
%    \end{macro}
%    \begin{macrocode}
\AddLineBeginAux{%
  \string\providecommand{\string\HypDest@Use}[1]{}%
}
%    \end{macrocode}
%
% \subsection{Generation of destination names}
%
%    Counter |HypDest| is used for identifying destinations.
%    \begin{macrocode}
\newcounter{HypDest}
%    \end{macrocode}
%
%    \begin{macrocode}
\ifHypDest@name
%    \end{macrocode}
%
%    \begin{macro}{\HypDest@HexChar}
%    Destination names are generated by automatically
%    numbering with the help of package \xpackage{alphalph}.
%    \cs{HypDest@HexChar} converts a number of the range 1 until 252
%    into the hexadecimal representation of the string character.
%    \begin{macrocode}
  \def\HypDest@HexChar#1{%
    \ifcase#1\or
%    \end{macrocode}
%    Avoid zero byte because of C strings in PDF viewer
%    applications.
%    \begin{macrocode}
      01\or 02\or 03\or 04\or 05\or 06\or 07\or
%    \end{macrocode}
%    Omit carriage return (13/\verb|^^0d|).
%    It needs quoting, otherwise it would be converted
%    to line feed (10/\verb|^^0a|).
%    \begin{macrocode}
      08\or 09\or 0A\or 0B\or 0C\or 0E\or 0F\or
      10\or 11\or 12\or 13\or 14\or 15\or 16\or 17\or
      18\or 19\or 1A\or 1B\or 1C\or 1D\or 1E\or 1F\or
      20\or 21\or 22\or 23\or 24\or 25\or 26\or 27\or
%    \end{macrocode}
%    Omit left and right parentheses (40/\verb|^^28|, 41/\verb|^^39|),
%    they need quoting in general.
%    \begin{macrocode}
      2A\or 2B\or 2C\or 2D\or 2E\or 2F\or
      30\or 31\or 32\or 33\or 34\or 35\or 36\or 37\or
      38\or 39\or 3A\or 3B\or 3C\or 3D\or 3E\or 3F\or
      40\or 41\or 42\or 43\or 44\or 45\or 46\or 47\or
      48\or 49\or 4A\or 4B\or 4C\or 4D\or 4E\or 4F\or
      50\or 51\or 52\or 53\or 54\or 55\or 56\or 57\or
%    \end{macrocode}
%    Omit backslash (92/\verb|^^5C|), it needs quoting.
%    \begin{macrocode}
      58\or 59\or 5A\or 5B\or 5D\or 5E\or 5F\or
      60\or 61\or 62\or 63\or 64\or 65\or 66\or 67\or
      68\or 69\or 6A\or 6B\or 6C\or 6D\or 6E\or 6F\or
      70\or 71\or 72\or 73\or 74\or 75\or 76\or 77\or
      78\or 79\or 7A\or 7B\or 7C\or 7D\or 7E\or 7F\or
      80\or 81\or 82\or 83\or 84\or 85\or 86\or 87\or
      88\or 89\or 8A\or 8B\or 8C\or 8D\or 8E\or 8F\or
      90\or 91\or 92\or 93\or 94\or 95\or 96\or 97\or
      98\or 99\or 9A\or 9B\or 9C\or 9D\or 9E\or 9F\or
      A0\or A1\or A2\or A3\or A4\or A5\or A6\or A7\or
      A8\or A9\or AA\or AB\or AC\or AD\or AE\or AF\or
      B0\or B1\or B2\or B3\or B4\or B5\or B6\or B7\or
      B8\or B9\or BA\or BB\or BC\or BD\or BE\or BF\or
      C0\or C1\or C2\or C3\or C4\or C5\or C6\or C7\or
      C8\or C9\or CA\or CB\or CC\or CD\or CE\or CF\or
      D0\or D1\or D2\or D3\or D4\or D5\or D6\or D7\or
      D8\or D9\or DA\or DB\or DC\or DD\or DE\or DF\or
      E0\or E1\or E2\or E3\or E4\or E5\or E6\or E7\or
      E8\or E9\or EA\or EB\or EC\or ED\or EE\or EF\or
      F0\or F1\or F2\or F3\or F4\or F5\or F6\or F7\or
%    \end{macrocode}
%    Avoid 255 (0xFF) to get rid of a possible unicode
%    marker at the begin of the string.
%    \begin{macrocode}
      F8\or F9\or FA\or FB\or FC\or FD\or FE%
    \fi
  }%
%    \end{macrocode}
%    \end{macro}
%    \begin{macro}{HypDest@HexString}
%    Now package \xpackage{alphalph} comes into play.
%    \cs{HypDest@HexString} is defined and converts
%    a positive number into a string, given in hexadecimal
%    representation.
%    \begin{macrocode}
  \newalphalph\HypDest@HexString\HypDest@HexChar{250}%
%    \end{macrocode}
%    \end{macro}
%    \begin{macro}{\theHypDest}
%    For use, the hexadecimal string is converted back.
%    \begin{macrocode}
  \renewcommand*{\theHypDest}{%
    \pdf@unescapehex{\HypDest@HexString{\value{HypDest}}}%
  }%
%    \end{macrocode}
%    \end{macro}
%
%    With option \xoption{num} we use the number directly.
%    \begin{macrocode}
\else
  \renewcommand*{\theHypDest}{%
    \number\value{HypDest}%
  }%
\fi
%    \end{macrocode}
%
% \subsection{Assign destination names}
%
%    \begin{macro}{\HypDest@Prefix}
%    The new destination names are remembered in macros whose names
%    start with prefix \cs{HypDest@Prefix}.
%    \begin{macrocode}
\edef\HypDest@Prefix{HypDest\string:}
%    \end{macrocode}
%    \end{macro}
%
%    \begin{macro}{\HypDest@Use}
%    During the first read of the auxiliary files, the used destinations
%    get fresh generated short destination names. Also for the old
%    destination names we use the hexadecimal representation. That
%    avoid problems with arbitrary names.
%    \begin{macrocode}
\def\HypDest@Use#1{%
  \begingroup
    \edef\x{%
      \expandafter\noexpand
      \csname\HypDest@Prefix\pdf@unescapehex{#1}\endcsname
    }%
    \expandafter\ifx\x\relax
      \stepcounter{HypDest}%
      \expandafter\xdef\x{\theHypDest}%
      \let\on@line\@empty
      \ifHypDest@name
        \HypDest@VerboseInfo{%
          Use: (\pdf@unescapehex{#1}) -\string> %
          0x\pdf@escapehex{\x} (\number\value{HypDest})%
        }%
      \else
        \HypDest@VerboseInfo{%
          Use: (\pdf@unescapehex{#1}) -\string> num \x
        }%
      \fi
    \fi
  \endgroup
}
%    \end{macrocode}
%    \end{macro}
%
%    After the first \xfile{.aux} file processing the destination names
%    are assigned and we can disable \cs{HypDest@Use}.
%    \begin{macrocode}
\AtBeginDocument{%
  \let\HypDest@Use\@gobble
}
%    \end{macrocode}
%
%    \begin{macro}{\HypDest@MarkUsed}
%    Destinations that are actually used are marked by \cs{HypDest@MarkUsed}.
%    \cs{nofiles} is respected.
%    \begin{macrocode}
\def\HypDest@MarkUsed#1{%
  \HypDest@VerboseInfo{%
    MarkUsed: (#1)%
  }%
  \if@filesw
    \immediate\write\@auxout{%
      \string\HypDest@Use{\pdf@escapehex{#1}}%
    }%
  \fi
}%
%    \end{macrocode}
%    \end{macro}
%
% \subsection{Redefinition of \xpackage{hyperref}'s hooks}
%
%    Package \xpackage{hyperref} can be loaded later, therefore
%    we redefine \xpackage{hyperref}'s macros at |\begin{document}|.
%    \begin{macrocode}
\HypDest@PrependDocument{%
%    \end{macrocode}
%
%    Check hyperref version.
%    \begin{macrocode}
  \@ifpackagelater{hyperref}{2006/06/01}{}{%
    \PackageError{hypdestopt}{%
      hyperref 2006/06/01 v6.75a or later is required%
    }\@ehc
  }%
%    \end{macrocode}
%
% \subsubsection{Destination setting}
%
%    \begin{macrocode}
  \ifHypDest@name
    \let\HypDest@Org@DestName\Hy@DestName
    \renewcommand*{\Hy@DestName}[2]{%
      \EdefUnescapeString\HypDest@temp{#1}%
      \@ifundefined{\HypDest@Prefix\HypDest@temp}{%
        \HypDest@VerboseInfo{%
          DestName: (\HypDest@temp) unused%
        }%
      }{%
        \HypDest@Org@DestName{%
          \csname\HypDest@Prefix\HypDest@temp\endcsname
        }{#2}%
        \HypDest@VerboseInfo{%
          DestName: (\HypDest@temp) %
          0x\pdf@escapehex{%
            \csname\HypDest@Prefix\HypDest@temp\endcsname
          }%
        }%
      }%
    }%
  \else
    \renewcommand*{\Hy@DestName}[2]{%
      \EdefUnescapeString\HypDest@temp{#1}%
      \@ifundefined{\HypDest@Prefix\HypDest@temp}{%
        \HypDest@VerboseInfo{%
          DestName: (\HypDest@temp) unused%
        }%
      }{%
        \pdfdest num%
        \csname\HypDest@Prefix\HypDest@temp\endcsname#2\relax
        \HypDest@VerboseInfo{%
          DestName: (\HypDest@temp) %
          num \csname\HypDest@Prefix\HypDest@temp\endcsname
        }%
      }%
    }%
  \fi
%    \end{macrocode}
%
% \subsubsection{Links}
%
%    \begin{macrocode}
  \let\HypDest@Org@StartlinkName\Hy@StartlinkName
  \ifHypDest@name
    \renewcommand*{\Hy@StartlinkName}[2]{%
      \HypDest@MarkUsed{#2}%
      \HypDest@Org@StartlinkName{#1}{%
        \@ifundefined{\HypDest@Prefix#2}{%
          #2%
        }{%
          \csname\HypDest@Prefix#2\endcsname
        }%
      }%
    }%
  \else
    \renewcommand*{\Hy@StartlinkName}[2]{%
      \HypDest@MarkUsed{#2}%
      \@ifundefined{\HypDest@Prefix#2}{%
        \HypDest@Org@StartlinkName{#1}{#2}%
      }{%
        \pdfstartlink attr{#1}%
                      goto num\csname\HypDest@Prefix#2\endcsname
        \relax
      }%
    }%
  \fi
%    \end{macrocode}
%
% \subsubsection{Outlines of package \xpackage{hyperref}}
%
%    \begin{macrocode}
  \let\HypDest@Org@OutlineName\Hy@OutlineName
  \ifHypDest@name
    \renewcommand*{\Hy@OutlineName}[4]{%
      \HypDest@Org@OutlineName{#1}{%
        \@ifundefined{\HypDest@Prefix#2}{%
          #2%
        }{%
          \csname\HypDest@Prefix#2\endcsname
        }%
      }{#3}{#4}%
    }%
  \else
    \renewcommand*{\Hy@OutlineName}[4]{%
      \@ifundefined{\HypDest@Prefix#2}{%
        \HypDest@Org@OutlineName{#1}{#2}{#3}{#4}%
      }{%
        \pdfoutline goto num\csname\HypDest@Prefix#2\endcsname
                    count#3{#4}%
      }%
    }%
  \fi
%    \end{macrocode}
%    Because \cs{Hy@OutlineName} is called after the \xfile{.out} file
%    is written in the previous run. Therefore we mark the destination
%    earlier in \cs{@@writetorep}.
%    \begin{macrocode}
  \let\HypDest@Org@@writetorep\@@writetorep
  \renewcommand*{\@@writetorep}[5]{%
    \begingroup
      \edef\Hy@tempa{#5}%
      \ifx\Hy@tempa\Hy@bookmarkstype
        \HypDest@MarkUsed{#3}%
      \fi
    \endgroup
    \HypDest@Org@@writetorep{#1}{#2}{#3}{#4}{#5}%
  }%
%    \end{macrocode}
%
% \subsubsection{Outlines of package \xpackage{bookmark}}
%
%    \begin{macrocode}
  \@ifpackageloaded{bookmark}{%
    \@ifpackagelater{bookmark}{2008/08/08}{%
      \renewcommand*{\BKM@DefGotoNameAction}[2]{%
        \@ifundefined{\HypDest@Prefix#2}{%
          \edef#1{goto name{hypdestopt\string :unknown}}%
        }{%
          \ifHypDest@name
            \edef#1{goto name{\csname\HypDest@Prefix#2\endcsname}}%
          \else
            \edef#1{goto num\csname\HypDest@Prefix#2\endcsname}%
          \fi
        }%
      }%
      \def\BKM@HypDestOptHook{%
        \ifx\BKM@dest\@empty
        \else
          \ifx\BKM@gotor\@empty
            \HypDest@MarkUsed\BKM@dest
          \fi
        \fi
      }%
    }{%
      \@PackageError{hypdestopt}{%
        Package `bookmark' is too old.\MessageBreak
        Version 2008/08/08 or later is needed%
      }\@ehc
    }%
  }{}%
%    \end{macrocode}
%
%    \begin{macrocode}
}
%    \end{macrocode}
%
%
%    \begin{macrocode}
%</package>
%    \end{macrocode}
%
% \section{Installation}
%
% \subsection{Download}
%
% \paragraph{Package.} This package is available on
% CTAN\footnote{\url{ftp://ftp.ctan.org/tex-archive/}}:
% \begin{description}
% \item[\CTAN{macros/latex/contrib/oberdiek/hypdestopt.dtx}] The source file.
% \item[\CTAN{macros/latex/contrib/oberdiek/hypdestopt.pdf}] Documentation.
% \end{description}
%
%
% \paragraph{Bundle.} All the packages of the bundle `oberdiek'
% are also available in a TDS compliant ZIP archive. There
% the packages are already unpacked and the documentation files
% are generated. The files and directories obey the TDS standard.
% \begin{description}
% \item[\CTAN{install/macros/latex/contrib/oberdiek.tds.zip}]
% \end{description}
% \emph{TDS} refers to the standard ``A Directory Structure
% for \TeX\ Files'' (\CTAN{tds/tds.pdf}). Directories
% with \xfile{texmf} in their name are usually organized this way.
%
% \subsection{Bundle installation}
%
% \paragraph{Unpacking.} Unpack the \xfile{oberdiek.tds.zip} in the
% TDS tree (also known as \xfile{texmf} tree) of your choice.
% Example (linux):
% \begin{quote}
%   |unzip oberdiek.tds.zip -d ~/texmf|
% \end{quote}
%
% \paragraph{Script installation.}
% Check the directory \xfile{TDS:scripts/oberdiek/} for
% scripts that need further installation steps.
% Package \xpackage{attachfile2} comes with the Perl script
% \xfile{pdfatfi.pl} that should be installed in such a way
% that it can be called as \texttt{pdfatfi}.
% Example (linux):
% \begin{quote}
%   |chmod +x scripts/oberdiek/pdfatfi.pl|\\
%   |cp scripts/oberdiek/pdfatfi.pl /usr/local/bin/|
% \end{quote}
%
% \subsection{Package installation}
%
% \paragraph{Unpacking.} The \xfile{.dtx} file is a self-extracting
% \docstrip\ archive. The files are extracted by running the
% \xfile{.dtx} through \plainTeX:
% \begin{quote}
%   \verb|tex hypdestopt.dtx|
% \end{quote}
%
% \paragraph{TDS.} Now the different files must be moved into
% the different directories in your installation TDS tree
% (also known as \xfile{texmf} tree):
% \begin{quote}
% \def\t{^^A
% \begin{tabular}{@{}>{\ttfamily}l@{ $\rightarrow$ }>{\ttfamily}l@{}}
%   hypdestopt.sty & tex/latex/oberdiek/hypdestopt.sty\\
%   hypdestopt.pdf & doc/latex/oberdiek/hypdestopt.pdf\\
%   hypdestopt.dtx & source/latex/oberdiek/hypdestopt.dtx\\
% \end{tabular}^^A
% }^^A
% \sbox0{\t}^^A
% \ifdim\wd0>\linewidth
%   \begingroup
%     \advance\linewidth by\leftmargin
%     \advance\linewidth by\rightmargin
%   \edef\x{\endgroup
%     \def\noexpand\lw{\the\linewidth}^^A
%   }\x
%   \def\lwbox{^^A
%     \leavevmode
%     \hbox to \linewidth{^^A
%       \kern-\leftmargin\relax
%       \hss
%       \usebox0
%       \hss
%       \kern-\rightmargin\relax
%     }^^A
%   }^^A
%   \ifdim\wd0>\lw
%     \sbox0{\small\t}^^A
%     \ifdim\wd0>\linewidth
%       \ifdim\wd0>\lw
%         \sbox0{\footnotesize\t}^^A
%         \ifdim\wd0>\linewidth
%           \ifdim\wd0>\lw
%             \sbox0{\scriptsize\t}^^A
%             \ifdim\wd0>\linewidth
%               \ifdim\wd0>\lw
%                 \sbox0{\tiny\t}^^A
%                 \ifdim\wd0>\linewidth
%                   \lwbox
%                 \else
%                   \usebox0
%                 \fi
%               \else
%                 \lwbox
%               \fi
%             \else
%               \usebox0
%             \fi
%           \else
%             \lwbox
%           \fi
%         \else
%           \usebox0
%         \fi
%       \else
%         \lwbox
%       \fi
%     \else
%       \usebox0
%     \fi
%   \else
%     \lwbox
%   \fi
% \else
%   \usebox0
% \fi
% \end{quote}
% If you have a \xfile{docstrip.cfg} that configures and enables \docstrip's
% TDS installing feature, then some files can already be in the right
% place, see the documentation of \docstrip.
%
% \subsection{Refresh file name databases}
%
% If your \TeX~distribution
% (\teTeX, \mikTeX, \dots) relies on file name databases, you must refresh
% these. For example, \teTeX\ users run \verb|texhash| or
% \verb|mktexlsr|.
%
% \subsection{Some details for the interested}
%
% \paragraph{Attached source.}
%
% The PDF documentation on CTAN also includes the
% \xfile{.dtx} source file. It can be extracted by
% AcrobatReader 6 or higher. Another option is \textsf{pdftk},
% e.g. unpack the file into the current directory:
% \begin{quote}
%   \verb|pdftk hypdestopt.pdf unpack_files output .|
% \end{quote}
%
% \paragraph{Unpacking with \LaTeX.}
% The \xfile{.dtx} chooses its action depending on the format:
% \begin{description}
% \item[\plainTeX:] Run \docstrip\ and extract the files.
% \item[\LaTeX:] Generate the documentation.
% \end{description}
% If you insist on using \LaTeX\ for \docstrip\ (really,
% \docstrip\ does not need \LaTeX), then inform the autodetect routine
% about your intention:
% \begin{quote}
%   \verb|latex \let\install=y% \iffalse meta-comment
%
% File: hypdestopt.dtx
% Version: 2011/05/13 v2.3
% Info: Hyperref destination optimizer
%
% Copyright (C) 2006-2008, 2011 by
%    Heiko Oberdiek <heiko.oberdiek at googlemail.com>
%
% This work may be distributed and/or modified under the
% conditions of the LaTeX Project Public License, either
% version 1.3c of this license or (at your option) any later
% version. This version of this license is in
%    http://www.latex-project.org/lppl/lppl-1-3c.txt
% and the latest version of this license is in
%    http://www.latex-project.org/lppl.txt
% and version 1.3 or later is part of all distributions of
% LaTeX version 2005/12/01 or later.
%
% This work has the LPPL maintenance status "maintained".
%
% This Current Maintainer of this work is Heiko Oberdiek.
%
% This work consists of the main source file hypdestopt.dtx
% and the derived files
%    hypdestopt.sty, hypdestopt.pdf, hypdestopt.ins, hypdestopt.drv.
%
% Distribution:
%    CTAN:macros/latex/contrib/oberdiek/hypdestopt.dtx
%    CTAN:macros/latex/contrib/oberdiek/hypdestopt.pdf
%
% Unpacking:
%    (a) If hypdestopt.ins is present:
%           tex hypdestopt.ins
%    (b) Without hypdestopt.ins:
%           tex hypdestopt.dtx
%    (c) If you insist on using LaTeX
%           latex \let\install=y\input{hypdestopt.dtx}
%        (quote the arguments according to the demands of your shell)
%
% Documentation:
%    (a) If hypdestopt.drv is present:
%           latex hypdestopt.drv
%    (b) Without hypdestopt.drv:
%           latex hypdestopt.dtx; ...
%    The class ltxdoc loads the configuration file ltxdoc.cfg
%    if available. Here you can specify further options, e.g.
%    use A4 as paper format:
%       \PassOptionsToClass{a4paper}{article}
%
%    Programm calls to get the documentation (example):
%       pdflatex hypdestopt.dtx
%       makeindex -s gind.ist hypdestopt.idx
%       pdflatex hypdestopt.dtx
%       makeindex -s gind.ist hypdestopt.idx
%       pdflatex hypdestopt.dtx
%
% Installation:
%    TDS:tex/latex/oberdiek/hypdestopt.sty
%    TDS:doc/latex/oberdiek/hypdestopt.pdf
%    TDS:source/latex/oberdiek/hypdestopt.dtx
%
%<*ignore>
\begingroup
  \catcode123=1 %
  \catcode125=2 %
  \def\x{LaTeX2e}%
\expandafter\endgroup
\ifcase 0\ifx\install y1\fi\expandafter
         \ifx\csname processbatchFile\endcsname\relax\else1\fi
         \ifx\fmtname\x\else 1\fi\relax
\else\csname fi\endcsname
%</ignore>
%<*install>
\input docstrip.tex
\Msg{************************************************************************}
\Msg{* Installation}
\Msg{* Package: hypdestopt 2011/05/13 v2.3 Hyperref destination optimizer (HO)}
\Msg{************************************************************************}

\keepsilent
\askforoverwritefalse

\let\MetaPrefix\relax
\preamble

This is a generated file.

Project: hypdestopt
Version: 2011/05/13 v2.3

Copyright (C) 2006-2008, 2011 by
   Heiko Oberdiek <heiko.oberdiek at googlemail.com>

This work may be distributed and/or modified under the
conditions of the LaTeX Project Public License, either
version 1.3c of this license or (at your option) any later
version. This version of this license is in
   http://www.latex-project.org/lppl/lppl-1-3c.txt
and the latest version of this license is in
   http://www.latex-project.org/lppl.txt
and version 1.3 or later is part of all distributions of
LaTeX version 2005/12/01 or later.

This work has the LPPL maintenance status "maintained".

This Current Maintainer of this work is Heiko Oberdiek.

This work consists of the main source file hypdestopt.dtx
and the derived files
   hypdestopt.sty, hypdestopt.pdf, hypdestopt.ins, hypdestopt.drv.

\endpreamble
\let\MetaPrefix\DoubleperCent

\generate{%
  \file{hypdestopt.ins}{\from{hypdestopt.dtx}{install}}%
  \file{hypdestopt.drv}{\from{hypdestopt.dtx}{driver}}%
  \usedir{tex/latex/oberdiek}%
  \file{hypdestopt.sty}{\from{hypdestopt.dtx}{package}}%
  \nopreamble
  \nopostamble
  \usedir{source/latex/oberdiek/catalogue}%
  \file{hypdestopt.xml}{\from{hypdestopt.dtx}{catalogue}}%
}

\catcode32=13\relax% active space
\let =\space%
\Msg{************************************************************************}
\Msg{*}
\Msg{* To finish the installation you have to move the following}
\Msg{* file into a directory searched by TeX:}
\Msg{*}
\Msg{*     hypdestopt.sty}
\Msg{*}
\Msg{* To produce the documentation run the file `hypdestopt.drv'}
\Msg{* through LaTeX.}
\Msg{*}
\Msg{* Happy TeXing!}
\Msg{*}
\Msg{************************************************************************}

\endbatchfile
%</install>
%<*ignore>
\fi
%</ignore>
%<*driver>
\NeedsTeXFormat{LaTeX2e}
\ProvidesFile{hypdestopt.drv}%
  [2011/05/13 v2.3 Hyperref destination optimizer (HO)]%
\documentclass{ltxdoc}
\usepackage{holtxdoc}[2011/11/22]
\begin{document}
  \DocInput{hypdestopt.dtx}%
\end{document}
%</driver>
% \fi
%
% \CheckSum{565}
%
% \CharacterTable
%  {Upper-case    \A\B\C\D\E\F\G\H\I\J\K\L\M\N\O\P\Q\R\S\T\U\V\W\X\Y\Z
%   Lower-case    \a\b\c\d\e\f\g\h\i\j\k\l\m\n\o\p\q\r\s\t\u\v\w\x\y\z
%   Digits        \0\1\2\3\4\5\6\7\8\9
%   Exclamation   \!     Double quote  \"     Hash (number) \#
%   Dollar        \$     Percent       \%     Ampersand     \&
%   Acute accent  \'     Left paren    \(     Right paren   \)
%   Asterisk      \*     Plus          \+     Comma         \,
%   Minus         \-     Point         \.     Solidus       \/
%   Colon         \:     Semicolon     \;     Less than     \<
%   Equals        \=     Greater than  \>     Question mark \?
%   Commercial at \@     Left bracket  \[     Backslash     \\
%   Right bracket \]     Circumflex    \^     Underscore    \_
%   Grave accent  \`     Left brace    \{     Vertical bar  \|
%   Right brace   \}     Tilde         \~}
%
% \GetFileInfo{hypdestopt.drv}
%
% \title{The \xpackage{hypdestopt} package}
% \date{2011/05/13 v2.3}
% \author{Heiko Oberdiek\\\xemail{heiko.oberdiek at googlemail.com}}
%
% \maketitle
%
% \begin{abstract}
% Package \xpackage{hypdestopt} supports \xpackage{hyperref}'s
% \xoption{pdftex} driver. It removes unnecessary destinations
% and shortens the destination names or uses numbered destinations
% to get smaller PDF files.
% \end{abstract}
%
% \tableofcontents
%
% \section{User interface}
%
% \subsection{Introduction}
%
% Before PDF-1.5 annotations and destinations cannot be compressed.
% If the destination names are not needed for external use, the
% file size can be decreased by the following means:
% \begin{itemize}
% \item Unused destinations are removed.
% \item The destination names are shortened (option \xoption{name}).
% \item Using numbered destinations (option \xoption{num}).
% \end{itemize}
%
% \subsection{Requirements}
%
% \begin{itemize}
% \item Package \xpackage{hyperref} 2006/06/01 v6.75a or newer
%       (\cite{hyperref}).
% \item Package \xpackage{alphalph} 2006/05/30 v1.4 or newer
%       (\cite{alphalph}), if option \xoption{name} is used.
% \item Package \xpackage{ifpdf} (\cite{ifpdf}).
% \item \pdfTeX\ 1.30.0 or newer.
% \item \pdfTeX\ in PDF mode.
% \item \eTeX\ extensions enabled.
% \item Probably an additional compile run of \pdfLaTeX\ is necessary.
% \end{itemize}
%
% In the first compile runs you can get warnings such as:
%\begin{quote}
%\begin{verbatim}
%! pdfTeX warning (dest): name{...} has been referenced ...
%\end{verbatim}
%\end{quote}
% These warnings should vanish in later compile runs.
% However these warnings also can occur without this package.
% The package does not cure them, thus these warnings will remain,
% but the destination name can be different. In such cases test
% without package, too.
%
% \subsection{Use}
%
% If the requirements are met, load the package:
%\begin{quote}
%\verb|\usepackage{hypdestopt}|
%\end{quote}
%
% The following options are supported:
% \begin{description}
% \item[\xoption{verbose}:] Verbose debug output is enabled and written
%   in the protocol file.
% \item[\xoption{num}:] Numbered destinations are used. The file size
%   is smaller, because names are no longer used.
%   This is the default.
% \item[\xoption{name}:] Destinations are identified by names.
% \end{description}
%
% \subsection{Limitations}
%
% \begin{itemize}
% \item Forget this package, if you need preserved destination names.
% \item Destination name strings use all bytes (0..255) except
%       the carriage return (13), left parenthesis (40), right
%       parenthesis (41), and backslash (92), because they
%       must be quoted in general and therefore occupy two bytes
%       instead of one.
%
%       Further the zero byte (0) is avoided for programs
%       that implement strings using zero terminated C strings.
%       And 255 (0xFF) is avoided to get rid of a possible
%       unicode marker at the begin.
%
%       So far I have not seen problems with:
%       \begin{itemize}
%       \item AcrobatReader 5.08/Linux
%       \item AcrobatReader 7.0/Linux
%       \item xpdf 3.00
%       \item Ghostscript 8.50
%       \item gv 3.5.8
%       \item GSview 4.6
%       \end{itemize}
%       But I have not tested all and all possible PDF viewers.
% \item Use of named destinations (\cs{pdfdest}, \cs{pdfoutline},
%       \cs{pdfstartlink}, \dots) that are not supported by this
%       package.
% \item Currently only \xpackage{hyperref} with \pdfTeX\ in PDF
%       mode is supported.
% \end{itemize}
%
% \subsection{Future}
%
% A more general approach is a PDF postprocessor that takes
% a PDF file, performs some transformations and writes the
% result in a more optimized PDF file. Then it does not depend,
% how the original PDF file was generated and further improvements
% are easier to apply. For example, the destination names could be sorted:
% often used destination names would then be shorter than seldom used ones.
%
% \StopEventually{
% }
%
% \section{Implementation}
%
% \subsection{Identification}
%
%    \begin{macrocode}
%<*package>
\NeedsTeXFormat{LaTeX2e}
\ProvidesPackage{hypdestopt}%
  [2011/05/13 v2.3 Hyperref destination optimizer (HO)]%
%    \end{macrocode}
%
% \subsection{Options}
%
% \subsubsection{Option \xoption{verbose}}
%
%    \begin{macrocode}
\newif\ifHypDest@Verbose
\DeclareOption{verbose}{\HypDest@Verbosetrue}
%    \end{macrocode}
%
%    \begin{macro}{\HypDest@VerboseInfo}
%    Wrapper for verbose messages.
%    \begin{macrocode}
\def\HypDest@VerboseInfo#1{%
  \ifHypDest@Verbose
    \PackageInfo{hypdestopt}{#1}%
  \fi
}
%    \end{macrocode}
%    \end{macro}
%
% \subsubsection{Options \xoption{num} and \xoption{name}}
%
%    The options \xoption{num} or \xoption{name} specify
%    the method, how destinations are referenced (by name or
%    number). Default is option \xoption{num}.
%    \begin{macrocode}
\newif\ifHypDest@name
\DeclareOption{num}{\HypDest@namefalse}
\DeclareOption{name}{\HypDest@nametrue}
%    \end{macrocode}
%
%    \begin{macrocode}
\ProcessOptions*\relax
%    \end{macrocode}
%
% \subsection{Check requirements}
%
%    First \pdfTeX\ must running in PDF mode.
%    \begin{macrocode}
\RequirePackage{ifpdf}[2007/09/09]
\RequirePackage{pdftexcmds}[2007/11/11]
\ifpdf
\else
  \PackageError{hypdestopt}{%
    This package requires pdfTeX in PDF mode%
  }\@ehc
  \expandafter\endinput
\fi
%    \end{macrocode}
%    The version of \pdfTeX\ must not be too old, because
%    \cs{pdfescapehex} and \cs{pdfunescapehex} are used.
%    \begin{macrocode}
\begingroup\expandafter\expandafter\expandafter\endgroup
\expandafter\ifx\csname pdf@escapehex\endcsname\relax
  \PackageError{hypdestopt}{%
    This pdfTeX is too old, at least 1.30.0 is required%
  }\@ehc
  \expandafter\endinput
\fi
%    \end{macrocode}
%    Features of \eTeX\ are used, e.g. \cs{numexpr}.
%    \begin{macrocode}
\begingroup\expandafter\expandafter\expandafter\endgroup
\expandafter\ifx\csname numexpr\endcsname\relax
  \PackageError{hypdestopt}{%
    e-TeX features are missing%
  }\@ehc
  \expandafter\endinput
\fi
%    \end{macrocode}
%    Package \xpackage{alphalph} provides \cs{newalphalph} since
%    version 2006/05/30 v1.4.
%    \begin{macrocode}
\ifHypDest@name
  \RequirePackage{alphalph}[2006/05/30]%
\fi
%    \end{macrocode}
%    \begin{macrocode}
\RequirePackage{auxhook}[2009/12/14]
\RequirePackage{pdfescape}[2007/04/21]
%    \end{macrocode}
%
% \subsection{Preamble for auxiliary file}
%
%    Provide dummy definitions for the macros that are used in the
%    auxiliary files. If the package is used no longer, then these
%    commands will not generate errors.
%
%    \begin{macro}{\HypDest@PrependDocument}
%    We add our stuff in front of the \cs{AtBeginDocument} hook
%    to ensure that we are before \xpackage{hyperref}'s stuff.
%    \begin{macrocode}
\long\def\HypDest@PrependDocument#1{%
  \begingroup
    \toks\z@{#1}%
    \toks\tw@\expandafter{\@begindocumenthook}%
    \xdef\@begindocumenthook{\the\toks\z@\the\toks\tw@}%
  \endgroup
}
%    \end{macrocode}
%    \end{macro}
%    \begin{macrocode}
\AddLineBeginAux{%
  \string\providecommand{\string\HypDest@Use}[1]{}%
}
%    \end{macrocode}
%
% \subsection{Generation of destination names}
%
%    Counter |HypDest| is used for identifying destinations.
%    \begin{macrocode}
\newcounter{HypDest}
%    \end{macrocode}
%
%    \begin{macrocode}
\ifHypDest@name
%    \end{macrocode}
%
%    \begin{macro}{\HypDest@HexChar}
%    Destination names are generated by automatically
%    numbering with the help of package \xpackage{alphalph}.
%    \cs{HypDest@HexChar} converts a number of the range 1 until 252
%    into the hexadecimal representation of the string character.
%    \begin{macrocode}
  \def\HypDest@HexChar#1{%
    \ifcase#1\or
%    \end{macrocode}
%    Avoid zero byte because of C strings in PDF viewer
%    applications.
%    \begin{macrocode}
      01\or 02\or 03\or 04\or 05\or 06\or 07\or
%    \end{macrocode}
%    Omit carriage return (13/\verb|^^0d|).
%    It needs quoting, otherwise it would be converted
%    to line feed (10/\verb|^^0a|).
%    \begin{macrocode}
      08\or 09\or 0A\or 0B\or 0C\or 0E\or 0F\or
      10\or 11\or 12\or 13\or 14\or 15\or 16\or 17\or
      18\or 19\or 1A\or 1B\or 1C\or 1D\or 1E\or 1F\or
      20\or 21\or 22\or 23\or 24\or 25\or 26\or 27\or
%    \end{macrocode}
%    Omit left and right parentheses (40/\verb|^^28|, 41/\verb|^^39|),
%    they need quoting in general.
%    \begin{macrocode}
      2A\or 2B\or 2C\or 2D\or 2E\or 2F\or
      30\or 31\or 32\or 33\or 34\or 35\or 36\or 37\or
      38\or 39\or 3A\or 3B\or 3C\or 3D\or 3E\or 3F\or
      40\or 41\or 42\or 43\or 44\or 45\or 46\or 47\or
      48\or 49\or 4A\or 4B\or 4C\or 4D\or 4E\or 4F\or
      50\or 51\or 52\or 53\or 54\or 55\or 56\or 57\or
%    \end{macrocode}
%    Omit backslash (92/\verb|^^5C|), it needs quoting.
%    \begin{macrocode}
      58\or 59\or 5A\or 5B\or 5D\or 5E\or 5F\or
      60\or 61\or 62\or 63\or 64\or 65\or 66\or 67\or
      68\or 69\or 6A\or 6B\or 6C\or 6D\or 6E\or 6F\or
      70\or 71\or 72\or 73\or 74\or 75\or 76\or 77\or
      78\or 79\or 7A\or 7B\or 7C\or 7D\or 7E\or 7F\or
      80\or 81\or 82\or 83\or 84\or 85\or 86\or 87\or
      88\or 89\or 8A\or 8B\or 8C\or 8D\or 8E\or 8F\or
      90\or 91\or 92\or 93\or 94\or 95\or 96\or 97\or
      98\or 99\or 9A\or 9B\or 9C\or 9D\or 9E\or 9F\or
      A0\or A1\or A2\or A3\or A4\or A5\or A6\or A7\or
      A8\or A9\or AA\or AB\or AC\or AD\or AE\or AF\or
      B0\or B1\or B2\or B3\or B4\or B5\or B6\or B7\or
      B8\or B9\or BA\or BB\or BC\or BD\or BE\or BF\or
      C0\or C1\or C2\or C3\or C4\or C5\or C6\or C7\or
      C8\or C9\or CA\or CB\or CC\or CD\or CE\or CF\or
      D0\or D1\or D2\or D3\or D4\or D5\or D6\or D7\or
      D8\or D9\or DA\or DB\or DC\or DD\or DE\or DF\or
      E0\or E1\or E2\or E3\or E4\or E5\or E6\or E7\or
      E8\or E9\or EA\or EB\or EC\or ED\or EE\or EF\or
      F0\or F1\or F2\or F3\or F4\or F5\or F6\or F7\or
%    \end{macrocode}
%    Avoid 255 (0xFF) to get rid of a possible unicode
%    marker at the begin of the string.
%    \begin{macrocode}
      F8\or F9\or FA\or FB\or FC\or FD\or FE%
    \fi
  }%
%    \end{macrocode}
%    \end{macro}
%    \begin{macro}{HypDest@HexString}
%    Now package \xpackage{alphalph} comes into play.
%    \cs{HypDest@HexString} is defined and converts
%    a positive number into a string, given in hexadecimal
%    representation.
%    \begin{macrocode}
  \newalphalph\HypDest@HexString\HypDest@HexChar{250}%
%    \end{macrocode}
%    \end{macro}
%    \begin{macro}{\theHypDest}
%    For use, the hexadecimal string is converted back.
%    \begin{macrocode}
  \renewcommand*{\theHypDest}{%
    \pdf@unescapehex{\HypDest@HexString{\value{HypDest}}}%
  }%
%    \end{macrocode}
%    \end{macro}
%
%    With option \xoption{num} we use the number directly.
%    \begin{macrocode}
\else
  \renewcommand*{\theHypDest}{%
    \number\value{HypDest}%
  }%
\fi
%    \end{macrocode}
%
% \subsection{Assign destination names}
%
%    \begin{macro}{\HypDest@Prefix}
%    The new destination names are remembered in macros whose names
%    start with prefix \cs{HypDest@Prefix}.
%    \begin{macrocode}
\edef\HypDest@Prefix{HypDest\string:}
%    \end{macrocode}
%    \end{macro}
%
%    \begin{macro}{\HypDest@Use}
%    During the first read of the auxiliary files, the used destinations
%    get fresh generated short destination names. Also for the old
%    destination names we use the hexadecimal representation. That
%    avoid problems with arbitrary names.
%    \begin{macrocode}
\def\HypDest@Use#1{%
  \begingroup
    \edef\x{%
      \expandafter\noexpand
      \csname\HypDest@Prefix\pdf@unescapehex{#1}\endcsname
    }%
    \expandafter\ifx\x\relax
      \stepcounter{HypDest}%
      \expandafter\xdef\x{\theHypDest}%
      \let\on@line\@empty
      \ifHypDest@name
        \HypDest@VerboseInfo{%
          Use: (\pdf@unescapehex{#1}) -\string> %
          0x\pdf@escapehex{\x} (\number\value{HypDest})%
        }%
      \else
        \HypDest@VerboseInfo{%
          Use: (\pdf@unescapehex{#1}) -\string> num \x
        }%
      \fi
    \fi
  \endgroup
}
%    \end{macrocode}
%    \end{macro}
%
%    After the first \xfile{.aux} file processing the destination names
%    are assigned and we can disable \cs{HypDest@Use}.
%    \begin{macrocode}
\AtBeginDocument{%
  \let\HypDest@Use\@gobble
}
%    \end{macrocode}
%
%    \begin{macro}{\HypDest@MarkUsed}
%    Destinations that are actually used are marked by \cs{HypDest@MarkUsed}.
%    \cs{nofiles} is respected.
%    \begin{macrocode}
\def\HypDest@MarkUsed#1{%
  \HypDest@VerboseInfo{%
    MarkUsed: (#1)%
  }%
  \if@filesw
    \immediate\write\@auxout{%
      \string\HypDest@Use{\pdf@escapehex{#1}}%
    }%
  \fi
}%
%    \end{macrocode}
%    \end{macro}
%
% \subsection{Redefinition of \xpackage{hyperref}'s hooks}
%
%    Package \xpackage{hyperref} can be loaded later, therefore
%    we redefine \xpackage{hyperref}'s macros at |\begin{document}|.
%    \begin{macrocode}
\HypDest@PrependDocument{%
%    \end{macrocode}
%
%    Check hyperref version.
%    \begin{macrocode}
  \@ifpackagelater{hyperref}{2006/06/01}{}{%
    \PackageError{hypdestopt}{%
      hyperref 2006/06/01 v6.75a or later is required%
    }\@ehc
  }%
%    \end{macrocode}
%
% \subsubsection{Destination setting}
%
%    \begin{macrocode}
  \ifHypDest@name
    \let\HypDest@Org@DestName\Hy@DestName
    \renewcommand*{\Hy@DestName}[2]{%
      \EdefUnescapeString\HypDest@temp{#1}%
      \@ifundefined{\HypDest@Prefix\HypDest@temp}{%
        \HypDest@VerboseInfo{%
          DestName: (\HypDest@temp) unused%
        }%
      }{%
        \HypDest@Org@DestName{%
          \csname\HypDest@Prefix\HypDest@temp\endcsname
        }{#2}%
        \HypDest@VerboseInfo{%
          DestName: (\HypDest@temp) %
          0x\pdf@escapehex{%
            \csname\HypDest@Prefix\HypDest@temp\endcsname
          }%
        }%
      }%
    }%
  \else
    \renewcommand*{\Hy@DestName}[2]{%
      \EdefUnescapeString\HypDest@temp{#1}%
      \@ifundefined{\HypDest@Prefix\HypDest@temp}{%
        \HypDest@VerboseInfo{%
          DestName: (\HypDest@temp) unused%
        }%
      }{%
        \pdfdest num%
        \csname\HypDest@Prefix\HypDest@temp\endcsname#2\relax
        \HypDest@VerboseInfo{%
          DestName: (\HypDest@temp) %
          num \csname\HypDest@Prefix\HypDest@temp\endcsname
        }%
      }%
    }%
  \fi
%    \end{macrocode}
%
% \subsubsection{Links}
%
%    \begin{macrocode}
  \let\HypDest@Org@StartlinkName\Hy@StartlinkName
  \ifHypDest@name
    \renewcommand*{\Hy@StartlinkName}[2]{%
      \HypDest@MarkUsed{#2}%
      \HypDest@Org@StartlinkName{#1}{%
        \@ifundefined{\HypDest@Prefix#2}{%
          #2%
        }{%
          \csname\HypDest@Prefix#2\endcsname
        }%
      }%
    }%
  \else
    \renewcommand*{\Hy@StartlinkName}[2]{%
      \HypDest@MarkUsed{#2}%
      \@ifundefined{\HypDest@Prefix#2}{%
        \HypDest@Org@StartlinkName{#1}{#2}%
      }{%
        \pdfstartlink attr{#1}%
                      goto num\csname\HypDest@Prefix#2\endcsname
        \relax
      }%
    }%
  \fi
%    \end{macrocode}
%
% \subsubsection{Outlines of package \xpackage{hyperref}}
%
%    \begin{macrocode}
  \let\HypDest@Org@OutlineName\Hy@OutlineName
  \ifHypDest@name
    \renewcommand*{\Hy@OutlineName}[4]{%
      \HypDest@Org@OutlineName{#1}{%
        \@ifundefined{\HypDest@Prefix#2}{%
          #2%
        }{%
          \csname\HypDest@Prefix#2\endcsname
        }%
      }{#3}{#4}%
    }%
  \else
    \renewcommand*{\Hy@OutlineName}[4]{%
      \@ifundefined{\HypDest@Prefix#2}{%
        \HypDest@Org@OutlineName{#1}{#2}{#3}{#4}%
      }{%
        \pdfoutline goto num\csname\HypDest@Prefix#2\endcsname
                    count#3{#4}%
      }%
    }%
  \fi
%    \end{macrocode}
%    Because \cs{Hy@OutlineName} is called after the \xfile{.out} file
%    is written in the previous run. Therefore we mark the destination
%    earlier in \cs{@@writetorep}.
%    \begin{macrocode}
  \let\HypDest@Org@@writetorep\@@writetorep
  \renewcommand*{\@@writetorep}[5]{%
    \begingroup
      \edef\Hy@tempa{#5}%
      \ifx\Hy@tempa\Hy@bookmarkstype
        \HypDest@MarkUsed{#3}%
      \fi
    \endgroup
    \HypDest@Org@@writetorep{#1}{#2}{#3}{#4}{#5}%
  }%
%    \end{macrocode}
%
% \subsubsection{Outlines of package \xpackage{bookmark}}
%
%    \begin{macrocode}
  \@ifpackageloaded{bookmark}{%
    \@ifpackagelater{bookmark}{2008/08/08}{%
      \renewcommand*{\BKM@DefGotoNameAction}[2]{%
        \@ifundefined{\HypDest@Prefix#2}{%
          \edef#1{goto name{hypdestopt\string :unknown}}%
        }{%
          \ifHypDest@name
            \edef#1{goto name{\csname\HypDest@Prefix#2\endcsname}}%
          \else
            \edef#1{goto num\csname\HypDest@Prefix#2\endcsname}%
          \fi
        }%
      }%
      \def\BKM@HypDestOptHook{%
        \ifx\BKM@dest\@empty
        \else
          \ifx\BKM@gotor\@empty
            \HypDest@MarkUsed\BKM@dest
          \fi
        \fi
      }%
    }{%
      \@PackageError{hypdestopt}{%
        Package `bookmark' is too old.\MessageBreak
        Version 2008/08/08 or later is needed%
      }\@ehc
    }%
  }{}%
%    \end{macrocode}
%
%    \begin{macrocode}
}
%    \end{macrocode}
%
%
%    \begin{macrocode}
%</package>
%    \end{macrocode}
%
% \section{Installation}
%
% \subsection{Download}
%
% \paragraph{Package.} This package is available on
% CTAN\footnote{\url{ftp://ftp.ctan.org/tex-archive/}}:
% \begin{description}
% \item[\CTAN{macros/latex/contrib/oberdiek/hypdestopt.dtx}] The source file.
% \item[\CTAN{macros/latex/contrib/oberdiek/hypdestopt.pdf}] Documentation.
% \end{description}
%
%
% \paragraph{Bundle.} All the packages of the bundle `oberdiek'
% are also available in a TDS compliant ZIP archive. There
% the packages are already unpacked and the documentation files
% are generated. The files and directories obey the TDS standard.
% \begin{description}
% \item[\CTAN{install/macros/latex/contrib/oberdiek.tds.zip}]
% \end{description}
% \emph{TDS} refers to the standard ``A Directory Structure
% for \TeX\ Files'' (\CTAN{tds/tds.pdf}). Directories
% with \xfile{texmf} in their name are usually organized this way.
%
% \subsection{Bundle installation}
%
% \paragraph{Unpacking.} Unpack the \xfile{oberdiek.tds.zip} in the
% TDS tree (also known as \xfile{texmf} tree) of your choice.
% Example (linux):
% \begin{quote}
%   |unzip oberdiek.tds.zip -d ~/texmf|
% \end{quote}
%
% \paragraph{Script installation.}
% Check the directory \xfile{TDS:scripts/oberdiek/} for
% scripts that need further installation steps.
% Package \xpackage{attachfile2} comes with the Perl script
% \xfile{pdfatfi.pl} that should be installed in such a way
% that it can be called as \texttt{pdfatfi}.
% Example (linux):
% \begin{quote}
%   |chmod +x scripts/oberdiek/pdfatfi.pl|\\
%   |cp scripts/oberdiek/pdfatfi.pl /usr/local/bin/|
% \end{quote}
%
% \subsection{Package installation}
%
% \paragraph{Unpacking.} The \xfile{.dtx} file is a self-extracting
% \docstrip\ archive. The files are extracted by running the
% \xfile{.dtx} through \plainTeX:
% \begin{quote}
%   \verb|tex hypdestopt.dtx|
% \end{quote}
%
% \paragraph{TDS.} Now the different files must be moved into
% the different directories in your installation TDS tree
% (also known as \xfile{texmf} tree):
% \begin{quote}
% \def\t{^^A
% \begin{tabular}{@{}>{\ttfamily}l@{ $\rightarrow$ }>{\ttfamily}l@{}}
%   hypdestopt.sty & tex/latex/oberdiek/hypdestopt.sty\\
%   hypdestopt.pdf & doc/latex/oberdiek/hypdestopt.pdf\\
%   hypdestopt.dtx & source/latex/oberdiek/hypdestopt.dtx\\
% \end{tabular}^^A
% }^^A
% \sbox0{\t}^^A
% \ifdim\wd0>\linewidth
%   \begingroup
%     \advance\linewidth by\leftmargin
%     \advance\linewidth by\rightmargin
%   \edef\x{\endgroup
%     \def\noexpand\lw{\the\linewidth}^^A
%   }\x
%   \def\lwbox{^^A
%     \leavevmode
%     \hbox to \linewidth{^^A
%       \kern-\leftmargin\relax
%       \hss
%       \usebox0
%       \hss
%       \kern-\rightmargin\relax
%     }^^A
%   }^^A
%   \ifdim\wd0>\lw
%     \sbox0{\small\t}^^A
%     \ifdim\wd0>\linewidth
%       \ifdim\wd0>\lw
%         \sbox0{\footnotesize\t}^^A
%         \ifdim\wd0>\linewidth
%           \ifdim\wd0>\lw
%             \sbox0{\scriptsize\t}^^A
%             \ifdim\wd0>\linewidth
%               \ifdim\wd0>\lw
%                 \sbox0{\tiny\t}^^A
%                 \ifdim\wd0>\linewidth
%                   \lwbox
%                 \else
%                   \usebox0
%                 \fi
%               \else
%                 \lwbox
%               \fi
%             \else
%               \usebox0
%             \fi
%           \else
%             \lwbox
%           \fi
%         \else
%           \usebox0
%         \fi
%       \else
%         \lwbox
%       \fi
%     \else
%       \usebox0
%     \fi
%   \else
%     \lwbox
%   \fi
% \else
%   \usebox0
% \fi
% \end{quote}
% If you have a \xfile{docstrip.cfg} that configures and enables \docstrip's
% TDS installing feature, then some files can already be in the right
% place, see the documentation of \docstrip.
%
% \subsection{Refresh file name databases}
%
% If your \TeX~distribution
% (\teTeX, \mikTeX, \dots) relies on file name databases, you must refresh
% these. For example, \teTeX\ users run \verb|texhash| or
% \verb|mktexlsr|.
%
% \subsection{Some details for the interested}
%
% \paragraph{Attached source.}
%
% The PDF documentation on CTAN also includes the
% \xfile{.dtx} source file. It can be extracted by
% AcrobatReader 6 or higher. Another option is \textsf{pdftk},
% e.g. unpack the file into the current directory:
% \begin{quote}
%   \verb|pdftk hypdestopt.pdf unpack_files output .|
% \end{quote}
%
% \paragraph{Unpacking with \LaTeX.}
% The \xfile{.dtx} chooses its action depending on the format:
% \begin{description}
% \item[\plainTeX:] Run \docstrip\ and extract the files.
% \item[\LaTeX:] Generate the documentation.
% \end{description}
% If you insist on using \LaTeX\ for \docstrip\ (really,
% \docstrip\ does not need \LaTeX), then inform the autodetect routine
% about your intention:
% \begin{quote}
%   \verb|latex \let\install=y\input{hypdestopt.dtx}|
% \end{quote}
% Do not forget to quote the argument according to the demands
% of your shell.
%
% \paragraph{Generating the documentation.}
% You can use both the \xfile{.dtx} or the \xfile{.drv} to generate
% the documentation. The process can be configured by the
% configuration file \xfile{ltxdoc.cfg}. For instance, put this
% line into this file, if you want to have A4 as paper format:
% \begin{quote}
%   \verb|\PassOptionsToClass{a4paper}{article}|
% \end{quote}
% An example follows how to generate the
% documentation with pdf\LaTeX:
% \begin{quote}
%\begin{verbatim}
%pdflatex hypdestopt.dtx
%makeindex -s gind.ist hypdestopt.idx
%pdflatex hypdestopt.dtx
%makeindex -s gind.ist hypdestopt.idx
%pdflatex hypdestopt.dtx
%\end{verbatim}
% \end{quote}
%
% \section{Catalogue}
%
% The following XML file can be used as source for the
% \href{http://mirror.ctan.org/help/Catalogue/catalogue.html}{\TeX\ Catalogue}.
% The elements \texttt{caption} and \texttt{description} are imported
% from the original XML file from the Catalogue.
% The name of the XML file in the Catalogue is \xfile{hypdestopt.xml}.
%    \begin{macrocode}
%<*catalogue>
<?xml version='1.0' encoding='us-ascii'?>
<!DOCTYPE entry SYSTEM 'catalogue.dtd'>
<entry datestamp='$Date$' modifier='$Author$' id='hypdestopt'>
  <name>hypdestopt</name>
  <caption>Hyperref destination optimizer.</caption>
  <authorref id='auth:oberdiek'/>
  <copyright owner='Heiko Oberdiek' year='2006-2008,2011'/>
  <license type='lppl1.3'/>
  <version number='2.3'/>
  <description>
    This package supports <xref refid='hyperref'>hyperref</xref>'s
    pdftex driver. It removes unnecessary destinations
    and shortens the destination names or uses numbered destinations
    to get smaller PDF files.
    <p/>
    The package is part of the <xref refid='oberdiek'>oberdiek</xref>
    bundle.
  </description>
  <documentation details='Package documentation'
      href='ctan:/macros/latex/contrib/oberdiek/hypdestopt.pdf'/>
  <ctan file='true' path='/macros/latex/contrib/oberdiek/hypdestopt.dtx'/>
  <miktex location='oberdiek'/>
  <texlive location='oberdiek'/>
  <install path='/macros/latex/contrib/oberdiek/oberdiek.tds.zip'/>
</entry>
%</catalogue>
%    \end{macrocode}
%
% \begin{thebibliography}{9}
%
% \bibitem{alphalph}
%   Heiko Oberdiek: \textit{The \xpackage{alphalph} package};
%   2006/05/30 v1.4;
%   \CTAN{macros/latex/contrib/oberdiek/alphalph.pdf}.
%
% \bibitem{hyperref}
%   Sebastian Rahtz, Heiko Oberdiek:
%   \textit{The \xpackage{hyperref} package};
%   2006/06/01 v6.75a;
%   \CTAN{macros/latex/contrib/hyperref/}.
%
% \bibitem{ifpdf}
%   Heiko Oberdiek: \textit{The \xpackage{ifpdf} package};
%   2006/02/20 v1.4;
%   \CTAN{macros/latex/contrib/oberdiek/ifpdf.pdf}.
%
% \end{thebibliography}
%
% \begin{History}
%   \begin{Version}{2006/06/01 v1.0}
%   \item
%     First version.
%   \end{Version}
%   \begin{Version}{2006/06/01 v2.0}
%   \item
%     New method for referencing destinations by number; an idea
%     proposed by Lars Hellstr\"om in the mailing list LATEX-L.
%   \item
%     Options \xoption{name} and \xoption{num} added.
%   \end{Version}
%   \begin{Version}{2007/11/11 v2.1}
%   \item
%     Use of package \xpackage{pdftexcmds} for \LuaTeX\ support.
%   \end{Version}
%   \begin{Version}{2008/08/08 v2.2}
%   \item
%     Support for package \xpackage{bookmark} added.
%   \end{Version}
%   \begin{Version}{2011/05/13 v2.3}
%   \item
%     Fix for \cs{Hy@DestName} if the destination name contains
%     special characters.
%   \item
%     Fix for option \xoption{name} and package \xpackage{bookmark}.
%   \end{Version}
% \end{History}
%
% \PrintIndex
%
% \Finale
\endinput
|
% \end{quote}
% Do not forget to quote the argument according to the demands
% of your shell.
%
% \paragraph{Generating the documentation.}
% You can use both the \xfile{.dtx} or the \xfile{.drv} to generate
% the documentation. The process can be configured by the
% configuration file \xfile{ltxdoc.cfg}. For instance, put this
% line into this file, if you want to have A4 as paper format:
% \begin{quote}
%   \verb|\PassOptionsToClass{a4paper}{article}|
% \end{quote}
% An example follows how to generate the
% documentation with pdf\LaTeX:
% \begin{quote}
%\begin{verbatim}
%pdflatex hypdestopt.dtx
%makeindex -s gind.ist hypdestopt.idx
%pdflatex hypdestopt.dtx
%makeindex -s gind.ist hypdestopt.idx
%pdflatex hypdestopt.dtx
%\end{verbatim}
% \end{quote}
%
% \section{Catalogue}
%
% The following XML file can be used as source for the
% \href{http://mirror.ctan.org/help/Catalogue/catalogue.html}{\TeX\ Catalogue}.
% The elements \texttt{caption} and \texttt{description} are imported
% from the original XML file from the Catalogue.
% The name of the XML file in the Catalogue is \xfile{hypdestopt.xml}.
%    \begin{macrocode}
%<*catalogue>
<?xml version='1.0' encoding='us-ascii'?>
<!DOCTYPE entry SYSTEM 'catalogue.dtd'>
<entry datestamp='$Date$' modifier='$Author$' id='hypdestopt'>
  <name>hypdestopt</name>
  <caption>Hyperref destination optimizer.</caption>
  <authorref id='auth:oberdiek'/>
  <copyright owner='Heiko Oberdiek' year='2006-2008,2011'/>
  <license type='lppl1.3'/>
  <version number='2.3'/>
  <description>
    This package supports <xref refid='hyperref'>hyperref</xref>'s
    pdftex driver. It removes unnecessary destinations
    and shortens the destination names or uses numbered destinations
    to get smaller PDF files.
    <p/>
    The package is part of the <xref refid='oberdiek'>oberdiek</xref>
    bundle.
  </description>
  <documentation details='Package documentation'
      href='ctan:/macros/latex/contrib/oberdiek/hypdestopt.pdf'/>
  <ctan file='true' path='/macros/latex/contrib/oberdiek/hypdestopt.dtx'/>
  <miktex location='oberdiek'/>
  <texlive location='oberdiek'/>
  <install path='/macros/latex/contrib/oberdiek/oberdiek.tds.zip'/>
</entry>
%</catalogue>
%    \end{macrocode}
%
% \begin{thebibliography}{9}
%
% \bibitem{alphalph}
%   Heiko Oberdiek: \textit{The \xpackage{alphalph} package};
%   2006/05/30 v1.4;
%   \CTAN{macros/latex/contrib/oberdiek/alphalph.pdf}.
%
% \bibitem{hyperref}
%   Sebastian Rahtz, Heiko Oberdiek:
%   \textit{The \xpackage{hyperref} package};
%   2006/06/01 v6.75a;
%   \CTAN{macros/latex/contrib/hyperref/}.
%
% \bibitem{ifpdf}
%   Heiko Oberdiek: \textit{The \xpackage{ifpdf} package};
%   2006/02/20 v1.4;
%   \CTAN{macros/latex/contrib/oberdiek/ifpdf.pdf}.
%
% \end{thebibliography}
%
% \begin{History}
%   \begin{Version}{2006/06/01 v1.0}
%   \item
%     First version.
%   \end{Version}
%   \begin{Version}{2006/06/01 v2.0}
%   \item
%     New method for referencing destinations by number; an idea
%     proposed by Lars Hellstr\"om in the mailing list LATEX-L.
%   \item
%     Options \xoption{name} and \xoption{num} added.
%   \end{Version}
%   \begin{Version}{2007/11/11 v2.1}
%   \item
%     Use of package \xpackage{pdftexcmds} for \LuaTeX\ support.
%   \end{Version}
%   \begin{Version}{2008/08/08 v2.2}
%   \item
%     Support for package \xpackage{bookmark} added.
%   \end{Version}
%   \begin{Version}{2011/05/13 v2.3}
%   \item
%     Fix for \cs{Hy@DestName} if the destination name contains
%     special characters.
%   \item
%     Fix for option \xoption{name} and package \xpackage{bookmark}.
%   \end{Version}
% \end{History}
%
% \PrintIndex
%
% \Finale
\endinput

%        (quote the arguments according to the demands of your shell)
%
% Documentation:
%    (a) If hypdestopt.drv is present:
%           latex hypdestopt.drv
%    (b) Without hypdestopt.drv:
%           latex hypdestopt.dtx; ...
%    The class ltxdoc loads the configuration file ltxdoc.cfg
%    if available. Here you can specify further options, e.g.
%    use A4 as paper format:
%       \PassOptionsToClass{a4paper}{article}
%
%    Programm calls to get the documentation (example):
%       pdflatex hypdestopt.dtx
%       makeindex -s gind.ist hypdestopt.idx
%       pdflatex hypdestopt.dtx
%       makeindex -s gind.ist hypdestopt.idx
%       pdflatex hypdestopt.dtx
%
% Installation:
%    TDS:tex/latex/oberdiek/hypdestopt.sty
%    TDS:doc/latex/oberdiek/hypdestopt.pdf
%    TDS:source/latex/oberdiek/hypdestopt.dtx
%
%<*ignore>
\begingroup
  \catcode123=1 %
  \catcode125=2 %
  \def\x{LaTeX2e}%
\expandafter\endgroup
\ifcase 0\ifx\install y1\fi\expandafter
         \ifx\csname processbatchFile\endcsname\relax\else1\fi
         \ifx\fmtname\x\else 1\fi\relax
\else\csname fi\endcsname
%</ignore>
%<*install>
\input docstrip.tex
\Msg{************************************************************************}
\Msg{* Installation}
\Msg{* Package: hypdestopt 2011/05/13 v2.3 Hyperref destination optimizer (HO)}
\Msg{************************************************************************}

\keepsilent
\askforoverwritefalse

\let\MetaPrefix\relax
\preamble

This is a generated file.

Project: hypdestopt
Version: 2011/05/13 v2.3

Copyright (C) 2006-2008, 2011 by
   Heiko Oberdiek <heiko.oberdiek at googlemail.com>

This work may be distributed and/or modified under the
conditions of the LaTeX Project Public License, either
version 1.3c of this license or (at your option) any later
version. This version of this license is in
   http://www.latex-project.org/lppl/lppl-1-3c.txt
and the latest version of this license is in
   http://www.latex-project.org/lppl.txt
and version 1.3 or later is part of all distributions of
LaTeX version 2005/12/01 or later.

This work has the LPPL maintenance status "maintained".

This Current Maintainer of this work is Heiko Oberdiek.

This work consists of the main source file hypdestopt.dtx
and the derived files
   hypdestopt.sty, hypdestopt.pdf, hypdestopt.ins, hypdestopt.drv.

\endpreamble
\let\MetaPrefix\DoubleperCent

\generate{%
  \file{hypdestopt.ins}{\from{hypdestopt.dtx}{install}}%
  \file{hypdestopt.drv}{\from{hypdestopt.dtx}{driver}}%
  \usedir{tex/latex/oberdiek}%
  \file{hypdestopt.sty}{\from{hypdestopt.dtx}{package}}%
  \nopreamble
  \nopostamble
  \usedir{source/latex/oberdiek/catalogue}%
  \file{hypdestopt.xml}{\from{hypdestopt.dtx}{catalogue}}%
}

\catcode32=13\relax% active space
\let =\space%
\Msg{************************************************************************}
\Msg{*}
\Msg{* To finish the installation you have to move the following}
\Msg{* file into a directory searched by TeX:}
\Msg{*}
\Msg{*     hypdestopt.sty}
\Msg{*}
\Msg{* To produce the documentation run the file `hypdestopt.drv'}
\Msg{* through LaTeX.}
\Msg{*}
\Msg{* Happy TeXing!}
\Msg{*}
\Msg{************************************************************************}

\endbatchfile
%</install>
%<*ignore>
\fi
%</ignore>
%<*driver>
\NeedsTeXFormat{LaTeX2e}
\ProvidesFile{hypdestopt.drv}%
  [2011/05/13 v2.3 Hyperref destination optimizer (HO)]%
\documentclass{ltxdoc}
\usepackage{holtxdoc}[2011/11/22]
\begin{document}
  \DocInput{hypdestopt.dtx}%
\end{document}
%</driver>
% \fi
%
% \CheckSum{565}
%
% \CharacterTable
%  {Upper-case    \A\B\C\D\E\F\G\H\I\J\K\L\M\N\O\P\Q\R\S\T\U\V\W\X\Y\Z
%   Lower-case    \a\b\c\d\e\f\g\h\i\j\k\l\m\n\o\p\q\r\s\t\u\v\w\x\y\z
%   Digits        \0\1\2\3\4\5\6\7\8\9
%   Exclamation   \!     Double quote  \"     Hash (number) \#
%   Dollar        \$     Percent       \%     Ampersand     \&
%   Acute accent  \'     Left paren    \(     Right paren   \)
%   Asterisk      \*     Plus          \+     Comma         \,
%   Minus         \-     Point         \.     Solidus       \/
%   Colon         \:     Semicolon     \;     Less than     \<
%   Equals        \=     Greater than  \>     Question mark \?
%   Commercial at \@     Left bracket  \[     Backslash     \\
%   Right bracket \]     Circumflex    \^     Underscore    \_
%   Grave accent  \`     Left brace    \{     Vertical bar  \|
%   Right brace   \}     Tilde         \~}
%
% \GetFileInfo{hypdestopt.drv}
%
% \title{The \xpackage{hypdestopt} package}
% \date{2011/05/13 v2.3}
% \author{Heiko Oberdiek\\\xemail{heiko.oberdiek at googlemail.com}}
%
% \maketitle
%
% \begin{abstract}
% Package \xpackage{hypdestopt} supports \xpackage{hyperref}'s
% \xoption{pdftex} driver. It removes unnecessary destinations
% and shortens the destination names or uses numbered destinations
% to get smaller PDF files.
% \end{abstract}
%
% \tableofcontents
%
% \section{User interface}
%
% \subsection{Introduction}
%
% Before PDF-1.5 annotations and destinations cannot be compressed.
% If the destination names are not needed for external use, the
% file size can be decreased by the following means:
% \begin{itemize}
% \item Unused destinations are removed.
% \item The destination names are shortened (option \xoption{name}).
% \item Using numbered destinations (option \xoption{num}).
% \end{itemize}
%
% \subsection{Requirements}
%
% \begin{itemize}
% \item Package \xpackage{hyperref} 2006/06/01 v6.75a or newer
%       (\cite{hyperref}).
% \item Package \xpackage{alphalph} 2006/05/30 v1.4 or newer
%       (\cite{alphalph}), if option \xoption{name} is used.
% \item Package \xpackage{ifpdf} (\cite{ifpdf}).
% \item \pdfTeX\ 1.30.0 or newer.
% \item \pdfTeX\ in PDF mode.
% \item \eTeX\ extensions enabled.
% \item Probably an additional compile run of \pdfLaTeX\ is necessary.
% \end{itemize}
%
% In the first compile runs you can get warnings such as:
%\begin{quote}
%\begin{verbatim}
%! pdfTeX warning (dest): name{...} has been referenced ...
%\end{verbatim}
%\end{quote}
% These warnings should vanish in later compile runs.
% However these warnings also can occur without this package.
% The package does not cure them, thus these warnings will remain,
% but the destination name can be different. In such cases test
% without package, too.
%
% \subsection{Use}
%
% If the requirements are met, load the package:
%\begin{quote}
%\verb|\usepackage{hypdestopt}|
%\end{quote}
%
% The following options are supported:
% \begin{description}
% \item[\xoption{verbose}:] Verbose debug output is enabled and written
%   in the protocol file.
% \item[\xoption{num}:] Numbered destinations are used. The file size
%   is smaller, because names are no longer used.
%   This is the default.
% \item[\xoption{name}:] Destinations are identified by names.
% \end{description}
%
% \subsection{Limitations}
%
% \begin{itemize}
% \item Forget this package, if you need preserved destination names.
% \item Destination name strings use all bytes (0..255) except
%       the carriage return (13), left parenthesis (40), right
%       parenthesis (41), and backslash (92), because they
%       must be quoted in general and therefore occupy two bytes
%       instead of one.
%
%       Further the zero byte (0) is avoided for programs
%       that implement strings using zero terminated C strings.
%       And 255 (0xFF) is avoided to get rid of a possible
%       unicode marker at the begin.
%
%       So far I have not seen problems with:
%       \begin{itemize}
%       \item AcrobatReader 5.08/Linux
%       \item AcrobatReader 7.0/Linux
%       \item xpdf 3.00
%       \item Ghostscript 8.50
%       \item gv 3.5.8
%       \item GSview 4.6
%       \end{itemize}
%       But I have not tested all and all possible PDF viewers.
% \item Use of named destinations (\cs{pdfdest}, \cs{pdfoutline},
%       \cs{pdfstartlink}, \dots) that are not supported by this
%       package.
% \item Currently only \xpackage{hyperref} with \pdfTeX\ in PDF
%       mode is supported.
% \end{itemize}
%
% \subsection{Future}
%
% A more general approach is a PDF postprocessor that takes
% a PDF file, performs some transformations and writes the
% result in a more optimized PDF file. Then it does not depend,
% how the original PDF file was generated and further improvements
% are easier to apply. For example, the destination names could be sorted:
% often used destination names would then be shorter than seldom used ones.
%
% \StopEventually{
% }
%
% \section{Implementation}
%
% \subsection{Identification}
%
%    \begin{macrocode}
%<*package>
\NeedsTeXFormat{LaTeX2e}
\ProvidesPackage{hypdestopt}%
  [2011/05/13 v2.3 Hyperref destination optimizer (HO)]%
%    \end{macrocode}
%
% \subsection{Options}
%
% \subsubsection{Option \xoption{verbose}}
%
%    \begin{macrocode}
\newif\ifHypDest@Verbose
\DeclareOption{verbose}{\HypDest@Verbosetrue}
%    \end{macrocode}
%
%    \begin{macro}{\HypDest@VerboseInfo}
%    Wrapper for verbose messages.
%    \begin{macrocode}
\def\HypDest@VerboseInfo#1{%
  \ifHypDest@Verbose
    \PackageInfo{hypdestopt}{#1}%
  \fi
}
%    \end{macrocode}
%    \end{macro}
%
% \subsubsection{Options \xoption{num} and \xoption{name}}
%
%    The options \xoption{num} or \xoption{name} specify
%    the method, how destinations are referenced (by name or
%    number). Default is option \xoption{num}.
%    \begin{macrocode}
\newif\ifHypDest@name
\DeclareOption{num}{\HypDest@namefalse}
\DeclareOption{name}{\HypDest@nametrue}
%    \end{macrocode}
%
%    \begin{macrocode}
\ProcessOptions*\relax
%    \end{macrocode}
%
% \subsection{Check requirements}
%
%    First \pdfTeX\ must running in PDF mode.
%    \begin{macrocode}
\RequirePackage{ifpdf}[2007/09/09]
\RequirePackage{pdftexcmds}[2007/11/11]
\ifpdf
\else
  \PackageError{hypdestopt}{%
    This package requires pdfTeX in PDF mode%
  }\@ehc
  \expandafter\endinput
\fi
%    \end{macrocode}
%    The version of \pdfTeX\ must not be too old, because
%    \cs{pdfescapehex} and \cs{pdfunescapehex} are used.
%    \begin{macrocode}
\begingroup\expandafter\expandafter\expandafter\endgroup
\expandafter\ifx\csname pdf@escapehex\endcsname\relax
  \PackageError{hypdestopt}{%
    This pdfTeX is too old, at least 1.30.0 is required%
  }\@ehc
  \expandafter\endinput
\fi
%    \end{macrocode}
%    Features of \eTeX\ are used, e.g. \cs{numexpr}.
%    \begin{macrocode}
\begingroup\expandafter\expandafter\expandafter\endgroup
\expandafter\ifx\csname numexpr\endcsname\relax
  \PackageError{hypdestopt}{%
    e-TeX features are missing%
  }\@ehc
  \expandafter\endinput
\fi
%    \end{macrocode}
%    Package \xpackage{alphalph} provides \cs{newalphalph} since
%    version 2006/05/30 v1.4.
%    \begin{macrocode}
\ifHypDest@name
  \RequirePackage{alphalph}[2006/05/30]%
\fi
%    \end{macrocode}
%    \begin{macrocode}
\RequirePackage{auxhook}[2009/12/14]
\RequirePackage{pdfescape}[2007/04/21]
%    \end{macrocode}
%
% \subsection{Preamble for auxiliary file}
%
%    Provide dummy definitions for the macros that are used in the
%    auxiliary files. If the package is used no longer, then these
%    commands will not generate errors.
%
%    \begin{macro}{\HypDest@PrependDocument}
%    We add our stuff in front of the \cs{AtBeginDocument} hook
%    to ensure that we are before \xpackage{hyperref}'s stuff.
%    \begin{macrocode}
\long\def\HypDest@PrependDocument#1{%
  \begingroup
    \toks\z@{#1}%
    \toks\tw@\expandafter{\@begindocumenthook}%
    \xdef\@begindocumenthook{\the\toks\z@\the\toks\tw@}%
  \endgroup
}
%    \end{macrocode}
%    \end{macro}
%    \begin{macrocode}
\AddLineBeginAux{%
  \string\providecommand{\string\HypDest@Use}[1]{}%
}
%    \end{macrocode}
%
% \subsection{Generation of destination names}
%
%    Counter |HypDest| is used for identifying destinations.
%    \begin{macrocode}
\newcounter{HypDest}
%    \end{macrocode}
%
%    \begin{macrocode}
\ifHypDest@name
%    \end{macrocode}
%
%    \begin{macro}{\HypDest@HexChar}
%    Destination names are generated by automatically
%    numbering with the help of package \xpackage{alphalph}.
%    \cs{HypDest@HexChar} converts a number of the range 1 until 252
%    into the hexadecimal representation of the string character.
%    \begin{macrocode}
  \def\HypDest@HexChar#1{%
    \ifcase#1\or
%    \end{macrocode}
%    Avoid zero byte because of C strings in PDF viewer
%    applications.
%    \begin{macrocode}
      01\or 02\or 03\or 04\or 05\or 06\or 07\or
%    \end{macrocode}
%    Omit carriage return (13/\verb|^^0d|).
%    It needs quoting, otherwise it would be converted
%    to line feed (10/\verb|^^0a|).
%    \begin{macrocode}
      08\or 09\or 0A\or 0B\or 0C\or 0E\or 0F\or
      10\or 11\or 12\or 13\or 14\or 15\or 16\or 17\or
      18\or 19\or 1A\or 1B\or 1C\or 1D\or 1E\or 1F\or
      20\or 21\or 22\or 23\or 24\or 25\or 26\or 27\or
%    \end{macrocode}
%    Omit left and right parentheses (40/\verb|^^28|, 41/\verb|^^39|),
%    they need quoting in general.
%    \begin{macrocode}
      2A\or 2B\or 2C\or 2D\or 2E\or 2F\or
      30\or 31\or 32\or 33\or 34\or 35\or 36\or 37\or
      38\or 39\or 3A\or 3B\or 3C\or 3D\or 3E\or 3F\or
      40\or 41\or 42\or 43\or 44\or 45\or 46\or 47\or
      48\or 49\or 4A\or 4B\or 4C\or 4D\or 4E\or 4F\or
      50\or 51\or 52\or 53\or 54\or 55\or 56\or 57\or
%    \end{macrocode}
%    Omit backslash (92/\verb|^^5C|), it needs quoting.
%    \begin{macrocode}
      58\or 59\or 5A\or 5B\or 5D\or 5E\or 5F\or
      60\or 61\or 62\or 63\or 64\or 65\or 66\or 67\or
      68\or 69\or 6A\or 6B\or 6C\or 6D\or 6E\or 6F\or
      70\or 71\or 72\or 73\or 74\or 75\or 76\or 77\or
      78\or 79\or 7A\or 7B\or 7C\or 7D\or 7E\or 7F\or
      80\or 81\or 82\or 83\or 84\or 85\or 86\or 87\or
      88\or 89\or 8A\or 8B\or 8C\or 8D\or 8E\or 8F\or
      90\or 91\or 92\or 93\or 94\or 95\or 96\or 97\or
      98\or 99\or 9A\or 9B\or 9C\or 9D\or 9E\or 9F\or
      A0\or A1\or A2\or A3\or A4\or A5\or A6\or A7\or
      A8\or A9\or AA\or AB\or AC\or AD\or AE\or AF\or
      B0\or B1\or B2\or B3\or B4\or B5\or B6\or B7\or
      B8\or B9\or BA\or BB\or BC\or BD\or BE\or BF\or
      C0\or C1\or C2\or C3\or C4\or C5\or C6\or C7\or
      C8\or C9\or CA\or CB\or CC\or CD\or CE\or CF\or
      D0\or D1\or D2\or D3\or D4\or D5\or D6\or D7\or
      D8\or D9\or DA\or DB\or DC\or DD\or DE\or DF\or
      E0\or E1\or E2\or E3\or E4\or E5\or E6\or E7\or
      E8\or E9\or EA\or EB\or EC\or ED\or EE\or EF\or
      F0\or F1\or F2\or F3\or F4\or F5\or F6\or F7\or
%    \end{macrocode}
%    Avoid 255 (0xFF) to get rid of a possible unicode
%    marker at the begin of the string.
%    \begin{macrocode}
      F8\or F9\or FA\or FB\or FC\or FD\or FE%
    \fi
  }%
%    \end{macrocode}
%    \end{macro}
%    \begin{macro}{HypDest@HexString}
%    Now package \xpackage{alphalph} comes into play.
%    \cs{HypDest@HexString} is defined and converts
%    a positive number into a string, given in hexadecimal
%    representation.
%    \begin{macrocode}
  \newalphalph\HypDest@HexString\HypDest@HexChar{250}%
%    \end{macrocode}
%    \end{macro}
%    \begin{macro}{\theHypDest}
%    For use, the hexadecimal string is converted back.
%    \begin{macrocode}
  \renewcommand*{\theHypDest}{%
    \pdf@unescapehex{\HypDest@HexString{\value{HypDest}}}%
  }%
%    \end{macrocode}
%    \end{macro}
%
%    With option \xoption{num} we use the number directly.
%    \begin{macrocode}
\else
  \renewcommand*{\theHypDest}{%
    \number\value{HypDest}%
  }%
\fi
%    \end{macrocode}
%
% \subsection{Assign destination names}
%
%    \begin{macro}{\HypDest@Prefix}
%    The new destination names are remembered in macros whose names
%    start with prefix \cs{HypDest@Prefix}.
%    \begin{macrocode}
\edef\HypDest@Prefix{HypDest\string:}
%    \end{macrocode}
%    \end{macro}
%
%    \begin{macro}{\HypDest@Use}
%    During the first read of the auxiliary files, the used destinations
%    get fresh generated short destination names. Also for the old
%    destination names we use the hexadecimal representation. That
%    avoid problems with arbitrary names.
%    \begin{macrocode}
\def\HypDest@Use#1{%
  \begingroup
    \edef\x{%
      \expandafter\noexpand
      \csname\HypDest@Prefix\pdf@unescapehex{#1}\endcsname
    }%
    \expandafter\ifx\x\relax
      \stepcounter{HypDest}%
      \expandafter\xdef\x{\theHypDest}%
      \let\on@line\@empty
      \ifHypDest@name
        \HypDest@VerboseInfo{%
          Use: (\pdf@unescapehex{#1}) -\string> %
          0x\pdf@escapehex{\x} (\number\value{HypDest})%
        }%
      \else
        \HypDest@VerboseInfo{%
          Use: (\pdf@unescapehex{#1}) -\string> num \x
        }%
      \fi
    \fi
  \endgroup
}
%    \end{macrocode}
%    \end{macro}
%
%    After the first \xfile{.aux} file processing the destination names
%    are assigned and we can disable \cs{HypDest@Use}.
%    \begin{macrocode}
\AtBeginDocument{%
  \let\HypDest@Use\@gobble
}
%    \end{macrocode}
%
%    \begin{macro}{\HypDest@MarkUsed}
%    Destinations that are actually used are marked by \cs{HypDest@MarkUsed}.
%    \cs{nofiles} is respected.
%    \begin{macrocode}
\def\HypDest@MarkUsed#1{%
  \HypDest@VerboseInfo{%
    MarkUsed: (#1)%
  }%
  \if@filesw
    \immediate\write\@auxout{%
      \string\HypDest@Use{\pdf@escapehex{#1}}%
    }%
  \fi
}%
%    \end{macrocode}
%    \end{macro}
%
% \subsection{Redefinition of \xpackage{hyperref}'s hooks}
%
%    Package \xpackage{hyperref} can be loaded later, therefore
%    we redefine \xpackage{hyperref}'s macros at |\begin{document}|.
%    \begin{macrocode}
\HypDest@PrependDocument{%
%    \end{macrocode}
%
%    Check hyperref version.
%    \begin{macrocode}
  \@ifpackagelater{hyperref}{2006/06/01}{}{%
    \PackageError{hypdestopt}{%
      hyperref 2006/06/01 v6.75a or later is required%
    }\@ehc
  }%
%    \end{macrocode}
%
% \subsubsection{Destination setting}
%
%    \begin{macrocode}
  \ifHypDest@name
    \let\HypDest@Org@DestName\Hy@DestName
    \renewcommand*{\Hy@DestName}[2]{%
      \EdefUnescapeString\HypDest@temp{#1}%
      \@ifundefined{\HypDest@Prefix\HypDest@temp}{%
        \HypDest@VerboseInfo{%
          DestName: (\HypDest@temp) unused%
        }%
      }{%
        \HypDest@Org@DestName{%
          \csname\HypDest@Prefix\HypDest@temp\endcsname
        }{#2}%
        \HypDest@VerboseInfo{%
          DestName: (\HypDest@temp) %
          0x\pdf@escapehex{%
            \csname\HypDest@Prefix\HypDest@temp\endcsname
          }%
        }%
      }%
    }%
  \else
    \renewcommand*{\Hy@DestName}[2]{%
      \EdefUnescapeString\HypDest@temp{#1}%
      \@ifundefined{\HypDest@Prefix\HypDest@temp}{%
        \HypDest@VerboseInfo{%
          DestName: (\HypDest@temp) unused%
        }%
      }{%
        \pdfdest num%
        \csname\HypDest@Prefix\HypDest@temp\endcsname#2\relax
        \HypDest@VerboseInfo{%
          DestName: (\HypDest@temp) %
          num \csname\HypDest@Prefix\HypDest@temp\endcsname
        }%
      }%
    }%
  \fi
%    \end{macrocode}
%
% \subsubsection{Links}
%
%    \begin{macrocode}
  \let\HypDest@Org@StartlinkName\Hy@StartlinkName
  \ifHypDest@name
    \renewcommand*{\Hy@StartlinkName}[2]{%
      \HypDest@MarkUsed{#2}%
      \HypDest@Org@StartlinkName{#1}{%
        \@ifundefined{\HypDest@Prefix#2}{%
          #2%
        }{%
          \csname\HypDest@Prefix#2\endcsname
        }%
      }%
    }%
  \else
    \renewcommand*{\Hy@StartlinkName}[2]{%
      \HypDest@MarkUsed{#2}%
      \@ifundefined{\HypDest@Prefix#2}{%
        \HypDest@Org@StartlinkName{#1}{#2}%
      }{%
        \pdfstartlink attr{#1}%
                      goto num\csname\HypDest@Prefix#2\endcsname
        \relax
      }%
    }%
  \fi
%    \end{macrocode}
%
% \subsubsection{Outlines of package \xpackage{hyperref}}
%
%    \begin{macrocode}
  \let\HypDest@Org@OutlineName\Hy@OutlineName
  \ifHypDest@name
    \renewcommand*{\Hy@OutlineName}[4]{%
      \HypDest@Org@OutlineName{#1}{%
        \@ifundefined{\HypDest@Prefix#2}{%
          #2%
        }{%
          \csname\HypDest@Prefix#2\endcsname
        }%
      }{#3}{#4}%
    }%
  \else
    \renewcommand*{\Hy@OutlineName}[4]{%
      \@ifundefined{\HypDest@Prefix#2}{%
        \HypDest@Org@OutlineName{#1}{#2}{#3}{#4}%
      }{%
        \pdfoutline goto num\csname\HypDest@Prefix#2\endcsname
                    count#3{#4}%
      }%
    }%
  \fi
%    \end{macrocode}
%    Because \cs{Hy@OutlineName} is called after the \xfile{.out} file
%    is written in the previous run. Therefore we mark the destination
%    earlier in \cs{@@writetorep}.
%    \begin{macrocode}
  \let\HypDest@Org@@writetorep\@@writetorep
  \renewcommand*{\@@writetorep}[5]{%
    \begingroup
      \edef\Hy@tempa{#5}%
      \ifx\Hy@tempa\Hy@bookmarkstype
        \HypDest@MarkUsed{#3}%
      \fi
    \endgroup
    \HypDest@Org@@writetorep{#1}{#2}{#3}{#4}{#5}%
  }%
%    \end{macrocode}
%
% \subsubsection{Outlines of package \xpackage{bookmark}}
%
%    \begin{macrocode}
  \@ifpackageloaded{bookmark}{%
    \@ifpackagelater{bookmark}{2008/08/08}{%
      \renewcommand*{\BKM@DefGotoNameAction}[2]{%
        \@ifundefined{\HypDest@Prefix#2}{%
          \edef#1{goto name{hypdestopt\string :unknown}}%
        }{%
          \ifHypDest@name
            \edef#1{goto name{\csname\HypDest@Prefix#2\endcsname}}%
          \else
            \edef#1{goto num\csname\HypDest@Prefix#2\endcsname}%
          \fi
        }%
      }%
      \def\BKM@HypDestOptHook{%
        \ifx\BKM@dest\@empty
        \else
          \ifx\BKM@gotor\@empty
            \HypDest@MarkUsed\BKM@dest
          \fi
        \fi
      }%
    }{%
      \@PackageError{hypdestopt}{%
        Package `bookmark' is too old.\MessageBreak
        Version 2008/08/08 or later is needed%
      }\@ehc
    }%
  }{}%
%    \end{macrocode}
%
%    \begin{macrocode}
}
%    \end{macrocode}
%
%
%    \begin{macrocode}
%</package>
%    \end{macrocode}
%
% \section{Installation}
%
% \subsection{Download}
%
% \paragraph{Package.} This package is available on
% CTAN\footnote{\url{ftp://ftp.ctan.org/tex-archive/}}:
% \begin{description}
% \item[\CTAN{macros/latex/contrib/oberdiek/hypdestopt.dtx}] The source file.
% \item[\CTAN{macros/latex/contrib/oberdiek/hypdestopt.pdf}] Documentation.
% \end{description}
%
%
% \paragraph{Bundle.} All the packages of the bundle `oberdiek'
% are also available in a TDS compliant ZIP archive. There
% the packages are already unpacked and the documentation files
% are generated. The files and directories obey the TDS standard.
% \begin{description}
% \item[\CTAN{install/macros/latex/contrib/oberdiek.tds.zip}]
% \end{description}
% \emph{TDS} refers to the standard ``A Directory Structure
% for \TeX\ Files'' (\CTAN{tds/tds.pdf}). Directories
% with \xfile{texmf} in their name are usually organized this way.
%
% \subsection{Bundle installation}
%
% \paragraph{Unpacking.} Unpack the \xfile{oberdiek.tds.zip} in the
% TDS tree (also known as \xfile{texmf} tree) of your choice.
% Example (linux):
% \begin{quote}
%   |unzip oberdiek.tds.zip -d ~/texmf|
% \end{quote}
%
% \paragraph{Script installation.}
% Check the directory \xfile{TDS:scripts/oberdiek/} for
% scripts that need further installation steps.
% Package \xpackage{attachfile2} comes with the Perl script
% \xfile{pdfatfi.pl} that should be installed in such a way
% that it can be called as \texttt{pdfatfi}.
% Example (linux):
% \begin{quote}
%   |chmod +x scripts/oberdiek/pdfatfi.pl|\\
%   |cp scripts/oberdiek/pdfatfi.pl /usr/local/bin/|
% \end{quote}
%
% \subsection{Package installation}
%
% \paragraph{Unpacking.} The \xfile{.dtx} file is a self-extracting
% \docstrip\ archive. The files are extracted by running the
% \xfile{.dtx} through \plainTeX:
% \begin{quote}
%   \verb|tex hypdestopt.dtx|
% \end{quote}
%
% \paragraph{TDS.} Now the different files must be moved into
% the different directories in your installation TDS tree
% (also known as \xfile{texmf} tree):
% \begin{quote}
% \def\t{^^A
% \begin{tabular}{@{}>{\ttfamily}l@{ $\rightarrow$ }>{\ttfamily}l@{}}
%   hypdestopt.sty & tex/latex/oberdiek/hypdestopt.sty\\
%   hypdestopt.pdf & doc/latex/oberdiek/hypdestopt.pdf\\
%   hypdestopt.dtx & source/latex/oberdiek/hypdestopt.dtx\\
% \end{tabular}^^A
% }^^A
% \sbox0{\t}^^A
% \ifdim\wd0>\linewidth
%   \begingroup
%     \advance\linewidth by\leftmargin
%     \advance\linewidth by\rightmargin
%   \edef\x{\endgroup
%     \def\noexpand\lw{\the\linewidth}^^A
%   }\x
%   \def\lwbox{^^A
%     \leavevmode
%     \hbox to \linewidth{^^A
%       \kern-\leftmargin\relax
%       \hss
%       \usebox0
%       \hss
%       \kern-\rightmargin\relax
%     }^^A
%   }^^A
%   \ifdim\wd0>\lw
%     \sbox0{\small\t}^^A
%     \ifdim\wd0>\linewidth
%       \ifdim\wd0>\lw
%         \sbox0{\footnotesize\t}^^A
%         \ifdim\wd0>\linewidth
%           \ifdim\wd0>\lw
%             \sbox0{\scriptsize\t}^^A
%             \ifdim\wd0>\linewidth
%               \ifdim\wd0>\lw
%                 \sbox0{\tiny\t}^^A
%                 \ifdim\wd0>\linewidth
%                   \lwbox
%                 \else
%                   \usebox0
%                 \fi
%               \else
%                 \lwbox
%               \fi
%             \else
%               \usebox0
%             \fi
%           \else
%             \lwbox
%           \fi
%         \else
%           \usebox0
%         \fi
%       \else
%         \lwbox
%       \fi
%     \else
%       \usebox0
%     \fi
%   \else
%     \lwbox
%   \fi
% \else
%   \usebox0
% \fi
% \end{quote}
% If you have a \xfile{docstrip.cfg} that configures and enables \docstrip's
% TDS installing feature, then some files can already be in the right
% place, see the documentation of \docstrip.
%
% \subsection{Refresh file name databases}
%
% If your \TeX~distribution
% (\teTeX, \mikTeX, \dots) relies on file name databases, you must refresh
% these. For example, \teTeX\ users run \verb|texhash| or
% \verb|mktexlsr|.
%
% \subsection{Some details for the interested}
%
% \paragraph{Attached source.}
%
% The PDF documentation on CTAN also includes the
% \xfile{.dtx} source file. It can be extracted by
% AcrobatReader 6 or higher. Another option is \textsf{pdftk},
% e.g. unpack the file into the current directory:
% \begin{quote}
%   \verb|pdftk hypdestopt.pdf unpack_files output .|
% \end{quote}
%
% \paragraph{Unpacking with \LaTeX.}
% The \xfile{.dtx} chooses its action depending on the format:
% \begin{description}
% \item[\plainTeX:] Run \docstrip\ and extract the files.
% \item[\LaTeX:] Generate the documentation.
% \end{description}
% If you insist on using \LaTeX\ for \docstrip\ (really,
% \docstrip\ does not need \LaTeX), then inform the autodetect routine
% about your intention:
% \begin{quote}
%   \verb|latex \let\install=y% \iffalse meta-comment
%
% File: hypdestopt.dtx
% Version: 2011/05/13 v2.3
% Info: Hyperref destination optimizer
%
% Copyright (C) 2006-2008, 2011 by
%    Heiko Oberdiek <heiko.oberdiek at googlemail.com>
%
% This work may be distributed and/or modified under the
% conditions of the LaTeX Project Public License, either
% version 1.3c of this license or (at your option) any later
% version. This version of this license is in
%    http://www.latex-project.org/lppl/lppl-1-3c.txt
% and the latest version of this license is in
%    http://www.latex-project.org/lppl.txt
% and version 1.3 or later is part of all distributions of
% LaTeX version 2005/12/01 or later.
%
% This work has the LPPL maintenance status "maintained".
%
% This Current Maintainer of this work is Heiko Oberdiek.
%
% This work consists of the main source file hypdestopt.dtx
% and the derived files
%    hypdestopt.sty, hypdestopt.pdf, hypdestopt.ins, hypdestopt.drv.
%
% Distribution:
%    CTAN:macros/latex/contrib/oberdiek/hypdestopt.dtx
%    CTAN:macros/latex/contrib/oberdiek/hypdestopt.pdf
%
% Unpacking:
%    (a) If hypdestopt.ins is present:
%           tex hypdestopt.ins
%    (b) Without hypdestopt.ins:
%           tex hypdestopt.dtx
%    (c) If you insist on using LaTeX
%           latex \let\install=y% \iffalse meta-comment
%
% File: hypdestopt.dtx
% Version: 2011/05/13 v2.3
% Info: Hyperref destination optimizer
%
% Copyright (C) 2006-2008, 2011 by
%    Heiko Oberdiek <heiko.oberdiek at googlemail.com>
%
% This work may be distributed and/or modified under the
% conditions of the LaTeX Project Public License, either
% version 1.3c of this license or (at your option) any later
% version. This version of this license is in
%    http://www.latex-project.org/lppl/lppl-1-3c.txt
% and the latest version of this license is in
%    http://www.latex-project.org/lppl.txt
% and version 1.3 or later is part of all distributions of
% LaTeX version 2005/12/01 or later.
%
% This work has the LPPL maintenance status "maintained".
%
% This Current Maintainer of this work is Heiko Oberdiek.
%
% This work consists of the main source file hypdestopt.dtx
% and the derived files
%    hypdestopt.sty, hypdestopt.pdf, hypdestopt.ins, hypdestopt.drv.
%
% Distribution:
%    CTAN:macros/latex/contrib/oberdiek/hypdestopt.dtx
%    CTAN:macros/latex/contrib/oberdiek/hypdestopt.pdf
%
% Unpacking:
%    (a) If hypdestopt.ins is present:
%           tex hypdestopt.ins
%    (b) Without hypdestopt.ins:
%           tex hypdestopt.dtx
%    (c) If you insist on using LaTeX
%           latex \let\install=y\input{hypdestopt.dtx}
%        (quote the arguments according to the demands of your shell)
%
% Documentation:
%    (a) If hypdestopt.drv is present:
%           latex hypdestopt.drv
%    (b) Without hypdestopt.drv:
%           latex hypdestopt.dtx; ...
%    The class ltxdoc loads the configuration file ltxdoc.cfg
%    if available. Here you can specify further options, e.g.
%    use A4 as paper format:
%       \PassOptionsToClass{a4paper}{article}
%
%    Programm calls to get the documentation (example):
%       pdflatex hypdestopt.dtx
%       makeindex -s gind.ist hypdestopt.idx
%       pdflatex hypdestopt.dtx
%       makeindex -s gind.ist hypdestopt.idx
%       pdflatex hypdestopt.dtx
%
% Installation:
%    TDS:tex/latex/oberdiek/hypdestopt.sty
%    TDS:doc/latex/oberdiek/hypdestopt.pdf
%    TDS:source/latex/oberdiek/hypdestopt.dtx
%
%<*ignore>
\begingroup
  \catcode123=1 %
  \catcode125=2 %
  \def\x{LaTeX2e}%
\expandafter\endgroup
\ifcase 0\ifx\install y1\fi\expandafter
         \ifx\csname processbatchFile\endcsname\relax\else1\fi
         \ifx\fmtname\x\else 1\fi\relax
\else\csname fi\endcsname
%</ignore>
%<*install>
\input docstrip.tex
\Msg{************************************************************************}
\Msg{* Installation}
\Msg{* Package: hypdestopt 2011/05/13 v2.3 Hyperref destination optimizer (HO)}
\Msg{************************************************************************}

\keepsilent
\askforoverwritefalse

\let\MetaPrefix\relax
\preamble

This is a generated file.

Project: hypdestopt
Version: 2011/05/13 v2.3

Copyright (C) 2006-2008, 2011 by
   Heiko Oberdiek <heiko.oberdiek at googlemail.com>

This work may be distributed and/or modified under the
conditions of the LaTeX Project Public License, either
version 1.3c of this license or (at your option) any later
version. This version of this license is in
   http://www.latex-project.org/lppl/lppl-1-3c.txt
and the latest version of this license is in
   http://www.latex-project.org/lppl.txt
and version 1.3 or later is part of all distributions of
LaTeX version 2005/12/01 or later.

This work has the LPPL maintenance status "maintained".

This Current Maintainer of this work is Heiko Oberdiek.

This work consists of the main source file hypdestopt.dtx
and the derived files
   hypdestopt.sty, hypdestopt.pdf, hypdestopt.ins, hypdestopt.drv.

\endpreamble
\let\MetaPrefix\DoubleperCent

\generate{%
  \file{hypdestopt.ins}{\from{hypdestopt.dtx}{install}}%
  \file{hypdestopt.drv}{\from{hypdestopt.dtx}{driver}}%
  \usedir{tex/latex/oberdiek}%
  \file{hypdestopt.sty}{\from{hypdestopt.dtx}{package}}%
  \nopreamble
  \nopostamble
  \usedir{source/latex/oberdiek/catalogue}%
  \file{hypdestopt.xml}{\from{hypdestopt.dtx}{catalogue}}%
}

\catcode32=13\relax% active space
\let =\space%
\Msg{************************************************************************}
\Msg{*}
\Msg{* To finish the installation you have to move the following}
\Msg{* file into a directory searched by TeX:}
\Msg{*}
\Msg{*     hypdestopt.sty}
\Msg{*}
\Msg{* To produce the documentation run the file `hypdestopt.drv'}
\Msg{* through LaTeX.}
\Msg{*}
\Msg{* Happy TeXing!}
\Msg{*}
\Msg{************************************************************************}

\endbatchfile
%</install>
%<*ignore>
\fi
%</ignore>
%<*driver>
\NeedsTeXFormat{LaTeX2e}
\ProvidesFile{hypdestopt.drv}%
  [2011/05/13 v2.3 Hyperref destination optimizer (HO)]%
\documentclass{ltxdoc}
\usepackage{holtxdoc}[2011/11/22]
\begin{document}
  \DocInput{hypdestopt.dtx}%
\end{document}
%</driver>
% \fi
%
% \CheckSum{565}
%
% \CharacterTable
%  {Upper-case    \A\B\C\D\E\F\G\H\I\J\K\L\M\N\O\P\Q\R\S\T\U\V\W\X\Y\Z
%   Lower-case    \a\b\c\d\e\f\g\h\i\j\k\l\m\n\o\p\q\r\s\t\u\v\w\x\y\z
%   Digits        \0\1\2\3\4\5\6\7\8\9
%   Exclamation   \!     Double quote  \"     Hash (number) \#
%   Dollar        \$     Percent       \%     Ampersand     \&
%   Acute accent  \'     Left paren    \(     Right paren   \)
%   Asterisk      \*     Plus          \+     Comma         \,
%   Minus         \-     Point         \.     Solidus       \/
%   Colon         \:     Semicolon     \;     Less than     \<
%   Equals        \=     Greater than  \>     Question mark \?
%   Commercial at \@     Left bracket  \[     Backslash     \\
%   Right bracket \]     Circumflex    \^     Underscore    \_
%   Grave accent  \`     Left brace    \{     Vertical bar  \|
%   Right brace   \}     Tilde         \~}
%
% \GetFileInfo{hypdestopt.drv}
%
% \title{The \xpackage{hypdestopt} package}
% \date{2011/05/13 v2.3}
% \author{Heiko Oberdiek\\\xemail{heiko.oberdiek at googlemail.com}}
%
% \maketitle
%
% \begin{abstract}
% Package \xpackage{hypdestopt} supports \xpackage{hyperref}'s
% \xoption{pdftex} driver. It removes unnecessary destinations
% and shortens the destination names or uses numbered destinations
% to get smaller PDF files.
% \end{abstract}
%
% \tableofcontents
%
% \section{User interface}
%
% \subsection{Introduction}
%
% Before PDF-1.5 annotations and destinations cannot be compressed.
% If the destination names are not needed for external use, the
% file size can be decreased by the following means:
% \begin{itemize}
% \item Unused destinations are removed.
% \item The destination names are shortened (option \xoption{name}).
% \item Using numbered destinations (option \xoption{num}).
% \end{itemize}
%
% \subsection{Requirements}
%
% \begin{itemize}
% \item Package \xpackage{hyperref} 2006/06/01 v6.75a or newer
%       (\cite{hyperref}).
% \item Package \xpackage{alphalph} 2006/05/30 v1.4 or newer
%       (\cite{alphalph}), if option \xoption{name} is used.
% \item Package \xpackage{ifpdf} (\cite{ifpdf}).
% \item \pdfTeX\ 1.30.0 or newer.
% \item \pdfTeX\ in PDF mode.
% \item \eTeX\ extensions enabled.
% \item Probably an additional compile run of \pdfLaTeX\ is necessary.
% \end{itemize}
%
% In the first compile runs you can get warnings such as:
%\begin{quote}
%\begin{verbatim}
%! pdfTeX warning (dest): name{...} has been referenced ...
%\end{verbatim}
%\end{quote}
% These warnings should vanish in later compile runs.
% However these warnings also can occur without this package.
% The package does not cure them, thus these warnings will remain,
% but the destination name can be different. In such cases test
% without package, too.
%
% \subsection{Use}
%
% If the requirements are met, load the package:
%\begin{quote}
%\verb|\usepackage{hypdestopt}|
%\end{quote}
%
% The following options are supported:
% \begin{description}
% \item[\xoption{verbose}:] Verbose debug output is enabled and written
%   in the protocol file.
% \item[\xoption{num}:] Numbered destinations are used. The file size
%   is smaller, because names are no longer used.
%   This is the default.
% \item[\xoption{name}:] Destinations are identified by names.
% \end{description}
%
% \subsection{Limitations}
%
% \begin{itemize}
% \item Forget this package, if you need preserved destination names.
% \item Destination name strings use all bytes (0..255) except
%       the carriage return (13), left parenthesis (40), right
%       parenthesis (41), and backslash (92), because they
%       must be quoted in general and therefore occupy two bytes
%       instead of one.
%
%       Further the zero byte (0) is avoided for programs
%       that implement strings using zero terminated C strings.
%       And 255 (0xFF) is avoided to get rid of a possible
%       unicode marker at the begin.
%
%       So far I have not seen problems with:
%       \begin{itemize}
%       \item AcrobatReader 5.08/Linux
%       \item AcrobatReader 7.0/Linux
%       \item xpdf 3.00
%       \item Ghostscript 8.50
%       \item gv 3.5.8
%       \item GSview 4.6
%       \end{itemize}
%       But I have not tested all and all possible PDF viewers.
% \item Use of named destinations (\cs{pdfdest}, \cs{pdfoutline},
%       \cs{pdfstartlink}, \dots) that are not supported by this
%       package.
% \item Currently only \xpackage{hyperref} with \pdfTeX\ in PDF
%       mode is supported.
% \end{itemize}
%
% \subsection{Future}
%
% A more general approach is a PDF postprocessor that takes
% a PDF file, performs some transformations and writes the
% result in a more optimized PDF file. Then it does not depend,
% how the original PDF file was generated and further improvements
% are easier to apply. For example, the destination names could be sorted:
% often used destination names would then be shorter than seldom used ones.
%
% \StopEventually{
% }
%
% \section{Implementation}
%
% \subsection{Identification}
%
%    \begin{macrocode}
%<*package>
\NeedsTeXFormat{LaTeX2e}
\ProvidesPackage{hypdestopt}%
  [2011/05/13 v2.3 Hyperref destination optimizer (HO)]%
%    \end{macrocode}
%
% \subsection{Options}
%
% \subsubsection{Option \xoption{verbose}}
%
%    \begin{macrocode}
\newif\ifHypDest@Verbose
\DeclareOption{verbose}{\HypDest@Verbosetrue}
%    \end{macrocode}
%
%    \begin{macro}{\HypDest@VerboseInfo}
%    Wrapper for verbose messages.
%    \begin{macrocode}
\def\HypDest@VerboseInfo#1{%
  \ifHypDest@Verbose
    \PackageInfo{hypdestopt}{#1}%
  \fi
}
%    \end{macrocode}
%    \end{macro}
%
% \subsubsection{Options \xoption{num} and \xoption{name}}
%
%    The options \xoption{num} or \xoption{name} specify
%    the method, how destinations are referenced (by name or
%    number). Default is option \xoption{num}.
%    \begin{macrocode}
\newif\ifHypDest@name
\DeclareOption{num}{\HypDest@namefalse}
\DeclareOption{name}{\HypDest@nametrue}
%    \end{macrocode}
%
%    \begin{macrocode}
\ProcessOptions*\relax
%    \end{macrocode}
%
% \subsection{Check requirements}
%
%    First \pdfTeX\ must running in PDF mode.
%    \begin{macrocode}
\RequirePackage{ifpdf}[2007/09/09]
\RequirePackage{pdftexcmds}[2007/11/11]
\ifpdf
\else
  \PackageError{hypdestopt}{%
    This package requires pdfTeX in PDF mode%
  }\@ehc
  \expandafter\endinput
\fi
%    \end{macrocode}
%    The version of \pdfTeX\ must not be too old, because
%    \cs{pdfescapehex} and \cs{pdfunescapehex} are used.
%    \begin{macrocode}
\begingroup\expandafter\expandafter\expandafter\endgroup
\expandafter\ifx\csname pdf@escapehex\endcsname\relax
  \PackageError{hypdestopt}{%
    This pdfTeX is too old, at least 1.30.0 is required%
  }\@ehc
  \expandafter\endinput
\fi
%    \end{macrocode}
%    Features of \eTeX\ are used, e.g. \cs{numexpr}.
%    \begin{macrocode}
\begingroup\expandafter\expandafter\expandafter\endgroup
\expandafter\ifx\csname numexpr\endcsname\relax
  \PackageError{hypdestopt}{%
    e-TeX features are missing%
  }\@ehc
  \expandafter\endinput
\fi
%    \end{macrocode}
%    Package \xpackage{alphalph} provides \cs{newalphalph} since
%    version 2006/05/30 v1.4.
%    \begin{macrocode}
\ifHypDest@name
  \RequirePackage{alphalph}[2006/05/30]%
\fi
%    \end{macrocode}
%    \begin{macrocode}
\RequirePackage{auxhook}[2009/12/14]
\RequirePackage{pdfescape}[2007/04/21]
%    \end{macrocode}
%
% \subsection{Preamble for auxiliary file}
%
%    Provide dummy definitions for the macros that are used in the
%    auxiliary files. If the package is used no longer, then these
%    commands will not generate errors.
%
%    \begin{macro}{\HypDest@PrependDocument}
%    We add our stuff in front of the \cs{AtBeginDocument} hook
%    to ensure that we are before \xpackage{hyperref}'s stuff.
%    \begin{macrocode}
\long\def\HypDest@PrependDocument#1{%
  \begingroup
    \toks\z@{#1}%
    \toks\tw@\expandafter{\@begindocumenthook}%
    \xdef\@begindocumenthook{\the\toks\z@\the\toks\tw@}%
  \endgroup
}
%    \end{macrocode}
%    \end{macro}
%    \begin{macrocode}
\AddLineBeginAux{%
  \string\providecommand{\string\HypDest@Use}[1]{}%
}
%    \end{macrocode}
%
% \subsection{Generation of destination names}
%
%    Counter |HypDest| is used for identifying destinations.
%    \begin{macrocode}
\newcounter{HypDest}
%    \end{macrocode}
%
%    \begin{macrocode}
\ifHypDest@name
%    \end{macrocode}
%
%    \begin{macro}{\HypDest@HexChar}
%    Destination names are generated by automatically
%    numbering with the help of package \xpackage{alphalph}.
%    \cs{HypDest@HexChar} converts a number of the range 1 until 252
%    into the hexadecimal representation of the string character.
%    \begin{macrocode}
  \def\HypDest@HexChar#1{%
    \ifcase#1\or
%    \end{macrocode}
%    Avoid zero byte because of C strings in PDF viewer
%    applications.
%    \begin{macrocode}
      01\or 02\or 03\or 04\or 05\or 06\or 07\or
%    \end{macrocode}
%    Omit carriage return (13/\verb|^^0d|).
%    It needs quoting, otherwise it would be converted
%    to line feed (10/\verb|^^0a|).
%    \begin{macrocode}
      08\or 09\or 0A\or 0B\or 0C\or 0E\or 0F\or
      10\or 11\or 12\or 13\or 14\or 15\or 16\or 17\or
      18\or 19\or 1A\or 1B\or 1C\or 1D\or 1E\or 1F\or
      20\or 21\or 22\or 23\or 24\or 25\or 26\or 27\or
%    \end{macrocode}
%    Omit left and right parentheses (40/\verb|^^28|, 41/\verb|^^39|),
%    they need quoting in general.
%    \begin{macrocode}
      2A\or 2B\or 2C\or 2D\or 2E\or 2F\or
      30\or 31\or 32\or 33\or 34\or 35\or 36\or 37\or
      38\or 39\or 3A\or 3B\or 3C\or 3D\or 3E\or 3F\or
      40\or 41\or 42\or 43\or 44\or 45\or 46\or 47\or
      48\or 49\or 4A\or 4B\or 4C\or 4D\or 4E\or 4F\or
      50\or 51\or 52\or 53\or 54\or 55\or 56\or 57\or
%    \end{macrocode}
%    Omit backslash (92/\verb|^^5C|), it needs quoting.
%    \begin{macrocode}
      58\or 59\or 5A\or 5B\or 5D\or 5E\or 5F\or
      60\or 61\or 62\or 63\or 64\or 65\or 66\or 67\or
      68\or 69\or 6A\or 6B\or 6C\or 6D\or 6E\or 6F\or
      70\or 71\or 72\or 73\or 74\or 75\or 76\or 77\or
      78\or 79\or 7A\or 7B\or 7C\or 7D\or 7E\or 7F\or
      80\or 81\or 82\or 83\or 84\or 85\or 86\or 87\or
      88\or 89\or 8A\or 8B\or 8C\or 8D\or 8E\or 8F\or
      90\or 91\or 92\or 93\or 94\or 95\or 96\or 97\or
      98\or 99\or 9A\or 9B\or 9C\or 9D\or 9E\or 9F\or
      A0\or A1\or A2\or A3\or A4\or A5\or A6\or A7\or
      A8\or A9\or AA\or AB\or AC\or AD\or AE\or AF\or
      B0\or B1\or B2\or B3\or B4\or B5\or B6\or B7\or
      B8\or B9\or BA\or BB\or BC\or BD\or BE\or BF\or
      C0\or C1\or C2\or C3\or C4\or C5\or C6\or C7\or
      C8\or C9\or CA\or CB\or CC\or CD\or CE\or CF\or
      D0\or D1\or D2\or D3\or D4\or D5\or D6\or D7\or
      D8\or D9\or DA\or DB\or DC\or DD\or DE\or DF\or
      E0\or E1\or E2\or E3\or E4\or E5\or E6\or E7\or
      E8\or E9\or EA\or EB\or EC\or ED\or EE\or EF\or
      F0\or F1\or F2\or F3\or F4\or F5\or F6\or F7\or
%    \end{macrocode}
%    Avoid 255 (0xFF) to get rid of a possible unicode
%    marker at the begin of the string.
%    \begin{macrocode}
      F8\or F9\or FA\or FB\or FC\or FD\or FE%
    \fi
  }%
%    \end{macrocode}
%    \end{macro}
%    \begin{macro}{HypDest@HexString}
%    Now package \xpackage{alphalph} comes into play.
%    \cs{HypDest@HexString} is defined and converts
%    a positive number into a string, given in hexadecimal
%    representation.
%    \begin{macrocode}
  \newalphalph\HypDest@HexString\HypDest@HexChar{250}%
%    \end{macrocode}
%    \end{macro}
%    \begin{macro}{\theHypDest}
%    For use, the hexadecimal string is converted back.
%    \begin{macrocode}
  \renewcommand*{\theHypDest}{%
    \pdf@unescapehex{\HypDest@HexString{\value{HypDest}}}%
  }%
%    \end{macrocode}
%    \end{macro}
%
%    With option \xoption{num} we use the number directly.
%    \begin{macrocode}
\else
  \renewcommand*{\theHypDest}{%
    \number\value{HypDest}%
  }%
\fi
%    \end{macrocode}
%
% \subsection{Assign destination names}
%
%    \begin{macro}{\HypDest@Prefix}
%    The new destination names are remembered in macros whose names
%    start with prefix \cs{HypDest@Prefix}.
%    \begin{macrocode}
\edef\HypDest@Prefix{HypDest\string:}
%    \end{macrocode}
%    \end{macro}
%
%    \begin{macro}{\HypDest@Use}
%    During the first read of the auxiliary files, the used destinations
%    get fresh generated short destination names. Also for the old
%    destination names we use the hexadecimal representation. That
%    avoid problems with arbitrary names.
%    \begin{macrocode}
\def\HypDest@Use#1{%
  \begingroup
    \edef\x{%
      \expandafter\noexpand
      \csname\HypDest@Prefix\pdf@unescapehex{#1}\endcsname
    }%
    \expandafter\ifx\x\relax
      \stepcounter{HypDest}%
      \expandafter\xdef\x{\theHypDest}%
      \let\on@line\@empty
      \ifHypDest@name
        \HypDest@VerboseInfo{%
          Use: (\pdf@unescapehex{#1}) -\string> %
          0x\pdf@escapehex{\x} (\number\value{HypDest})%
        }%
      \else
        \HypDest@VerboseInfo{%
          Use: (\pdf@unescapehex{#1}) -\string> num \x
        }%
      \fi
    \fi
  \endgroup
}
%    \end{macrocode}
%    \end{macro}
%
%    After the first \xfile{.aux} file processing the destination names
%    are assigned and we can disable \cs{HypDest@Use}.
%    \begin{macrocode}
\AtBeginDocument{%
  \let\HypDest@Use\@gobble
}
%    \end{macrocode}
%
%    \begin{macro}{\HypDest@MarkUsed}
%    Destinations that are actually used are marked by \cs{HypDest@MarkUsed}.
%    \cs{nofiles} is respected.
%    \begin{macrocode}
\def\HypDest@MarkUsed#1{%
  \HypDest@VerboseInfo{%
    MarkUsed: (#1)%
  }%
  \if@filesw
    \immediate\write\@auxout{%
      \string\HypDest@Use{\pdf@escapehex{#1}}%
    }%
  \fi
}%
%    \end{macrocode}
%    \end{macro}
%
% \subsection{Redefinition of \xpackage{hyperref}'s hooks}
%
%    Package \xpackage{hyperref} can be loaded later, therefore
%    we redefine \xpackage{hyperref}'s macros at |\begin{document}|.
%    \begin{macrocode}
\HypDest@PrependDocument{%
%    \end{macrocode}
%
%    Check hyperref version.
%    \begin{macrocode}
  \@ifpackagelater{hyperref}{2006/06/01}{}{%
    \PackageError{hypdestopt}{%
      hyperref 2006/06/01 v6.75a or later is required%
    }\@ehc
  }%
%    \end{macrocode}
%
% \subsubsection{Destination setting}
%
%    \begin{macrocode}
  \ifHypDest@name
    \let\HypDest@Org@DestName\Hy@DestName
    \renewcommand*{\Hy@DestName}[2]{%
      \EdefUnescapeString\HypDest@temp{#1}%
      \@ifundefined{\HypDest@Prefix\HypDest@temp}{%
        \HypDest@VerboseInfo{%
          DestName: (\HypDest@temp) unused%
        }%
      }{%
        \HypDest@Org@DestName{%
          \csname\HypDest@Prefix\HypDest@temp\endcsname
        }{#2}%
        \HypDest@VerboseInfo{%
          DestName: (\HypDest@temp) %
          0x\pdf@escapehex{%
            \csname\HypDest@Prefix\HypDest@temp\endcsname
          }%
        }%
      }%
    }%
  \else
    \renewcommand*{\Hy@DestName}[2]{%
      \EdefUnescapeString\HypDest@temp{#1}%
      \@ifundefined{\HypDest@Prefix\HypDest@temp}{%
        \HypDest@VerboseInfo{%
          DestName: (\HypDest@temp) unused%
        }%
      }{%
        \pdfdest num%
        \csname\HypDest@Prefix\HypDest@temp\endcsname#2\relax
        \HypDest@VerboseInfo{%
          DestName: (\HypDest@temp) %
          num \csname\HypDest@Prefix\HypDest@temp\endcsname
        }%
      }%
    }%
  \fi
%    \end{macrocode}
%
% \subsubsection{Links}
%
%    \begin{macrocode}
  \let\HypDest@Org@StartlinkName\Hy@StartlinkName
  \ifHypDest@name
    \renewcommand*{\Hy@StartlinkName}[2]{%
      \HypDest@MarkUsed{#2}%
      \HypDest@Org@StartlinkName{#1}{%
        \@ifundefined{\HypDest@Prefix#2}{%
          #2%
        }{%
          \csname\HypDest@Prefix#2\endcsname
        }%
      }%
    }%
  \else
    \renewcommand*{\Hy@StartlinkName}[2]{%
      \HypDest@MarkUsed{#2}%
      \@ifundefined{\HypDest@Prefix#2}{%
        \HypDest@Org@StartlinkName{#1}{#2}%
      }{%
        \pdfstartlink attr{#1}%
                      goto num\csname\HypDest@Prefix#2\endcsname
        \relax
      }%
    }%
  \fi
%    \end{macrocode}
%
% \subsubsection{Outlines of package \xpackage{hyperref}}
%
%    \begin{macrocode}
  \let\HypDest@Org@OutlineName\Hy@OutlineName
  \ifHypDest@name
    \renewcommand*{\Hy@OutlineName}[4]{%
      \HypDest@Org@OutlineName{#1}{%
        \@ifundefined{\HypDest@Prefix#2}{%
          #2%
        }{%
          \csname\HypDest@Prefix#2\endcsname
        }%
      }{#3}{#4}%
    }%
  \else
    \renewcommand*{\Hy@OutlineName}[4]{%
      \@ifundefined{\HypDest@Prefix#2}{%
        \HypDest@Org@OutlineName{#1}{#2}{#3}{#4}%
      }{%
        \pdfoutline goto num\csname\HypDest@Prefix#2\endcsname
                    count#3{#4}%
      }%
    }%
  \fi
%    \end{macrocode}
%    Because \cs{Hy@OutlineName} is called after the \xfile{.out} file
%    is written in the previous run. Therefore we mark the destination
%    earlier in \cs{@@writetorep}.
%    \begin{macrocode}
  \let\HypDest@Org@@writetorep\@@writetorep
  \renewcommand*{\@@writetorep}[5]{%
    \begingroup
      \edef\Hy@tempa{#5}%
      \ifx\Hy@tempa\Hy@bookmarkstype
        \HypDest@MarkUsed{#3}%
      \fi
    \endgroup
    \HypDest@Org@@writetorep{#1}{#2}{#3}{#4}{#5}%
  }%
%    \end{macrocode}
%
% \subsubsection{Outlines of package \xpackage{bookmark}}
%
%    \begin{macrocode}
  \@ifpackageloaded{bookmark}{%
    \@ifpackagelater{bookmark}{2008/08/08}{%
      \renewcommand*{\BKM@DefGotoNameAction}[2]{%
        \@ifundefined{\HypDest@Prefix#2}{%
          \edef#1{goto name{hypdestopt\string :unknown}}%
        }{%
          \ifHypDest@name
            \edef#1{goto name{\csname\HypDest@Prefix#2\endcsname}}%
          \else
            \edef#1{goto num\csname\HypDest@Prefix#2\endcsname}%
          \fi
        }%
      }%
      \def\BKM@HypDestOptHook{%
        \ifx\BKM@dest\@empty
        \else
          \ifx\BKM@gotor\@empty
            \HypDest@MarkUsed\BKM@dest
          \fi
        \fi
      }%
    }{%
      \@PackageError{hypdestopt}{%
        Package `bookmark' is too old.\MessageBreak
        Version 2008/08/08 or later is needed%
      }\@ehc
    }%
  }{}%
%    \end{macrocode}
%
%    \begin{macrocode}
}
%    \end{macrocode}
%
%
%    \begin{macrocode}
%</package>
%    \end{macrocode}
%
% \section{Installation}
%
% \subsection{Download}
%
% \paragraph{Package.} This package is available on
% CTAN\footnote{\url{ftp://ftp.ctan.org/tex-archive/}}:
% \begin{description}
% \item[\CTAN{macros/latex/contrib/oberdiek/hypdestopt.dtx}] The source file.
% \item[\CTAN{macros/latex/contrib/oberdiek/hypdestopt.pdf}] Documentation.
% \end{description}
%
%
% \paragraph{Bundle.} All the packages of the bundle `oberdiek'
% are also available in a TDS compliant ZIP archive. There
% the packages are already unpacked and the documentation files
% are generated. The files and directories obey the TDS standard.
% \begin{description}
% \item[\CTAN{install/macros/latex/contrib/oberdiek.tds.zip}]
% \end{description}
% \emph{TDS} refers to the standard ``A Directory Structure
% for \TeX\ Files'' (\CTAN{tds/tds.pdf}). Directories
% with \xfile{texmf} in their name are usually organized this way.
%
% \subsection{Bundle installation}
%
% \paragraph{Unpacking.} Unpack the \xfile{oberdiek.tds.zip} in the
% TDS tree (also known as \xfile{texmf} tree) of your choice.
% Example (linux):
% \begin{quote}
%   |unzip oberdiek.tds.zip -d ~/texmf|
% \end{quote}
%
% \paragraph{Script installation.}
% Check the directory \xfile{TDS:scripts/oberdiek/} for
% scripts that need further installation steps.
% Package \xpackage{attachfile2} comes with the Perl script
% \xfile{pdfatfi.pl} that should be installed in such a way
% that it can be called as \texttt{pdfatfi}.
% Example (linux):
% \begin{quote}
%   |chmod +x scripts/oberdiek/pdfatfi.pl|\\
%   |cp scripts/oberdiek/pdfatfi.pl /usr/local/bin/|
% \end{quote}
%
% \subsection{Package installation}
%
% \paragraph{Unpacking.} The \xfile{.dtx} file is a self-extracting
% \docstrip\ archive. The files are extracted by running the
% \xfile{.dtx} through \plainTeX:
% \begin{quote}
%   \verb|tex hypdestopt.dtx|
% \end{quote}
%
% \paragraph{TDS.} Now the different files must be moved into
% the different directories in your installation TDS tree
% (also known as \xfile{texmf} tree):
% \begin{quote}
% \def\t{^^A
% \begin{tabular}{@{}>{\ttfamily}l@{ $\rightarrow$ }>{\ttfamily}l@{}}
%   hypdestopt.sty & tex/latex/oberdiek/hypdestopt.sty\\
%   hypdestopt.pdf & doc/latex/oberdiek/hypdestopt.pdf\\
%   hypdestopt.dtx & source/latex/oberdiek/hypdestopt.dtx\\
% \end{tabular}^^A
% }^^A
% \sbox0{\t}^^A
% \ifdim\wd0>\linewidth
%   \begingroup
%     \advance\linewidth by\leftmargin
%     \advance\linewidth by\rightmargin
%   \edef\x{\endgroup
%     \def\noexpand\lw{\the\linewidth}^^A
%   }\x
%   \def\lwbox{^^A
%     \leavevmode
%     \hbox to \linewidth{^^A
%       \kern-\leftmargin\relax
%       \hss
%       \usebox0
%       \hss
%       \kern-\rightmargin\relax
%     }^^A
%   }^^A
%   \ifdim\wd0>\lw
%     \sbox0{\small\t}^^A
%     \ifdim\wd0>\linewidth
%       \ifdim\wd0>\lw
%         \sbox0{\footnotesize\t}^^A
%         \ifdim\wd0>\linewidth
%           \ifdim\wd0>\lw
%             \sbox0{\scriptsize\t}^^A
%             \ifdim\wd0>\linewidth
%               \ifdim\wd0>\lw
%                 \sbox0{\tiny\t}^^A
%                 \ifdim\wd0>\linewidth
%                   \lwbox
%                 \else
%                   \usebox0
%                 \fi
%               \else
%                 \lwbox
%               \fi
%             \else
%               \usebox0
%             \fi
%           \else
%             \lwbox
%           \fi
%         \else
%           \usebox0
%         \fi
%       \else
%         \lwbox
%       \fi
%     \else
%       \usebox0
%     \fi
%   \else
%     \lwbox
%   \fi
% \else
%   \usebox0
% \fi
% \end{quote}
% If you have a \xfile{docstrip.cfg} that configures and enables \docstrip's
% TDS installing feature, then some files can already be in the right
% place, see the documentation of \docstrip.
%
% \subsection{Refresh file name databases}
%
% If your \TeX~distribution
% (\teTeX, \mikTeX, \dots) relies on file name databases, you must refresh
% these. For example, \teTeX\ users run \verb|texhash| or
% \verb|mktexlsr|.
%
% \subsection{Some details for the interested}
%
% \paragraph{Attached source.}
%
% The PDF documentation on CTAN also includes the
% \xfile{.dtx} source file. It can be extracted by
% AcrobatReader 6 or higher. Another option is \textsf{pdftk},
% e.g. unpack the file into the current directory:
% \begin{quote}
%   \verb|pdftk hypdestopt.pdf unpack_files output .|
% \end{quote}
%
% \paragraph{Unpacking with \LaTeX.}
% The \xfile{.dtx} chooses its action depending on the format:
% \begin{description}
% \item[\plainTeX:] Run \docstrip\ and extract the files.
% \item[\LaTeX:] Generate the documentation.
% \end{description}
% If you insist on using \LaTeX\ for \docstrip\ (really,
% \docstrip\ does not need \LaTeX), then inform the autodetect routine
% about your intention:
% \begin{quote}
%   \verb|latex \let\install=y\input{hypdestopt.dtx}|
% \end{quote}
% Do not forget to quote the argument according to the demands
% of your shell.
%
% \paragraph{Generating the documentation.}
% You can use both the \xfile{.dtx} or the \xfile{.drv} to generate
% the documentation. The process can be configured by the
% configuration file \xfile{ltxdoc.cfg}. For instance, put this
% line into this file, if you want to have A4 as paper format:
% \begin{quote}
%   \verb|\PassOptionsToClass{a4paper}{article}|
% \end{quote}
% An example follows how to generate the
% documentation with pdf\LaTeX:
% \begin{quote}
%\begin{verbatim}
%pdflatex hypdestopt.dtx
%makeindex -s gind.ist hypdestopt.idx
%pdflatex hypdestopt.dtx
%makeindex -s gind.ist hypdestopt.idx
%pdflatex hypdestopt.dtx
%\end{verbatim}
% \end{quote}
%
% \section{Catalogue}
%
% The following XML file can be used as source for the
% \href{http://mirror.ctan.org/help/Catalogue/catalogue.html}{\TeX\ Catalogue}.
% The elements \texttt{caption} and \texttt{description} are imported
% from the original XML file from the Catalogue.
% The name of the XML file in the Catalogue is \xfile{hypdestopt.xml}.
%    \begin{macrocode}
%<*catalogue>
<?xml version='1.0' encoding='us-ascii'?>
<!DOCTYPE entry SYSTEM 'catalogue.dtd'>
<entry datestamp='$Date$' modifier='$Author$' id='hypdestopt'>
  <name>hypdestopt</name>
  <caption>Hyperref destination optimizer.</caption>
  <authorref id='auth:oberdiek'/>
  <copyright owner='Heiko Oberdiek' year='2006-2008,2011'/>
  <license type='lppl1.3'/>
  <version number='2.3'/>
  <description>
    This package supports <xref refid='hyperref'>hyperref</xref>'s
    pdftex driver. It removes unnecessary destinations
    and shortens the destination names or uses numbered destinations
    to get smaller PDF files.
    <p/>
    The package is part of the <xref refid='oberdiek'>oberdiek</xref>
    bundle.
  </description>
  <documentation details='Package documentation'
      href='ctan:/macros/latex/contrib/oberdiek/hypdestopt.pdf'/>
  <ctan file='true' path='/macros/latex/contrib/oberdiek/hypdestopt.dtx'/>
  <miktex location='oberdiek'/>
  <texlive location='oberdiek'/>
  <install path='/macros/latex/contrib/oberdiek/oberdiek.tds.zip'/>
</entry>
%</catalogue>
%    \end{macrocode}
%
% \begin{thebibliography}{9}
%
% \bibitem{alphalph}
%   Heiko Oberdiek: \textit{The \xpackage{alphalph} package};
%   2006/05/30 v1.4;
%   \CTAN{macros/latex/contrib/oberdiek/alphalph.pdf}.
%
% \bibitem{hyperref}
%   Sebastian Rahtz, Heiko Oberdiek:
%   \textit{The \xpackage{hyperref} package};
%   2006/06/01 v6.75a;
%   \CTAN{macros/latex/contrib/hyperref/}.
%
% \bibitem{ifpdf}
%   Heiko Oberdiek: \textit{The \xpackage{ifpdf} package};
%   2006/02/20 v1.4;
%   \CTAN{macros/latex/contrib/oberdiek/ifpdf.pdf}.
%
% \end{thebibliography}
%
% \begin{History}
%   \begin{Version}{2006/06/01 v1.0}
%   \item
%     First version.
%   \end{Version}
%   \begin{Version}{2006/06/01 v2.0}
%   \item
%     New method for referencing destinations by number; an idea
%     proposed by Lars Hellstr\"om in the mailing list LATEX-L.
%   \item
%     Options \xoption{name} and \xoption{num} added.
%   \end{Version}
%   \begin{Version}{2007/11/11 v2.1}
%   \item
%     Use of package \xpackage{pdftexcmds} for \LuaTeX\ support.
%   \end{Version}
%   \begin{Version}{2008/08/08 v2.2}
%   \item
%     Support for package \xpackage{bookmark} added.
%   \end{Version}
%   \begin{Version}{2011/05/13 v2.3}
%   \item
%     Fix for \cs{Hy@DestName} if the destination name contains
%     special characters.
%   \item
%     Fix for option \xoption{name} and package \xpackage{bookmark}.
%   \end{Version}
% \end{History}
%
% \PrintIndex
%
% \Finale
\endinput

%        (quote the arguments according to the demands of your shell)
%
% Documentation:
%    (a) If hypdestopt.drv is present:
%           latex hypdestopt.drv
%    (b) Without hypdestopt.drv:
%           latex hypdestopt.dtx; ...
%    The class ltxdoc loads the configuration file ltxdoc.cfg
%    if available. Here you can specify further options, e.g.
%    use A4 as paper format:
%       \PassOptionsToClass{a4paper}{article}
%
%    Programm calls to get the documentation (example):
%       pdflatex hypdestopt.dtx
%       makeindex -s gind.ist hypdestopt.idx
%       pdflatex hypdestopt.dtx
%       makeindex -s gind.ist hypdestopt.idx
%       pdflatex hypdestopt.dtx
%
% Installation:
%    TDS:tex/latex/oberdiek/hypdestopt.sty
%    TDS:doc/latex/oberdiek/hypdestopt.pdf
%    TDS:source/latex/oberdiek/hypdestopt.dtx
%
%<*ignore>
\begingroup
  \catcode123=1 %
  \catcode125=2 %
  \def\x{LaTeX2e}%
\expandafter\endgroup
\ifcase 0\ifx\install y1\fi\expandafter
         \ifx\csname processbatchFile\endcsname\relax\else1\fi
         \ifx\fmtname\x\else 1\fi\relax
\else\csname fi\endcsname
%</ignore>
%<*install>
\input docstrip.tex
\Msg{************************************************************************}
\Msg{* Installation}
\Msg{* Package: hypdestopt 2011/05/13 v2.3 Hyperref destination optimizer (HO)}
\Msg{************************************************************************}

\keepsilent
\askforoverwritefalse

\let\MetaPrefix\relax
\preamble

This is a generated file.

Project: hypdestopt
Version: 2011/05/13 v2.3

Copyright (C) 2006-2008, 2011 by
   Heiko Oberdiek <heiko.oberdiek at googlemail.com>

This work may be distributed and/or modified under the
conditions of the LaTeX Project Public License, either
version 1.3c of this license or (at your option) any later
version. This version of this license is in
   http://www.latex-project.org/lppl/lppl-1-3c.txt
and the latest version of this license is in
   http://www.latex-project.org/lppl.txt
and version 1.3 or later is part of all distributions of
LaTeX version 2005/12/01 or later.

This work has the LPPL maintenance status "maintained".

This Current Maintainer of this work is Heiko Oberdiek.

This work consists of the main source file hypdestopt.dtx
and the derived files
   hypdestopt.sty, hypdestopt.pdf, hypdestopt.ins, hypdestopt.drv.

\endpreamble
\let\MetaPrefix\DoubleperCent

\generate{%
  \file{hypdestopt.ins}{\from{hypdestopt.dtx}{install}}%
  \file{hypdestopt.drv}{\from{hypdestopt.dtx}{driver}}%
  \usedir{tex/latex/oberdiek}%
  \file{hypdestopt.sty}{\from{hypdestopt.dtx}{package}}%
  \nopreamble
  \nopostamble
  \usedir{source/latex/oberdiek/catalogue}%
  \file{hypdestopt.xml}{\from{hypdestopt.dtx}{catalogue}}%
}

\catcode32=13\relax% active space
\let =\space%
\Msg{************************************************************************}
\Msg{*}
\Msg{* To finish the installation you have to move the following}
\Msg{* file into a directory searched by TeX:}
\Msg{*}
\Msg{*     hypdestopt.sty}
\Msg{*}
\Msg{* To produce the documentation run the file `hypdestopt.drv'}
\Msg{* through LaTeX.}
\Msg{*}
\Msg{* Happy TeXing!}
\Msg{*}
\Msg{************************************************************************}

\endbatchfile
%</install>
%<*ignore>
\fi
%</ignore>
%<*driver>
\NeedsTeXFormat{LaTeX2e}
\ProvidesFile{hypdestopt.drv}%
  [2011/05/13 v2.3 Hyperref destination optimizer (HO)]%
\documentclass{ltxdoc}
\usepackage{holtxdoc}[2011/11/22]
\begin{document}
  \DocInput{hypdestopt.dtx}%
\end{document}
%</driver>
% \fi
%
% \CheckSum{565}
%
% \CharacterTable
%  {Upper-case    \A\B\C\D\E\F\G\H\I\J\K\L\M\N\O\P\Q\R\S\T\U\V\W\X\Y\Z
%   Lower-case    \a\b\c\d\e\f\g\h\i\j\k\l\m\n\o\p\q\r\s\t\u\v\w\x\y\z
%   Digits        \0\1\2\3\4\5\6\7\8\9
%   Exclamation   \!     Double quote  \"     Hash (number) \#
%   Dollar        \$     Percent       \%     Ampersand     \&
%   Acute accent  \'     Left paren    \(     Right paren   \)
%   Asterisk      \*     Plus          \+     Comma         \,
%   Minus         \-     Point         \.     Solidus       \/
%   Colon         \:     Semicolon     \;     Less than     \<
%   Equals        \=     Greater than  \>     Question mark \?
%   Commercial at \@     Left bracket  \[     Backslash     \\
%   Right bracket \]     Circumflex    \^     Underscore    \_
%   Grave accent  \`     Left brace    \{     Vertical bar  \|
%   Right brace   \}     Tilde         \~}
%
% \GetFileInfo{hypdestopt.drv}
%
% \title{The \xpackage{hypdestopt} package}
% \date{2011/05/13 v2.3}
% \author{Heiko Oberdiek\\\xemail{heiko.oberdiek at googlemail.com}}
%
% \maketitle
%
% \begin{abstract}
% Package \xpackage{hypdestopt} supports \xpackage{hyperref}'s
% \xoption{pdftex} driver. It removes unnecessary destinations
% and shortens the destination names or uses numbered destinations
% to get smaller PDF files.
% \end{abstract}
%
% \tableofcontents
%
% \section{User interface}
%
% \subsection{Introduction}
%
% Before PDF-1.5 annotations and destinations cannot be compressed.
% If the destination names are not needed for external use, the
% file size can be decreased by the following means:
% \begin{itemize}
% \item Unused destinations are removed.
% \item The destination names are shortened (option \xoption{name}).
% \item Using numbered destinations (option \xoption{num}).
% \end{itemize}
%
% \subsection{Requirements}
%
% \begin{itemize}
% \item Package \xpackage{hyperref} 2006/06/01 v6.75a or newer
%       (\cite{hyperref}).
% \item Package \xpackage{alphalph} 2006/05/30 v1.4 or newer
%       (\cite{alphalph}), if option \xoption{name} is used.
% \item Package \xpackage{ifpdf} (\cite{ifpdf}).
% \item \pdfTeX\ 1.30.0 or newer.
% \item \pdfTeX\ in PDF mode.
% \item \eTeX\ extensions enabled.
% \item Probably an additional compile run of \pdfLaTeX\ is necessary.
% \end{itemize}
%
% In the first compile runs you can get warnings such as:
%\begin{quote}
%\begin{verbatim}
%! pdfTeX warning (dest): name{...} has been referenced ...
%\end{verbatim}
%\end{quote}
% These warnings should vanish in later compile runs.
% However these warnings also can occur without this package.
% The package does not cure them, thus these warnings will remain,
% but the destination name can be different. In such cases test
% without package, too.
%
% \subsection{Use}
%
% If the requirements are met, load the package:
%\begin{quote}
%\verb|\usepackage{hypdestopt}|
%\end{quote}
%
% The following options are supported:
% \begin{description}
% \item[\xoption{verbose}:] Verbose debug output is enabled and written
%   in the protocol file.
% \item[\xoption{num}:] Numbered destinations are used. The file size
%   is smaller, because names are no longer used.
%   This is the default.
% \item[\xoption{name}:] Destinations are identified by names.
% \end{description}
%
% \subsection{Limitations}
%
% \begin{itemize}
% \item Forget this package, if you need preserved destination names.
% \item Destination name strings use all bytes (0..255) except
%       the carriage return (13), left parenthesis (40), right
%       parenthesis (41), and backslash (92), because they
%       must be quoted in general and therefore occupy two bytes
%       instead of one.
%
%       Further the zero byte (0) is avoided for programs
%       that implement strings using zero terminated C strings.
%       And 255 (0xFF) is avoided to get rid of a possible
%       unicode marker at the begin.
%
%       So far I have not seen problems with:
%       \begin{itemize}
%       \item AcrobatReader 5.08/Linux
%       \item AcrobatReader 7.0/Linux
%       \item xpdf 3.00
%       \item Ghostscript 8.50
%       \item gv 3.5.8
%       \item GSview 4.6
%       \end{itemize}
%       But I have not tested all and all possible PDF viewers.
% \item Use of named destinations (\cs{pdfdest}, \cs{pdfoutline},
%       \cs{pdfstartlink}, \dots) that are not supported by this
%       package.
% \item Currently only \xpackage{hyperref} with \pdfTeX\ in PDF
%       mode is supported.
% \end{itemize}
%
% \subsection{Future}
%
% A more general approach is a PDF postprocessor that takes
% a PDF file, performs some transformations and writes the
% result in a more optimized PDF file. Then it does not depend,
% how the original PDF file was generated and further improvements
% are easier to apply. For example, the destination names could be sorted:
% often used destination names would then be shorter than seldom used ones.
%
% \StopEventually{
% }
%
% \section{Implementation}
%
% \subsection{Identification}
%
%    \begin{macrocode}
%<*package>
\NeedsTeXFormat{LaTeX2e}
\ProvidesPackage{hypdestopt}%
  [2011/05/13 v2.3 Hyperref destination optimizer (HO)]%
%    \end{macrocode}
%
% \subsection{Options}
%
% \subsubsection{Option \xoption{verbose}}
%
%    \begin{macrocode}
\newif\ifHypDest@Verbose
\DeclareOption{verbose}{\HypDest@Verbosetrue}
%    \end{macrocode}
%
%    \begin{macro}{\HypDest@VerboseInfo}
%    Wrapper for verbose messages.
%    \begin{macrocode}
\def\HypDest@VerboseInfo#1{%
  \ifHypDest@Verbose
    \PackageInfo{hypdestopt}{#1}%
  \fi
}
%    \end{macrocode}
%    \end{macro}
%
% \subsubsection{Options \xoption{num} and \xoption{name}}
%
%    The options \xoption{num} or \xoption{name} specify
%    the method, how destinations are referenced (by name or
%    number). Default is option \xoption{num}.
%    \begin{macrocode}
\newif\ifHypDest@name
\DeclareOption{num}{\HypDest@namefalse}
\DeclareOption{name}{\HypDest@nametrue}
%    \end{macrocode}
%
%    \begin{macrocode}
\ProcessOptions*\relax
%    \end{macrocode}
%
% \subsection{Check requirements}
%
%    First \pdfTeX\ must running in PDF mode.
%    \begin{macrocode}
\RequirePackage{ifpdf}[2007/09/09]
\RequirePackage{pdftexcmds}[2007/11/11]
\ifpdf
\else
  \PackageError{hypdestopt}{%
    This package requires pdfTeX in PDF mode%
  }\@ehc
  \expandafter\endinput
\fi
%    \end{macrocode}
%    The version of \pdfTeX\ must not be too old, because
%    \cs{pdfescapehex} and \cs{pdfunescapehex} are used.
%    \begin{macrocode}
\begingroup\expandafter\expandafter\expandafter\endgroup
\expandafter\ifx\csname pdf@escapehex\endcsname\relax
  \PackageError{hypdestopt}{%
    This pdfTeX is too old, at least 1.30.0 is required%
  }\@ehc
  \expandafter\endinput
\fi
%    \end{macrocode}
%    Features of \eTeX\ are used, e.g. \cs{numexpr}.
%    \begin{macrocode}
\begingroup\expandafter\expandafter\expandafter\endgroup
\expandafter\ifx\csname numexpr\endcsname\relax
  \PackageError{hypdestopt}{%
    e-TeX features are missing%
  }\@ehc
  \expandafter\endinput
\fi
%    \end{macrocode}
%    Package \xpackage{alphalph} provides \cs{newalphalph} since
%    version 2006/05/30 v1.4.
%    \begin{macrocode}
\ifHypDest@name
  \RequirePackage{alphalph}[2006/05/30]%
\fi
%    \end{macrocode}
%    \begin{macrocode}
\RequirePackage{auxhook}[2009/12/14]
\RequirePackage{pdfescape}[2007/04/21]
%    \end{macrocode}
%
% \subsection{Preamble for auxiliary file}
%
%    Provide dummy definitions for the macros that are used in the
%    auxiliary files. If the package is used no longer, then these
%    commands will not generate errors.
%
%    \begin{macro}{\HypDest@PrependDocument}
%    We add our stuff in front of the \cs{AtBeginDocument} hook
%    to ensure that we are before \xpackage{hyperref}'s stuff.
%    \begin{macrocode}
\long\def\HypDest@PrependDocument#1{%
  \begingroup
    \toks\z@{#1}%
    \toks\tw@\expandafter{\@begindocumenthook}%
    \xdef\@begindocumenthook{\the\toks\z@\the\toks\tw@}%
  \endgroup
}
%    \end{macrocode}
%    \end{macro}
%    \begin{macrocode}
\AddLineBeginAux{%
  \string\providecommand{\string\HypDest@Use}[1]{}%
}
%    \end{macrocode}
%
% \subsection{Generation of destination names}
%
%    Counter |HypDest| is used for identifying destinations.
%    \begin{macrocode}
\newcounter{HypDest}
%    \end{macrocode}
%
%    \begin{macrocode}
\ifHypDest@name
%    \end{macrocode}
%
%    \begin{macro}{\HypDest@HexChar}
%    Destination names are generated by automatically
%    numbering with the help of package \xpackage{alphalph}.
%    \cs{HypDest@HexChar} converts a number of the range 1 until 252
%    into the hexadecimal representation of the string character.
%    \begin{macrocode}
  \def\HypDest@HexChar#1{%
    \ifcase#1\or
%    \end{macrocode}
%    Avoid zero byte because of C strings in PDF viewer
%    applications.
%    \begin{macrocode}
      01\or 02\or 03\or 04\or 05\or 06\or 07\or
%    \end{macrocode}
%    Omit carriage return (13/\verb|^^0d|).
%    It needs quoting, otherwise it would be converted
%    to line feed (10/\verb|^^0a|).
%    \begin{macrocode}
      08\or 09\or 0A\or 0B\or 0C\or 0E\or 0F\or
      10\or 11\or 12\or 13\or 14\or 15\or 16\or 17\or
      18\or 19\or 1A\or 1B\or 1C\or 1D\or 1E\or 1F\or
      20\or 21\or 22\or 23\or 24\or 25\or 26\or 27\or
%    \end{macrocode}
%    Omit left and right parentheses (40/\verb|^^28|, 41/\verb|^^39|),
%    they need quoting in general.
%    \begin{macrocode}
      2A\or 2B\or 2C\or 2D\or 2E\or 2F\or
      30\or 31\or 32\or 33\or 34\or 35\or 36\or 37\or
      38\or 39\or 3A\or 3B\or 3C\or 3D\or 3E\or 3F\or
      40\or 41\or 42\or 43\or 44\or 45\or 46\or 47\or
      48\or 49\or 4A\or 4B\or 4C\or 4D\or 4E\or 4F\or
      50\or 51\or 52\or 53\or 54\or 55\or 56\or 57\or
%    \end{macrocode}
%    Omit backslash (92/\verb|^^5C|), it needs quoting.
%    \begin{macrocode}
      58\or 59\or 5A\or 5B\or 5D\or 5E\or 5F\or
      60\or 61\or 62\or 63\or 64\or 65\or 66\or 67\or
      68\or 69\or 6A\or 6B\or 6C\or 6D\or 6E\or 6F\or
      70\or 71\or 72\or 73\or 74\or 75\or 76\or 77\or
      78\or 79\or 7A\or 7B\or 7C\or 7D\or 7E\or 7F\or
      80\or 81\or 82\or 83\or 84\or 85\or 86\or 87\or
      88\or 89\or 8A\or 8B\or 8C\or 8D\or 8E\or 8F\or
      90\or 91\or 92\or 93\or 94\or 95\or 96\or 97\or
      98\or 99\or 9A\or 9B\or 9C\or 9D\or 9E\or 9F\or
      A0\or A1\or A2\or A3\or A4\or A5\or A6\or A7\or
      A8\or A9\or AA\or AB\or AC\or AD\or AE\or AF\or
      B0\or B1\or B2\or B3\or B4\or B5\or B6\or B7\or
      B8\or B9\or BA\or BB\or BC\or BD\or BE\or BF\or
      C0\or C1\or C2\or C3\or C4\or C5\or C6\or C7\or
      C8\or C9\or CA\or CB\or CC\or CD\or CE\or CF\or
      D0\or D1\or D2\or D3\or D4\or D5\or D6\or D7\or
      D8\or D9\or DA\or DB\or DC\or DD\or DE\or DF\or
      E0\or E1\or E2\or E3\or E4\or E5\or E6\or E7\or
      E8\or E9\or EA\or EB\or EC\or ED\or EE\or EF\or
      F0\or F1\or F2\or F3\or F4\or F5\or F6\or F7\or
%    \end{macrocode}
%    Avoid 255 (0xFF) to get rid of a possible unicode
%    marker at the begin of the string.
%    \begin{macrocode}
      F8\or F9\or FA\or FB\or FC\or FD\or FE%
    \fi
  }%
%    \end{macrocode}
%    \end{macro}
%    \begin{macro}{HypDest@HexString}
%    Now package \xpackage{alphalph} comes into play.
%    \cs{HypDest@HexString} is defined and converts
%    a positive number into a string, given in hexadecimal
%    representation.
%    \begin{macrocode}
  \newalphalph\HypDest@HexString\HypDest@HexChar{250}%
%    \end{macrocode}
%    \end{macro}
%    \begin{macro}{\theHypDest}
%    For use, the hexadecimal string is converted back.
%    \begin{macrocode}
  \renewcommand*{\theHypDest}{%
    \pdf@unescapehex{\HypDest@HexString{\value{HypDest}}}%
  }%
%    \end{macrocode}
%    \end{macro}
%
%    With option \xoption{num} we use the number directly.
%    \begin{macrocode}
\else
  \renewcommand*{\theHypDest}{%
    \number\value{HypDest}%
  }%
\fi
%    \end{macrocode}
%
% \subsection{Assign destination names}
%
%    \begin{macro}{\HypDest@Prefix}
%    The new destination names are remembered in macros whose names
%    start with prefix \cs{HypDest@Prefix}.
%    \begin{macrocode}
\edef\HypDest@Prefix{HypDest\string:}
%    \end{macrocode}
%    \end{macro}
%
%    \begin{macro}{\HypDest@Use}
%    During the first read of the auxiliary files, the used destinations
%    get fresh generated short destination names. Also for the old
%    destination names we use the hexadecimal representation. That
%    avoid problems with arbitrary names.
%    \begin{macrocode}
\def\HypDest@Use#1{%
  \begingroup
    \edef\x{%
      \expandafter\noexpand
      \csname\HypDest@Prefix\pdf@unescapehex{#1}\endcsname
    }%
    \expandafter\ifx\x\relax
      \stepcounter{HypDest}%
      \expandafter\xdef\x{\theHypDest}%
      \let\on@line\@empty
      \ifHypDest@name
        \HypDest@VerboseInfo{%
          Use: (\pdf@unescapehex{#1}) -\string> %
          0x\pdf@escapehex{\x} (\number\value{HypDest})%
        }%
      \else
        \HypDest@VerboseInfo{%
          Use: (\pdf@unescapehex{#1}) -\string> num \x
        }%
      \fi
    \fi
  \endgroup
}
%    \end{macrocode}
%    \end{macro}
%
%    After the first \xfile{.aux} file processing the destination names
%    are assigned and we can disable \cs{HypDest@Use}.
%    \begin{macrocode}
\AtBeginDocument{%
  \let\HypDest@Use\@gobble
}
%    \end{macrocode}
%
%    \begin{macro}{\HypDest@MarkUsed}
%    Destinations that are actually used are marked by \cs{HypDest@MarkUsed}.
%    \cs{nofiles} is respected.
%    \begin{macrocode}
\def\HypDest@MarkUsed#1{%
  \HypDest@VerboseInfo{%
    MarkUsed: (#1)%
  }%
  \if@filesw
    \immediate\write\@auxout{%
      \string\HypDest@Use{\pdf@escapehex{#1}}%
    }%
  \fi
}%
%    \end{macrocode}
%    \end{macro}
%
% \subsection{Redefinition of \xpackage{hyperref}'s hooks}
%
%    Package \xpackage{hyperref} can be loaded later, therefore
%    we redefine \xpackage{hyperref}'s macros at |\begin{document}|.
%    \begin{macrocode}
\HypDest@PrependDocument{%
%    \end{macrocode}
%
%    Check hyperref version.
%    \begin{macrocode}
  \@ifpackagelater{hyperref}{2006/06/01}{}{%
    \PackageError{hypdestopt}{%
      hyperref 2006/06/01 v6.75a or later is required%
    }\@ehc
  }%
%    \end{macrocode}
%
% \subsubsection{Destination setting}
%
%    \begin{macrocode}
  \ifHypDest@name
    \let\HypDest@Org@DestName\Hy@DestName
    \renewcommand*{\Hy@DestName}[2]{%
      \EdefUnescapeString\HypDest@temp{#1}%
      \@ifundefined{\HypDest@Prefix\HypDest@temp}{%
        \HypDest@VerboseInfo{%
          DestName: (\HypDest@temp) unused%
        }%
      }{%
        \HypDest@Org@DestName{%
          \csname\HypDest@Prefix\HypDest@temp\endcsname
        }{#2}%
        \HypDest@VerboseInfo{%
          DestName: (\HypDest@temp) %
          0x\pdf@escapehex{%
            \csname\HypDest@Prefix\HypDest@temp\endcsname
          }%
        }%
      }%
    }%
  \else
    \renewcommand*{\Hy@DestName}[2]{%
      \EdefUnescapeString\HypDest@temp{#1}%
      \@ifundefined{\HypDest@Prefix\HypDest@temp}{%
        \HypDest@VerboseInfo{%
          DestName: (\HypDest@temp) unused%
        }%
      }{%
        \pdfdest num%
        \csname\HypDest@Prefix\HypDest@temp\endcsname#2\relax
        \HypDest@VerboseInfo{%
          DestName: (\HypDest@temp) %
          num \csname\HypDest@Prefix\HypDest@temp\endcsname
        }%
      }%
    }%
  \fi
%    \end{macrocode}
%
% \subsubsection{Links}
%
%    \begin{macrocode}
  \let\HypDest@Org@StartlinkName\Hy@StartlinkName
  \ifHypDest@name
    \renewcommand*{\Hy@StartlinkName}[2]{%
      \HypDest@MarkUsed{#2}%
      \HypDest@Org@StartlinkName{#1}{%
        \@ifundefined{\HypDest@Prefix#2}{%
          #2%
        }{%
          \csname\HypDest@Prefix#2\endcsname
        }%
      }%
    }%
  \else
    \renewcommand*{\Hy@StartlinkName}[2]{%
      \HypDest@MarkUsed{#2}%
      \@ifundefined{\HypDest@Prefix#2}{%
        \HypDest@Org@StartlinkName{#1}{#2}%
      }{%
        \pdfstartlink attr{#1}%
                      goto num\csname\HypDest@Prefix#2\endcsname
        \relax
      }%
    }%
  \fi
%    \end{macrocode}
%
% \subsubsection{Outlines of package \xpackage{hyperref}}
%
%    \begin{macrocode}
  \let\HypDest@Org@OutlineName\Hy@OutlineName
  \ifHypDest@name
    \renewcommand*{\Hy@OutlineName}[4]{%
      \HypDest@Org@OutlineName{#1}{%
        \@ifundefined{\HypDest@Prefix#2}{%
          #2%
        }{%
          \csname\HypDest@Prefix#2\endcsname
        }%
      }{#3}{#4}%
    }%
  \else
    \renewcommand*{\Hy@OutlineName}[4]{%
      \@ifundefined{\HypDest@Prefix#2}{%
        \HypDest@Org@OutlineName{#1}{#2}{#3}{#4}%
      }{%
        \pdfoutline goto num\csname\HypDest@Prefix#2\endcsname
                    count#3{#4}%
      }%
    }%
  \fi
%    \end{macrocode}
%    Because \cs{Hy@OutlineName} is called after the \xfile{.out} file
%    is written in the previous run. Therefore we mark the destination
%    earlier in \cs{@@writetorep}.
%    \begin{macrocode}
  \let\HypDest@Org@@writetorep\@@writetorep
  \renewcommand*{\@@writetorep}[5]{%
    \begingroup
      \edef\Hy@tempa{#5}%
      \ifx\Hy@tempa\Hy@bookmarkstype
        \HypDest@MarkUsed{#3}%
      \fi
    \endgroup
    \HypDest@Org@@writetorep{#1}{#2}{#3}{#4}{#5}%
  }%
%    \end{macrocode}
%
% \subsubsection{Outlines of package \xpackage{bookmark}}
%
%    \begin{macrocode}
  \@ifpackageloaded{bookmark}{%
    \@ifpackagelater{bookmark}{2008/08/08}{%
      \renewcommand*{\BKM@DefGotoNameAction}[2]{%
        \@ifundefined{\HypDest@Prefix#2}{%
          \edef#1{goto name{hypdestopt\string :unknown}}%
        }{%
          \ifHypDest@name
            \edef#1{goto name{\csname\HypDest@Prefix#2\endcsname}}%
          \else
            \edef#1{goto num\csname\HypDest@Prefix#2\endcsname}%
          \fi
        }%
      }%
      \def\BKM@HypDestOptHook{%
        \ifx\BKM@dest\@empty
        \else
          \ifx\BKM@gotor\@empty
            \HypDest@MarkUsed\BKM@dest
          \fi
        \fi
      }%
    }{%
      \@PackageError{hypdestopt}{%
        Package `bookmark' is too old.\MessageBreak
        Version 2008/08/08 or later is needed%
      }\@ehc
    }%
  }{}%
%    \end{macrocode}
%
%    \begin{macrocode}
}
%    \end{macrocode}
%
%
%    \begin{macrocode}
%</package>
%    \end{macrocode}
%
% \section{Installation}
%
% \subsection{Download}
%
% \paragraph{Package.} This package is available on
% CTAN\footnote{\url{ftp://ftp.ctan.org/tex-archive/}}:
% \begin{description}
% \item[\CTAN{macros/latex/contrib/oberdiek/hypdestopt.dtx}] The source file.
% \item[\CTAN{macros/latex/contrib/oberdiek/hypdestopt.pdf}] Documentation.
% \end{description}
%
%
% \paragraph{Bundle.} All the packages of the bundle `oberdiek'
% are also available in a TDS compliant ZIP archive. There
% the packages are already unpacked and the documentation files
% are generated. The files and directories obey the TDS standard.
% \begin{description}
% \item[\CTAN{install/macros/latex/contrib/oberdiek.tds.zip}]
% \end{description}
% \emph{TDS} refers to the standard ``A Directory Structure
% for \TeX\ Files'' (\CTAN{tds/tds.pdf}). Directories
% with \xfile{texmf} in their name are usually organized this way.
%
% \subsection{Bundle installation}
%
% \paragraph{Unpacking.} Unpack the \xfile{oberdiek.tds.zip} in the
% TDS tree (also known as \xfile{texmf} tree) of your choice.
% Example (linux):
% \begin{quote}
%   |unzip oberdiek.tds.zip -d ~/texmf|
% \end{quote}
%
% \paragraph{Script installation.}
% Check the directory \xfile{TDS:scripts/oberdiek/} for
% scripts that need further installation steps.
% Package \xpackage{attachfile2} comes with the Perl script
% \xfile{pdfatfi.pl} that should be installed in such a way
% that it can be called as \texttt{pdfatfi}.
% Example (linux):
% \begin{quote}
%   |chmod +x scripts/oberdiek/pdfatfi.pl|\\
%   |cp scripts/oberdiek/pdfatfi.pl /usr/local/bin/|
% \end{quote}
%
% \subsection{Package installation}
%
% \paragraph{Unpacking.} The \xfile{.dtx} file is a self-extracting
% \docstrip\ archive. The files are extracted by running the
% \xfile{.dtx} through \plainTeX:
% \begin{quote}
%   \verb|tex hypdestopt.dtx|
% \end{quote}
%
% \paragraph{TDS.} Now the different files must be moved into
% the different directories in your installation TDS tree
% (also known as \xfile{texmf} tree):
% \begin{quote}
% \def\t{^^A
% \begin{tabular}{@{}>{\ttfamily}l@{ $\rightarrow$ }>{\ttfamily}l@{}}
%   hypdestopt.sty & tex/latex/oberdiek/hypdestopt.sty\\
%   hypdestopt.pdf & doc/latex/oberdiek/hypdestopt.pdf\\
%   hypdestopt.dtx & source/latex/oberdiek/hypdestopt.dtx\\
% \end{tabular}^^A
% }^^A
% \sbox0{\t}^^A
% \ifdim\wd0>\linewidth
%   \begingroup
%     \advance\linewidth by\leftmargin
%     \advance\linewidth by\rightmargin
%   \edef\x{\endgroup
%     \def\noexpand\lw{\the\linewidth}^^A
%   }\x
%   \def\lwbox{^^A
%     \leavevmode
%     \hbox to \linewidth{^^A
%       \kern-\leftmargin\relax
%       \hss
%       \usebox0
%       \hss
%       \kern-\rightmargin\relax
%     }^^A
%   }^^A
%   \ifdim\wd0>\lw
%     \sbox0{\small\t}^^A
%     \ifdim\wd0>\linewidth
%       \ifdim\wd0>\lw
%         \sbox0{\footnotesize\t}^^A
%         \ifdim\wd0>\linewidth
%           \ifdim\wd0>\lw
%             \sbox0{\scriptsize\t}^^A
%             \ifdim\wd0>\linewidth
%               \ifdim\wd0>\lw
%                 \sbox0{\tiny\t}^^A
%                 \ifdim\wd0>\linewidth
%                   \lwbox
%                 \else
%                   \usebox0
%                 \fi
%               \else
%                 \lwbox
%               \fi
%             \else
%               \usebox0
%             \fi
%           \else
%             \lwbox
%           \fi
%         \else
%           \usebox0
%         \fi
%       \else
%         \lwbox
%       \fi
%     \else
%       \usebox0
%     \fi
%   \else
%     \lwbox
%   \fi
% \else
%   \usebox0
% \fi
% \end{quote}
% If you have a \xfile{docstrip.cfg} that configures and enables \docstrip's
% TDS installing feature, then some files can already be in the right
% place, see the documentation of \docstrip.
%
% \subsection{Refresh file name databases}
%
% If your \TeX~distribution
% (\teTeX, \mikTeX, \dots) relies on file name databases, you must refresh
% these. For example, \teTeX\ users run \verb|texhash| or
% \verb|mktexlsr|.
%
% \subsection{Some details for the interested}
%
% \paragraph{Attached source.}
%
% The PDF documentation on CTAN also includes the
% \xfile{.dtx} source file. It can be extracted by
% AcrobatReader 6 or higher. Another option is \textsf{pdftk},
% e.g. unpack the file into the current directory:
% \begin{quote}
%   \verb|pdftk hypdestopt.pdf unpack_files output .|
% \end{quote}
%
% \paragraph{Unpacking with \LaTeX.}
% The \xfile{.dtx} chooses its action depending on the format:
% \begin{description}
% \item[\plainTeX:] Run \docstrip\ and extract the files.
% \item[\LaTeX:] Generate the documentation.
% \end{description}
% If you insist on using \LaTeX\ for \docstrip\ (really,
% \docstrip\ does not need \LaTeX), then inform the autodetect routine
% about your intention:
% \begin{quote}
%   \verb|latex \let\install=y% \iffalse meta-comment
%
% File: hypdestopt.dtx
% Version: 2011/05/13 v2.3
% Info: Hyperref destination optimizer
%
% Copyright (C) 2006-2008, 2011 by
%    Heiko Oberdiek <heiko.oberdiek at googlemail.com>
%
% This work may be distributed and/or modified under the
% conditions of the LaTeX Project Public License, either
% version 1.3c of this license or (at your option) any later
% version. This version of this license is in
%    http://www.latex-project.org/lppl/lppl-1-3c.txt
% and the latest version of this license is in
%    http://www.latex-project.org/lppl.txt
% and version 1.3 or later is part of all distributions of
% LaTeX version 2005/12/01 or later.
%
% This work has the LPPL maintenance status "maintained".
%
% This Current Maintainer of this work is Heiko Oberdiek.
%
% This work consists of the main source file hypdestopt.dtx
% and the derived files
%    hypdestopt.sty, hypdestopt.pdf, hypdestopt.ins, hypdestopt.drv.
%
% Distribution:
%    CTAN:macros/latex/contrib/oberdiek/hypdestopt.dtx
%    CTAN:macros/latex/contrib/oberdiek/hypdestopt.pdf
%
% Unpacking:
%    (a) If hypdestopt.ins is present:
%           tex hypdestopt.ins
%    (b) Without hypdestopt.ins:
%           tex hypdestopt.dtx
%    (c) If you insist on using LaTeX
%           latex \let\install=y\input{hypdestopt.dtx}
%        (quote the arguments according to the demands of your shell)
%
% Documentation:
%    (a) If hypdestopt.drv is present:
%           latex hypdestopt.drv
%    (b) Without hypdestopt.drv:
%           latex hypdestopt.dtx; ...
%    The class ltxdoc loads the configuration file ltxdoc.cfg
%    if available. Here you can specify further options, e.g.
%    use A4 as paper format:
%       \PassOptionsToClass{a4paper}{article}
%
%    Programm calls to get the documentation (example):
%       pdflatex hypdestopt.dtx
%       makeindex -s gind.ist hypdestopt.idx
%       pdflatex hypdestopt.dtx
%       makeindex -s gind.ist hypdestopt.idx
%       pdflatex hypdestopt.dtx
%
% Installation:
%    TDS:tex/latex/oberdiek/hypdestopt.sty
%    TDS:doc/latex/oberdiek/hypdestopt.pdf
%    TDS:source/latex/oberdiek/hypdestopt.dtx
%
%<*ignore>
\begingroup
  \catcode123=1 %
  \catcode125=2 %
  \def\x{LaTeX2e}%
\expandafter\endgroup
\ifcase 0\ifx\install y1\fi\expandafter
         \ifx\csname processbatchFile\endcsname\relax\else1\fi
         \ifx\fmtname\x\else 1\fi\relax
\else\csname fi\endcsname
%</ignore>
%<*install>
\input docstrip.tex
\Msg{************************************************************************}
\Msg{* Installation}
\Msg{* Package: hypdestopt 2011/05/13 v2.3 Hyperref destination optimizer (HO)}
\Msg{************************************************************************}

\keepsilent
\askforoverwritefalse

\let\MetaPrefix\relax
\preamble

This is a generated file.

Project: hypdestopt
Version: 2011/05/13 v2.3

Copyright (C) 2006-2008, 2011 by
   Heiko Oberdiek <heiko.oberdiek at googlemail.com>

This work may be distributed and/or modified under the
conditions of the LaTeX Project Public License, either
version 1.3c of this license or (at your option) any later
version. This version of this license is in
   http://www.latex-project.org/lppl/lppl-1-3c.txt
and the latest version of this license is in
   http://www.latex-project.org/lppl.txt
and version 1.3 or later is part of all distributions of
LaTeX version 2005/12/01 or later.

This work has the LPPL maintenance status "maintained".

This Current Maintainer of this work is Heiko Oberdiek.

This work consists of the main source file hypdestopt.dtx
and the derived files
   hypdestopt.sty, hypdestopt.pdf, hypdestopt.ins, hypdestopt.drv.

\endpreamble
\let\MetaPrefix\DoubleperCent

\generate{%
  \file{hypdestopt.ins}{\from{hypdestopt.dtx}{install}}%
  \file{hypdestopt.drv}{\from{hypdestopt.dtx}{driver}}%
  \usedir{tex/latex/oberdiek}%
  \file{hypdestopt.sty}{\from{hypdestopt.dtx}{package}}%
  \nopreamble
  \nopostamble
  \usedir{source/latex/oberdiek/catalogue}%
  \file{hypdestopt.xml}{\from{hypdestopt.dtx}{catalogue}}%
}

\catcode32=13\relax% active space
\let =\space%
\Msg{************************************************************************}
\Msg{*}
\Msg{* To finish the installation you have to move the following}
\Msg{* file into a directory searched by TeX:}
\Msg{*}
\Msg{*     hypdestopt.sty}
\Msg{*}
\Msg{* To produce the documentation run the file `hypdestopt.drv'}
\Msg{* through LaTeX.}
\Msg{*}
\Msg{* Happy TeXing!}
\Msg{*}
\Msg{************************************************************************}

\endbatchfile
%</install>
%<*ignore>
\fi
%</ignore>
%<*driver>
\NeedsTeXFormat{LaTeX2e}
\ProvidesFile{hypdestopt.drv}%
  [2011/05/13 v2.3 Hyperref destination optimizer (HO)]%
\documentclass{ltxdoc}
\usepackage{holtxdoc}[2011/11/22]
\begin{document}
  \DocInput{hypdestopt.dtx}%
\end{document}
%</driver>
% \fi
%
% \CheckSum{565}
%
% \CharacterTable
%  {Upper-case    \A\B\C\D\E\F\G\H\I\J\K\L\M\N\O\P\Q\R\S\T\U\V\W\X\Y\Z
%   Lower-case    \a\b\c\d\e\f\g\h\i\j\k\l\m\n\o\p\q\r\s\t\u\v\w\x\y\z
%   Digits        \0\1\2\3\4\5\6\7\8\9
%   Exclamation   \!     Double quote  \"     Hash (number) \#
%   Dollar        \$     Percent       \%     Ampersand     \&
%   Acute accent  \'     Left paren    \(     Right paren   \)
%   Asterisk      \*     Plus          \+     Comma         \,
%   Minus         \-     Point         \.     Solidus       \/
%   Colon         \:     Semicolon     \;     Less than     \<
%   Equals        \=     Greater than  \>     Question mark \?
%   Commercial at \@     Left bracket  \[     Backslash     \\
%   Right bracket \]     Circumflex    \^     Underscore    \_
%   Grave accent  \`     Left brace    \{     Vertical bar  \|
%   Right brace   \}     Tilde         \~}
%
% \GetFileInfo{hypdestopt.drv}
%
% \title{The \xpackage{hypdestopt} package}
% \date{2011/05/13 v2.3}
% \author{Heiko Oberdiek\\\xemail{heiko.oberdiek at googlemail.com}}
%
% \maketitle
%
% \begin{abstract}
% Package \xpackage{hypdestopt} supports \xpackage{hyperref}'s
% \xoption{pdftex} driver. It removes unnecessary destinations
% and shortens the destination names or uses numbered destinations
% to get smaller PDF files.
% \end{abstract}
%
% \tableofcontents
%
% \section{User interface}
%
% \subsection{Introduction}
%
% Before PDF-1.5 annotations and destinations cannot be compressed.
% If the destination names are not needed for external use, the
% file size can be decreased by the following means:
% \begin{itemize}
% \item Unused destinations are removed.
% \item The destination names are shortened (option \xoption{name}).
% \item Using numbered destinations (option \xoption{num}).
% \end{itemize}
%
% \subsection{Requirements}
%
% \begin{itemize}
% \item Package \xpackage{hyperref} 2006/06/01 v6.75a or newer
%       (\cite{hyperref}).
% \item Package \xpackage{alphalph} 2006/05/30 v1.4 or newer
%       (\cite{alphalph}), if option \xoption{name} is used.
% \item Package \xpackage{ifpdf} (\cite{ifpdf}).
% \item \pdfTeX\ 1.30.0 or newer.
% \item \pdfTeX\ in PDF mode.
% \item \eTeX\ extensions enabled.
% \item Probably an additional compile run of \pdfLaTeX\ is necessary.
% \end{itemize}
%
% In the first compile runs you can get warnings such as:
%\begin{quote}
%\begin{verbatim}
%! pdfTeX warning (dest): name{...} has been referenced ...
%\end{verbatim}
%\end{quote}
% These warnings should vanish in later compile runs.
% However these warnings also can occur without this package.
% The package does not cure them, thus these warnings will remain,
% but the destination name can be different. In such cases test
% without package, too.
%
% \subsection{Use}
%
% If the requirements are met, load the package:
%\begin{quote}
%\verb|\usepackage{hypdestopt}|
%\end{quote}
%
% The following options are supported:
% \begin{description}
% \item[\xoption{verbose}:] Verbose debug output is enabled and written
%   in the protocol file.
% \item[\xoption{num}:] Numbered destinations are used. The file size
%   is smaller, because names are no longer used.
%   This is the default.
% \item[\xoption{name}:] Destinations are identified by names.
% \end{description}
%
% \subsection{Limitations}
%
% \begin{itemize}
% \item Forget this package, if you need preserved destination names.
% \item Destination name strings use all bytes (0..255) except
%       the carriage return (13), left parenthesis (40), right
%       parenthesis (41), and backslash (92), because they
%       must be quoted in general and therefore occupy two bytes
%       instead of one.
%
%       Further the zero byte (0) is avoided for programs
%       that implement strings using zero terminated C strings.
%       And 255 (0xFF) is avoided to get rid of a possible
%       unicode marker at the begin.
%
%       So far I have not seen problems with:
%       \begin{itemize}
%       \item AcrobatReader 5.08/Linux
%       \item AcrobatReader 7.0/Linux
%       \item xpdf 3.00
%       \item Ghostscript 8.50
%       \item gv 3.5.8
%       \item GSview 4.6
%       \end{itemize}
%       But I have not tested all and all possible PDF viewers.
% \item Use of named destinations (\cs{pdfdest}, \cs{pdfoutline},
%       \cs{pdfstartlink}, \dots) that are not supported by this
%       package.
% \item Currently only \xpackage{hyperref} with \pdfTeX\ in PDF
%       mode is supported.
% \end{itemize}
%
% \subsection{Future}
%
% A more general approach is a PDF postprocessor that takes
% a PDF file, performs some transformations and writes the
% result in a more optimized PDF file. Then it does not depend,
% how the original PDF file was generated and further improvements
% are easier to apply. For example, the destination names could be sorted:
% often used destination names would then be shorter than seldom used ones.
%
% \StopEventually{
% }
%
% \section{Implementation}
%
% \subsection{Identification}
%
%    \begin{macrocode}
%<*package>
\NeedsTeXFormat{LaTeX2e}
\ProvidesPackage{hypdestopt}%
  [2011/05/13 v2.3 Hyperref destination optimizer (HO)]%
%    \end{macrocode}
%
% \subsection{Options}
%
% \subsubsection{Option \xoption{verbose}}
%
%    \begin{macrocode}
\newif\ifHypDest@Verbose
\DeclareOption{verbose}{\HypDest@Verbosetrue}
%    \end{macrocode}
%
%    \begin{macro}{\HypDest@VerboseInfo}
%    Wrapper for verbose messages.
%    \begin{macrocode}
\def\HypDest@VerboseInfo#1{%
  \ifHypDest@Verbose
    \PackageInfo{hypdestopt}{#1}%
  \fi
}
%    \end{macrocode}
%    \end{macro}
%
% \subsubsection{Options \xoption{num} and \xoption{name}}
%
%    The options \xoption{num} or \xoption{name} specify
%    the method, how destinations are referenced (by name or
%    number). Default is option \xoption{num}.
%    \begin{macrocode}
\newif\ifHypDest@name
\DeclareOption{num}{\HypDest@namefalse}
\DeclareOption{name}{\HypDest@nametrue}
%    \end{macrocode}
%
%    \begin{macrocode}
\ProcessOptions*\relax
%    \end{macrocode}
%
% \subsection{Check requirements}
%
%    First \pdfTeX\ must running in PDF mode.
%    \begin{macrocode}
\RequirePackage{ifpdf}[2007/09/09]
\RequirePackage{pdftexcmds}[2007/11/11]
\ifpdf
\else
  \PackageError{hypdestopt}{%
    This package requires pdfTeX in PDF mode%
  }\@ehc
  \expandafter\endinput
\fi
%    \end{macrocode}
%    The version of \pdfTeX\ must not be too old, because
%    \cs{pdfescapehex} and \cs{pdfunescapehex} are used.
%    \begin{macrocode}
\begingroup\expandafter\expandafter\expandafter\endgroup
\expandafter\ifx\csname pdf@escapehex\endcsname\relax
  \PackageError{hypdestopt}{%
    This pdfTeX is too old, at least 1.30.0 is required%
  }\@ehc
  \expandafter\endinput
\fi
%    \end{macrocode}
%    Features of \eTeX\ are used, e.g. \cs{numexpr}.
%    \begin{macrocode}
\begingroup\expandafter\expandafter\expandafter\endgroup
\expandafter\ifx\csname numexpr\endcsname\relax
  \PackageError{hypdestopt}{%
    e-TeX features are missing%
  }\@ehc
  \expandafter\endinput
\fi
%    \end{macrocode}
%    Package \xpackage{alphalph} provides \cs{newalphalph} since
%    version 2006/05/30 v1.4.
%    \begin{macrocode}
\ifHypDest@name
  \RequirePackage{alphalph}[2006/05/30]%
\fi
%    \end{macrocode}
%    \begin{macrocode}
\RequirePackage{auxhook}[2009/12/14]
\RequirePackage{pdfescape}[2007/04/21]
%    \end{macrocode}
%
% \subsection{Preamble for auxiliary file}
%
%    Provide dummy definitions for the macros that are used in the
%    auxiliary files. If the package is used no longer, then these
%    commands will not generate errors.
%
%    \begin{macro}{\HypDest@PrependDocument}
%    We add our stuff in front of the \cs{AtBeginDocument} hook
%    to ensure that we are before \xpackage{hyperref}'s stuff.
%    \begin{macrocode}
\long\def\HypDest@PrependDocument#1{%
  \begingroup
    \toks\z@{#1}%
    \toks\tw@\expandafter{\@begindocumenthook}%
    \xdef\@begindocumenthook{\the\toks\z@\the\toks\tw@}%
  \endgroup
}
%    \end{macrocode}
%    \end{macro}
%    \begin{macrocode}
\AddLineBeginAux{%
  \string\providecommand{\string\HypDest@Use}[1]{}%
}
%    \end{macrocode}
%
% \subsection{Generation of destination names}
%
%    Counter |HypDest| is used for identifying destinations.
%    \begin{macrocode}
\newcounter{HypDest}
%    \end{macrocode}
%
%    \begin{macrocode}
\ifHypDest@name
%    \end{macrocode}
%
%    \begin{macro}{\HypDest@HexChar}
%    Destination names are generated by automatically
%    numbering with the help of package \xpackage{alphalph}.
%    \cs{HypDest@HexChar} converts a number of the range 1 until 252
%    into the hexadecimal representation of the string character.
%    \begin{macrocode}
  \def\HypDest@HexChar#1{%
    \ifcase#1\or
%    \end{macrocode}
%    Avoid zero byte because of C strings in PDF viewer
%    applications.
%    \begin{macrocode}
      01\or 02\or 03\or 04\or 05\or 06\or 07\or
%    \end{macrocode}
%    Omit carriage return (13/\verb|^^0d|).
%    It needs quoting, otherwise it would be converted
%    to line feed (10/\verb|^^0a|).
%    \begin{macrocode}
      08\or 09\or 0A\or 0B\or 0C\or 0E\or 0F\or
      10\or 11\or 12\or 13\or 14\or 15\or 16\or 17\or
      18\or 19\or 1A\or 1B\or 1C\or 1D\or 1E\or 1F\or
      20\or 21\or 22\or 23\or 24\or 25\or 26\or 27\or
%    \end{macrocode}
%    Omit left and right parentheses (40/\verb|^^28|, 41/\verb|^^39|),
%    they need quoting in general.
%    \begin{macrocode}
      2A\or 2B\or 2C\or 2D\or 2E\or 2F\or
      30\or 31\or 32\or 33\or 34\or 35\or 36\or 37\or
      38\or 39\or 3A\or 3B\or 3C\or 3D\or 3E\or 3F\or
      40\or 41\or 42\or 43\or 44\or 45\or 46\or 47\or
      48\or 49\or 4A\or 4B\or 4C\or 4D\or 4E\or 4F\or
      50\or 51\or 52\or 53\or 54\or 55\or 56\or 57\or
%    \end{macrocode}
%    Omit backslash (92/\verb|^^5C|), it needs quoting.
%    \begin{macrocode}
      58\or 59\or 5A\or 5B\or 5D\or 5E\or 5F\or
      60\or 61\or 62\or 63\or 64\or 65\or 66\or 67\or
      68\or 69\or 6A\or 6B\or 6C\or 6D\or 6E\or 6F\or
      70\or 71\or 72\or 73\or 74\or 75\or 76\or 77\or
      78\or 79\or 7A\or 7B\or 7C\or 7D\or 7E\or 7F\or
      80\or 81\or 82\or 83\or 84\or 85\or 86\or 87\or
      88\or 89\or 8A\or 8B\or 8C\or 8D\or 8E\or 8F\or
      90\or 91\or 92\or 93\or 94\or 95\or 96\or 97\or
      98\or 99\or 9A\or 9B\or 9C\or 9D\or 9E\or 9F\or
      A0\or A1\or A2\or A3\or A4\or A5\or A6\or A7\or
      A8\or A9\or AA\or AB\or AC\or AD\or AE\or AF\or
      B0\or B1\or B2\or B3\or B4\or B5\or B6\or B7\or
      B8\or B9\or BA\or BB\or BC\or BD\or BE\or BF\or
      C0\or C1\or C2\or C3\or C4\or C5\or C6\or C7\or
      C8\or C9\or CA\or CB\or CC\or CD\or CE\or CF\or
      D0\or D1\or D2\or D3\or D4\or D5\or D6\or D7\or
      D8\or D9\or DA\or DB\or DC\or DD\or DE\or DF\or
      E0\or E1\or E2\or E3\or E4\or E5\or E6\or E7\or
      E8\or E9\or EA\or EB\or EC\or ED\or EE\or EF\or
      F0\or F1\or F2\or F3\or F4\or F5\or F6\or F7\or
%    \end{macrocode}
%    Avoid 255 (0xFF) to get rid of a possible unicode
%    marker at the begin of the string.
%    \begin{macrocode}
      F8\or F9\or FA\or FB\or FC\or FD\or FE%
    \fi
  }%
%    \end{macrocode}
%    \end{macro}
%    \begin{macro}{HypDest@HexString}
%    Now package \xpackage{alphalph} comes into play.
%    \cs{HypDest@HexString} is defined and converts
%    a positive number into a string, given in hexadecimal
%    representation.
%    \begin{macrocode}
  \newalphalph\HypDest@HexString\HypDest@HexChar{250}%
%    \end{macrocode}
%    \end{macro}
%    \begin{macro}{\theHypDest}
%    For use, the hexadecimal string is converted back.
%    \begin{macrocode}
  \renewcommand*{\theHypDest}{%
    \pdf@unescapehex{\HypDest@HexString{\value{HypDest}}}%
  }%
%    \end{macrocode}
%    \end{macro}
%
%    With option \xoption{num} we use the number directly.
%    \begin{macrocode}
\else
  \renewcommand*{\theHypDest}{%
    \number\value{HypDest}%
  }%
\fi
%    \end{macrocode}
%
% \subsection{Assign destination names}
%
%    \begin{macro}{\HypDest@Prefix}
%    The new destination names are remembered in macros whose names
%    start with prefix \cs{HypDest@Prefix}.
%    \begin{macrocode}
\edef\HypDest@Prefix{HypDest\string:}
%    \end{macrocode}
%    \end{macro}
%
%    \begin{macro}{\HypDest@Use}
%    During the first read of the auxiliary files, the used destinations
%    get fresh generated short destination names. Also for the old
%    destination names we use the hexadecimal representation. That
%    avoid problems with arbitrary names.
%    \begin{macrocode}
\def\HypDest@Use#1{%
  \begingroup
    \edef\x{%
      \expandafter\noexpand
      \csname\HypDest@Prefix\pdf@unescapehex{#1}\endcsname
    }%
    \expandafter\ifx\x\relax
      \stepcounter{HypDest}%
      \expandafter\xdef\x{\theHypDest}%
      \let\on@line\@empty
      \ifHypDest@name
        \HypDest@VerboseInfo{%
          Use: (\pdf@unescapehex{#1}) -\string> %
          0x\pdf@escapehex{\x} (\number\value{HypDest})%
        }%
      \else
        \HypDest@VerboseInfo{%
          Use: (\pdf@unescapehex{#1}) -\string> num \x
        }%
      \fi
    \fi
  \endgroup
}
%    \end{macrocode}
%    \end{macro}
%
%    After the first \xfile{.aux} file processing the destination names
%    are assigned and we can disable \cs{HypDest@Use}.
%    \begin{macrocode}
\AtBeginDocument{%
  \let\HypDest@Use\@gobble
}
%    \end{macrocode}
%
%    \begin{macro}{\HypDest@MarkUsed}
%    Destinations that are actually used are marked by \cs{HypDest@MarkUsed}.
%    \cs{nofiles} is respected.
%    \begin{macrocode}
\def\HypDest@MarkUsed#1{%
  \HypDest@VerboseInfo{%
    MarkUsed: (#1)%
  }%
  \if@filesw
    \immediate\write\@auxout{%
      \string\HypDest@Use{\pdf@escapehex{#1}}%
    }%
  \fi
}%
%    \end{macrocode}
%    \end{macro}
%
% \subsection{Redefinition of \xpackage{hyperref}'s hooks}
%
%    Package \xpackage{hyperref} can be loaded later, therefore
%    we redefine \xpackage{hyperref}'s macros at |\begin{document}|.
%    \begin{macrocode}
\HypDest@PrependDocument{%
%    \end{macrocode}
%
%    Check hyperref version.
%    \begin{macrocode}
  \@ifpackagelater{hyperref}{2006/06/01}{}{%
    \PackageError{hypdestopt}{%
      hyperref 2006/06/01 v6.75a or later is required%
    }\@ehc
  }%
%    \end{macrocode}
%
% \subsubsection{Destination setting}
%
%    \begin{macrocode}
  \ifHypDest@name
    \let\HypDest@Org@DestName\Hy@DestName
    \renewcommand*{\Hy@DestName}[2]{%
      \EdefUnescapeString\HypDest@temp{#1}%
      \@ifundefined{\HypDest@Prefix\HypDest@temp}{%
        \HypDest@VerboseInfo{%
          DestName: (\HypDest@temp) unused%
        }%
      }{%
        \HypDest@Org@DestName{%
          \csname\HypDest@Prefix\HypDest@temp\endcsname
        }{#2}%
        \HypDest@VerboseInfo{%
          DestName: (\HypDest@temp) %
          0x\pdf@escapehex{%
            \csname\HypDest@Prefix\HypDest@temp\endcsname
          }%
        }%
      }%
    }%
  \else
    \renewcommand*{\Hy@DestName}[2]{%
      \EdefUnescapeString\HypDest@temp{#1}%
      \@ifundefined{\HypDest@Prefix\HypDest@temp}{%
        \HypDest@VerboseInfo{%
          DestName: (\HypDest@temp) unused%
        }%
      }{%
        \pdfdest num%
        \csname\HypDest@Prefix\HypDest@temp\endcsname#2\relax
        \HypDest@VerboseInfo{%
          DestName: (\HypDest@temp) %
          num \csname\HypDest@Prefix\HypDest@temp\endcsname
        }%
      }%
    }%
  \fi
%    \end{macrocode}
%
% \subsubsection{Links}
%
%    \begin{macrocode}
  \let\HypDest@Org@StartlinkName\Hy@StartlinkName
  \ifHypDest@name
    \renewcommand*{\Hy@StartlinkName}[2]{%
      \HypDest@MarkUsed{#2}%
      \HypDest@Org@StartlinkName{#1}{%
        \@ifundefined{\HypDest@Prefix#2}{%
          #2%
        }{%
          \csname\HypDest@Prefix#2\endcsname
        }%
      }%
    }%
  \else
    \renewcommand*{\Hy@StartlinkName}[2]{%
      \HypDest@MarkUsed{#2}%
      \@ifundefined{\HypDest@Prefix#2}{%
        \HypDest@Org@StartlinkName{#1}{#2}%
      }{%
        \pdfstartlink attr{#1}%
                      goto num\csname\HypDest@Prefix#2\endcsname
        \relax
      }%
    }%
  \fi
%    \end{macrocode}
%
% \subsubsection{Outlines of package \xpackage{hyperref}}
%
%    \begin{macrocode}
  \let\HypDest@Org@OutlineName\Hy@OutlineName
  \ifHypDest@name
    \renewcommand*{\Hy@OutlineName}[4]{%
      \HypDest@Org@OutlineName{#1}{%
        \@ifundefined{\HypDest@Prefix#2}{%
          #2%
        }{%
          \csname\HypDest@Prefix#2\endcsname
        }%
      }{#3}{#4}%
    }%
  \else
    \renewcommand*{\Hy@OutlineName}[4]{%
      \@ifundefined{\HypDest@Prefix#2}{%
        \HypDest@Org@OutlineName{#1}{#2}{#3}{#4}%
      }{%
        \pdfoutline goto num\csname\HypDest@Prefix#2\endcsname
                    count#3{#4}%
      }%
    }%
  \fi
%    \end{macrocode}
%    Because \cs{Hy@OutlineName} is called after the \xfile{.out} file
%    is written in the previous run. Therefore we mark the destination
%    earlier in \cs{@@writetorep}.
%    \begin{macrocode}
  \let\HypDest@Org@@writetorep\@@writetorep
  \renewcommand*{\@@writetorep}[5]{%
    \begingroup
      \edef\Hy@tempa{#5}%
      \ifx\Hy@tempa\Hy@bookmarkstype
        \HypDest@MarkUsed{#3}%
      \fi
    \endgroup
    \HypDest@Org@@writetorep{#1}{#2}{#3}{#4}{#5}%
  }%
%    \end{macrocode}
%
% \subsubsection{Outlines of package \xpackage{bookmark}}
%
%    \begin{macrocode}
  \@ifpackageloaded{bookmark}{%
    \@ifpackagelater{bookmark}{2008/08/08}{%
      \renewcommand*{\BKM@DefGotoNameAction}[2]{%
        \@ifundefined{\HypDest@Prefix#2}{%
          \edef#1{goto name{hypdestopt\string :unknown}}%
        }{%
          \ifHypDest@name
            \edef#1{goto name{\csname\HypDest@Prefix#2\endcsname}}%
          \else
            \edef#1{goto num\csname\HypDest@Prefix#2\endcsname}%
          \fi
        }%
      }%
      \def\BKM@HypDestOptHook{%
        \ifx\BKM@dest\@empty
        \else
          \ifx\BKM@gotor\@empty
            \HypDest@MarkUsed\BKM@dest
          \fi
        \fi
      }%
    }{%
      \@PackageError{hypdestopt}{%
        Package `bookmark' is too old.\MessageBreak
        Version 2008/08/08 or later is needed%
      }\@ehc
    }%
  }{}%
%    \end{macrocode}
%
%    \begin{macrocode}
}
%    \end{macrocode}
%
%
%    \begin{macrocode}
%</package>
%    \end{macrocode}
%
% \section{Installation}
%
% \subsection{Download}
%
% \paragraph{Package.} This package is available on
% CTAN\footnote{\url{ftp://ftp.ctan.org/tex-archive/}}:
% \begin{description}
% \item[\CTAN{macros/latex/contrib/oberdiek/hypdestopt.dtx}] The source file.
% \item[\CTAN{macros/latex/contrib/oberdiek/hypdestopt.pdf}] Documentation.
% \end{description}
%
%
% \paragraph{Bundle.} All the packages of the bundle `oberdiek'
% are also available in a TDS compliant ZIP archive. There
% the packages are already unpacked and the documentation files
% are generated. The files and directories obey the TDS standard.
% \begin{description}
% \item[\CTAN{install/macros/latex/contrib/oberdiek.tds.zip}]
% \end{description}
% \emph{TDS} refers to the standard ``A Directory Structure
% for \TeX\ Files'' (\CTAN{tds/tds.pdf}). Directories
% with \xfile{texmf} in their name are usually organized this way.
%
% \subsection{Bundle installation}
%
% \paragraph{Unpacking.} Unpack the \xfile{oberdiek.tds.zip} in the
% TDS tree (also known as \xfile{texmf} tree) of your choice.
% Example (linux):
% \begin{quote}
%   |unzip oberdiek.tds.zip -d ~/texmf|
% \end{quote}
%
% \paragraph{Script installation.}
% Check the directory \xfile{TDS:scripts/oberdiek/} for
% scripts that need further installation steps.
% Package \xpackage{attachfile2} comes with the Perl script
% \xfile{pdfatfi.pl} that should be installed in such a way
% that it can be called as \texttt{pdfatfi}.
% Example (linux):
% \begin{quote}
%   |chmod +x scripts/oberdiek/pdfatfi.pl|\\
%   |cp scripts/oberdiek/pdfatfi.pl /usr/local/bin/|
% \end{quote}
%
% \subsection{Package installation}
%
% \paragraph{Unpacking.} The \xfile{.dtx} file is a self-extracting
% \docstrip\ archive. The files are extracted by running the
% \xfile{.dtx} through \plainTeX:
% \begin{quote}
%   \verb|tex hypdestopt.dtx|
% \end{quote}
%
% \paragraph{TDS.} Now the different files must be moved into
% the different directories in your installation TDS tree
% (also known as \xfile{texmf} tree):
% \begin{quote}
% \def\t{^^A
% \begin{tabular}{@{}>{\ttfamily}l@{ $\rightarrow$ }>{\ttfamily}l@{}}
%   hypdestopt.sty & tex/latex/oberdiek/hypdestopt.sty\\
%   hypdestopt.pdf & doc/latex/oberdiek/hypdestopt.pdf\\
%   hypdestopt.dtx & source/latex/oberdiek/hypdestopt.dtx\\
% \end{tabular}^^A
% }^^A
% \sbox0{\t}^^A
% \ifdim\wd0>\linewidth
%   \begingroup
%     \advance\linewidth by\leftmargin
%     \advance\linewidth by\rightmargin
%   \edef\x{\endgroup
%     \def\noexpand\lw{\the\linewidth}^^A
%   }\x
%   \def\lwbox{^^A
%     \leavevmode
%     \hbox to \linewidth{^^A
%       \kern-\leftmargin\relax
%       \hss
%       \usebox0
%       \hss
%       \kern-\rightmargin\relax
%     }^^A
%   }^^A
%   \ifdim\wd0>\lw
%     \sbox0{\small\t}^^A
%     \ifdim\wd0>\linewidth
%       \ifdim\wd0>\lw
%         \sbox0{\footnotesize\t}^^A
%         \ifdim\wd0>\linewidth
%           \ifdim\wd0>\lw
%             \sbox0{\scriptsize\t}^^A
%             \ifdim\wd0>\linewidth
%               \ifdim\wd0>\lw
%                 \sbox0{\tiny\t}^^A
%                 \ifdim\wd0>\linewidth
%                   \lwbox
%                 \else
%                   \usebox0
%                 \fi
%               \else
%                 \lwbox
%               \fi
%             \else
%               \usebox0
%             \fi
%           \else
%             \lwbox
%           \fi
%         \else
%           \usebox0
%         \fi
%       \else
%         \lwbox
%       \fi
%     \else
%       \usebox0
%     \fi
%   \else
%     \lwbox
%   \fi
% \else
%   \usebox0
% \fi
% \end{quote}
% If you have a \xfile{docstrip.cfg} that configures and enables \docstrip's
% TDS installing feature, then some files can already be in the right
% place, see the documentation of \docstrip.
%
% \subsection{Refresh file name databases}
%
% If your \TeX~distribution
% (\teTeX, \mikTeX, \dots) relies on file name databases, you must refresh
% these. For example, \teTeX\ users run \verb|texhash| or
% \verb|mktexlsr|.
%
% \subsection{Some details for the interested}
%
% \paragraph{Attached source.}
%
% The PDF documentation on CTAN also includes the
% \xfile{.dtx} source file. It can be extracted by
% AcrobatReader 6 or higher. Another option is \textsf{pdftk},
% e.g. unpack the file into the current directory:
% \begin{quote}
%   \verb|pdftk hypdestopt.pdf unpack_files output .|
% \end{quote}
%
% \paragraph{Unpacking with \LaTeX.}
% The \xfile{.dtx} chooses its action depending on the format:
% \begin{description}
% \item[\plainTeX:] Run \docstrip\ and extract the files.
% \item[\LaTeX:] Generate the documentation.
% \end{description}
% If you insist on using \LaTeX\ for \docstrip\ (really,
% \docstrip\ does not need \LaTeX), then inform the autodetect routine
% about your intention:
% \begin{quote}
%   \verb|latex \let\install=y\input{hypdestopt.dtx}|
% \end{quote}
% Do not forget to quote the argument according to the demands
% of your shell.
%
% \paragraph{Generating the documentation.}
% You can use both the \xfile{.dtx} or the \xfile{.drv} to generate
% the documentation. The process can be configured by the
% configuration file \xfile{ltxdoc.cfg}. For instance, put this
% line into this file, if you want to have A4 as paper format:
% \begin{quote}
%   \verb|\PassOptionsToClass{a4paper}{article}|
% \end{quote}
% An example follows how to generate the
% documentation with pdf\LaTeX:
% \begin{quote}
%\begin{verbatim}
%pdflatex hypdestopt.dtx
%makeindex -s gind.ist hypdestopt.idx
%pdflatex hypdestopt.dtx
%makeindex -s gind.ist hypdestopt.idx
%pdflatex hypdestopt.dtx
%\end{verbatim}
% \end{quote}
%
% \section{Catalogue}
%
% The following XML file can be used as source for the
% \href{http://mirror.ctan.org/help/Catalogue/catalogue.html}{\TeX\ Catalogue}.
% The elements \texttt{caption} and \texttt{description} are imported
% from the original XML file from the Catalogue.
% The name of the XML file in the Catalogue is \xfile{hypdestopt.xml}.
%    \begin{macrocode}
%<*catalogue>
<?xml version='1.0' encoding='us-ascii'?>
<!DOCTYPE entry SYSTEM 'catalogue.dtd'>
<entry datestamp='$Date$' modifier='$Author$' id='hypdestopt'>
  <name>hypdestopt</name>
  <caption>Hyperref destination optimizer.</caption>
  <authorref id='auth:oberdiek'/>
  <copyright owner='Heiko Oberdiek' year='2006-2008,2011'/>
  <license type='lppl1.3'/>
  <version number='2.3'/>
  <description>
    This package supports <xref refid='hyperref'>hyperref</xref>'s
    pdftex driver. It removes unnecessary destinations
    and shortens the destination names or uses numbered destinations
    to get smaller PDF files.
    <p/>
    The package is part of the <xref refid='oberdiek'>oberdiek</xref>
    bundle.
  </description>
  <documentation details='Package documentation'
      href='ctan:/macros/latex/contrib/oberdiek/hypdestopt.pdf'/>
  <ctan file='true' path='/macros/latex/contrib/oberdiek/hypdestopt.dtx'/>
  <miktex location='oberdiek'/>
  <texlive location='oberdiek'/>
  <install path='/macros/latex/contrib/oberdiek/oberdiek.tds.zip'/>
</entry>
%</catalogue>
%    \end{macrocode}
%
% \begin{thebibliography}{9}
%
% \bibitem{alphalph}
%   Heiko Oberdiek: \textit{The \xpackage{alphalph} package};
%   2006/05/30 v1.4;
%   \CTAN{macros/latex/contrib/oberdiek/alphalph.pdf}.
%
% \bibitem{hyperref}
%   Sebastian Rahtz, Heiko Oberdiek:
%   \textit{The \xpackage{hyperref} package};
%   2006/06/01 v6.75a;
%   \CTAN{macros/latex/contrib/hyperref/}.
%
% \bibitem{ifpdf}
%   Heiko Oberdiek: \textit{The \xpackage{ifpdf} package};
%   2006/02/20 v1.4;
%   \CTAN{macros/latex/contrib/oberdiek/ifpdf.pdf}.
%
% \end{thebibliography}
%
% \begin{History}
%   \begin{Version}{2006/06/01 v1.0}
%   \item
%     First version.
%   \end{Version}
%   \begin{Version}{2006/06/01 v2.0}
%   \item
%     New method for referencing destinations by number; an idea
%     proposed by Lars Hellstr\"om in the mailing list LATEX-L.
%   \item
%     Options \xoption{name} and \xoption{num} added.
%   \end{Version}
%   \begin{Version}{2007/11/11 v2.1}
%   \item
%     Use of package \xpackage{pdftexcmds} for \LuaTeX\ support.
%   \end{Version}
%   \begin{Version}{2008/08/08 v2.2}
%   \item
%     Support for package \xpackage{bookmark} added.
%   \end{Version}
%   \begin{Version}{2011/05/13 v2.3}
%   \item
%     Fix for \cs{Hy@DestName} if the destination name contains
%     special characters.
%   \item
%     Fix for option \xoption{name} and package \xpackage{bookmark}.
%   \end{Version}
% \end{History}
%
% \PrintIndex
%
% \Finale
\endinput
|
% \end{quote}
% Do not forget to quote the argument according to the demands
% of your shell.
%
% \paragraph{Generating the documentation.}
% You can use both the \xfile{.dtx} or the \xfile{.drv} to generate
% the documentation. The process can be configured by the
% configuration file \xfile{ltxdoc.cfg}. For instance, put this
% line into this file, if you want to have A4 as paper format:
% \begin{quote}
%   \verb|\PassOptionsToClass{a4paper}{article}|
% \end{quote}
% An example follows how to generate the
% documentation with pdf\LaTeX:
% \begin{quote}
%\begin{verbatim}
%pdflatex hypdestopt.dtx
%makeindex -s gind.ist hypdestopt.idx
%pdflatex hypdestopt.dtx
%makeindex -s gind.ist hypdestopt.idx
%pdflatex hypdestopt.dtx
%\end{verbatim}
% \end{quote}
%
% \section{Catalogue}
%
% The following XML file can be used as source for the
% \href{http://mirror.ctan.org/help/Catalogue/catalogue.html}{\TeX\ Catalogue}.
% The elements \texttt{caption} and \texttt{description} are imported
% from the original XML file from the Catalogue.
% The name of the XML file in the Catalogue is \xfile{hypdestopt.xml}.
%    \begin{macrocode}
%<*catalogue>
<?xml version='1.0' encoding='us-ascii'?>
<!DOCTYPE entry SYSTEM 'catalogue.dtd'>
<entry datestamp='$Date$' modifier='$Author$' id='hypdestopt'>
  <name>hypdestopt</name>
  <caption>Hyperref destination optimizer.</caption>
  <authorref id='auth:oberdiek'/>
  <copyright owner='Heiko Oberdiek' year='2006-2008,2011'/>
  <license type='lppl1.3'/>
  <version number='2.3'/>
  <description>
    This package supports <xref refid='hyperref'>hyperref</xref>'s
    pdftex driver. It removes unnecessary destinations
    and shortens the destination names or uses numbered destinations
    to get smaller PDF files.
    <p/>
    The package is part of the <xref refid='oberdiek'>oberdiek</xref>
    bundle.
  </description>
  <documentation details='Package documentation'
      href='ctan:/macros/latex/contrib/oberdiek/hypdestopt.pdf'/>
  <ctan file='true' path='/macros/latex/contrib/oberdiek/hypdestopt.dtx'/>
  <miktex location='oberdiek'/>
  <texlive location='oberdiek'/>
  <install path='/macros/latex/contrib/oberdiek/oberdiek.tds.zip'/>
</entry>
%</catalogue>
%    \end{macrocode}
%
% \begin{thebibliography}{9}
%
% \bibitem{alphalph}
%   Heiko Oberdiek: \textit{The \xpackage{alphalph} package};
%   2006/05/30 v1.4;
%   \CTAN{macros/latex/contrib/oberdiek/alphalph.pdf}.
%
% \bibitem{hyperref}
%   Sebastian Rahtz, Heiko Oberdiek:
%   \textit{The \xpackage{hyperref} package};
%   2006/06/01 v6.75a;
%   \CTAN{macros/latex/contrib/hyperref/}.
%
% \bibitem{ifpdf}
%   Heiko Oberdiek: \textit{The \xpackage{ifpdf} package};
%   2006/02/20 v1.4;
%   \CTAN{macros/latex/contrib/oberdiek/ifpdf.pdf}.
%
% \end{thebibliography}
%
% \begin{History}
%   \begin{Version}{2006/06/01 v1.0}
%   \item
%     First version.
%   \end{Version}
%   \begin{Version}{2006/06/01 v2.0}
%   \item
%     New method for referencing destinations by number; an idea
%     proposed by Lars Hellstr\"om in the mailing list LATEX-L.
%   \item
%     Options \xoption{name} and \xoption{num} added.
%   \end{Version}
%   \begin{Version}{2007/11/11 v2.1}
%   \item
%     Use of package \xpackage{pdftexcmds} for \LuaTeX\ support.
%   \end{Version}
%   \begin{Version}{2008/08/08 v2.2}
%   \item
%     Support for package \xpackage{bookmark} added.
%   \end{Version}
%   \begin{Version}{2011/05/13 v2.3}
%   \item
%     Fix for \cs{Hy@DestName} if the destination name contains
%     special characters.
%   \item
%     Fix for option \xoption{name} and package \xpackage{bookmark}.
%   \end{Version}
% \end{History}
%
% \PrintIndex
%
% \Finale
\endinput
|
% \end{quote}
% Do not forget to quote the argument according to the demands
% of your shell.
%
% \paragraph{Generating the documentation.}
% You can use both the \xfile{.dtx} or the \xfile{.drv} to generate
% the documentation. The process can be configured by the
% configuration file \xfile{ltxdoc.cfg}. For instance, put this
% line into this file, if you want to have A4 as paper format:
% \begin{quote}
%   \verb|\PassOptionsToClass{a4paper}{article}|
% \end{quote}
% An example follows how to generate the
% documentation with pdf\LaTeX:
% \begin{quote}
%\begin{verbatim}
%pdflatex hypdestopt.dtx
%makeindex -s gind.ist hypdestopt.idx
%pdflatex hypdestopt.dtx
%makeindex -s gind.ist hypdestopt.idx
%pdflatex hypdestopt.dtx
%\end{verbatim}
% \end{quote}
%
% \section{Catalogue}
%
% The following XML file can be used as source for the
% \href{http://mirror.ctan.org/help/Catalogue/catalogue.html}{\TeX\ Catalogue}.
% The elements \texttt{caption} and \texttt{description} are imported
% from the original XML file from the Catalogue.
% The name of the XML file in the Catalogue is \xfile{hypdestopt.xml}.
%    \begin{macrocode}
%<*catalogue>
<?xml version='1.0' encoding='us-ascii'?>
<!DOCTYPE entry SYSTEM 'catalogue.dtd'>
<entry datestamp='$Date$' modifier='$Author$' id='hypdestopt'>
  <name>hypdestopt</name>
  <caption>Hyperref destination optimizer.</caption>
  <authorref id='auth:oberdiek'/>
  <copyright owner='Heiko Oberdiek' year='2006-2008,2011'/>
  <license type='lppl1.3'/>
  <version number='2.3'/>
  <description>
    This package supports <xref refid='hyperref'>hyperref</xref>'s
    pdftex driver. It removes unnecessary destinations
    and shortens the destination names or uses numbered destinations
    to get smaller PDF files.
    <p/>
    The package is part of the <xref refid='oberdiek'>oberdiek</xref>
    bundle.
  </description>
  <documentation details='Package documentation'
      href='ctan:/macros/latex/contrib/oberdiek/hypdestopt.pdf'/>
  <ctan file='true' path='/macros/latex/contrib/oberdiek/hypdestopt.dtx'/>
  <miktex location='oberdiek'/>
  <texlive location='oberdiek'/>
  <install path='/macros/latex/contrib/oberdiek/oberdiek.tds.zip'/>
</entry>
%</catalogue>
%    \end{macrocode}
%
% \begin{thebibliography}{9}
%
% \bibitem{alphalph}
%   Heiko Oberdiek: \textit{The \xpackage{alphalph} package};
%   2006/05/30 v1.4;
%   \CTAN{macros/latex/contrib/oberdiek/alphalph.pdf}.
%
% \bibitem{hyperref}
%   Sebastian Rahtz, Heiko Oberdiek:
%   \textit{The \xpackage{hyperref} package};
%   2006/06/01 v6.75a;
%   \CTAN{macros/latex/contrib/hyperref/}.
%
% \bibitem{ifpdf}
%   Heiko Oberdiek: \textit{The \xpackage{ifpdf} package};
%   2006/02/20 v1.4;
%   \CTAN{macros/latex/contrib/oberdiek/ifpdf.pdf}.
%
% \end{thebibliography}
%
% \begin{History}
%   \begin{Version}{2006/06/01 v1.0}
%   \item
%     First version.
%   \end{Version}
%   \begin{Version}{2006/06/01 v2.0}
%   \item
%     New method for referencing destinations by number; an idea
%     proposed by Lars Hellstr\"om in the mailing list LATEX-L.
%   \item
%     Options \xoption{name} and \xoption{num} added.
%   \end{Version}
%   \begin{Version}{2007/11/11 v2.1}
%   \item
%     Use of package \xpackage{pdftexcmds} for \LuaTeX\ support.
%   \end{Version}
%   \begin{Version}{2008/08/08 v2.2}
%   \item
%     Support for package \xpackage{bookmark} added.
%   \end{Version}
%   \begin{Version}{2011/05/13 v2.3}
%   \item
%     Fix for \cs{Hy@DestName} if the destination name contains
%     special characters.
%   \item
%     Fix for option \xoption{name} and package \xpackage{bookmark}.
%   \end{Version}
% \end{History}
%
% \PrintIndex
%
% \Finale
\endinput

%        (quote the arguments according to the demands of your shell)
%
% Documentation:
%    (a) If hypdestopt.drv is present:
%           latex hypdestopt.drv
%    (b) Without hypdestopt.drv:
%           latex hypdestopt.dtx; ...
%    The class ltxdoc loads the configuration file ltxdoc.cfg
%    if available. Here you can specify further options, e.g.
%    use A4 as paper format:
%       \PassOptionsToClass{a4paper}{article}
%
%    Programm calls to get the documentation (example):
%       pdflatex hypdestopt.dtx
%       makeindex -s gind.ist hypdestopt.idx
%       pdflatex hypdestopt.dtx
%       makeindex -s gind.ist hypdestopt.idx
%       pdflatex hypdestopt.dtx
%
% Installation:
%    TDS:tex/latex/oberdiek/hypdestopt.sty
%    TDS:doc/latex/oberdiek/hypdestopt.pdf
%    TDS:source/latex/oberdiek/hypdestopt.dtx
%
%<*ignore>
\begingroup
  \catcode123=1 %
  \catcode125=2 %
  \def\x{LaTeX2e}%
\expandafter\endgroup
\ifcase 0\ifx\install y1\fi\expandafter
         \ifx\csname processbatchFile\endcsname\relax\else1\fi
         \ifx\fmtname\x\else 1\fi\relax
\else\csname fi\endcsname
%</ignore>
%<*install>
\input docstrip.tex
\Msg{************************************************************************}
\Msg{* Installation}
\Msg{* Package: hypdestopt 2011/05/13 v2.3 Hyperref destination optimizer (HO)}
\Msg{************************************************************************}

\keepsilent
\askforoverwritefalse

\let\MetaPrefix\relax
\preamble

This is a generated file.

Project: hypdestopt
Version: 2011/05/13 v2.3

Copyright (C) 2006-2008, 2011 by
   Heiko Oberdiek <heiko.oberdiek at googlemail.com>

This work may be distributed and/or modified under the
conditions of the LaTeX Project Public License, either
version 1.3c of this license or (at your option) any later
version. This version of this license is in
   http://www.latex-project.org/lppl/lppl-1-3c.txt
and the latest version of this license is in
   http://www.latex-project.org/lppl.txt
and version 1.3 or later is part of all distributions of
LaTeX version 2005/12/01 or later.

This work has the LPPL maintenance status "maintained".

This Current Maintainer of this work is Heiko Oberdiek.

This work consists of the main source file hypdestopt.dtx
and the derived files
   hypdestopt.sty, hypdestopt.pdf, hypdestopt.ins, hypdestopt.drv.

\endpreamble
\let\MetaPrefix\DoubleperCent

\generate{%
  \file{hypdestopt.ins}{\from{hypdestopt.dtx}{install}}%
  \file{hypdestopt.drv}{\from{hypdestopt.dtx}{driver}}%
  \usedir{tex/latex/oberdiek}%
  \file{hypdestopt.sty}{\from{hypdestopt.dtx}{package}}%
  \nopreamble
  \nopostamble
  \usedir{source/latex/oberdiek/catalogue}%
  \file{hypdestopt.xml}{\from{hypdestopt.dtx}{catalogue}}%
}

\catcode32=13\relax% active space
\let =\space%
\Msg{************************************************************************}
\Msg{*}
\Msg{* To finish the installation you have to move the following}
\Msg{* file into a directory searched by TeX:}
\Msg{*}
\Msg{*     hypdestopt.sty}
\Msg{*}
\Msg{* To produce the documentation run the file `hypdestopt.drv'}
\Msg{* through LaTeX.}
\Msg{*}
\Msg{* Happy TeXing!}
\Msg{*}
\Msg{************************************************************************}

\endbatchfile
%</install>
%<*ignore>
\fi
%</ignore>
%<*driver>
\NeedsTeXFormat{LaTeX2e}
\ProvidesFile{hypdestopt.drv}%
  [2011/05/13 v2.3 Hyperref destination optimizer (HO)]%
\documentclass{ltxdoc}
\usepackage{holtxdoc}[2011/11/22]
\begin{document}
  \DocInput{hypdestopt.dtx}%
\end{document}
%</driver>
% \fi
%
% \CheckSum{565}
%
% \CharacterTable
%  {Upper-case    \A\B\C\D\E\F\G\H\I\J\K\L\M\N\O\P\Q\R\S\T\U\V\W\X\Y\Z
%   Lower-case    \a\b\c\d\e\f\g\h\i\j\k\l\m\n\o\p\q\r\s\t\u\v\w\x\y\z
%   Digits        \0\1\2\3\4\5\6\7\8\9
%   Exclamation   \!     Double quote  \"     Hash (number) \#
%   Dollar        \$     Percent       \%     Ampersand     \&
%   Acute accent  \'     Left paren    \(     Right paren   \)
%   Asterisk      \*     Plus          \+     Comma         \,
%   Minus         \-     Point         \.     Solidus       \/
%   Colon         \:     Semicolon     \;     Less than     \<
%   Equals        \=     Greater than  \>     Question mark \?
%   Commercial at \@     Left bracket  \[     Backslash     \\
%   Right bracket \]     Circumflex    \^     Underscore    \_
%   Grave accent  \`     Left brace    \{     Vertical bar  \|
%   Right brace   \}     Tilde         \~}
%
% \GetFileInfo{hypdestopt.drv}
%
% \title{The \xpackage{hypdestopt} package}
% \date{2011/05/13 v2.3}
% \author{Heiko Oberdiek\\\xemail{heiko.oberdiek at googlemail.com}}
%
% \maketitle
%
% \begin{abstract}
% Package \xpackage{hypdestopt} supports \xpackage{hyperref}'s
% \xoption{pdftex} driver. It removes unnecessary destinations
% and shortens the destination names or uses numbered destinations
% to get smaller PDF files.
% \end{abstract}
%
% \tableofcontents
%
% \section{User interface}
%
% \subsection{Introduction}
%
% Before PDF-1.5 annotations and destinations cannot be compressed.
% If the destination names are not needed for external use, the
% file size can be decreased by the following means:
% \begin{itemize}
% \item Unused destinations are removed.
% \item The destination names are shortened (option \xoption{name}).
% \item Using numbered destinations (option \xoption{num}).
% \end{itemize}
%
% \subsection{Requirements}
%
% \begin{itemize}
% \item Package \xpackage{hyperref} 2006/06/01 v6.75a or newer
%       (\cite{hyperref}).
% \item Package \xpackage{alphalph} 2006/05/30 v1.4 or newer
%       (\cite{alphalph}), if option \xoption{name} is used.
% \item Package \xpackage{ifpdf} (\cite{ifpdf}).
% \item \pdfTeX\ 1.30.0 or newer.
% \item \pdfTeX\ in PDF mode.
% \item \eTeX\ extensions enabled.
% \item Probably an additional compile run of \pdfLaTeX\ is necessary.
% \end{itemize}
%
% In the first compile runs you can get warnings such as:
%\begin{quote}
%\begin{verbatim}
%! pdfTeX warning (dest): name{...} has been referenced ...
%\end{verbatim}
%\end{quote}
% These warnings should vanish in later compile runs.
% However these warnings also can occur without this package.
% The package does not cure them, thus these warnings will remain,
% but the destination name can be different. In such cases test
% without package, too.
%
% \subsection{Use}
%
% If the requirements are met, load the package:
%\begin{quote}
%\verb|\usepackage{hypdestopt}|
%\end{quote}
%
% The following options are supported:
% \begin{description}
% \item[\xoption{verbose}:] Verbose debug output is enabled and written
%   in the protocol file.
% \item[\xoption{num}:] Numbered destinations are used. The file size
%   is smaller, because names are no longer used.
%   This is the default.
% \item[\xoption{name}:] Destinations are identified by names.
% \end{description}
%
% \subsection{Limitations}
%
% \begin{itemize}
% \item Forget this package, if you need preserved destination names.
% \item Destination name strings use all bytes (0..255) except
%       the carriage return (13), left parenthesis (40), right
%       parenthesis (41), and backslash (92), because they
%       must be quoted in general and therefore occupy two bytes
%       instead of one.
%
%       Further the zero byte (0) is avoided for programs
%       that implement strings using zero terminated C strings.
%       And 255 (0xFF) is avoided to get rid of a possible
%       unicode marker at the begin.
%
%       So far I have not seen problems with:
%       \begin{itemize}
%       \item AcrobatReader 5.08/Linux
%       \item AcrobatReader 7.0/Linux
%       \item xpdf 3.00
%       \item Ghostscript 8.50
%       \item gv 3.5.8
%       \item GSview 4.6
%       \end{itemize}
%       But I have not tested all and all possible PDF viewers.
% \item Use of named destinations (\cs{pdfdest}, \cs{pdfoutline},
%       \cs{pdfstartlink}, \dots) that are not supported by this
%       package.
% \item Currently only \xpackage{hyperref} with \pdfTeX\ in PDF
%       mode is supported.
% \end{itemize}
%
% \subsection{Future}
%
% A more general approach is a PDF postprocessor that takes
% a PDF file, performs some transformations and writes the
% result in a more optimized PDF file. Then it does not depend,
% how the original PDF file was generated and further improvements
% are easier to apply. For example, the destination names could be sorted:
% often used destination names would then be shorter than seldom used ones.
%
% \StopEventually{
% }
%
% \section{Implementation}
%
% \subsection{Identification}
%
%    \begin{macrocode}
%<*package>
\NeedsTeXFormat{LaTeX2e}
\ProvidesPackage{hypdestopt}%
  [2011/05/13 v2.3 Hyperref destination optimizer (HO)]%
%    \end{macrocode}
%
% \subsection{Options}
%
% \subsubsection{Option \xoption{verbose}}
%
%    \begin{macrocode}
\newif\ifHypDest@Verbose
\DeclareOption{verbose}{\HypDest@Verbosetrue}
%    \end{macrocode}
%
%    \begin{macro}{\HypDest@VerboseInfo}
%    Wrapper for verbose messages.
%    \begin{macrocode}
\def\HypDest@VerboseInfo#1{%
  \ifHypDest@Verbose
    \PackageInfo{hypdestopt}{#1}%
  \fi
}
%    \end{macrocode}
%    \end{macro}
%
% \subsubsection{Options \xoption{num} and \xoption{name}}
%
%    The options \xoption{num} or \xoption{name} specify
%    the method, how destinations are referenced (by name or
%    number). Default is option \xoption{num}.
%    \begin{macrocode}
\newif\ifHypDest@name
\DeclareOption{num}{\HypDest@namefalse}
\DeclareOption{name}{\HypDest@nametrue}
%    \end{macrocode}
%
%    \begin{macrocode}
\ProcessOptions*\relax
%    \end{macrocode}
%
% \subsection{Check requirements}
%
%    First \pdfTeX\ must running in PDF mode.
%    \begin{macrocode}
\RequirePackage{ifpdf}[2007/09/09]
\RequirePackage{pdftexcmds}[2007/11/11]
\ifpdf
\else
  \PackageError{hypdestopt}{%
    This package requires pdfTeX in PDF mode%
  }\@ehc
  \expandafter\endinput
\fi
%    \end{macrocode}
%    The version of \pdfTeX\ must not be too old, because
%    \cs{pdfescapehex} and \cs{pdfunescapehex} are used.
%    \begin{macrocode}
\begingroup\expandafter\expandafter\expandafter\endgroup
\expandafter\ifx\csname pdf@escapehex\endcsname\relax
  \PackageError{hypdestopt}{%
    This pdfTeX is too old, at least 1.30.0 is required%
  }\@ehc
  \expandafter\endinput
\fi
%    \end{macrocode}
%    Features of \eTeX\ are used, e.g. \cs{numexpr}.
%    \begin{macrocode}
\begingroup\expandafter\expandafter\expandafter\endgroup
\expandafter\ifx\csname numexpr\endcsname\relax
  \PackageError{hypdestopt}{%
    e-TeX features are missing%
  }\@ehc
  \expandafter\endinput
\fi
%    \end{macrocode}
%    Package \xpackage{alphalph} provides \cs{newalphalph} since
%    version 2006/05/30 v1.4.
%    \begin{macrocode}
\ifHypDest@name
  \RequirePackage{alphalph}[2006/05/30]%
\fi
%    \end{macrocode}
%    \begin{macrocode}
\RequirePackage{auxhook}[2009/12/14]
\RequirePackage{pdfescape}[2007/04/21]
%    \end{macrocode}
%
% \subsection{Preamble for auxiliary file}
%
%    Provide dummy definitions for the macros that are used in the
%    auxiliary files. If the package is used no longer, then these
%    commands will not generate errors.
%
%    \begin{macro}{\HypDest@PrependDocument}
%    We add our stuff in front of the \cs{AtBeginDocument} hook
%    to ensure that we are before \xpackage{hyperref}'s stuff.
%    \begin{macrocode}
\long\def\HypDest@PrependDocument#1{%
  \begingroup
    \toks\z@{#1}%
    \toks\tw@\expandafter{\@begindocumenthook}%
    \xdef\@begindocumenthook{\the\toks\z@\the\toks\tw@}%
  \endgroup
}
%    \end{macrocode}
%    \end{macro}
%    \begin{macrocode}
\AddLineBeginAux{%
  \string\providecommand{\string\HypDest@Use}[1]{}%
}
%    \end{macrocode}
%
% \subsection{Generation of destination names}
%
%    Counter |HypDest| is used for identifying destinations.
%    \begin{macrocode}
\newcounter{HypDest}
%    \end{macrocode}
%
%    \begin{macrocode}
\ifHypDest@name
%    \end{macrocode}
%
%    \begin{macro}{\HypDest@HexChar}
%    Destination names are generated by automatically
%    numbering with the help of package \xpackage{alphalph}.
%    \cs{HypDest@HexChar} converts a number of the range 1 until 252
%    into the hexadecimal representation of the string character.
%    \begin{macrocode}
  \def\HypDest@HexChar#1{%
    \ifcase#1\or
%    \end{macrocode}
%    Avoid zero byte because of C strings in PDF viewer
%    applications.
%    \begin{macrocode}
      01\or 02\or 03\or 04\or 05\or 06\or 07\or
%    \end{macrocode}
%    Omit carriage return (13/\verb|^^0d|).
%    It needs quoting, otherwise it would be converted
%    to line feed (10/\verb|^^0a|).
%    \begin{macrocode}
      08\or 09\or 0A\or 0B\or 0C\or 0E\or 0F\or
      10\or 11\or 12\or 13\or 14\or 15\or 16\or 17\or
      18\or 19\or 1A\or 1B\or 1C\or 1D\or 1E\or 1F\or
      20\or 21\or 22\or 23\or 24\or 25\or 26\or 27\or
%    \end{macrocode}
%    Omit left and right parentheses (40/\verb|^^28|, 41/\verb|^^39|),
%    they need quoting in general.
%    \begin{macrocode}
      2A\or 2B\or 2C\or 2D\or 2E\or 2F\or
      30\or 31\or 32\or 33\or 34\or 35\or 36\or 37\or
      38\or 39\or 3A\or 3B\or 3C\or 3D\or 3E\or 3F\or
      40\or 41\or 42\or 43\or 44\or 45\or 46\or 47\or
      48\or 49\or 4A\or 4B\or 4C\or 4D\or 4E\or 4F\or
      50\or 51\or 52\or 53\or 54\or 55\or 56\or 57\or
%    \end{macrocode}
%    Omit backslash (92/\verb|^^5C|), it needs quoting.
%    \begin{macrocode}
      58\or 59\or 5A\or 5B\or 5D\or 5E\or 5F\or
      60\or 61\or 62\or 63\or 64\or 65\or 66\or 67\or
      68\or 69\or 6A\or 6B\or 6C\or 6D\or 6E\or 6F\or
      70\or 71\or 72\or 73\or 74\or 75\or 76\or 77\or
      78\or 79\or 7A\or 7B\or 7C\or 7D\or 7E\or 7F\or
      80\or 81\or 82\or 83\or 84\or 85\or 86\or 87\or
      88\or 89\or 8A\or 8B\or 8C\or 8D\or 8E\or 8F\or
      90\or 91\or 92\or 93\or 94\or 95\or 96\or 97\or
      98\or 99\or 9A\or 9B\or 9C\or 9D\or 9E\or 9F\or
      A0\or A1\or A2\or A3\or A4\or A5\or A6\or A7\or
      A8\or A9\or AA\or AB\or AC\or AD\or AE\or AF\or
      B0\or B1\or B2\or B3\or B4\or B5\or B6\or B7\or
      B8\or B9\or BA\or BB\or BC\or BD\or BE\or BF\or
      C0\or C1\or C2\or C3\or C4\or C5\or C6\or C7\or
      C8\or C9\or CA\or CB\or CC\or CD\or CE\or CF\or
      D0\or D1\or D2\or D3\or D4\or D5\or D6\or D7\or
      D8\or D9\or DA\or DB\or DC\or DD\or DE\or DF\or
      E0\or E1\or E2\or E3\or E4\or E5\or E6\or E7\or
      E8\or E9\or EA\or EB\or EC\or ED\or EE\or EF\or
      F0\or F1\or F2\or F3\or F4\or F5\or F6\or F7\or
%    \end{macrocode}
%    Avoid 255 (0xFF) to get rid of a possible unicode
%    marker at the begin of the string.
%    \begin{macrocode}
      F8\or F9\or FA\or FB\or FC\or FD\or FE%
    \fi
  }%
%    \end{macrocode}
%    \end{macro}
%    \begin{macro}{HypDest@HexString}
%    Now package \xpackage{alphalph} comes into play.
%    \cs{HypDest@HexString} is defined and converts
%    a positive number into a string, given in hexadecimal
%    representation.
%    \begin{macrocode}
  \newalphalph\HypDest@HexString\HypDest@HexChar{250}%
%    \end{macrocode}
%    \end{macro}
%    \begin{macro}{\theHypDest}
%    For use, the hexadecimal string is converted back.
%    \begin{macrocode}
  \renewcommand*{\theHypDest}{%
    \pdf@unescapehex{\HypDest@HexString{\value{HypDest}}}%
  }%
%    \end{macrocode}
%    \end{macro}
%
%    With option \xoption{num} we use the number directly.
%    \begin{macrocode}
\else
  \renewcommand*{\theHypDest}{%
    \number\value{HypDest}%
  }%
\fi
%    \end{macrocode}
%
% \subsection{Assign destination names}
%
%    \begin{macro}{\HypDest@Prefix}
%    The new destination names are remembered in macros whose names
%    start with prefix \cs{HypDest@Prefix}.
%    \begin{macrocode}
\edef\HypDest@Prefix{HypDest\string:}
%    \end{macrocode}
%    \end{macro}
%
%    \begin{macro}{\HypDest@Use}
%    During the first read of the auxiliary files, the used destinations
%    get fresh generated short destination names. Also for the old
%    destination names we use the hexadecimal representation. That
%    avoid problems with arbitrary names.
%    \begin{macrocode}
\def\HypDest@Use#1{%
  \begingroup
    \edef\x{%
      \expandafter\noexpand
      \csname\HypDest@Prefix\pdf@unescapehex{#1}\endcsname
    }%
    \expandafter\ifx\x\relax
      \stepcounter{HypDest}%
      \expandafter\xdef\x{\theHypDest}%
      \let\on@line\@empty
      \ifHypDest@name
        \HypDest@VerboseInfo{%
          Use: (\pdf@unescapehex{#1}) -\string> %
          0x\pdf@escapehex{\x} (\number\value{HypDest})%
        }%
      \else
        \HypDest@VerboseInfo{%
          Use: (\pdf@unescapehex{#1}) -\string> num \x
        }%
      \fi
    \fi
  \endgroup
}
%    \end{macrocode}
%    \end{macro}
%
%    After the first \xfile{.aux} file processing the destination names
%    are assigned and we can disable \cs{HypDest@Use}.
%    \begin{macrocode}
\AtBeginDocument{%
  \let\HypDest@Use\@gobble
}
%    \end{macrocode}
%
%    \begin{macro}{\HypDest@MarkUsed}
%    Destinations that are actually used are marked by \cs{HypDest@MarkUsed}.
%    \cs{nofiles} is respected.
%    \begin{macrocode}
\def\HypDest@MarkUsed#1{%
  \HypDest@VerboseInfo{%
    MarkUsed: (#1)%
  }%
  \if@filesw
    \immediate\write\@auxout{%
      \string\HypDest@Use{\pdf@escapehex{#1}}%
    }%
  \fi
}%
%    \end{macrocode}
%    \end{macro}
%
% \subsection{Redefinition of \xpackage{hyperref}'s hooks}
%
%    Package \xpackage{hyperref} can be loaded later, therefore
%    we redefine \xpackage{hyperref}'s macros at |\begin{document}|.
%    \begin{macrocode}
\HypDest@PrependDocument{%
%    \end{macrocode}
%
%    Check hyperref version.
%    \begin{macrocode}
  \@ifpackagelater{hyperref}{2006/06/01}{}{%
    \PackageError{hypdestopt}{%
      hyperref 2006/06/01 v6.75a or later is required%
    }\@ehc
  }%
%    \end{macrocode}
%
% \subsubsection{Destination setting}
%
%    \begin{macrocode}
  \ifHypDest@name
    \let\HypDest@Org@DestName\Hy@DestName
    \renewcommand*{\Hy@DestName}[2]{%
      \EdefUnescapeString\HypDest@temp{#1}%
      \@ifundefined{\HypDest@Prefix\HypDest@temp}{%
        \HypDest@VerboseInfo{%
          DestName: (\HypDest@temp) unused%
        }%
      }{%
        \HypDest@Org@DestName{%
          \csname\HypDest@Prefix\HypDest@temp\endcsname
        }{#2}%
        \HypDest@VerboseInfo{%
          DestName: (\HypDest@temp) %
          0x\pdf@escapehex{%
            \csname\HypDest@Prefix\HypDest@temp\endcsname
          }%
        }%
      }%
    }%
  \else
    \renewcommand*{\Hy@DestName}[2]{%
      \EdefUnescapeString\HypDest@temp{#1}%
      \@ifundefined{\HypDest@Prefix\HypDest@temp}{%
        \HypDest@VerboseInfo{%
          DestName: (\HypDest@temp) unused%
        }%
      }{%
        \pdfdest num%
        \csname\HypDest@Prefix\HypDest@temp\endcsname#2\relax
        \HypDest@VerboseInfo{%
          DestName: (\HypDest@temp) %
          num \csname\HypDest@Prefix\HypDest@temp\endcsname
        }%
      }%
    }%
  \fi
%    \end{macrocode}
%
% \subsubsection{Links}
%
%    \begin{macrocode}
  \let\HypDest@Org@StartlinkName\Hy@StartlinkName
  \ifHypDest@name
    \renewcommand*{\Hy@StartlinkName}[2]{%
      \HypDest@MarkUsed{#2}%
      \HypDest@Org@StartlinkName{#1}{%
        \@ifundefined{\HypDest@Prefix#2}{%
          #2%
        }{%
          \csname\HypDest@Prefix#2\endcsname
        }%
      }%
    }%
  \else
    \renewcommand*{\Hy@StartlinkName}[2]{%
      \HypDest@MarkUsed{#2}%
      \@ifundefined{\HypDest@Prefix#2}{%
        \HypDest@Org@StartlinkName{#1}{#2}%
      }{%
        \pdfstartlink attr{#1}%
                      goto num\csname\HypDest@Prefix#2\endcsname
        \relax
      }%
    }%
  \fi
%    \end{macrocode}
%
% \subsubsection{Outlines of package \xpackage{hyperref}}
%
%    \begin{macrocode}
  \let\HypDest@Org@OutlineName\Hy@OutlineName
  \ifHypDest@name
    \renewcommand*{\Hy@OutlineName}[4]{%
      \HypDest@Org@OutlineName{#1}{%
        \@ifundefined{\HypDest@Prefix#2}{%
          #2%
        }{%
          \csname\HypDest@Prefix#2\endcsname
        }%
      }{#3}{#4}%
    }%
  \else
    \renewcommand*{\Hy@OutlineName}[4]{%
      \@ifundefined{\HypDest@Prefix#2}{%
        \HypDest@Org@OutlineName{#1}{#2}{#3}{#4}%
      }{%
        \pdfoutline goto num\csname\HypDest@Prefix#2\endcsname
                    count#3{#4}%
      }%
    }%
  \fi
%    \end{macrocode}
%    Because \cs{Hy@OutlineName} is called after the \xfile{.out} file
%    is written in the previous run. Therefore we mark the destination
%    earlier in \cs{@@writetorep}.
%    \begin{macrocode}
  \let\HypDest@Org@@writetorep\@@writetorep
  \renewcommand*{\@@writetorep}[5]{%
    \begingroup
      \edef\Hy@tempa{#5}%
      \ifx\Hy@tempa\Hy@bookmarkstype
        \HypDest@MarkUsed{#3}%
      \fi
    \endgroup
    \HypDest@Org@@writetorep{#1}{#2}{#3}{#4}{#5}%
  }%
%    \end{macrocode}
%
% \subsubsection{Outlines of package \xpackage{bookmark}}
%
%    \begin{macrocode}
  \@ifpackageloaded{bookmark}{%
    \@ifpackagelater{bookmark}{2008/08/08}{%
      \renewcommand*{\BKM@DefGotoNameAction}[2]{%
        \@ifundefined{\HypDest@Prefix#2}{%
          \edef#1{goto name{hypdestopt\string :unknown}}%
        }{%
          \ifHypDest@name
            \edef#1{goto name{\csname\HypDest@Prefix#2\endcsname}}%
          \else
            \edef#1{goto num\csname\HypDest@Prefix#2\endcsname}%
          \fi
        }%
      }%
      \def\BKM@HypDestOptHook{%
        \ifx\BKM@dest\@empty
        \else
          \ifx\BKM@gotor\@empty
            \HypDest@MarkUsed\BKM@dest
          \fi
        \fi
      }%
    }{%
      \@PackageError{hypdestopt}{%
        Package `bookmark' is too old.\MessageBreak
        Version 2008/08/08 or later is needed%
      }\@ehc
    }%
  }{}%
%    \end{macrocode}
%
%    \begin{macrocode}
}
%    \end{macrocode}
%
%
%    \begin{macrocode}
%</package>
%    \end{macrocode}
%
% \section{Installation}
%
% \subsection{Download}
%
% \paragraph{Package.} This package is available on
% CTAN\footnote{\url{ftp://ftp.ctan.org/tex-archive/}}:
% \begin{description}
% \item[\CTAN{macros/latex/contrib/oberdiek/hypdestopt.dtx}] The source file.
% \item[\CTAN{macros/latex/contrib/oberdiek/hypdestopt.pdf}] Documentation.
% \end{description}
%
%
% \paragraph{Bundle.} All the packages of the bundle `oberdiek'
% are also available in a TDS compliant ZIP archive. There
% the packages are already unpacked and the documentation files
% are generated. The files and directories obey the TDS standard.
% \begin{description}
% \item[\CTAN{install/macros/latex/contrib/oberdiek.tds.zip}]
% \end{description}
% \emph{TDS} refers to the standard ``A Directory Structure
% for \TeX\ Files'' (\CTAN{tds/tds.pdf}). Directories
% with \xfile{texmf} in their name are usually organized this way.
%
% \subsection{Bundle installation}
%
% \paragraph{Unpacking.} Unpack the \xfile{oberdiek.tds.zip} in the
% TDS tree (also known as \xfile{texmf} tree) of your choice.
% Example (linux):
% \begin{quote}
%   |unzip oberdiek.tds.zip -d ~/texmf|
% \end{quote}
%
% \paragraph{Script installation.}
% Check the directory \xfile{TDS:scripts/oberdiek/} for
% scripts that need further installation steps.
% Package \xpackage{attachfile2} comes with the Perl script
% \xfile{pdfatfi.pl} that should be installed in such a way
% that it can be called as \texttt{pdfatfi}.
% Example (linux):
% \begin{quote}
%   |chmod +x scripts/oberdiek/pdfatfi.pl|\\
%   |cp scripts/oberdiek/pdfatfi.pl /usr/local/bin/|
% \end{quote}
%
% \subsection{Package installation}
%
% \paragraph{Unpacking.} The \xfile{.dtx} file is a self-extracting
% \docstrip\ archive. The files are extracted by running the
% \xfile{.dtx} through \plainTeX:
% \begin{quote}
%   \verb|tex hypdestopt.dtx|
% \end{quote}
%
% \paragraph{TDS.} Now the different files must be moved into
% the different directories in your installation TDS tree
% (also known as \xfile{texmf} tree):
% \begin{quote}
% \def\t{^^A
% \begin{tabular}{@{}>{\ttfamily}l@{ $\rightarrow$ }>{\ttfamily}l@{}}
%   hypdestopt.sty & tex/latex/oberdiek/hypdestopt.sty\\
%   hypdestopt.pdf & doc/latex/oberdiek/hypdestopt.pdf\\
%   hypdestopt.dtx & source/latex/oberdiek/hypdestopt.dtx\\
% \end{tabular}^^A
% }^^A
% \sbox0{\t}^^A
% \ifdim\wd0>\linewidth
%   \begingroup
%     \advance\linewidth by\leftmargin
%     \advance\linewidth by\rightmargin
%   \edef\x{\endgroup
%     \def\noexpand\lw{\the\linewidth}^^A
%   }\x
%   \def\lwbox{^^A
%     \leavevmode
%     \hbox to \linewidth{^^A
%       \kern-\leftmargin\relax
%       \hss
%       \usebox0
%       \hss
%       \kern-\rightmargin\relax
%     }^^A
%   }^^A
%   \ifdim\wd0>\lw
%     \sbox0{\small\t}^^A
%     \ifdim\wd0>\linewidth
%       \ifdim\wd0>\lw
%         \sbox0{\footnotesize\t}^^A
%         \ifdim\wd0>\linewidth
%           \ifdim\wd0>\lw
%             \sbox0{\scriptsize\t}^^A
%             \ifdim\wd0>\linewidth
%               \ifdim\wd0>\lw
%                 \sbox0{\tiny\t}^^A
%                 \ifdim\wd0>\linewidth
%                   \lwbox
%                 \else
%                   \usebox0
%                 \fi
%               \else
%                 \lwbox
%               \fi
%             \else
%               \usebox0
%             \fi
%           \else
%             \lwbox
%           \fi
%         \else
%           \usebox0
%         \fi
%       \else
%         \lwbox
%       \fi
%     \else
%       \usebox0
%     \fi
%   \else
%     \lwbox
%   \fi
% \else
%   \usebox0
% \fi
% \end{quote}
% If you have a \xfile{docstrip.cfg} that configures and enables \docstrip's
% TDS installing feature, then some files can already be in the right
% place, see the documentation of \docstrip.
%
% \subsection{Refresh file name databases}
%
% If your \TeX~distribution
% (\teTeX, \mikTeX, \dots) relies on file name databases, you must refresh
% these. For example, \teTeX\ users run \verb|texhash| or
% \verb|mktexlsr|.
%
% \subsection{Some details for the interested}
%
% \paragraph{Attached source.}
%
% The PDF documentation on CTAN also includes the
% \xfile{.dtx} source file. It can be extracted by
% AcrobatReader 6 or higher. Another option is \textsf{pdftk},
% e.g. unpack the file into the current directory:
% \begin{quote}
%   \verb|pdftk hypdestopt.pdf unpack_files output .|
% \end{quote}
%
% \paragraph{Unpacking with \LaTeX.}
% The \xfile{.dtx} chooses its action depending on the format:
% \begin{description}
% \item[\plainTeX:] Run \docstrip\ and extract the files.
% \item[\LaTeX:] Generate the documentation.
% \end{description}
% If you insist on using \LaTeX\ for \docstrip\ (really,
% \docstrip\ does not need \LaTeX), then inform the autodetect routine
% about your intention:
% \begin{quote}
%   \verb|latex \let\install=y% \iffalse meta-comment
%
% File: hypdestopt.dtx
% Version: 2011/05/13 v2.3
% Info: Hyperref destination optimizer
%
% Copyright (C) 2006-2008, 2011 by
%    Heiko Oberdiek <heiko.oberdiek at googlemail.com>
%
% This work may be distributed and/or modified under the
% conditions of the LaTeX Project Public License, either
% version 1.3c of this license or (at your option) any later
% version. This version of this license is in
%    http://www.latex-project.org/lppl/lppl-1-3c.txt
% and the latest version of this license is in
%    http://www.latex-project.org/lppl.txt
% and version 1.3 or later is part of all distributions of
% LaTeX version 2005/12/01 or later.
%
% This work has the LPPL maintenance status "maintained".
%
% This Current Maintainer of this work is Heiko Oberdiek.
%
% This work consists of the main source file hypdestopt.dtx
% and the derived files
%    hypdestopt.sty, hypdestopt.pdf, hypdestopt.ins, hypdestopt.drv.
%
% Distribution:
%    CTAN:macros/latex/contrib/oberdiek/hypdestopt.dtx
%    CTAN:macros/latex/contrib/oberdiek/hypdestopt.pdf
%
% Unpacking:
%    (a) If hypdestopt.ins is present:
%           tex hypdestopt.ins
%    (b) Without hypdestopt.ins:
%           tex hypdestopt.dtx
%    (c) If you insist on using LaTeX
%           latex \let\install=y% \iffalse meta-comment
%
% File: hypdestopt.dtx
% Version: 2011/05/13 v2.3
% Info: Hyperref destination optimizer
%
% Copyright (C) 2006-2008, 2011 by
%    Heiko Oberdiek <heiko.oberdiek at googlemail.com>
%
% This work may be distributed and/or modified under the
% conditions of the LaTeX Project Public License, either
% version 1.3c of this license or (at your option) any later
% version. This version of this license is in
%    http://www.latex-project.org/lppl/lppl-1-3c.txt
% and the latest version of this license is in
%    http://www.latex-project.org/lppl.txt
% and version 1.3 or later is part of all distributions of
% LaTeX version 2005/12/01 or later.
%
% This work has the LPPL maintenance status "maintained".
%
% This Current Maintainer of this work is Heiko Oberdiek.
%
% This work consists of the main source file hypdestopt.dtx
% and the derived files
%    hypdestopt.sty, hypdestopt.pdf, hypdestopt.ins, hypdestopt.drv.
%
% Distribution:
%    CTAN:macros/latex/contrib/oberdiek/hypdestopt.dtx
%    CTAN:macros/latex/contrib/oberdiek/hypdestopt.pdf
%
% Unpacking:
%    (a) If hypdestopt.ins is present:
%           tex hypdestopt.ins
%    (b) Without hypdestopt.ins:
%           tex hypdestopt.dtx
%    (c) If you insist on using LaTeX
%           latex \let\install=y% \iffalse meta-comment
%
% File: hypdestopt.dtx
% Version: 2011/05/13 v2.3
% Info: Hyperref destination optimizer
%
% Copyright (C) 2006-2008, 2011 by
%    Heiko Oberdiek <heiko.oberdiek at googlemail.com>
%
% This work may be distributed and/or modified under the
% conditions of the LaTeX Project Public License, either
% version 1.3c of this license or (at your option) any later
% version. This version of this license is in
%    http://www.latex-project.org/lppl/lppl-1-3c.txt
% and the latest version of this license is in
%    http://www.latex-project.org/lppl.txt
% and version 1.3 or later is part of all distributions of
% LaTeX version 2005/12/01 or later.
%
% This work has the LPPL maintenance status "maintained".
%
% This Current Maintainer of this work is Heiko Oberdiek.
%
% This work consists of the main source file hypdestopt.dtx
% and the derived files
%    hypdestopt.sty, hypdestopt.pdf, hypdestopt.ins, hypdestopt.drv.
%
% Distribution:
%    CTAN:macros/latex/contrib/oberdiek/hypdestopt.dtx
%    CTAN:macros/latex/contrib/oberdiek/hypdestopt.pdf
%
% Unpacking:
%    (a) If hypdestopt.ins is present:
%           tex hypdestopt.ins
%    (b) Without hypdestopt.ins:
%           tex hypdestopt.dtx
%    (c) If you insist on using LaTeX
%           latex \let\install=y\input{hypdestopt.dtx}
%        (quote the arguments according to the demands of your shell)
%
% Documentation:
%    (a) If hypdestopt.drv is present:
%           latex hypdestopt.drv
%    (b) Without hypdestopt.drv:
%           latex hypdestopt.dtx; ...
%    The class ltxdoc loads the configuration file ltxdoc.cfg
%    if available. Here you can specify further options, e.g.
%    use A4 as paper format:
%       \PassOptionsToClass{a4paper}{article}
%
%    Programm calls to get the documentation (example):
%       pdflatex hypdestopt.dtx
%       makeindex -s gind.ist hypdestopt.idx
%       pdflatex hypdestopt.dtx
%       makeindex -s gind.ist hypdestopt.idx
%       pdflatex hypdestopt.dtx
%
% Installation:
%    TDS:tex/latex/oberdiek/hypdestopt.sty
%    TDS:doc/latex/oberdiek/hypdestopt.pdf
%    TDS:source/latex/oberdiek/hypdestopt.dtx
%
%<*ignore>
\begingroup
  \catcode123=1 %
  \catcode125=2 %
  \def\x{LaTeX2e}%
\expandafter\endgroup
\ifcase 0\ifx\install y1\fi\expandafter
         \ifx\csname processbatchFile\endcsname\relax\else1\fi
         \ifx\fmtname\x\else 1\fi\relax
\else\csname fi\endcsname
%</ignore>
%<*install>
\input docstrip.tex
\Msg{************************************************************************}
\Msg{* Installation}
\Msg{* Package: hypdestopt 2011/05/13 v2.3 Hyperref destination optimizer (HO)}
\Msg{************************************************************************}

\keepsilent
\askforoverwritefalse

\let\MetaPrefix\relax
\preamble

This is a generated file.

Project: hypdestopt
Version: 2011/05/13 v2.3

Copyright (C) 2006-2008, 2011 by
   Heiko Oberdiek <heiko.oberdiek at googlemail.com>

This work may be distributed and/or modified under the
conditions of the LaTeX Project Public License, either
version 1.3c of this license or (at your option) any later
version. This version of this license is in
   http://www.latex-project.org/lppl/lppl-1-3c.txt
and the latest version of this license is in
   http://www.latex-project.org/lppl.txt
and version 1.3 or later is part of all distributions of
LaTeX version 2005/12/01 or later.

This work has the LPPL maintenance status "maintained".

This Current Maintainer of this work is Heiko Oberdiek.

This work consists of the main source file hypdestopt.dtx
and the derived files
   hypdestopt.sty, hypdestopt.pdf, hypdestopt.ins, hypdestopt.drv.

\endpreamble
\let\MetaPrefix\DoubleperCent

\generate{%
  \file{hypdestopt.ins}{\from{hypdestopt.dtx}{install}}%
  \file{hypdestopt.drv}{\from{hypdestopt.dtx}{driver}}%
  \usedir{tex/latex/oberdiek}%
  \file{hypdestopt.sty}{\from{hypdestopt.dtx}{package}}%
  \nopreamble
  \nopostamble
  \usedir{source/latex/oberdiek/catalogue}%
  \file{hypdestopt.xml}{\from{hypdestopt.dtx}{catalogue}}%
}

\catcode32=13\relax% active space
\let =\space%
\Msg{************************************************************************}
\Msg{*}
\Msg{* To finish the installation you have to move the following}
\Msg{* file into a directory searched by TeX:}
\Msg{*}
\Msg{*     hypdestopt.sty}
\Msg{*}
\Msg{* To produce the documentation run the file `hypdestopt.drv'}
\Msg{* through LaTeX.}
\Msg{*}
\Msg{* Happy TeXing!}
\Msg{*}
\Msg{************************************************************************}

\endbatchfile
%</install>
%<*ignore>
\fi
%</ignore>
%<*driver>
\NeedsTeXFormat{LaTeX2e}
\ProvidesFile{hypdestopt.drv}%
  [2011/05/13 v2.3 Hyperref destination optimizer (HO)]%
\documentclass{ltxdoc}
\usepackage{holtxdoc}[2011/11/22]
\begin{document}
  \DocInput{hypdestopt.dtx}%
\end{document}
%</driver>
% \fi
%
% \CheckSum{565}
%
% \CharacterTable
%  {Upper-case    \A\B\C\D\E\F\G\H\I\J\K\L\M\N\O\P\Q\R\S\T\U\V\W\X\Y\Z
%   Lower-case    \a\b\c\d\e\f\g\h\i\j\k\l\m\n\o\p\q\r\s\t\u\v\w\x\y\z
%   Digits        \0\1\2\3\4\5\6\7\8\9
%   Exclamation   \!     Double quote  \"     Hash (number) \#
%   Dollar        \$     Percent       \%     Ampersand     \&
%   Acute accent  \'     Left paren    \(     Right paren   \)
%   Asterisk      \*     Plus          \+     Comma         \,
%   Minus         \-     Point         \.     Solidus       \/
%   Colon         \:     Semicolon     \;     Less than     \<
%   Equals        \=     Greater than  \>     Question mark \?
%   Commercial at \@     Left bracket  \[     Backslash     \\
%   Right bracket \]     Circumflex    \^     Underscore    \_
%   Grave accent  \`     Left brace    \{     Vertical bar  \|
%   Right brace   \}     Tilde         \~}
%
% \GetFileInfo{hypdestopt.drv}
%
% \title{The \xpackage{hypdestopt} package}
% \date{2011/05/13 v2.3}
% \author{Heiko Oberdiek\\\xemail{heiko.oberdiek at googlemail.com}}
%
% \maketitle
%
% \begin{abstract}
% Package \xpackage{hypdestopt} supports \xpackage{hyperref}'s
% \xoption{pdftex} driver. It removes unnecessary destinations
% and shortens the destination names or uses numbered destinations
% to get smaller PDF files.
% \end{abstract}
%
% \tableofcontents
%
% \section{User interface}
%
% \subsection{Introduction}
%
% Before PDF-1.5 annotations and destinations cannot be compressed.
% If the destination names are not needed for external use, the
% file size can be decreased by the following means:
% \begin{itemize}
% \item Unused destinations are removed.
% \item The destination names are shortened (option \xoption{name}).
% \item Using numbered destinations (option \xoption{num}).
% \end{itemize}
%
% \subsection{Requirements}
%
% \begin{itemize}
% \item Package \xpackage{hyperref} 2006/06/01 v6.75a or newer
%       (\cite{hyperref}).
% \item Package \xpackage{alphalph} 2006/05/30 v1.4 or newer
%       (\cite{alphalph}), if option \xoption{name} is used.
% \item Package \xpackage{ifpdf} (\cite{ifpdf}).
% \item \pdfTeX\ 1.30.0 or newer.
% \item \pdfTeX\ in PDF mode.
% \item \eTeX\ extensions enabled.
% \item Probably an additional compile run of \pdfLaTeX\ is necessary.
% \end{itemize}
%
% In the first compile runs you can get warnings such as:
%\begin{quote}
%\begin{verbatim}
%! pdfTeX warning (dest): name{...} has been referenced ...
%\end{verbatim}
%\end{quote}
% These warnings should vanish in later compile runs.
% However these warnings also can occur without this package.
% The package does not cure them, thus these warnings will remain,
% but the destination name can be different. In such cases test
% without package, too.
%
% \subsection{Use}
%
% If the requirements are met, load the package:
%\begin{quote}
%\verb|\usepackage{hypdestopt}|
%\end{quote}
%
% The following options are supported:
% \begin{description}
% \item[\xoption{verbose}:] Verbose debug output is enabled and written
%   in the protocol file.
% \item[\xoption{num}:] Numbered destinations are used. The file size
%   is smaller, because names are no longer used.
%   This is the default.
% \item[\xoption{name}:] Destinations are identified by names.
% \end{description}
%
% \subsection{Limitations}
%
% \begin{itemize}
% \item Forget this package, if you need preserved destination names.
% \item Destination name strings use all bytes (0..255) except
%       the carriage return (13), left parenthesis (40), right
%       parenthesis (41), and backslash (92), because they
%       must be quoted in general and therefore occupy two bytes
%       instead of one.
%
%       Further the zero byte (0) is avoided for programs
%       that implement strings using zero terminated C strings.
%       And 255 (0xFF) is avoided to get rid of a possible
%       unicode marker at the begin.
%
%       So far I have not seen problems with:
%       \begin{itemize}
%       \item AcrobatReader 5.08/Linux
%       \item AcrobatReader 7.0/Linux
%       \item xpdf 3.00
%       \item Ghostscript 8.50
%       \item gv 3.5.8
%       \item GSview 4.6
%       \end{itemize}
%       But I have not tested all and all possible PDF viewers.
% \item Use of named destinations (\cs{pdfdest}, \cs{pdfoutline},
%       \cs{pdfstartlink}, \dots) that are not supported by this
%       package.
% \item Currently only \xpackage{hyperref} with \pdfTeX\ in PDF
%       mode is supported.
% \end{itemize}
%
% \subsection{Future}
%
% A more general approach is a PDF postprocessor that takes
% a PDF file, performs some transformations and writes the
% result in a more optimized PDF file. Then it does not depend,
% how the original PDF file was generated and further improvements
% are easier to apply. For example, the destination names could be sorted:
% often used destination names would then be shorter than seldom used ones.
%
% \StopEventually{
% }
%
% \section{Implementation}
%
% \subsection{Identification}
%
%    \begin{macrocode}
%<*package>
\NeedsTeXFormat{LaTeX2e}
\ProvidesPackage{hypdestopt}%
  [2011/05/13 v2.3 Hyperref destination optimizer (HO)]%
%    \end{macrocode}
%
% \subsection{Options}
%
% \subsubsection{Option \xoption{verbose}}
%
%    \begin{macrocode}
\newif\ifHypDest@Verbose
\DeclareOption{verbose}{\HypDest@Verbosetrue}
%    \end{macrocode}
%
%    \begin{macro}{\HypDest@VerboseInfo}
%    Wrapper for verbose messages.
%    \begin{macrocode}
\def\HypDest@VerboseInfo#1{%
  \ifHypDest@Verbose
    \PackageInfo{hypdestopt}{#1}%
  \fi
}
%    \end{macrocode}
%    \end{macro}
%
% \subsubsection{Options \xoption{num} and \xoption{name}}
%
%    The options \xoption{num} or \xoption{name} specify
%    the method, how destinations are referenced (by name or
%    number). Default is option \xoption{num}.
%    \begin{macrocode}
\newif\ifHypDest@name
\DeclareOption{num}{\HypDest@namefalse}
\DeclareOption{name}{\HypDest@nametrue}
%    \end{macrocode}
%
%    \begin{macrocode}
\ProcessOptions*\relax
%    \end{macrocode}
%
% \subsection{Check requirements}
%
%    First \pdfTeX\ must running in PDF mode.
%    \begin{macrocode}
\RequirePackage{ifpdf}[2007/09/09]
\RequirePackage{pdftexcmds}[2007/11/11]
\ifpdf
\else
  \PackageError{hypdestopt}{%
    This package requires pdfTeX in PDF mode%
  }\@ehc
  \expandafter\endinput
\fi
%    \end{macrocode}
%    The version of \pdfTeX\ must not be too old, because
%    \cs{pdfescapehex} and \cs{pdfunescapehex} are used.
%    \begin{macrocode}
\begingroup\expandafter\expandafter\expandafter\endgroup
\expandafter\ifx\csname pdf@escapehex\endcsname\relax
  \PackageError{hypdestopt}{%
    This pdfTeX is too old, at least 1.30.0 is required%
  }\@ehc
  \expandafter\endinput
\fi
%    \end{macrocode}
%    Features of \eTeX\ are used, e.g. \cs{numexpr}.
%    \begin{macrocode}
\begingroup\expandafter\expandafter\expandafter\endgroup
\expandafter\ifx\csname numexpr\endcsname\relax
  \PackageError{hypdestopt}{%
    e-TeX features are missing%
  }\@ehc
  \expandafter\endinput
\fi
%    \end{macrocode}
%    Package \xpackage{alphalph} provides \cs{newalphalph} since
%    version 2006/05/30 v1.4.
%    \begin{macrocode}
\ifHypDest@name
  \RequirePackage{alphalph}[2006/05/30]%
\fi
%    \end{macrocode}
%    \begin{macrocode}
\RequirePackage{auxhook}[2009/12/14]
\RequirePackage{pdfescape}[2007/04/21]
%    \end{macrocode}
%
% \subsection{Preamble for auxiliary file}
%
%    Provide dummy definitions for the macros that are used in the
%    auxiliary files. If the package is used no longer, then these
%    commands will not generate errors.
%
%    \begin{macro}{\HypDest@PrependDocument}
%    We add our stuff in front of the \cs{AtBeginDocument} hook
%    to ensure that we are before \xpackage{hyperref}'s stuff.
%    \begin{macrocode}
\long\def\HypDest@PrependDocument#1{%
  \begingroup
    \toks\z@{#1}%
    \toks\tw@\expandafter{\@begindocumenthook}%
    \xdef\@begindocumenthook{\the\toks\z@\the\toks\tw@}%
  \endgroup
}
%    \end{macrocode}
%    \end{macro}
%    \begin{macrocode}
\AddLineBeginAux{%
  \string\providecommand{\string\HypDest@Use}[1]{}%
}
%    \end{macrocode}
%
% \subsection{Generation of destination names}
%
%    Counter |HypDest| is used for identifying destinations.
%    \begin{macrocode}
\newcounter{HypDest}
%    \end{macrocode}
%
%    \begin{macrocode}
\ifHypDest@name
%    \end{macrocode}
%
%    \begin{macro}{\HypDest@HexChar}
%    Destination names are generated by automatically
%    numbering with the help of package \xpackage{alphalph}.
%    \cs{HypDest@HexChar} converts a number of the range 1 until 252
%    into the hexadecimal representation of the string character.
%    \begin{macrocode}
  \def\HypDest@HexChar#1{%
    \ifcase#1\or
%    \end{macrocode}
%    Avoid zero byte because of C strings in PDF viewer
%    applications.
%    \begin{macrocode}
      01\or 02\or 03\or 04\or 05\or 06\or 07\or
%    \end{macrocode}
%    Omit carriage return (13/\verb|^^0d|).
%    It needs quoting, otherwise it would be converted
%    to line feed (10/\verb|^^0a|).
%    \begin{macrocode}
      08\or 09\or 0A\or 0B\or 0C\or 0E\or 0F\or
      10\or 11\or 12\or 13\or 14\or 15\or 16\or 17\or
      18\or 19\or 1A\or 1B\or 1C\or 1D\or 1E\or 1F\or
      20\or 21\or 22\or 23\or 24\or 25\or 26\or 27\or
%    \end{macrocode}
%    Omit left and right parentheses (40/\verb|^^28|, 41/\verb|^^39|),
%    they need quoting in general.
%    \begin{macrocode}
      2A\or 2B\or 2C\or 2D\or 2E\or 2F\or
      30\or 31\or 32\or 33\or 34\or 35\or 36\or 37\or
      38\or 39\or 3A\or 3B\or 3C\or 3D\or 3E\or 3F\or
      40\or 41\or 42\or 43\or 44\or 45\or 46\or 47\or
      48\or 49\or 4A\or 4B\or 4C\or 4D\or 4E\or 4F\or
      50\or 51\or 52\or 53\or 54\or 55\or 56\or 57\or
%    \end{macrocode}
%    Omit backslash (92/\verb|^^5C|), it needs quoting.
%    \begin{macrocode}
      58\or 59\or 5A\or 5B\or 5D\or 5E\or 5F\or
      60\or 61\or 62\or 63\or 64\or 65\or 66\or 67\or
      68\or 69\or 6A\or 6B\or 6C\or 6D\or 6E\or 6F\or
      70\or 71\or 72\or 73\or 74\or 75\or 76\or 77\or
      78\or 79\or 7A\or 7B\or 7C\or 7D\or 7E\or 7F\or
      80\or 81\or 82\or 83\or 84\or 85\or 86\or 87\or
      88\or 89\or 8A\or 8B\or 8C\or 8D\or 8E\or 8F\or
      90\or 91\or 92\or 93\or 94\or 95\or 96\or 97\or
      98\or 99\or 9A\or 9B\or 9C\or 9D\or 9E\or 9F\or
      A0\or A1\or A2\or A3\or A4\or A5\or A6\or A7\or
      A8\or A9\or AA\or AB\or AC\or AD\or AE\or AF\or
      B0\or B1\or B2\or B3\or B4\or B5\or B6\or B7\or
      B8\or B9\or BA\or BB\or BC\or BD\or BE\or BF\or
      C0\or C1\or C2\or C3\or C4\or C5\or C6\or C7\or
      C8\or C9\or CA\or CB\or CC\or CD\or CE\or CF\or
      D0\or D1\or D2\or D3\or D4\or D5\or D6\or D7\or
      D8\or D9\or DA\or DB\or DC\or DD\or DE\or DF\or
      E0\or E1\or E2\or E3\or E4\or E5\or E6\or E7\or
      E8\or E9\or EA\or EB\or EC\or ED\or EE\or EF\or
      F0\or F1\or F2\or F3\or F4\or F5\or F6\or F7\or
%    \end{macrocode}
%    Avoid 255 (0xFF) to get rid of a possible unicode
%    marker at the begin of the string.
%    \begin{macrocode}
      F8\or F9\or FA\or FB\or FC\or FD\or FE%
    \fi
  }%
%    \end{macrocode}
%    \end{macro}
%    \begin{macro}{HypDest@HexString}
%    Now package \xpackage{alphalph} comes into play.
%    \cs{HypDest@HexString} is defined and converts
%    a positive number into a string, given in hexadecimal
%    representation.
%    \begin{macrocode}
  \newalphalph\HypDest@HexString\HypDest@HexChar{250}%
%    \end{macrocode}
%    \end{macro}
%    \begin{macro}{\theHypDest}
%    For use, the hexadecimal string is converted back.
%    \begin{macrocode}
  \renewcommand*{\theHypDest}{%
    \pdf@unescapehex{\HypDest@HexString{\value{HypDest}}}%
  }%
%    \end{macrocode}
%    \end{macro}
%
%    With option \xoption{num} we use the number directly.
%    \begin{macrocode}
\else
  \renewcommand*{\theHypDest}{%
    \number\value{HypDest}%
  }%
\fi
%    \end{macrocode}
%
% \subsection{Assign destination names}
%
%    \begin{macro}{\HypDest@Prefix}
%    The new destination names are remembered in macros whose names
%    start with prefix \cs{HypDest@Prefix}.
%    \begin{macrocode}
\edef\HypDest@Prefix{HypDest\string:}
%    \end{macrocode}
%    \end{macro}
%
%    \begin{macro}{\HypDest@Use}
%    During the first read of the auxiliary files, the used destinations
%    get fresh generated short destination names. Also for the old
%    destination names we use the hexadecimal representation. That
%    avoid problems with arbitrary names.
%    \begin{macrocode}
\def\HypDest@Use#1{%
  \begingroup
    \edef\x{%
      \expandafter\noexpand
      \csname\HypDest@Prefix\pdf@unescapehex{#1}\endcsname
    }%
    \expandafter\ifx\x\relax
      \stepcounter{HypDest}%
      \expandafter\xdef\x{\theHypDest}%
      \let\on@line\@empty
      \ifHypDest@name
        \HypDest@VerboseInfo{%
          Use: (\pdf@unescapehex{#1}) -\string> %
          0x\pdf@escapehex{\x} (\number\value{HypDest})%
        }%
      \else
        \HypDest@VerboseInfo{%
          Use: (\pdf@unescapehex{#1}) -\string> num \x
        }%
      \fi
    \fi
  \endgroup
}
%    \end{macrocode}
%    \end{macro}
%
%    After the first \xfile{.aux} file processing the destination names
%    are assigned and we can disable \cs{HypDest@Use}.
%    \begin{macrocode}
\AtBeginDocument{%
  \let\HypDest@Use\@gobble
}
%    \end{macrocode}
%
%    \begin{macro}{\HypDest@MarkUsed}
%    Destinations that are actually used are marked by \cs{HypDest@MarkUsed}.
%    \cs{nofiles} is respected.
%    \begin{macrocode}
\def\HypDest@MarkUsed#1{%
  \HypDest@VerboseInfo{%
    MarkUsed: (#1)%
  }%
  \if@filesw
    \immediate\write\@auxout{%
      \string\HypDest@Use{\pdf@escapehex{#1}}%
    }%
  \fi
}%
%    \end{macrocode}
%    \end{macro}
%
% \subsection{Redefinition of \xpackage{hyperref}'s hooks}
%
%    Package \xpackage{hyperref} can be loaded later, therefore
%    we redefine \xpackage{hyperref}'s macros at |\begin{document}|.
%    \begin{macrocode}
\HypDest@PrependDocument{%
%    \end{macrocode}
%
%    Check hyperref version.
%    \begin{macrocode}
  \@ifpackagelater{hyperref}{2006/06/01}{}{%
    \PackageError{hypdestopt}{%
      hyperref 2006/06/01 v6.75a or later is required%
    }\@ehc
  }%
%    \end{macrocode}
%
% \subsubsection{Destination setting}
%
%    \begin{macrocode}
  \ifHypDest@name
    \let\HypDest@Org@DestName\Hy@DestName
    \renewcommand*{\Hy@DestName}[2]{%
      \EdefUnescapeString\HypDest@temp{#1}%
      \@ifundefined{\HypDest@Prefix\HypDest@temp}{%
        \HypDest@VerboseInfo{%
          DestName: (\HypDest@temp) unused%
        }%
      }{%
        \HypDest@Org@DestName{%
          \csname\HypDest@Prefix\HypDest@temp\endcsname
        }{#2}%
        \HypDest@VerboseInfo{%
          DestName: (\HypDest@temp) %
          0x\pdf@escapehex{%
            \csname\HypDest@Prefix\HypDest@temp\endcsname
          }%
        }%
      }%
    }%
  \else
    \renewcommand*{\Hy@DestName}[2]{%
      \EdefUnescapeString\HypDest@temp{#1}%
      \@ifundefined{\HypDest@Prefix\HypDest@temp}{%
        \HypDest@VerboseInfo{%
          DestName: (\HypDest@temp) unused%
        }%
      }{%
        \pdfdest num%
        \csname\HypDest@Prefix\HypDest@temp\endcsname#2\relax
        \HypDest@VerboseInfo{%
          DestName: (\HypDest@temp) %
          num \csname\HypDest@Prefix\HypDest@temp\endcsname
        }%
      }%
    }%
  \fi
%    \end{macrocode}
%
% \subsubsection{Links}
%
%    \begin{macrocode}
  \let\HypDest@Org@StartlinkName\Hy@StartlinkName
  \ifHypDest@name
    \renewcommand*{\Hy@StartlinkName}[2]{%
      \HypDest@MarkUsed{#2}%
      \HypDest@Org@StartlinkName{#1}{%
        \@ifundefined{\HypDest@Prefix#2}{%
          #2%
        }{%
          \csname\HypDest@Prefix#2\endcsname
        }%
      }%
    }%
  \else
    \renewcommand*{\Hy@StartlinkName}[2]{%
      \HypDest@MarkUsed{#2}%
      \@ifundefined{\HypDest@Prefix#2}{%
        \HypDest@Org@StartlinkName{#1}{#2}%
      }{%
        \pdfstartlink attr{#1}%
                      goto num\csname\HypDest@Prefix#2\endcsname
        \relax
      }%
    }%
  \fi
%    \end{macrocode}
%
% \subsubsection{Outlines of package \xpackage{hyperref}}
%
%    \begin{macrocode}
  \let\HypDest@Org@OutlineName\Hy@OutlineName
  \ifHypDest@name
    \renewcommand*{\Hy@OutlineName}[4]{%
      \HypDest@Org@OutlineName{#1}{%
        \@ifundefined{\HypDest@Prefix#2}{%
          #2%
        }{%
          \csname\HypDest@Prefix#2\endcsname
        }%
      }{#3}{#4}%
    }%
  \else
    \renewcommand*{\Hy@OutlineName}[4]{%
      \@ifundefined{\HypDest@Prefix#2}{%
        \HypDest@Org@OutlineName{#1}{#2}{#3}{#4}%
      }{%
        \pdfoutline goto num\csname\HypDest@Prefix#2\endcsname
                    count#3{#4}%
      }%
    }%
  \fi
%    \end{macrocode}
%    Because \cs{Hy@OutlineName} is called after the \xfile{.out} file
%    is written in the previous run. Therefore we mark the destination
%    earlier in \cs{@@writetorep}.
%    \begin{macrocode}
  \let\HypDest@Org@@writetorep\@@writetorep
  \renewcommand*{\@@writetorep}[5]{%
    \begingroup
      \edef\Hy@tempa{#5}%
      \ifx\Hy@tempa\Hy@bookmarkstype
        \HypDest@MarkUsed{#3}%
      \fi
    \endgroup
    \HypDest@Org@@writetorep{#1}{#2}{#3}{#4}{#5}%
  }%
%    \end{macrocode}
%
% \subsubsection{Outlines of package \xpackage{bookmark}}
%
%    \begin{macrocode}
  \@ifpackageloaded{bookmark}{%
    \@ifpackagelater{bookmark}{2008/08/08}{%
      \renewcommand*{\BKM@DefGotoNameAction}[2]{%
        \@ifundefined{\HypDest@Prefix#2}{%
          \edef#1{goto name{hypdestopt\string :unknown}}%
        }{%
          \ifHypDest@name
            \edef#1{goto name{\csname\HypDest@Prefix#2\endcsname}}%
          \else
            \edef#1{goto num\csname\HypDest@Prefix#2\endcsname}%
          \fi
        }%
      }%
      \def\BKM@HypDestOptHook{%
        \ifx\BKM@dest\@empty
        \else
          \ifx\BKM@gotor\@empty
            \HypDest@MarkUsed\BKM@dest
          \fi
        \fi
      }%
    }{%
      \@PackageError{hypdestopt}{%
        Package `bookmark' is too old.\MessageBreak
        Version 2008/08/08 or later is needed%
      }\@ehc
    }%
  }{}%
%    \end{macrocode}
%
%    \begin{macrocode}
}
%    \end{macrocode}
%
%
%    \begin{macrocode}
%</package>
%    \end{macrocode}
%
% \section{Installation}
%
% \subsection{Download}
%
% \paragraph{Package.} This package is available on
% CTAN\footnote{\url{ftp://ftp.ctan.org/tex-archive/}}:
% \begin{description}
% \item[\CTAN{macros/latex/contrib/oberdiek/hypdestopt.dtx}] The source file.
% \item[\CTAN{macros/latex/contrib/oberdiek/hypdestopt.pdf}] Documentation.
% \end{description}
%
%
% \paragraph{Bundle.} All the packages of the bundle `oberdiek'
% are also available in a TDS compliant ZIP archive. There
% the packages are already unpacked and the documentation files
% are generated. The files and directories obey the TDS standard.
% \begin{description}
% \item[\CTAN{install/macros/latex/contrib/oberdiek.tds.zip}]
% \end{description}
% \emph{TDS} refers to the standard ``A Directory Structure
% for \TeX\ Files'' (\CTAN{tds/tds.pdf}). Directories
% with \xfile{texmf} in their name are usually organized this way.
%
% \subsection{Bundle installation}
%
% \paragraph{Unpacking.} Unpack the \xfile{oberdiek.tds.zip} in the
% TDS tree (also known as \xfile{texmf} tree) of your choice.
% Example (linux):
% \begin{quote}
%   |unzip oberdiek.tds.zip -d ~/texmf|
% \end{quote}
%
% \paragraph{Script installation.}
% Check the directory \xfile{TDS:scripts/oberdiek/} for
% scripts that need further installation steps.
% Package \xpackage{attachfile2} comes with the Perl script
% \xfile{pdfatfi.pl} that should be installed in such a way
% that it can be called as \texttt{pdfatfi}.
% Example (linux):
% \begin{quote}
%   |chmod +x scripts/oberdiek/pdfatfi.pl|\\
%   |cp scripts/oberdiek/pdfatfi.pl /usr/local/bin/|
% \end{quote}
%
% \subsection{Package installation}
%
% \paragraph{Unpacking.} The \xfile{.dtx} file is a self-extracting
% \docstrip\ archive. The files are extracted by running the
% \xfile{.dtx} through \plainTeX:
% \begin{quote}
%   \verb|tex hypdestopt.dtx|
% \end{quote}
%
% \paragraph{TDS.} Now the different files must be moved into
% the different directories in your installation TDS tree
% (also known as \xfile{texmf} tree):
% \begin{quote}
% \def\t{^^A
% \begin{tabular}{@{}>{\ttfamily}l@{ $\rightarrow$ }>{\ttfamily}l@{}}
%   hypdestopt.sty & tex/latex/oberdiek/hypdestopt.sty\\
%   hypdestopt.pdf & doc/latex/oberdiek/hypdestopt.pdf\\
%   hypdestopt.dtx & source/latex/oberdiek/hypdestopt.dtx\\
% \end{tabular}^^A
% }^^A
% \sbox0{\t}^^A
% \ifdim\wd0>\linewidth
%   \begingroup
%     \advance\linewidth by\leftmargin
%     \advance\linewidth by\rightmargin
%   \edef\x{\endgroup
%     \def\noexpand\lw{\the\linewidth}^^A
%   }\x
%   \def\lwbox{^^A
%     \leavevmode
%     \hbox to \linewidth{^^A
%       \kern-\leftmargin\relax
%       \hss
%       \usebox0
%       \hss
%       \kern-\rightmargin\relax
%     }^^A
%   }^^A
%   \ifdim\wd0>\lw
%     \sbox0{\small\t}^^A
%     \ifdim\wd0>\linewidth
%       \ifdim\wd0>\lw
%         \sbox0{\footnotesize\t}^^A
%         \ifdim\wd0>\linewidth
%           \ifdim\wd0>\lw
%             \sbox0{\scriptsize\t}^^A
%             \ifdim\wd0>\linewidth
%               \ifdim\wd0>\lw
%                 \sbox0{\tiny\t}^^A
%                 \ifdim\wd0>\linewidth
%                   \lwbox
%                 \else
%                   \usebox0
%                 \fi
%               \else
%                 \lwbox
%               \fi
%             \else
%               \usebox0
%             \fi
%           \else
%             \lwbox
%           \fi
%         \else
%           \usebox0
%         \fi
%       \else
%         \lwbox
%       \fi
%     \else
%       \usebox0
%     \fi
%   \else
%     \lwbox
%   \fi
% \else
%   \usebox0
% \fi
% \end{quote}
% If you have a \xfile{docstrip.cfg} that configures and enables \docstrip's
% TDS installing feature, then some files can already be in the right
% place, see the documentation of \docstrip.
%
% \subsection{Refresh file name databases}
%
% If your \TeX~distribution
% (\teTeX, \mikTeX, \dots) relies on file name databases, you must refresh
% these. For example, \teTeX\ users run \verb|texhash| or
% \verb|mktexlsr|.
%
% \subsection{Some details for the interested}
%
% \paragraph{Attached source.}
%
% The PDF documentation on CTAN also includes the
% \xfile{.dtx} source file. It can be extracted by
% AcrobatReader 6 or higher. Another option is \textsf{pdftk},
% e.g. unpack the file into the current directory:
% \begin{quote}
%   \verb|pdftk hypdestopt.pdf unpack_files output .|
% \end{quote}
%
% \paragraph{Unpacking with \LaTeX.}
% The \xfile{.dtx} chooses its action depending on the format:
% \begin{description}
% \item[\plainTeX:] Run \docstrip\ and extract the files.
% \item[\LaTeX:] Generate the documentation.
% \end{description}
% If you insist on using \LaTeX\ for \docstrip\ (really,
% \docstrip\ does not need \LaTeX), then inform the autodetect routine
% about your intention:
% \begin{quote}
%   \verb|latex \let\install=y\input{hypdestopt.dtx}|
% \end{quote}
% Do not forget to quote the argument according to the demands
% of your shell.
%
% \paragraph{Generating the documentation.}
% You can use both the \xfile{.dtx} or the \xfile{.drv} to generate
% the documentation. The process can be configured by the
% configuration file \xfile{ltxdoc.cfg}. For instance, put this
% line into this file, if you want to have A4 as paper format:
% \begin{quote}
%   \verb|\PassOptionsToClass{a4paper}{article}|
% \end{quote}
% An example follows how to generate the
% documentation with pdf\LaTeX:
% \begin{quote}
%\begin{verbatim}
%pdflatex hypdestopt.dtx
%makeindex -s gind.ist hypdestopt.idx
%pdflatex hypdestopt.dtx
%makeindex -s gind.ist hypdestopt.idx
%pdflatex hypdestopt.dtx
%\end{verbatim}
% \end{quote}
%
% \section{Catalogue}
%
% The following XML file can be used as source for the
% \href{http://mirror.ctan.org/help/Catalogue/catalogue.html}{\TeX\ Catalogue}.
% The elements \texttt{caption} and \texttt{description} are imported
% from the original XML file from the Catalogue.
% The name of the XML file in the Catalogue is \xfile{hypdestopt.xml}.
%    \begin{macrocode}
%<*catalogue>
<?xml version='1.0' encoding='us-ascii'?>
<!DOCTYPE entry SYSTEM 'catalogue.dtd'>
<entry datestamp='$Date$' modifier='$Author$' id='hypdestopt'>
  <name>hypdestopt</name>
  <caption>Hyperref destination optimizer.</caption>
  <authorref id='auth:oberdiek'/>
  <copyright owner='Heiko Oberdiek' year='2006-2008,2011'/>
  <license type='lppl1.3'/>
  <version number='2.3'/>
  <description>
    This package supports <xref refid='hyperref'>hyperref</xref>'s
    pdftex driver. It removes unnecessary destinations
    and shortens the destination names or uses numbered destinations
    to get smaller PDF files.
    <p/>
    The package is part of the <xref refid='oberdiek'>oberdiek</xref>
    bundle.
  </description>
  <documentation details='Package documentation'
      href='ctan:/macros/latex/contrib/oberdiek/hypdestopt.pdf'/>
  <ctan file='true' path='/macros/latex/contrib/oberdiek/hypdestopt.dtx'/>
  <miktex location='oberdiek'/>
  <texlive location='oberdiek'/>
  <install path='/macros/latex/contrib/oberdiek/oberdiek.tds.zip'/>
</entry>
%</catalogue>
%    \end{macrocode}
%
% \begin{thebibliography}{9}
%
% \bibitem{alphalph}
%   Heiko Oberdiek: \textit{The \xpackage{alphalph} package};
%   2006/05/30 v1.4;
%   \CTAN{macros/latex/contrib/oberdiek/alphalph.pdf}.
%
% \bibitem{hyperref}
%   Sebastian Rahtz, Heiko Oberdiek:
%   \textit{The \xpackage{hyperref} package};
%   2006/06/01 v6.75a;
%   \CTAN{macros/latex/contrib/hyperref/}.
%
% \bibitem{ifpdf}
%   Heiko Oberdiek: \textit{The \xpackage{ifpdf} package};
%   2006/02/20 v1.4;
%   \CTAN{macros/latex/contrib/oberdiek/ifpdf.pdf}.
%
% \end{thebibliography}
%
% \begin{History}
%   \begin{Version}{2006/06/01 v1.0}
%   \item
%     First version.
%   \end{Version}
%   \begin{Version}{2006/06/01 v2.0}
%   \item
%     New method for referencing destinations by number; an idea
%     proposed by Lars Hellstr\"om in the mailing list LATEX-L.
%   \item
%     Options \xoption{name} and \xoption{num} added.
%   \end{Version}
%   \begin{Version}{2007/11/11 v2.1}
%   \item
%     Use of package \xpackage{pdftexcmds} for \LuaTeX\ support.
%   \end{Version}
%   \begin{Version}{2008/08/08 v2.2}
%   \item
%     Support for package \xpackage{bookmark} added.
%   \end{Version}
%   \begin{Version}{2011/05/13 v2.3}
%   \item
%     Fix for \cs{Hy@DestName} if the destination name contains
%     special characters.
%   \item
%     Fix for option \xoption{name} and package \xpackage{bookmark}.
%   \end{Version}
% \end{History}
%
% \PrintIndex
%
% \Finale
\endinput

%        (quote the arguments according to the demands of your shell)
%
% Documentation:
%    (a) If hypdestopt.drv is present:
%           latex hypdestopt.drv
%    (b) Without hypdestopt.drv:
%           latex hypdestopt.dtx; ...
%    The class ltxdoc loads the configuration file ltxdoc.cfg
%    if available. Here you can specify further options, e.g.
%    use A4 as paper format:
%       \PassOptionsToClass{a4paper}{article}
%
%    Programm calls to get the documentation (example):
%       pdflatex hypdestopt.dtx
%       makeindex -s gind.ist hypdestopt.idx
%       pdflatex hypdestopt.dtx
%       makeindex -s gind.ist hypdestopt.idx
%       pdflatex hypdestopt.dtx
%
% Installation:
%    TDS:tex/latex/oberdiek/hypdestopt.sty
%    TDS:doc/latex/oberdiek/hypdestopt.pdf
%    TDS:source/latex/oberdiek/hypdestopt.dtx
%
%<*ignore>
\begingroup
  \catcode123=1 %
  \catcode125=2 %
  \def\x{LaTeX2e}%
\expandafter\endgroup
\ifcase 0\ifx\install y1\fi\expandafter
         \ifx\csname processbatchFile\endcsname\relax\else1\fi
         \ifx\fmtname\x\else 1\fi\relax
\else\csname fi\endcsname
%</ignore>
%<*install>
\input docstrip.tex
\Msg{************************************************************************}
\Msg{* Installation}
\Msg{* Package: hypdestopt 2011/05/13 v2.3 Hyperref destination optimizer (HO)}
\Msg{************************************************************************}

\keepsilent
\askforoverwritefalse

\let\MetaPrefix\relax
\preamble

This is a generated file.

Project: hypdestopt
Version: 2011/05/13 v2.3

Copyright (C) 2006-2008, 2011 by
   Heiko Oberdiek <heiko.oberdiek at googlemail.com>

This work may be distributed and/or modified under the
conditions of the LaTeX Project Public License, either
version 1.3c of this license or (at your option) any later
version. This version of this license is in
   http://www.latex-project.org/lppl/lppl-1-3c.txt
and the latest version of this license is in
   http://www.latex-project.org/lppl.txt
and version 1.3 or later is part of all distributions of
LaTeX version 2005/12/01 or later.

This work has the LPPL maintenance status "maintained".

This Current Maintainer of this work is Heiko Oberdiek.

This work consists of the main source file hypdestopt.dtx
and the derived files
   hypdestopt.sty, hypdestopt.pdf, hypdestopt.ins, hypdestopt.drv.

\endpreamble
\let\MetaPrefix\DoubleperCent

\generate{%
  \file{hypdestopt.ins}{\from{hypdestopt.dtx}{install}}%
  \file{hypdestopt.drv}{\from{hypdestopt.dtx}{driver}}%
  \usedir{tex/latex/oberdiek}%
  \file{hypdestopt.sty}{\from{hypdestopt.dtx}{package}}%
  \nopreamble
  \nopostamble
  \usedir{source/latex/oberdiek/catalogue}%
  \file{hypdestopt.xml}{\from{hypdestopt.dtx}{catalogue}}%
}

\catcode32=13\relax% active space
\let =\space%
\Msg{************************************************************************}
\Msg{*}
\Msg{* To finish the installation you have to move the following}
\Msg{* file into a directory searched by TeX:}
\Msg{*}
\Msg{*     hypdestopt.sty}
\Msg{*}
\Msg{* To produce the documentation run the file `hypdestopt.drv'}
\Msg{* through LaTeX.}
\Msg{*}
\Msg{* Happy TeXing!}
\Msg{*}
\Msg{************************************************************************}

\endbatchfile
%</install>
%<*ignore>
\fi
%</ignore>
%<*driver>
\NeedsTeXFormat{LaTeX2e}
\ProvidesFile{hypdestopt.drv}%
  [2011/05/13 v2.3 Hyperref destination optimizer (HO)]%
\documentclass{ltxdoc}
\usepackage{holtxdoc}[2011/11/22]
\begin{document}
  \DocInput{hypdestopt.dtx}%
\end{document}
%</driver>
% \fi
%
% \CheckSum{565}
%
% \CharacterTable
%  {Upper-case    \A\B\C\D\E\F\G\H\I\J\K\L\M\N\O\P\Q\R\S\T\U\V\W\X\Y\Z
%   Lower-case    \a\b\c\d\e\f\g\h\i\j\k\l\m\n\o\p\q\r\s\t\u\v\w\x\y\z
%   Digits        \0\1\2\3\4\5\6\7\8\9
%   Exclamation   \!     Double quote  \"     Hash (number) \#
%   Dollar        \$     Percent       \%     Ampersand     \&
%   Acute accent  \'     Left paren    \(     Right paren   \)
%   Asterisk      \*     Plus          \+     Comma         \,
%   Minus         \-     Point         \.     Solidus       \/
%   Colon         \:     Semicolon     \;     Less than     \<
%   Equals        \=     Greater than  \>     Question mark \?
%   Commercial at \@     Left bracket  \[     Backslash     \\
%   Right bracket \]     Circumflex    \^     Underscore    \_
%   Grave accent  \`     Left brace    \{     Vertical bar  \|
%   Right brace   \}     Tilde         \~}
%
% \GetFileInfo{hypdestopt.drv}
%
% \title{The \xpackage{hypdestopt} package}
% \date{2011/05/13 v2.3}
% \author{Heiko Oberdiek\\\xemail{heiko.oberdiek at googlemail.com}}
%
% \maketitle
%
% \begin{abstract}
% Package \xpackage{hypdestopt} supports \xpackage{hyperref}'s
% \xoption{pdftex} driver. It removes unnecessary destinations
% and shortens the destination names or uses numbered destinations
% to get smaller PDF files.
% \end{abstract}
%
% \tableofcontents
%
% \section{User interface}
%
% \subsection{Introduction}
%
% Before PDF-1.5 annotations and destinations cannot be compressed.
% If the destination names are not needed for external use, the
% file size can be decreased by the following means:
% \begin{itemize}
% \item Unused destinations are removed.
% \item The destination names are shortened (option \xoption{name}).
% \item Using numbered destinations (option \xoption{num}).
% \end{itemize}
%
% \subsection{Requirements}
%
% \begin{itemize}
% \item Package \xpackage{hyperref} 2006/06/01 v6.75a or newer
%       (\cite{hyperref}).
% \item Package \xpackage{alphalph} 2006/05/30 v1.4 or newer
%       (\cite{alphalph}), if option \xoption{name} is used.
% \item Package \xpackage{ifpdf} (\cite{ifpdf}).
% \item \pdfTeX\ 1.30.0 or newer.
% \item \pdfTeX\ in PDF mode.
% \item \eTeX\ extensions enabled.
% \item Probably an additional compile run of \pdfLaTeX\ is necessary.
% \end{itemize}
%
% In the first compile runs you can get warnings such as:
%\begin{quote}
%\begin{verbatim}
%! pdfTeX warning (dest): name{...} has been referenced ...
%\end{verbatim}
%\end{quote}
% These warnings should vanish in later compile runs.
% However these warnings also can occur without this package.
% The package does not cure them, thus these warnings will remain,
% but the destination name can be different. In such cases test
% without package, too.
%
% \subsection{Use}
%
% If the requirements are met, load the package:
%\begin{quote}
%\verb|\usepackage{hypdestopt}|
%\end{quote}
%
% The following options are supported:
% \begin{description}
% \item[\xoption{verbose}:] Verbose debug output is enabled and written
%   in the protocol file.
% \item[\xoption{num}:] Numbered destinations are used. The file size
%   is smaller, because names are no longer used.
%   This is the default.
% \item[\xoption{name}:] Destinations are identified by names.
% \end{description}
%
% \subsection{Limitations}
%
% \begin{itemize}
% \item Forget this package, if you need preserved destination names.
% \item Destination name strings use all bytes (0..255) except
%       the carriage return (13), left parenthesis (40), right
%       parenthesis (41), and backslash (92), because they
%       must be quoted in general and therefore occupy two bytes
%       instead of one.
%
%       Further the zero byte (0) is avoided for programs
%       that implement strings using zero terminated C strings.
%       And 255 (0xFF) is avoided to get rid of a possible
%       unicode marker at the begin.
%
%       So far I have not seen problems with:
%       \begin{itemize}
%       \item AcrobatReader 5.08/Linux
%       \item AcrobatReader 7.0/Linux
%       \item xpdf 3.00
%       \item Ghostscript 8.50
%       \item gv 3.5.8
%       \item GSview 4.6
%       \end{itemize}
%       But I have not tested all and all possible PDF viewers.
% \item Use of named destinations (\cs{pdfdest}, \cs{pdfoutline},
%       \cs{pdfstartlink}, \dots) that are not supported by this
%       package.
% \item Currently only \xpackage{hyperref} with \pdfTeX\ in PDF
%       mode is supported.
% \end{itemize}
%
% \subsection{Future}
%
% A more general approach is a PDF postprocessor that takes
% a PDF file, performs some transformations and writes the
% result in a more optimized PDF file. Then it does not depend,
% how the original PDF file was generated and further improvements
% are easier to apply. For example, the destination names could be sorted:
% often used destination names would then be shorter than seldom used ones.
%
% \StopEventually{
% }
%
% \section{Implementation}
%
% \subsection{Identification}
%
%    \begin{macrocode}
%<*package>
\NeedsTeXFormat{LaTeX2e}
\ProvidesPackage{hypdestopt}%
  [2011/05/13 v2.3 Hyperref destination optimizer (HO)]%
%    \end{macrocode}
%
% \subsection{Options}
%
% \subsubsection{Option \xoption{verbose}}
%
%    \begin{macrocode}
\newif\ifHypDest@Verbose
\DeclareOption{verbose}{\HypDest@Verbosetrue}
%    \end{macrocode}
%
%    \begin{macro}{\HypDest@VerboseInfo}
%    Wrapper for verbose messages.
%    \begin{macrocode}
\def\HypDest@VerboseInfo#1{%
  \ifHypDest@Verbose
    \PackageInfo{hypdestopt}{#1}%
  \fi
}
%    \end{macrocode}
%    \end{macro}
%
% \subsubsection{Options \xoption{num} and \xoption{name}}
%
%    The options \xoption{num} or \xoption{name} specify
%    the method, how destinations are referenced (by name or
%    number). Default is option \xoption{num}.
%    \begin{macrocode}
\newif\ifHypDest@name
\DeclareOption{num}{\HypDest@namefalse}
\DeclareOption{name}{\HypDest@nametrue}
%    \end{macrocode}
%
%    \begin{macrocode}
\ProcessOptions*\relax
%    \end{macrocode}
%
% \subsection{Check requirements}
%
%    First \pdfTeX\ must running in PDF mode.
%    \begin{macrocode}
\RequirePackage{ifpdf}[2007/09/09]
\RequirePackage{pdftexcmds}[2007/11/11]
\ifpdf
\else
  \PackageError{hypdestopt}{%
    This package requires pdfTeX in PDF mode%
  }\@ehc
  \expandafter\endinput
\fi
%    \end{macrocode}
%    The version of \pdfTeX\ must not be too old, because
%    \cs{pdfescapehex} and \cs{pdfunescapehex} are used.
%    \begin{macrocode}
\begingroup\expandafter\expandafter\expandafter\endgroup
\expandafter\ifx\csname pdf@escapehex\endcsname\relax
  \PackageError{hypdestopt}{%
    This pdfTeX is too old, at least 1.30.0 is required%
  }\@ehc
  \expandafter\endinput
\fi
%    \end{macrocode}
%    Features of \eTeX\ are used, e.g. \cs{numexpr}.
%    \begin{macrocode}
\begingroup\expandafter\expandafter\expandafter\endgroup
\expandafter\ifx\csname numexpr\endcsname\relax
  \PackageError{hypdestopt}{%
    e-TeX features are missing%
  }\@ehc
  \expandafter\endinput
\fi
%    \end{macrocode}
%    Package \xpackage{alphalph} provides \cs{newalphalph} since
%    version 2006/05/30 v1.4.
%    \begin{macrocode}
\ifHypDest@name
  \RequirePackage{alphalph}[2006/05/30]%
\fi
%    \end{macrocode}
%    \begin{macrocode}
\RequirePackage{auxhook}[2009/12/14]
\RequirePackage{pdfescape}[2007/04/21]
%    \end{macrocode}
%
% \subsection{Preamble for auxiliary file}
%
%    Provide dummy definitions for the macros that are used in the
%    auxiliary files. If the package is used no longer, then these
%    commands will not generate errors.
%
%    \begin{macro}{\HypDest@PrependDocument}
%    We add our stuff in front of the \cs{AtBeginDocument} hook
%    to ensure that we are before \xpackage{hyperref}'s stuff.
%    \begin{macrocode}
\long\def\HypDest@PrependDocument#1{%
  \begingroup
    \toks\z@{#1}%
    \toks\tw@\expandafter{\@begindocumenthook}%
    \xdef\@begindocumenthook{\the\toks\z@\the\toks\tw@}%
  \endgroup
}
%    \end{macrocode}
%    \end{macro}
%    \begin{macrocode}
\AddLineBeginAux{%
  \string\providecommand{\string\HypDest@Use}[1]{}%
}
%    \end{macrocode}
%
% \subsection{Generation of destination names}
%
%    Counter |HypDest| is used for identifying destinations.
%    \begin{macrocode}
\newcounter{HypDest}
%    \end{macrocode}
%
%    \begin{macrocode}
\ifHypDest@name
%    \end{macrocode}
%
%    \begin{macro}{\HypDest@HexChar}
%    Destination names are generated by automatically
%    numbering with the help of package \xpackage{alphalph}.
%    \cs{HypDest@HexChar} converts a number of the range 1 until 252
%    into the hexadecimal representation of the string character.
%    \begin{macrocode}
  \def\HypDest@HexChar#1{%
    \ifcase#1\or
%    \end{macrocode}
%    Avoid zero byte because of C strings in PDF viewer
%    applications.
%    \begin{macrocode}
      01\or 02\or 03\or 04\or 05\or 06\or 07\or
%    \end{macrocode}
%    Omit carriage return (13/\verb|^^0d|).
%    It needs quoting, otherwise it would be converted
%    to line feed (10/\verb|^^0a|).
%    \begin{macrocode}
      08\or 09\or 0A\or 0B\or 0C\or 0E\or 0F\or
      10\or 11\or 12\or 13\or 14\or 15\or 16\or 17\or
      18\or 19\or 1A\or 1B\or 1C\or 1D\or 1E\or 1F\or
      20\or 21\or 22\or 23\or 24\or 25\or 26\or 27\or
%    \end{macrocode}
%    Omit left and right parentheses (40/\verb|^^28|, 41/\verb|^^39|),
%    they need quoting in general.
%    \begin{macrocode}
      2A\or 2B\or 2C\or 2D\or 2E\or 2F\or
      30\or 31\or 32\or 33\or 34\or 35\or 36\or 37\or
      38\or 39\or 3A\or 3B\or 3C\or 3D\or 3E\or 3F\or
      40\or 41\or 42\or 43\or 44\or 45\or 46\or 47\or
      48\or 49\or 4A\or 4B\or 4C\or 4D\or 4E\or 4F\or
      50\or 51\or 52\or 53\or 54\or 55\or 56\or 57\or
%    \end{macrocode}
%    Omit backslash (92/\verb|^^5C|), it needs quoting.
%    \begin{macrocode}
      58\or 59\or 5A\or 5B\or 5D\or 5E\or 5F\or
      60\or 61\or 62\or 63\or 64\or 65\or 66\or 67\or
      68\or 69\or 6A\or 6B\or 6C\or 6D\or 6E\or 6F\or
      70\or 71\or 72\or 73\or 74\or 75\or 76\or 77\or
      78\or 79\or 7A\or 7B\or 7C\or 7D\or 7E\or 7F\or
      80\or 81\or 82\or 83\or 84\or 85\or 86\or 87\or
      88\or 89\or 8A\or 8B\or 8C\or 8D\or 8E\or 8F\or
      90\or 91\or 92\or 93\or 94\or 95\or 96\or 97\or
      98\or 99\or 9A\or 9B\or 9C\or 9D\or 9E\or 9F\or
      A0\or A1\or A2\or A3\or A4\or A5\or A6\or A7\or
      A8\or A9\or AA\or AB\or AC\or AD\or AE\or AF\or
      B0\or B1\or B2\or B3\or B4\or B5\or B6\or B7\or
      B8\or B9\or BA\or BB\or BC\or BD\or BE\or BF\or
      C0\or C1\or C2\or C3\or C4\or C5\or C6\or C7\or
      C8\or C9\or CA\or CB\or CC\or CD\or CE\or CF\or
      D0\or D1\or D2\or D3\or D4\or D5\or D6\or D7\or
      D8\or D9\or DA\or DB\or DC\or DD\or DE\or DF\or
      E0\or E1\or E2\or E3\or E4\or E5\or E6\or E7\or
      E8\or E9\or EA\or EB\or EC\or ED\or EE\or EF\or
      F0\or F1\or F2\or F3\or F4\or F5\or F6\or F7\or
%    \end{macrocode}
%    Avoid 255 (0xFF) to get rid of a possible unicode
%    marker at the begin of the string.
%    \begin{macrocode}
      F8\or F9\or FA\or FB\or FC\or FD\or FE%
    \fi
  }%
%    \end{macrocode}
%    \end{macro}
%    \begin{macro}{HypDest@HexString}
%    Now package \xpackage{alphalph} comes into play.
%    \cs{HypDest@HexString} is defined and converts
%    a positive number into a string, given in hexadecimal
%    representation.
%    \begin{macrocode}
  \newalphalph\HypDest@HexString\HypDest@HexChar{250}%
%    \end{macrocode}
%    \end{macro}
%    \begin{macro}{\theHypDest}
%    For use, the hexadecimal string is converted back.
%    \begin{macrocode}
  \renewcommand*{\theHypDest}{%
    \pdf@unescapehex{\HypDest@HexString{\value{HypDest}}}%
  }%
%    \end{macrocode}
%    \end{macro}
%
%    With option \xoption{num} we use the number directly.
%    \begin{macrocode}
\else
  \renewcommand*{\theHypDest}{%
    \number\value{HypDest}%
  }%
\fi
%    \end{macrocode}
%
% \subsection{Assign destination names}
%
%    \begin{macro}{\HypDest@Prefix}
%    The new destination names are remembered in macros whose names
%    start with prefix \cs{HypDest@Prefix}.
%    \begin{macrocode}
\edef\HypDest@Prefix{HypDest\string:}
%    \end{macrocode}
%    \end{macro}
%
%    \begin{macro}{\HypDest@Use}
%    During the first read of the auxiliary files, the used destinations
%    get fresh generated short destination names. Also for the old
%    destination names we use the hexadecimal representation. That
%    avoid problems with arbitrary names.
%    \begin{macrocode}
\def\HypDest@Use#1{%
  \begingroup
    \edef\x{%
      \expandafter\noexpand
      \csname\HypDest@Prefix\pdf@unescapehex{#1}\endcsname
    }%
    \expandafter\ifx\x\relax
      \stepcounter{HypDest}%
      \expandafter\xdef\x{\theHypDest}%
      \let\on@line\@empty
      \ifHypDest@name
        \HypDest@VerboseInfo{%
          Use: (\pdf@unescapehex{#1}) -\string> %
          0x\pdf@escapehex{\x} (\number\value{HypDest})%
        }%
      \else
        \HypDest@VerboseInfo{%
          Use: (\pdf@unescapehex{#1}) -\string> num \x
        }%
      \fi
    \fi
  \endgroup
}
%    \end{macrocode}
%    \end{macro}
%
%    After the first \xfile{.aux} file processing the destination names
%    are assigned and we can disable \cs{HypDest@Use}.
%    \begin{macrocode}
\AtBeginDocument{%
  \let\HypDest@Use\@gobble
}
%    \end{macrocode}
%
%    \begin{macro}{\HypDest@MarkUsed}
%    Destinations that are actually used are marked by \cs{HypDest@MarkUsed}.
%    \cs{nofiles} is respected.
%    \begin{macrocode}
\def\HypDest@MarkUsed#1{%
  \HypDest@VerboseInfo{%
    MarkUsed: (#1)%
  }%
  \if@filesw
    \immediate\write\@auxout{%
      \string\HypDest@Use{\pdf@escapehex{#1}}%
    }%
  \fi
}%
%    \end{macrocode}
%    \end{macro}
%
% \subsection{Redefinition of \xpackage{hyperref}'s hooks}
%
%    Package \xpackage{hyperref} can be loaded later, therefore
%    we redefine \xpackage{hyperref}'s macros at |\begin{document}|.
%    \begin{macrocode}
\HypDest@PrependDocument{%
%    \end{macrocode}
%
%    Check hyperref version.
%    \begin{macrocode}
  \@ifpackagelater{hyperref}{2006/06/01}{}{%
    \PackageError{hypdestopt}{%
      hyperref 2006/06/01 v6.75a or later is required%
    }\@ehc
  }%
%    \end{macrocode}
%
% \subsubsection{Destination setting}
%
%    \begin{macrocode}
  \ifHypDest@name
    \let\HypDest@Org@DestName\Hy@DestName
    \renewcommand*{\Hy@DestName}[2]{%
      \EdefUnescapeString\HypDest@temp{#1}%
      \@ifundefined{\HypDest@Prefix\HypDest@temp}{%
        \HypDest@VerboseInfo{%
          DestName: (\HypDest@temp) unused%
        }%
      }{%
        \HypDest@Org@DestName{%
          \csname\HypDest@Prefix\HypDest@temp\endcsname
        }{#2}%
        \HypDest@VerboseInfo{%
          DestName: (\HypDest@temp) %
          0x\pdf@escapehex{%
            \csname\HypDest@Prefix\HypDest@temp\endcsname
          }%
        }%
      }%
    }%
  \else
    \renewcommand*{\Hy@DestName}[2]{%
      \EdefUnescapeString\HypDest@temp{#1}%
      \@ifundefined{\HypDest@Prefix\HypDest@temp}{%
        \HypDest@VerboseInfo{%
          DestName: (\HypDest@temp) unused%
        }%
      }{%
        \pdfdest num%
        \csname\HypDest@Prefix\HypDest@temp\endcsname#2\relax
        \HypDest@VerboseInfo{%
          DestName: (\HypDest@temp) %
          num \csname\HypDest@Prefix\HypDest@temp\endcsname
        }%
      }%
    }%
  \fi
%    \end{macrocode}
%
% \subsubsection{Links}
%
%    \begin{macrocode}
  \let\HypDest@Org@StartlinkName\Hy@StartlinkName
  \ifHypDest@name
    \renewcommand*{\Hy@StartlinkName}[2]{%
      \HypDest@MarkUsed{#2}%
      \HypDest@Org@StartlinkName{#1}{%
        \@ifundefined{\HypDest@Prefix#2}{%
          #2%
        }{%
          \csname\HypDest@Prefix#2\endcsname
        }%
      }%
    }%
  \else
    \renewcommand*{\Hy@StartlinkName}[2]{%
      \HypDest@MarkUsed{#2}%
      \@ifundefined{\HypDest@Prefix#2}{%
        \HypDest@Org@StartlinkName{#1}{#2}%
      }{%
        \pdfstartlink attr{#1}%
                      goto num\csname\HypDest@Prefix#2\endcsname
        \relax
      }%
    }%
  \fi
%    \end{macrocode}
%
% \subsubsection{Outlines of package \xpackage{hyperref}}
%
%    \begin{macrocode}
  \let\HypDest@Org@OutlineName\Hy@OutlineName
  \ifHypDest@name
    \renewcommand*{\Hy@OutlineName}[4]{%
      \HypDest@Org@OutlineName{#1}{%
        \@ifundefined{\HypDest@Prefix#2}{%
          #2%
        }{%
          \csname\HypDest@Prefix#2\endcsname
        }%
      }{#3}{#4}%
    }%
  \else
    \renewcommand*{\Hy@OutlineName}[4]{%
      \@ifundefined{\HypDest@Prefix#2}{%
        \HypDest@Org@OutlineName{#1}{#2}{#3}{#4}%
      }{%
        \pdfoutline goto num\csname\HypDest@Prefix#2\endcsname
                    count#3{#4}%
      }%
    }%
  \fi
%    \end{macrocode}
%    Because \cs{Hy@OutlineName} is called after the \xfile{.out} file
%    is written in the previous run. Therefore we mark the destination
%    earlier in \cs{@@writetorep}.
%    \begin{macrocode}
  \let\HypDest@Org@@writetorep\@@writetorep
  \renewcommand*{\@@writetorep}[5]{%
    \begingroup
      \edef\Hy@tempa{#5}%
      \ifx\Hy@tempa\Hy@bookmarkstype
        \HypDest@MarkUsed{#3}%
      \fi
    \endgroup
    \HypDest@Org@@writetorep{#1}{#2}{#3}{#4}{#5}%
  }%
%    \end{macrocode}
%
% \subsubsection{Outlines of package \xpackage{bookmark}}
%
%    \begin{macrocode}
  \@ifpackageloaded{bookmark}{%
    \@ifpackagelater{bookmark}{2008/08/08}{%
      \renewcommand*{\BKM@DefGotoNameAction}[2]{%
        \@ifundefined{\HypDest@Prefix#2}{%
          \edef#1{goto name{hypdestopt\string :unknown}}%
        }{%
          \ifHypDest@name
            \edef#1{goto name{\csname\HypDest@Prefix#2\endcsname}}%
          \else
            \edef#1{goto num\csname\HypDest@Prefix#2\endcsname}%
          \fi
        }%
      }%
      \def\BKM@HypDestOptHook{%
        \ifx\BKM@dest\@empty
        \else
          \ifx\BKM@gotor\@empty
            \HypDest@MarkUsed\BKM@dest
          \fi
        \fi
      }%
    }{%
      \@PackageError{hypdestopt}{%
        Package `bookmark' is too old.\MessageBreak
        Version 2008/08/08 or later is needed%
      }\@ehc
    }%
  }{}%
%    \end{macrocode}
%
%    \begin{macrocode}
}
%    \end{macrocode}
%
%
%    \begin{macrocode}
%</package>
%    \end{macrocode}
%
% \section{Installation}
%
% \subsection{Download}
%
% \paragraph{Package.} This package is available on
% CTAN\footnote{\url{ftp://ftp.ctan.org/tex-archive/}}:
% \begin{description}
% \item[\CTAN{macros/latex/contrib/oberdiek/hypdestopt.dtx}] The source file.
% \item[\CTAN{macros/latex/contrib/oberdiek/hypdestopt.pdf}] Documentation.
% \end{description}
%
%
% \paragraph{Bundle.} All the packages of the bundle `oberdiek'
% are also available in a TDS compliant ZIP archive. There
% the packages are already unpacked and the documentation files
% are generated. The files and directories obey the TDS standard.
% \begin{description}
% \item[\CTAN{install/macros/latex/contrib/oberdiek.tds.zip}]
% \end{description}
% \emph{TDS} refers to the standard ``A Directory Structure
% for \TeX\ Files'' (\CTAN{tds/tds.pdf}). Directories
% with \xfile{texmf} in their name are usually organized this way.
%
% \subsection{Bundle installation}
%
% \paragraph{Unpacking.} Unpack the \xfile{oberdiek.tds.zip} in the
% TDS tree (also known as \xfile{texmf} tree) of your choice.
% Example (linux):
% \begin{quote}
%   |unzip oberdiek.tds.zip -d ~/texmf|
% \end{quote}
%
% \paragraph{Script installation.}
% Check the directory \xfile{TDS:scripts/oberdiek/} for
% scripts that need further installation steps.
% Package \xpackage{attachfile2} comes with the Perl script
% \xfile{pdfatfi.pl} that should be installed in such a way
% that it can be called as \texttt{pdfatfi}.
% Example (linux):
% \begin{quote}
%   |chmod +x scripts/oberdiek/pdfatfi.pl|\\
%   |cp scripts/oberdiek/pdfatfi.pl /usr/local/bin/|
% \end{quote}
%
% \subsection{Package installation}
%
% \paragraph{Unpacking.} The \xfile{.dtx} file is a self-extracting
% \docstrip\ archive. The files are extracted by running the
% \xfile{.dtx} through \plainTeX:
% \begin{quote}
%   \verb|tex hypdestopt.dtx|
% \end{quote}
%
% \paragraph{TDS.} Now the different files must be moved into
% the different directories in your installation TDS tree
% (also known as \xfile{texmf} tree):
% \begin{quote}
% \def\t{^^A
% \begin{tabular}{@{}>{\ttfamily}l@{ $\rightarrow$ }>{\ttfamily}l@{}}
%   hypdestopt.sty & tex/latex/oberdiek/hypdestopt.sty\\
%   hypdestopt.pdf & doc/latex/oberdiek/hypdestopt.pdf\\
%   hypdestopt.dtx & source/latex/oberdiek/hypdestopt.dtx\\
% \end{tabular}^^A
% }^^A
% \sbox0{\t}^^A
% \ifdim\wd0>\linewidth
%   \begingroup
%     \advance\linewidth by\leftmargin
%     \advance\linewidth by\rightmargin
%   \edef\x{\endgroup
%     \def\noexpand\lw{\the\linewidth}^^A
%   }\x
%   \def\lwbox{^^A
%     \leavevmode
%     \hbox to \linewidth{^^A
%       \kern-\leftmargin\relax
%       \hss
%       \usebox0
%       \hss
%       \kern-\rightmargin\relax
%     }^^A
%   }^^A
%   \ifdim\wd0>\lw
%     \sbox0{\small\t}^^A
%     \ifdim\wd0>\linewidth
%       \ifdim\wd0>\lw
%         \sbox0{\footnotesize\t}^^A
%         \ifdim\wd0>\linewidth
%           \ifdim\wd0>\lw
%             \sbox0{\scriptsize\t}^^A
%             \ifdim\wd0>\linewidth
%               \ifdim\wd0>\lw
%                 \sbox0{\tiny\t}^^A
%                 \ifdim\wd0>\linewidth
%                   \lwbox
%                 \else
%                   \usebox0
%                 \fi
%               \else
%                 \lwbox
%               \fi
%             \else
%               \usebox0
%             \fi
%           \else
%             \lwbox
%           \fi
%         \else
%           \usebox0
%         \fi
%       \else
%         \lwbox
%       \fi
%     \else
%       \usebox0
%     \fi
%   \else
%     \lwbox
%   \fi
% \else
%   \usebox0
% \fi
% \end{quote}
% If you have a \xfile{docstrip.cfg} that configures and enables \docstrip's
% TDS installing feature, then some files can already be in the right
% place, see the documentation of \docstrip.
%
% \subsection{Refresh file name databases}
%
% If your \TeX~distribution
% (\teTeX, \mikTeX, \dots) relies on file name databases, you must refresh
% these. For example, \teTeX\ users run \verb|texhash| or
% \verb|mktexlsr|.
%
% \subsection{Some details for the interested}
%
% \paragraph{Attached source.}
%
% The PDF documentation on CTAN also includes the
% \xfile{.dtx} source file. It can be extracted by
% AcrobatReader 6 or higher. Another option is \textsf{pdftk},
% e.g. unpack the file into the current directory:
% \begin{quote}
%   \verb|pdftk hypdestopt.pdf unpack_files output .|
% \end{quote}
%
% \paragraph{Unpacking with \LaTeX.}
% The \xfile{.dtx} chooses its action depending on the format:
% \begin{description}
% \item[\plainTeX:] Run \docstrip\ and extract the files.
% \item[\LaTeX:] Generate the documentation.
% \end{description}
% If you insist on using \LaTeX\ for \docstrip\ (really,
% \docstrip\ does not need \LaTeX), then inform the autodetect routine
% about your intention:
% \begin{quote}
%   \verb|latex \let\install=y% \iffalse meta-comment
%
% File: hypdestopt.dtx
% Version: 2011/05/13 v2.3
% Info: Hyperref destination optimizer
%
% Copyright (C) 2006-2008, 2011 by
%    Heiko Oberdiek <heiko.oberdiek at googlemail.com>
%
% This work may be distributed and/or modified under the
% conditions of the LaTeX Project Public License, either
% version 1.3c of this license or (at your option) any later
% version. This version of this license is in
%    http://www.latex-project.org/lppl/lppl-1-3c.txt
% and the latest version of this license is in
%    http://www.latex-project.org/lppl.txt
% and version 1.3 or later is part of all distributions of
% LaTeX version 2005/12/01 or later.
%
% This work has the LPPL maintenance status "maintained".
%
% This Current Maintainer of this work is Heiko Oberdiek.
%
% This work consists of the main source file hypdestopt.dtx
% and the derived files
%    hypdestopt.sty, hypdestopt.pdf, hypdestopt.ins, hypdestopt.drv.
%
% Distribution:
%    CTAN:macros/latex/contrib/oberdiek/hypdestopt.dtx
%    CTAN:macros/latex/contrib/oberdiek/hypdestopt.pdf
%
% Unpacking:
%    (a) If hypdestopt.ins is present:
%           tex hypdestopt.ins
%    (b) Without hypdestopt.ins:
%           tex hypdestopt.dtx
%    (c) If you insist on using LaTeX
%           latex \let\install=y\input{hypdestopt.dtx}
%        (quote the arguments according to the demands of your shell)
%
% Documentation:
%    (a) If hypdestopt.drv is present:
%           latex hypdestopt.drv
%    (b) Without hypdestopt.drv:
%           latex hypdestopt.dtx; ...
%    The class ltxdoc loads the configuration file ltxdoc.cfg
%    if available. Here you can specify further options, e.g.
%    use A4 as paper format:
%       \PassOptionsToClass{a4paper}{article}
%
%    Programm calls to get the documentation (example):
%       pdflatex hypdestopt.dtx
%       makeindex -s gind.ist hypdestopt.idx
%       pdflatex hypdestopt.dtx
%       makeindex -s gind.ist hypdestopt.idx
%       pdflatex hypdestopt.dtx
%
% Installation:
%    TDS:tex/latex/oberdiek/hypdestopt.sty
%    TDS:doc/latex/oberdiek/hypdestopt.pdf
%    TDS:source/latex/oberdiek/hypdestopt.dtx
%
%<*ignore>
\begingroup
  \catcode123=1 %
  \catcode125=2 %
  \def\x{LaTeX2e}%
\expandafter\endgroup
\ifcase 0\ifx\install y1\fi\expandafter
         \ifx\csname processbatchFile\endcsname\relax\else1\fi
         \ifx\fmtname\x\else 1\fi\relax
\else\csname fi\endcsname
%</ignore>
%<*install>
\input docstrip.tex
\Msg{************************************************************************}
\Msg{* Installation}
\Msg{* Package: hypdestopt 2011/05/13 v2.3 Hyperref destination optimizer (HO)}
\Msg{************************************************************************}

\keepsilent
\askforoverwritefalse

\let\MetaPrefix\relax
\preamble

This is a generated file.

Project: hypdestopt
Version: 2011/05/13 v2.3

Copyright (C) 2006-2008, 2011 by
   Heiko Oberdiek <heiko.oberdiek at googlemail.com>

This work may be distributed and/or modified under the
conditions of the LaTeX Project Public License, either
version 1.3c of this license or (at your option) any later
version. This version of this license is in
   http://www.latex-project.org/lppl/lppl-1-3c.txt
and the latest version of this license is in
   http://www.latex-project.org/lppl.txt
and version 1.3 or later is part of all distributions of
LaTeX version 2005/12/01 or later.

This work has the LPPL maintenance status "maintained".

This Current Maintainer of this work is Heiko Oberdiek.

This work consists of the main source file hypdestopt.dtx
and the derived files
   hypdestopt.sty, hypdestopt.pdf, hypdestopt.ins, hypdestopt.drv.

\endpreamble
\let\MetaPrefix\DoubleperCent

\generate{%
  \file{hypdestopt.ins}{\from{hypdestopt.dtx}{install}}%
  \file{hypdestopt.drv}{\from{hypdestopt.dtx}{driver}}%
  \usedir{tex/latex/oberdiek}%
  \file{hypdestopt.sty}{\from{hypdestopt.dtx}{package}}%
  \nopreamble
  \nopostamble
  \usedir{source/latex/oberdiek/catalogue}%
  \file{hypdestopt.xml}{\from{hypdestopt.dtx}{catalogue}}%
}

\catcode32=13\relax% active space
\let =\space%
\Msg{************************************************************************}
\Msg{*}
\Msg{* To finish the installation you have to move the following}
\Msg{* file into a directory searched by TeX:}
\Msg{*}
\Msg{*     hypdestopt.sty}
\Msg{*}
\Msg{* To produce the documentation run the file `hypdestopt.drv'}
\Msg{* through LaTeX.}
\Msg{*}
\Msg{* Happy TeXing!}
\Msg{*}
\Msg{************************************************************************}

\endbatchfile
%</install>
%<*ignore>
\fi
%</ignore>
%<*driver>
\NeedsTeXFormat{LaTeX2e}
\ProvidesFile{hypdestopt.drv}%
  [2011/05/13 v2.3 Hyperref destination optimizer (HO)]%
\documentclass{ltxdoc}
\usepackage{holtxdoc}[2011/11/22]
\begin{document}
  \DocInput{hypdestopt.dtx}%
\end{document}
%</driver>
% \fi
%
% \CheckSum{565}
%
% \CharacterTable
%  {Upper-case    \A\B\C\D\E\F\G\H\I\J\K\L\M\N\O\P\Q\R\S\T\U\V\W\X\Y\Z
%   Lower-case    \a\b\c\d\e\f\g\h\i\j\k\l\m\n\o\p\q\r\s\t\u\v\w\x\y\z
%   Digits        \0\1\2\3\4\5\6\7\8\9
%   Exclamation   \!     Double quote  \"     Hash (number) \#
%   Dollar        \$     Percent       \%     Ampersand     \&
%   Acute accent  \'     Left paren    \(     Right paren   \)
%   Asterisk      \*     Plus          \+     Comma         \,
%   Minus         \-     Point         \.     Solidus       \/
%   Colon         \:     Semicolon     \;     Less than     \<
%   Equals        \=     Greater than  \>     Question mark \?
%   Commercial at \@     Left bracket  \[     Backslash     \\
%   Right bracket \]     Circumflex    \^     Underscore    \_
%   Grave accent  \`     Left brace    \{     Vertical bar  \|
%   Right brace   \}     Tilde         \~}
%
% \GetFileInfo{hypdestopt.drv}
%
% \title{The \xpackage{hypdestopt} package}
% \date{2011/05/13 v2.3}
% \author{Heiko Oberdiek\\\xemail{heiko.oberdiek at googlemail.com}}
%
% \maketitle
%
% \begin{abstract}
% Package \xpackage{hypdestopt} supports \xpackage{hyperref}'s
% \xoption{pdftex} driver. It removes unnecessary destinations
% and shortens the destination names or uses numbered destinations
% to get smaller PDF files.
% \end{abstract}
%
% \tableofcontents
%
% \section{User interface}
%
% \subsection{Introduction}
%
% Before PDF-1.5 annotations and destinations cannot be compressed.
% If the destination names are not needed for external use, the
% file size can be decreased by the following means:
% \begin{itemize}
% \item Unused destinations are removed.
% \item The destination names are shortened (option \xoption{name}).
% \item Using numbered destinations (option \xoption{num}).
% \end{itemize}
%
% \subsection{Requirements}
%
% \begin{itemize}
% \item Package \xpackage{hyperref} 2006/06/01 v6.75a or newer
%       (\cite{hyperref}).
% \item Package \xpackage{alphalph} 2006/05/30 v1.4 or newer
%       (\cite{alphalph}), if option \xoption{name} is used.
% \item Package \xpackage{ifpdf} (\cite{ifpdf}).
% \item \pdfTeX\ 1.30.0 or newer.
% \item \pdfTeX\ in PDF mode.
% \item \eTeX\ extensions enabled.
% \item Probably an additional compile run of \pdfLaTeX\ is necessary.
% \end{itemize}
%
% In the first compile runs you can get warnings such as:
%\begin{quote}
%\begin{verbatim}
%! pdfTeX warning (dest): name{...} has been referenced ...
%\end{verbatim}
%\end{quote}
% These warnings should vanish in later compile runs.
% However these warnings also can occur without this package.
% The package does not cure them, thus these warnings will remain,
% but the destination name can be different. In such cases test
% without package, too.
%
% \subsection{Use}
%
% If the requirements are met, load the package:
%\begin{quote}
%\verb|\usepackage{hypdestopt}|
%\end{quote}
%
% The following options are supported:
% \begin{description}
% \item[\xoption{verbose}:] Verbose debug output is enabled and written
%   in the protocol file.
% \item[\xoption{num}:] Numbered destinations are used. The file size
%   is smaller, because names are no longer used.
%   This is the default.
% \item[\xoption{name}:] Destinations are identified by names.
% \end{description}
%
% \subsection{Limitations}
%
% \begin{itemize}
% \item Forget this package, if you need preserved destination names.
% \item Destination name strings use all bytes (0..255) except
%       the carriage return (13), left parenthesis (40), right
%       parenthesis (41), and backslash (92), because they
%       must be quoted in general and therefore occupy two bytes
%       instead of one.
%
%       Further the zero byte (0) is avoided for programs
%       that implement strings using zero terminated C strings.
%       And 255 (0xFF) is avoided to get rid of a possible
%       unicode marker at the begin.
%
%       So far I have not seen problems with:
%       \begin{itemize}
%       \item AcrobatReader 5.08/Linux
%       \item AcrobatReader 7.0/Linux
%       \item xpdf 3.00
%       \item Ghostscript 8.50
%       \item gv 3.5.8
%       \item GSview 4.6
%       \end{itemize}
%       But I have not tested all and all possible PDF viewers.
% \item Use of named destinations (\cs{pdfdest}, \cs{pdfoutline},
%       \cs{pdfstartlink}, \dots) that are not supported by this
%       package.
% \item Currently only \xpackage{hyperref} with \pdfTeX\ in PDF
%       mode is supported.
% \end{itemize}
%
% \subsection{Future}
%
% A more general approach is a PDF postprocessor that takes
% a PDF file, performs some transformations and writes the
% result in a more optimized PDF file. Then it does not depend,
% how the original PDF file was generated and further improvements
% are easier to apply. For example, the destination names could be sorted:
% often used destination names would then be shorter than seldom used ones.
%
% \StopEventually{
% }
%
% \section{Implementation}
%
% \subsection{Identification}
%
%    \begin{macrocode}
%<*package>
\NeedsTeXFormat{LaTeX2e}
\ProvidesPackage{hypdestopt}%
  [2011/05/13 v2.3 Hyperref destination optimizer (HO)]%
%    \end{macrocode}
%
% \subsection{Options}
%
% \subsubsection{Option \xoption{verbose}}
%
%    \begin{macrocode}
\newif\ifHypDest@Verbose
\DeclareOption{verbose}{\HypDest@Verbosetrue}
%    \end{macrocode}
%
%    \begin{macro}{\HypDest@VerboseInfo}
%    Wrapper for verbose messages.
%    \begin{macrocode}
\def\HypDest@VerboseInfo#1{%
  \ifHypDest@Verbose
    \PackageInfo{hypdestopt}{#1}%
  \fi
}
%    \end{macrocode}
%    \end{macro}
%
% \subsubsection{Options \xoption{num} and \xoption{name}}
%
%    The options \xoption{num} or \xoption{name} specify
%    the method, how destinations are referenced (by name or
%    number). Default is option \xoption{num}.
%    \begin{macrocode}
\newif\ifHypDest@name
\DeclareOption{num}{\HypDest@namefalse}
\DeclareOption{name}{\HypDest@nametrue}
%    \end{macrocode}
%
%    \begin{macrocode}
\ProcessOptions*\relax
%    \end{macrocode}
%
% \subsection{Check requirements}
%
%    First \pdfTeX\ must running in PDF mode.
%    \begin{macrocode}
\RequirePackage{ifpdf}[2007/09/09]
\RequirePackage{pdftexcmds}[2007/11/11]
\ifpdf
\else
  \PackageError{hypdestopt}{%
    This package requires pdfTeX in PDF mode%
  }\@ehc
  \expandafter\endinput
\fi
%    \end{macrocode}
%    The version of \pdfTeX\ must not be too old, because
%    \cs{pdfescapehex} and \cs{pdfunescapehex} are used.
%    \begin{macrocode}
\begingroup\expandafter\expandafter\expandafter\endgroup
\expandafter\ifx\csname pdf@escapehex\endcsname\relax
  \PackageError{hypdestopt}{%
    This pdfTeX is too old, at least 1.30.0 is required%
  }\@ehc
  \expandafter\endinput
\fi
%    \end{macrocode}
%    Features of \eTeX\ are used, e.g. \cs{numexpr}.
%    \begin{macrocode}
\begingroup\expandafter\expandafter\expandafter\endgroup
\expandafter\ifx\csname numexpr\endcsname\relax
  \PackageError{hypdestopt}{%
    e-TeX features are missing%
  }\@ehc
  \expandafter\endinput
\fi
%    \end{macrocode}
%    Package \xpackage{alphalph} provides \cs{newalphalph} since
%    version 2006/05/30 v1.4.
%    \begin{macrocode}
\ifHypDest@name
  \RequirePackage{alphalph}[2006/05/30]%
\fi
%    \end{macrocode}
%    \begin{macrocode}
\RequirePackage{auxhook}[2009/12/14]
\RequirePackage{pdfescape}[2007/04/21]
%    \end{macrocode}
%
% \subsection{Preamble for auxiliary file}
%
%    Provide dummy definitions for the macros that are used in the
%    auxiliary files. If the package is used no longer, then these
%    commands will not generate errors.
%
%    \begin{macro}{\HypDest@PrependDocument}
%    We add our stuff in front of the \cs{AtBeginDocument} hook
%    to ensure that we are before \xpackage{hyperref}'s stuff.
%    \begin{macrocode}
\long\def\HypDest@PrependDocument#1{%
  \begingroup
    \toks\z@{#1}%
    \toks\tw@\expandafter{\@begindocumenthook}%
    \xdef\@begindocumenthook{\the\toks\z@\the\toks\tw@}%
  \endgroup
}
%    \end{macrocode}
%    \end{macro}
%    \begin{macrocode}
\AddLineBeginAux{%
  \string\providecommand{\string\HypDest@Use}[1]{}%
}
%    \end{macrocode}
%
% \subsection{Generation of destination names}
%
%    Counter |HypDest| is used for identifying destinations.
%    \begin{macrocode}
\newcounter{HypDest}
%    \end{macrocode}
%
%    \begin{macrocode}
\ifHypDest@name
%    \end{macrocode}
%
%    \begin{macro}{\HypDest@HexChar}
%    Destination names are generated by automatically
%    numbering with the help of package \xpackage{alphalph}.
%    \cs{HypDest@HexChar} converts a number of the range 1 until 252
%    into the hexadecimal representation of the string character.
%    \begin{macrocode}
  \def\HypDest@HexChar#1{%
    \ifcase#1\or
%    \end{macrocode}
%    Avoid zero byte because of C strings in PDF viewer
%    applications.
%    \begin{macrocode}
      01\or 02\or 03\or 04\or 05\or 06\or 07\or
%    \end{macrocode}
%    Omit carriage return (13/\verb|^^0d|).
%    It needs quoting, otherwise it would be converted
%    to line feed (10/\verb|^^0a|).
%    \begin{macrocode}
      08\or 09\or 0A\or 0B\or 0C\or 0E\or 0F\or
      10\or 11\or 12\or 13\or 14\or 15\or 16\or 17\or
      18\or 19\or 1A\or 1B\or 1C\or 1D\or 1E\or 1F\or
      20\or 21\or 22\or 23\or 24\or 25\or 26\or 27\or
%    \end{macrocode}
%    Omit left and right parentheses (40/\verb|^^28|, 41/\verb|^^39|),
%    they need quoting in general.
%    \begin{macrocode}
      2A\or 2B\or 2C\or 2D\or 2E\or 2F\or
      30\or 31\or 32\or 33\or 34\or 35\or 36\or 37\or
      38\or 39\or 3A\or 3B\or 3C\or 3D\or 3E\or 3F\or
      40\or 41\or 42\or 43\or 44\or 45\or 46\or 47\or
      48\or 49\or 4A\or 4B\or 4C\or 4D\or 4E\or 4F\or
      50\or 51\or 52\or 53\or 54\or 55\or 56\or 57\or
%    \end{macrocode}
%    Omit backslash (92/\verb|^^5C|), it needs quoting.
%    \begin{macrocode}
      58\or 59\or 5A\or 5B\or 5D\or 5E\or 5F\or
      60\or 61\or 62\or 63\or 64\or 65\or 66\or 67\or
      68\or 69\or 6A\or 6B\or 6C\or 6D\or 6E\or 6F\or
      70\or 71\or 72\or 73\or 74\or 75\or 76\or 77\or
      78\or 79\or 7A\or 7B\or 7C\or 7D\or 7E\or 7F\or
      80\or 81\or 82\or 83\or 84\or 85\or 86\or 87\or
      88\or 89\or 8A\or 8B\or 8C\or 8D\or 8E\or 8F\or
      90\or 91\or 92\or 93\or 94\or 95\or 96\or 97\or
      98\or 99\or 9A\or 9B\or 9C\or 9D\or 9E\or 9F\or
      A0\or A1\or A2\or A3\or A4\or A5\or A6\or A7\or
      A8\or A9\or AA\or AB\or AC\or AD\or AE\or AF\or
      B0\or B1\or B2\or B3\or B4\or B5\or B6\or B7\or
      B8\or B9\or BA\or BB\or BC\or BD\or BE\or BF\or
      C0\or C1\or C2\or C3\or C4\or C5\or C6\or C7\or
      C8\or C9\or CA\or CB\or CC\or CD\or CE\or CF\or
      D0\or D1\or D2\or D3\or D4\or D5\or D6\or D7\or
      D8\or D9\or DA\or DB\or DC\or DD\or DE\or DF\or
      E0\or E1\or E2\or E3\or E4\or E5\or E6\or E7\or
      E8\or E9\or EA\or EB\or EC\or ED\or EE\or EF\or
      F0\or F1\or F2\or F3\or F4\or F5\or F6\or F7\or
%    \end{macrocode}
%    Avoid 255 (0xFF) to get rid of a possible unicode
%    marker at the begin of the string.
%    \begin{macrocode}
      F8\or F9\or FA\or FB\or FC\or FD\or FE%
    \fi
  }%
%    \end{macrocode}
%    \end{macro}
%    \begin{macro}{HypDest@HexString}
%    Now package \xpackage{alphalph} comes into play.
%    \cs{HypDest@HexString} is defined and converts
%    a positive number into a string, given in hexadecimal
%    representation.
%    \begin{macrocode}
  \newalphalph\HypDest@HexString\HypDest@HexChar{250}%
%    \end{macrocode}
%    \end{macro}
%    \begin{macro}{\theHypDest}
%    For use, the hexadecimal string is converted back.
%    \begin{macrocode}
  \renewcommand*{\theHypDest}{%
    \pdf@unescapehex{\HypDest@HexString{\value{HypDest}}}%
  }%
%    \end{macrocode}
%    \end{macro}
%
%    With option \xoption{num} we use the number directly.
%    \begin{macrocode}
\else
  \renewcommand*{\theHypDest}{%
    \number\value{HypDest}%
  }%
\fi
%    \end{macrocode}
%
% \subsection{Assign destination names}
%
%    \begin{macro}{\HypDest@Prefix}
%    The new destination names are remembered in macros whose names
%    start with prefix \cs{HypDest@Prefix}.
%    \begin{macrocode}
\edef\HypDest@Prefix{HypDest\string:}
%    \end{macrocode}
%    \end{macro}
%
%    \begin{macro}{\HypDest@Use}
%    During the first read of the auxiliary files, the used destinations
%    get fresh generated short destination names. Also for the old
%    destination names we use the hexadecimal representation. That
%    avoid problems with arbitrary names.
%    \begin{macrocode}
\def\HypDest@Use#1{%
  \begingroup
    \edef\x{%
      \expandafter\noexpand
      \csname\HypDest@Prefix\pdf@unescapehex{#1}\endcsname
    }%
    \expandafter\ifx\x\relax
      \stepcounter{HypDest}%
      \expandafter\xdef\x{\theHypDest}%
      \let\on@line\@empty
      \ifHypDest@name
        \HypDest@VerboseInfo{%
          Use: (\pdf@unescapehex{#1}) -\string> %
          0x\pdf@escapehex{\x} (\number\value{HypDest})%
        }%
      \else
        \HypDest@VerboseInfo{%
          Use: (\pdf@unescapehex{#1}) -\string> num \x
        }%
      \fi
    \fi
  \endgroup
}
%    \end{macrocode}
%    \end{macro}
%
%    After the first \xfile{.aux} file processing the destination names
%    are assigned and we can disable \cs{HypDest@Use}.
%    \begin{macrocode}
\AtBeginDocument{%
  \let\HypDest@Use\@gobble
}
%    \end{macrocode}
%
%    \begin{macro}{\HypDest@MarkUsed}
%    Destinations that are actually used are marked by \cs{HypDest@MarkUsed}.
%    \cs{nofiles} is respected.
%    \begin{macrocode}
\def\HypDest@MarkUsed#1{%
  \HypDest@VerboseInfo{%
    MarkUsed: (#1)%
  }%
  \if@filesw
    \immediate\write\@auxout{%
      \string\HypDest@Use{\pdf@escapehex{#1}}%
    }%
  \fi
}%
%    \end{macrocode}
%    \end{macro}
%
% \subsection{Redefinition of \xpackage{hyperref}'s hooks}
%
%    Package \xpackage{hyperref} can be loaded later, therefore
%    we redefine \xpackage{hyperref}'s macros at |\begin{document}|.
%    \begin{macrocode}
\HypDest@PrependDocument{%
%    \end{macrocode}
%
%    Check hyperref version.
%    \begin{macrocode}
  \@ifpackagelater{hyperref}{2006/06/01}{}{%
    \PackageError{hypdestopt}{%
      hyperref 2006/06/01 v6.75a or later is required%
    }\@ehc
  }%
%    \end{macrocode}
%
% \subsubsection{Destination setting}
%
%    \begin{macrocode}
  \ifHypDest@name
    \let\HypDest@Org@DestName\Hy@DestName
    \renewcommand*{\Hy@DestName}[2]{%
      \EdefUnescapeString\HypDest@temp{#1}%
      \@ifundefined{\HypDest@Prefix\HypDest@temp}{%
        \HypDest@VerboseInfo{%
          DestName: (\HypDest@temp) unused%
        }%
      }{%
        \HypDest@Org@DestName{%
          \csname\HypDest@Prefix\HypDest@temp\endcsname
        }{#2}%
        \HypDest@VerboseInfo{%
          DestName: (\HypDest@temp) %
          0x\pdf@escapehex{%
            \csname\HypDest@Prefix\HypDest@temp\endcsname
          }%
        }%
      }%
    }%
  \else
    \renewcommand*{\Hy@DestName}[2]{%
      \EdefUnescapeString\HypDest@temp{#1}%
      \@ifundefined{\HypDest@Prefix\HypDest@temp}{%
        \HypDest@VerboseInfo{%
          DestName: (\HypDest@temp) unused%
        }%
      }{%
        \pdfdest num%
        \csname\HypDest@Prefix\HypDest@temp\endcsname#2\relax
        \HypDest@VerboseInfo{%
          DestName: (\HypDest@temp) %
          num \csname\HypDest@Prefix\HypDest@temp\endcsname
        }%
      }%
    }%
  \fi
%    \end{macrocode}
%
% \subsubsection{Links}
%
%    \begin{macrocode}
  \let\HypDest@Org@StartlinkName\Hy@StartlinkName
  \ifHypDest@name
    \renewcommand*{\Hy@StartlinkName}[2]{%
      \HypDest@MarkUsed{#2}%
      \HypDest@Org@StartlinkName{#1}{%
        \@ifundefined{\HypDest@Prefix#2}{%
          #2%
        }{%
          \csname\HypDest@Prefix#2\endcsname
        }%
      }%
    }%
  \else
    \renewcommand*{\Hy@StartlinkName}[2]{%
      \HypDest@MarkUsed{#2}%
      \@ifundefined{\HypDest@Prefix#2}{%
        \HypDest@Org@StartlinkName{#1}{#2}%
      }{%
        \pdfstartlink attr{#1}%
                      goto num\csname\HypDest@Prefix#2\endcsname
        \relax
      }%
    }%
  \fi
%    \end{macrocode}
%
% \subsubsection{Outlines of package \xpackage{hyperref}}
%
%    \begin{macrocode}
  \let\HypDest@Org@OutlineName\Hy@OutlineName
  \ifHypDest@name
    \renewcommand*{\Hy@OutlineName}[4]{%
      \HypDest@Org@OutlineName{#1}{%
        \@ifundefined{\HypDest@Prefix#2}{%
          #2%
        }{%
          \csname\HypDest@Prefix#2\endcsname
        }%
      }{#3}{#4}%
    }%
  \else
    \renewcommand*{\Hy@OutlineName}[4]{%
      \@ifundefined{\HypDest@Prefix#2}{%
        \HypDest@Org@OutlineName{#1}{#2}{#3}{#4}%
      }{%
        \pdfoutline goto num\csname\HypDest@Prefix#2\endcsname
                    count#3{#4}%
      }%
    }%
  \fi
%    \end{macrocode}
%    Because \cs{Hy@OutlineName} is called after the \xfile{.out} file
%    is written in the previous run. Therefore we mark the destination
%    earlier in \cs{@@writetorep}.
%    \begin{macrocode}
  \let\HypDest@Org@@writetorep\@@writetorep
  \renewcommand*{\@@writetorep}[5]{%
    \begingroup
      \edef\Hy@tempa{#5}%
      \ifx\Hy@tempa\Hy@bookmarkstype
        \HypDest@MarkUsed{#3}%
      \fi
    \endgroup
    \HypDest@Org@@writetorep{#1}{#2}{#3}{#4}{#5}%
  }%
%    \end{macrocode}
%
% \subsubsection{Outlines of package \xpackage{bookmark}}
%
%    \begin{macrocode}
  \@ifpackageloaded{bookmark}{%
    \@ifpackagelater{bookmark}{2008/08/08}{%
      \renewcommand*{\BKM@DefGotoNameAction}[2]{%
        \@ifundefined{\HypDest@Prefix#2}{%
          \edef#1{goto name{hypdestopt\string :unknown}}%
        }{%
          \ifHypDest@name
            \edef#1{goto name{\csname\HypDest@Prefix#2\endcsname}}%
          \else
            \edef#1{goto num\csname\HypDest@Prefix#2\endcsname}%
          \fi
        }%
      }%
      \def\BKM@HypDestOptHook{%
        \ifx\BKM@dest\@empty
        \else
          \ifx\BKM@gotor\@empty
            \HypDest@MarkUsed\BKM@dest
          \fi
        \fi
      }%
    }{%
      \@PackageError{hypdestopt}{%
        Package `bookmark' is too old.\MessageBreak
        Version 2008/08/08 or later is needed%
      }\@ehc
    }%
  }{}%
%    \end{macrocode}
%
%    \begin{macrocode}
}
%    \end{macrocode}
%
%
%    \begin{macrocode}
%</package>
%    \end{macrocode}
%
% \section{Installation}
%
% \subsection{Download}
%
% \paragraph{Package.} This package is available on
% CTAN\footnote{\url{ftp://ftp.ctan.org/tex-archive/}}:
% \begin{description}
% \item[\CTAN{macros/latex/contrib/oberdiek/hypdestopt.dtx}] The source file.
% \item[\CTAN{macros/latex/contrib/oberdiek/hypdestopt.pdf}] Documentation.
% \end{description}
%
%
% \paragraph{Bundle.} All the packages of the bundle `oberdiek'
% are also available in a TDS compliant ZIP archive. There
% the packages are already unpacked and the documentation files
% are generated. The files and directories obey the TDS standard.
% \begin{description}
% \item[\CTAN{install/macros/latex/contrib/oberdiek.tds.zip}]
% \end{description}
% \emph{TDS} refers to the standard ``A Directory Structure
% for \TeX\ Files'' (\CTAN{tds/tds.pdf}). Directories
% with \xfile{texmf} in their name are usually organized this way.
%
% \subsection{Bundle installation}
%
% \paragraph{Unpacking.} Unpack the \xfile{oberdiek.tds.zip} in the
% TDS tree (also known as \xfile{texmf} tree) of your choice.
% Example (linux):
% \begin{quote}
%   |unzip oberdiek.tds.zip -d ~/texmf|
% \end{quote}
%
% \paragraph{Script installation.}
% Check the directory \xfile{TDS:scripts/oberdiek/} for
% scripts that need further installation steps.
% Package \xpackage{attachfile2} comes with the Perl script
% \xfile{pdfatfi.pl} that should be installed in such a way
% that it can be called as \texttt{pdfatfi}.
% Example (linux):
% \begin{quote}
%   |chmod +x scripts/oberdiek/pdfatfi.pl|\\
%   |cp scripts/oberdiek/pdfatfi.pl /usr/local/bin/|
% \end{quote}
%
% \subsection{Package installation}
%
% \paragraph{Unpacking.} The \xfile{.dtx} file is a self-extracting
% \docstrip\ archive. The files are extracted by running the
% \xfile{.dtx} through \plainTeX:
% \begin{quote}
%   \verb|tex hypdestopt.dtx|
% \end{quote}
%
% \paragraph{TDS.} Now the different files must be moved into
% the different directories in your installation TDS tree
% (also known as \xfile{texmf} tree):
% \begin{quote}
% \def\t{^^A
% \begin{tabular}{@{}>{\ttfamily}l@{ $\rightarrow$ }>{\ttfamily}l@{}}
%   hypdestopt.sty & tex/latex/oberdiek/hypdestopt.sty\\
%   hypdestopt.pdf & doc/latex/oberdiek/hypdestopt.pdf\\
%   hypdestopt.dtx & source/latex/oberdiek/hypdestopt.dtx\\
% \end{tabular}^^A
% }^^A
% \sbox0{\t}^^A
% \ifdim\wd0>\linewidth
%   \begingroup
%     \advance\linewidth by\leftmargin
%     \advance\linewidth by\rightmargin
%   \edef\x{\endgroup
%     \def\noexpand\lw{\the\linewidth}^^A
%   }\x
%   \def\lwbox{^^A
%     \leavevmode
%     \hbox to \linewidth{^^A
%       \kern-\leftmargin\relax
%       \hss
%       \usebox0
%       \hss
%       \kern-\rightmargin\relax
%     }^^A
%   }^^A
%   \ifdim\wd0>\lw
%     \sbox0{\small\t}^^A
%     \ifdim\wd0>\linewidth
%       \ifdim\wd0>\lw
%         \sbox0{\footnotesize\t}^^A
%         \ifdim\wd0>\linewidth
%           \ifdim\wd0>\lw
%             \sbox0{\scriptsize\t}^^A
%             \ifdim\wd0>\linewidth
%               \ifdim\wd0>\lw
%                 \sbox0{\tiny\t}^^A
%                 \ifdim\wd0>\linewidth
%                   \lwbox
%                 \else
%                   \usebox0
%                 \fi
%               \else
%                 \lwbox
%               \fi
%             \else
%               \usebox0
%             \fi
%           \else
%             \lwbox
%           \fi
%         \else
%           \usebox0
%         \fi
%       \else
%         \lwbox
%       \fi
%     \else
%       \usebox0
%     \fi
%   \else
%     \lwbox
%   \fi
% \else
%   \usebox0
% \fi
% \end{quote}
% If you have a \xfile{docstrip.cfg} that configures and enables \docstrip's
% TDS installing feature, then some files can already be in the right
% place, see the documentation of \docstrip.
%
% \subsection{Refresh file name databases}
%
% If your \TeX~distribution
% (\teTeX, \mikTeX, \dots) relies on file name databases, you must refresh
% these. For example, \teTeX\ users run \verb|texhash| or
% \verb|mktexlsr|.
%
% \subsection{Some details for the interested}
%
% \paragraph{Attached source.}
%
% The PDF documentation on CTAN also includes the
% \xfile{.dtx} source file. It can be extracted by
% AcrobatReader 6 or higher. Another option is \textsf{pdftk},
% e.g. unpack the file into the current directory:
% \begin{quote}
%   \verb|pdftk hypdestopt.pdf unpack_files output .|
% \end{quote}
%
% \paragraph{Unpacking with \LaTeX.}
% The \xfile{.dtx} chooses its action depending on the format:
% \begin{description}
% \item[\plainTeX:] Run \docstrip\ and extract the files.
% \item[\LaTeX:] Generate the documentation.
% \end{description}
% If you insist on using \LaTeX\ for \docstrip\ (really,
% \docstrip\ does not need \LaTeX), then inform the autodetect routine
% about your intention:
% \begin{quote}
%   \verb|latex \let\install=y\input{hypdestopt.dtx}|
% \end{quote}
% Do not forget to quote the argument according to the demands
% of your shell.
%
% \paragraph{Generating the documentation.}
% You can use both the \xfile{.dtx} or the \xfile{.drv} to generate
% the documentation. The process can be configured by the
% configuration file \xfile{ltxdoc.cfg}. For instance, put this
% line into this file, if you want to have A4 as paper format:
% \begin{quote}
%   \verb|\PassOptionsToClass{a4paper}{article}|
% \end{quote}
% An example follows how to generate the
% documentation with pdf\LaTeX:
% \begin{quote}
%\begin{verbatim}
%pdflatex hypdestopt.dtx
%makeindex -s gind.ist hypdestopt.idx
%pdflatex hypdestopt.dtx
%makeindex -s gind.ist hypdestopt.idx
%pdflatex hypdestopt.dtx
%\end{verbatim}
% \end{quote}
%
% \section{Catalogue}
%
% The following XML file can be used as source for the
% \href{http://mirror.ctan.org/help/Catalogue/catalogue.html}{\TeX\ Catalogue}.
% The elements \texttt{caption} and \texttt{description} are imported
% from the original XML file from the Catalogue.
% The name of the XML file in the Catalogue is \xfile{hypdestopt.xml}.
%    \begin{macrocode}
%<*catalogue>
<?xml version='1.0' encoding='us-ascii'?>
<!DOCTYPE entry SYSTEM 'catalogue.dtd'>
<entry datestamp='$Date$' modifier='$Author$' id='hypdestopt'>
  <name>hypdestopt</name>
  <caption>Hyperref destination optimizer.</caption>
  <authorref id='auth:oberdiek'/>
  <copyright owner='Heiko Oberdiek' year='2006-2008,2011'/>
  <license type='lppl1.3'/>
  <version number='2.3'/>
  <description>
    This package supports <xref refid='hyperref'>hyperref</xref>'s
    pdftex driver. It removes unnecessary destinations
    and shortens the destination names or uses numbered destinations
    to get smaller PDF files.
    <p/>
    The package is part of the <xref refid='oberdiek'>oberdiek</xref>
    bundle.
  </description>
  <documentation details='Package documentation'
      href='ctan:/macros/latex/contrib/oberdiek/hypdestopt.pdf'/>
  <ctan file='true' path='/macros/latex/contrib/oberdiek/hypdestopt.dtx'/>
  <miktex location='oberdiek'/>
  <texlive location='oberdiek'/>
  <install path='/macros/latex/contrib/oberdiek/oberdiek.tds.zip'/>
</entry>
%</catalogue>
%    \end{macrocode}
%
% \begin{thebibliography}{9}
%
% \bibitem{alphalph}
%   Heiko Oberdiek: \textit{The \xpackage{alphalph} package};
%   2006/05/30 v1.4;
%   \CTAN{macros/latex/contrib/oberdiek/alphalph.pdf}.
%
% \bibitem{hyperref}
%   Sebastian Rahtz, Heiko Oberdiek:
%   \textit{The \xpackage{hyperref} package};
%   2006/06/01 v6.75a;
%   \CTAN{macros/latex/contrib/hyperref/}.
%
% \bibitem{ifpdf}
%   Heiko Oberdiek: \textit{The \xpackage{ifpdf} package};
%   2006/02/20 v1.4;
%   \CTAN{macros/latex/contrib/oberdiek/ifpdf.pdf}.
%
% \end{thebibliography}
%
% \begin{History}
%   \begin{Version}{2006/06/01 v1.0}
%   \item
%     First version.
%   \end{Version}
%   \begin{Version}{2006/06/01 v2.0}
%   \item
%     New method for referencing destinations by number; an idea
%     proposed by Lars Hellstr\"om in the mailing list LATEX-L.
%   \item
%     Options \xoption{name} and \xoption{num} added.
%   \end{Version}
%   \begin{Version}{2007/11/11 v2.1}
%   \item
%     Use of package \xpackage{pdftexcmds} for \LuaTeX\ support.
%   \end{Version}
%   \begin{Version}{2008/08/08 v2.2}
%   \item
%     Support for package \xpackage{bookmark} added.
%   \end{Version}
%   \begin{Version}{2011/05/13 v2.3}
%   \item
%     Fix for \cs{Hy@DestName} if the destination name contains
%     special characters.
%   \item
%     Fix for option \xoption{name} and package \xpackage{bookmark}.
%   \end{Version}
% \end{History}
%
% \PrintIndex
%
% \Finale
\endinput
|
% \end{quote}
% Do not forget to quote the argument according to the demands
% of your shell.
%
% \paragraph{Generating the documentation.}
% You can use both the \xfile{.dtx} or the \xfile{.drv} to generate
% the documentation. The process can be configured by the
% configuration file \xfile{ltxdoc.cfg}. For instance, put this
% line into this file, if you want to have A4 as paper format:
% \begin{quote}
%   \verb|\PassOptionsToClass{a4paper}{article}|
% \end{quote}
% An example follows how to generate the
% documentation with pdf\LaTeX:
% \begin{quote}
%\begin{verbatim}
%pdflatex hypdestopt.dtx
%makeindex -s gind.ist hypdestopt.idx
%pdflatex hypdestopt.dtx
%makeindex -s gind.ist hypdestopt.idx
%pdflatex hypdestopt.dtx
%\end{verbatim}
% \end{quote}
%
% \section{Catalogue}
%
% The following XML file can be used as source for the
% \href{http://mirror.ctan.org/help/Catalogue/catalogue.html}{\TeX\ Catalogue}.
% The elements \texttt{caption} and \texttt{description} are imported
% from the original XML file from the Catalogue.
% The name of the XML file in the Catalogue is \xfile{hypdestopt.xml}.
%    \begin{macrocode}
%<*catalogue>
<?xml version='1.0' encoding='us-ascii'?>
<!DOCTYPE entry SYSTEM 'catalogue.dtd'>
<entry datestamp='$Date$' modifier='$Author$' id='hypdestopt'>
  <name>hypdestopt</name>
  <caption>Hyperref destination optimizer.</caption>
  <authorref id='auth:oberdiek'/>
  <copyright owner='Heiko Oberdiek' year='2006-2008,2011'/>
  <license type='lppl1.3'/>
  <version number='2.3'/>
  <description>
    This package supports <xref refid='hyperref'>hyperref</xref>'s
    pdftex driver. It removes unnecessary destinations
    and shortens the destination names or uses numbered destinations
    to get smaller PDF files.
    <p/>
    The package is part of the <xref refid='oberdiek'>oberdiek</xref>
    bundle.
  </description>
  <documentation details='Package documentation'
      href='ctan:/macros/latex/contrib/oberdiek/hypdestopt.pdf'/>
  <ctan file='true' path='/macros/latex/contrib/oberdiek/hypdestopt.dtx'/>
  <miktex location='oberdiek'/>
  <texlive location='oberdiek'/>
  <install path='/macros/latex/contrib/oberdiek/oberdiek.tds.zip'/>
</entry>
%</catalogue>
%    \end{macrocode}
%
% \begin{thebibliography}{9}
%
% \bibitem{alphalph}
%   Heiko Oberdiek: \textit{The \xpackage{alphalph} package};
%   2006/05/30 v1.4;
%   \CTAN{macros/latex/contrib/oberdiek/alphalph.pdf}.
%
% \bibitem{hyperref}
%   Sebastian Rahtz, Heiko Oberdiek:
%   \textit{The \xpackage{hyperref} package};
%   2006/06/01 v6.75a;
%   \CTAN{macros/latex/contrib/hyperref/}.
%
% \bibitem{ifpdf}
%   Heiko Oberdiek: \textit{The \xpackage{ifpdf} package};
%   2006/02/20 v1.4;
%   \CTAN{macros/latex/contrib/oberdiek/ifpdf.pdf}.
%
% \end{thebibliography}
%
% \begin{History}
%   \begin{Version}{2006/06/01 v1.0}
%   \item
%     First version.
%   \end{Version}
%   \begin{Version}{2006/06/01 v2.0}
%   \item
%     New method for referencing destinations by number; an idea
%     proposed by Lars Hellstr\"om in the mailing list LATEX-L.
%   \item
%     Options \xoption{name} and \xoption{num} added.
%   \end{Version}
%   \begin{Version}{2007/11/11 v2.1}
%   \item
%     Use of package \xpackage{pdftexcmds} for \LuaTeX\ support.
%   \end{Version}
%   \begin{Version}{2008/08/08 v2.2}
%   \item
%     Support for package \xpackage{bookmark} added.
%   \end{Version}
%   \begin{Version}{2011/05/13 v2.3}
%   \item
%     Fix for \cs{Hy@DestName} if the destination name contains
%     special characters.
%   \item
%     Fix for option \xoption{name} and package \xpackage{bookmark}.
%   \end{Version}
% \end{History}
%
% \PrintIndex
%
% \Finale
\endinput

%        (quote the arguments according to the demands of your shell)
%
% Documentation:
%    (a) If hypdestopt.drv is present:
%           latex hypdestopt.drv
%    (b) Without hypdestopt.drv:
%           latex hypdestopt.dtx; ...
%    The class ltxdoc loads the configuration file ltxdoc.cfg
%    if available. Here you can specify further options, e.g.
%    use A4 as paper format:
%       \PassOptionsToClass{a4paper}{article}
%
%    Programm calls to get the documentation (example):
%       pdflatex hypdestopt.dtx
%       makeindex -s gind.ist hypdestopt.idx
%       pdflatex hypdestopt.dtx
%       makeindex -s gind.ist hypdestopt.idx
%       pdflatex hypdestopt.dtx
%
% Installation:
%    TDS:tex/latex/oberdiek/hypdestopt.sty
%    TDS:doc/latex/oberdiek/hypdestopt.pdf
%    TDS:source/latex/oberdiek/hypdestopt.dtx
%
%<*ignore>
\begingroup
  \catcode123=1 %
  \catcode125=2 %
  \def\x{LaTeX2e}%
\expandafter\endgroup
\ifcase 0\ifx\install y1\fi\expandafter
         \ifx\csname processbatchFile\endcsname\relax\else1\fi
         \ifx\fmtname\x\else 1\fi\relax
\else\csname fi\endcsname
%</ignore>
%<*install>
\input docstrip.tex
\Msg{************************************************************************}
\Msg{* Installation}
\Msg{* Package: hypdestopt 2011/05/13 v2.3 Hyperref destination optimizer (HO)}
\Msg{************************************************************************}

\keepsilent
\askforoverwritefalse

\let\MetaPrefix\relax
\preamble

This is a generated file.

Project: hypdestopt
Version: 2011/05/13 v2.3

Copyright (C) 2006-2008, 2011 by
   Heiko Oberdiek <heiko.oberdiek at googlemail.com>

This work may be distributed and/or modified under the
conditions of the LaTeX Project Public License, either
version 1.3c of this license or (at your option) any later
version. This version of this license is in
   http://www.latex-project.org/lppl/lppl-1-3c.txt
and the latest version of this license is in
   http://www.latex-project.org/lppl.txt
and version 1.3 or later is part of all distributions of
LaTeX version 2005/12/01 or later.

This work has the LPPL maintenance status "maintained".

This Current Maintainer of this work is Heiko Oberdiek.

This work consists of the main source file hypdestopt.dtx
and the derived files
   hypdestopt.sty, hypdestopt.pdf, hypdestopt.ins, hypdestopt.drv.

\endpreamble
\let\MetaPrefix\DoubleperCent

\generate{%
  \file{hypdestopt.ins}{\from{hypdestopt.dtx}{install}}%
  \file{hypdestopt.drv}{\from{hypdestopt.dtx}{driver}}%
  \usedir{tex/latex/oberdiek}%
  \file{hypdestopt.sty}{\from{hypdestopt.dtx}{package}}%
  \nopreamble
  \nopostamble
  \usedir{source/latex/oberdiek/catalogue}%
  \file{hypdestopt.xml}{\from{hypdestopt.dtx}{catalogue}}%
}

\catcode32=13\relax% active space
\let =\space%
\Msg{************************************************************************}
\Msg{*}
\Msg{* To finish the installation you have to move the following}
\Msg{* file into a directory searched by TeX:}
\Msg{*}
\Msg{*     hypdestopt.sty}
\Msg{*}
\Msg{* To produce the documentation run the file `hypdestopt.drv'}
\Msg{* through LaTeX.}
\Msg{*}
\Msg{* Happy TeXing!}
\Msg{*}
\Msg{************************************************************************}

\endbatchfile
%</install>
%<*ignore>
\fi
%</ignore>
%<*driver>
\NeedsTeXFormat{LaTeX2e}
\ProvidesFile{hypdestopt.drv}%
  [2011/05/13 v2.3 Hyperref destination optimizer (HO)]%
\documentclass{ltxdoc}
\usepackage{holtxdoc}[2011/11/22]
\begin{document}
  \DocInput{hypdestopt.dtx}%
\end{document}
%</driver>
% \fi
%
% \CheckSum{565}
%
% \CharacterTable
%  {Upper-case    \A\B\C\D\E\F\G\H\I\J\K\L\M\N\O\P\Q\R\S\T\U\V\W\X\Y\Z
%   Lower-case    \a\b\c\d\e\f\g\h\i\j\k\l\m\n\o\p\q\r\s\t\u\v\w\x\y\z
%   Digits        \0\1\2\3\4\5\6\7\8\9
%   Exclamation   \!     Double quote  \"     Hash (number) \#
%   Dollar        \$     Percent       \%     Ampersand     \&
%   Acute accent  \'     Left paren    \(     Right paren   \)
%   Asterisk      \*     Plus          \+     Comma         \,
%   Minus         \-     Point         \.     Solidus       \/
%   Colon         \:     Semicolon     \;     Less than     \<
%   Equals        \=     Greater than  \>     Question mark \?
%   Commercial at \@     Left bracket  \[     Backslash     \\
%   Right bracket \]     Circumflex    \^     Underscore    \_
%   Grave accent  \`     Left brace    \{     Vertical bar  \|
%   Right brace   \}     Tilde         \~}
%
% \GetFileInfo{hypdestopt.drv}
%
% \title{The \xpackage{hypdestopt} package}
% \date{2011/05/13 v2.3}
% \author{Heiko Oberdiek\\\xemail{heiko.oberdiek at googlemail.com}}
%
% \maketitle
%
% \begin{abstract}
% Package \xpackage{hypdestopt} supports \xpackage{hyperref}'s
% \xoption{pdftex} driver. It removes unnecessary destinations
% and shortens the destination names or uses numbered destinations
% to get smaller PDF files.
% \end{abstract}
%
% \tableofcontents
%
% \section{User interface}
%
% \subsection{Introduction}
%
% Before PDF-1.5 annotations and destinations cannot be compressed.
% If the destination names are not needed for external use, the
% file size can be decreased by the following means:
% \begin{itemize}
% \item Unused destinations are removed.
% \item The destination names are shortened (option \xoption{name}).
% \item Using numbered destinations (option \xoption{num}).
% \end{itemize}
%
% \subsection{Requirements}
%
% \begin{itemize}
% \item Package \xpackage{hyperref} 2006/06/01 v6.75a or newer
%       (\cite{hyperref}).
% \item Package \xpackage{alphalph} 2006/05/30 v1.4 or newer
%       (\cite{alphalph}), if option \xoption{name} is used.
% \item Package \xpackage{ifpdf} (\cite{ifpdf}).
% \item \pdfTeX\ 1.30.0 or newer.
% \item \pdfTeX\ in PDF mode.
% \item \eTeX\ extensions enabled.
% \item Probably an additional compile run of \pdfLaTeX\ is necessary.
% \end{itemize}
%
% In the first compile runs you can get warnings such as:
%\begin{quote}
%\begin{verbatim}
%! pdfTeX warning (dest): name{...} has been referenced ...
%\end{verbatim}
%\end{quote}
% These warnings should vanish in later compile runs.
% However these warnings also can occur without this package.
% The package does not cure them, thus these warnings will remain,
% but the destination name can be different. In such cases test
% without package, too.
%
% \subsection{Use}
%
% If the requirements are met, load the package:
%\begin{quote}
%\verb|\usepackage{hypdestopt}|
%\end{quote}
%
% The following options are supported:
% \begin{description}
% \item[\xoption{verbose}:] Verbose debug output is enabled and written
%   in the protocol file.
% \item[\xoption{num}:] Numbered destinations are used. The file size
%   is smaller, because names are no longer used.
%   This is the default.
% \item[\xoption{name}:] Destinations are identified by names.
% \end{description}
%
% \subsection{Limitations}
%
% \begin{itemize}
% \item Forget this package, if you need preserved destination names.
% \item Destination name strings use all bytes (0..255) except
%       the carriage return (13), left parenthesis (40), right
%       parenthesis (41), and backslash (92), because they
%       must be quoted in general and therefore occupy two bytes
%       instead of one.
%
%       Further the zero byte (0) is avoided for programs
%       that implement strings using zero terminated C strings.
%       And 255 (0xFF) is avoided to get rid of a possible
%       unicode marker at the begin.
%
%       So far I have not seen problems with:
%       \begin{itemize}
%       \item AcrobatReader 5.08/Linux
%       \item AcrobatReader 7.0/Linux
%       \item xpdf 3.00
%       \item Ghostscript 8.50
%       \item gv 3.5.8
%       \item GSview 4.6
%       \end{itemize}
%       But I have not tested all and all possible PDF viewers.
% \item Use of named destinations (\cs{pdfdest}, \cs{pdfoutline},
%       \cs{pdfstartlink}, \dots) that are not supported by this
%       package.
% \item Currently only \xpackage{hyperref} with \pdfTeX\ in PDF
%       mode is supported.
% \end{itemize}
%
% \subsection{Future}
%
% A more general approach is a PDF postprocessor that takes
% a PDF file, performs some transformations and writes the
% result in a more optimized PDF file. Then it does not depend,
% how the original PDF file was generated and further improvements
% are easier to apply. For example, the destination names could be sorted:
% often used destination names would then be shorter than seldom used ones.
%
% \StopEventually{
% }
%
% \section{Implementation}
%
% \subsection{Identification}
%
%    \begin{macrocode}
%<*package>
\NeedsTeXFormat{LaTeX2e}
\ProvidesPackage{hypdestopt}%
  [2011/05/13 v2.3 Hyperref destination optimizer (HO)]%
%    \end{macrocode}
%
% \subsection{Options}
%
% \subsubsection{Option \xoption{verbose}}
%
%    \begin{macrocode}
\newif\ifHypDest@Verbose
\DeclareOption{verbose}{\HypDest@Verbosetrue}
%    \end{macrocode}
%
%    \begin{macro}{\HypDest@VerboseInfo}
%    Wrapper for verbose messages.
%    \begin{macrocode}
\def\HypDest@VerboseInfo#1{%
  \ifHypDest@Verbose
    \PackageInfo{hypdestopt}{#1}%
  \fi
}
%    \end{macrocode}
%    \end{macro}
%
% \subsubsection{Options \xoption{num} and \xoption{name}}
%
%    The options \xoption{num} or \xoption{name} specify
%    the method, how destinations are referenced (by name or
%    number). Default is option \xoption{num}.
%    \begin{macrocode}
\newif\ifHypDest@name
\DeclareOption{num}{\HypDest@namefalse}
\DeclareOption{name}{\HypDest@nametrue}
%    \end{macrocode}
%
%    \begin{macrocode}
\ProcessOptions*\relax
%    \end{macrocode}
%
% \subsection{Check requirements}
%
%    First \pdfTeX\ must running in PDF mode.
%    \begin{macrocode}
\RequirePackage{ifpdf}[2007/09/09]
\RequirePackage{pdftexcmds}[2007/11/11]
\ifpdf
\else
  \PackageError{hypdestopt}{%
    This package requires pdfTeX in PDF mode%
  }\@ehc
  \expandafter\endinput
\fi
%    \end{macrocode}
%    The version of \pdfTeX\ must not be too old, because
%    \cs{pdfescapehex} and \cs{pdfunescapehex} are used.
%    \begin{macrocode}
\begingroup\expandafter\expandafter\expandafter\endgroup
\expandafter\ifx\csname pdf@escapehex\endcsname\relax
  \PackageError{hypdestopt}{%
    This pdfTeX is too old, at least 1.30.0 is required%
  }\@ehc
  \expandafter\endinput
\fi
%    \end{macrocode}
%    Features of \eTeX\ are used, e.g. \cs{numexpr}.
%    \begin{macrocode}
\begingroup\expandafter\expandafter\expandafter\endgroup
\expandafter\ifx\csname numexpr\endcsname\relax
  \PackageError{hypdestopt}{%
    e-TeX features are missing%
  }\@ehc
  \expandafter\endinput
\fi
%    \end{macrocode}
%    Package \xpackage{alphalph} provides \cs{newalphalph} since
%    version 2006/05/30 v1.4.
%    \begin{macrocode}
\ifHypDest@name
  \RequirePackage{alphalph}[2006/05/30]%
\fi
%    \end{macrocode}
%    \begin{macrocode}
\RequirePackage{auxhook}[2009/12/14]
\RequirePackage{pdfescape}[2007/04/21]
%    \end{macrocode}
%
% \subsection{Preamble for auxiliary file}
%
%    Provide dummy definitions for the macros that are used in the
%    auxiliary files. If the package is used no longer, then these
%    commands will not generate errors.
%
%    \begin{macro}{\HypDest@PrependDocument}
%    We add our stuff in front of the \cs{AtBeginDocument} hook
%    to ensure that we are before \xpackage{hyperref}'s stuff.
%    \begin{macrocode}
\long\def\HypDest@PrependDocument#1{%
  \begingroup
    \toks\z@{#1}%
    \toks\tw@\expandafter{\@begindocumenthook}%
    \xdef\@begindocumenthook{\the\toks\z@\the\toks\tw@}%
  \endgroup
}
%    \end{macrocode}
%    \end{macro}
%    \begin{macrocode}
\AddLineBeginAux{%
  \string\providecommand{\string\HypDest@Use}[1]{}%
}
%    \end{macrocode}
%
% \subsection{Generation of destination names}
%
%    Counter |HypDest| is used for identifying destinations.
%    \begin{macrocode}
\newcounter{HypDest}
%    \end{macrocode}
%
%    \begin{macrocode}
\ifHypDest@name
%    \end{macrocode}
%
%    \begin{macro}{\HypDest@HexChar}
%    Destination names are generated by automatically
%    numbering with the help of package \xpackage{alphalph}.
%    \cs{HypDest@HexChar} converts a number of the range 1 until 252
%    into the hexadecimal representation of the string character.
%    \begin{macrocode}
  \def\HypDest@HexChar#1{%
    \ifcase#1\or
%    \end{macrocode}
%    Avoid zero byte because of C strings in PDF viewer
%    applications.
%    \begin{macrocode}
      01\or 02\or 03\or 04\or 05\or 06\or 07\or
%    \end{macrocode}
%    Omit carriage return (13/\verb|^^0d|).
%    It needs quoting, otherwise it would be converted
%    to line feed (10/\verb|^^0a|).
%    \begin{macrocode}
      08\or 09\or 0A\or 0B\or 0C\or 0E\or 0F\or
      10\or 11\or 12\or 13\or 14\or 15\or 16\or 17\or
      18\or 19\or 1A\or 1B\or 1C\or 1D\or 1E\or 1F\or
      20\or 21\or 22\or 23\or 24\or 25\or 26\or 27\or
%    \end{macrocode}
%    Omit left and right parentheses (40/\verb|^^28|, 41/\verb|^^39|),
%    they need quoting in general.
%    \begin{macrocode}
      2A\or 2B\or 2C\or 2D\or 2E\or 2F\or
      30\or 31\or 32\or 33\or 34\or 35\or 36\or 37\or
      38\or 39\or 3A\or 3B\or 3C\or 3D\or 3E\or 3F\or
      40\or 41\or 42\or 43\or 44\or 45\or 46\or 47\or
      48\or 49\or 4A\or 4B\or 4C\or 4D\or 4E\or 4F\or
      50\or 51\or 52\or 53\or 54\or 55\or 56\or 57\or
%    \end{macrocode}
%    Omit backslash (92/\verb|^^5C|), it needs quoting.
%    \begin{macrocode}
      58\or 59\or 5A\or 5B\or 5D\or 5E\or 5F\or
      60\or 61\or 62\or 63\or 64\or 65\or 66\or 67\or
      68\or 69\or 6A\or 6B\or 6C\or 6D\or 6E\or 6F\or
      70\or 71\or 72\or 73\or 74\or 75\or 76\or 77\or
      78\or 79\or 7A\or 7B\or 7C\or 7D\or 7E\or 7F\or
      80\or 81\or 82\or 83\or 84\or 85\or 86\or 87\or
      88\or 89\or 8A\or 8B\or 8C\or 8D\or 8E\or 8F\or
      90\or 91\or 92\or 93\or 94\or 95\or 96\or 97\or
      98\or 99\or 9A\or 9B\or 9C\or 9D\or 9E\or 9F\or
      A0\or A1\or A2\or A3\or A4\or A5\or A6\or A7\or
      A8\or A9\or AA\or AB\or AC\or AD\or AE\or AF\or
      B0\or B1\or B2\or B3\or B4\or B5\or B6\or B7\or
      B8\or B9\or BA\or BB\or BC\or BD\or BE\or BF\or
      C0\or C1\or C2\or C3\or C4\or C5\or C6\or C7\or
      C8\or C9\or CA\or CB\or CC\or CD\or CE\or CF\or
      D0\or D1\or D2\or D3\or D4\or D5\or D6\or D7\or
      D8\or D9\or DA\or DB\or DC\or DD\or DE\or DF\or
      E0\or E1\or E2\or E3\or E4\or E5\or E6\or E7\or
      E8\or E9\or EA\or EB\or EC\or ED\or EE\or EF\or
      F0\or F1\or F2\or F3\or F4\or F5\or F6\or F7\or
%    \end{macrocode}
%    Avoid 255 (0xFF) to get rid of a possible unicode
%    marker at the begin of the string.
%    \begin{macrocode}
      F8\or F9\or FA\or FB\or FC\or FD\or FE%
    \fi
  }%
%    \end{macrocode}
%    \end{macro}
%    \begin{macro}{HypDest@HexString}
%    Now package \xpackage{alphalph} comes into play.
%    \cs{HypDest@HexString} is defined and converts
%    a positive number into a string, given in hexadecimal
%    representation.
%    \begin{macrocode}
  \newalphalph\HypDest@HexString\HypDest@HexChar{250}%
%    \end{macrocode}
%    \end{macro}
%    \begin{macro}{\theHypDest}
%    For use, the hexadecimal string is converted back.
%    \begin{macrocode}
  \renewcommand*{\theHypDest}{%
    \pdf@unescapehex{\HypDest@HexString{\value{HypDest}}}%
  }%
%    \end{macrocode}
%    \end{macro}
%
%    With option \xoption{num} we use the number directly.
%    \begin{macrocode}
\else
  \renewcommand*{\theHypDest}{%
    \number\value{HypDest}%
  }%
\fi
%    \end{macrocode}
%
% \subsection{Assign destination names}
%
%    \begin{macro}{\HypDest@Prefix}
%    The new destination names are remembered in macros whose names
%    start with prefix \cs{HypDest@Prefix}.
%    \begin{macrocode}
\edef\HypDest@Prefix{HypDest\string:}
%    \end{macrocode}
%    \end{macro}
%
%    \begin{macro}{\HypDest@Use}
%    During the first read of the auxiliary files, the used destinations
%    get fresh generated short destination names. Also for the old
%    destination names we use the hexadecimal representation. That
%    avoid problems with arbitrary names.
%    \begin{macrocode}
\def\HypDest@Use#1{%
  \begingroup
    \edef\x{%
      \expandafter\noexpand
      \csname\HypDest@Prefix\pdf@unescapehex{#1}\endcsname
    }%
    \expandafter\ifx\x\relax
      \stepcounter{HypDest}%
      \expandafter\xdef\x{\theHypDest}%
      \let\on@line\@empty
      \ifHypDest@name
        \HypDest@VerboseInfo{%
          Use: (\pdf@unescapehex{#1}) -\string> %
          0x\pdf@escapehex{\x} (\number\value{HypDest})%
        }%
      \else
        \HypDest@VerboseInfo{%
          Use: (\pdf@unescapehex{#1}) -\string> num \x
        }%
      \fi
    \fi
  \endgroup
}
%    \end{macrocode}
%    \end{macro}
%
%    After the first \xfile{.aux} file processing the destination names
%    are assigned and we can disable \cs{HypDest@Use}.
%    \begin{macrocode}
\AtBeginDocument{%
  \let\HypDest@Use\@gobble
}
%    \end{macrocode}
%
%    \begin{macro}{\HypDest@MarkUsed}
%    Destinations that are actually used are marked by \cs{HypDest@MarkUsed}.
%    \cs{nofiles} is respected.
%    \begin{macrocode}
\def\HypDest@MarkUsed#1{%
  \HypDest@VerboseInfo{%
    MarkUsed: (#1)%
  }%
  \if@filesw
    \immediate\write\@auxout{%
      \string\HypDest@Use{\pdf@escapehex{#1}}%
    }%
  \fi
}%
%    \end{macrocode}
%    \end{macro}
%
% \subsection{Redefinition of \xpackage{hyperref}'s hooks}
%
%    Package \xpackage{hyperref} can be loaded later, therefore
%    we redefine \xpackage{hyperref}'s macros at |\begin{document}|.
%    \begin{macrocode}
\HypDest@PrependDocument{%
%    \end{macrocode}
%
%    Check hyperref version.
%    \begin{macrocode}
  \@ifpackagelater{hyperref}{2006/06/01}{}{%
    \PackageError{hypdestopt}{%
      hyperref 2006/06/01 v6.75a or later is required%
    }\@ehc
  }%
%    \end{macrocode}
%
% \subsubsection{Destination setting}
%
%    \begin{macrocode}
  \ifHypDest@name
    \let\HypDest@Org@DestName\Hy@DestName
    \renewcommand*{\Hy@DestName}[2]{%
      \EdefUnescapeString\HypDest@temp{#1}%
      \@ifundefined{\HypDest@Prefix\HypDest@temp}{%
        \HypDest@VerboseInfo{%
          DestName: (\HypDest@temp) unused%
        }%
      }{%
        \HypDest@Org@DestName{%
          \csname\HypDest@Prefix\HypDest@temp\endcsname
        }{#2}%
        \HypDest@VerboseInfo{%
          DestName: (\HypDest@temp) %
          0x\pdf@escapehex{%
            \csname\HypDest@Prefix\HypDest@temp\endcsname
          }%
        }%
      }%
    }%
  \else
    \renewcommand*{\Hy@DestName}[2]{%
      \EdefUnescapeString\HypDest@temp{#1}%
      \@ifundefined{\HypDest@Prefix\HypDest@temp}{%
        \HypDest@VerboseInfo{%
          DestName: (\HypDest@temp) unused%
        }%
      }{%
        \pdfdest num%
        \csname\HypDest@Prefix\HypDest@temp\endcsname#2\relax
        \HypDest@VerboseInfo{%
          DestName: (\HypDest@temp) %
          num \csname\HypDest@Prefix\HypDest@temp\endcsname
        }%
      }%
    }%
  \fi
%    \end{macrocode}
%
% \subsubsection{Links}
%
%    \begin{macrocode}
  \let\HypDest@Org@StartlinkName\Hy@StartlinkName
  \ifHypDest@name
    \renewcommand*{\Hy@StartlinkName}[2]{%
      \HypDest@MarkUsed{#2}%
      \HypDest@Org@StartlinkName{#1}{%
        \@ifundefined{\HypDest@Prefix#2}{%
          #2%
        }{%
          \csname\HypDest@Prefix#2\endcsname
        }%
      }%
    }%
  \else
    \renewcommand*{\Hy@StartlinkName}[2]{%
      \HypDest@MarkUsed{#2}%
      \@ifundefined{\HypDest@Prefix#2}{%
        \HypDest@Org@StartlinkName{#1}{#2}%
      }{%
        \pdfstartlink attr{#1}%
                      goto num\csname\HypDest@Prefix#2\endcsname
        \relax
      }%
    }%
  \fi
%    \end{macrocode}
%
% \subsubsection{Outlines of package \xpackage{hyperref}}
%
%    \begin{macrocode}
  \let\HypDest@Org@OutlineName\Hy@OutlineName
  \ifHypDest@name
    \renewcommand*{\Hy@OutlineName}[4]{%
      \HypDest@Org@OutlineName{#1}{%
        \@ifundefined{\HypDest@Prefix#2}{%
          #2%
        }{%
          \csname\HypDest@Prefix#2\endcsname
        }%
      }{#3}{#4}%
    }%
  \else
    \renewcommand*{\Hy@OutlineName}[4]{%
      \@ifundefined{\HypDest@Prefix#2}{%
        \HypDest@Org@OutlineName{#1}{#2}{#3}{#4}%
      }{%
        \pdfoutline goto num\csname\HypDest@Prefix#2\endcsname
                    count#3{#4}%
      }%
    }%
  \fi
%    \end{macrocode}
%    Because \cs{Hy@OutlineName} is called after the \xfile{.out} file
%    is written in the previous run. Therefore we mark the destination
%    earlier in \cs{@@writetorep}.
%    \begin{macrocode}
  \let\HypDest@Org@@writetorep\@@writetorep
  \renewcommand*{\@@writetorep}[5]{%
    \begingroup
      \edef\Hy@tempa{#5}%
      \ifx\Hy@tempa\Hy@bookmarkstype
        \HypDest@MarkUsed{#3}%
      \fi
    \endgroup
    \HypDest@Org@@writetorep{#1}{#2}{#3}{#4}{#5}%
  }%
%    \end{macrocode}
%
% \subsubsection{Outlines of package \xpackage{bookmark}}
%
%    \begin{macrocode}
  \@ifpackageloaded{bookmark}{%
    \@ifpackagelater{bookmark}{2008/08/08}{%
      \renewcommand*{\BKM@DefGotoNameAction}[2]{%
        \@ifundefined{\HypDest@Prefix#2}{%
          \edef#1{goto name{hypdestopt\string :unknown}}%
        }{%
          \ifHypDest@name
            \edef#1{goto name{\csname\HypDest@Prefix#2\endcsname}}%
          \else
            \edef#1{goto num\csname\HypDest@Prefix#2\endcsname}%
          \fi
        }%
      }%
      \def\BKM@HypDestOptHook{%
        \ifx\BKM@dest\@empty
        \else
          \ifx\BKM@gotor\@empty
            \HypDest@MarkUsed\BKM@dest
          \fi
        \fi
      }%
    }{%
      \@PackageError{hypdestopt}{%
        Package `bookmark' is too old.\MessageBreak
        Version 2008/08/08 or later is needed%
      }\@ehc
    }%
  }{}%
%    \end{macrocode}
%
%    \begin{macrocode}
}
%    \end{macrocode}
%
%
%    \begin{macrocode}
%</package>
%    \end{macrocode}
%
% \section{Installation}
%
% \subsection{Download}
%
% \paragraph{Package.} This package is available on
% CTAN\footnote{\url{ftp://ftp.ctan.org/tex-archive/}}:
% \begin{description}
% \item[\CTAN{macros/latex/contrib/oberdiek/hypdestopt.dtx}] The source file.
% \item[\CTAN{macros/latex/contrib/oberdiek/hypdestopt.pdf}] Documentation.
% \end{description}
%
%
% \paragraph{Bundle.} All the packages of the bundle `oberdiek'
% are also available in a TDS compliant ZIP archive. There
% the packages are already unpacked and the documentation files
% are generated. The files and directories obey the TDS standard.
% \begin{description}
% \item[\CTAN{install/macros/latex/contrib/oberdiek.tds.zip}]
% \end{description}
% \emph{TDS} refers to the standard ``A Directory Structure
% for \TeX\ Files'' (\CTAN{tds/tds.pdf}). Directories
% with \xfile{texmf} in their name are usually organized this way.
%
% \subsection{Bundle installation}
%
% \paragraph{Unpacking.} Unpack the \xfile{oberdiek.tds.zip} in the
% TDS tree (also known as \xfile{texmf} tree) of your choice.
% Example (linux):
% \begin{quote}
%   |unzip oberdiek.tds.zip -d ~/texmf|
% \end{quote}
%
% \paragraph{Script installation.}
% Check the directory \xfile{TDS:scripts/oberdiek/} for
% scripts that need further installation steps.
% Package \xpackage{attachfile2} comes with the Perl script
% \xfile{pdfatfi.pl} that should be installed in such a way
% that it can be called as \texttt{pdfatfi}.
% Example (linux):
% \begin{quote}
%   |chmod +x scripts/oberdiek/pdfatfi.pl|\\
%   |cp scripts/oberdiek/pdfatfi.pl /usr/local/bin/|
% \end{quote}
%
% \subsection{Package installation}
%
% \paragraph{Unpacking.} The \xfile{.dtx} file is a self-extracting
% \docstrip\ archive. The files are extracted by running the
% \xfile{.dtx} through \plainTeX:
% \begin{quote}
%   \verb|tex hypdestopt.dtx|
% \end{quote}
%
% \paragraph{TDS.} Now the different files must be moved into
% the different directories in your installation TDS tree
% (also known as \xfile{texmf} tree):
% \begin{quote}
% \def\t{^^A
% \begin{tabular}{@{}>{\ttfamily}l@{ $\rightarrow$ }>{\ttfamily}l@{}}
%   hypdestopt.sty & tex/latex/oberdiek/hypdestopt.sty\\
%   hypdestopt.pdf & doc/latex/oberdiek/hypdestopt.pdf\\
%   hypdestopt.dtx & source/latex/oberdiek/hypdestopt.dtx\\
% \end{tabular}^^A
% }^^A
% \sbox0{\t}^^A
% \ifdim\wd0>\linewidth
%   \begingroup
%     \advance\linewidth by\leftmargin
%     \advance\linewidth by\rightmargin
%   \edef\x{\endgroup
%     \def\noexpand\lw{\the\linewidth}^^A
%   }\x
%   \def\lwbox{^^A
%     \leavevmode
%     \hbox to \linewidth{^^A
%       \kern-\leftmargin\relax
%       \hss
%       \usebox0
%       \hss
%       \kern-\rightmargin\relax
%     }^^A
%   }^^A
%   \ifdim\wd0>\lw
%     \sbox0{\small\t}^^A
%     \ifdim\wd0>\linewidth
%       \ifdim\wd0>\lw
%         \sbox0{\footnotesize\t}^^A
%         \ifdim\wd0>\linewidth
%           \ifdim\wd0>\lw
%             \sbox0{\scriptsize\t}^^A
%             \ifdim\wd0>\linewidth
%               \ifdim\wd0>\lw
%                 \sbox0{\tiny\t}^^A
%                 \ifdim\wd0>\linewidth
%                   \lwbox
%                 \else
%                   \usebox0
%                 \fi
%               \else
%                 \lwbox
%               \fi
%             \else
%               \usebox0
%             \fi
%           \else
%             \lwbox
%           \fi
%         \else
%           \usebox0
%         \fi
%       \else
%         \lwbox
%       \fi
%     \else
%       \usebox0
%     \fi
%   \else
%     \lwbox
%   \fi
% \else
%   \usebox0
% \fi
% \end{quote}
% If you have a \xfile{docstrip.cfg} that configures and enables \docstrip's
% TDS installing feature, then some files can already be in the right
% place, see the documentation of \docstrip.
%
% \subsection{Refresh file name databases}
%
% If your \TeX~distribution
% (\teTeX, \mikTeX, \dots) relies on file name databases, you must refresh
% these. For example, \teTeX\ users run \verb|texhash| or
% \verb|mktexlsr|.
%
% \subsection{Some details for the interested}
%
% \paragraph{Attached source.}
%
% The PDF documentation on CTAN also includes the
% \xfile{.dtx} source file. It can be extracted by
% AcrobatReader 6 or higher. Another option is \textsf{pdftk},
% e.g. unpack the file into the current directory:
% \begin{quote}
%   \verb|pdftk hypdestopt.pdf unpack_files output .|
% \end{quote}
%
% \paragraph{Unpacking with \LaTeX.}
% The \xfile{.dtx} chooses its action depending on the format:
% \begin{description}
% \item[\plainTeX:] Run \docstrip\ and extract the files.
% \item[\LaTeX:] Generate the documentation.
% \end{description}
% If you insist on using \LaTeX\ for \docstrip\ (really,
% \docstrip\ does not need \LaTeX), then inform the autodetect routine
% about your intention:
% \begin{quote}
%   \verb|latex \let\install=y% \iffalse meta-comment
%
% File: hypdestopt.dtx
% Version: 2011/05/13 v2.3
% Info: Hyperref destination optimizer
%
% Copyright (C) 2006-2008, 2011 by
%    Heiko Oberdiek <heiko.oberdiek at googlemail.com>
%
% This work may be distributed and/or modified under the
% conditions of the LaTeX Project Public License, either
% version 1.3c of this license or (at your option) any later
% version. This version of this license is in
%    http://www.latex-project.org/lppl/lppl-1-3c.txt
% and the latest version of this license is in
%    http://www.latex-project.org/lppl.txt
% and version 1.3 or later is part of all distributions of
% LaTeX version 2005/12/01 or later.
%
% This work has the LPPL maintenance status "maintained".
%
% This Current Maintainer of this work is Heiko Oberdiek.
%
% This work consists of the main source file hypdestopt.dtx
% and the derived files
%    hypdestopt.sty, hypdestopt.pdf, hypdestopt.ins, hypdestopt.drv.
%
% Distribution:
%    CTAN:macros/latex/contrib/oberdiek/hypdestopt.dtx
%    CTAN:macros/latex/contrib/oberdiek/hypdestopt.pdf
%
% Unpacking:
%    (a) If hypdestopt.ins is present:
%           tex hypdestopt.ins
%    (b) Without hypdestopt.ins:
%           tex hypdestopt.dtx
%    (c) If you insist on using LaTeX
%           latex \let\install=y% \iffalse meta-comment
%
% File: hypdestopt.dtx
% Version: 2011/05/13 v2.3
% Info: Hyperref destination optimizer
%
% Copyright (C) 2006-2008, 2011 by
%    Heiko Oberdiek <heiko.oberdiek at googlemail.com>
%
% This work may be distributed and/or modified under the
% conditions of the LaTeX Project Public License, either
% version 1.3c of this license or (at your option) any later
% version. This version of this license is in
%    http://www.latex-project.org/lppl/lppl-1-3c.txt
% and the latest version of this license is in
%    http://www.latex-project.org/lppl.txt
% and version 1.3 or later is part of all distributions of
% LaTeX version 2005/12/01 or later.
%
% This work has the LPPL maintenance status "maintained".
%
% This Current Maintainer of this work is Heiko Oberdiek.
%
% This work consists of the main source file hypdestopt.dtx
% and the derived files
%    hypdestopt.sty, hypdestopt.pdf, hypdestopt.ins, hypdestopt.drv.
%
% Distribution:
%    CTAN:macros/latex/contrib/oberdiek/hypdestopt.dtx
%    CTAN:macros/latex/contrib/oberdiek/hypdestopt.pdf
%
% Unpacking:
%    (a) If hypdestopt.ins is present:
%           tex hypdestopt.ins
%    (b) Without hypdestopt.ins:
%           tex hypdestopt.dtx
%    (c) If you insist on using LaTeX
%           latex \let\install=y\input{hypdestopt.dtx}
%        (quote the arguments according to the demands of your shell)
%
% Documentation:
%    (a) If hypdestopt.drv is present:
%           latex hypdestopt.drv
%    (b) Without hypdestopt.drv:
%           latex hypdestopt.dtx; ...
%    The class ltxdoc loads the configuration file ltxdoc.cfg
%    if available. Here you can specify further options, e.g.
%    use A4 as paper format:
%       \PassOptionsToClass{a4paper}{article}
%
%    Programm calls to get the documentation (example):
%       pdflatex hypdestopt.dtx
%       makeindex -s gind.ist hypdestopt.idx
%       pdflatex hypdestopt.dtx
%       makeindex -s gind.ist hypdestopt.idx
%       pdflatex hypdestopt.dtx
%
% Installation:
%    TDS:tex/latex/oberdiek/hypdestopt.sty
%    TDS:doc/latex/oberdiek/hypdestopt.pdf
%    TDS:source/latex/oberdiek/hypdestopt.dtx
%
%<*ignore>
\begingroup
  \catcode123=1 %
  \catcode125=2 %
  \def\x{LaTeX2e}%
\expandafter\endgroup
\ifcase 0\ifx\install y1\fi\expandafter
         \ifx\csname processbatchFile\endcsname\relax\else1\fi
         \ifx\fmtname\x\else 1\fi\relax
\else\csname fi\endcsname
%</ignore>
%<*install>
\input docstrip.tex
\Msg{************************************************************************}
\Msg{* Installation}
\Msg{* Package: hypdestopt 2011/05/13 v2.3 Hyperref destination optimizer (HO)}
\Msg{************************************************************************}

\keepsilent
\askforoverwritefalse

\let\MetaPrefix\relax
\preamble

This is a generated file.

Project: hypdestopt
Version: 2011/05/13 v2.3

Copyright (C) 2006-2008, 2011 by
   Heiko Oberdiek <heiko.oberdiek at googlemail.com>

This work may be distributed and/or modified under the
conditions of the LaTeX Project Public License, either
version 1.3c of this license or (at your option) any later
version. This version of this license is in
   http://www.latex-project.org/lppl/lppl-1-3c.txt
and the latest version of this license is in
   http://www.latex-project.org/lppl.txt
and version 1.3 or later is part of all distributions of
LaTeX version 2005/12/01 or later.

This work has the LPPL maintenance status "maintained".

This Current Maintainer of this work is Heiko Oberdiek.

This work consists of the main source file hypdestopt.dtx
and the derived files
   hypdestopt.sty, hypdestopt.pdf, hypdestopt.ins, hypdestopt.drv.

\endpreamble
\let\MetaPrefix\DoubleperCent

\generate{%
  \file{hypdestopt.ins}{\from{hypdestopt.dtx}{install}}%
  \file{hypdestopt.drv}{\from{hypdestopt.dtx}{driver}}%
  \usedir{tex/latex/oberdiek}%
  \file{hypdestopt.sty}{\from{hypdestopt.dtx}{package}}%
  \nopreamble
  \nopostamble
  \usedir{source/latex/oberdiek/catalogue}%
  \file{hypdestopt.xml}{\from{hypdestopt.dtx}{catalogue}}%
}

\catcode32=13\relax% active space
\let =\space%
\Msg{************************************************************************}
\Msg{*}
\Msg{* To finish the installation you have to move the following}
\Msg{* file into a directory searched by TeX:}
\Msg{*}
\Msg{*     hypdestopt.sty}
\Msg{*}
\Msg{* To produce the documentation run the file `hypdestopt.drv'}
\Msg{* through LaTeX.}
\Msg{*}
\Msg{* Happy TeXing!}
\Msg{*}
\Msg{************************************************************************}

\endbatchfile
%</install>
%<*ignore>
\fi
%</ignore>
%<*driver>
\NeedsTeXFormat{LaTeX2e}
\ProvidesFile{hypdestopt.drv}%
  [2011/05/13 v2.3 Hyperref destination optimizer (HO)]%
\documentclass{ltxdoc}
\usepackage{holtxdoc}[2011/11/22]
\begin{document}
  \DocInput{hypdestopt.dtx}%
\end{document}
%</driver>
% \fi
%
% \CheckSum{565}
%
% \CharacterTable
%  {Upper-case    \A\B\C\D\E\F\G\H\I\J\K\L\M\N\O\P\Q\R\S\T\U\V\W\X\Y\Z
%   Lower-case    \a\b\c\d\e\f\g\h\i\j\k\l\m\n\o\p\q\r\s\t\u\v\w\x\y\z
%   Digits        \0\1\2\3\4\5\6\7\8\9
%   Exclamation   \!     Double quote  \"     Hash (number) \#
%   Dollar        \$     Percent       \%     Ampersand     \&
%   Acute accent  \'     Left paren    \(     Right paren   \)
%   Asterisk      \*     Plus          \+     Comma         \,
%   Minus         \-     Point         \.     Solidus       \/
%   Colon         \:     Semicolon     \;     Less than     \<
%   Equals        \=     Greater than  \>     Question mark \?
%   Commercial at \@     Left bracket  \[     Backslash     \\
%   Right bracket \]     Circumflex    \^     Underscore    \_
%   Grave accent  \`     Left brace    \{     Vertical bar  \|
%   Right brace   \}     Tilde         \~}
%
% \GetFileInfo{hypdestopt.drv}
%
% \title{The \xpackage{hypdestopt} package}
% \date{2011/05/13 v2.3}
% \author{Heiko Oberdiek\\\xemail{heiko.oberdiek at googlemail.com}}
%
% \maketitle
%
% \begin{abstract}
% Package \xpackage{hypdestopt} supports \xpackage{hyperref}'s
% \xoption{pdftex} driver. It removes unnecessary destinations
% and shortens the destination names or uses numbered destinations
% to get smaller PDF files.
% \end{abstract}
%
% \tableofcontents
%
% \section{User interface}
%
% \subsection{Introduction}
%
% Before PDF-1.5 annotations and destinations cannot be compressed.
% If the destination names are not needed for external use, the
% file size can be decreased by the following means:
% \begin{itemize}
% \item Unused destinations are removed.
% \item The destination names are shortened (option \xoption{name}).
% \item Using numbered destinations (option \xoption{num}).
% \end{itemize}
%
% \subsection{Requirements}
%
% \begin{itemize}
% \item Package \xpackage{hyperref} 2006/06/01 v6.75a or newer
%       (\cite{hyperref}).
% \item Package \xpackage{alphalph} 2006/05/30 v1.4 or newer
%       (\cite{alphalph}), if option \xoption{name} is used.
% \item Package \xpackage{ifpdf} (\cite{ifpdf}).
% \item \pdfTeX\ 1.30.0 or newer.
% \item \pdfTeX\ in PDF mode.
% \item \eTeX\ extensions enabled.
% \item Probably an additional compile run of \pdfLaTeX\ is necessary.
% \end{itemize}
%
% In the first compile runs you can get warnings such as:
%\begin{quote}
%\begin{verbatim}
%! pdfTeX warning (dest): name{...} has been referenced ...
%\end{verbatim}
%\end{quote}
% These warnings should vanish in later compile runs.
% However these warnings also can occur without this package.
% The package does not cure them, thus these warnings will remain,
% but the destination name can be different. In such cases test
% without package, too.
%
% \subsection{Use}
%
% If the requirements are met, load the package:
%\begin{quote}
%\verb|\usepackage{hypdestopt}|
%\end{quote}
%
% The following options are supported:
% \begin{description}
% \item[\xoption{verbose}:] Verbose debug output is enabled and written
%   in the protocol file.
% \item[\xoption{num}:] Numbered destinations are used. The file size
%   is smaller, because names are no longer used.
%   This is the default.
% \item[\xoption{name}:] Destinations are identified by names.
% \end{description}
%
% \subsection{Limitations}
%
% \begin{itemize}
% \item Forget this package, if you need preserved destination names.
% \item Destination name strings use all bytes (0..255) except
%       the carriage return (13), left parenthesis (40), right
%       parenthesis (41), and backslash (92), because they
%       must be quoted in general and therefore occupy two bytes
%       instead of one.
%
%       Further the zero byte (0) is avoided for programs
%       that implement strings using zero terminated C strings.
%       And 255 (0xFF) is avoided to get rid of a possible
%       unicode marker at the begin.
%
%       So far I have not seen problems with:
%       \begin{itemize}
%       \item AcrobatReader 5.08/Linux
%       \item AcrobatReader 7.0/Linux
%       \item xpdf 3.00
%       \item Ghostscript 8.50
%       \item gv 3.5.8
%       \item GSview 4.6
%       \end{itemize}
%       But I have not tested all and all possible PDF viewers.
% \item Use of named destinations (\cs{pdfdest}, \cs{pdfoutline},
%       \cs{pdfstartlink}, \dots) that are not supported by this
%       package.
% \item Currently only \xpackage{hyperref} with \pdfTeX\ in PDF
%       mode is supported.
% \end{itemize}
%
% \subsection{Future}
%
% A more general approach is a PDF postprocessor that takes
% a PDF file, performs some transformations and writes the
% result in a more optimized PDF file. Then it does not depend,
% how the original PDF file was generated and further improvements
% are easier to apply. For example, the destination names could be sorted:
% often used destination names would then be shorter than seldom used ones.
%
% \StopEventually{
% }
%
% \section{Implementation}
%
% \subsection{Identification}
%
%    \begin{macrocode}
%<*package>
\NeedsTeXFormat{LaTeX2e}
\ProvidesPackage{hypdestopt}%
  [2011/05/13 v2.3 Hyperref destination optimizer (HO)]%
%    \end{macrocode}
%
% \subsection{Options}
%
% \subsubsection{Option \xoption{verbose}}
%
%    \begin{macrocode}
\newif\ifHypDest@Verbose
\DeclareOption{verbose}{\HypDest@Verbosetrue}
%    \end{macrocode}
%
%    \begin{macro}{\HypDest@VerboseInfo}
%    Wrapper for verbose messages.
%    \begin{macrocode}
\def\HypDest@VerboseInfo#1{%
  \ifHypDest@Verbose
    \PackageInfo{hypdestopt}{#1}%
  \fi
}
%    \end{macrocode}
%    \end{macro}
%
% \subsubsection{Options \xoption{num} and \xoption{name}}
%
%    The options \xoption{num} or \xoption{name} specify
%    the method, how destinations are referenced (by name or
%    number). Default is option \xoption{num}.
%    \begin{macrocode}
\newif\ifHypDest@name
\DeclareOption{num}{\HypDest@namefalse}
\DeclareOption{name}{\HypDest@nametrue}
%    \end{macrocode}
%
%    \begin{macrocode}
\ProcessOptions*\relax
%    \end{macrocode}
%
% \subsection{Check requirements}
%
%    First \pdfTeX\ must running in PDF mode.
%    \begin{macrocode}
\RequirePackage{ifpdf}[2007/09/09]
\RequirePackage{pdftexcmds}[2007/11/11]
\ifpdf
\else
  \PackageError{hypdestopt}{%
    This package requires pdfTeX in PDF mode%
  }\@ehc
  \expandafter\endinput
\fi
%    \end{macrocode}
%    The version of \pdfTeX\ must not be too old, because
%    \cs{pdfescapehex} and \cs{pdfunescapehex} are used.
%    \begin{macrocode}
\begingroup\expandafter\expandafter\expandafter\endgroup
\expandafter\ifx\csname pdf@escapehex\endcsname\relax
  \PackageError{hypdestopt}{%
    This pdfTeX is too old, at least 1.30.0 is required%
  }\@ehc
  \expandafter\endinput
\fi
%    \end{macrocode}
%    Features of \eTeX\ are used, e.g. \cs{numexpr}.
%    \begin{macrocode}
\begingroup\expandafter\expandafter\expandafter\endgroup
\expandafter\ifx\csname numexpr\endcsname\relax
  \PackageError{hypdestopt}{%
    e-TeX features are missing%
  }\@ehc
  \expandafter\endinput
\fi
%    \end{macrocode}
%    Package \xpackage{alphalph} provides \cs{newalphalph} since
%    version 2006/05/30 v1.4.
%    \begin{macrocode}
\ifHypDest@name
  \RequirePackage{alphalph}[2006/05/30]%
\fi
%    \end{macrocode}
%    \begin{macrocode}
\RequirePackage{auxhook}[2009/12/14]
\RequirePackage{pdfescape}[2007/04/21]
%    \end{macrocode}
%
% \subsection{Preamble for auxiliary file}
%
%    Provide dummy definitions for the macros that are used in the
%    auxiliary files. If the package is used no longer, then these
%    commands will not generate errors.
%
%    \begin{macro}{\HypDest@PrependDocument}
%    We add our stuff in front of the \cs{AtBeginDocument} hook
%    to ensure that we are before \xpackage{hyperref}'s stuff.
%    \begin{macrocode}
\long\def\HypDest@PrependDocument#1{%
  \begingroup
    \toks\z@{#1}%
    \toks\tw@\expandafter{\@begindocumenthook}%
    \xdef\@begindocumenthook{\the\toks\z@\the\toks\tw@}%
  \endgroup
}
%    \end{macrocode}
%    \end{macro}
%    \begin{macrocode}
\AddLineBeginAux{%
  \string\providecommand{\string\HypDest@Use}[1]{}%
}
%    \end{macrocode}
%
% \subsection{Generation of destination names}
%
%    Counter |HypDest| is used for identifying destinations.
%    \begin{macrocode}
\newcounter{HypDest}
%    \end{macrocode}
%
%    \begin{macrocode}
\ifHypDest@name
%    \end{macrocode}
%
%    \begin{macro}{\HypDest@HexChar}
%    Destination names are generated by automatically
%    numbering with the help of package \xpackage{alphalph}.
%    \cs{HypDest@HexChar} converts a number of the range 1 until 252
%    into the hexadecimal representation of the string character.
%    \begin{macrocode}
  \def\HypDest@HexChar#1{%
    \ifcase#1\or
%    \end{macrocode}
%    Avoid zero byte because of C strings in PDF viewer
%    applications.
%    \begin{macrocode}
      01\or 02\or 03\or 04\or 05\or 06\or 07\or
%    \end{macrocode}
%    Omit carriage return (13/\verb|^^0d|).
%    It needs quoting, otherwise it would be converted
%    to line feed (10/\verb|^^0a|).
%    \begin{macrocode}
      08\or 09\or 0A\or 0B\or 0C\or 0E\or 0F\or
      10\or 11\or 12\or 13\or 14\or 15\or 16\or 17\or
      18\or 19\or 1A\or 1B\or 1C\or 1D\or 1E\or 1F\or
      20\or 21\or 22\or 23\or 24\or 25\or 26\or 27\or
%    \end{macrocode}
%    Omit left and right parentheses (40/\verb|^^28|, 41/\verb|^^39|),
%    they need quoting in general.
%    \begin{macrocode}
      2A\or 2B\or 2C\or 2D\or 2E\or 2F\or
      30\or 31\or 32\or 33\or 34\or 35\or 36\or 37\or
      38\or 39\or 3A\or 3B\or 3C\or 3D\or 3E\or 3F\or
      40\or 41\or 42\or 43\or 44\or 45\or 46\or 47\or
      48\or 49\or 4A\or 4B\or 4C\or 4D\or 4E\or 4F\or
      50\or 51\or 52\or 53\or 54\or 55\or 56\or 57\or
%    \end{macrocode}
%    Omit backslash (92/\verb|^^5C|), it needs quoting.
%    \begin{macrocode}
      58\or 59\or 5A\or 5B\or 5D\or 5E\or 5F\or
      60\or 61\or 62\or 63\or 64\or 65\or 66\or 67\or
      68\or 69\or 6A\or 6B\or 6C\or 6D\or 6E\or 6F\or
      70\or 71\or 72\or 73\or 74\or 75\or 76\or 77\or
      78\or 79\or 7A\or 7B\or 7C\or 7D\or 7E\or 7F\or
      80\or 81\or 82\or 83\or 84\or 85\or 86\or 87\or
      88\or 89\or 8A\or 8B\or 8C\or 8D\or 8E\or 8F\or
      90\or 91\or 92\or 93\or 94\or 95\or 96\or 97\or
      98\or 99\or 9A\or 9B\or 9C\or 9D\or 9E\or 9F\or
      A0\or A1\or A2\or A3\or A4\or A5\or A6\or A7\or
      A8\or A9\or AA\or AB\or AC\or AD\or AE\or AF\or
      B0\or B1\or B2\or B3\or B4\or B5\or B6\or B7\or
      B8\or B9\or BA\or BB\or BC\or BD\or BE\or BF\or
      C0\or C1\or C2\or C3\or C4\or C5\or C6\or C7\or
      C8\or C9\or CA\or CB\or CC\or CD\or CE\or CF\or
      D0\or D1\or D2\or D3\or D4\or D5\or D6\or D7\or
      D8\or D9\or DA\or DB\or DC\or DD\or DE\or DF\or
      E0\or E1\or E2\or E3\or E4\or E5\or E6\or E7\or
      E8\or E9\or EA\or EB\or EC\or ED\or EE\or EF\or
      F0\or F1\or F2\or F3\or F4\or F5\or F6\or F7\or
%    \end{macrocode}
%    Avoid 255 (0xFF) to get rid of a possible unicode
%    marker at the begin of the string.
%    \begin{macrocode}
      F8\or F9\or FA\or FB\or FC\or FD\or FE%
    \fi
  }%
%    \end{macrocode}
%    \end{macro}
%    \begin{macro}{HypDest@HexString}
%    Now package \xpackage{alphalph} comes into play.
%    \cs{HypDest@HexString} is defined and converts
%    a positive number into a string, given in hexadecimal
%    representation.
%    \begin{macrocode}
  \newalphalph\HypDest@HexString\HypDest@HexChar{250}%
%    \end{macrocode}
%    \end{macro}
%    \begin{macro}{\theHypDest}
%    For use, the hexadecimal string is converted back.
%    \begin{macrocode}
  \renewcommand*{\theHypDest}{%
    \pdf@unescapehex{\HypDest@HexString{\value{HypDest}}}%
  }%
%    \end{macrocode}
%    \end{macro}
%
%    With option \xoption{num} we use the number directly.
%    \begin{macrocode}
\else
  \renewcommand*{\theHypDest}{%
    \number\value{HypDest}%
  }%
\fi
%    \end{macrocode}
%
% \subsection{Assign destination names}
%
%    \begin{macro}{\HypDest@Prefix}
%    The new destination names are remembered in macros whose names
%    start with prefix \cs{HypDest@Prefix}.
%    \begin{macrocode}
\edef\HypDest@Prefix{HypDest\string:}
%    \end{macrocode}
%    \end{macro}
%
%    \begin{macro}{\HypDest@Use}
%    During the first read of the auxiliary files, the used destinations
%    get fresh generated short destination names. Also for the old
%    destination names we use the hexadecimal representation. That
%    avoid problems with arbitrary names.
%    \begin{macrocode}
\def\HypDest@Use#1{%
  \begingroup
    \edef\x{%
      \expandafter\noexpand
      \csname\HypDest@Prefix\pdf@unescapehex{#1}\endcsname
    }%
    \expandafter\ifx\x\relax
      \stepcounter{HypDest}%
      \expandafter\xdef\x{\theHypDest}%
      \let\on@line\@empty
      \ifHypDest@name
        \HypDest@VerboseInfo{%
          Use: (\pdf@unescapehex{#1}) -\string> %
          0x\pdf@escapehex{\x} (\number\value{HypDest})%
        }%
      \else
        \HypDest@VerboseInfo{%
          Use: (\pdf@unescapehex{#1}) -\string> num \x
        }%
      \fi
    \fi
  \endgroup
}
%    \end{macrocode}
%    \end{macro}
%
%    After the first \xfile{.aux} file processing the destination names
%    are assigned and we can disable \cs{HypDest@Use}.
%    \begin{macrocode}
\AtBeginDocument{%
  \let\HypDest@Use\@gobble
}
%    \end{macrocode}
%
%    \begin{macro}{\HypDest@MarkUsed}
%    Destinations that are actually used are marked by \cs{HypDest@MarkUsed}.
%    \cs{nofiles} is respected.
%    \begin{macrocode}
\def\HypDest@MarkUsed#1{%
  \HypDest@VerboseInfo{%
    MarkUsed: (#1)%
  }%
  \if@filesw
    \immediate\write\@auxout{%
      \string\HypDest@Use{\pdf@escapehex{#1}}%
    }%
  \fi
}%
%    \end{macrocode}
%    \end{macro}
%
% \subsection{Redefinition of \xpackage{hyperref}'s hooks}
%
%    Package \xpackage{hyperref} can be loaded later, therefore
%    we redefine \xpackage{hyperref}'s macros at |\begin{document}|.
%    \begin{macrocode}
\HypDest@PrependDocument{%
%    \end{macrocode}
%
%    Check hyperref version.
%    \begin{macrocode}
  \@ifpackagelater{hyperref}{2006/06/01}{}{%
    \PackageError{hypdestopt}{%
      hyperref 2006/06/01 v6.75a or later is required%
    }\@ehc
  }%
%    \end{macrocode}
%
% \subsubsection{Destination setting}
%
%    \begin{macrocode}
  \ifHypDest@name
    \let\HypDest@Org@DestName\Hy@DestName
    \renewcommand*{\Hy@DestName}[2]{%
      \EdefUnescapeString\HypDest@temp{#1}%
      \@ifundefined{\HypDest@Prefix\HypDest@temp}{%
        \HypDest@VerboseInfo{%
          DestName: (\HypDest@temp) unused%
        }%
      }{%
        \HypDest@Org@DestName{%
          \csname\HypDest@Prefix\HypDest@temp\endcsname
        }{#2}%
        \HypDest@VerboseInfo{%
          DestName: (\HypDest@temp) %
          0x\pdf@escapehex{%
            \csname\HypDest@Prefix\HypDest@temp\endcsname
          }%
        }%
      }%
    }%
  \else
    \renewcommand*{\Hy@DestName}[2]{%
      \EdefUnescapeString\HypDest@temp{#1}%
      \@ifundefined{\HypDest@Prefix\HypDest@temp}{%
        \HypDest@VerboseInfo{%
          DestName: (\HypDest@temp) unused%
        }%
      }{%
        \pdfdest num%
        \csname\HypDest@Prefix\HypDest@temp\endcsname#2\relax
        \HypDest@VerboseInfo{%
          DestName: (\HypDest@temp) %
          num \csname\HypDest@Prefix\HypDest@temp\endcsname
        }%
      }%
    }%
  \fi
%    \end{macrocode}
%
% \subsubsection{Links}
%
%    \begin{macrocode}
  \let\HypDest@Org@StartlinkName\Hy@StartlinkName
  \ifHypDest@name
    \renewcommand*{\Hy@StartlinkName}[2]{%
      \HypDest@MarkUsed{#2}%
      \HypDest@Org@StartlinkName{#1}{%
        \@ifundefined{\HypDest@Prefix#2}{%
          #2%
        }{%
          \csname\HypDest@Prefix#2\endcsname
        }%
      }%
    }%
  \else
    \renewcommand*{\Hy@StartlinkName}[2]{%
      \HypDest@MarkUsed{#2}%
      \@ifundefined{\HypDest@Prefix#2}{%
        \HypDest@Org@StartlinkName{#1}{#2}%
      }{%
        \pdfstartlink attr{#1}%
                      goto num\csname\HypDest@Prefix#2\endcsname
        \relax
      }%
    }%
  \fi
%    \end{macrocode}
%
% \subsubsection{Outlines of package \xpackage{hyperref}}
%
%    \begin{macrocode}
  \let\HypDest@Org@OutlineName\Hy@OutlineName
  \ifHypDest@name
    \renewcommand*{\Hy@OutlineName}[4]{%
      \HypDest@Org@OutlineName{#1}{%
        \@ifundefined{\HypDest@Prefix#2}{%
          #2%
        }{%
          \csname\HypDest@Prefix#2\endcsname
        }%
      }{#3}{#4}%
    }%
  \else
    \renewcommand*{\Hy@OutlineName}[4]{%
      \@ifundefined{\HypDest@Prefix#2}{%
        \HypDest@Org@OutlineName{#1}{#2}{#3}{#4}%
      }{%
        \pdfoutline goto num\csname\HypDest@Prefix#2\endcsname
                    count#3{#4}%
      }%
    }%
  \fi
%    \end{macrocode}
%    Because \cs{Hy@OutlineName} is called after the \xfile{.out} file
%    is written in the previous run. Therefore we mark the destination
%    earlier in \cs{@@writetorep}.
%    \begin{macrocode}
  \let\HypDest@Org@@writetorep\@@writetorep
  \renewcommand*{\@@writetorep}[5]{%
    \begingroup
      \edef\Hy@tempa{#5}%
      \ifx\Hy@tempa\Hy@bookmarkstype
        \HypDest@MarkUsed{#3}%
      \fi
    \endgroup
    \HypDest@Org@@writetorep{#1}{#2}{#3}{#4}{#5}%
  }%
%    \end{macrocode}
%
% \subsubsection{Outlines of package \xpackage{bookmark}}
%
%    \begin{macrocode}
  \@ifpackageloaded{bookmark}{%
    \@ifpackagelater{bookmark}{2008/08/08}{%
      \renewcommand*{\BKM@DefGotoNameAction}[2]{%
        \@ifundefined{\HypDest@Prefix#2}{%
          \edef#1{goto name{hypdestopt\string :unknown}}%
        }{%
          \ifHypDest@name
            \edef#1{goto name{\csname\HypDest@Prefix#2\endcsname}}%
          \else
            \edef#1{goto num\csname\HypDest@Prefix#2\endcsname}%
          \fi
        }%
      }%
      \def\BKM@HypDestOptHook{%
        \ifx\BKM@dest\@empty
        \else
          \ifx\BKM@gotor\@empty
            \HypDest@MarkUsed\BKM@dest
          \fi
        \fi
      }%
    }{%
      \@PackageError{hypdestopt}{%
        Package `bookmark' is too old.\MessageBreak
        Version 2008/08/08 or later is needed%
      }\@ehc
    }%
  }{}%
%    \end{macrocode}
%
%    \begin{macrocode}
}
%    \end{macrocode}
%
%
%    \begin{macrocode}
%</package>
%    \end{macrocode}
%
% \section{Installation}
%
% \subsection{Download}
%
% \paragraph{Package.} This package is available on
% CTAN\footnote{\url{ftp://ftp.ctan.org/tex-archive/}}:
% \begin{description}
% \item[\CTAN{macros/latex/contrib/oberdiek/hypdestopt.dtx}] The source file.
% \item[\CTAN{macros/latex/contrib/oberdiek/hypdestopt.pdf}] Documentation.
% \end{description}
%
%
% \paragraph{Bundle.} All the packages of the bundle `oberdiek'
% are also available in a TDS compliant ZIP archive. There
% the packages are already unpacked and the documentation files
% are generated. The files and directories obey the TDS standard.
% \begin{description}
% \item[\CTAN{install/macros/latex/contrib/oberdiek.tds.zip}]
% \end{description}
% \emph{TDS} refers to the standard ``A Directory Structure
% for \TeX\ Files'' (\CTAN{tds/tds.pdf}). Directories
% with \xfile{texmf} in their name are usually organized this way.
%
% \subsection{Bundle installation}
%
% \paragraph{Unpacking.} Unpack the \xfile{oberdiek.tds.zip} in the
% TDS tree (also known as \xfile{texmf} tree) of your choice.
% Example (linux):
% \begin{quote}
%   |unzip oberdiek.tds.zip -d ~/texmf|
% \end{quote}
%
% \paragraph{Script installation.}
% Check the directory \xfile{TDS:scripts/oberdiek/} for
% scripts that need further installation steps.
% Package \xpackage{attachfile2} comes with the Perl script
% \xfile{pdfatfi.pl} that should be installed in such a way
% that it can be called as \texttt{pdfatfi}.
% Example (linux):
% \begin{quote}
%   |chmod +x scripts/oberdiek/pdfatfi.pl|\\
%   |cp scripts/oberdiek/pdfatfi.pl /usr/local/bin/|
% \end{quote}
%
% \subsection{Package installation}
%
% \paragraph{Unpacking.} The \xfile{.dtx} file is a self-extracting
% \docstrip\ archive. The files are extracted by running the
% \xfile{.dtx} through \plainTeX:
% \begin{quote}
%   \verb|tex hypdestopt.dtx|
% \end{quote}
%
% \paragraph{TDS.} Now the different files must be moved into
% the different directories in your installation TDS tree
% (also known as \xfile{texmf} tree):
% \begin{quote}
% \def\t{^^A
% \begin{tabular}{@{}>{\ttfamily}l@{ $\rightarrow$ }>{\ttfamily}l@{}}
%   hypdestopt.sty & tex/latex/oberdiek/hypdestopt.sty\\
%   hypdestopt.pdf & doc/latex/oberdiek/hypdestopt.pdf\\
%   hypdestopt.dtx & source/latex/oberdiek/hypdestopt.dtx\\
% \end{tabular}^^A
% }^^A
% \sbox0{\t}^^A
% \ifdim\wd0>\linewidth
%   \begingroup
%     \advance\linewidth by\leftmargin
%     \advance\linewidth by\rightmargin
%   \edef\x{\endgroup
%     \def\noexpand\lw{\the\linewidth}^^A
%   }\x
%   \def\lwbox{^^A
%     \leavevmode
%     \hbox to \linewidth{^^A
%       \kern-\leftmargin\relax
%       \hss
%       \usebox0
%       \hss
%       \kern-\rightmargin\relax
%     }^^A
%   }^^A
%   \ifdim\wd0>\lw
%     \sbox0{\small\t}^^A
%     \ifdim\wd0>\linewidth
%       \ifdim\wd0>\lw
%         \sbox0{\footnotesize\t}^^A
%         \ifdim\wd0>\linewidth
%           \ifdim\wd0>\lw
%             \sbox0{\scriptsize\t}^^A
%             \ifdim\wd0>\linewidth
%               \ifdim\wd0>\lw
%                 \sbox0{\tiny\t}^^A
%                 \ifdim\wd0>\linewidth
%                   \lwbox
%                 \else
%                   \usebox0
%                 \fi
%               \else
%                 \lwbox
%               \fi
%             \else
%               \usebox0
%             \fi
%           \else
%             \lwbox
%           \fi
%         \else
%           \usebox0
%         \fi
%       \else
%         \lwbox
%       \fi
%     \else
%       \usebox0
%     \fi
%   \else
%     \lwbox
%   \fi
% \else
%   \usebox0
% \fi
% \end{quote}
% If you have a \xfile{docstrip.cfg} that configures and enables \docstrip's
% TDS installing feature, then some files can already be in the right
% place, see the documentation of \docstrip.
%
% \subsection{Refresh file name databases}
%
% If your \TeX~distribution
% (\teTeX, \mikTeX, \dots) relies on file name databases, you must refresh
% these. For example, \teTeX\ users run \verb|texhash| or
% \verb|mktexlsr|.
%
% \subsection{Some details for the interested}
%
% \paragraph{Attached source.}
%
% The PDF documentation on CTAN also includes the
% \xfile{.dtx} source file. It can be extracted by
% AcrobatReader 6 or higher. Another option is \textsf{pdftk},
% e.g. unpack the file into the current directory:
% \begin{quote}
%   \verb|pdftk hypdestopt.pdf unpack_files output .|
% \end{quote}
%
% \paragraph{Unpacking with \LaTeX.}
% The \xfile{.dtx} chooses its action depending on the format:
% \begin{description}
% \item[\plainTeX:] Run \docstrip\ and extract the files.
% \item[\LaTeX:] Generate the documentation.
% \end{description}
% If you insist on using \LaTeX\ for \docstrip\ (really,
% \docstrip\ does not need \LaTeX), then inform the autodetect routine
% about your intention:
% \begin{quote}
%   \verb|latex \let\install=y\input{hypdestopt.dtx}|
% \end{quote}
% Do not forget to quote the argument according to the demands
% of your shell.
%
% \paragraph{Generating the documentation.}
% You can use both the \xfile{.dtx} or the \xfile{.drv} to generate
% the documentation. The process can be configured by the
% configuration file \xfile{ltxdoc.cfg}. For instance, put this
% line into this file, if you want to have A4 as paper format:
% \begin{quote}
%   \verb|\PassOptionsToClass{a4paper}{article}|
% \end{quote}
% An example follows how to generate the
% documentation with pdf\LaTeX:
% \begin{quote}
%\begin{verbatim}
%pdflatex hypdestopt.dtx
%makeindex -s gind.ist hypdestopt.idx
%pdflatex hypdestopt.dtx
%makeindex -s gind.ist hypdestopt.idx
%pdflatex hypdestopt.dtx
%\end{verbatim}
% \end{quote}
%
% \section{Catalogue}
%
% The following XML file can be used as source for the
% \href{http://mirror.ctan.org/help/Catalogue/catalogue.html}{\TeX\ Catalogue}.
% The elements \texttt{caption} and \texttt{description} are imported
% from the original XML file from the Catalogue.
% The name of the XML file in the Catalogue is \xfile{hypdestopt.xml}.
%    \begin{macrocode}
%<*catalogue>
<?xml version='1.0' encoding='us-ascii'?>
<!DOCTYPE entry SYSTEM 'catalogue.dtd'>
<entry datestamp='$Date$' modifier='$Author$' id='hypdestopt'>
  <name>hypdestopt</name>
  <caption>Hyperref destination optimizer.</caption>
  <authorref id='auth:oberdiek'/>
  <copyright owner='Heiko Oberdiek' year='2006-2008,2011'/>
  <license type='lppl1.3'/>
  <version number='2.3'/>
  <description>
    This package supports <xref refid='hyperref'>hyperref</xref>'s
    pdftex driver. It removes unnecessary destinations
    and shortens the destination names or uses numbered destinations
    to get smaller PDF files.
    <p/>
    The package is part of the <xref refid='oberdiek'>oberdiek</xref>
    bundle.
  </description>
  <documentation details='Package documentation'
      href='ctan:/macros/latex/contrib/oberdiek/hypdestopt.pdf'/>
  <ctan file='true' path='/macros/latex/contrib/oberdiek/hypdestopt.dtx'/>
  <miktex location='oberdiek'/>
  <texlive location='oberdiek'/>
  <install path='/macros/latex/contrib/oberdiek/oberdiek.tds.zip'/>
</entry>
%</catalogue>
%    \end{macrocode}
%
% \begin{thebibliography}{9}
%
% \bibitem{alphalph}
%   Heiko Oberdiek: \textit{The \xpackage{alphalph} package};
%   2006/05/30 v1.4;
%   \CTAN{macros/latex/contrib/oberdiek/alphalph.pdf}.
%
% \bibitem{hyperref}
%   Sebastian Rahtz, Heiko Oberdiek:
%   \textit{The \xpackage{hyperref} package};
%   2006/06/01 v6.75a;
%   \CTAN{macros/latex/contrib/hyperref/}.
%
% \bibitem{ifpdf}
%   Heiko Oberdiek: \textit{The \xpackage{ifpdf} package};
%   2006/02/20 v1.4;
%   \CTAN{macros/latex/contrib/oberdiek/ifpdf.pdf}.
%
% \end{thebibliography}
%
% \begin{History}
%   \begin{Version}{2006/06/01 v1.0}
%   \item
%     First version.
%   \end{Version}
%   \begin{Version}{2006/06/01 v2.0}
%   \item
%     New method for referencing destinations by number; an idea
%     proposed by Lars Hellstr\"om in the mailing list LATEX-L.
%   \item
%     Options \xoption{name} and \xoption{num} added.
%   \end{Version}
%   \begin{Version}{2007/11/11 v2.1}
%   \item
%     Use of package \xpackage{pdftexcmds} for \LuaTeX\ support.
%   \end{Version}
%   \begin{Version}{2008/08/08 v2.2}
%   \item
%     Support for package \xpackage{bookmark} added.
%   \end{Version}
%   \begin{Version}{2011/05/13 v2.3}
%   \item
%     Fix for \cs{Hy@DestName} if the destination name contains
%     special characters.
%   \item
%     Fix for option \xoption{name} and package \xpackage{bookmark}.
%   \end{Version}
% \end{History}
%
% \PrintIndex
%
% \Finale
\endinput

%        (quote the arguments according to the demands of your shell)
%
% Documentation:
%    (a) If hypdestopt.drv is present:
%           latex hypdestopt.drv
%    (b) Without hypdestopt.drv:
%           latex hypdestopt.dtx; ...
%    The class ltxdoc loads the configuration file ltxdoc.cfg
%    if available. Here you can specify further options, e.g.
%    use A4 as paper format:
%       \PassOptionsToClass{a4paper}{article}
%
%    Programm calls to get the documentation (example):
%       pdflatex hypdestopt.dtx
%       makeindex -s gind.ist hypdestopt.idx
%       pdflatex hypdestopt.dtx
%       makeindex -s gind.ist hypdestopt.idx
%       pdflatex hypdestopt.dtx
%
% Installation:
%    TDS:tex/latex/oberdiek/hypdestopt.sty
%    TDS:doc/latex/oberdiek/hypdestopt.pdf
%    TDS:source/latex/oberdiek/hypdestopt.dtx
%
%<*ignore>
\begingroup
  \catcode123=1 %
  \catcode125=2 %
  \def\x{LaTeX2e}%
\expandafter\endgroup
\ifcase 0\ifx\install y1\fi\expandafter
         \ifx\csname processbatchFile\endcsname\relax\else1\fi
         \ifx\fmtname\x\else 1\fi\relax
\else\csname fi\endcsname
%</ignore>
%<*install>
\input docstrip.tex
\Msg{************************************************************************}
\Msg{* Installation}
\Msg{* Package: hypdestopt 2011/05/13 v2.3 Hyperref destination optimizer (HO)}
\Msg{************************************************************************}

\keepsilent
\askforoverwritefalse

\let\MetaPrefix\relax
\preamble

This is a generated file.

Project: hypdestopt
Version: 2011/05/13 v2.3

Copyright (C) 2006-2008, 2011 by
   Heiko Oberdiek <heiko.oberdiek at googlemail.com>

This work may be distributed and/or modified under the
conditions of the LaTeX Project Public License, either
version 1.3c of this license or (at your option) any later
version. This version of this license is in
   http://www.latex-project.org/lppl/lppl-1-3c.txt
and the latest version of this license is in
   http://www.latex-project.org/lppl.txt
and version 1.3 or later is part of all distributions of
LaTeX version 2005/12/01 or later.

This work has the LPPL maintenance status "maintained".

This Current Maintainer of this work is Heiko Oberdiek.

This work consists of the main source file hypdestopt.dtx
and the derived files
   hypdestopt.sty, hypdestopt.pdf, hypdestopt.ins, hypdestopt.drv.

\endpreamble
\let\MetaPrefix\DoubleperCent

\generate{%
  \file{hypdestopt.ins}{\from{hypdestopt.dtx}{install}}%
  \file{hypdestopt.drv}{\from{hypdestopt.dtx}{driver}}%
  \usedir{tex/latex/oberdiek}%
  \file{hypdestopt.sty}{\from{hypdestopt.dtx}{package}}%
  \nopreamble
  \nopostamble
  \usedir{source/latex/oberdiek/catalogue}%
  \file{hypdestopt.xml}{\from{hypdestopt.dtx}{catalogue}}%
}

\catcode32=13\relax% active space
\let =\space%
\Msg{************************************************************************}
\Msg{*}
\Msg{* To finish the installation you have to move the following}
\Msg{* file into a directory searched by TeX:}
\Msg{*}
\Msg{*     hypdestopt.sty}
\Msg{*}
\Msg{* To produce the documentation run the file `hypdestopt.drv'}
\Msg{* through LaTeX.}
\Msg{*}
\Msg{* Happy TeXing!}
\Msg{*}
\Msg{************************************************************************}

\endbatchfile
%</install>
%<*ignore>
\fi
%</ignore>
%<*driver>
\NeedsTeXFormat{LaTeX2e}
\ProvidesFile{hypdestopt.drv}%
  [2011/05/13 v2.3 Hyperref destination optimizer (HO)]%
\documentclass{ltxdoc}
\usepackage{holtxdoc}[2011/11/22]
\begin{document}
  \DocInput{hypdestopt.dtx}%
\end{document}
%</driver>
% \fi
%
% \CheckSum{565}
%
% \CharacterTable
%  {Upper-case    \A\B\C\D\E\F\G\H\I\J\K\L\M\N\O\P\Q\R\S\T\U\V\W\X\Y\Z
%   Lower-case    \a\b\c\d\e\f\g\h\i\j\k\l\m\n\o\p\q\r\s\t\u\v\w\x\y\z
%   Digits        \0\1\2\3\4\5\6\7\8\9
%   Exclamation   \!     Double quote  \"     Hash (number) \#
%   Dollar        \$     Percent       \%     Ampersand     \&
%   Acute accent  \'     Left paren    \(     Right paren   \)
%   Asterisk      \*     Plus          \+     Comma         \,
%   Minus         \-     Point         \.     Solidus       \/
%   Colon         \:     Semicolon     \;     Less than     \<
%   Equals        \=     Greater than  \>     Question mark \?
%   Commercial at \@     Left bracket  \[     Backslash     \\
%   Right bracket \]     Circumflex    \^     Underscore    \_
%   Grave accent  \`     Left brace    \{     Vertical bar  \|
%   Right brace   \}     Tilde         \~}
%
% \GetFileInfo{hypdestopt.drv}
%
% \title{The \xpackage{hypdestopt} package}
% \date{2011/05/13 v2.3}
% \author{Heiko Oberdiek\\\xemail{heiko.oberdiek at googlemail.com}}
%
% \maketitle
%
% \begin{abstract}
% Package \xpackage{hypdestopt} supports \xpackage{hyperref}'s
% \xoption{pdftex} driver. It removes unnecessary destinations
% and shortens the destination names or uses numbered destinations
% to get smaller PDF files.
% \end{abstract}
%
% \tableofcontents
%
% \section{User interface}
%
% \subsection{Introduction}
%
% Before PDF-1.5 annotations and destinations cannot be compressed.
% If the destination names are not needed for external use, the
% file size can be decreased by the following means:
% \begin{itemize}
% \item Unused destinations are removed.
% \item The destination names are shortened (option \xoption{name}).
% \item Using numbered destinations (option \xoption{num}).
% \end{itemize}
%
% \subsection{Requirements}
%
% \begin{itemize}
% \item Package \xpackage{hyperref} 2006/06/01 v6.75a or newer
%       (\cite{hyperref}).
% \item Package \xpackage{alphalph} 2006/05/30 v1.4 or newer
%       (\cite{alphalph}), if option \xoption{name} is used.
% \item Package \xpackage{ifpdf} (\cite{ifpdf}).
% \item \pdfTeX\ 1.30.0 or newer.
% \item \pdfTeX\ in PDF mode.
% \item \eTeX\ extensions enabled.
% \item Probably an additional compile run of \pdfLaTeX\ is necessary.
% \end{itemize}
%
% In the first compile runs you can get warnings such as:
%\begin{quote}
%\begin{verbatim}
%! pdfTeX warning (dest): name{...} has been referenced ...
%\end{verbatim}
%\end{quote}
% These warnings should vanish in later compile runs.
% However these warnings also can occur without this package.
% The package does not cure them, thus these warnings will remain,
% but the destination name can be different. In such cases test
% without package, too.
%
% \subsection{Use}
%
% If the requirements are met, load the package:
%\begin{quote}
%\verb|\usepackage{hypdestopt}|
%\end{quote}
%
% The following options are supported:
% \begin{description}
% \item[\xoption{verbose}:] Verbose debug output is enabled and written
%   in the protocol file.
% \item[\xoption{num}:] Numbered destinations are used. The file size
%   is smaller, because names are no longer used.
%   This is the default.
% \item[\xoption{name}:] Destinations are identified by names.
% \end{description}
%
% \subsection{Limitations}
%
% \begin{itemize}
% \item Forget this package, if you need preserved destination names.
% \item Destination name strings use all bytes (0..255) except
%       the carriage return (13), left parenthesis (40), right
%       parenthesis (41), and backslash (92), because they
%       must be quoted in general and therefore occupy two bytes
%       instead of one.
%
%       Further the zero byte (0) is avoided for programs
%       that implement strings using zero terminated C strings.
%       And 255 (0xFF) is avoided to get rid of a possible
%       unicode marker at the begin.
%
%       So far I have not seen problems with:
%       \begin{itemize}
%       \item AcrobatReader 5.08/Linux
%       \item AcrobatReader 7.0/Linux
%       \item xpdf 3.00
%       \item Ghostscript 8.50
%       \item gv 3.5.8
%       \item GSview 4.6
%       \end{itemize}
%       But I have not tested all and all possible PDF viewers.
% \item Use of named destinations (\cs{pdfdest}, \cs{pdfoutline},
%       \cs{pdfstartlink}, \dots) that are not supported by this
%       package.
% \item Currently only \xpackage{hyperref} with \pdfTeX\ in PDF
%       mode is supported.
% \end{itemize}
%
% \subsection{Future}
%
% A more general approach is a PDF postprocessor that takes
% a PDF file, performs some transformations and writes the
% result in a more optimized PDF file. Then it does not depend,
% how the original PDF file was generated and further improvements
% are easier to apply. For example, the destination names could be sorted:
% often used destination names would then be shorter than seldom used ones.
%
% \StopEventually{
% }
%
% \section{Implementation}
%
% \subsection{Identification}
%
%    \begin{macrocode}
%<*package>
\NeedsTeXFormat{LaTeX2e}
\ProvidesPackage{hypdestopt}%
  [2011/05/13 v2.3 Hyperref destination optimizer (HO)]%
%    \end{macrocode}
%
% \subsection{Options}
%
% \subsubsection{Option \xoption{verbose}}
%
%    \begin{macrocode}
\newif\ifHypDest@Verbose
\DeclareOption{verbose}{\HypDest@Verbosetrue}
%    \end{macrocode}
%
%    \begin{macro}{\HypDest@VerboseInfo}
%    Wrapper for verbose messages.
%    \begin{macrocode}
\def\HypDest@VerboseInfo#1{%
  \ifHypDest@Verbose
    \PackageInfo{hypdestopt}{#1}%
  \fi
}
%    \end{macrocode}
%    \end{macro}
%
% \subsubsection{Options \xoption{num} and \xoption{name}}
%
%    The options \xoption{num} or \xoption{name} specify
%    the method, how destinations are referenced (by name or
%    number). Default is option \xoption{num}.
%    \begin{macrocode}
\newif\ifHypDest@name
\DeclareOption{num}{\HypDest@namefalse}
\DeclareOption{name}{\HypDest@nametrue}
%    \end{macrocode}
%
%    \begin{macrocode}
\ProcessOptions*\relax
%    \end{macrocode}
%
% \subsection{Check requirements}
%
%    First \pdfTeX\ must running in PDF mode.
%    \begin{macrocode}
\RequirePackage{ifpdf}[2007/09/09]
\RequirePackage{pdftexcmds}[2007/11/11]
\ifpdf
\else
  \PackageError{hypdestopt}{%
    This package requires pdfTeX in PDF mode%
  }\@ehc
  \expandafter\endinput
\fi
%    \end{macrocode}
%    The version of \pdfTeX\ must not be too old, because
%    \cs{pdfescapehex} and \cs{pdfunescapehex} are used.
%    \begin{macrocode}
\begingroup\expandafter\expandafter\expandafter\endgroup
\expandafter\ifx\csname pdf@escapehex\endcsname\relax
  \PackageError{hypdestopt}{%
    This pdfTeX is too old, at least 1.30.0 is required%
  }\@ehc
  \expandafter\endinput
\fi
%    \end{macrocode}
%    Features of \eTeX\ are used, e.g. \cs{numexpr}.
%    \begin{macrocode}
\begingroup\expandafter\expandafter\expandafter\endgroup
\expandafter\ifx\csname numexpr\endcsname\relax
  \PackageError{hypdestopt}{%
    e-TeX features are missing%
  }\@ehc
  \expandafter\endinput
\fi
%    \end{macrocode}
%    Package \xpackage{alphalph} provides \cs{newalphalph} since
%    version 2006/05/30 v1.4.
%    \begin{macrocode}
\ifHypDest@name
  \RequirePackage{alphalph}[2006/05/30]%
\fi
%    \end{macrocode}
%    \begin{macrocode}
\RequirePackage{auxhook}[2009/12/14]
\RequirePackage{pdfescape}[2007/04/21]
%    \end{macrocode}
%
% \subsection{Preamble for auxiliary file}
%
%    Provide dummy definitions for the macros that are used in the
%    auxiliary files. If the package is used no longer, then these
%    commands will not generate errors.
%
%    \begin{macro}{\HypDest@PrependDocument}
%    We add our stuff in front of the \cs{AtBeginDocument} hook
%    to ensure that we are before \xpackage{hyperref}'s stuff.
%    \begin{macrocode}
\long\def\HypDest@PrependDocument#1{%
  \begingroup
    \toks\z@{#1}%
    \toks\tw@\expandafter{\@begindocumenthook}%
    \xdef\@begindocumenthook{\the\toks\z@\the\toks\tw@}%
  \endgroup
}
%    \end{macrocode}
%    \end{macro}
%    \begin{macrocode}
\AddLineBeginAux{%
  \string\providecommand{\string\HypDest@Use}[1]{}%
}
%    \end{macrocode}
%
% \subsection{Generation of destination names}
%
%    Counter |HypDest| is used for identifying destinations.
%    \begin{macrocode}
\newcounter{HypDest}
%    \end{macrocode}
%
%    \begin{macrocode}
\ifHypDest@name
%    \end{macrocode}
%
%    \begin{macro}{\HypDest@HexChar}
%    Destination names are generated by automatically
%    numbering with the help of package \xpackage{alphalph}.
%    \cs{HypDest@HexChar} converts a number of the range 1 until 252
%    into the hexadecimal representation of the string character.
%    \begin{macrocode}
  \def\HypDest@HexChar#1{%
    \ifcase#1\or
%    \end{macrocode}
%    Avoid zero byte because of C strings in PDF viewer
%    applications.
%    \begin{macrocode}
      01\or 02\or 03\or 04\or 05\or 06\or 07\or
%    \end{macrocode}
%    Omit carriage return (13/\verb|^^0d|).
%    It needs quoting, otherwise it would be converted
%    to line feed (10/\verb|^^0a|).
%    \begin{macrocode}
      08\or 09\or 0A\or 0B\or 0C\or 0E\or 0F\or
      10\or 11\or 12\or 13\or 14\or 15\or 16\or 17\or
      18\or 19\or 1A\or 1B\or 1C\or 1D\or 1E\or 1F\or
      20\or 21\or 22\or 23\or 24\or 25\or 26\or 27\or
%    \end{macrocode}
%    Omit left and right parentheses (40/\verb|^^28|, 41/\verb|^^39|),
%    they need quoting in general.
%    \begin{macrocode}
      2A\or 2B\or 2C\or 2D\or 2E\or 2F\or
      30\or 31\or 32\or 33\or 34\or 35\or 36\or 37\or
      38\or 39\or 3A\or 3B\or 3C\or 3D\or 3E\or 3F\or
      40\or 41\or 42\or 43\or 44\or 45\or 46\or 47\or
      48\or 49\or 4A\or 4B\or 4C\or 4D\or 4E\or 4F\or
      50\or 51\or 52\or 53\or 54\or 55\or 56\or 57\or
%    \end{macrocode}
%    Omit backslash (92/\verb|^^5C|), it needs quoting.
%    \begin{macrocode}
      58\or 59\or 5A\or 5B\or 5D\or 5E\or 5F\or
      60\or 61\or 62\or 63\or 64\or 65\or 66\or 67\or
      68\or 69\or 6A\or 6B\or 6C\or 6D\or 6E\or 6F\or
      70\or 71\or 72\or 73\or 74\or 75\or 76\or 77\or
      78\or 79\or 7A\or 7B\or 7C\or 7D\or 7E\or 7F\or
      80\or 81\or 82\or 83\or 84\or 85\or 86\or 87\or
      88\or 89\or 8A\or 8B\or 8C\or 8D\or 8E\or 8F\or
      90\or 91\or 92\or 93\or 94\or 95\or 96\or 97\or
      98\or 99\or 9A\or 9B\or 9C\or 9D\or 9E\or 9F\or
      A0\or A1\or A2\or A3\or A4\or A5\or A6\or A7\or
      A8\or A9\or AA\or AB\or AC\or AD\or AE\or AF\or
      B0\or B1\or B2\or B3\or B4\or B5\or B6\or B7\or
      B8\or B9\or BA\or BB\or BC\or BD\or BE\or BF\or
      C0\or C1\or C2\or C3\or C4\or C5\or C6\or C7\or
      C8\or C9\or CA\or CB\or CC\or CD\or CE\or CF\or
      D0\or D1\or D2\or D3\or D4\or D5\or D6\or D7\or
      D8\or D9\or DA\or DB\or DC\or DD\or DE\or DF\or
      E0\or E1\or E2\or E3\or E4\or E5\or E6\or E7\or
      E8\or E9\or EA\or EB\or EC\or ED\or EE\or EF\or
      F0\or F1\or F2\or F3\or F4\or F5\or F6\or F7\or
%    \end{macrocode}
%    Avoid 255 (0xFF) to get rid of a possible unicode
%    marker at the begin of the string.
%    \begin{macrocode}
      F8\or F9\or FA\or FB\or FC\or FD\or FE%
    \fi
  }%
%    \end{macrocode}
%    \end{macro}
%    \begin{macro}{HypDest@HexString}
%    Now package \xpackage{alphalph} comes into play.
%    \cs{HypDest@HexString} is defined and converts
%    a positive number into a string, given in hexadecimal
%    representation.
%    \begin{macrocode}
  \newalphalph\HypDest@HexString\HypDest@HexChar{250}%
%    \end{macrocode}
%    \end{macro}
%    \begin{macro}{\theHypDest}
%    For use, the hexadecimal string is converted back.
%    \begin{macrocode}
  \renewcommand*{\theHypDest}{%
    \pdf@unescapehex{\HypDest@HexString{\value{HypDest}}}%
  }%
%    \end{macrocode}
%    \end{macro}
%
%    With option \xoption{num} we use the number directly.
%    \begin{macrocode}
\else
  \renewcommand*{\theHypDest}{%
    \number\value{HypDest}%
  }%
\fi
%    \end{macrocode}
%
% \subsection{Assign destination names}
%
%    \begin{macro}{\HypDest@Prefix}
%    The new destination names are remembered in macros whose names
%    start with prefix \cs{HypDest@Prefix}.
%    \begin{macrocode}
\edef\HypDest@Prefix{HypDest\string:}
%    \end{macrocode}
%    \end{macro}
%
%    \begin{macro}{\HypDest@Use}
%    During the first read of the auxiliary files, the used destinations
%    get fresh generated short destination names. Also for the old
%    destination names we use the hexadecimal representation. That
%    avoid problems with arbitrary names.
%    \begin{macrocode}
\def\HypDest@Use#1{%
  \begingroup
    \edef\x{%
      \expandafter\noexpand
      \csname\HypDest@Prefix\pdf@unescapehex{#1}\endcsname
    }%
    \expandafter\ifx\x\relax
      \stepcounter{HypDest}%
      \expandafter\xdef\x{\theHypDest}%
      \let\on@line\@empty
      \ifHypDest@name
        \HypDest@VerboseInfo{%
          Use: (\pdf@unescapehex{#1}) -\string> %
          0x\pdf@escapehex{\x} (\number\value{HypDest})%
        }%
      \else
        \HypDest@VerboseInfo{%
          Use: (\pdf@unescapehex{#1}) -\string> num \x
        }%
      \fi
    \fi
  \endgroup
}
%    \end{macrocode}
%    \end{macro}
%
%    After the first \xfile{.aux} file processing the destination names
%    are assigned and we can disable \cs{HypDest@Use}.
%    \begin{macrocode}
\AtBeginDocument{%
  \let\HypDest@Use\@gobble
}
%    \end{macrocode}
%
%    \begin{macro}{\HypDest@MarkUsed}
%    Destinations that are actually used are marked by \cs{HypDest@MarkUsed}.
%    \cs{nofiles} is respected.
%    \begin{macrocode}
\def\HypDest@MarkUsed#1{%
  \HypDest@VerboseInfo{%
    MarkUsed: (#1)%
  }%
  \if@filesw
    \immediate\write\@auxout{%
      \string\HypDest@Use{\pdf@escapehex{#1}}%
    }%
  \fi
}%
%    \end{macrocode}
%    \end{macro}
%
% \subsection{Redefinition of \xpackage{hyperref}'s hooks}
%
%    Package \xpackage{hyperref} can be loaded later, therefore
%    we redefine \xpackage{hyperref}'s macros at |\begin{document}|.
%    \begin{macrocode}
\HypDest@PrependDocument{%
%    \end{macrocode}
%
%    Check hyperref version.
%    \begin{macrocode}
  \@ifpackagelater{hyperref}{2006/06/01}{}{%
    \PackageError{hypdestopt}{%
      hyperref 2006/06/01 v6.75a or later is required%
    }\@ehc
  }%
%    \end{macrocode}
%
% \subsubsection{Destination setting}
%
%    \begin{macrocode}
  \ifHypDest@name
    \let\HypDest@Org@DestName\Hy@DestName
    \renewcommand*{\Hy@DestName}[2]{%
      \EdefUnescapeString\HypDest@temp{#1}%
      \@ifundefined{\HypDest@Prefix\HypDest@temp}{%
        \HypDest@VerboseInfo{%
          DestName: (\HypDest@temp) unused%
        }%
      }{%
        \HypDest@Org@DestName{%
          \csname\HypDest@Prefix\HypDest@temp\endcsname
        }{#2}%
        \HypDest@VerboseInfo{%
          DestName: (\HypDest@temp) %
          0x\pdf@escapehex{%
            \csname\HypDest@Prefix\HypDest@temp\endcsname
          }%
        }%
      }%
    }%
  \else
    \renewcommand*{\Hy@DestName}[2]{%
      \EdefUnescapeString\HypDest@temp{#1}%
      \@ifundefined{\HypDest@Prefix\HypDest@temp}{%
        \HypDest@VerboseInfo{%
          DestName: (\HypDest@temp) unused%
        }%
      }{%
        \pdfdest num%
        \csname\HypDest@Prefix\HypDest@temp\endcsname#2\relax
        \HypDest@VerboseInfo{%
          DestName: (\HypDest@temp) %
          num \csname\HypDest@Prefix\HypDest@temp\endcsname
        }%
      }%
    }%
  \fi
%    \end{macrocode}
%
% \subsubsection{Links}
%
%    \begin{macrocode}
  \let\HypDest@Org@StartlinkName\Hy@StartlinkName
  \ifHypDest@name
    \renewcommand*{\Hy@StartlinkName}[2]{%
      \HypDest@MarkUsed{#2}%
      \HypDest@Org@StartlinkName{#1}{%
        \@ifundefined{\HypDest@Prefix#2}{%
          #2%
        }{%
          \csname\HypDest@Prefix#2\endcsname
        }%
      }%
    }%
  \else
    \renewcommand*{\Hy@StartlinkName}[2]{%
      \HypDest@MarkUsed{#2}%
      \@ifundefined{\HypDest@Prefix#2}{%
        \HypDest@Org@StartlinkName{#1}{#2}%
      }{%
        \pdfstartlink attr{#1}%
                      goto num\csname\HypDest@Prefix#2\endcsname
        \relax
      }%
    }%
  \fi
%    \end{macrocode}
%
% \subsubsection{Outlines of package \xpackage{hyperref}}
%
%    \begin{macrocode}
  \let\HypDest@Org@OutlineName\Hy@OutlineName
  \ifHypDest@name
    \renewcommand*{\Hy@OutlineName}[4]{%
      \HypDest@Org@OutlineName{#1}{%
        \@ifundefined{\HypDest@Prefix#2}{%
          #2%
        }{%
          \csname\HypDest@Prefix#2\endcsname
        }%
      }{#3}{#4}%
    }%
  \else
    \renewcommand*{\Hy@OutlineName}[4]{%
      \@ifundefined{\HypDest@Prefix#2}{%
        \HypDest@Org@OutlineName{#1}{#2}{#3}{#4}%
      }{%
        \pdfoutline goto num\csname\HypDest@Prefix#2\endcsname
                    count#3{#4}%
      }%
    }%
  \fi
%    \end{macrocode}
%    Because \cs{Hy@OutlineName} is called after the \xfile{.out} file
%    is written in the previous run. Therefore we mark the destination
%    earlier in \cs{@@writetorep}.
%    \begin{macrocode}
  \let\HypDest@Org@@writetorep\@@writetorep
  \renewcommand*{\@@writetorep}[5]{%
    \begingroup
      \edef\Hy@tempa{#5}%
      \ifx\Hy@tempa\Hy@bookmarkstype
        \HypDest@MarkUsed{#3}%
      \fi
    \endgroup
    \HypDest@Org@@writetorep{#1}{#2}{#3}{#4}{#5}%
  }%
%    \end{macrocode}
%
% \subsubsection{Outlines of package \xpackage{bookmark}}
%
%    \begin{macrocode}
  \@ifpackageloaded{bookmark}{%
    \@ifpackagelater{bookmark}{2008/08/08}{%
      \renewcommand*{\BKM@DefGotoNameAction}[2]{%
        \@ifundefined{\HypDest@Prefix#2}{%
          \edef#1{goto name{hypdestopt\string :unknown}}%
        }{%
          \ifHypDest@name
            \edef#1{goto name{\csname\HypDest@Prefix#2\endcsname}}%
          \else
            \edef#1{goto num\csname\HypDest@Prefix#2\endcsname}%
          \fi
        }%
      }%
      \def\BKM@HypDestOptHook{%
        \ifx\BKM@dest\@empty
        \else
          \ifx\BKM@gotor\@empty
            \HypDest@MarkUsed\BKM@dest
          \fi
        \fi
      }%
    }{%
      \@PackageError{hypdestopt}{%
        Package `bookmark' is too old.\MessageBreak
        Version 2008/08/08 or later is needed%
      }\@ehc
    }%
  }{}%
%    \end{macrocode}
%
%    \begin{macrocode}
}
%    \end{macrocode}
%
%
%    \begin{macrocode}
%</package>
%    \end{macrocode}
%
% \section{Installation}
%
% \subsection{Download}
%
% \paragraph{Package.} This package is available on
% CTAN\footnote{\url{ftp://ftp.ctan.org/tex-archive/}}:
% \begin{description}
% \item[\CTAN{macros/latex/contrib/oberdiek/hypdestopt.dtx}] The source file.
% \item[\CTAN{macros/latex/contrib/oberdiek/hypdestopt.pdf}] Documentation.
% \end{description}
%
%
% \paragraph{Bundle.} All the packages of the bundle `oberdiek'
% are also available in a TDS compliant ZIP archive. There
% the packages are already unpacked and the documentation files
% are generated. The files and directories obey the TDS standard.
% \begin{description}
% \item[\CTAN{install/macros/latex/contrib/oberdiek.tds.zip}]
% \end{description}
% \emph{TDS} refers to the standard ``A Directory Structure
% for \TeX\ Files'' (\CTAN{tds/tds.pdf}). Directories
% with \xfile{texmf} in their name are usually organized this way.
%
% \subsection{Bundle installation}
%
% \paragraph{Unpacking.} Unpack the \xfile{oberdiek.tds.zip} in the
% TDS tree (also known as \xfile{texmf} tree) of your choice.
% Example (linux):
% \begin{quote}
%   |unzip oberdiek.tds.zip -d ~/texmf|
% \end{quote}
%
% \paragraph{Script installation.}
% Check the directory \xfile{TDS:scripts/oberdiek/} for
% scripts that need further installation steps.
% Package \xpackage{attachfile2} comes with the Perl script
% \xfile{pdfatfi.pl} that should be installed in such a way
% that it can be called as \texttt{pdfatfi}.
% Example (linux):
% \begin{quote}
%   |chmod +x scripts/oberdiek/pdfatfi.pl|\\
%   |cp scripts/oberdiek/pdfatfi.pl /usr/local/bin/|
% \end{quote}
%
% \subsection{Package installation}
%
% \paragraph{Unpacking.} The \xfile{.dtx} file is a self-extracting
% \docstrip\ archive. The files are extracted by running the
% \xfile{.dtx} through \plainTeX:
% \begin{quote}
%   \verb|tex hypdestopt.dtx|
% \end{quote}
%
% \paragraph{TDS.} Now the different files must be moved into
% the different directories in your installation TDS tree
% (also known as \xfile{texmf} tree):
% \begin{quote}
% \def\t{^^A
% \begin{tabular}{@{}>{\ttfamily}l@{ $\rightarrow$ }>{\ttfamily}l@{}}
%   hypdestopt.sty & tex/latex/oberdiek/hypdestopt.sty\\
%   hypdestopt.pdf & doc/latex/oberdiek/hypdestopt.pdf\\
%   hypdestopt.dtx & source/latex/oberdiek/hypdestopt.dtx\\
% \end{tabular}^^A
% }^^A
% \sbox0{\t}^^A
% \ifdim\wd0>\linewidth
%   \begingroup
%     \advance\linewidth by\leftmargin
%     \advance\linewidth by\rightmargin
%   \edef\x{\endgroup
%     \def\noexpand\lw{\the\linewidth}^^A
%   }\x
%   \def\lwbox{^^A
%     \leavevmode
%     \hbox to \linewidth{^^A
%       \kern-\leftmargin\relax
%       \hss
%       \usebox0
%       \hss
%       \kern-\rightmargin\relax
%     }^^A
%   }^^A
%   \ifdim\wd0>\lw
%     \sbox0{\small\t}^^A
%     \ifdim\wd0>\linewidth
%       \ifdim\wd0>\lw
%         \sbox0{\footnotesize\t}^^A
%         \ifdim\wd0>\linewidth
%           \ifdim\wd0>\lw
%             \sbox0{\scriptsize\t}^^A
%             \ifdim\wd0>\linewidth
%               \ifdim\wd0>\lw
%                 \sbox0{\tiny\t}^^A
%                 \ifdim\wd0>\linewidth
%                   \lwbox
%                 \else
%                   \usebox0
%                 \fi
%               \else
%                 \lwbox
%               \fi
%             \else
%               \usebox0
%             \fi
%           \else
%             \lwbox
%           \fi
%         \else
%           \usebox0
%         \fi
%       \else
%         \lwbox
%       \fi
%     \else
%       \usebox0
%     \fi
%   \else
%     \lwbox
%   \fi
% \else
%   \usebox0
% \fi
% \end{quote}
% If you have a \xfile{docstrip.cfg} that configures and enables \docstrip's
% TDS installing feature, then some files can already be in the right
% place, see the documentation of \docstrip.
%
% \subsection{Refresh file name databases}
%
% If your \TeX~distribution
% (\teTeX, \mikTeX, \dots) relies on file name databases, you must refresh
% these. For example, \teTeX\ users run \verb|texhash| or
% \verb|mktexlsr|.
%
% \subsection{Some details for the interested}
%
% \paragraph{Attached source.}
%
% The PDF documentation on CTAN also includes the
% \xfile{.dtx} source file. It can be extracted by
% AcrobatReader 6 or higher. Another option is \textsf{pdftk},
% e.g. unpack the file into the current directory:
% \begin{quote}
%   \verb|pdftk hypdestopt.pdf unpack_files output .|
% \end{quote}
%
% \paragraph{Unpacking with \LaTeX.}
% The \xfile{.dtx} chooses its action depending on the format:
% \begin{description}
% \item[\plainTeX:] Run \docstrip\ and extract the files.
% \item[\LaTeX:] Generate the documentation.
% \end{description}
% If you insist on using \LaTeX\ for \docstrip\ (really,
% \docstrip\ does not need \LaTeX), then inform the autodetect routine
% about your intention:
% \begin{quote}
%   \verb|latex \let\install=y% \iffalse meta-comment
%
% File: hypdestopt.dtx
% Version: 2011/05/13 v2.3
% Info: Hyperref destination optimizer
%
% Copyright (C) 2006-2008, 2011 by
%    Heiko Oberdiek <heiko.oberdiek at googlemail.com>
%
% This work may be distributed and/or modified under the
% conditions of the LaTeX Project Public License, either
% version 1.3c of this license or (at your option) any later
% version. This version of this license is in
%    http://www.latex-project.org/lppl/lppl-1-3c.txt
% and the latest version of this license is in
%    http://www.latex-project.org/lppl.txt
% and version 1.3 or later is part of all distributions of
% LaTeX version 2005/12/01 or later.
%
% This work has the LPPL maintenance status "maintained".
%
% This Current Maintainer of this work is Heiko Oberdiek.
%
% This work consists of the main source file hypdestopt.dtx
% and the derived files
%    hypdestopt.sty, hypdestopt.pdf, hypdestopt.ins, hypdestopt.drv.
%
% Distribution:
%    CTAN:macros/latex/contrib/oberdiek/hypdestopt.dtx
%    CTAN:macros/latex/contrib/oberdiek/hypdestopt.pdf
%
% Unpacking:
%    (a) If hypdestopt.ins is present:
%           tex hypdestopt.ins
%    (b) Without hypdestopt.ins:
%           tex hypdestopt.dtx
%    (c) If you insist on using LaTeX
%           latex \let\install=y\input{hypdestopt.dtx}
%        (quote the arguments according to the demands of your shell)
%
% Documentation:
%    (a) If hypdestopt.drv is present:
%           latex hypdestopt.drv
%    (b) Without hypdestopt.drv:
%           latex hypdestopt.dtx; ...
%    The class ltxdoc loads the configuration file ltxdoc.cfg
%    if available. Here you can specify further options, e.g.
%    use A4 as paper format:
%       \PassOptionsToClass{a4paper}{article}
%
%    Programm calls to get the documentation (example):
%       pdflatex hypdestopt.dtx
%       makeindex -s gind.ist hypdestopt.idx
%       pdflatex hypdestopt.dtx
%       makeindex -s gind.ist hypdestopt.idx
%       pdflatex hypdestopt.dtx
%
% Installation:
%    TDS:tex/latex/oberdiek/hypdestopt.sty
%    TDS:doc/latex/oberdiek/hypdestopt.pdf
%    TDS:source/latex/oberdiek/hypdestopt.dtx
%
%<*ignore>
\begingroup
  \catcode123=1 %
  \catcode125=2 %
  \def\x{LaTeX2e}%
\expandafter\endgroup
\ifcase 0\ifx\install y1\fi\expandafter
         \ifx\csname processbatchFile\endcsname\relax\else1\fi
         \ifx\fmtname\x\else 1\fi\relax
\else\csname fi\endcsname
%</ignore>
%<*install>
\input docstrip.tex
\Msg{************************************************************************}
\Msg{* Installation}
\Msg{* Package: hypdestopt 2011/05/13 v2.3 Hyperref destination optimizer (HO)}
\Msg{************************************************************************}

\keepsilent
\askforoverwritefalse

\let\MetaPrefix\relax
\preamble

This is a generated file.

Project: hypdestopt
Version: 2011/05/13 v2.3

Copyright (C) 2006-2008, 2011 by
   Heiko Oberdiek <heiko.oberdiek at googlemail.com>

This work may be distributed and/or modified under the
conditions of the LaTeX Project Public License, either
version 1.3c of this license or (at your option) any later
version. This version of this license is in
   http://www.latex-project.org/lppl/lppl-1-3c.txt
and the latest version of this license is in
   http://www.latex-project.org/lppl.txt
and version 1.3 or later is part of all distributions of
LaTeX version 2005/12/01 or later.

This work has the LPPL maintenance status "maintained".

This Current Maintainer of this work is Heiko Oberdiek.

This work consists of the main source file hypdestopt.dtx
and the derived files
   hypdestopt.sty, hypdestopt.pdf, hypdestopt.ins, hypdestopt.drv.

\endpreamble
\let\MetaPrefix\DoubleperCent

\generate{%
  \file{hypdestopt.ins}{\from{hypdestopt.dtx}{install}}%
  \file{hypdestopt.drv}{\from{hypdestopt.dtx}{driver}}%
  \usedir{tex/latex/oberdiek}%
  \file{hypdestopt.sty}{\from{hypdestopt.dtx}{package}}%
  \nopreamble
  \nopostamble
  \usedir{source/latex/oberdiek/catalogue}%
  \file{hypdestopt.xml}{\from{hypdestopt.dtx}{catalogue}}%
}

\catcode32=13\relax% active space
\let =\space%
\Msg{************************************************************************}
\Msg{*}
\Msg{* To finish the installation you have to move the following}
\Msg{* file into a directory searched by TeX:}
\Msg{*}
\Msg{*     hypdestopt.sty}
\Msg{*}
\Msg{* To produce the documentation run the file `hypdestopt.drv'}
\Msg{* through LaTeX.}
\Msg{*}
\Msg{* Happy TeXing!}
\Msg{*}
\Msg{************************************************************************}

\endbatchfile
%</install>
%<*ignore>
\fi
%</ignore>
%<*driver>
\NeedsTeXFormat{LaTeX2e}
\ProvidesFile{hypdestopt.drv}%
  [2011/05/13 v2.3 Hyperref destination optimizer (HO)]%
\documentclass{ltxdoc}
\usepackage{holtxdoc}[2011/11/22]
\begin{document}
  \DocInput{hypdestopt.dtx}%
\end{document}
%</driver>
% \fi
%
% \CheckSum{565}
%
% \CharacterTable
%  {Upper-case    \A\B\C\D\E\F\G\H\I\J\K\L\M\N\O\P\Q\R\S\T\U\V\W\X\Y\Z
%   Lower-case    \a\b\c\d\e\f\g\h\i\j\k\l\m\n\o\p\q\r\s\t\u\v\w\x\y\z
%   Digits        \0\1\2\3\4\5\6\7\8\9
%   Exclamation   \!     Double quote  \"     Hash (number) \#
%   Dollar        \$     Percent       \%     Ampersand     \&
%   Acute accent  \'     Left paren    \(     Right paren   \)
%   Asterisk      \*     Plus          \+     Comma         \,
%   Minus         \-     Point         \.     Solidus       \/
%   Colon         \:     Semicolon     \;     Less than     \<
%   Equals        \=     Greater than  \>     Question mark \?
%   Commercial at \@     Left bracket  \[     Backslash     \\
%   Right bracket \]     Circumflex    \^     Underscore    \_
%   Grave accent  \`     Left brace    \{     Vertical bar  \|
%   Right brace   \}     Tilde         \~}
%
% \GetFileInfo{hypdestopt.drv}
%
% \title{The \xpackage{hypdestopt} package}
% \date{2011/05/13 v2.3}
% \author{Heiko Oberdiek\\\xemail{heiko.oberdiek at googlemail.com}}
%
% \maketitle
%
% \begin{abstract}
% Package \xpackage{hypdestopt} supports \xpackage{hyperref}'s
% \xoption{pdftex} driver. It removes unnecessary destinations
% and shortens the destination names or uses numbered destinations
% to get smaller PDF files.
% \end{abstract}
%
% \tableofcontents
%
% \section{User interface}
%
% \subsection{Introduction}
%
% Before PDF-1.5 annotations and destinations cannot be compressed.
% If the destination names are not needed for external use, the
% file size can be decreased by the following means:
% \begin{itemize}
% \item Unused destinations are removed.
% \item The destination names are shortened (option \xoption{name}).
% \item Using numbered destinations (option \xoption{num}).
% \end{itemize}
%
% \subsection{Requirements}
%
% \begin{itemize}
% \item Package \xpackage{hyperref} 2006/06/01 v6.75a or newer
%       (\cite{hyperref}).
% \item Package \xpackage{alphalph} 2006/05/30 v1.4 or newer
%       (\cite{alphalph}), if option \xoption{name} is used.
% \item Package \xpackage{ifpdf} (\cite{ifpdf}).
% \item \pdfTeX\ 1.30.0 or newer.
% \item \pdfTeX\ in PDF mode.
% \item \eTeX\ extensions enabled.
% \item Probably an additional compile run of \pdfLaTeX\ is necessary.
% \end{itemize}
%
% In the first compile runs you can get warnings such as:
%\begin{quote}
%\begin{verbatim}
%! pdfTeX warning (dest): name{...} has been referenced ...
%\end{verbatim}
%\end{quote}
% These warnings should vanish in later compile runs.
% However these warnings also can occur without this package.
% The package does not cure them, thus these warnings will remain,
% but the destination name can be different. In such cases test
% without package, too.
%
% \subsection{Use}
%
% If the requirements are met, load the package:
%\begin{quote}
%\verb|\usepackage{hypdestopt}|
%\end{quote}
%
% The following options are supported:
% \begin{description}
% \item[\xoption{verbose}:] Verbose debug output is enabled and written
%   in the protocol file.
% \item[\xoption{num}:] Numbered destinations are used. The file size
%   is smaller, because names are no longer used.
%   This is the default.
% \item[\xoption{name}:] Destinations are identified by names.
% \end{description}
%
% \subsection{Limitations}
%
% \begin{itemize}
% \item Forget this package, if you need preserved destination names.
% \item Destination name strings use all bytes (0..255) except
%       the carriage return (13), left parenthesis (40), right
%       parenthesis (41), and backslash (92), because they
%       must be quoted in general and therefore occupy two bytes
%       instead of one.
%
%       Further the zero byte (0) is avoided for programs
%       that implement strings using zero terminated C strings.
%       And 255 (0xFF) is avoided to get rid of a possible
%       unicode marker at the begin.
%
%       So far I have not seen problems with:
%       \begin{itemize}
%       \item AcrobatReader 5.08/Linux
%       \item AcrobatReader 7.0/Linux
%       \item xpdf 3.00
%       \item Ghostscript 8.50
%       \item gv 3.5.8
%       \item GSview 4.6
%       \end{itemize}
%       But I have not tested all and all possible PDF viewers.
% \item Use of named destinations (\cs{pdfdest}, \cs{pdfoutline},
%       \cs{pdfstartlink}, \dots) that are not supported by this
%       package.
% \item Currently only \xpackage{hyperref} with \pdfTeX\ in PDF
%       mode is supported.
% \end{itemize}
%
% \subsection{Future}
%
% A more general approach is a PDF postprocessor that takes
% a PDF file, performs some transformations and writes the
% result in a more optimized PDF file. Then it does not depend,
% how the original PDF file was generated and further improvements
% are easier to apply. For example, the destination names could be sorted:
% often used destination names would then be shorter than seldom used ones.
%
% \StopEventually{
% }
%
% \section{Implementation}
%
% \subsection{Identification}
%
%    \begin{macrocode}
%<*package>
\NeedsTeXFormat{LaTeX2e}
\ProvidesPackage{hypdestopt}%
  [2011/05/13 v2.3 Hyperref destination optimizer (HO)]%
%    \end{macrocode}
%
% \subsection{Options}
%
% \subsubsection{Option \xoption{verbose}}
%
%    \begin{macrocode}
\newif\ifHypDest@Verbose
\DeclareOption{verbose}{\HypDest@Verbosetrue}
%    \end{macrocode}
%
%    \begin{macro}{\HypDest@VerboseInfo}
%    Wrapper for verbose messages.
%    \begin{macrocode}
\def\HypDest@VerboseInfo#1{%
  \ifHypDest@Verbose
    \PackageInfo{hypdestopt}{#1}%
  \fi
}
%    \end{macrocode}
%    \end{macro}
%
% \subsubsection{Options \xoption{num} and \xoption{name}}
%
%    The options \xoption{num} or \xoption{name} specify
%    the method, how destinations are referenced (by name or
%    number). Default is option \xoption{num}.
%    \begin{macrocode}
\newif\ifHypDest@name
\DeclareOption{num}{\HypDest@namefalse}
\DeclareOption{name}{\HypDest@nametrue}
%    \end{macrocode}
%
%    \begin{macrocode}
\ProcessOptions*\relax
%    \end{macrocode}
%
% \subsection{Check requirements}
%
%    First \pdfTeX\ must running in PDF mode.
%    \begin{macrocode}
\RequirePackage{ifpdf}[2007/09/09]
\RequirePackage{pdftexcmds}[2007/11/11]
\ifpdf
\else
  \PackageError{hypdestopt}{%
    This package requires pdfTeX in PDF mode%
  }\@ehc
  \expandafter\endinput
\fi
%    \end{macrocode}
%    The version of \pdfTeX\ must not be too old, because
%    \cs{pdfescapehex} and \cs{pdfunescapehex} are used.
%    \begin{macrocode}
\begingroup\expandafter\expandafter\expandafter\endgroup
\expandafter\ifx\csname pdf@escapehex\endcsname\relax
  \PackageError{hypdestopt}{%
    This pdfTeX is too old, at least 1.30.0 is required%
  }\@ehc
  \expandafter\endinput
\fi
%    \end{macrocode}
%    Features of \eTeX\ are used, e.g. \cs{numexpr}.
%    \begin{macrocode}
\begingroup\expandafter\expandafter\expandafter\endgroup
\expandafter\ifx\csname numexpr\endcsname\relax
  \PackageError{hypdestopt}{%
    e-TeX features are missing%
  }\@ehc
  \expandafter\endinput
\fi
%    \end{macrocode}
%    Package \xpackage{alphalph} provides \cs{newalphalph} since
%    version 2006/05/30 v1.4.
%    \begin{macrocode}
\ifHypDest@name
  \RequirePackage{alphalph}[2006/05/30]%
\fi
%    \end{macrocode}
%    \begin{macrocode}
\RequirePackage{auxhook}[2009/12/14]
\RequirePackage{pdfescape}[2007/04/21]
%    \end{macrocode}
%
% \subsection{Preamble for auxiliary file}
%
%    Provide dummy definitions for the macros that are used in the
%    auxiliary files. If the package is used no longer, then these
%    commands will not generate errors.
%
%    \begin{macro}{\HypDest@PrependDocument}
%    We add our stuff in front of the \cs{AtBeginDocument} hook
%    to ensure that we are before \xpackage{hyperref}'s stuff.
%    \begin{macrocode}
\long\def\HypDest@PrependDocument#1{%
  \begingroup
    \toks\z@{#1}%
    \toks\tw@\expandafter{\@begindocumenthook}%
    \xdef\@begindocumenthook{\the\toks\z@\the\toks\tw@}%
  \endgroup
}
%    \end{macrocode}
%    \end{macro}
%    \begin{macrocode}
\AddLineBeginAux{%
  \string\providecommand{\string\HypDest@Use}[1]{}%
}
%    \end{macrocode}
%
% \subsection{Generation of destination names}
%
%    Counter |HypDest| is used for identifying destinations.
%    \begin{macrocode}
\newcounter{HypDest}
%    \end{macrocode}
%
%    \begin{macrocode}
\ifHypDest@name
%    \end{macrocode}
%
%    \begin{macro}{\HypDest@HexChar}
%    Destination names are generated by automatically
%    numbering with the help of package \xpackage{alphalph}.
%    \cs{HypDest@HexChar} converts a number of the range 1 until 252
%    into the hexadecimal representation of the string character.
%    \begin{macrocode}
  \def\HypDest@HexChar#1{%
    \ifcase#1\or
%    \end{macrocode}
%    Avoid zero byte because of C strings in PDF viewer
%    applications.
%    \begin{macrocode}
      01\or 02\or 03\or 04\or 05\or 06\or 07\or
%    \end{macrocode}
%    Omit carriage return (13/\verb|^^0d|).
%    It needs quoting, otherwise it would be converted
%    to line feed (10/\verb|^^0a|).
%    \begin{macrocode}
      08\or 09\or 0A\or 0B\or 0C\or 0E\or 0F\or
      10\or 11\or 12\or 13\or 14\or 15\or 16\or 17\or
      18\or 19\or 1A\or 1B\or 1C\or 1D\or 1E\or 1F\or
      20\or 21\or 22\or 23\or 24\or 25\or 26\or 27\or
%    \end{macrocode}
%    Omit left and right parentheses (40/\verb|^^28|, 41/\verb|^^39|),
%    they need quoting in general.
%    \begin{macrocode}
      2A\or 2B\or 2C\or 2D\or 2E\or 2F\or
      30\or 31\or 32\or 33\or 34\or 35\or 36\or 37\or
      38\or 39\or 3A\or 3B\or 3C\or 3D\or 3E\or 3F\or
      40\or 41\or 42\or 43\or 44\or 45\or 46\or 47\or
      48\or 49\or 4A\or 4B\or 4C\or 4D\or 4E\or 4F\or
      50\or 51\or 52\or 53\or 54\or 55\or 56\or 57\or
%    \end{macrocode}
%    Omit backslash (92/\verb|^^5C|), it needs quoting.
%    \begin{macrocode}
      58\or 59\or 5A\or 5B\or 5D\or 5E\or 5F\or
      60\or 61\or 62\or 63\or 64\or 65\or 66\or 67\or
      68\or 69\or 6A\or 6B\or 6C\or 6D\or 6E\or 6F\or
      70\or 71\or 72\or 73\or 74\or 75\or 76\or 77\or
      78\or 79\or 7A\or 7B\or 7C\or 7D\or 7E\or 7F\or
      80\or 81\or 82\or 83\or 84\or 85\or 86\or 87\or
      88\or 89\or 8A\or 8B\or 8C\or 8D\or 8E\or 8F\or
      90\or 91\or 92\or 93\or 94\or 95\or 96\or 97\or
      98\or 99\or 9A\or 9B\or 9C\or 9D\or 9E\or 9F\or
      A0\or A1\or A2\or A3\or A4\or A5\or A6\or A7\or
      A8\or A9\or AA\or AB\or AC\or AD\or AE\or AF\or
      B0\or B1\or B2\or B3\or B4\or B5\or B6\or B7\or
      B8\or B9\or BA\or BB\or BC\or BD\or BE\or BF\or
      C0\or C1\or C2\or C3\or C4\or C5\or C6\or C7\or
      C8\or C9\or CA\or CB\or CC\or CD\or CE\or CF\or
      D0\or D1\or D2\or D3\or D4\or D5\or D6\or D7\or
      D8\or D9\or DA\or DB\or DC\or DD\or DE\or DF\or
      E0\or E1\or E2\or E3\or E4\or E5\or E6\or E7\or
      E8\or E9\or EA\or EB\or EC\or ED\or EE\or EF\or
      F0\or F1\or F2\or F3\or F4\or F5\or F6\or F7\or
%    \end{macrocode}
%    Avoid 255 (0xFF) to get rid of a possible unicode
%    marker at the begin of the string.
%    \begin{macrocode}
      F8\or F9\or FA\or FB\or FC\or FD\or FE%
    \fi
  }%
%    \end{macrocode}
%    \end{macro}
%    \begin{macro}{HypDest@HexString}
%    Now package \xpackage{alphalph} comes into play.
%    \cs{HypDest@HexString} is defined and converts
%    a positive number into a string, given in hexadecimal
%    representation.
%    \begin{macrocode}
  \newalphalph\HypDest@HexString\HypDest@HexChar{250}%
%    \end{macrocode}
%    \end{macro}
%    \begin{macro}{\theHypDest}
%    For use, the hexadecimal string is converted back.
%    \begin{macrocode}
  \renewcommand*{\theHypDest}{%
    \pdf@unescapehex{\HypDest@HexString{\value{HypDest}}}%
  }%
%    \end{macrocode}
%    \end{macro}
%
%    With option \xoption{num} we use the number directly.
%    \begin{macrocode}
\else
  \renewcommand*{\theHypDest}{%
    \number\value{HypDest}%
  }%
\fi
%    \end{macrocode}
%
% \subsection{Assign destination names}
%
%    \begin{macro}{\HypDest@Prefix}
%    The new destination names are remembered in macros whose names
%    start with prefix \cs{HypDest@Prefix}.
%    \begin{macrocode}
\edef\HypDest@Prefix{HypDest\string:}
%    \end{macrocode}
%    \end{macro}
%
%    \begin{macro}{\HypDest@Use}
%    During the first read of the auxiliary files, the used destinations
%    get fresh generated short destination names. Also for the old
%    destination names we use the hexadecimal representation. That
%    avoid problems with arbitrary names.
%    \begin{macrocode}
\def\HypDest@Use#1{%
  \begingroup
    \edef\x{%
      \expandafter\noexpand
      \csname\HypDest@Prefix\pdf@unescapehex{#1}\endcsname
    }%
    \expandafter\ifx\x\relax
      \stepcounter{HypDest}%
      \expandafter\xdef\x{\theHypDest}%
      \let\on@line\@empty
      \ifHypDest@name
        \HypDest@VerboseInfo{%
          Use: (\pdf@unescapehex{#1}) -\string> %
          0x\pdf@escapehex{\x} (\number\value{HypDest})%
        }%
      \else
        \HypDest@VerboseInfo{%
          Use: (\pdf@unescapehex{#1}) -\string> num \x
        }%
      \fi
    \fi
  \endgroup
}
%    \end{macrocode}
%    \end{macro}
%
%    After the first \xfile{.aux} file processing the destination names
%    are assigned and we can disable \cs{HypDest@Use}.
%    \begin{macrocode}
\AtBeginDocument{%
  \let\HypDest@Use\@gobble
}
%    \end{macrocode}
%
%    \begin{macro}{\HypDest@MarkUsed}
%    Destinations that are actually used are marked by \cs{HypDest@MarkUsed}.
%    \cs{nofiles} is respected.
%    \begin{macrocode}
\def\HypDest@MarkUsed#1{%
  \HypDest@VerboseInfo{%
    MarkUsed: (#1)%
  }%
  \if@filesw
    \immediate\write\@auxout{%
      \string\HypDest@Use{\pdf@escapehex{#1}}%
    }%
  \fi
}%
%    \end{macrocode}
%    \end{macro}
%
% \subsection{Redefinition of \xpackage{hyperref}'s hooks}
%
%    Package \xpackage{hyperref} can be loaded later, therefore
%    we redefine \xpackage{hyperref}'s macros at |\begin{document}|.
%    \begin{macrocode}
\HypDest@PrependDocument{%
%    \end{macrocode}
%
%    Check hyperref version.
%    \begin{macrocode}
  \@ifpackagelater{hyperref}{2006/06/01}{}{%
    \PackageError{hypdestopt}{%
      hyperref 2006/06/01 v6.75a or later is required%
    }\@ehc
  }%
%    \end{macrocode}
%
% \subsubsection{Destination setting}
%
%    \begin{macrocode}
  \ifHypDest@name
    \let\HypDest@Org@DestName\Hy@DestName
    \renewcommand*{\Hy@DestName}[2]{%
      \EdefUnescapeString\HypDest@temp{#1}%
      \@ifundefined{\HypDest@Prefix\HypDest@temp}{%
        \HypDest@VerboseInfo{%
          DestName: (\HypDest@temp) unused%
        }%
      }{%
        \HypDest@Org@DestName{%
          \csname\HypDest@Prefix\HypDest@temp\endcsname
        }{#2}%
        \HypDest@VerboseInfo{%
          DestName: (\HypDest@temp) %
          0x\pdf@escapehex{%
            \csname\HypDest@Prefix\HypDest@temp\endcsname
          }%
        }%
      }%
    }%
  \else
    \renewcommand*{\Hy@DestName}[2]{%
      \EdefUnescapeString\HypDest@temp{#1}%
      \@ifundefined{\HypDest@Prefix\HypDest@temp}{%
        \HypDest@VerboseInfo{%
          DestName: (\HypDest@temp) unused%
        }%
      }{%
        \pdfdest num%
        \csname\HypDest@Prefix\HypDest@temp\endcsname#2\relax
        \HypDest@VerboseInfo{%
          DestName: (\HypDest@temp) %
          num \csname\HypDest@Prefix\HypDest@temp\endcsname
        }%
      }%
    }%
  \fi
%    \end{macrocode}
%
% \subsubsection{Links}
%
%    \begin{macrocode}
  \let\HypDest@Org@StartlinkName\Hy@StartlinkName
  \ifHypDest@name
    \renewcommand*{\Hy@StartlinkName}[2]{%
      \HypDest@MarkUsed{#2}%
      \HypDest@Org@StartlinkName{#1}{%
        \@ifundefined{\HypDest@Prefix#2}{%
          #2%
        }{%
          \csname\HypDest@Prefix#2\endcsname
        }%
      }%
    }%
  \else
    \renewcommand*{\Hy@StartlinkName}[2]{%
      \HypDest@MarkUsed{#2}%
      \@ifundefined{\HypDest@Prefix#2}{%
        \HypDest@Org@StartlinkName{#1}{#2}%
      }{%
        \pdfstartlink attr{#1}%
                      goto num\csname\HypDest@Prefix#2\endcsname
        \relax
      }%
    }%
  \fi
%    \end{macrocode}
%
% \subsubsection{Outlines of package \xpackage{hyperref}}
%
%    \begin{macrocode}
  \let\HypDest@Org@OutlineName\Hy@OutlineName
  \ifHypDest@name
    \renewcommand*{\Hy@OutlineName}[4]{%
      \HypDest@Org@OutlineName{#1}{%
        \@ifundefined{\HypDest@Prefix#2}{%
          #2%
        }{%
          \csname\HypDest@Prefix#2\endcsname
        }%
      }{#3}{#4}%
    }%
  \else
    \renewcommand*{\Hy@OutlineName}[4]{%
      \@ifundefined{\HypDest@Prefix#2}{%
        \HypDest@Org@OutlineName{#1}{#2}{#3}{#4}%
      }{%
        \pdfoutline goto num\csname\HypDest@Prefix#2\endcsname
                    count#3{#4}%
      }%
    }%
  \fi
%    \end{macrocode}
%    Because \cs{Hy@OutlineName} is called after the \xfile{.out} file
%    is written in the previous run. Therefore we mark the destination
%    earlier in \cs{@@writetorep}.
%    \begin{macrocode}
  \let\HypDest@Org@@writetorep\@@writetorep
  \renewcommand*{\@@writetorep}[5]{%
    \begingroup
      \edef\Hy@tempa{#5}%
      \ifx\Hy@tempa\Hy@bookmarkstype
        \HypDest@MarkUsed{#3}%
      \fi
    \endgroup
    \HypDest@Org@@writetorep{#1}{#2}{#3}{#4}{#5}%
  }%
%    \end{macrocode}
%
% \subsubsection{Outlines of package \xpackage{bookmark}}
%
%    \begin{macrocode}
  \@ifpackageloaded{bookmark}{%
    \@ifpackagelater{bookmark}{2008/08/08}{%
      \renewcommand*{\BKM@DefGotoNameAction}[2]{%
        \@ifundefined{\HypDest@Prefix#2}{%
          \edef#1{goto name{hypdestopt\string :unknown}}%
        }{%
          \ifHypDest@name
            \edef#1{goto name{\csname\HypDest@Prefix#2\endcsname}}%
          \else
            \edef#1{goto num\csname\HypDest@Prefix#2\endcsname}%
          \fi
        }%
      }%
      \def\BKM@HypDestOptHook{%
        \ifx\BKM@dest\@empty
        \else
          \ifx\BKM@gotor\@empty
            \HypDest@MarkUsed\BKM@dest
          \fi
        \fi
      }%
    }{%
      \@PackageError{hypdestopt}{%
        Package `bookmark' is too old.\MessageBreak
        Version 2008/08/08 or later is needed%
      }\@ehc
    }%
  }{}%
%    \end{macrocode}
%
%    \begin{macrocode}
}
%    \end{macrocode}
%
%
%    \begin{macrocode}
%</package>
%    \end{macrocode}
%
% \section{Installation}
%
% \subsection{Download}
%
% \paragraph{Package.} This package is available on
% CTAN\footnote{\url{ftp://ftp.ctan.org/tex-archive/}}:
% \begin{description}
% \item[\CTAN{macros/latex/contrib/oberdiek/hypdestopt.dtx}] The source file.
% \item[\CTAN{macros/latex/contrib/oberdiek/hypdestopt.pdf}] Documentation.
% \end{description}
%
%
% \paragraph{Bundle.} All the packages of the bundle `oberdiek'
% are also available in a TDS compliant ZIP archive. There
% the packages are already unpacked and the documentation files
% are generated. The files and directories obey the TDS standard.
% \begin{description}
% \item[\CTAN{install/macros/latex/contrib/oberdiek.tds.zip}]
% \end{description}
% \emph{TDS} refers to the standard ``A Directory Structure
% for \TeX\ Files'' (\CTAN{tds/tds.pdf}). Directories
% with \xfile{texmf} in their name are usually organized this way.
%
% \subsection{Bundle installation}
%
% \paragraph{Unpacking.} Unpack the \xfile{oberdiek.tds.zip} in the
% TDS tree (also known as \xfile{texmf} tree) of your choice.
% Example (linux):
% \begin{quote}
%   |unzip oberdiek.tds.zip -d ~/texmf|
% \end{quote}
%
% \paragraph{Script installation.}
% Check the directory \xfile{TDS:scripts/oberdiek/} for
% scripts that need further installation steps.
% Package \xpackage{attachfile2} comes with the Perl script
% \xfile{pdfatfi.pl} that should be installed in such a way
% that it can be called as \texttt{pdfatfi}.
% Example (linux):
% \begin{quote}
%   |chmod +x scripts/oberdiek/pdfatfi.pl|\\
%   |cp scripts/oberdiek/pdfatfi.pl /usr/local/bin/|
% \end{quote}
%
% \subsection{Package installation}
%
% \paragraph{Unpacking.} The \xfile{.dtx} file is a self-extracting
% \docstrip\ archive. The files are extracted by running the
% \xfile{.dtx} through \plainTeX:
% \begin{quote}
%   \verb|tex hypdestopt.dtx|
% \end{quote}
%
% \paragraph{TDS.} Now the different files must be moved into
% the different directories in your installation TDS tree
% (also known as \xfile{texmf} tree):
% \begin{quote}
% \def\t{^^A
% \begin{tabular}{@{}>{\ttfamily}l@{ $\rightarrow$ }>{\ttfamily}l@{}}
%   hypdestopt.sty & tex/latex/oberdiek/hypdestopt.sty\\
%   hypdestopt.pdf & doc/latex/oberdiek/hypdestopt.pdf\\
%   hypdestopt.dtx & source/latex/oberdiek/hypdestopt.dtx\\
% \end{tabular}^^A
% }^^A
% \sbox0{\t}^^A
% \ifdim\wd0>\linewidth
%   \begingroup
%     \advance\linewidth by\leftmargin
%     \advance\linewidth by\rightmargin
%   \edef\x{\endgroup
%     \def\noexpand\lw{\the\linewidth}^^A
%   }\x
%   \def\lwbox{^^A
%     \leavevmode
%     \hbox to \linewidth{^^A
%       \kern-\leftmargin\relax
%       \hss
%       \usebox0
%       \hss
%       \kern-\rightmargin\relax
%     }^^A
%   }^^A
%   \ifdim\wd0>\lw
%     \sbox0{\small\t}^^A
%     \ifdim\wd0>\linewidth
%       \ifdim\wd0>\lw
%         \sbox0{\footnotesize\t}^^A
%         \ifdim\wd0>\linewidth
%           \ifdim\wd0>\lw
%             \sbox0{\scriptsize\t}^^A
%             \ifdim\wd0>\linewidth
%               \ifdim\wd0>\lw
%                 \sbox0{\tiny\t}^^A
%                 \ifdim\wd0>\linewidth
%                   \lwbox
%                 \else
%                   \usebox0
%                 \fi
%               \else
%                 \lwbox
%               \fi
%             \else
%               \usebox0
%             \fi
%           \else
%             \lwbox
%           \fi
%         \else
%           \usebox0
%         \fi
%       \else
%         \lwbox
%       \fi
%     \else
%       \usebox0
%     \fi
%   \else
%     \lwbox
%   \fi
% \else
%   \usebox0
% \fi
% \end{quote}
% If you have a \xfile{docstrip.cfg} that configures and enables \docstrip's
% TDS installing feature, then some files can already be in the right
% place, see the documentation of \docstrip.
%
% \subsection{Refresh file name databases}
%
% If your \TeX~distribution
% (\teTeX, \mikTeX, \dots) relies on file name databases, you must refresh
% these. For example, \teTeX\ users run \verb|texhash| or
% \verb|mktexlsr|.
%
% \subsection{Some details for the interested}
%
% \paragraph{Attached source.}
%
% The PDF documentation on CTAN also includes the
% \xfile{.dtx} source file. It can be extracted by
% AcrobatReader 6 or higher. Another option is \textsf{pdftk},
% e.g. unpack the file into the current directory:
% \begin{quote}
%   \verb|pdftk hypdestopt.pdf unpack_files output .|
% \end{quote}
%
% \paragraph{Unpacking with \LaTeX.}
% The \xfile{.dtx} chooses its action depending on the format:
% \begin{description}
% \item[\plainTeX:] Run \docstrip\ and extract the files.
% \item[\LaTeX:] Generate the documentation.
% \end{description}
% If you insist on using \LaTeX\ for \docstrip\ (really,
% \docstrip\ does not need \LaTeX), then inform the autodetect routine
% about your intention:
% \begin{quote}
%   \verb|latex \let\install=y\input{hypdestopt.dtx}|
% \end{quote}
% Do not forget to quote the argument according to the demands
% of your shell.
%
% \paragraph{Generating the documentation.}
% You can use both the \xfile{.dtx} or the \xfile{.drv} to generate
% the documentation. The process can be configured by the
% configuration file \xfile{ltxdoc.cfg}. For instance, put this
% line into this file, if you want to have A4 as paper format:
% \begin{quote}
%   \verb|\PassOptionsToClass{a4paper}{article}|
% \end{quote}
% An example follows how to generate the
% documentation with pdf\LaTeX:
% \begin{quote}
%\begin{verbatim}
%pdflatex hypdestopt.dtx
%makeindex -s gind.ist hypdestopt.idx
%pdflatex hypdestopt.dtx
%makeindex -s gind.ist hypdestopt.idx
%pdflatex hypdestopt.dtx
%\end{verbatim}
% \end{quote}
%
% \section{Catalogue}
%
% The following XML file can be used as source for the
% \href{http://mirror.ctan.org/help/Catalogue/catalogue.html}{\TeX\ Catalogue}.
% The elements \texttt{caption} and \texttt{description} are imported
% from the original XML file from the Catalogue.
% The name of the XML file in the Catalogue is \xfile{hypdestopt.xml}.
%    \begin{macrocode}
%<*catalogue>
<?xml version='1.0' encoding='us-ascii'?>
<!DOCTYPE entry SYSTEM 'catalogue.dtd'>
<entry datestamp='$Date$' modifier='$Author$' id='hypdestopt'>
  <name>hypdestopt</name>
  <caption>Hyperref destination optimizer.</caption>
  <authorref id='auth:oberdiek'/>
  <copyright owner='Heiko Oberdiek' year='2006-2008,2011'/>
  <license type='lppl1.3'/>
  <version number='2.3'/>
  <description>
    This package supports <xref refid='hyperref'>hyperref</xref>'s
    pdftex driver. It removes unnecessary destinations
    and shortens the destination names or uses numbered destinations
    to get smaller PDF files.
    <p/>
    The package is part of the <xref refid='oberdiek'>oberdiek</xref>
    bundle.
  </description>
  <documentation details='Package documentation'
      href='ctan:/macros/latex/contrib/oberdiek/hypdestopt.pdf'/>
  <ctan file='true' path='/macros/latex/contrib/oberdiek/hypdestopt.dtx'/>
  <miktex location='oberdiek'/>
  <texlive location='oberdiek'/>
  <install path='/macros/latex/contrib/oberdiek/oberdiek.tds.zip'/>
</entry>
%</catalogue>
%    \end{macrocode}
%
% \begin{thebibliography}{9}
%
% \bibitem{alphalph}
%   Heiko Oberdiek: \textit{The \xpackage{alphalph} package};
%   2006/05/30 v1.4;
%   \CTAN{macros/latex/contrib/oberdiek/alphalph.pdf}.
%
% \bibitem{hyperref}
%   Sebastian Rahtz, Heiko Oberdiek:
%   \textit{The \xpackage{hyperref} package};
%   2006/06/01 v6.75a;
%   \CTAN{macros/latex/contrib/hyperref/}.
%
% \bibitem{ifpdf}
%   Heiko Oberdiek: \textit{The \xpackage{ifpdf} package};
%   2006/02/20 v1.4;
%   \CTAN{macros/latex/contrib/oberdiek/ifpdf.pdf}.
%
% \end{thebibliography}
%
% \begin{History}
%   \begin{Version}{2006/06/01 v1.0}
%   \item
%     First version.
%   \end{Version}
%   \begin{Version}{2006/06/01 v2.0}
%   \item
%     New method for referencing destinations by number; an idea
%     proposed by Lars Hellstr\"om in the mailing list LATEX-L.
%   \item
%     Options \xoption{name} and \xoption{num} added.
%   \end{Version}
%   \begin{Version}{2007/11/11 v2.1}
%   \item
%     Use of package \xpackage{pdftexcmds} for \LuaTeX\ support.
%   \end{Version}
%   \begin{Version}{2008/08/08 v2.2}
%   \item
%     Support for package \xpackage{bookmark} added.
%   \end{Version}
%   \begin{Version}{2011/05/13 v2.3}
%   \item
%     Fix for \cs{Hy@DestName} if the destination name contains
%     special characters.
%   \item
%     Fix for option \xoption{name} and package \xpackage{bookmark}.
%   \end{Version}
% \end{History}
%
% \PrintIndex
%
% \Finale
\endinput
|
% \end{quote}
% Do not forget to quote the argument according to the demands
% of your shell.
%
% \paragraph{Generating the documentation.}
% You can use both the \xfile{.dtx} or the \xfile{.drv} to generate
% the documentation. The process can be configured by the
% configuration file \xfile{ltxdoc.cfg}. For instance, put this
% line into this file, if you want to have A4 as paper format:
% \begin{quote}
%   \verb|\PassOptionsToClass{a4paper}{article}|
% \end{quote}
% An example follows how to generate the
% documentation with pdf\LaTeX:
% \begin{quote}
%\begin{verbatim}
%pdflatex hypdestopt.dtx
%makeindex -s gind.ist hypdestopt.idx
%pdflatex hypdestopt.dtx
%makeindex -s gind.ist hypdestopt.idx
%pdflatex hypdestopt.dtx
%\end{verbatim}
% \end{quote}
%
% \section{Catalogue}
%
% The following XML file can be used as source for the
% \href{http://mirror.ctan.org/help/Catalogue/catalogue.html}{\TeX\ Catalogue}.
% The elements \texttt{caption} and \texttt{description} are imported
% from the original XML file from the Catalogue.
% The name of the XML file in the Catalogue is \xfile{hypdestopt.xml}.
%    \begin{macrocode}
%<*catalogue>
<?xml version='1.0' encoding='us-ascii'?>
<!DOCTYPE entry SYSTEM 'catalogue.dtd'>
<entry datestamp='$Date$' modifier='$Author$' id='hypdestopt'>
  <name>hypdestopt</name>
  <caption>Hyperref destination optimizer.</caption>
  <authorref id='auth:oberdiek'/>
  <copyright owner='Heiko Oberdiek' year='2006-2008,2011'/>
  <license type='lppl1.3'/>
  <version number='2.3'/>
  <description>
    This package supports <xref refid='hyperref'>hyperref</xref>'s
    pdftex driver. It removes unnecessary destinations
    and shortens the destination names or uses numbered destinations
    to get smaller PDF files.
    <p/>
    The package is part of the <xref refid='oberdiek'>oberdiek</xref>
    bundle.
  </description>
  <documentation details='Package documentation'
      href='ctan:/macros/latex/contrib/oberdiek/hypdestopt.pdf'/>
  <ctan file='true' path='/macros/latex/contrib/oberdiek/hypdestopt.dtx'/>
  <miktex location='oberdiek'/>
  <texlive location='oberdiek'/>
  <install path='/macros/latex/contrib/oberdiek/oberdiek.tds.zip'/>
</entry>
%</catalogue>
%    \end{macrocode}
%
% \begin{thebibliography}{9}
%
% \bibitem{alphalph}
%   Heiko Oberdiek: \textit{The \xpackage{alphalph} package};
%   2006/05/30 v1.4;
%   \CTAN{macros/latex/contrib/oberdiek/alphalph.pdf}.
%
% \bibitem{hyperref}
%   Sebastian Rahtz, Heiko Oberdiek:
%   \textit{The \xpackage{hyperref} package};
%   2006/06/01 v6.75a;
%   \CTAN{macros/latex/contrib/hyperref/}.
%
% \bibitem{ifpdf}
%   Heiko Oberdiek: \textit{The \xpackage{ifpdf} package};
%   2006/02/20 v1.4;
%   \CTAN{macros/latex/contrib/oberdiek/ifpdf.pdf}.
%
% \end{thebibliography}
%
% \begin{History}
%   \begin{Version}{2006/06/01 v1.0}
%   \item
%     First version.
%   \end{Version}
%   \begin{Version}{2006/06/01 v2.0}
%   \item
%     New method for referencing destinations by number; an idea
%     proposed by Lars Hellstr\"om in the mailing list LATEX-L.
%   \item
%     Options \xoption{name} and \xoption{num} added.
%   \end{Version}
%   \begin{Version}{2007/11/11 v2.1}
%   \item
%     Use of package \xpackage{pdftexcmds} for \LuaTeX\ support.
%   \end{Version}
%   \begin{Version}{2008/08/08 v2.2}
%   \item
%     Support for package \xpackage{bookmark} added.
%   \end{Version}
%   \begin{Version}{2011/05/13 v2.3}
%   \item
%     Fix for \cs{Hy@DestName} if the destination name contains
%     special characters.
%   \item
%     Fix for option \xoption{name} and package \xpackage{bookmark}.
%   \end{Version}
% \end{History}
%
% \PrintIndex
%
% \Finale
\endinput
|
% \end{quote}
% Do not forget to quote the argument according to the demands
% of your shell.
%
% \paragraph{Generating the documentation.}
% You can use both the \xfile{.dtx} or the \xfile{.drv} to generate
% the documentation. The process can be configured by the
% configuration file \xfile{ltxdoc.cfg}. For instance, put this
% line into this file, if you want to have A4 as paper format:
% \begin{quote}
%   \verb|\PassOptionsToClass{a4paper}{article}|
% \end{quote}
% An example follows how to generate the
% documentation with pdf\LaTeX:
% \begin{quote}
%\begin{verbatim}
%pdflatex hypdestopt.dtx
%makeindex -s gind.ist hypdestopt.idx
%pdflatex hypdestopt.dtx
%makeindex -s gind.ist hypdestopt.idx
%pdflatex hypdestopt.dtx
%\end{verbatim}
% \end{quote}
%
% \section{Catalogue}
%
% The following XML file can be used as source for the
% \href{http://mirror.ctan.org/help/Catalogue/catalogue.html}{\TeX\ Catalogue}.
% The elements \texttt{caption} and \texttt{description} are imported
% from the original XML file from the Catalogue.
% The name of the XML file in the Catalogue is \xfile{hypdestopt.xml}.
%    \begin{macrocode}
%<*catalogue>
<?xml version='1.0' encoding='us-ascii'?>
<!DOCTYPE entry SYSTEM 'catalogue.dtd'>
<entry datestamp='$Date$' modifier='$Author$' id='hypdestopt'>
  <name>hypdestopt</name>
  <caption>Hyperref destination optimizer.</caption>
  <authorref id='auth:oberdiek'/>
  <copyright owner='Heiko Oberdiek' year='2006-2008,2011'/>
  <license type='lppl1.3'/>
  <version number='2.3'/>
  <description>
    This package supports <xref refid='hyperref'>hyperref</xref>'s
    pdftex driver. It removes unnecessary destinations
    and shortens the destination names or uses numbered destinations
    to get smaller PDF files.
    <p/>
    The package is part of the <xref refid='oberdiek'>oberdiek</xref>
    bundle.
  </description>
  <documentation details='Package documentation'
      href='ctan:/macros/latex/contrib/oberdiek/hypdestopt.pdf'/>
  <ctan file='true' path='/macros/latex/contrib/oberdiek/hypdestopt.dtx'/>
  <miktex location='oberdiek'/>
  <texlive location='oberdiek'/>
  <install path='/macros/latex/contrib/oberdiek/oberdiek.tds.zip'/>
</entry>
%</catalogue>
%    \end{macrocode}
%
% \begin{thebibliography}{9}
%
% \bibitem{alphalph}
%   Heiko Oberdiek: \textit{The \xpackage{alphalph} package};
%   2006/05/30 v1.4;
%   \CTAN{macros/latex/contrib/oberdiek/alphalph.pdf}.
%
% \bibitem{hyperref}
%   Sebastian Rahtz, Heiko Oberdiek:
%   \textit{The \xpackage{hyperref} package};
%   2006/06/01 v6.75a;
%   \CTAN{macros/latex/contrib/hyperref/}.
%
% \bibitem{ifpdf}
%   Heiko Oberdiek: \textit{The \xpackage{ifpdf} package};
%   2006/02/20 v1.4;
%   \CTAN{macros/latex/contrib/oberdiek/ifpdf.pdf}.
%
% \end{thebibliography}
%
% \begin{History}
%   \begin{Version}{2006/06/01 v1.0}
%   \item
%     First version.
%   \end{Version}
%   \begin{Version}{2006/06/01 v2.0}
%   \item
%     New method for referencing destinations by number; an idea
%     proposed by Lars Hellstr\"om in the mailing list LATEX-L.
%   \item
%     Options \xoption{name} and \xoption{num} added.
%   \end{Version}
%   \begin{Version}{2007/11/11 v2.1}
%   \item
%     Use of package \xpackage{pdftexcmds} for \LuaTeX\ support.
%   \end{Version}
%   \begin{Version}{2008/08/08 v2.2}
%   \item
%     Support for package \xpackage{bookmark} added.
%   \end{Version}
%   \begin{Version}{2011/05/13 v2.3}
%   \item
%     Fix for \cs{Hy@DestName} if the destination name contains
%     special characters.
%   \item
%     Fix for option \xoption{name} and package \xpackage{bookmark}.
%   \end{Version}
% \end{History}
%
% \PrintIndex
%
% \Finale
\endinput
|
% \end{quote}
% Do not forget to quote the argument according to the demands
% of your shell.
%
% \paragraph{Generating the documentation.}
% You can use both the \xfile{.dtx} or the \xfile{.drv} to generate
% the documentation. The process can be configured by the
% configuration file \xfile{ltxdoc.cfg}. For instance, put this
% line into this file, if you want to have A4 as paper format:
% \begin{quote}
%   \verb|\PassOptionsToClass{a4paper}{article}|
% \end{quote}
% An example follows how to generate the
% documentation with pdf\LaTeX:
% \begin{quote}
%\begin{verbatim}
%pdflatex hypdestopt.dtx
%makeindex -s gind.ist hypdestopt.idx
%pdflatex hypdestopt.dtx
%makeindex -s gind.ist hypdestopt.idx
%pdflatex hypdestopt.dtx
%\end{verbatim}
% \end{quote}
%
% \section{Catalogue}
%
% The following XML file can be used as source for the
% \href{http://mirror.ctan.org/help/Catalogue/catalogue.html}{\TeX\ Catalogue}.
% The elements \texttt{caption} and \texttt{description} are imported
% from the original XML file from the Catalogue.
% The name of the XML file in the Catalogue is \xfile{hypdestopt.xml}.
%    \begin{macrocode}
%<*catalogue>
<?xml version='1.0' encoding='us-ascii'?>
<!DOCTYPE entry SYSTEM 'catalogue.dtd'>
<entry datestamp='$Date$' modifier='$Author$' id='hypdestopt'>
  <name>hypdestopt</name>
  <caption>Hyperref destination optimizer.</caption>
  <authorref id='auth:oberdiek'/>
  <copyright owner='Heiko Oberdiek' year='2006-2008,2011'/>
  <license type='lppl1.3'/>
  <version number='2.3'/>
  <description>
    This package supports <xref refid='hyperref'>hyperref</xref>'s
    pdftex driver. It removes unnecessary destinations
    and shortens the destination names or uses numbered destinations
    to get smaller PDF files.
    <p/>
    The package is part of the <xref refid='oberdiek'>oberdiek</xref>
    bundle.
  </description>
  <documentation details='Package documentation'
      href='ctan:/macros/latex/contrib/oberdiek/hypdestopt.pdf'/>
  <ctan file='true' path='/macros/latex/contrib/oberdiek/hypdestopt.dtx'/>
  <miktex location='oberdiek'/>
  <texlive location='oberdiek'/>
  <install path='/macros/latex/contrib/oberdiek/oberdiek.tds.zip'/>
</entry>
%</catalogue>
%    \end{macrocode}
%
% \begin{thebibliography}{9}
%
% \bibitem{alphalph}
%   Heiko Oberdiek: \textit{The \xpackage{alphalph} package};
%   2006/05/30 v1.4;
%   \CTAN{macros/latex/contrib/oberdiek/alphalph.pdf}.
%
% \bibitem{hyperref}
%   Sebastian Rahtz, Heiko Oberdiek:
%   \textit{The \xpackage{hyperref} package};
%   2006/06/01 v6.75a;
%   \CTAN{macros/latex/contrib/hyperref/}.
%
% \bibitem{ifpdf}
%   Heiko Oberdiek: \textit{The \xpackage{ifpdf} package};
%   2006/02/20 v1.4;
%   \CTAN{macros/latex/contrib/oberdiek/ifpdf.pdf}.
%
% \end{thebibliography}
%
% \begin{History}
%   \begin{Version}{2006/06/01 v1.0}
%   \item
%     First version.
%   \end{Version}
%   \begin{Version}{2006/06/01 v2.0}
%   \item
%     New method for referencing destinations by number; an idea
%     proposed by Lars Hellstr\"om in the mailing list LATEX-L.
%   \item
%     Options \xoption{name} and \xoption{num} added.
%   \end{Version}
%   \begin{Version}{2007/11/11 v2.1}
%   \item
%     Use of package \xpackage{pdftexcmds} for \LuaTeX\ support.
%   \end{Version}
%   \begin{Version}{2008/08/08 v2.2}
%   \item
%     Support for package \xpackage{bookmark} added.
%   \end{Version}
%   \begin{Version}{2011/05/13 v2.3}
%   \item
%     Fix for \cs{Hy@DestName} if the destination name contains
%     special characters.
%   \item
%     Fix for option \xoption{name} and package \xpackage{bookmark}.
%   \end{Version}
% \end{History}
%
% \PrintIndex
%
% \Finale
\endinput
